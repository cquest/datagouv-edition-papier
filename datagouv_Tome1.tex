\documentclass[a4paper, 11pt]{book}
\usepackage[french]{babel}
\usepackage{makeidx}
\usepackage{hyperref}
\usepackage{gensymb}
\usepackage{eurosym}
\usepackage{pifont}
\usepackage{textgreek}
\usepackage{imakeidx}
\usepackage{wrapfig}
\usepackage{graphicx}
\usepackage{longtable}
\usepackage{qrcode}
\usepackage{booktabs}
\usepackage{parskip}
\usepackage{needspace}
\usepackage{minitoc}
\dominitoc
\mtcselectlanguage{french}

\makeindex[intoc]

\hypersetup{
    colorlinks=true,
    linkcolor=black,
    filecolor=magenta,      
    urlcolor=black,
}

\title{data.gouv.fr}
\author{Etalab}
\date{2019-04-01} 
\begin{document}
{
\frontmatter
\begin{titlepage}
\thispagestyle{empty}
\vspace*{\fill}

\includegraphics[width=8cm]{images/datagouv.png}
\newline
\rule{\textwidth}{0.2pt}
\Large \par \hfill Édition 2019
\vspace*{\fill}
\newline
\begin{center} 
\LARGE Tome 1 - Le "Best Of"
\par
\end{center}

\vspace*{\fill}
\includegraphics[width=4cm]{images/marianne.png}
\hfill
\includegraphics[width=4cm]{images/logo-etalab.png}

\pagenumbering{gobble}
\end{titlepage}
\pagebreak
\normalsize
\pagenumbering{roman}
\section*{Préface}

\emph{data.gouv.fr} est le portail unique interministériel destiné à
rassembler et à mettre à disposition librement l'ensemble des
informations publiques de l'Etat, de ses établissements publics
administratifs et, si elles le souhaitent, des collectivités
territoriales et des personnes de droit public ou de droit privé
chargées d'une mission de service public.
(décret n\degree 2011-194 du 21 février 2011)

\noindent
Vous avez entre les mains la toute première édition papier de \emph{data.gouv.fr}
\par
\noindent
Elle a été produite par \emph{Etalab \& Alumni}, avec le langage \href{https://fr.wikipedia.org/wiki/LaTeX}{\LaTeX}.
\par
\noindent
\textbf{Notre environnement est fragile, merci de n’imprimer ce document qu’en cas de nécessité.}

\par
\vspace*{\fill}
Ce document est sous \href{https://www.etalab.gouv.fr/wp-content/uploads/2017/04/ETALAB-Licence-Ouverte-v2.0.pdf}{Licence Ouverte 2.0}
\clearpage
\setcounter{tocdepth}{0}
\tableofcontents

\mainmatter

\chapter{Données de référence}

Ce chapitre liste les jeux de données de référence définis
par le \href{https://www.legifrance.gouv.fr/affichTexte.do?cidTexte=JORFTEXT000034194946&categorieLien=id}{décret n\degree 2017-331}
du 14 mars 2017 relatif au service public de mise à disposition des données de référence codifié dans
l'\href{https://www.legifrance.gouv.fr/affichCodeArticle.do;jsessionid=19672D71250ADD261EC92AD65C5FE97B.tplgfr36s_3?cidTexte=LEGITEXT000031366350&idArticle=LEGIARTI000034196073&categorieLien=id#LEGIARTI000034196073}{article R321-5 et suivants du Code des Relations entre le Public et l'Administration}.
\par
\noindent
Voir aussi \url{https://www.data.gouv.fr/fr/reference}

\minitoc

\clearpage
\needspace{12\baselineskip}
\clearpage\section{Base Adresse Nationale
}

\begin{center}
  \includegraphics[width=3cm]{images/orga/18_18b929a270482faefabb18f5d2b4fd-100.png}
\end{center}

  \begin{wrapfigure}{r}{2.5cm}
    \centering
    \qrcode[nolink]{https://data.gouv.fr/dataset/5530fbacc751df5ff937dddb}
  \end{wrapfigure}

Licence : \textbf{Open Data Commons Open Database License (ODbL)
}\newline
Créé le : 2015-04-17\newline
Modifié le : 2018-11-28\newline
Granularité : au point d'intérêt\newline
Mise à jour : hebdomadaire\newline
Popularité : 29 réutilisations,  35 suivis\newline
Mots-clé : \emph{aucun
}\newline
Permalien : \url{https://data.gouv.fr/dataset/5530fbacc751df5ff937dddb}\newline

\par
\noindent
    La Base Adresse Nationale est une base de données qui a pour but de
référencer l'intégralité des adresses du territoire français.

Elle contient la position géographique de plus de 20 millions
d'adresses.

Elle est constituée par la collaboration entre Etalab, La Poste, l'IGN,
la DGFiP et OpenStreetMap France.

Le projet est co-gouverné par l'Administrateur Général des Données et le
Conseil National de l'Information Géographique.

Elle est diffusée sur le site
\href{https://adresse.data.gouv.fr}{adresse.data.gouv.fr} développé par
la mission Etalab de la Direction interministérielle du numérique et du
système d'information et de communication de l'État (DINSIC).

Les données sont disponibles sous licence ODbL 1.0.

Un \href{https://adresse.data.gouv.fr/api}{service de géocodage gratuit}
est mis à disposition par la mission Etalab.


\vspace{0.5cm}
\needspace{12\baselineskip}
\clearpage\section{Base Sirene des entreprises et de leurs établissements (SIREN,
SIRET){[}fin le 30 avril 2019{]}
}

\begin{center}
  \includegraphics[width=3cm]{images/orga/db_fbfd745ae543f6856ed59e3d44eb71-100.jpg}
\end{center}
\index{associations}\index{companies}\index{entreprises}\index{etablissements}\index{immatriculation}\index{register}\index{registre}\index{siren}\index{sirene}\index{siret}
  \begin{wrapfigure}{r}{2.5cm}
    \centering
    \qrcode[nolink]{https://data.gouv.fr/dataset/5862206588ee38254d3f4e5e}
  \end{wrapfigure}

Licence : \textbf{Licence Ouverte
}\newline
Créé le : 2016-12-27\newline
Modifié le : 2019-03-16\newline
Mise à jour : quotienne\newline
Popularité : 43 réutilisations,  66 suivis\newline
Mots-clé : \emph{associations, companies, entreprises, etablissements, immatriculation, register, registre, siren, sirene, siret
}\newline
Permalien : \url{https://data.gouv.fr/dataset/5862206588ee38254d3f4e5e}\newline

\par
\noindent
    \texttt{Suite\ à\ plusieurs\ demandes\ émanant\ d\textquotesingle{}utilisateurs\ des\ fichiers\ Sirene,\ l\textquotesingle{}Insee\ a\ décidé\ de\ prolonger\ \ de\ trois\ mois\ la\ mise\ à\ disposition\ des\ fichiers\ à\ l\textquotesingle{}ancien\ format\ (dessins\ L2\ et\ XL2\ :\ fichiers\ stocks,\ mises\ à\ jour\ mensuelles\ et\ quotidiennes).\ Cette\ offre\ prendra\ donc\ fin\ le\ 30\ avril\ 2019.\ La\ mise\ en\ concordance\ semestrielle\ ne\ se\ fera\ pas\ sur\ ce\ jeu\ de\ données.}ATTENTION
! Ce jeu de données est remplacé par un
\href{https://www.data.gouv.fr/fr/datasets/base-sirene-des-entreprises-et-de-leurs-etablissements-siren-siret/}{nouveau
jeu de données} et prendra fin le \textbf{30 avril 2019} \emph{(au lieu
du 31 janvier 2019)}.

\textbf{La base Sirene contenant des données à caractère personnel,
l'Insee attire votre attention sur les obligations légales qui en
découlent} : - Le traitement de ces données relève des obligations de
déclaration de la Loi 78-17 du 6 janvier 1978 modifiée, dite Loi
CNIL~:\url{https://www.cnil.fr/fr/loi-78-17-du-6-janvier-1978-modifiee}-
Selon votre usage du jeu de données, il est de votre responsabilité de
tenir compte du statut de diffusion le plus récent de chaque personne
physique. En effet, l'article A123-96 du code de commerce dispose que~:
``Toute personne physique peut demander soit directement lors de ses
formalités de création ou de modification, soit par lettre adressée au
directeur général de l'Institut national de la statistique et des études
économiques, que les informations du répertoire la concernant ne
puissent être utilisées par des tiers autres que les organismes
habilités au titre de l'article R. 123-224 ou les administrations, à des
fins de prospection, notamment commerciale.''

\textbf{Fichiers} : Trois types de fichiers compactés (format ZIP) sont
mis à disposition (1 fichier stock et 2 fichiers de mises à jour
mensuelles ou quotidiennes, France entière). Chaque fichier compacté
(Format ZIP) contient un fichier de données en format CSV.

Le fichier stock et le fichier des mises à jour mensuelles sont déposés
le premier jour de chaque mois avec les données (dessin de fichier L2)
au dernier jour du mois précédent. Le fichier des mises à jour
mensuelles étant au même dessin que le fichier stock, vous pouvez faire
le choix chaque mois, d'annuler et remplacer le fichier stock par le
nouveau fichier stock ou de l'actualiser avec le fichier des mises à
jour mensuelles. Tous les mouvements relatifs à un établissement au
cours du mois sont agrégés, rendant l'information facilement utilisable.

Le fichier des mises à jour quotidiennes est déposé dans la nuit suivant
la journée de gestion qu'il concerne (dessin de fichier XL2).

\emph{\textbf{NB}~: Les fichiers sont déposés un peu plus tardivement
début janvier et début juillet en raison d'opérations de mise en
concordance semestrielle de la base de diffusion.}

Le site de l'Insee www.sirene.fr fournit des informations sur le contenu
de ces bases, ainsi que la
\href{http://sirene.fr/sirene/public/static/documentation}{documentation}
associée et une \href{http://sirene.fr/sirene/public/faq}{Foire Aux
Questions} pour vous aider, comprenant plus de 50 questions-réponses.

Pour toute demande de création, de modification ou de changement
concernant votre situation administrative, nous vous invitons à
contacter le Centre de formalités des entreprises dont vous dépendez :
\url{https://www.service-public.fr/professionnels-entreprises/vosdroits/F24023}.{]}(https://www.service-public.fr/professionnels-entreprises/vosdroits/F24023{]}(https://www.service-public.fr/professionnels-entreprises/vosdroits/F24023).)


\vspace{0.5cm}
\needspace{12\baselineskip}
\clearpage\section{Code Officiel Géographique (COG)
}

\begin{center}
  \includegraphics[width=3cm]{images/orga/db_fbfd745ae543f6856ed59e3d44eb71-100.jpg}
\end{center}
\index{code!geographique}\index{decoupages!administratifs}
  \begin{wrapfigure}{r}{2.5cm}
    \centering
    \qrcode[nolink]{https://data.gouv.fr/dataset/58c984b088ee386cdb1261f3}
  \end{wrapfigure}

Licence : \textbf{Licence Ouverte
}\newline
Créé le : 2017-03-15\newline
Modifié le : 2018-05-28\newline
Granularité : à la commune\newline
Mise à jour : annuelle\newline
Popularité : 8 réutilisations,  18 suivis\newline
Mots-clé : \emph{code-geographique, decoupages-administratifs
}\newline
Permalien : \url{https://data.gouv.fr/dataset/58c984b088ee386cdb1261f3}\newline

\par
\noindent
    Le code officiel géographique rassemble les codes et libellés des
communes, des cantons, des arrondissements, des départements, des
régions, des collectivités d'outre-mer et des pays et territoires
étrangers au 1er janvier de chaque année.


\vspace{0.5cm}
\needspace{12\baselineskip}
\clearpage\section{Plan Cadastral Informatisé
}

\begin{center}
  \includegraphics[width=3cm]{images/orga/68_0660a939e7495d94117e0d9845d6f1-100.png}
\end{center}
\index{batiments}\index{cadastre}\index{parcelles}
  \begin{wrapfigure}{r}{2.5cm}
    \centering
    \qrcode[nolink]{https://data.gouv.fr/dataset/58e5924b88ee3802ca255566}
  \end{wrapfigure}

Licence : \textbf{Licence Ouverte
}\newline
Créé le : 2017-04-06\newline
Modifié le : 2018-10-31\newline
Mise à jour : trimestrielle\newline
Popularité : 3 réutilisations,  54 suivis\newline
Mots-clé : \emph{batiments, cadastre, parcelles
}\newline
Permalien : \url{https://data.gouv.fr/dataset/58e5924b88ee3802ca255566}\newline

\par
\noindent
    Le plan cadastral est un assemblage d'environ \textbf{600 000 feuilles}
ou planches représentant chacune une section ou une partie d'une section
cadastrale.

Il couvre la France entière, à l'exception de la ville de Strasbourg et
de quelques communes voisines, pour des raisons historiques liée à
l'occupation de l'Alsace-Moselle par l'Allemagne entre 1871 et 1918. PCI
Vecteur et PCI Image

Pour des questions pratiques et techniques, le Plan Cadastral
Informatisé existe sous la forme de \textbf{deux produits
complémentaires} : le PCI Vecteur et le PCI Image.

Le \textbf{PCI Vecteur} regroupe les feuilles qui ont été numérisées et
couvre l'essentiel du territoire.

Le \textbf{PCI Image} regroupe les feuilles qui n'ont été que scannées,
et complète la couverture. Couverture

\textbf{33 682 communes} sont couvertes par le PCI Vecteur, sur un total
de près de 35 400. Les plans des autres communes sont disponibles dans
le PCI Image.

Strasbourg et les communes limitrophes ne sont actuellement pas gérées
au format PCI.

Les collectivités d'outre-mer de Saint-Martin et de Saint-Barthelemy
sont présentes et historiquement intégrées dans le département de la
Guadeloupe (971). Formats disponibles

Les données du PCI Vecteur sont disponibles dans plusieurs formats :

\begin{itemize}

\item
  Format
  \href{https://www.data.gouv.fr/s/resources/plan-cadastral-informatise/20170906-150737/standard_edigeo_2013.pdf}{EDIGÉO}
  en projection Lambert 93 ;
\item
  Format
  \href{https://www.data.gouv.fr/s/resources/plan-cadastral-informatise/20170906-150737/standard_edigeo_2013.pdf}{EDIGÉO}
  en projection Lambert CC 9 zones ;
\item
  Format DXF-PCI en projection Lambert 93 ;
\item
  Format DXF-PCI en projection Lambert CC 9 zones.
\end{itemize}

Les données du PCI Image sont disponibles au format TIFF. Modèle de
données

Chaque commune est subdivisée en sections, elles-mêmes subdivisées en
feuilles (ou planches). Une feuille cadastrale comporte des parcelles,
qui peuvent supporter des bâtiments, ainsi que de nombreux autres objets
d'habillage ou de gestion. Pour plus de précision, veuillez vous
reporter à la documentation du standard
\href{https://www.data.gouv.fr/s/resources/plan-cadastral-informatise/20170906-150737/standard_edigeo_2013.pdf}{EDIGÉO}.
Mise à disposition

Les données sont mises à disposition de deux manières :

\begin{itemize}

\item
  En téléchargement direct à la feuille ou en archive départementale. Ce
  sont ces URL qu'il faut utiliser si vous souhaitez automatiser la
  récupération des données et bénéficier des meilleures performances.
\item
  Via un outil en ligne pour les archives communales. Les données sont
  alors produites à la volée.
\end{itemize}

Les deux modes de mise à disposition sont accessibles sur
\href{https://cadastre.data.gouv.fr/datasets/plan-cadastral-informatise}{cadastre.data.gouv.fr}.
Millésimes disponibles :

\begin{itemize}
\item
  13 février 2017
\item
  14 mai 2017
\item
  6 juillet 2017
\item
  12 octobre 2017
\item
  2 janvier 2018
\item
  3 avril 2018
\item
  29 juin 2018 (plus récent) Voir aussi
\item
  \href{https://cadastre.data.gouv.fr/datasets/cadastre-etalab}{Données
  cadastrales retravaillées par Etalab (formats GeoJSON et Shapefile)}
\item
  \href{https://cadastre.data.gouv.fr/datasets/cadastre-strasbourg}{Données
  cadastrales Eurométropole de Strasbourg}
\end{itemize}


\vspace{0.5cm}
\needspace{12\baselineskip}
\clearpage\section{Référentiel à grande échelle (RGE)
}

\begin{center}
  \includegraphics[width=3cm]{images/orga/1b_e4985396724faf9f6e1122baa7b65c-100.png}
\end{center}
\index{bd!ortho}\index{bd!topo}\index{cours!deau}\index{hydrographie}\index{hydronymes}\index{lacs}\index{ortho!image!aerienne}\index{orthoimagerie}\index{rge}\index{rivieres}\index{topographie}
  \begin{wrapfigure}{r}{2.5cm}
    \centering
    \qrcode[nolink]{https://data.gouv.fr/dataset/58e5842688ee386c65805755}
  \end{wrapfigure}

Licence : \textbf{License Not Specified
}\newline
Créé le : 2017-04-06\newline
Modifié le : 2017-06-23\newline
Mise à jour : irrégulière\newline
Popularité : 2 réutilisations,  8 suivis\newline
Mots-clé : \emph{bd-ortho, bd-topo, cours-deau, hydrographie, hydronymes, lacs, ortho-image-aerienne, orthoimagerie, rge, rivieres, topographie
}\newline
Permalien : \url{https://data.gouv.fr/dataset/58e5842688ee386c65805755}\newline

\par
\noindent
    L'État a confié à l'IGN le développement du référentiel à grande échelle
(RGE). Pour ce faire, il fait appel à ses moyens propres ainsi qu'à des
partenariats avec des producteurs principalement de la sphère publique.
Le RGE est constitué des composantes orthophotographique, topographique
et adresse, parcellaire et altimétrique.

\textbf{Spécifications du RGE :}

Les spécifications du RGE sont accessibles à partir de liens
référérencés ci-dessous.

\textbf{Accès en téléchargement aux données du RGE :}

Les données du RGE sous Licence Ouverte (actuellement : la BD TOPO thème
hydrographique, BD ORTHO 5m, certains départements de BD ORTHO 50 cm)
sont accessibles à partir des liens listés ci-dessous.

Les autres données du RGE, soumises aux licences IGN en vigueur, sont
accessibles \href{http://professionnels.ign.fr/rge}{depuis le site de
l'IGN}.

\textbf{API d'accès au RGE :}

L'ensemble des données du RGE est disponible en web service via les
géoservices du Géoportail

\begin{itemize}

\item
  services de consultation au standard WMS et WMTS
\item
  services vecteur au standard WFS
\item
  autres géoservices (géocodage adresse et parcelles, alticodage, etc)
\end{itemize}

L'URL d'obtention de la clé d'accès est indiquée ci-dessous. La
documentation d'utilisation des géoservices est accessible
sur\url{https://geoservices.ign.fr/} \textbf{Signalement et remontée
d'informations sur le RGE :}

L'IGN propose plusieurs canaux de signalement :

\begin{itemize}

\item
  sur le \href{https://www.geoportail.gouv.fr/carte}{site Géoportail},
  via la rubrique outils/signaler une anomalie
\item
  via le service dédié
  \href{https://espacecollaboratif.ign.fr/}{espacecollaboratif.ign.fr},
  nécessitant une inscription préalable. L'espace collaboratif propose
  une \href{https://espacecollaboratif.ign.fr/api/doc/georem}{API}ainsi
  que des plugins pour les logiciels
  \href{http://logiciels.ign.fr/?-RIPart-Geoconcept-}{Geoconcept},
  \href{http://logiciels.ign.fr/?-RIPart-ArcMap-}{ArcMap}et
  \href{http://logiciels.ign.fr/?-RIPart-QGIS-}{QGIS}.
\item
  via l'appli Espace Collaboratif disponible sur
  Android\url{https://play.google.com/store/apps/details?id=fr.ign.guichet\&hl=fr}et
  iOS\url{https://itunes.apple.com/tw/app/id1245621439.}
\end{itemize}


\vspace{0.5cm}
\needspace{12\baselineskip}
\clearpage\section{Référentiel de l'organisation administrative de l'Etat
}

\begin{center}
  \includegraphics[width=3cm]{images/orga/08_98902e29244685862bcdd3198cef7b-100.png}
\end{center}
\index{administration}\index{administration!generale}\index{administration!publique}\index{annuaire}\index{annuaire!de!services}\index{autorites!publiques}\index{base}\index{coordonnees}\index{coordonnees!geographiques}\index{dila}\index{nationale}\index{organisation!administrative}\index{organisme!public}\index{referentiel}\index{service!public}\index{services!d!utilite!publique}\index{services!de!l!etat}
  \begin{wrapfigure}{r}{2.5cm}
    \centering
    \qrcode[nolink]{https://data.gouv.fr/dataset/57343feb88ee3823b0d1b934}
  \end{wrapfigure}

Licence : \textbf{Licence Ouverte
}\newline
Créé le : 2016-05-12\newline
Modifié le : 2019-02-07\newline
De 2016-05-12 à 2026-05-12\newline
Mise à jour : hebdomadaire\newline
Popularité : 4 réutilisations,  14 suivis\newline
Mots-clé : \emph{administration, administration-generale, administration-publique, annuaire, annuaire-de-services, autorites-publiques, base, coordonnees, coordonnees-geographiques, dila, nationale, organisation-administrative, organisme-public, referentiel, service-public, services-d-utilite-publique, services-de-l-etat
}\newline
Permalien : \url{https://data.gouv.fr/dataset/57343feb88ee3823b0d1b934}\newline

\par
\noindent
    \href{https://www.legifrance.gouv.fr/eli/decret/2017/3/14/PRMJ1636987D/jo/texte}{Le
décret du 14 mars 2017} a institué le \textbf{Service Public de la
Donnée}.

Celui-ci met à la disposition du public \textbf{9 jeux de données de
référence} parmi lesquels la base nationale de l'organisation
administrative de l'Etat, produite et diffusée par la Dila sur son site
\href{https://lannuaire.service-public.fr/}{service-public.fr}.

Le Référentiel de l'organisation administrative de l'Etat, nouvelle
appellation de la base, comprend toutes les institutions régies par la
Constitution de la Ve République et les administrations qui en
dépendent, soit environ 6000 organismes. Le périmètre couvre \textbf{les
services centraux de l'Etat}, jusqu'au niveau des bureaux.

Le référentiel comprend les missions, l'organisation hiérarchique des
services et leurs coordonnées complètes.

\href{https://echanges.dila.gouv.fr/OPENDATA/RefOrgaAdminEtat/FluxHistorique/}{Accéder
à l'historique des versions}

Vous pouvez nous écrire ou vous abonner à une alerte par mail adressé à
: \textbf{donnees-dila@dila.gouv.fr}


\vspace{0.5cm}
\needspace{12\baselineskip}
\clearpage\section{Registre parcellaire graphique (RPG) : contours des parcelles et îlots
culturaux et leur groupe de cultures majoritaire
}

\begin{center}
  \includegraphics[width=3cm]{images/orga/1b_e4985396724faf9f6e1122baa7b65c-100.png}
\end{center}
\index{agriculture}\index{aides}\index{europe}\index{politique!agricole!commune}\index{registre!parcellaire!graphique}
  \begin{wrapfigure}{r}{2.5cm}
    \centering
    \qrcode[nolink]{https://data.gouv.fr/dataset/58d8d8a0c751df17537c66be}
  \end{wrapfigure}

Licence : \textbf{Licence Ouverte
}\newline
Créé le : 2017-03-27\newline
Modifié le : 2018-11-13\newline
De 2013-01-01 à 2017-12-31\newline
Mise à jour : annuelle\newline
Popularité : 2 réutilisations,  7 suivis\newline
Mots-clé : \emph{agriculture, aides, europe, politique-agricole-commune, registre-parcellaire-graphique
}\newline
Permalien : \url{https://data.gouv.fr/dataset/58d8d8a0c751df17537c66be}\newline

\par
\noindent
    Le registre parcellaire graphique est une base de données géographiques
servant de référence à l'instruction des aides de la politique agricole
commune (PAC). La version anonymisée diffusée ici dans le cadre du
service public de mise à disposition des données de référence contient
les données graphiques des parcelles (depuis 2015) et îlots (éditions
2014 et antérieures). munis de leur culture principale. Ces données sont
produites par l'agence de services et de paiement (ASP) depuis 2007.

La réutilisation du RPG est gratuite pour tous les usages, y compris
commerciaux, selon les termes de la ``licence ouverte'' version 1.0.

Les éditions du Registre parcellaire graphique antérieures à 2013 (2010,
2011 et 2012) sur data.gouv.fr sont disponibles sur
\href{https://www.data.gouv.fr/fr/organizations/agence-de-services-et-de-paiement-asp/}{la
page de l'Agence de service et de paiement}

Les données anonymes du RPG sont millésimées et contiennent des
parcelles et îlots correspondant à ceux déclarés pour la campagne N dans
leur situation connue et arrêtée par l'administration, en général au 1er
janvier de l'année N+1. Ces données couvrent l'ensemble du territoire
français hors Mayotte (y compris les collectivités d'outre-mer de
Saint-Barthélemy et de Saint-Martin).

Format : shapefile

\textbf{Projections disponibles :}

Dans les systèmes géodésiques légaux :

\begin{itemize}

\item
  En métropole : (RGF93) projection Lambert-93
\item
  En outre-mer : (système légal) projections UTM
\end{itemize}

\textbf{Découpages disponibles :}

Le RPG est disponible en téléchargement : - France entière (à compter de
l'édition 2015) - par région (à compter de l'édition 2015) - France
entière par région (de l'édition 2013 à l'édition 2014)

\textbf{API d'accès :}

Le RPG France entière est disponible en web service via les géoservices
du Géoportail - services de consultation au standard WMS et WMTS -
services vecteur au standard WFS.

L'URL pour l'obtention d'une clé d'accès est indiquée ci-dessous. La
documentation d'utilisation des géoservices est accessible
sur\url{https://geoservices.ign.fr/}


\vspace{0.5cm}
\needspace{12\baselineskip}
\clearpage\section{Répertoire National des Associations
}

\begin{center}
  \includegraphics[width=3cm]{images/orga/8e_0925ee5c44429d9e37131258773cb7-100.png}
\end{center}
\index{1901}\index{asso}\index{association}\index{associations}\index{loi!1901}\index{rna}\index{waldec}
  \begin{wrapfigure}{r}{2.5cm}
    \centering
    \qrcode[nolink]{https://data.gouv.fr/dataset/58e53811c751df03df38f42d}
  \end{wrapfigure}

Licence : \textbf{Licence Ouverte
}\newline
Créé le : 2017-04-05\newline
Modifié le : 2019-03-09\newline
De 1901-01-01 à 2017-04-05\newline
Granularité : au point d'intérêt\newline
Mise à jour : mensuelle\newline
Popularité : 6 réutilisations,  22 suivis\newline
Mots-clé : \emph{1901, asso, association, associations, loi-1901, rna, waldec
}\newline
Permalien : \url{https://data.gouv.fr/dataset/58e53811c751df03df38f42d}\newline

\par
\noindent
    Le Répertoire National des Associations (RNA) contient l'ensemble des
associations relevant de la loi 1901, à savoir toutes les associations
de France, dont le siège est déclaré en métropole ou dans les
départements d'outre-mer, sauf dans les départements de la Moselle, du
Bas-Rhin et du Haut-Rhin, qui relèvent du régime du Concordat. Le RNA
contient également les associations reconnues d'utilité publique (dites
``ARUP'').

Les associations relevant de la loi 1901 sont déclarées en préfecture ou
en sous-préfecture (au greffe des associations) : la création et les
changements de statuts, tels que la modification du nom, du siège, des
dirigeants, etc. doivent être déclarés et sont mis à jour dans la base
du RNA. La mise à jour est effective une fois les données validées par
le greffe ou suite à la publication au Journal officiel des associations
et des fondations d'entreprise (JOAFE) d'une création, d'une dissolution
(obligatoire) ou d'un changement de situation (publication au JOAFE non
obligatoires).

La base du Répertoire National des Associations (RNA) est alimentée par
la validation par les greffes des associations et la publication au
Journal officiel des associations et des fondations d'entreprise des
créations et modifications des associations relevant de la loi 1901.

\textbf{Diffusion des données du Répertoire National des Associations
(RNA)}

Conformément aux dispositions de la loi pour une République numérique,
les données du Répertoire National des Associations (RNA), produites par
la Direction des libertés publiques et des affaires juridiques (DLPAJ)
du Ministère de l'intérieur, sont aujourd'hui accessibles ci-dessous.

Cette base, qui comprend toutes les associations relevant de la loi 1901
intègre désormais le service public de la donnée. Vous pouvez maintenant
télécharger l'intégralité de la base, ainsi que la documentation
associée. Les mises à jour quotidiennes seront également téléchargeables
prochainement.

Le contenu des données à télécharger est scindé en deux extractions :

\emph{RNA\_waldec} : liste des associations disposant d'un n\degree{}
RNA. Toutes les associations créées ou ayant déclaré un changement de
situation depuis 2009 disposent d'un n\degree{} RNA.

\emph{RNA\_import} : liste des associations créées depuis 1901 et qui
n'ont pas effectué de déclaration de changement de situation depuis
2009.

\emph{RNA\_Liste\_donnees\_diffusees} : Ce fichier décrit les données
exposées par le RNA pour les fichiers typés ``import'' et ``waldec''.

\textbf{Contenu des extractions :}

\begin{itemize}

\item
  le cas échéant, le n\degree{} RNA
\item
  le nom de l'association et son sigle
\item
  l'objet de l'association et son objet social
\item
  l'adresse du siège
\item
  le cas échéant, l'adresse de gestion
\item
  le cas échéant, le site internet de l'association
\end{itemize}

\textbf{Fréquence d'extraction}

La fréquence est pour le moment mensuelle.


\vspace{0.5cm}
\needspace{12\baselineskip}
\clearpage\section{Répertoire Opérationnel des Métiers et des Emplois (ROME)
}

\begin{center}
  \includegraphics[width=3cm]{images/orga/79_02694fd7de42b9901a0d220857e9fc-100.png}
\end{center}
\index{competences}\index{demande!d!emploi}\index{fiche!metier}\index{offre!d!emploi}\index{pole!emploi}\index{rome}\index{skills}
  \begin{wrapfigure}{r}{2.5cm}
    \centering
    \qrcode[nolink]{https://data.gouv.fr/dataset/58da857388ee384902e505f5}
  \end{wrapfigure}

Licence : \textbf{Licence Ouverte
}\newline
Créé le : 2017-03-28\newline
Modifié le : 2018-03-20\newline
Mise à jour : trimestrielle\newline
Popularité : 2 réutilisations,  8 suivis\newline
Mots-clé : \emph{competences, demande-d-emploi, fiche-metier, offre-d-emploi, pole-emploi, rome, skills
}\newline
Permalien : \url{https://data.gouv.fr/dataset/58da857388ee384902e505f5}\newline

\par
\noindent
    Dans un contexte marqué par de fortes mutations de l'environnement
économique et social, le ROME (Répertoire Opérationnel des Métiers et
des Emplois) est un outil au service de la mobilité professionnelle et
du rapprochement entre offres et candidats.

Le ROME a été construit par les équipes de Pôle emploi avec la
contribution d'un large réseau de partenaires (entreprises, branches et
syndicats professionnels, AFPA\ldots{}), en s'appuyant sur une démarche
pragmatique : inventaire des dénominations d'emplois/métiers les plus
courantes, analyse des activités et compétences, regroupement des
emplois selon un principe d'équivalence ou de proximité.

En décembre 2016, les référentiels de compétences du ROME évoluent afin
d'améliorer la transversalité lors du rapprochement entre l'offre et la
demande. Cette évolution consiste à : • réorganiser les compétences en
savoir-faire et savoirs • reformuler les libellés en les simplifiant et
les décontextualisant.


\vspace{0.5cm}

    \clearpage

    \chapter{Données des services publics certifiés}

    Ce chapitre liste une sélection des jeux de données produits par les services publics
    répertoriés sur le portail national \href{https://data.gouv.fr/}{data.gouv.fr}
    ayant fait l'objet d'au moins une réutilisation publiés sous Licence Ouverte ou ODbL.

    \minitoc

    \clearpage

    

\clearpage
\section{Agence Bio}


\begin{center}
  \includegraphics[width=3cm]{images/orga/fe_3a4e927b7e41199b04501e769ad967-100.png}
\end{center}


Créée en novembre 2001, l'Agence française pour le développement et la
promotion de l'agriculture biologique, est une plateforme nationale
d'information et d'actions qui s'inscrit dans une dynamique de
développement, de promotion et de structuration de l'agriculture
biologique française.


\vspace{0.5cm}

\needspace{12\baselineskip}
\subsection*{Agriculture biologique 2008-2011 - nombre d'opérateurs engagés en
agriculture biologique
}
  \begin{wrapfigure}{r}{2.5cm}
    \centering
    \qrcode[nolink]{https://data.gouv.fr/dataset/53698e90a3a729239d2034d3}
  \end{wrapfigure}

Licence : \textbf{Licence Ouverte
}\newline
Créé le : 2013-07-08\newline
Modifié le : 2016-03-15\newline
De 2008-01-01 à 2011-12-31\newline
Mise à jour : annuelle\newline
Popularité : 4 réutilisations,  3 suivis\newline
Mots-clé : \emph{aucun
}\newline
Permalien : \url{https://data.gouv.fr/dataset/53698e90a3a729239d2034d3}\newline

\par
\noindent
    Nombre d'opérateurs engagés en agriculture biologique sur la période
2008-2011 inclus.


\vspace{0.5cm}
\needspace{12\baselineskip}
\subsection*{Agriculture biologique 2008-2011 - productions animales bio - têtes par
département
}
  \begin{wrapfigure}{r}{2.5cm}
    \centering
    \qrcode[nolink]{https://data.gouv.fr/dataset/53698e90a3a729239d2034d4}
  \end{wrapfigure}

Licence : \textbf{Licence Ouverte
}\newline
Créé le : 2013-07-08\newline
Modifié le : 2015-12-03\newline
De 2008-01-01 à 2011-12-31\newline
Mise à jour : annuelle\newline
Popularité : 1 réutilisation,  0 suivi\newline
Mots-clé : \emph{aucun
}\newline
Permalien : \url{https://data.gouv.fr/dataset/53698e90a3a729239d2034d4}\newline

\par
\noindent
    Nombre de têtes bio par département sur la période 2008-2011 inclus.


\vspace{0.5cm}
\needspace{3\baselineskip} \rule{4cm}{0.25pt}\newline\textbf{Aussi disponible du même producteur :}\begin{itemize}
\item \href{https://data.gouv.fr/dataset/53698e91a3a729239d2034d5}{Agriculture biologique 2008-2011 - productions végétales - surfaces par département}
\end{itemize}

\clearpage
\section{Agence de l'eau Loire-Bretagne}


\begin{center}
  \includegraphics[width=3cm]{images/orga/2015-03-30_ba1032847c03494fbc20488c3d8bb474_LOGO_AELB_CAS1_Q-100.png}
\end{center}


L'agence de l'eau est un établissement public de l'Etat, sous tutelle du
ministère du développement durable.

\textbf{Notre mission :} contribuer à préserver les ressources en eau et
les milieux aquatiques en apportant une aide technique et financière aux
collectivités, acteurs économiques, associations pour réduire les
pollutions, mieux gérer les ressources, restaurer les cours d'eau et les
zones humides. Nous finançons ces aides grâce aux redevances que nous
percevons auprès des utilisateurs d'eau.

Une partie importante de notre activité porte sur la connaissance des
milieux aquatiques et de leur fonctionnement. Autre volet important :
l'animation de la concertation, en particulier au travers du Comité de
bassin. Ce ``parlement de l'eau'' est composé de 190 représentants des
différents acteurs de l'eau. C'est lui qui élabore la stratégie du
bassin pour un ``bon état des eaux'' (pour les connaisseurs, le
``Sdage'').

Le \textbf{bassin Loire-Bretagne}, correspond au bassin hydrographique
de la Loire et de ses affluents, au bassin de la Vilaine et aux bassins
côtiers bretons et vendéens. C'est environ le tiers du territoire
métropolitain, 336 communautés de communes, plus de 7000 communes, 36
départements et 8 régions en tout ou partie et presque 13 millions
d'habitants.


\vspace{0.5cm}

\needspace{12\baselineskip}
\subsection*{Données du Schéma Directeur d'Aménagement et de Gestion des Eaux (SDAGE)
2016/2021 du bassin Loire-Bretagne
}\index{eau}\index{sdage}
  \begin{wrapfigure}{r}{2.5cm}
    \centering
    \qrcode[nolink]{https://data.gouv.fr/dataset/5716032088ee381f27f5262c}
  \end{wrapfigure}

Licence : \textbf{Licence Ouverte
}\newline
Créé le : 2016-04-19\newline
Modifié le : 2016-05-12\newline
Granularité : à la région\newline
Popularité : 1 réutilisation,  1 suivi\newline
Mots-clé : \emph{eau, sdage
}\newline
Permalien : \url{https://data.gouv.fr/dataset/5716032088ee381f27f5262c}\newline

\par
\noindent
    Pour chaque carte du Sdage 2016-2021, vous pouvez visualiser et
télécharger la carte au format image telle qu'elle apparaît dans le
document SDAGE (Schéma Directeur d'Aménagement et de Gestion des Eaux)
ou directement dans l'application Carmen.


\vspace{0.5cm}
\needspace{12\baselineskip}
\subsection*{Qualité des eaux de surface continentales dans le bassin Loire-Bretagne
de 1971 à nos jours
}\index{analyse}\index{cours!d!eau}\index{eau!de!surface}\index{plan!d!eau}\index{qualite!biologique}\index{qualite!de!l!eau}\index{riviere}
  \begin{wrapfigure}{r}{2.5cm}
    \centering
    \qrcode[nolink]{https://data.gouv.fr/dataset/5732f25988ee387e56d1b934}
  \end{wrapfigure}

Licence : \textbf{Licence Ouverte
}\newline
Créé le : 2016-05-11\newline
Modifié le : 2016-06-28\newline
Granularité : au point d'intérêt\newline
Mise à jour : quotienne\newline
Popularité : 2 réutilisations,  0 suivi\newline
Mots-clé : \emph{analyse, cours-d-eau, eau-de-surface, plan-d-eau, qualite-biologique, qualite-de-l-eau, riviere
}\newline
Permalien : \url{https://data.gouv.fr/dataset/5732f25988ee387e56d1b934}\newline

\par
\noindent
    La qualité des eaux de surface continentales (cours d'eau et plans
d'eau) de 1971 à nos jours dans le bassin Loire-Bretagne (banque OSUR).
Les données portent sur les analyses physico-chimiques et
hydrobiologiques.


\vspace{0.5cm}
\needspace{3\baselineskip} \rule{4cm}{0.25pt}\newline\textbf{Aussi disponible du même producteur :}\begin{itemize}
\item \href{https://data.gouv.fr/dataset/57751dce88ee38487b0ddd2c}{Cartes de l'assainissement non collectif (ANC) sur le bassin Loire-Bretagne }
\item \href{https://data.gouv.fr/dataset/58a5bf77c751df255e9dfc79}{Les contrats territoriaux du bassin Loire Bretagne}
\item \href{https://data.gouv.fr/dataset/551c0d68c751df77e74b1077}{Les prélèvements d'eau pour l'industrie de 2008 à 2014 dans le bassin Loire-Bretagne}
\item \href{https://data.gouv.fr/dataset/59195ee0c751df2f55e7e68e}{Les Sage du bassin Loire-Bretagne}
\item \href{https://data.gouv.fr/dataset/5a1295cdc751df7acd724663}{Liste des décisions d'aide attribuées lors du 10 ème programme}
\end{itemize}

\clearpage
\section{Agence de services et de paiement (ASP)}


\begin{center}
  \includegraphics[width=3cm]{images/orga/15_f5c37f8118492fba1843f0aaa5d63c-100.png}
\end{center}


L'Agence de services et de paiement (ASP) est un établissement public
interministériel qui contribue à la mise en œuvre de politiques
publiques.

Elle mène des missions pour le compte de plus de 130 donneurs d'ordre :
l'Union européenne - elle est le premier payeur européen d'aides
agricoles -, plus de 10 ministères, la quasi-totalité des régions,
plusieurs dizaines de départements et des établissements publics.

L'ASP, avec ses 26 délégations régionales, est présente sur tout le
territoire, en métropole et à l'outre-mer.


\vspace{0.5cm}

\needspace{12\baselineskip}
\subsection*{Aides perçues par les personnes morales au titre de la Politique
Agricole Commune
}\index{agriculture}\index{europe}\index{politique!agricole!commune}
  \begin{wrapfigure}{r}{2.5cm}
    \centering
    \qrcode[nolink]{https://data.gouv.fr/dataset/53698ec6a3a729239d20355d}
  \end{wrapfigure}

Licence : \textbf{Licence Ouverte
}\newline
Créé le : 2013-09-05\newline
Modifié le : 2016-03-12\newline
De 2010-01-01 à 2012-12-31\newline
Granularité : à la commune\newline
Mise à jour : ponctuelle\newline
Popularité : 6 réutilisations,  14 suivis\newline
Mots-clé : \emph{agriculture, europe, politique-agricole-commune
}\newline
Permalien : \url{https://data.gouv.fr/dataset/53698ec6a3a729239d20355d}\newline

\par
\noindent
    Le Règlement communautaire (CE) n\degree{}259/2008 a imposé aux Etats
Membres de publier les montants d'aides perçus par chaque bénéficiaire
au titre de la PAC. A la suite de l'arrêt de la Cour de Justice de
l'Union européenne rendu le 9 novembre 2010, la base juridique de la
publication des bénéficiaires des paiements de la PAC a été
partiellement invalidée : elle ne concerne désormais plus que les
personnes morales. Les données disponibles au travers d'Etalab, sont
donc pour chaque forme sociétaire concernée, les montants perçus au
titre du Feader d'une part (hors et inclus prêts bonifiés) et du Feaga
(paiements directs et autres) d'autre part.


\vspace{0.5cm}
\needspace{12\baselineskip}
\subsection*{Registre Parcellaire Graphique 2010: contours des îlots culturaux et
leur groupe de cultures majoritaire des exploitations
}
  \begin{wrapfigure}{r}{2.5cm}
    \centering
    \qrcode[nolink]{https://data.gouv.fr/dataset/53699edca3a729239d205faf}
  \end{wrapfigure}

Licence : \textbf{Licence Ouverte
}\newline
Créé le : 2013-08-16\newline
Modifié le : 2017-10-10\newline
De 2010-04-01 à 2010-06-11\newline
Granularité : à la commune\newline
Mise à jour : ponctuelle\newline
Popularité : 5 réutilisations,  3 suivis\newline
Mots-clé : \emph{aucun
}\newline
Permalien : \url{https://data.gouv.fr/dataset/53699edca3a729239d205faf}\newline

\par
\noindent
    Le Règlement communautaire (CE) n\degree{}1593/2000 a institué
l'obligation, dans tous les Etats Membres, de localiser et d'identifier
les parcelles agricoles.Pour répondre à cette exigence, la France a mis
en place le Registre Parcellaire Graphique (RPG) qui est un système
d'information géographique permettant l'identification des parcelles
agricoles. Ainsi, chaque année, les agriculteurs adressent à
l'administration un dossier de déclaration de surfaces qui comprend
notamment le dessin des îlots de culture qu'ils exploitent et les
cultures qui y sont pratiquées. La localisation des îlots se fait à
l'échelle du 1:5000 sur le fond photographique de la BD Ortho (IGN) et
leur mise à jour est annuelle. Cette base de données constitue donc une
description à grande échelle et régulièrement mise à jour de la majorité
des terres agricoles.Les données disponibles au travers d'Etalab, pour
un département métropolitain donné, sont les contours des îlots «
anonymisés » du RPG et leur occupation culturale représentée par le
groupe de cultures majoritaire de l'îlot. Elles correspondent au
millésime 2010.Ces données brutes, en mode vecteur, sont encodées en
fichiers au format Shapefile selon un modèle de données décrit dans un
fichier associé.La géométrie des îlots est en Latitude/Longitude dans le
système de projection Lambert 93.Le contenu attributaire fourni est la
valeur du code du groupe de cultures majoritaire dans l'îlot.


\vspace{0.5cm}
\needspace{12\baselineskip}
\subsection*{Registre Parcellaire Graphique 2011: contours des îlots culturaux et
leur groupe de cultures majoritaire des exploitations
}\index{registre!parcellaire!graphique}\index{rpg}
  \begin{wrapfigure}{r}{2.5cm}
    \centering
    \qrcode[nolink]{https://data.gouv.fr/dataset/53fd262ea3a729390ba568a7}
  \end{wrapfigure}

Licence : \textbf{Licence Ouverte
}\newline
Créé le : 2014-02-11\newline
Modifié le : 2016-01-13\newline
De 2011-01-01 à 2011-12-31\newline
Granularité : à la commune\newline
Mise à jour : ponctuelle\newline
Popularité : 1 réutilisation,  0 suivi\newline
Mots-clé : \emph{registre-parcellaire-graphique, rpg
}\newline
Permalien : \url{https://data.gouv.fr/dataset/53fd262ea3a729390ba568a7}\newline

\par
\noindent
    Le Règlement communautaire (CE) n\degree{}1593/2000 a institué
l'obligation, dans tous les Etats Membres, de localiser et d'identifier
les parcelles agricoles.Pour répondre à cette exigence, la France a mis
en place le Registre Parcellaire Graphique (RPG) qui est un système
d'information géographique permettant l'identification des parcelles
agricoles. Ainsi, chaque année, les agriculteurs adressent à
l'administration un dossier de déclaration de surfaces qui comprend
notamment le dessin des îlots de culture qu'ils exploitent et les
cultures qui y sont pratiquées. La localisation des îlots se fait à
l'échelle du 1:5000 sur le fond photographique de la BD Ortho (IGN) et
leur mise à jour est annuelle. Cette base de données constitue donc une
description à grande échelle et régulièrement mise à jour de la majorité
des terres agricoles.Les données disponibles au travers d'Etalab, pour
un département métropolitain donné, sont les contours des îlots «
anonymisés » du RPG et leur occupation culturale représentée par le
groupe de cultures majoritaire de l'îlot. Elles correspondent au
millésime 2011.Ces données brutes, en mode vecteur, sont encodées en
fichiers au format Shapefile selon un modèle de données décrit dans un
fichier associé.La géométrie des îlots est en Latitude/Longitude dans le
système de projection Lambert 93.Le contenu attributaire fourni est la
valeur du code du groupe de cultures majoritaire dans l'îlot.


\vspace{0.5cm}
\needspace{12\baselineskip}
\subsection*{Registre Parcellaire Graphique 2012: contours des îlots culturaux et
leur groupe de cultures majoritaire des exploitations
}\index{cultures}\index{registre!parcellaire!graphique}\index{rpg}
  \begin{wrapfigure}{r}{2.5cm}
    \centering
    \qrcode[nolink]{https://data.gouv.fr/dataset/53699edca3a729239d205fb3}
  \end{wrapfigure}

Licence : \textbf{Licence Ouverte
}\newline
Créé le : 2014-02-11\newline
Modifié le : 2016-03-16\newline
De 2012-01-01 à 2012-12-31\newline
Granularité : à la commune\newline
Mise à jour : ponctuelle\newline
Popularité : 3 réutilisations,  9 suivis\newline
Mots-clé : \emph{cultures, registre-parcellaire-graphique, rpg
}\newline
Permalien : \url{https://data.gouv.fr/dataset/53699edca3a729239d205fb3}\newline

\par
\noindent
    Le Règlement communautaire (CE) n\degree{}1593/2000 a institué
l'obligation, dans tous les Etats Membres, de localiser et d'identifier
les parcelles agricoles.Pour répondre à cette exigence, la France a mis
en place le Registre Parcellaire Graphique (RPG) qui est un système
d'information géographique permettant l'identification des parcelles
agricoles. Ainsi, chaque année, les agriculteurs adressent à
l'administration un dossier de déclaration de surfaces qui comprend
notamment le dessin des îlots de culture qu'ils exploitent et les
cultures qui y sont pratiquées. La localisation des îlots se fait à
l'échelle du 1:5000 sur le fond photographique de la BD Ortho (IGN) et
leur mise à jour est annuelle. Cette base de données constitue donc une
description à grande échelle et régulièrement mise à jour de la majorité
des terres agricoles.Les données disponibles au travers d'Etalab, pour
un département métropolitain donné, sont les contours des îlots «
anonymisés » du RPG et leur occupation culturale représentée par le
groupe de cultures majoritaire de l'îlot. Elles correspondent au
millésime 2012.Ces données brutes, en mode vecteur, sont encodées en
fichiers au format Shapefile selon un modèle de données décrit dans un
fichier associé.La géométrie des îlots est en Latitude/Longitude dans le
système de projection Lambert 93.Le contenu attributaire fourni est la
valeur du code du groupe de cultures majoritaire dans l'îlot.


\vspace{0.5cm}
\needspace{3\baselineskip} \rule{4cm}{0.25pt}\newline\textbf{Aussi disponible du même producteur :}\begin{itemize}
\item \href{https://data.gouv.fr/dataset/53698ec1a3a729239d203552}{Aides de la politique agricole commune (PAC)}
\end{itemize}

\clearpage
\section{Agence du Numérique}


\begin{center}
  \includegraphics[width=3cm]{images/orga/fa_349f7213f4483daaf111e878884ebb-100.png}
\end{center}


Rattachée au ministère de l'Économie, l'Agence du Numérique a pour
mission d'impulser et de soutenir des actions préparant la société
française aux révolutions numériques. Elle intervient toujours en
soutien à des écosystèmes territoriaux associant acteurs publics et
privés, avec un rôle d'animation et de soutien à des initiatives
locales.

Pour atteindre cet objectif, l'Agence du Numérique pilote trois
politiques publiques complémentaires :

\begin{itemize}

\item
  Le Plan France Très Haut Débit vise à déployer de nouvelles
  infrastructures numériques sur l'ensemble du territoire pour apporter
  un accès à un Internet très haut débit sur l'ensemble du territoire
  d'ici 2022,
\item
  Le Programme Société Numérique vise à donner à tous les citoyens la
  capacité de saisir les nombreuses opportunités qu'offre le
  développement du numérique,
\item
  L'Initiative French Tech a pour objectif de soutenir la croissance des
  startups en France et à l'international.
\end{itemize}


\vspace{0.5cm}

\needspace{12\baselineskip}
\subsection*{Baromètre du Numérique 2007-2016
}\index{agence!du!numerique}\index{arcep}\index{barometre}\index{barometre!du!numerique}\index{dematerialisation}\index{numerique}\index{usages}
  \begin{wrapfigure}{r}{2.5cm}
    \centering
    \qrcode[nolink]{https://data.gouv.fr/dataset/583d4f5388ee3846c6c65bb3}
  \end{wrapfigure}

Licence : \textbf{Open Data Commons Open Database License (ODbL)
}\newline
Créé le : 2016-11-29\newline
Modifié le : 2016-12-13\newline
Mise à jour : annuelle\newline
Popularité : 1 réutilisation,  2 suivis\newline
Mots-clé : \emph{agence-du-numerique, arcep, barometre, barometre-du-numerique, dematerialisation, numerique, usages
}\newline
Permalien : \url{https://data.gouv.fr/dataset/583d4f5388ee3846c6c65bb3}\newline

\par
\noindent
    Le Baromètre du Numérique est une étude de référence sur l'adoption par
les Français des équipements et des usages numériques. Il est le fruit
d'une collaboration entre le Conseil général de l'économie (CGE) et
l'Autorité de régulation des communications électroniques et des postes
(Arcep) depuis 2003, à laquelle l'Agence du Numérique s'est associée
pour l'édition 2016.

Chaque année, au mois de juin, le Centre de recherche pour l'étude et
l'observation des conditions de vie (Credoc) interroge en face-à-face un
échantillon de plus de 2 000 personnes représentatif de la population
française. Des questions leur sont posées sur la nature de leurs
équipements (téléphonie, ordinateur, tablette, etc.) et de leurs usages
numériques (réseaux sociaux, e-commerce, administration en ligne, etc.).

\textbf{\href{https://www.data.gouv.fr/fr/datasets/barometre-du-numerique-2007-2016-1/}{Données
publiées et maintenues par l'Arcep, à retrouver sur leur espace
data.gouv.fr} ou via les liens vers leur API ci-dessous :}


\vspace{0.5cm}
\needspace{12\baselineskip}
\subsection*{Contributions à la concertation Territoire Numérique
}
  \begin{wrapfigure}{r}{2.5cm}
    \centering
    \qrcode[nolink]{https://data.gouv.fr/dataset/58da6b15c751df262746558d}
  \end{wrapfigure}

Licence : \textbf{Licence Ouverte
}\newline
Créé le : 2017-03-28\newline
Modifié le : 2017-03-28\newline
De 2017-02-08 à 2017-03-24\newline
Mise à jour : irrégulière\newline
Popularité : 1 réutilisation,  0 suivi\newline
Mots-clé : \emph{aucun
}\newline
Permalien : \url{https://data.gouv.fr/dataset/58da6b15c751df262746558d}\newline

\par
\noindent
    L'article 69 de la Loi pour une République numérique introduit la
possibilité pour les collectivités qui le souhaitent de mettre en place
des stratégies de développement des usages et services numériques. Il
s'agit notamment de ``favoriser l'équilibre de l'offre de services
numériques sur le territoire ainsi que la mise en place de ressources
mutualisées, publiques et privées, y compris en matière de médiation
numérique''.

Dans ce cadre, l'Agence du Numérique est chargée de rédiger le
document-cadre intitulé Orientations nationales pour le développement
des usages et des services numériques dans les territoires. Dans ce
cadre une concertation a été menée du 8 février au 24 mars une
concertation nationale a été menée.

Variables : 1. Débat : Titre de la question 1. Contribution collective :
Nom de l'atelier contributif\\
1. Nombre de participants : Nombre de participants à l'atelier publié\\
1. user-id : identifiant anonymisé du contributeur ou du porte-parole de
l'atelier collectif 1. Contenu : titre de la contribution


\vspace{0.5cm}
\needspace{12\baselineskip}
\subsection*{Enquête Capacity / World Internet Project
}\index{agence!du!numerique}\index{empowerment}\index{enquete}\index{marsouin}\index{pouvoir!dagir}\index{sociologie}\index{usages!numeriques}\index{wip}\index{world!internet!project}
  \begin{wrapfigure}{r}{2.5cm}
    \centering
    \qrcode[nolink]{https://data.gouv.fr/dataset/58d2f1cbc751df71d2ce9186}
  \end{wrapfigure}

Licence : \textbf{Open Data Commons Open Database License (ODbL)
}\newline
Créé le : 2017-03-22\newline
Modifié le : 2017-03-22\newline
Granularité : au pays\newline
Mise à jour : ponctuelle\newline
Popularité : 1 réutilisation,  1 suivi\newline
Mots-clé : \emph{agence-du-numerique, empowerment, enquete, marsouin, pouvoir-dagir, sociologie, usages-numeriques, wip, world-internet-project
}\newline
Permalien : \url{https://data.gouv.fr/dataset/58d2f1cbc751df71d2ce9186}\newline

\par
\noindent
    Les thématiques abordées sont : la connectivité et l'équipement
numériques, la diversité et l'intensité des usages, les compétences
numériques, les attitudes et représentations vis-à-vis du numérique et
le pouvoir d'agir.

Les données récoltées visent notamment à alimenter les travaux entrepris
dans le cadre du projet de recherche « Capacity », portant sur les
réalités de l'empowerment par les usages numériques en France, ainsi que
les réflexions de l'Agence du numérique, partie prenante de cette
enquête.

Le questionnaire intègre également les « questions communes » à chaque
pays membre du WIP (World Internet Project) auquel M@rsouin participe
pour la première fois en 2016 en tant que représentant de la France.

L'enquête s'est déroulée du 17 novembre au 8 décembre 2016 et a permis
d'obtenir les réponses de 2036 français représentatifs de la population
française des 18 ans et plus. Les interviews ont été réalisées en
face-à-face.

La représentativité de l'échantillon est assurée par la méthode des
quotas sur le sexe, l'âge, la profession, la catégorie d'agglomération
et la région géographique (UDA5) du répondant. Les quotas sont croisés
sur le sexe et l'âge et indépendants sur la profession, la catégorie
d'agglomération et la région.


\vspace{0.5cm}
\needspace{12\baselineskip}
\subsection*{Niveau des débits sur les réseaux d'accès à Internet : ADSL, câble,
Fibre FttH (T2 2015 - T2 2017)
}\index{adsl}\index{agence!du!numerique}\index{cable}\index{couverture!internet}\index{fibre!optique}\index{ftth}\index{internet}\index{mission!tres!haut!debit}\index{open!data!couverture!numerique}\index{plan!france!tres!haut!debit}\index{reseaux}\index{thd}\index{tres!haut!debit}
  \begin{wrapfigure}{r}{2.5cm}
    \centering
    \qrcode[nolink]{https://data.gouv.fr/dataset/5694060588ee386ac0af0bf4}
  \end{wrapfigure}

Licence : \textbf{Licence Ouverte
}\newline
Créé le : 2016-01-11\newline
Modifié le : 2018-06-29\newline
Popularité : 5 réutilisations,  7 suivis\newline
Mots-clé : \emph{adsl, agence-du-numerique, cable, couverture-internet, fibre-optique, ftth, internet, mission-tres-haut-debit, open-data-couverture-numerique, plan-france-tres-haut-debit, reseaux, thd, tres-haut-debit
}\newline
Permalien : \url{https://data.gouv.fr/dataset/5694060588ee386ac0af0bf4}\newline

\par
\noindent
    Les données présentent la répartition du nombre de locaux (logements,
entreprises, sites publics) par classe de débits pour chaque commune du
T2 2015 au T2 2017. Elles sont affichées toutes technologies confondues
et par technologie (ADSL, câble, fibre FttH).

Ces données ne sont plus actualisées.


\vspace{0.5cm}
\needspace{3\baselineskip} \rule{4cm}{0.25pt}\newline\textbf{Aussi disponible du même producteur :}\begin{itemize}
\item \href{https://data.gouv.fr/dataset/587e5b3488ee387e829b81a4}{Enquêtes régionales sur les usages numériques des ménages}
\item \href{https://data.gouv.fr/dataset/587379c788ee384d1e0bfefe}{Liste des jeux de données ouvertes des principaux observatoires numériques français}
\end{itemize}

\clearpage
\section{Agence nationale de sécurité sanitaire, de l'alimentation, de l'environnement et du travail (Anses)}


\begin{center}
  \includegraphics[width=3cm]{images/orga/06_75fedf299e42a5ae0eea17522ced09-100.png}
\end{center}


L'Anses - Agence nationale de sécurité sanitaire de l'alimentation, de
l'environnement et du travail - est une instance scientifique
intervenant dans les domaines de l'alimentation, de l'environnement, du
travail, de la santé et du bien-être des animaux et de la santé des
végétaux.

Le cœur de l'action de l'Anses est de mettre en œuvre une expertise
scientifique indépendante et pluraliste afin d'évaluer les risques
sanitaires et de proposer aux autorités compétentes toute mesure de
nature à préserver la santé publique.

Par ses activités de veille, d'expertise, de recherche et de référence,
l'Agence couvre l'ensemble des risques (microbiologiques, physiques ou
chimiques) auxquels un individu peut être exposé, volontairement ou non,
à tous les moments de sa vie, qu'il s'agisse d'expositions sur son lieu
de travail, pendant ses transports, ses loisirs ou via son alimentation.
Cette activité repose sur la mise en œuvre d'une expertise scientifique
indépendante et pluraliste au sein de collectifs d'experts, en intégrant
les dimensions socio-économiques du risque.

Reconnue comme une agence de référence dans les domaines de
l'alimentation, de la santé et du bien-être des animaux, de la santé des
végétaux, de l'environnement et du travail, l'Anses a noué nombre de
partenariats et est impliquée dans des programmes de recherche
nationaux, européens et internationaux.

L'Agence est un établissement public à caractère administratif, placée
sous la tutelle des ministères chargés de la santé, de l'agriculture, de
l'environnement, du travail et de la consommation.


\vspace{0.5cm}

\needspace{12\baselineskip}
\subsection*{Données de consommations et habitudes alimentaires de l'étude INCA 2
}\index{alimentation}\index{consommation}\index{habitudes!alimentaires}
  \begin{wrapfigure}{r}{2.5cm}
    \centering
    \qrcode[nolink]{https://data.gouv.fr/dataset/5422875e88ee38334e5b9e15}
  \end{wrapfigure}

Licence : \textbf{Licence Ouverte
}\newline
Créé le : 2014-09-24\newline
Modifié le : 2016-03-16\newline
De 2006-01-01 à 2007-04-30\newline
Popularité : 3 réutilisations,  3 suivis\newline
Mots-clé : \emph{alimentation, consommation, habitudes-alimentaires
}\newline
Permalien : \url{https://data.gouv.fr/dataset/5422875e88ee38334e5b9e15}\newline

\par
\noindent
    Les études INCA constituent un des outils indispensables à l'évaluation
du risque. En effet, elles fournissent à un moment donné une
photographie des habitudes de consommations alimentaires de la
population en France métropolitaine. Combinées aux bases de données de
l'Anses sur la composition des aliments (plusieurs millions de données),
ces données permettent de connaître les apports en substances bénéfiques
présentes dans notre alimentation (vitamines, acides gras essentiels,
etc.)

Les données disponibles intègrent l'ensemble des thématiques couvertes
par l'étude : -les caractéristiques des individus ayant participé à
l'étude, notamment démographiques et socioéconomiques, critères de choix
des aliments, préparation et conservation des aliments, habitudes de
vie, état de santé, attitudes et opinions en alimentation ; -les apports
nutritionnels journaliers en 38 nutriments des individus ; -les
consommations alimentaires détaillées et quantifiées sur la semaine des
individus et le descriptif des occasions de consommation (lieu, durée,
etc.) ; -les consommations de compléments alimentaires (type de
compléments alimentaires et quantité, contexte et motivations d'achat)
et les apports nutritionnels issus des compléments alimentaires au
niveau individuel. Plus d'informations sur
\href{http://www.anses.fr}{www.anses.fr}


\vspace{0.5cm}
\needspace{12\baselineskip}
\subsection*{Données régionales EAT2 (Etude de l'Alimentation Totale)
}\index{alimentation!humaine}\index{dioxines}\index{risque!sanitaire}
  \begin{wrapfigure}{r}{2.5cm}
    \centering
    \qrcode[nolink]{https://data.gouv.fr/dataset/53699334a3a729239d204120}
  \end{wrapfigure}

Licence : \textbf{Licence Ouverte
}\newline
Créé le : 2014-01-17\newline
Modifié le : 2017-01-25\newline
De 2006-01-01 à 2011-12-31\newline
Granularité : à la région\newline
Mise à jour : ponctuelle\newline
Popularité : 1 réutilisation,  7 suivis\newline
Mots-clé : \emph{alimentation-humaine, dioxines, risque-sanitaire
}\newline
Permalien : \url{https://data.gouv.fr/dataset/53699334a3a729239d204120}\newline

\par
\noindent
    Les données publiées constituent une analyse des éventuelles différences
interrégionales en termes d'exposition à une douzaine de substances
(notamment dioxines, PCB, acrylamide, plomb ou arsenic), pour lesquelles
un risque sanitaire dû à une contamination par l'alimentation n'avait pu
être exclu. Ces données mettent en évidence une faible variabilité des
expositions entre inter-régions, en France, pour les composés chimiques
considérés.


\vspace{0.5cm}
\needspace{3\baselineskip} \rule{4cm}{0.25pt}\newline\textbf{Aussi disponible du même producteur :}\begin{itemize}
\item \href{https://data.gouv.fr/dataset/5c6c16388b4c412a1032642a}{Bisphénol A}
\item \href{https://data.gouv.fr/dataset/5c5da3c2634f415ae64ace8c}{Cartographie des bases de données existantes dans le domaine de la santé et de la sécurité au travail}
\item \href{https://data.gouv.fr/dataset/5bfe5a5b8b4c410e486b0bb4}{Données Etude de l’Alimentation Totale infantile}
\item \href{https://data.gouv.fr/dataset/575e9fac88ee38072a640390}{Données ouvertes du catalogue E-Phy des produits phytopharmaceutiques, matières fertilisantes et supports de culture, adjuvants, produits mixtes et mélanges}
\end{itemize}

\clearpage
\section{Agence nationale des fréquences}


\begin{center}
  \includegraphics[width=3cm]{images/orga/28_84a29053514049aa04bbad9e00f155-100.png}
\end{center}


L'ANFR gère le spectre des fréquences. Les fréquences radioélectriques
sont une ressource naturelle disponible en quantité limitée.
Irremplaçables pour communiquer avec les objets mobiles et pour créer
des infrastructures flexibles et à faible coût, elles sont déterminantes
pour la compétitivité. Des secteurs entiers reposent sur l'accès à cette
ressource : communications mobiles, audiovisuel hertzien, liaisons
satellite, transports, industrie militaire, météorologie\ldots{} Les
fréquences contribuent ainsi à l'innovation et à la création de nombreux
emplois.

L'ANFR remplit trois missions principales.

\begin{itemize}

\item
  Elle contribue à l'élaboration de la réglementation internationale et
  à la planification des bandes de fréquences.
\item
  Elle gère les bandes de fréquences nationales et les sites
  radioélectriques.
\item
  Elle contrôle les installations ainsi que leurs émissions et
  intervient dans le traitement des brouillages.
\end{itemize}

L'ANFR est un établissement public. A son conseil d'administration sont
représentés plusieurs ministères et régulateurs pour une prise en compte
des intérêts de tous les utilisateurs de fréquences. Sa tutelle est
assurée par la Ministre chargée du Numérique.

Les sites internet de l'ANFR : \href{http://www.anfr.fr}{www.anfr.fr} ;
\href{http://www.cartoradio.fr}{www.cartoradio.fr} ;
\href{http://www.recevoirlatnt.fr}{www.recevoirlatnt.fr} Suivez l'ANFR
sur twitter : @anfr


\vspace{0.5cm}

\needspace{12\baselineskip}
\subsection*{Données sur les installations radioélectriques de plus de 5 watts
}\index{antennes}\index{frequence}\index{station}
  \begin{wrapfigure}{r}{2.5cm}
    \centering
    \qrcode[nolink]{https://data.gouv.fr/dataset/551d4ff3c751df55da0cd89f}
  \end{wrapfigure}

Licence : \textbf{Licence Ouverte
}\newline
Créé le : 2015-04-02\newline
Modifié le : 2019-03-12\newline
Granularité : à la commune\newline
Mise à jour : mensuelle\newline
Popularité : 5 réutilisations,  22 suivis\newline
Mots-clé : \emph{antennes, frequence, station
}\newline
Permalien : \url{https://data.gouv.fr/dataset/551d4ff3c751df55da0cd89f}\newline

\par
\noindent
    Installations radioélectriques de plus de 5 watts, hormis celles de
l'Aviation Civile et des ministères de la Défense et de l'Intérieur. Les
données présentées proviennent d'une base de données de l'ANFR alimentée
par tous les exploitants d'installations radioélectriques, publics ou
privés, dans le cadre de la procédure administrative prévue par
l'article L.43 du code des postes et communications électroniques.


\vspace{0.5cm}
\needspace{3\baselineskip} \rule{4cm}{0.25pt}\newline\textbf{Aussi disponible du même producteur :}\begin{itemize}
\item \href{https://data.gouv.fr/dataset/574edb7388ee3843b484e966}{données de mesures de niveau de champ}
\item \href{https://data.gouv.fr/dataset/574c51fe88ee385366d1b934}{Données radio maritime}
\end{itemize}

\clearpage
\section{Agence pour l'enseignement français à l'étranger}


\begin{center}
  \includegraphics[width=3cm]{images/orga/96_dbd25970664d909abcec69cda4b334-100.png}
\end{center}


Créée en 1990, l'Agence pour l'enseignement français à l'étranger (AEFE)
est un établissement public national placé sous la tutelle du ministère
des Affaires étrangères. Elle assure les missions de service public
relatives à l'éducation en faveur des enfants français résidant hors de
France et contribue au rayonnement de la langue et de la culture
françaises ainsi qu'au renforcement des relations entre les systèmes
éducatifs français et étrangers. Le principal objectif de l'AEFE est de
servir et promouvoir un réseau scolaire unique au monde, constitué en
2012 de 480 établissements implantés dans 130 pays.


\vspace{0.5cm}

\needspace{12\baselineskip}
\subsection*{Guide des établissements d'enseignement délivrant un programme homologué
par le ministère de l'éducation nationale
}
  \begin{wrapfigure}{r}{2.5cm}
    \centering
    \qrcode[nolink]{https://data.gouv.fr/dataset/53699608a3a729239d2048b4}
  \end{wrapfigure}

Licence : \textbf{Licence Ouverte
}\newline
Créé le : 2013-07-08\newline
Modifié le : 2015-11-22\newline
De 2011-01-01 à 2011-12-31\newline
Mise à jour : annuelle\newline
Popularité : 1 réutilisation,  2 suivis\newline
Mots-clé : \emph{aucun
}\newline
Permalien : \url{https://data.gouv.fr/dataset/53699608a3a729239d2048b4}\newline

\par
\noindent
    Liste d'informations éditées annuellement dans un guide papier
permettant de trouver par pays les établissements scolaires à programme
français


\vspace{0.5cm}
\needspace{3\baselineskip} \rule{4cm}{0.25pt}\newline\textbf{Aussi disponible du même producteur :}\begin{itemize}
\item \href{https://data.gouv.fr/dataset/536995f7a3a729239d204889}{Géolocalisation des établissements du réseau d'enseignement de l'AEFE}
\end{itemize}

\clearpage
\section{Agence Technique de l'information sur l'Hospitalisation (ATIH)}


\begin{center}
  \includegraphics[width=3cm]{images/orga/9e_9ca637c309405aa3879b7ba724a2ba-100.jpg}
\end{center}


L'Agence technique de l'information sur l'hospitalisation (ATIH) a été
instituée par le décret n\degree{}2000-1282 du 26 décembre 2000, repris
dans le code de la santé publique aux articles R. 6113-33 et suivants
qui précisent ses missions, son organisation et son fonctionnement
budgétaire et comptable Initialement circonscrit aux travaux techniques
concourant à la mise en œuvre et à l'accessibilité aux tiers du
programme de médicalisation du système d'information (PMSI) ainsi que
des travaux relatifs aux nomenclatures de santé, son périmètre
d'activité s'est élargi avec la mise en place de la tarification à
l'activité (T2A) en 2004. Le PMSI est en effet devenu un outil de
pilotage contribuant à mesurer la performance des établissements de
santé et non plus seulement un outil descriptif de l'activité médicale.


\vspace{0.5cm}

\needspace{12\baselineskip}
\subsection*{Référentiel national de coûts 2011 pour les cliniques
}
  \begin{wrapfigure}{r}{2.5cm}
    \centering
    \qrcode[nolink]{https://data.gouv.fr/dataset/53699ed7a3a729239d205fa4}
  \end{wrapfigure}

Licence : \textbf{Licence Ouverte
}\newline
Créé le : 2013-07-08\newline
Modifié le : 2016-01-31\newline
De 2011-01-01 à 2011-12-31\newline
Mise à jour : annuelle\newline
Popularité : 1 réutilisation,  0 suivi\newline
Mots-clé : \emph{aucun
}\newline
Permalien : \url{https://data.gouv.fr/dataset/53699ed7a3a729239d205fa4}\newline

\par
\noindent
    Cette base de données fournit les coûts moyens 2011 des séjours
hospitaliers en médecine, chirurgie, obstétrique et odontologie (MCO)
dans les établissements de santé privés. Ce référentiel se présente sous
la forme d'une échelle de coûts organisée selon la classification (v11e)
des groupes homogènes de malades (GHM). Les coûts moyens ont été
calculés à partir des données issues de l'étude nationale de coûts à
méthodologie commune (ENCC) portant sur l'activité 2011.\\
Ce référentiel a été élaboré à partir d'un échantillon de 73
établissements de santé, redressé avec les données nationales
recueillies par le programme de médicalisation des systèmes
d'information (PMSI). En plus de l'accès aux bases détaillées des coûts
moyens, il est possible de consulter des fiches de synthèse selon
différentes catégories d'activité : catégorie majeure de diagnostic
(CMD), sous-CMD, racine et GHM. Sur ces fiches, les évolutions de coûts
par rapport à l'exercice 2010 sont précisées. Un document de synthèse «
Principaux résultats issus du référentiel 2011 » détaille les principaux
résultats des données de coûts évalués à partir des données 2011 et leur
évolution par rapport à 2010. Après une présentation générale, les
résultats sont analysés par CMD, activité, niveaux de sévérité, etc. En
complément, deux guides permettent de faciliter la lecture et la
compréhension de ces informations : - « Guide pratique Référentiel 2011
», qui recense les données détaillées du référentiel et explicite
l'accès aux données et l'utilisation des différents onglets; - « Guide
technique Référentiel 2011 », qui présente les modalités de calcul du
coût moyen et les indicateurs statistiques associés.


\vspace{0.5cm}
\needspace{3\baselineskip} \rule{4cm}{0.25pt}\newline\textbf{Aussi disponible du même producteur :}\begin{itemize}
\item \href{https://data.gouv.fr/dataset/5369926ba3a729239d203f0b}{Dépassements d'honoraires dans les cliniques privées}
\item \href{https://data.gouv.fr/dataset/5369952ca3a729239d20464d}{Eventail des cas}
\item \href{https://data.gouv.fr/dataset/536995cda3a729239d20480e}{Fréquence des séjours selon codes diagnostics principaux ou actes classant}
\item \href{https://data.gouv.fr/dataset/536997f9a3a729239d204e27}{Les catégories majeures de diagnostic par établissement de santé}
\item \href{https://data.gouv.fr/dataset/53699da2a3a729239d205ca4}{Présentation des bases}
\item \href{https://data.gouv.fr/dataset/53699ecfa3a729239d205f8f}{Référentiel de coûts pour l'activité de soins de suite et de réadaptation (SSR) des établissements de santé (DAF)}
\item \href{https://data.gouv.fr/dataset/53699ecfa3a729239d205f90}{Référentiel de coûts pour l'activité de soins de suite et de réadaptation (SSR) des établissements de santé (OQN)}
\item \href{https://data.gouv.fr/dataset/53699ed8a3a729239d205fa5}{Référentiel national de coûts 2011 pour les hôpitaux}
\item \href{https://data.gouv.fr/dataset/5369a0b5a3a729239d20641b}{Statistiques par GHM}
\item \href{https://data.gouv.fr/dataset/5369a33aa3a729239d2069fd}{Valeurs nationales de coûts pour l'hospitalisation à domicile en 2011}
\end{itemize}

\clearpage
\section{Airaq}


\begin{center}
  \includegraphics[width=3cm]{images/orga/10_623590087244f5b8037f2ae6d518ba-100.png}
\end{center}


Depuis 1980, la qualité de l'air ambiant fait l'objet d'une
réglementation communautaire.

En France, l'Etat confie la surveillance de la qualité de l'air à une
trentaine d'associations loi 1901, agréées chaque année par le Ministère
en charge de l'Ecologie. Elles constituent le Réseau National ATMO de
surveillance et d'Information sur l'Air.

AIRAQ est l'Association Agréée pour la Surveillance de la Qualité de
l'Air en Aquitaine * Services de l'Etat et de l'ADEME, * Collectivités :
Région, départements, communes, groupements de communes, * Entreprises
et activités (ou leur groupement) contribuant à l'émission de substances
surveillées (en particulier assujettis à la TGAP Air), * Associations
agréées de la protection de l'Environnement et de consommateurs,
professions de santé et personnalités qualifiées.

Mesurer Surveiller en permanence la qualité de l'air conformément à la
règlementation

Les polluants mesurés sont ceux pour lesquels des effets sur la santé ou
sur l'environnement ont été établis ou sont pressentis.

\begin{itemize}

\item
  dioxyde de soufre ( SO2 )
\item
  oxydes d'azote ( NOx )
\item
  particules fines ( PM10 et PM2.5 )
\item
  ozone ( O3 )
\item
  Métaux lourds
\item
  monoxyde de carbone ( CO )
\item
  benzène, toluène, Ethylbenzène, xylène ( BTEX )
\item
  certains métaux lourds (Arsenic, Nickel, Cadmium, Plomb)
\item
  les Hydrocarbures Aromatiques Polycycliques ( HAP )
\item
  certains produits phytosanitaires
\end{itemize}

Sur les 14 sites sous surveillance (Bordeaux, Pau, Bayonne, Périgueux,
Agen, Arcachon, Marmande, Mont de Marsan, Dax, Ambès, Lacq, Tartas,
Iraty et Le Temple ) sont réparties une ou plusieurs stations de mesures
fixes dans lesquelles se trouvent un ou plusieurs analyseurs
fonctionnant en automatique et mesurant des polluants spécifiques.

AIRAQ dispose aussi de 2 laboratoires mobiles de surveillance et de 6
préleveurs.


\vspace{0.5cm}

\needspace{12\baselineskip}
\subsection*{Surveillance de la qualité de l'air station proximité automobile de
Bordeaux Bastide
}\index{air}\index{circulation}\index{environnement}\index{pollution}
  \begin{wrapfigure}{r}{2.5cm}
    \centering
    \qrcode[nolink]{https://data.gouv.fr/dataset/5369a131a3a729239d206546}
  \end{wrapfigure}

Licence : \textbf{Licence Ouverte
}\newline
Créé le : 2013-09-18\newline
Modifié le : 2015-04-04\newline
Mise à jour : annuelle\newline
Popularité : 1 réutilisation,  0 suivi\newline
Mots-clé : \emph{air, circulation, environnement, pollution
}\newline
Permalien : \url{https://data.gouv.fr/dataset/5369a131a3a729239d206546}\newline

\par
\noindent
    En Aquitaine, AIRAQ est en charge de la surveillance de la qualité de
l'air. Le dispositif mis en place permet ainsi de connaitre en
permanence les concentrations relevées dans l'air ambiant.

AIRAQ a souhaité s'initier dans la démarche « inventaire » car cet outil
s'avère être précieux dans le cadre des missions des AASQA. En effet,
posséder un inventaire des émissions permet de répondre plus précisément
à la mission de surveillance de la pollution atmosphérique. Cette donnée
est nécessaire pour une meilleure modélisation de la qualité de l'air à
l'échelle régionale.


\vspace{0.5cm}
\needspace{12\baselineskip}
\subsection*{Surveillance de la qualité de l'air station proximité automobile de
Bordeaux Gambetta
}\index{air}\index{circulation}\index{environnement}\index{pollution}
  \begin{wrapfigure}{r}{2.5cm}
    \centering
    \qrcode[nolink]{https://data.gouv.fr/dataset/5369a132a3a729239d206547}
  \end{wrapfigure}

Licence : \textbf{Licence Ouverte
}\newline
Créé le : 2013-09-18\newline
Modifié le : 2015-09-28\newline
Mise à jour : annuelle\newline
Popularité : 1 réutilisation,  0 suivi\newline
Mots-clé : \emph{air, circulation, environnement, pollution
}\newline
Permalien : \url{https://data.gouv.fr/dataset/5369a132a3a729239d206547}\newline

\par
\noindent
    En Aquitaine, AIRAQ est en charge de la surveillance de la qualité de
l'air. Le dispositif mis en place permet ainsi de connaitre en
permanence les concentrations relevées dans l'air ambiant.

AIRAQ a souhaité s'initier dans la démarche « inventaire » car cet outil
s'avère être précieux dans le cadre des missions des AASQA. En effet,
posséder un inventaire des émissions permet de répondre plus précisément
à la mission de surveillance de la pollution atmosphérique. Cette donnée
est nécessaire pour une meilleure modélisation de la qualité de l'air à
l'échelle régionale.


\vspace{0.5cm}
\needspace{3\baselineskip} \rule{4cm}{0.25pt}\newline\textbf{Aussi disponible du même producteur :}\begin{itemize}
\item \href{https://data.gouv.fr/dataset/5369a130a3a729239d206543}{Surveillance de la qualité de l’air Station périurbaines de fond Ambès}
\item \href{https://data.gouv.fr/dataset/5369a131a3a729239d206544}{Surveillance de la qualité de l’air Station périurbaines de fond de Saint-Sulpice-et-Cameyrac}
\item \href{https://data.gouv.fr/dataset/5369a131a3a729239d206545}{Surveillance de la qualité de l’air Station périurbaines de fond Léognan}
\item \href{https://data.gouv.fr/dataset/5369a132a3a729239d206548}{Surveillance de la qualité de l’air station proximité automobile de Mérignac}
\item \href{https://data.gouv.fr/dataset/5369a133a3a729239d206549}{Surveillance de la qualité de l’air station urbaine de fond : Bassens}
\end{itemize}

\clearpage
\section{Airparif}


\begin{center}
  \includegraphics[width=3cm]{images/orga/b2_fa6dc8ae814e4782aec3e6020c2dac-100.jpg}
\end{center}


Observatoire indépendant de la qualité de l'air au service de la santé
et de l'action, Airparif est l'association chargée de la surveillance et
de l'information sur la qualité de l'air en Île-de-France, agréé par le
Ministère de la Transition Écologique et Solidaire. Airparif a pour
mission de mettre en œuvre une surveillance de la qualité de l'air et de
fournir une information fiable et régulière aux autorités et au public,
afin de permettre l'amélioration durable de la santé des franciliens et
de l'environnement. Airparif regroupe près de 150 membres qui sont
représentés au sein de 4 collèges : l'État, les collectivités
territoriales, les acteurs économiques et le monde associatif -
personnes qualifiées. Les actions d'Airparif se déclinent suivant trois
axes : Surveiller l'air respiré par les Franciliens , Comprendre la
pollution atmosphérique et ses impacts, Accompagner les Franciliens et
les partenaires d'Airparif pour améliorer la qualité de l'air.


\vspace{0.5cm}

\needspace{12\baselineskip}
\subsection*{Indices Qualité de l'air (Citeair) journaliers par polluant sur
l'Île-de-France, les départements, les communes franciliennes et les
arrondissements parisiens
}\index{qualite!de!lair}
  \begin{wrapfigure}{r}{2.5cm}
    \centering
    \qrcode[nolink]{https://data.gouv.fr/dataset/5a4651eb88ee380bb9eff81e}
  \end{wrapfigure}

Licence : \textbf{Open Data Commons Open Database License (ODbL)
}\newline
Créé le : 2017-12-29\newline
Modifié le : 2018-07-03\newline
De 2014-01-01 à 2018-02-28\newline
Granularité : à la commune\newline
Mise à jour : mensuelle\newline
Popularité : 2 réutilisations,  1 suivi\newline
Mots-clé : \emph{qualite-de-lair
}\newline
Permalien : \url{https://data.gouv.fr/dataset/5a4651eb88ee380bb9eff81e}\newline

\par
\noindent
    Les indices de qualité de l'air sont des outils de communication qui
permettent de décrire périodiquement sous une forme simple
(qualificatif, chiffre) l'état global de la qualité de l'air dans une
aire géographique donnée. Ces indices ne sont pas des outils de
déclenchement ou de gestion des actions prévues par les procédures
préfectorales en cas d'épisodes de pollution de l'air ambiant. Pour plus
d'information sur l'indice européen citeair
:\url{http://www.airparif.fr/reglementation/indice-qualite-air-europeen.}Les
indices donnés sont des indices de fond, à savoir hors influence directe
des sources de pollution. Les communes et arrondissements sont désignés
par le code insee, la région par ``0'' et les départements par leur
numéro (75, 77, 78, 91, 92, 93, 94, 95)


\vspace{0.5cm}
\needspace{3\baselineskip} \rule{4cm}{0.25pt}\newline\textbf{Aussi disponible du même producteur :}\begin{itemize}
\item \href{https://data.gouv.fr/dataset/5b50571188ee38324ff491b8}{Cartographie horaire des niveaux de polluants  en Île-de-France}
\item \href{https://data.gouv.fr/dataset/5b5046da88ee3821c064a580}{Concentrations annuelles moyennes de polluants dans l'air ambiant issues du réseau permanent de mesures en région Île-de-France sur 5 ans glissants}
\item \href{https://data.gouv.fr/dataset/5b4cbeabc751df5c2b5fae20}{Concentrations horaires de polluants dans l'air ambiant issues du réseau permanent de mesures automatiques en région Île-de-France}
\item \href{https://data.gouv.fr/dataset/5b4cc322c751df5c2b5fae21}{Concentrations mensuelles moyennes de polluants dans l'air ambiant issues du réseau permanent de mesures en région Île-de-France sur un an glissant}
\item \href{https://data.gouv.fr/dataset/5b504ebb88ee382b449fc376}{Concentrations moyennes annuelles des polluants réglementés en Île-de-France	}
\item \href{https://data.gouv.fr/dataset/5b4cc14bc751df5fa8ae0e76}{Concentrations moyennes journalières de polluants dans l'air ambiant issues du réseau permanent de mesures automatiques en région Île-de-France}
\item \href{https://data.gouv.fr/dataset/5b4cb9e1c751df54deb76e66}{Évaluation des émissions de polluants atmosphériques et de gaz à effet de serre totales de la région Île-de-France en 2015}
\item \href{https://data.gouv.fr/dataset/5b4cb383c751df4acc48b254}{Évaluation des émissions de polluants atmosphériques et de gaz à effet de serre totales par département de la région Île-de-France en 2015}
\item \href{https://data.gouv.fr/dataset/5b4cb666c751df4cf8bca9de}{Évaluation des émissions de polluants atmosphériques et de gaz à effet de serre totales par EPCI de la région Île-de-France en 2015}
\item \href{https://data.gouv.fr/dataset/5b4f33c588ee38263fdfb66e}{Exposition des populations et territoires aux polluants atmosphériques NO2 et PM10  pour la région Île-de-France}
\item \href{https://data.gouv.fr/dataset/5ba0ac15634f411171af593d}{Guide d'utilisation des flux de données}
\item \href{https://data.gouv.fr/dataset/5ba0b6fb8b4c411d4bdba5e1}{Indice de qualité de l'air (Atmo)}
\end{itemize}

\clearpage
\section{Assemblée nationale}


\begin{center}
  \includegraphics[width=3cm]{images/orga/2015-06-19_b9a6ba86634147d5b6c8758274bc657d_assemblee-nationale-100.jpg}
\end{center}


L'Assemblée nationale est l'une des deux chambres du Parlement français
au cœur de la démocratie comme le montrent les textes qui régissent le
fonctionnement de la Vème République. Représenter le peuple français,
légiférer et contrôler l'action du Gouvernement : telles sont les
missions des 577 députés élus au suffrage universel direct. Chaque
année, ce sont ainsi en moyenne une centaine de lois qui sont adoptées,
plus de 1000 heures de débats qui ont lieu dans l'hémicycle, entre 20
000 et 25 000 questions qui sont posées au Gouvernement par écrit ou par
oral, plus de 300 rapports qui sont adoptés par les commissions sur les
sujets les plus divers.

Le site principal et source pour l'open data de l'Assemblée nationale
est data\url{http://data.assemblee-nationale.fr}et les données Etalab en
sont un mirroir.


\vspace{0.5cm}

\needspace{12\baselineskip}
\subsection*{Questions au Gouvernement
}\index{gouvernement}\index{questions}\index{reponses}\index{seances!publiques}
  \begin{wrapfigure}{r}{2.5cm}
    \centering
    \qrcode[nolink]{https://data.gouv.fr/dataset/5588331ec751df7971a453b9}
  \end{wrapfigure}

Licence : \textbf{Licence Ouverte
}\newline
Créé le : 2015-06-22\newline
Modifié le : 2018-05-29\newline
Granularité : au pays\newline
Mise à jour : quotienne\newline
Popularité : 1 réutilisation,  0 suivi\newline
Mots-clé : \emph{gouvernement, questions, reponses, seances-publiques
}\newline
Permalien : \url{https://data.gouv.fr/dataset/5588331ec751df7971a453b9}\newline

\par
\noindent
    Ce jeu rassemble l'ensemble des questions posées au Gouvernement ainsi
que les réponses de ce dernier lors des séances publiques dédiées les
mardis et mercredis depuis le début de la quatorzième législature (juin
2012). Ces données sont disponibles en format XML et JSON.


\vspace{0.5cm}
\needspace{12\baselineskip}
\subsection*{Réserve parlementaire
}\index{deputes}\index{gouvernement}\index{repartition}\index{reserve}\index{reserve!parlementaire}
  \begin{wrapfigure}{r}{2.5cm}
    \centering
    \qrcode[nolink]{https://data.gouv.fr/dataset/558837cfc751df7991a453bc}
  \end{wrapfigure}

Licence : \textbf{Licence Ouverte
}\newline
Créé le : 2015-06-22\newline
Modifié le : 2016-03-07\newline
Granularité : à la commune\newline
Mise à jour : annuelle\newline
Popularité : 1 réutilisation,  9 suivis\newline
Mots-clé : \emph{deputes, gouvernement, repartition, reserve, reserve-parlementaire
}\newline
Permalien : \url{https://data.gouv.fr/dataset/558837cfc751df7991a453bc}\newline

\par
\noindent
    Ce jeu rassemble les informations relatives à la répartition de la
réserve parlementaire attribuée par les députés.


\vspace{0.5cm}
\needspace{3\baselineskip} \rule{4cm}{0.25pt}\newline\textbf{Aussi disponible du même producteur :}\begin{itemize}
\item \href{https://data.gouv.fr/dataset/5591aa74c751df3deca453b9}{Amendements XIV ième législature}
\item \href{https://data.gouv.fr/dataset/5b1e5e26c751df50d162459c}{LE BUDGET DE L'ASSEMBLÉE NATIONALE}
\item \href{https://data.gouv.fr/dataset/5588355ac751df7991a453bb}{Questions écrites}
\item \href{https://data.gouv.fr/dataset/55883443c751df7991a453ba}{Questions orales sans débat}
\item \href{https://data.gouv.fr/dataset/55883731c751df760ca453b9}{Représentants d'intéret}
\end{itemize}

\clearpage
\section{ATOUT FRANCE - Agence de développement touristique de la France}


\begin{center}
  \includegraphics[width=3cm]{images/orga/8f_fa6726d3ec4fcaa6407ad51ab5518d-100.png}
\end{center}


L'Agence de développement touristique de la France, est chargée par la
loi du 22 juillet 2009 sur le développement et la modernisation des
services touristiques, de contribuer au développement de l'industrie
touristique, premier secteur économique français et de l'ensemble de ses
acteurs. Atout France met en oeuvre tous les moyens nécessaires afin de
remplir les trois objectifs qui lui ont été fixés : * Promouvoir et
développer la Marque Rendez-vous en France à l'international, pour
donner envie au monde de visiter la France. * Adapter l'offre française
à la demande touristique nationale et internationale, améliorer la
qualité en s'appuyant à la fois sur nos services d'ingénierie et sur
notre connaissance des marchés émetteurs. * Accompagner chacun de nos
partenaires, privés comme publics, en vue d'accroître leur compétitivité
économique.


\vspace{0.5cm}

\needspace{12\baselineskip}
\subsection*{Hébergements collectifs classés en France
}
  \begin{wrapfigure}{r}{2.5cm}
    \centering
    \qrcode[nolink]{https://data.gouv.fr/dataset/59c36c6a88ee3826d5998a35}
  \end{wrapfigure}

Licence : \textbf{Licence Ouverte
}\newline
Créé le : 2017-09-21\newline
Modifié le : 2017-09-21\newline
Granularité : au point d'intérêt\newline
Mise à jour : quotienne\newline
Popularité : 1 réutilisation,  0 suivi\newline
Mots-clé : \emph{aucun
}\newline
Permalien : \url{https://data.gouv.fr/dataset/59c36c6a88ee3826d5998a35}\newline

\par
\noindent
    Liste des hébergements collectifs classés comprenant les données
publiques mentionnées à l'article 4 de l'arrêté du 23 décembre 2009
fixant les normes et la procédure de classement des hôtels de tourisme.
Les modes d'hébergement concernés sont : Hôtel de tourisme, Camping,
Village de vacances, Résidence de tourisme, Parcs résidentiels de
loisirs.


\vspace{0.5cm}

\clearpage
\section{Autolib' Métropole}


\begin{center}
  \includegraphics[width=3cm]{images/orga/2015-07-23_dbef63f006514137b6a72612d9705fd0_Autolib-metropole-100.jpg}
\end{center}


Autolib' Métropole est un syndicat mixte, que les communes ou
établissements publics d'Ile-de-France rejoignent sur la base du
volontariat pour déployer le service Autolib' sur leur territoire.

L'exploitation d'Autolib' a été confiée à la Société Autolib' (filiale
du Groupe Bolloré) via une délégation de service public.


\vspace{0.5cm}

\needspace{12\baselineskip}
\subsection*{Stations Autolib'
}\index{autolib}\index{autopartage}\index{bornes!de!recharge}
  \begin{wrapfigure}{r}{2.5cm}
    \centering
    \qrcode[nolink]{https://data.gouv.fr/dataset/55b10d70c751df7e3610ccb6}
  \end{wrapfigure}

Licence : \textbf{Licence Ouverte
}\newline
Créé le : 2015-07-23\newline
Modifié le : 2016-03-16\newline
Granularité : au point d'intérêt\newline
Mise à jour : trimestrielle\newline
Popularité : 1 réutilisation,  1 suivi\newline
Mots-clé : \emph{autolib, autopartage, bornes-de-recharge
}\newline
Permalien : \url{https://data.gouv.fr/dataset/55b10d70c751df7e3610ccb6}\newline

\par
\noindent
    L'ensemble des stations Autolib' ouvertes au public en Île-de-France.


\vspace{0.5cm}

\clearpage
\section{Autorité de régulation des communications électroniques et des postes (ARCEP)}


\begin{center}
  \includegraphics[width=3cm]{images/orga/71_d1d847d7ec45a8a5f7fc0845a6994c-100.png}
\end{center}


L'Autorité de régulation des communications électroniques et des postes
(ARCEP) a été créée par la loi du 26 juillet 1996 pour préparer et
accompagner l'ouverture à la concurrence du secteur et veiller à la
fourniture et au financement du service universel des
télécommunications. La loi du 20 mai 2005 relative à la régulation des
activités postales a étendu la compétence de l'Autorité au secteur
postal. Autorité administrative indépendante, l'ARCEP assure, au nom de
l'Etat, et sous le contrôle du Parlement et du juge, la régulation des
secteurs des communications électroniques et postales.


\vspace{0.5cm}

\needspace{12\baselineskip}
\subsection*{Baromètre du Numérique
}\index{administration!en!ligne}\index{agence!du!numerique}\index{arcep}\index{barometre}\index{cge}\index{credoc}\index{equipements}\index{numerique}\index{tic}\index{usages}
  \begin{wrapfigure}{r}{2.5cm}
    \centering
    \qrcode[nolink]{https://data.gouv.fr/dataset/5838300988ee386317c65bb3}
  \end{wrapfigure}

Licence : \textbf{Open Data Commons Open Database License (ODbL)
}\newline
Créé le : 2016-11-25\newline
Modifié le : 2018-12-05\newline
De 2007-01-01 à 2018-12-03\newline
Granularité : au pays\newline
Mise à jour : annuelle\newline
Popularité : 2 réutilisations,  4 suivis\newline
Mots-clé : \emph{administration-en-ligne, agence-du-numerique, arcep, barometre, cge, credoc, equipements, numerique, tic, usages
}\newline
Permalien : \url{https://data.gouv.fr/dataset/5838300988ee386317c65bb3}\newline

\par
\noindent
    Le \textbf{Baromètre du Numérique} est une étude de référence sur
l'adoption par les Français des \textbf{équipements et des usages
numériques}. Il est le fruit d'une collaboration entre le Conseil
général de l'économie (\textbf{CGE}) et l'Autorité de régulation des
communications électroniques et des postes (\textbf{Arcep}) depuis 2003,
à laquelle l'\textbf{Agence du Numérique} s'est associée depuis
l'édition 2016.

Chaque année, au mois de juin, le Centre de recherche pour l'étude et
l'observation des conditions de vie (\textbf{Credoc}) interroge en
face-à-face un \textbf{échantillon de plus de 2 000 personnes}
représentatif de la population française. Des questions leur sont posées
sur la nature de leurs équipements (téléphonie, ordinateur, tablette,
etc.) et de leurs usages numériques (réseaux sociaux, e-commerce,
administration en ligne, etc.).


\vspace{0.5cm}
\needspace{3\baselineskip} \rule{4cm}{0.25pt}\newline\textbf{Aussi disponible du même producteur :}\begin{itemize}
\item \href{https://data.gouv.fr/dataset/5b4371e2c751df037a54e146}{Base de population}
\item \href{https://data.gouv.fr/dataset/5c489d728b4c413564a1d8f8}{Découpage administratif des COM St Martin et St Barthélemy "Format Admin-Express"}
\item \href{https://data.gouv.fr/dataset/593aac1488ee384e8b6c4e74}{Données de l'observatoire postal}
\item \href{https://data.gouv.fr/dataset/547d8f73c751df4a9f090fca}{Données du dispositif de mesure de la qualité du service fixe d'accès à internet}
\item \href{https://data.gouv.fr/dataset/5940fe5088ee38123bb45d4d}{Etude sur les équipements et usages des PME et ETI}
\item \href{https://data.gouv.fr/dataset/571f335288ee3815cba19f12}{Indicateurs d’activité des opérateurs de communications électroniques}
\item \href{https://data.gouv.fr/dataset/5b2b6715c751df6acaf0c2ee}{Tableau bord du New Deal mobile}
\end{itemize}

\clearpage
\section{Base Adresse Nationale}


\begin{center}
  \includegraphics[width=3cm]{images/orga/18_18b929a270482faefabb18f5d2b4fd-100.png}
\end{center}


Elle est constituée par la collaboration entre:

\begin{itemize}

\item
  des acteurs nationaux tels que l'IGN, La Poste et la mission Etalab
\item
  des acteurs locaux tels que les collectivités, les communes, les SDIS,
\item
  des citoyens par exemple à travers le projet OpenStreetMap et
  l'association OpenStreetMap France.
\end{itemize}


\vspace{0.5cm}

\needspace{12\baselineskip}
\subsection*{Base Adresse Nationale
}
  \begin{wrapfigure}{r}{2.5cm}
    \centering
    \qrcode[nolink]{https://data.gouv.fr/dataset/5530fbacc751df5ff937dddb}
  \end{wrapfigure}

Licence : \textbf{Open Data Commons Open Database License (ODbL)
}\newline
Créé le : 2015-04-17\newline
Modifié le : 2018-11-28\newline
Granularité : au point d'intérêt\newline
Mise à jour : hebdomadaire\newline
Popularité : 29 réutilisations,  35 suivis\newline
Mots-clé : \emph{aucun
}\newline
Permalien : \url{https://data.gouv.fr/dataset/5530fbacc751df5ff937dddb}\newline

\par
\noindent
    La Base Adresse Nationale est une base de données qui a pour but de
référencer l'intégralité des adresses du territoire français.

Elle contient la position géographique de plus de 20 millions
d'adresses.

Elle est constituée par la collaboration entre Etalab, La Poste, l'IGN,
la DGFiP et OpenStreetMap France.

Le projet est co-gouverné par l'Administrateur Général des Données et le
Conseil National de l'Information Géographique.

Elle est diffusée sur le site
\href{https://adresse.data.gouv.fr}{adresse.data.gouv.fr} développé par
la mission Etalab de la Direction interministérielle du numérique et du
système d'information et de communication de l'État (DINSIC).

Les données sont disponibles sous licence ODbL 1.0.

Un \href{https://adresse.data.gouv.fr/api}{service de géocodage gratuit}
est mis à disposition par la mission Etalab.


\vspace{0.5cm}

\clearpage
\section{Bibliothèque nationale de France}


\begin{center}
  \includegraphics[width=3cm]{images/orga/f0_7d02be08b749529c3e58d34ff8c504-100.png}
\end{center}


La Bibliothèque nationale de France (BnF) a pour mission de collecter,
conserver et diffuser le patrimoine documentaire national.


\vspace{0.5cm}

\needspace{12\baselineskip}
\subsection*{Data.bnf.fr : les données de la BnF en RDF
}\index{archives}\index{auteur}\index{bibliotheque}\index{catalogue}\index{culture}\index{histoire}\index{litterature}\index{manuscrit}\index{patrimoine!numerise}\index{web!semantique}
  \begin{wrapfigure}{r}{2.5cm}
    \centering
    \qrcode[nolink]{https://data.gouv.fr/dataset/53699211a3a729239d203e0e}
  \end{wrapfigure}

Licence : \textbf{Licence Ouverte
}\newline
Créé le : 2013-07-08\newline
Modifié le : 2017-11-17\newline
Granularité : au pays\newline
Mise à jour : mensuelle\newline
Popularité : 2 réutilisations,  6 suivis\newline
Mots-clé : \emph{archives, auteur, bibliotheque, catalogue, culture, histoire, litterature, manuscrit, patrimoine-numerise, web-semantique
}\newline
Permalien : \url{https://data.gouv.fr/dataset/53699211a3a729239d203e0e}\newline

\par
\noindent
    La Bibliothèque nationale de France vous guide dans ses ressources
patrimoniales, en publiant des fiches de référence inédites sur les
auteurs, sur les œuvres et sur les thèmes. data.bnf.fr répertorie tous
les oeuvres pour un auteur, ainsi que toutes les éditions d'une œuvre,
avec les liens vers les documents en ligne sur Gallica lorsqu'ils ont
été numérisés. La sélection de ressources proposée est issue des
catalogues documentaires, des inventaires d'archives et de manuscrits,
ainsi que des collections numériques de la BnF. Elle valorise les grands
classiques de la littérature, de l'histoire et du droit. C'est en
fonction de vos usages et de vos préférences que l'ensemble sera
progressivement élargi à un nombre croissant de références
encyclopédiques. Le projet data.bnf.fr utilise les techniques du Web
sémantique qui favorisent une navigation plus fluide et plus intuitive
par les internautes entre les différents types de ressources.
L'utilisation de formats structurés pensés pour le Web (RDF) facilite
l'indexation, le partage et la réutilisation des données : la BnF met
ces données librement à votre disposition à condition d'en mentionner la
source.


\vspace{0.5cm}
\needspace{3\baselineskip} \rule{4cm}{0.25pt}\newline\textbf{Aussi disponible du même producteur :}\begin{itemize}
\item \href{https://data.gouv.fr/dataset/53698f51a3a729239d2036fb}{Base des reliures numérisées de la Bibliothèque nationale de France}
\item \href{https://data.gouv.fr/dataset/536c3d88a3a72933d8d1b392}{Bibliographie des éditions parisiennes du XVIe siècle}
\item \href{https://data.gouv.fr/dataset/536990caa3a729239d203ac1}{Collectes du web électoral par la BnF}
\item \href{https://data.gouv.fr/dataset/5a7c6d10c751df4dfc94dbd2}{Collectes thématiques du web par la BnF}
\item \href{https://data.gouv.fr/dataset/5369981ba3a729239d204e7c}{Les données complètes des catalogues de la BnF}
\item \href{https://data.gouv.fr/dataset/53699b12a3a729239d205643}{Observatoire du dépôt légal}
\end{itemize}

\clearpage
\section{Caisse nationale de l'assurance maladie}


\begin{center}
  \includegraphics[width=3cm]{images/orga/1e_3ae33c736f41fe8ed91bb9897e010d-100.jpg}
\end{center}


Assureur solidaire en santé, la Cnam définit les politiques de gestion
du risque et pilote le réseau d'organismes chargés de les mettre en
œuvre. Dans le cadre de la loi de réforme d'août 2004, la Caisse
nationale de l'assurance maladie (Cnam) a vu ses responsabilités
complétées, renforcées et soutenues par des outils qui lui permettent
d'optimiser le fonctionnement du système de soins. Elle met en œuvre le
parcours de soins coordonnés en plaçant le médecin traitant au cœur du
système. Elle veille à l'équilibre des dépenses avec les ressources
publiques qui lui sont affectées. Elle est ainsi l'acteur central du
système de soins dont elle assure la maîtrise médicalisée.


\vspace{0.5cm}

\needspace{12\baselineskip}
\subsection*{Annuaire santé de la Cnam
}\index{ameli}\index{annuaire!sante}\index{cnam}\index{cnamts}\index{coordonnees}\index{etablissement!de!sante}\index{medecin}\index{professionnels!de!sante}\index{sante}
  \begin{wrapfigure}{r}{2.5cm}
    \centering
    \qrcode[nolink]{https://data.gouv.fr/dataset/5ac4f20fc751df5567500c0a}
  \end{wrapfigure}

Licence : \textbf{Licence Ouverte
}\newline
Créé le : 2018-04-04\newline
Modifié le : 2019-03-01\newline
Granularité : au pays\newline
Popularité : 1 réutilisation,  3 suivis\newline
Mots-clé : \emph{ameli, annuaire-sante, cnam, cnamts, coordonnees, etablissement-de-sante, medecin, professionnels-de-sante, sante
}\newline
Permalien : \url{https://data.gouv.fr/dataset/5ac4f20fc751df5567500c0a}\newline

\par
\noindent
    L'annuaire santé répond à la mission générale d'information des assurés
en vue notamment de faciliter l'accès aux soins, confiée à l'Assurance
Maladie en application de l'article L.162-1-11 du Code de la Sécurité
sociale. Il permet de trouver les coordonnées des professionnels de
santé exerçant à titre libéral et celles des établissements de soins,
ainsi que les actes pratiqués. Il permet également de connaître le
secteur conventionnel auquel appartient un professionnel de santé, les
tarifs pratiqués, s'il accepte ou non la carte vitale, ou encore les
données tarifaires pour certaines prestations d'hospitalisation.

Le présent jeu de données contient des informations à caractère
personnel relatives aux professionnels de santé. Il est publié
conformément aux dispositions de
\href{https://www.legifrance.gouv.fr/affichCodeArticle.do?cidTexte=LEGITEXT000006072665\&idArticle=LEGIARTI000031923894\&dateTexte=\&categorieLien=cid}{l'article
L. 1461-2 du Code de la santé publique}. La réutilisation de ces données
est soumise au respect de la réglementation relative à la protection de
la vie privée.


\vspace{0.5cm}
\needspace{12\baselineskip}
\subsection*{Dépenses d' assurance maladie hors prestations hospitalières (données
nationales)
}\index{damir}\index{medecins}\index{prestations}\index{sante}
  \begin{wrapfigure}{r}{2.5cm}
    \centering
    \qrcode[nolink]{https://data.gouv.fr/dataset/54903bfac751df244d6accbc}
  \end{wrapfigure}

Licence : \textbf{Licence Ouverte
}\newline
Créé le : 2014-12-16\newline
Modifié le : 2018-09-14\newline
De 2010-01-01 à 2018-06-30\newline
Granularité : au pays\newline
Mise à jour : mensuelle\newline
Popularité : 7 réutilisations,  9 suivis\newline
Mots-clé : \emph{damir, medecins, prestations, sante
}\newline
Permalien : \url{https://data.gouv.fr/dataset/54903bfac751df244d6accbc}\newline

\par
\noindent
    Cette base de données contient l'ensemble des remboursements mensuels
effectués par le régime général de l'Assurance Maladie (hors prestations
hospitalières) par type de prestations (soins et prestations en
espèces), type d'exécutant (médecins par spécialité, chirurgiens
dentistes, auxiliaires médicaux, laboratoires d'analyse, pharmaciens,
\ldots{}) et par type de prescripteurs. Les dépenses sont indiquées en
montants remboursés et présentées au remboursement.

Concernant les dépenses de santé d'assurance maladie, trois autres jeux
de données, couvrant des champs différents, sont proposés : -
\href{https://www.data.gouv.fr/fr/datasets/depenses-d-assurance-maladie-hors-prestations-hospitalieres-par-caisse-primaire-departement/}{Dépenses
d'assurance maladie hors prestations hospitalières par caisse
primaire/département} : base de données regroupant l'ensemble des
remboursements (hors prestations hospitalières) effectués par le régime
général de l'Assurance Maladie en France ;

\begin{itemize}
\item
  \href{https://www.data.gouv.fr/fr/datasets/depenses-annuelles-d-assurance-maladie/}{Dépenses
  annuelles d'assurance maladie} : tableaux de bord portant sur
  l'ensemble des remboursements (y compris les prestations
  hospitalières) effectués par le régime général de l'Assurance Maladie
  en France métropolitaine ;
\item
  \href{https://www.data.gouv.fr/fr/datasets/open-damir-base-complete-sur-les-depenses-dassurance-maladie-inter-regimes/}{Open
  Damir : base complète sur les dépenses d'assurance maladie inter
  régimes} : base de données regroupant l'ensemble des remboursements (y
  compris prestations hospitalières facturées directement à l'Assurance
  Maladie) effectués par tous les régimes d'assurance maladie en France.
\end{itemize}


\vspace{0.5cm}
\needspace{12\baselineskip}
\subsection*{Dépenses d' assurance maladie hors prestations hospitalières par caisse
primaire/département
}\index{assurance!maladie}\index{damir}\index{prestations}\index{sante}
  \begin{wrapfigure}{r}{2.5cm}
    \centering
    \qrcode[nolink]{https://data.gouv.fr/dataset/54ca9d19c751df7e99467389}
  \end{wrapfigure}

Licence : \textbf{Licence Ouverte
}\newline
Créé le : 2015-01-29\newline
Modifié le : 2018-09-11\newline
De 2010-01-01 à 2016-12-31\newline
Granularité : au département\newline
Mise à jour : annuelle\newline
Popularité : 5 réutilisations,  4 suivis\newline
Mots-clé : \emph{assurance-maladie, damir, prestations, sante
}\newline
Permalien : \url{https://data.gouv.fr/dataset/54ca9d19c751df7e99467389}\newline

\par
\noindent
    Cette base de données contient l'ensemble des remboursements mensuels
effectués par le régime général de l'Assurance Maladie (hors prestations
hospitalières) \textbf{par caisse primaire/département}, par type de
prestations (soins et prestations en espèces), type d'exécutant
(médecins par spécialité, chirurgiens dentistes, auxiliaires médicaux,
laboratoires d'analyse, pharmaciens, \ldots{}) et par type de
prescripteurs. Les dépenses sont indiquées en montants remboursés et
présentées au remboursement.

Les tables mensuelles au format csv sont regroupées dans des fichiers
zip annuels. Deux présentations sont disponibles : une version complète
avec les libellés des variables et des regroupements pour certaines
variables et une version simplifiée, sans les libellés et sans les
regroupements. Les utilisateurs des tables simplifiées pourront se
rapporter au lexique de la nomenclature.

Concernant les dépenses de santé d'assurance maladie, trois autres jeux
de données, couvrant des champs différents, sont proposés :

\begin{itemize}
\item
  \href{https://www.data.gouv.fr/fr/datasets/depenses-d-assurance-maladie-hors-prestations-hospitalieres-donnees-nationales/}{Dépenses
  d'assurance maladie hors prestations hospitalières (données
  nationales)} : base de données regroupant l'ensemble des
  remboursements (hors prestations hospitalières) effectués par le
  régime général de l'Assurance Maladie en France ;
\item
  \href{https://www.data.gouv.fr/fr/datasets/depenses-annuelles-d-assurance-maladie/}{Dépenses
  annuelles d'assurance maladie}: tableaux de bord portant sur
  l'ensemble des remboursements (y compris les prestations
  hospitalières) effectués par le régime général de l'Assurance Maladie
  en France métropolitaine ;
\item
  \href{https://www.data.gouv.fr/fr/datasets/open-damir-base-complete-sur-les-depenses-dassurance-maladie-inter-regimes/}{Open
  Damir : base complète sur les dépenses d'assurance maladie inter
  régimes} : base de données regroupant l'ensemble des remboursements (y
  compris prestations hospitalières facturées directement à l'Assurance
  Maladie) effectués par tous les régimes d'assurance maladie en France.
\end{itemize}


\vspace{0.5cm}
\needspace{3\baselineskip} \rule{4cm}{0.25pt}\newline\textbf{Aussi disponible du même producteur :}\begin{itemize}
\item \href{https://data.gouv.fr/dataset/53698e59a3a729239d203439}{Actes de biologie médicale remboursés par l'Asurance Maladie }
\item \href{https://data.gouv.fr/dataset/550998a0c751df5433882844}{Activité des professionnels de santé libéraux}
\item \href{https://data.gouv.fr/dataset/53698f85a3a729239d203783}{Bénéficiaires de la CMU de base par département}
\item \href{https://data.gouv.fr/dataset/53698f86a3a729239d203785}{Bénéficiaires de la CMU de base par région}
\item \href{https://data.gouv.fr/dataset/55099acbc751df55ef882844}{Honoraires des professionnels de santé libéraux}
\item \href{https://data.gouv.fr/dataset/55fbe32188ee3813c2cda4f5}{Medicines refunded by the National Health Insurance }
\item \href{https://data.gouv.fr/dataset/58d3c14bc751df6883298f1c}{Open Bio: base complète sur les dépenses de biologie médicale interrégimes}
\item \href{https://data.gouv.fr/dataset/5b45f757c751df102525972a}{Open LPP : base complète sur les dépenses de dispositifs médicaux inscrits à la liste de produits et prestations (LPP) interrégimes}
\item \href{https://data.gouv.fr/dataset/55099b8dc751df57a6882844}{Patientèle des professionnels de santé libéraux}
\end{itemize}

\clearpage
\section{Caisse Nationale des Allocations familiales }


\begin{center}
  \includegraphics[width=3cm]{images/orga/2015-07-02_a1bc08e4a4dc44a89cb29697800a83b5_new_logo-100.jpg}
\end{center}


Cafdata, l'Open Data des Allocations familiales, est ouvert.

Rendez-vous sur\url{http://data.caf.fr}


\vspace{0.5cm}

\needspace{12\baselineskip}
\subsection*{Adresses des établissements d'accueil du jeune enfant percevant une
prestation de service Caf et nombre de places offertes.
}\index{carte}\index{cartographie}\index{commune}\index{creche}\index{eaje}\index{halte!garderie}\index{jeune!enfant}
  \begin{wrapfigure}{r}{2.5cm}
    \centering
    \qrcode[nolink]{https://data.gouv.fr/dataset/560d915ca3a7294a8d94aa0f}
  \end{wrapfigure}

Licence : \textbf{Licence Ouverte
}\newline
Créé le : 2016-06-22\newline
Modifié le : 2019-03-12\newline
Popularité : 1 réutilisation,  0 suivi\newline
Mots-clé : \emph{carte, cartographie, commune, creche, eaje, halte-garderie, jeune-enfant
}\newline
Permalien : \url{https://data.gouv.fr/dataset/560d915ca3a7294a8d94aa0f}\newline

\par
\noindent
    Ce jeu de données restitue l'adresse et le nombre de places de chaque
établissement d'accueil du jeune enfant (EAJE) implanté sur le
territoire français, et percevant une prestation de service versée par
la Caf au titre de l'exercice considéré.

Les EAJE sont des structures qui sont autorisées à accueillir de manière
non permanente, des enfants de moins de 6 ans. Ils regroupent plusieurs
catégories d'établissements conçus et aménagés afin de recevoir dans la
journée, collectivement ou chez un(e) assistant(e) maternel(le) exerçant
au sein d'un service d'accueil familial, de façon régulière ou
occasionnelle, ces enfants, sous la responsabilité de professionnels de
la petite enfance. Ils sont soumis au respect d'une réglementation
prévue dans le code de la santé publique (articles R 2324-16 et
suivants) et font l'objet d'une autorisation d'ouverture délivrée par
l'autorité compétente (Président du Conseil général pour les
gestionnaires privés et maire de la commune pour les gestionnaires
publics après avis des services de protection maternelle et infantile
(PMI). Les locaux respectent les normes de sécurité exigées pour les
établissements recevant du public et sont aménagés de façon à favoriser
l'éveil des enfants.\\
La capacité d'accueil s'exprime en nombre de places. Elle est définie
dans l'autorisation de fonctionnement délivrée par l'autorité compétente
après instruction des services de la protection maternelle et infantile
(Pmi). L'autorisation peut prévoir des capacités d'accueil différentes
suivant les périodes de l'année, de la semaine ou de la journée, compte
tenu des variations prévisibles des besoins d'accueil.

Les caisses d'Allocations familiales (Caf) contribuent au développement
de l'offre d'accueil en versant une subvention de fonctionnement aux
gestionnaires des EAJE : la prestation de service unique (PSU). La
prestation de service unique prend en charge 66 \% du prix de revient
horaire de l'accueil de l'enfant dans la limite du prix plafond fixé
annuellement par la Cnaf, déduction faite des participations familiales.
Elle est versée uniquement pour les enfants relevant du régime général.
Pour bénéficier de ces financements, les structures doivent répondre à
plusieurs conditions :

− être autorisées à fonctionner par les autorités compétentes ;

− être ouvertes à toute la population ;

− calculer les participations des familles à partir du barème national
des participations des familles précitées ;

− signer une convention d'objectifs et de financement avec la Caf.

Un même gestionnaire peut administrer plusieurs établissements, et un
établissement peut être divisé en plusieurs équipements, accueillant
chacun un certain nombre d'enfants.


\vspace{0.5cm}
\needspace{12\baselineskip}
\subsection*{Nombre de pièces arrivées
}\index{arrive}\index{piece}
  \begin{wrapfigure}{r}{2.5cm}
    \centering
    \qrcode[nolink]{https://data.gouv.fr/dataset/560d9160a3a7294ad6ff2d01}
  \end{wrapfigure}

Licence : \textbf{Licence Ouverte
}\newline
Créé le : 2016-06-22\newline
Modifié le : 2019-03-12\newline
Popularité : 1 réutilisation,  0 suivi\newline
Mots-clé : \emph{arrive, piece
}\newline
Permalien : \url{https://data.gouv.fr/dataset/560d9160a3a7294ad6ff2d01}\newline

\par
\noindent
    Il s'agit de l'ensemble des pièces arrivées liées à la gestion des
dossiers allocataires, quel que soit le domaine (PF, AS, contentieux,
etc.) et le lieu de réception (service courrier, guichets, centres
extérieurs, prestataires, etc.) Doivent être comptées l'ensemble des
pièces de l'ensemble des courriers, quel que soit le support (papier,
fax, courrier électronique, etc), son mode de traitement (masse ou
flux), le (ou les) service(s) destinataire(s) (Prestations Familiales,
Action Sociale, Contentieux \ldots{}) et le lieu de réception (service
courrier, guichets de la caisse, centres extérieurs de la caisse, centre
informatique ou prestataire de service pour le compte de la caisse),
qu'elles soient enregistrées ou non par SDP.

Seules les pièces arrivant de l'extérieur (flux entrants) et les flux
internes générant une charge doivent être comptés, les flux sortants ne
doivent pas être comptés.

La définition de courrier suit la convention adoptée dans SDP : un
courrier comptabilisé par SDP correspond à l'ensemble des pièces
arrivées le même jour pour un même allocataire (matricule identique).

Cet indicateur ne constitue qu'un élément partiel de l'appréciation de
la charge dans la mesure où le traitement d'une pièce peut être plus ou
moins long selon le type de pièce concerné.

Dans cet indicateur, certaines fluctuations observées mensuellement
peuvent résulter d'opérations ponctuelles (arrivées de déclarations de
ressources, des certificats de scolarité, campagnes d'information
locale, activités particulières d'action sociale, etc..).

La comptabilisation au mois le mois des pièces réceptionnées par des
prestataires de service peut s'avérer difficile. La caisse communiquera
leur chiffrage dès qu'elle en aura connaissance et un recyclage mensuel
affecté au(x) mois indiqué par elle sera réalisé.

L'affectation des pièces aux différents services (Prestations Légales,
Aides Financières Individuelles d'Action Sociale) s'effectue en
utilisant la Corbeille électronique grâce au service distribution
Corbeille. Par défaut, cette répartition est faite sur la base du
service auquel sont affectées les pièces dans le répertoire des pièces
SDP.

Le respect des préconisations (cf.~Documentations fonctionnelles de SDP)
quant à l'utilisation des codes état dans SDP garantit la cohérence
entre le champ défini pour l'indicateur et la réalité du calcul.


\vspace{0.5cm}
\needspace{3\baselineskip} \rule{4cm}{0.25pt}\newline\textbf{Aussi disponible du même producteur :}\begin{itemize}
\item \href{https://data.gouv.fr/dataset/595930d6a3a7291dcf9c824b}{Adresses des foyers jeunes travailleurs percevant une prestation de service Caf}
\item \href{https://data.gouv.fr/dataset/595930e3a3a7291dd09c81e3}{Adresses des lieux d'accueil enfants parents (LAEP) percevant une prestation de service de la Caf}
\item \href{https://data.gouv.fr/dataset/560d9160b595086cd501d755}{Allocation aux adultes handicapés (AAH) - par Caf}
\item \href{https://data.gouv.fr/dataset/560d915db595086cd501d74e}{Allocation de base de la Paje - par Caf}
\item \href{https://data.gouv.fr/dataset/560d915da3a7294ad6ff2cfc}{Allocation d'éducation de l'enfant handicapé (AEEH) - par Caf}
\item \href{https://data.gouv.fr/dataset/560d915db595086cd601d74f}{Allocation de logement familiale (ALF) - par Caf}
\item \href{https://data.gouv.fr/dataset/560d915da3a7294a8d94aa11}{Allocation de logement sociale (ALS) - par Caf}
\item \href{https://data.gouv.fr/dataset/560d915db595086cd501d74f}{Allocation de rentrée scolaire (ARS) - par Caf}
\item \href{https://data.gouv.fr/dataset/560d915da3a7294a8e94aa11}{Allocation de soutien familial (ASF) - par Caf}
\item \href{https://data.gouv.fr/dataset/560d915cb595086cd501d74b}{Amplitude d’ouverture de l’accueil physique}
\item \href{https://data.gouv.fr/dataset/560d915ca3a7294ad7ff2cfa}{Amplitude d’ouverture de l’accueil téléphonique}
\item \href{https://data.gouv.fr/dataset/560d915ca3a7294a8e94aa0f}{Antériorité du solde < 10 jours}
\item \href{https://data.gouv.fr/dataset/560d915cb595086cd601d74c}{Antériorité du solde <= 15 jours ouvrés}
\item \href{https://data.gouv.fr/dataset/560d915da3a7294a8d94aa10}{Attente à l'accueil <= 20 minutes}
\item \href{https://data.gouv.fr/dataset/560d9160b595086cd601d755}{Bénéficiaires de l’allocation aux adultes handicapés au cours du mois de décembre - par Commune}
\item \href{https://data.gouv.fr/dataset/560d915eb595086cd601d752}{Bénéficiaires  tous régimes de prestations familiales et sociales gérées par la branche Famille}
\item \href{https://data.gouv.fr/dataset/560d915cb595086cd501d74c}{Caractéristiques des foyers allocataires - Niveau national}
\item \href{https://data.gouv.fr/dataset/560d915fa3a7294ad6ff2cfe}{Complément de libre choix d'activité de la Paje (CLCA et COLCA) - par Caf}
\item \href{https://data.gouv.fr/dataset/560d915ea3a7294a8e94aa12}{Complément familial (CF) - par Caf}
\item \href{https://data.gouv.fr/dataset/560d915fa3a7294a8e94aa14}{Complément mode de garde (Paje) - par Caf}
\item \href{https://data.gouv.fr/dataset/560d915da3a7294ad7ff2cfb}{Délai de traitement des minima sociaux}
\item \href{https://data.gouv.fr/dataset/560d915da3a7294ad6ff2cfb}{Dénombrement et répartition des foyers allocataires selon l’âge du responsable dossier - par commune}
\item \href{https://data.gouv.fr/dataset/5b9b57d106e3e7772e2ffd31}{Dénombrement et répartition des foyers allocataires selon l’âge du responsable dossier - par EPCI}
\item \href{https://data.gouv.fr/dataset/560d9161b595086cd501d758}{Dépenses par prestation - National}
\item \href{https://data.gouv.fr/dataset/560d915db595086cd501d74d}{Enfants couverts par l'allocation de rentrée scolaire  (ARS) - par Caf}
\item \href{https://data.gouv.fr/dataset/560d915ca3a7294ad6ff2cfa}{Foyers allocataires à "bas revenus"  - par commune}
\item \href{https://data.gouv.fr/dataset/5b9b57d406e3e7772e2ffd32}{Foyers allocataires à "bas revenus" - par EPCI}
\item \href{https://data.gouv.fr/dataset/560d915da3a7294ad7ff2cfc}{Foyers allocataires percevant au moins une prestation légale et dénombrement des foyers allocataires par prestation - Niveau national}
\item \href{https://data.gouv.fr/dataset/560d915db595086cd601d74e}{Foyers allocataires percevant au moins une prestation légale - par Caf}
\item \href{https://data.gouv.fr/dataset/595930dea3a7291dd09c81e1}{Foyers allocataires percevant la prime d'activité (PPA) - niveau national}
\item \href{https://data.gouv.fr/dataset/59606638a3a7296407d6a7c1}{Foyers allocataires percevant la prime d'activité (PPA) - par Commune}
\item \href{https://data.gouv.fr/dataset/560d915ea3a7294a8d94aa12}{Foyers allocataires percevant la prime naissance ou adoption de la Paje - par Caf}
\item \href{https://data.gouv.fr/dataset/5b9b57e406e3e778bb2ffd31}{Foyers allocataires percevant le revenu de solidarité active (RSA) - par EPCI}
\item \href{https://data.gouv.fr/dataset/560d915eb595086cd601d751}{Foyers allocataires percevant le revenu de solidarité (RSO) - par commune d'Outre-mer}
\item \href{https://data.gouv.fr/dataset/5b9b57e606e3e778bb2ffd32}{Foyers allocataires percevant le revenu de solidarité (RSO) - par EPCI d'Outre-mer}
\item \href{https://data.gouv.fr/dataset/595930daa3a7291dcf9c824e}{Foyers allocataires percevant les Allocations familiales (AF) par commune}
\item \href{https://data.gouv.fr/dataset/5b9b57e706e3e778bb2ffd33}{Foyers allocataires percevant les Allocations familiales (AF) par EPCI}
\item \href{https://data.gouv.fr/dataset/560d9160b595086cd601d757}{Foyers  allocataires percevant  une aide au logement  en décembre par commune}
\item \href{https://data.gouv.fr/dataset/560d915fa3a7294ad7ff2cff}{Foyers allocataires percevant une aide au logement - par Caf}
\item \href{https://data.gouv.fr/dataset/560d915fa3a7294ad6ff2cff}{Foyers allocataires percevant une prestation enfance et jeunesse (AF, CF, ASF, AEEH et ARS) - par commune}
\item \href{https://data.gouv.fr/dataset/5b9b57eb06e3e778bb2ffd34}{Foyers allocataires percevant une prestation enfance et jeunesse (AF, CF, ASF, AEEH et ARS) - par EPCI}
\item \href{https://data.gouv.fr/dataset/595930e2a3a7291dcf9c824f}{Historique des bénéficiaires de prestations familiales et sociales tous régimes et CAF (France entière, Métropole et DOM)}
\item \href{https://data.gouv.fr/dataset/595930e2a3a7291dd09c81e2}{Historique des dépenses de prestations familiales et sociales tous régimes (France entière, Métropole, Dom)}
\item \href{https://data.gouv.fr/dataset/5bb7137506e3e71f74293eb8}{Indicateurs sur les assistants maternels - Accueil du jeune enfant}
\item \href{https://data.gouv.fr/dataset/560d915eb595086cd501d751}{Indicateur sur la part des prestations dans les ressources des foyers allocataires - par Commune}
\item \href{https://data.gouv.fr/dataset/560d915fb595086cd601d753}{Nombre d’allocataires Bornes interactives}
\item \href{https://data.gouv.fr/dataset/560d915fa3a7294a8d94aa15}{Nombre d’allocataires Internet}
\item \href{https://data.gouv.fr/dataset/560d9160a3a7294a8e94aa15}{Nombre d'appels téléphoniques aboutis}
\item \href{https://data.gouv.fr/dataset/560d915fa3a7294ad7ff2d00}{Nombre d'appels téléphoniques traités}
\item \href{https://data.gouv.fr/dataset/560d9160a3a7294ad6ff2d00}{Nombre de courriers arrivés}
\item et 40 autres jeux de données\end{itemize}

\clearpage
\section{Centre des monuments nationaux }


\begin{center}
  \includegraphics[width=3cm]{images/orga/57_97877b36d049b9809d2c4709da246f-100.jpg}
\end{center}


CMN


\vspace{0.5cm}

\needspace{12\baselineskip}
\subsection*{Liste des coordonnées GPS des monuments nationaux
}\index{culture}\index{monuments}
  \begin{wrapfigure}{r}{2.5cm}
    \centering
    \qrcode[nolink]{https://data.gouv.fr/dataset/536998eea3a729239d2050ca}
  \end{wrapfigure}

Licence : \textbf{Licence Ouverte
}\newline
Créé le : 2013-10-20\newline
Modifié le : 2016-08-04\newline
Mise à jour : annuelle\newline
Popularité : 2 réutilisations,  4 suivis\newline
Mots-clé : \emph{culture, monuments
}\newline
Permalien : \url{https://data.gouv.fr/dataset/536998eea3a729239d2050ca}\newline

\par
\noindent
    Liste des coordonnées GPS des monuments nationaux


\vspace{0.5cm}
\needspace{3\baselineskip} \rule{4cm}{0.25pt}\newline\textbf{Aussi disponible du même producteur :}\begin{itemize}
\item \href{https://data.gouv.fr/dataset/53698e6ba3a729239d203468}{Activités d'accueil des tournages et filmographies dans les monuments nationaux}
\item \href{https://data.gouv.fr/dataset/536990bea3a729239d203aa3}{Classement des monuments nationaux par catégorie}
\item \href{https://data.gouv.fr/dataset/536995d3a3a729239d20481e}{Fréquentation des monuments nationaux}
\item \href{https://data.gouv.fr/dataset/5873ae1fc751df465e41ed71}{Informations pratiques des monuments (horaires, contacts, prix)}
\item \href{https://data.gouv.fr/dataset/5369994ba3a729239d2051f5}{Liste des publications des éditions du patrimoine}
\item \href{https://data.gouv.fr/dataset/5369a165a3a729239d2065c7}{Tarifs des prises de vues cinématographiques dans les monuments nationaux}
\item \href{https://data.gouv.fr/dataset/5369a165a3a729239d2065c8}{Tarifs des prises de vues photographiques dans les monuments nationaux}
\end{itemize}

\clearpage
\section{Centre National des Œuvres Universitaires et Scolaires}


\begin{center}
  \includegraphics[width=3cm]{images/orga/17_b3b9199a2b49bd8a3d2723cc4954fc-100.jpg}
\end{center}


Le Centre National des Œuvres Universitaires et Scolaires
(\textbf{CNOUS}) constitue un réseau avec les vingt-huit Centres
Régionaux des Œuvres Universitaires (\textbf{CROUS}). Le CNOUS et les
CROUS s'engagent résolument vers l'ouverture des données produites par
nos systèmes d'information. Vous trouverez principalement dans ces jeux
de données : - des lieux de restauration avec leurs caractéristiques -
le détail des menus quotidiens servis dans nos structures de
restauration - les résidences universitaires avec leurs caractéristiques

Nous sommes à votre écoute concernant ces jeux de données, leurs
réutilisations et la mise à disposition de données non publiées
actuellement. Pour cela, une seule adresse : \textbf{opendata@cnous.fr}


\vspace{0.5cm}

\needspace{12\baselineskip}
\subsection*{Logements étudiants
}\index{crous}\index{etudiant}\index{handicape}\index{logement}\index{residence}
  \begin{wrapfigure}{r}{2.5cm}
    \centering
    \qrcode[nolink]{https://data.gouv.fr/dataset/5548d994c751df32e0a7b26c}
  \end{wrapfigure}

Licence : \textbf{Licence Ouverte
}\newline
Créé le : 2015-05-05\newline
Modifié le : 2017-10-27\newline
Granularité : à la région\newline
Mise à jour : quotienne\newline
Popularité : 1 réutilisation,  0 suivi\newline
Mots-clé : \emph{crous, etudiant, handicape, logement, residence
}\newline
Permalien : \url{https://data.gouv.fr/dataset/5548d994c751df32e0a7b26c}\newline

\par
\noindent
    Par région, flux recensant les logements proposés aux étudiants par le
réseau des \textbf{CROUS}. Les services associés renseignés :

\begin{itemize}

\item
  description
\item
  accessibilité des personnes à mobilité réduite
\item
  moyens d'accès
\item
  localisation GPS
\item
  loyers des logements par type
\item
  laverie, gardien et gardien
\item
  restaurant et cafétéria à proximité
\item
  coordonnées de contact
\end{itemize}


\vspace{0.5cm}
\needspace{3\baselineskip} \rule{4cm}{0.25pt}\newline\textbf{Aussi disponible du même producteur :}\begin{itemize}
\item \href{https://data.gouv.fr/dataset/5548d35cc751df0767a7b26c}{Actualités des CROUS}
\item \href{https://data.gouv.fr/dataset/55f27f8988ee383ebda46ec1}{Menus des restaurants, brasseries et cafétérias}
\item \href{https://data.gouv.fr/dataset/55f28fe088ee386774a46ec2}{Restaurants, brasseries et cafétérias des CROUS}
\end{itemize}

\clearpage
\section{Centre national du cinéma et de l'image animée}


\begin{center}
  \includegraphics[width=3cm]{images/orga/20_04deb5b8e84f3aae457be5728ead64-100.png}
\end{center}


Le succès du cinéma et de l'audiovisuel français est avant tout le
résultat du génie créatif français, mais seuls les pays qui développent
une politique publique ambitieuse peuvent préserver et développer une
industrie nationale de l'image animée. La manière la plus astucieuse de
faire -- et la plus économe pour les finances publiques -- c'est au CNC
qu'on la doit. Soixante-cinq ans après sa création par le Gouvernement
de Jean Monnet en 1946, le CNC reste une construction originale. C'est
un établissement public, qui dispose de recettes affectées pour apporter
des soutiens aux arts de l'image animée, et c'est aussi une
administration centrale, en charge de ce secteur sous l'autorité du
ministre de la Culture et de la Communication. Chargé de financer les
créateurs d'aujourd'hui et de demain, et de réguler les marchés du
cinéma et de l'audiovisuel, sa mission est aussi d'entretenir et de
valoriser la mémoire du passé. Venant au soutien des auteurs et des
artistes les plus singuliers, il veille aussi à la santé d'une industrie
qui les emploie. Profondément ancré dans le cinéma, le Centre intervient
désormais sur tous les champs de l'image animée, des œuvres
audiovisuelles jusqu'aux univers numériques interactifs.


\vspace{0.5cm}

\needspace{12\baselineskip}
\subsection*{Films ayant réalisé plus d'un million d'entrées
}\index{cinema}\index{frequentation}
  \begin{wrapfigure}{r}{2.5cm}
    \centering
    \qrcode[nolink]{https://data.gouv.fr/dataset/5437e88988ee387cb58f5e85}
  \end{wrapfigure}

Licence : \textbf{Licence Ouverte
}\newline
Créé le : 2014-10-10\newline
Modifié le : 2015-07-24\newline
De 2003-01-01 à 2013-12-31\newline
Granularité : au point d'intérêt\newline
Mise à jour : annuelle\newline
Popularité : 1 réutilisation,  0 suivi\newline
Mots-clé : \emph{cinema, frequentation
}\newline
Permalien : \url{https://data.gouv.fr/dataset/5437e88988ee387cb58f5e85}\newline

\par
\noindent
    Films ayant réalisé plus d'un million d'entrées


\vspace{0.5cm}
\needspace{12\baselineskip}
\subsection*{Liste des établissements cinématographiques actifs
}\index{art!et!essai}\index{cinema}\index{ecrans!cinema}\index{fauteuils!cinema}\index{salles!de!cinema}
  \begin{wrapfigure}{r}{2.5cm}
    \centering
    \qrcode[nolink]{https://data.gouv.fr/dataset/5437eb4e88ee387cb58f5e86}
  \end{wrapfigure}

Licence : \textbf{Licence Ouverte
}\newline
Créé le : 2014-10-10\newline
Modifié le : 2015-12-09\newline
De 2003-01-01 à 2013-12-31\newline
Granularité : à la commune\newline
Mise à jour : annuelle\newline
Popularité : 1 réutilisation,  2 suivis\newline
Mots-clé : \emph{art-et-essai, cinema, ecrans-cinema, fauteuils-cinema, salles-de-cinema
}\newline
Permalien : \url{https://data.gouv.fr/dataset/5437eb4e88ee387cb58f5e86}\newline

\par
\noindent
    Liste des établissements cinématographiques actifs


\vspace{0.5cm}
\needspace{3\baselineskip} \rule{4cm}{0.25pt}\newline\textbf{Aussi disponible du même producteur :}\begin{itemize}
\item \href{https://data.gouv.fr/dataset/545215a9c751df25e7edf3d6}{Audience de la télévision}
\item \href{https://data.gouv.fr/dataset/549825b9c751df483404805b}{Classement des films les plus diffusés sur les chaînes nationales gratuites depuis 1957}
\item \href{https://data.gouv.fr/dataset/54983728c751df635f04805b}{Consommation des ménages en vidéo à la demande}
\item \href{https://data.gouv.fr/dataset/54817bdec751df6e6252aa9a}{Dépenses des ménages en programmes audiovisuels}
\item \href{https://data.gouv.fr/dataset/5429829c88ee380327a59168}{distribution des films dans les salles de cinéma}
\item \href{https://data.gouv.fr/dataset/54983cafc751df7f3b04805f}{Données internationales sur le cinéma}
\item \href{https://data.gouv.fr/dataset/549949b5c751df74d104805a}{Durée de vie des films inédits (en salle)}
\item \href{https://data.gouv.fr/dataset/547eedf1c751df5197048e8b}{Exportation de programmes  audiovisuels}
\item \href{https://data.gouv.fr/dataset/54511377c751df28685a81f1}{Exportations des films cinématographiques}
\item \href{https://data.gouv.fr/dataset/548185d9c751df7dea52aa9a}{Films à la télévision }
\item \href{https://data.gouv.fr/dataset/547decfcc751df4105090fca}{Financement de la télévision}
\item \href{https://data.gouv.fr/dataset/54510dc1c751df1fdb5a81f3}{Géographie du cinéma : équipement et fréquentation }
\item \href{https://data.gouv.fr/dataset/549845f4c751df7b7504805a}{La diffusion en salles de courts métrages}
\item \href{https://data.gouv.fr/dataset/54982478c751df44c104805a}{Les meilleures audience des films à la télévision}
\item \href{https://data.gouv.fr/dataset/54dde5c2c751df66a8467389}{Liste des établissements cinématographiques avec leur adresse}
\item \href{https://data.gouv.fr/dataset/5499446fc751df6a86048068}{Liste des films cinématographiques agréés}
\item \href{https://data.gouv.fr/dataset/5498359fc751df61f004805b}{Marché de la vidéo physique}
\item \href{https://data.gouv.fr/dataset/5437e96d88ee387cb28f5e7d}{Meilleurs succès du cinéma depuis 1945}
\item \href{https://data.gouv.fr/dataset/54983672c751df020604805b}{Pratiques des consommateurs de vidéo physique}
\item \href{https://data.gouv.fr/dataset/54297d0588ee380326a59163}{Production de films cinématographiques - données statistiques}
\item \href{https://data.gouv.fr/dataset/547dee5cc751df51ff090fca}{Programmes audiovisuels (offre, consommation, coût de grille, investissements)}
\item \href{https://data.gouv.fr/dataset/549838c1c751df66f904805a}{Public de la vidéo à la demande}
\item \href{https://data.gouv.fr/dataset/542c11ce88ee386c42cb3929}{Public des films}
\item \href{https://data.gouv.fr/dataset/54340d0788ee38736ab698a2}{Public régional du cinéma}
\item \href{https://data.gouv.fr/dataset/54340b8e88ee387368b698a2}{Public selon les catégories d'établissements cinématographiques}
\item \href{https://data.gouv.fr/dataset/5498383ec751df7f3b04805c}{Références actives en vidéo à la demande}
\item \href{https://data.gouv.fr/dataset/549826ebc751df4bb304805b}{Télévision de rattrapage (ou à la demande) : offre, consommation et usage}
\end{itemize}

\clearpage
\section{CNIL}


\begin{center}
  \includegraphics[width=3cm]{images/orga/43_4f3dcc999b4e34b7cccf57f64f0b17-100.jpg}
\end{center}


La Commission nationale de l'informatique et des libertés est l'autorité
administrative indépendante en charge de la protection des données
personnelles en France.

En tant que régulateur, elle veille au respect du règlement général sur
la protection des données (RGPD) et de la loi Informatique et Libertés
modifiée :

\begin{itemize}
\item
  elle \textbf{conseille et accompagne} les responsables de projets
  numériques ;
\item
  elle accompagne les \textbf{délégués à la protection des données
  (DPO)} désignés par les entreprises, les associations et les services
  publics ;
\item
  elle analyse les conséquences des \textbf{innovations technologiques}
  sur la vie privée et les libertés, et émet des
  \textbf{recommandations} ;
\item
  elle \textbf{autorise} les traitements de données présentant une
  \textbf{sensibilité particulière} ;
\item
  elle a un pouvoir de \textbf{contrôle} et de \textbf{sanction
  administrative} ;
\item
  enfin, elle travaille en \textbf{étroite collaboration avec ses
  homologues} européens et internationaux.
\end{itemize}


\vspace{0.5cm}

\needspace{12\baselineskip}
\subsection*{Contrôles réalisés par la CNIL
}\index{cnil}\index{controle}\index{fichiers}\index{informatique!et!libertes}
  \begin{wrapfigure}{r}{2.5cm}
    \centering
    \qrcode[nolink]{https://data.gouv.fr/dataset/555c431bc751df5800190c78}
  \end{wrapfigure}

Licence : \textbf{Licence Ouverte
}\newline
Créé le : 2015-05-20\newline
Modifié le : 2018-06-20\newline
Granularité : à la commune\newline
Mise à jour : annuelle\newline
Popularité : 2 réutilisations,  3 suivis\newline
Mots-clé : \emph{cnil, controle, fichiers, informatique-et-libertes
}\newline
Permalien : \url{https://data.gouv.fr/dataset/555c431bc751df5800190c78}\newline

\par
\noindent
    Pour vérifier le respect de la loi Informatique et Libertés, la CNIL a
la possibilité de contrôler les fichiers enregistrant des données
personnelles.

Ce contrôle peut s'exercer :

\begin{itemize}
\item
  \textbf{sur place} (dans les locaux du responsable du fichier) ;
\item
  \textbf{sur convocation} (dans les locaux de la CNIL) ;
\item
  \textbf{sur pièces} (demande de documents) ;
\item
  et, depuis 2014, \textbf{en ligne} (contrôle de sites internet).
\end{itemize}

Cinq jeux de données sont publiés :

\begin{enumerate}
\def\labelenumi{\arabic{enumi}.}
\item
  nombre et types de contrôles réalisés chaque année depuis 1990 (csv et
  xls) ;
\item
  liste des contrôles réalisés en 2014 (csv et xls) ;
\item
  liste des contrôles réalisés en 2015 (csv et xls) ;
\item
  liste des contrôles réalisés en 2016 (csv et xls) ;
\item
  liste des contrôles réalisés en 2017 (csv et xls).
\end{enumerate}


\vspace{0.5cm}
\needspace{12\baselineskip}
\subsection*{Correspondants Informatique et Libertés (CIL)
}\index{cil}\index{cnil}\index{correspondant}\index{donnees!personnelles}
  \begin{wrapfigure}{r}{2.5cm}
    \centering
    \qrcode[nolink]{https://data.gouv.fr/dataset/555b5673c751df4821190c78}
  \end{wrapfigure}

Licence : \textbf{Licence Ouverte
}\newline
Créé le : 2015-05-19\newline
Modifié le : 2018-05-24\newline
De 2005-10-20 à 2018-05-24\newline
Granularité : à la commune\newline
Mise à jour : mensuelle\newline
Popularité : 4 réutilisations,  8 suivis\newline
Mots-clé : \emph{cil, cnil, correspondant, donnees-personnelles
}\newline
Permalien : \url{https://data.gouv.fr/dataset/555b5673c751df4821190c78}\newline

\par
\noindent
    Le correspondant Informatique et Libertés (CIL) était chargé de veiller,
avant l'entrée en application du règlement général sur la protection des
données (RGPD) le 25 mai 2018, au respect de la loi Informatique et
Libertés au sein de l'entreprise, du groupe, de l'association ou de
l'administration qui l'avait désigné.

Cette désignation était facultative.

La CNIL publie la liste des organismes privés et publics qui avaient
souhaité s'engager dans une démarche de conformité en désignant un CIL
avant la mise en place, par le RGPD, du
\href{https://www.cnil.fr/fr/devenir-delegue-la-protection-des-donnees}{délégué
à la protection des données (DPO)}.


\vspace{0.5cm}
\needspace{12\baselineskip}
\subsection*{Les délibérations de la CNIL
}\index{cnil}\index{commission}\index{deliberations}\index{donnees!personnelles}\index{informatique!et!libertes}
  \begin{wrapfigure}{r}{2.5cm}
    \centering
    \qrcode[nolink]{https://data.gouv.fr/dataset/53ca2dc6a3a7294a1ddd7848}
  \end{wrapfigure}

Licence : \textbf{Licence Ouverte
}\newline
Créé le : 2014-07-18\newline
Modifié le : 2017-07-26\newline
Granularité : à la commune\newline
Mise à jour : mensuelle\newline
Popularité : 4 réutilisations,  3 suivis\newline
Mots-clé : \emph{cnil, commission, deliberations, donnees-personnelles, informatique-et-libertes
}\newline
Permalien : \url{https://data.gouv.fr/dataset/53ca2dc6a3a7294a1ddd7848}\newline

\par
\noindent
    Toutes les délibérations de la CNIL depuis l'origine, la première
délibération ayant été rendue en 1979.

Leurs modalités de publication sont définies par la loi Informatique et
Libertés, par son décret d'application ainsi que par le règlement
intérieur de la Commission.

Les délibérations adoptées entre le 1er janvier 1979 et le 3 mai 2012
ont été publiées dans leur version mise à jour au plus tard le 4 mai
2012. Les délibérations adoptées postérieurement au 3 mai 2012 portent
mention de leur date de publication effective.

Les délibérations de la CNIL sont également accessibles sur Legifrance.


\vspace{0.5cm}
\needspace{12\baselineskip}
\subsection*{Protection des données personnelles dans le monde
}\index{afapdp}\index{cnil}\index{donnees!personnelles}\index{informatique!et!libertes}\index{monde}\index{protection!des!donnees}
  \begin{wrapfigure}{r}{2.5cm}
    \centering
    \qrcode[nolink]{https://data.gouv.fr/dataset/5728c163c751df28b64e291f}
  \end{wrapfigure}

Licence : \textbf{Licence Ouverte
}\newline
Créé le : 2016-05-03\newline
Modifié le : 2019-01-24\newline
Granularité : au pays\newline
Mise à jour : annuelle\newline
Popularité : 1 réutilisation,  2 suivis\newline
Mots-clé : \emph{afapdp, cnil, donnees-personnelles, informatique-et-libertes, monde, protection-des-donnees
}\newline
Permalien : \url{https://data.gouv.fr/dataset/5728c163c751df28b64e291f}\newline

\par
\noindent
    Tous les pays ne bénéficient pas d'une loi Informatique et Libertés.

Quels pays disposent d'une législation spécifique ou d'une autorité de
protection des données personnelles ? Quelles précautions prendre pour
transférer des données personnelles à l'international ?

La CNIL publie les différents niveaux de protection des données et les
coordonnées des ``CNIL'' à travers le monde.

Elle met également une cartographie interactive à disposition
\href{https://www.cnil.fr/fr/la-protection-des-donnees-dans-le-monde}{sur
son site}.


\vspace{0.5cm}
\needspace{3\baselineskip} \rule{4cm}{0.25pt}\newline\textbf{Aussi disponible du même producteur :}\begin{itemize}
\item \href{https://data.gouv.fr/dataset/555b6af2c751df4bb8190c7a}{Droit d'accès indirect (données générales)}
\item \href{https://data.gouv.fr/dataset/555b5347c751df49c8190c79}{Effectifs de la CNIL}
\item \href{https://data.gouv.fr/dataset/555c53c0c751df5d48190c78}{Marchés publics de la CNIL }
\item \href{https://data.gouv.fr/dataset/591acfbb88ee387bd5a92ab7}{Mises en demeure prononcées par la CNIL}
\item \href{https://data.gouv.fr/dataset/591af43d88ee3826b379093a}{Sanctions prononcées par la CNIL}
\end{itemize}

\clearpage
\section{Commissariat général à l'égalité des territoires}


\begin{center}
  \includegraphics[width=3cm]{images/orga/11_7d320c3ebc4048906eed0d200cbf73-100.png}
\end{center}


Le Commissariat général à l'égalité des territoires (CGET) est un
service de l'État placé sous l'autorité de la ministre de la Cohésion
des territoires et des Relations avec les collectivités territoriales.
Il appuie le Gouvernement dans la lutte contre les inégalités
territoriales et le soutien aux dynamiques territoriales, en concevant
et animant les politiques de la ville et d'aménagement du territoire
avec les acteurs locaux et les citoyens.


\vspace{0.5cm}

\needspace{12\baselineskip}
\subsection*{Fiches d'identité des Maisons de services au public
}\index{maisons!de!services!au!public}\index{msap}\index{services!au!public}\index{services!publics}
  \begin{wrapfigure}{r}{2.5cm}
    \centering
    \qrcode[nolink]{https://data.gouv.fr/dataset/5a9675b4c751df529aad26f2}
  \end{wrapfigure}

Licence : \textbf{Licence Ouverte
}\newline
Créé le : 2018-02-28\newline
Modifié le : 2018-06-27\newline
Popularité : 1 réutilisation,  2 suivis\newline
Mots-clé : \emph{maisons-de-services-au-public, msap, services-au-public, services-publics
}\newline
Permalien : \url{https://data.gouv.fr/dataset/5a9675b4c751df529aad26f2}\newline

\par
\noindent
    Fiches d'identité des Maisons de services au public, complétées par
chaque porteur de Maison de services au public (collectivité,
association, La Poste). Dans ce fichier, sont présentes les informations
suivantes : - adresse ; - horaires d'ouverture ; - contacts (téléphone
et mail) ; - informations pratiques d'accès ; - informations relatives à
la prise en charge de handicap ; - URL vers les réseaux sociaux ; -
partenaires nationaux déclarés (parmi Pôle emploi, La Caisse nationale
d'assurance maladie des travailleurs salariés (Assurance maladie), la
Caisse nationale d'assurance vieillesse (Assurance retraite), la Caisse
nationale des allocations familiales (Cnaf), la caisse centrale de la
mutualité sociale agricole (CCMSA), La Poste, GRDF) et des partenaires
locaux ; - équipements mis à disposition ; - autres services délivrés au
sein de la Maison ou dans les mêmes locaux.


\vspace{0.5cm}
\needspace{12\baselineskip}
\subsection*{Zone d'Aide à Finalité Régionale (AFR)
}\index{afr}\index{aide}\index{aide!a!finalite!regionale}\index{cget}\index{finalite}\index{regionale}
  \begin{wrapfigure}{r}{2.5cm}
    \centering
    \qrcode[nolink]{https://data.gouv.fr/dataset/5943d4c188ee38742a95eb0d}
  \end{wrapfigure}

Licence : \textbf{Licence Ouverte
}\newline
Créé le : 2017-06-16\newline
Modifié le : 2017-06-23\newline
Granularité : à la commune\newline
Mise à jour : ponctuelle\newline
Popularité : 1 réutilisation,  0 suivi\newline
Mots-clé : \emph{afr, aide, aide-a-finalite-regionale, cget, finalite, regionale
}\newline
Permalien : \url{https://data.gouv.fr/dataset/5943d4c188ee38742a95eb0d}\newline

\par
\noindent
    La Commission européenne a adopté la carte française des zones d'aides à
finalité régionale (AFR) pour la période 2014-2020, mise en œuvre par le
\href{https://www.legifrance.gouv.fr/affichTexte.do?cidTexte=JORFTEXT000029181847\&categorieLien=id}{décret
n\degree{} 2014-758 du 2 juillet 2014 relatif aux zones d'aide à
finalité régionale (AFR) et aux zones d'aide à l'investissement des
petites et moyennes entreprises} modifié par le
\href{https://www.legifrance.gouv.fr/jo_pdf.do?numJO=0\&dateJO=20151101\&numTexte=26\&pageDebut=20481\&pageFin=20482}{décret
n\degree{} 2015-1391 du 30 octobre 2015} et le
\href{https://www.legifrance.gouv.fr/affichTexte.do?cidTexte=JORFTEXT000034503194\&dateTexte=\&categorieLien=id}{décret
n\degree{} 2017-648 du 26 avril 2017}.

Elle remplace la carte des zones AFR 2007-2013 venue à expiration le 30
juin 2014.

Cette nouvelle carte délimite les zones, conditions et limites dans
lesquelles l'Etat et les collectivités locales pourront allouer aux
entreprises des aides à l'investissement et à la création d'emploi.

Elle détermine les taux plafonds d'aide à l'investissement qui varient
selon la fragilité des territoires, conformément aux règles européennes
:

\begin{itemize}

\item
  10 \% du coût des investissements productifs pour les grandes
  entreprises en métropole ;
\item
  45 à 70 \% du coût des investissements productifs pour les grandes
  entreprises dans les DOM ;
\item
  des bonifications de taux de 10 \% pour les moyennes entreprises et de
  20 \% pour les petites entreprises sont prévues dans chacune de ces
  zones.
\end{itemize}

Les zones AFR sont éligibles jusqu'au 31/12/2020.

Vous pouvez consulter la liste des communes classées en AFR pour la
période 2014-2020 sur le
\href{http://www.observatoire-des-territoires.gouv.fr/observatoire-des-territoires/fr/zones-daide-\%C3\%A0-finalit\%C3\%A9-r\%C3\%A9gionale}{site
de l'Observatoire des territoires}. Vous pouvez alors rechercher votre
commune et zoomer sur cette commune pour en apprécier le classement.

Pour plus d'informations se reporter au site du
\href{http://www.cget.gouv.fr/}{CGET}.


\vspace{0.5cm}
\needspace{12\baselineskip}
\subsection*{Zones de Revitalisation Rurale (ZRR)
}\index{cget}\index{revitalisation}\index{rurale}\index{zone}\index{zones}\index{zones!de!revitalisation!rurale}\index{zrr}
  \begin{wrapfigure}{r}{2.5cm}
    \centering
    \qrcode[nolink]{https://data.gouv.fr/dataset/5943d13588ee38742a95eb0c}
  \end{wrapfigure}

Licence : \textbf{Licence Ouverte
}\newline
Créé le : 2017-06-16\newline
Modifié le : 2018-06-08\newline
Granularité : à la commune\newline
Mise à jour : ponctuelle\newline
Popularité : 1 réutilisation,  1 suivi\newline
Mots-clé : \emph{cget, revitalisation, rurale, zone, zones, zones-de-revitalisation-rurale, zrr
}\newline
Permalien : \url{https://data.gouv.fr/dataset/5943d13588ee38742a95eb0c}\newline

\par
\noindent
    La liste des communes classées en zones de revitalisation rurale (ZRR)
selon l'arrêté du 22 février 2018 modifiant l'arrêté du 16 mars 2017 est
parue. L'Observatoire des Territoires du CGET met à disposition ce
nouveau classement sur son site internet et son application de
cartographie interactive.

\textbf{Comment les communes sont-elles désormais classées en ZRR ?}

La réforme des ZRR, votée en loi de finances rectificative pour 2015
(article 1465A du code général des impôts), a simplifié les critères de
classement des territoires pris en compte. Les critères sont désormais
examinés à l'échelon intercommunal et entrainent le classement de
l'ensemble des communes de l'EPCI.

Pour être classé en ZRR au 1er juillet 2017, l'EPCI doit avoir à la fois
: - une densité de population inférieure ou égale à la médiane des
densités par EPCI ; - un revenu fiscal par unité de consommation médian
inférieur ou égal à la médiane des revenus fiscaux médians.

Pour les DOM, les communes classées en ZRR sont définies par la loi.

La loi de finances pour 2018 a créé une nouvelle condition de classement
en ZRR, relative à la baisse de population au niveau de l'EPCI depuis 40
ans. Seules les communes de l'EPCI de Decazeville communauté sont
concernées.

Le classement des communes en ZRR est valable jusqu'au 31 décembre 2020.

Les communes précédemment classées en ZRR et qui ne le sont plus
bénéficient du maintien des effets du classement en ZRR jusqu'au 30 juin
2020 : - pour les communes de montagne en application de la loi de
modernisation de développement et de protection des territoires de
montagne du 28 décembre 2016 ; - pour les autres communes en application
de la loi de finances pour 2018.

Les arrêtes de classement sont les arrêtés du 16 mars et du 22 février
2018.

\textbf{Où consulter la liste des communes classées en ZRR ?}

Vous pouvez consulter la liste des communes classées en ZRR au 22
février 2018 sur le
\href{http://www.observatoire-des-territoires.gouv.fr/observatoire-des-territoires/fr/zones-de-revitalisation-rurale-zrr}{site
de l'Observatoire des territoires}. Vous pouvez alors rechercher votre
commune et zoomer sur cette commune et en apprécier son classement.


\vspace{0.5cm}
\needspace{3\baselineskip} \rule{4cm}{0.25pt}\newline\textbf{Aussi disponible du même producteur :}\begin{itemize}
\item \href{https://data.gouv.fr/dataset/5accb8e9c751df39d7b5a0eb}{Contrats urbains de cohésion sociale (Cucs)}
\item \href{https://data.gouv.fr/dataset/5b1f897388ee387593c0466e}{Quartiers de veille active}
\item \href{https://data.gouv.fr/dataset/5b27839288ee3827eee7079a}{Quartiers prioritaires de la politique de la ville - Habitat ancien dégradé}
\item \href{https://data.gouv.fr/dataset/5a69f47e88ee385c0c053a21}{Subventions politique de la ville}
\item \href{https://data.gouv.fr/dataset/5acc7eddc751df5e21efdf20}{Villes bénéficiaires du plan Action cœur de ville}
\item \href{https://data.gouv.fr/dataset/5a5784f9c751df1b5b1d530f}{Zones franches urbaines (ZFU)}
\item \href{https://data.gouv.fr/dataset/5a67414888ee385f0ca521eb}{Zones urbaines sensibles (ZUS)}
\end{itemize}

\clearpage
\section{Commission d’accès aux documents administratifs (CADA)}


\begin{center}
  \includegraphics[width=3cm]{images/orga/19_8d825fd63e408e8b287bbbb6070633-100.png}
\end{center}


La \href{http://www.cada.fr}{Commission d'accès aux documents
administratifs} est une autorité administrative indépendante et
consultative chargée de veiller à la liberté d'accès aux documents
administratifs. Sa composition garantit son indépendance.

Son rôle est principalement de rendre des avis sur le refus opposé par
l'administration aux demandes de communication des particuliers, des
entreprises, des associations, des syndicats et d'autres
d'administrations. Sa saisine est obligatoire avant tout recours
contentieux.

Elle conseille également les administrations sur le caractère
communicable des documents qu'elles détiennent, et peut être consultée
par le gouvernement sur des textes législatifs ou réglementaires, ou en
proposer la modification.

Elle informe le public sur le droit d'accès.


\vspace{0.5cm}

\needspace{12\baselineskip}
\subsection*{Avis et conseils de la CADA
}\index{avis}\index{cada}\index{conseil}\index{demandes}\index{documents!administratifs}
  \begin{wrapfigure}{r}{2.5cm}
    \centering
    \qrcode[nolink]{https://data.gouv.fr/dataset/53698f37a3a729239d2036a0}
  \end{wrapfigure}

Licence : \textbf{Licence Ouverte
}\newline
Créé le : 2014-04-11\newline
Modifié le : 2018-12-05\newline
De 1984-03-03 à 2018-09-06\newline
Mise à jour : mensuelle\newline
Popularité : 2 réutilisations,  9 suivis\newline
Mots-clé : \emph{avis, cada, conseil, demandes, documents-administratifs
}\newline
Permalien : \url{https://data.gouv.fr/dataset/53698f37a3a729239d2036a0}\newline

\par
\noindent
    Ce jeu de données contient les avis rendus par la commission d'accès aux
documents administratifs (CADA) ainsi que ses conseils à destination des
administrations, tels que défini par
\href{https://www.cada.fr/lacada/le-role-de-la-cada}{son rôle}.

Ces données ont été libérées suite à la décision n\degree{}25 du CIMAP
du 18 décembre 2013 concernant la suppression des redevances sur
certains jeux de données.

Les 4 000 avis et conseils présents jusqu'à fin 2012 constituent un
panel représentatif des demandes dont la commission a été saisie depuis
sa création en 1978.

Ils sont pour l'essentiel concentrés sur les années 2000-2012.

A partir de fin 2012, l'intégralité des avis et conseils est disponible.
Les nouveaux avis émis (et anonymisés) sont ajoutés sous la forme d'un
nouveau fichier CSV.


\vspace{0.5cm}

\clearpage
\section{Commission Nationale des Comptes de Campagne et des Financements Politiques (CNCCFP)}


\begin{center}
  \includegraphics[width=3cm]{images/orga/88_a391c214884532bdf097497dc8e9ff-100.png}
\end{center}


La commission a été créée par la loi n\degree{} 90-55 du 15 janvier 1990
relative à la limitation des dépenses électorales et à la clarification
du financement des activités politiques. Elle a été mise en place le 19
juin 1990.

La loi du 15 janvier 1990 définit la commission comme un organisme
collégial. Le Conseil constitutionnel a ajouté que la commission est une
``autorité administrative et non une juridiction'' (décision 91-1141 du
31 juillet 1991). Le Conseil d'État dans son rapport public 2001 avait
classé la commission dans les autorités administratives indépendantes,
statut qui a été juridiquement consacré par l'ordonnance n\degree{}
2003-1165 du 8 décembre 2003 portant simplifications administratives en
matière électorale.

Son rôle est de contrôler les comptes de campagne des candidats à
l'élection présidentielle (depuis 2007) et aux élections européennes,
législatives, sénatoriales (pour la première fois en 2014), régionales,
cantonales (puis départementales à partir de 2014), municipales,
territoriales et provinciales (0utre-Mer) dans les circonscriptions de
plus de 9000 habitants et de vérifier le respect par les partis de leurs
obligations comptables et financières.


\vspace{0.5cm}

\needspace{12\baselineskip}
\subsection*{Comptes des partis et groupements politiques
}\index{cnccfp}\index{comptes}\index{controle}\index{financement}\index{partis!politiques}\index{transparence}\index{vie!politique}
  \begin{wrapfigure}{r}{2.5cm}
    \centering
    \qrcode[nolink]{https://data.gouv.fr/dataset/53699158a3a729239d203c2f}
  \end{wrapfigure}

Licence : \textbf{Licence Ouverte
}\newline
Créé le : 2013-12-11\newline
Modifié le : 2019-01-11\newline
Granularité : au pays\newline
Mise à jour : annuelle\newline
Popularité : 1 réutilisation,  5 suivis\newline
Mots-clé : \emph{cnccfp, comptes, controle, financement, partis-politiques, transparence, vie-politique
}\newline
Permalien : \url{https://data.gouv.fr/dataset/53699158a3a729239d203c2f}\newline

\par
\noindent
    Données comptables des partis et groupements politiques issues des
comptes d'ensemble déposés à la commission au titre de l'article 11-7 de
la loi n\degree{} 88-226 du 11 mars 1988 relative à la transparence
financière de la vie politique. Figurent les postes comptables du bilan
et du compte de résultat tels que prévus par l'avis n\degree{} 95-02
relatif à la comptabilité des partis et groupements politiques. Certains
comptes déposés n'ont pas fait l'objet d'une certification par un ou
deux commissaires aux comptes. Concernant la décision de la CNCCFP
relative au respect des obligations comptables des partis, il convient
de se reporter à la publication générale des comptes des partis au titre
de l'exercice concerné publiée au Journal officiel de la République
française. Les données correspondent aux informations comptables
retranscrites des comptes déposés à la commission. L'élaboration des
comptes et l'imputation comptable des dépenses et recettes relèvent de
la compétence exclusive des partis. Seules les erreurs matérielles de
présentation ou les déséquilibres apparents dus à l'utilisation de
sommes arrondies ont été rectifiées. Chaque parti politique dispose d'un
numéro d'identification donné par la CNCCFP. Ainsi, en cas de changement
de dénomination du parti sans changement de personnalité morale, le
numéro du parti est conservé. ATTENTION : certains comptes peuvent être
en Francs CFP et non en euros.


\vspace{0.5cm}
\needspace{3\baselineskip} \rule{4cm}{0.25pt}\newline\textbf{Aussi disponible du même producteur :}\begin{itemize}
\item \href{https://data.gouv.fr/dataset/59848c13c751df68019a68af}{Compte de campagne - Elections partielles pour l'année 2016}
\item \href{https://data.gouv.fr/dataset/568f7829c751df18cfc664bc}{Comptes de campagne – Election des représentants au Parlement européen du 25 mai 2014}
\item \href{https://data.gouv.fr/dataset/5b645c528b4c414d40af517e}{Comptes de campagne – Élections législatives des 11 et 18 juin 2017}
\item \href{https://data.gouv.fr/dataset/568f7708c751df0fe6c664bc}{Comptes de campagne - Elections municipales des 23 et 30 mars 2014}
\item \href{https://data.gouv.fr/dataset/59848cb2c751df74d4875afe}{Comptes de campagne - Elections partielles pour l'année 2016}
\item \href{https://data.gouv.fr/dataset/59848d5fc751df68019a68b0}{Comptes de campagne - Elections partielles pour l'année 2016}
\item \href{https://data.gouv.fr/dataset/59847830c751df556bf78845}{Comptes de campagne - élections partielles pour l'année 2016 }
\item \href{https://data.gouv.fr/dataset/5bee813e8b4c41676c51fc8e}{Comptes de campagne - Elections sénatoriales du 24 septembre 2017}
\item \href{https://data.gouv.fr/dataset/568141c988ee38427faf0bf7}{Comptes de campagne - Elections sénatoriales du 28 septembre 2014}
\item \href{https://data.gouv.fr/dataset/59848b4ec751df72f1012615}{Comptes de campagnes - Elections partielles pour l'année 2016}
\item \href{https://data.gouv.fr/dataset/5819ed5288ee383645c65bb3}{Elections régionales des 6 et 13 décembre 2015}
\item \href{https://data.gouv.fr/dataset/537d6f18a3a72973a2dc028a}{Tableau général des élections législatives des 10 et 17 juin 2012}
\end{itemize}

\clearpage
\section{Communauté d'Agglomération Bar-le-Duc Sud Meuse}


\begin{center}
  \includegraphics[width=3cm]{images/orga/50_57ae17a70e4c8e96dc74e50f80c748-100.jpg}
\end{center}


Meuse Grand Sud


\vspace{0.5cm}

\needspace{12\baselineskip}
\subsection*{Liste des marchés de travaux 2015
}\index{marches!publics}
  \begin{wrapfigure}{r}{2.5cm}
    \centering
    \qrcode[nolink]{https://data.gouv.fr/dataset/56fd36e1c751df6df7c485cb}
  \end{wrapfigure}

Licence : \textbf{Licence Ouverte
}\newline
Créé le : 2016-03-31\newline
Modifié le : 2016-08-05\newline
Granularité : à l'EPCI\newline
Mise à jour : ponctuelle\newline
Popularité : 1 réutilisation,  0 suivi\newline
Mots-clé : \emph{marches-publics
}\newline
Permalien : \url{https://data.gouv.fr/dataset/56fd36e1c751df6df7c485cb}\newline

\par
\noindent
    Publication légale (article 133) des marchés de travaux passés par la
Communauté d'Agglomération Bar-le-Duc Sud Meuse en 2015


\vspace{0.5cm}
\needspace{3\baselineskip} \rule{4cm}{0.25pt}\newline\textbf{Aussi disponible du même producteur :}\begin{itemize}
\item \href{https://data.gouv.fr/dataset/594d1b8088ee387c6695abf7}{Actualités}
\item \href{https://data.gouv.fr/dataset/594d1bd988ee387c6695abf8}{Agenda}
\item \href{https://data.gouv.fr/dataset/56b2196fc751df3584c30483}{Annuaire des Associations de la Communauté d'Agglomération Meuse Grand Sud}
\item \href{https://data.gouv.fr/dataset/56b218ebc751df41eac30482}{Annuaire des équipements de la Communauté d'Agglomération Meuse Grand Sud}
\item \href{https://data.gouv.fr/dataset/56b21877c751df3201c30482}{Annuaire des services de la Communauté d'Agglomération Meuse Grand Sud}
\item \href{https://data.gouv.fr/dataset/587f43b288ee3869dc9b81a5}{Budget annexe Assainissement 2016}
\item \href{https://data.gouv.fr/dataset/587f695d88ee382aa59b81a5}{Budget annexe Bâtiment Industriel  2016}
\item \href{https://data.gouv.fr/dataset/587f6dd588ee382aa59b81a6}{Budget annexe Centre Affaires 2016}
\item \href{https://data.gouv.fr/dataset/587f7d57c751df3d3fae0a65}{Budget annexe Eau 2016}
\item \href{https://data.gouv.fr/dataset/587f7ffec751df5cfaae0a65}{Budget annexe Les annonciades 2016}
\item \href{https://data.gouv.fr/dataset/587f830388ee3827119b81a5}{Budget annexe Lotissement CA 2016}
\item \href{https://data.gouv.fr/dataset/587f84bb88ee38588a9b81a4}{Budget annexe Ordure Ménagère 2016}
\item \href{https://data.gouv.fr/dataset/587f880888ee38588b9b81a6}{Budget annexe SIVU 2016}
\item \href{https://data.gouv.fr/dataset/587f896988ee3856cb9b81a5}{Budget annexe Transport 2016}
\item \href{https://data.gouv.fr/dataset/587f71a5c751df2d62ae0a65}{Budget CIAS 2016}
\item \href{https://data.gouv.fr/dataset/587f74cb88ee3827119b81a4}{Budget CIAS Coquillotes 2016}
\item \href{https://data.gouv.fr/dataset/587f77e288ee382aa59b81a8}{Budget CIAS Ehpad 2016}
\item \href{https://data.gouv.fr/dataset/587f7a7b88ee382a7b9b81a4}{Budget CIAS SSIAD 2016}
\item \href{https://data.gouv.fr/dataset/57a4a34288ee384951d73ff9}{Budget primitif 2016 de la Communauté d'Agglomération Bar-le-Duc Sud Meuse}
\item \href{https://data.gouv.fr/dataset/5ad7356788ee385c897f8153}{Budget primitif 2018 de la Communauté d'Agglomération Bar-le-Duc Sud Meuse}
\item \href{https://data.gouv.fr/dataset/5adece5b88ee387ef644f599}{Budget primitif 2018 du SIVU de la Communauté d'Agglomération Bar-le-Duc Sude Meuse}
\item \href{https://data.gouv.fr/dataset/5912dd9488ee382457509b60}{Budgets primitifs 2017 de la Communauté d'Agglomération Bar-le-Duc Sud Meuse}
\item \href{https://data.gouv.fr/dataset/5912df8788ee382bc71ffd5c}{Budgets primitifs 2017 du CIAS de la Communauté d'Agglomération Bar-le-Duc Sud Meuse}
\item \href{https://data.gouv.fr/dataset/5ac34935c751df221172babd}{Budgets primitifs 2018 du CIAS de la Communauté d'Agglomération Bar-le-Duc Sud Meuse}
\item \href{https://data.gouv.fr/dataset/57a4a27588ee384951d73ff8}{Compte administratif 2015 de la Communauté d'Agglomération Bar-le-Duc Sud Meuse}
\item \href{https://data.gouv.fr/dataset/5adde3c588ee3805208d8d48}{Compte administratif 2017 de la Communauté d'Agglomération Bar-le-Duc Sud Meuse}
\item \href{https://data.gouv.fr/dataset/5addcdb588ee3861086f892e}{Compte administratif 2017 du CIAS de la Communauté d 'Agglomération Bar-le-Duc Sud Meuse}
\item \href{https://data.gouv.fr/dataset/5ae316f788ee3871681a523f}{Compte administratif 2017 du SIVU de la Communauté d'Agglomération Bar-le-Duc Sud Meuse}
\item \href{https://data.gouv.fr/dataset/5912d40c88ee380cf6470a7c}{Comptes administratifs 2016 de la Communauté d'Agglomération Bar-le-Duc Sud Meuse}
\item \href{https://data.gouv.fr/dataset/5aec0b77c751df5adf1a18fa}{Débat d'orientation budgétaire  2018 de la Communauté d'Agglomération Bar-le-Duc Sud Meuse}
\item \href{https://data.gouv.fr/dataset/5aec09dfc751df576c13dda8}{Débat d'orientation budgétaire  2018 du CIAS de la Communauté d'Agglomération Bar-le-Duc Sud Meuse}
\item \href{https://data.gouv.fr/dataset/5b9155218b4c41686f6c0202}{Données Essentielles de la Commande Publique de la CA et du CIAS Bar-le-Duc Sud Meuse}
\item \href{https://data.gouv.fr/dataset/56b219d8c751df2cdcc30484}{Horaires des services de la Communauté d'Agglomération Meuse Grand Sud}
\item \href{https://data.gouv.fr/dataset/56a65799c751df62a0ade714}{Liste des Elus Communautaires Meuse Grand Sud}
\item \href{https://data.gouv.fr/dataset/56fd380cc751df70a6c485cb}{Liste des marchés de fournitures 2015}
\item \href{https://data.gouv.fr/dataset/56fd365fc751df6bdec485cb}{Liste des marchés de fournitures CIAS 2015}
\item \href{https://data.gouv.fr/dataset/58de53efc751df5d99e3c268}{Liste des marchés de la Communauté d'Agglomération Bar-le-Duc Sud Meuse 2016}
\item \href{https://data.gouv.fr/dataset/56fd37b1c751df6ec7c485cb}{Liste des marchés de services 2015}
\item \href{https://data.gouv.fr/dataset/56fd35d7c751df6b8ec485cb}{Liste des marchés de services CIAS 2015}
\item \href{https://data.gouv.fr/dataset/56fd34a1c751df6963c485cb}{Liste des marchés de travaux CIAS 2015}
\item \href{https://data.gouv.fr/dataset/58de5448c751df5d99e3c269}{Liste des marchés du CIAS de la Communauté d'Agglomération Bar-le-Duc Sud Meuse 2016}
\item \href{https://data.gouv.fr/dataset/57a9ec20c751df622797bae5}{Statistiques Action Sociale}
\item \href{https://data.gouv.fr/dataset/58c035cb88ee385a385b297d}{Statistiques de consultation du site web}
\item \href{https://data.gouv.fr/dataset/58b9905c88ee384ff4b17a4d}{Statistiques de la page Facebook Meuse Grand Sud}
\end{itemize}

\clearpage
\section{Communauté urbaine de Bordeaux}


\begin{center}
  \includegraphics[width=3cm]{images/orga/08_a755ef4b5441248296fa5a7df5e3c8-100.jpg}
\end{center}


La Communauté urbaine de Bordeaux regroupe 28 communes autour de 3
objectifs : réaliser les grands équipements d’agglomération, moderniser
les services urbains et développer l’économie locale. Bref, relever les
défis d’'une grande métropole entrant dans le troisième millénaire. Les
missions de la Communauté urbaine de Bordeaux correspondent d’abord aux
12 compétences attribuées aux communautés urbaines par la loi du 31
décembre 1966 :

\begin{itemize}

\item
  Le développement économique
\item
  L’'urbanisme
\item
  L'habitat
\item
  L'environnement (tri, collecte et traitement des déchets)
\item
  L'eau et l'assainissement
\item
  Les transports urbains
\item
  La voirie – la signalisation
\item
  Le stationnement
\item
  Les abattoirs – le Marché d'’Intérêt National
\item
  Les parcs cimetières
\end{itemize}

Évolution des compétences de la Cub

Le bureau communautaire du 14 octobre 2010 a souhaité que soit engagée
une réflexion sur l'’évolution des compétences de la Cub, d'’une part
pour répondre à des contraintes réglementaires et, d’'autre part, pour
prendre en compte de nouvelles sollicitations et attentes des élus, des
citoyens et des partenaires.

Le 8 juillet 2011, le Conseil de Cub a approuvé la nécessité de faire
évoluer les compétences de la Communauté urbaine. Le 25 novembre 2011,
le Conseil de Cub a adopté une délibération permettant d’étendre les
compétences de la Cub, ce qui a été acté par l’'arrêté préfectoral en
date du 30 mars 2012 et concernant :

\begin{itemize}

\item
  l’'aménagement numérique du territoire
\item
  les aires de grand passage
\item
  l’'archéologie préventive
\item
  les réseaux de chaleur et de froid
\item
  le soutien et la promotion d’'une programmation culturelle des
  territoires de la métropole
\end{itemize}

Le Conseil de Cub, en date du 8 juillet 2011 a aussi ouvert la voie vers
l'’élargissement à d’`autres compétences, notamment dans le domaine de
la nature, de la propreté, du sport, du tourisme, de l''enseignement
supérieur et de la recherche, des parcs de stationnement.


\vspace{0.5cm}

\needspace{12\baselineskip}
\subsection*{Comptage du trafic CUB
}\index{circulation}\index{routes}\index{transport}
  \begin{wrapfigure}{r}{2.5cm}
    \centering
    \qrcode[nolink]{https://data.gouv.fr/dataset/53699133a3a729239d203bd0}
  \end{wrapfigure}

Licence : \textbf{Open Data Commons Open Database License (ODbL)
}\newline
Créé le : 2013-09-18\newline
Modifié le : 2016-02-24\newline
Mise à jour : annuelle\newline
Popularité : 2 réutilisations,  0 suivi\newline
Mots-clé : \emph{circulation, routes, transport
}\newline
Permalien : \url{https://data.gouv.fr/dataset/53699133a3a729239d203bd0}\newline

\par
\noindent
    en 2010 - débit routier aux heures de pointe 2009-2010 : comptage,
moyenne et évolution en 2011 - débit routier aux heures de pointe
20010-2011 : comptage, moyenne et évolution

La Communauté urbaine de Bordeaux a installé un réseau de capteurs pour
évaluer l'état de la circulation sur son territoire. Ce trafic peut
êtreconsultéen temps réel sur\url{http://www.lacub.fr/trafic.} Ce jeu de
données consolide sur 2009 et 2010 les comptages du débit routier aux
heures de pointes (matin et soir), la moyenne de ce débit sur les jours
ouvrés,et l'évolution de ces débits entreces 2années pour chaque
compteur.

Pour les compteurs de type SIREDO, ce tableau fourni en plus le nombre
de poids lourds et la vitesse moyenne de circulation ainsi que
leursévolutions entre 2009 et 2010.


\vspace{0.5cm}
\needspace{3\baselineskip} \rule{4cm}{0.25pt}\newline\textbf{Aussi disponible du même producteur :}\begin{itemize}
\item \href{https://data.gouv.fr/dataset/53698f08a3a729239d203626}{Arceau vélo de la CUB}
\item \href{https://data.gouv.fr/dataset/53698f0ca3a729239d203631}{Arrêt physique sur le réseau}
\item \href{https://data.gouv.fr/dataset/53698fa9a3a729239d2037dd}{Borne monétique de la CUB}
\item \href{https://data.gouv.fr/dataset/53699073a3a729239d2039e4}{Chemin d'une ligne}
\item \href{https://data.gouv.fr/dataset/536991daa3a729239d203d87}{Couloir de bus}
\item \href{https://data.gouv.fr/dataset/536992f2a3a729239d204065}{Disponibilité parking de la CUB en temps réel}
\item \href{https://data.gouv.fr/dataset/536994b3a3a729239d2044f1}{Équipement public de la CUB}
\item \href{https://data.gouv.fr/dataset/536998baa3a729239d20502c}{Ligne commerciale de transport de la CUB}
\item \href{https://data.gouv.fr/dataset/53699b22a3a729239d20566e}{Offres de services bus de la CUB}
\item \href{https://data.gouv.fr/dataset/53699b23a3a729239d205670}{Offres de services tramway}
\item \href{https://data.gouv.fr/dataset/53699b7aa3a729239d205749}{Parc relais de la CUB}
\item \href{https://data.gouv.fr/dataset/53699bd5a3a729239d205824}{Passage ou franchissement en temps réel}
\item \href{https://data.gouv.fr/dataset/53699c93a3a729239d2059e9}{Places de stationnement GIG GIC}
\item \href{https://data.gouv.fr/dataset/53699cfda3a729239d205b03}{Point vente titres transport (position)}
\item \href{https://data.gouv.fr/dataset/5369a046a3a729239d20631c}{Station VCUB}
\end{itemize}

\clearpage
\section{Commune de Brocas}


\begin{center}
  \includegraphics[width=3cm]{images/orga/2014-12-20_ac498003fceb43c5bd9f28730fffdd86_opendatabrocas-100.png}
\end{center}


Soucieuse d'efficacité de l'action publique, la commune de Brocas dans
les Landes (800 habitants) a choisi l'open data comme support de
transparence et d'efficience dans la vie publique et politique. Les
données en possession de cette commune (documents, photos, vidéos)
constituent un patrimoine immatériel qui peut être mis en valeur pour
l'ensemble de la collectivité.

Les habitants ont compris que l'open data pouvait leur rendre des
services, mettre leur territoire en avant et les associer au devenir de
leur village.

Cette démarche d'ouverture des données publiques s'intègre dans une
politique d'innovation ouverte et de développement des nouvelles
technologies et des logiciels libres dans la commune.


\vspace{0.5cm}

\needspace{12\baselineskip}
\subsection*{Commune de Brocas : Budget
}\index{aquitaine}\index{brocas}\index{budget!primitif}\index{depenses}\index{fonctionnement}\index{investissement}\index{landes}\index{politiques}\index{recettes}
  \begin{wrapfigure}{r}{2.5cm}
    \centering
    \qrcode[nolink]{https://data.gouv.fr/dataset/5369910ea3a729239d203b6c}
  \end{wrapfigure}

Licence : \textbf{Open Data Commons Open Database License (ODbL)
}\newline
Créé le : 2013-12-22\newline
Modifié le : 2015-10-26\newline
De 2008-01-01 à 2013-12-31\newline
Granularité : à la commune\newline
Mise à jour : annuelle\newline
Popularité : 1 réutilisation,  1 suivi\newline
Mots-clé : \emph{aquitaine, brocas, budget-primitif, depenses, fonctionnement, investissement, landes, politiques, recettes
}\newline
Permalien : \url{https://data.gouv.fr/dataset/5369910ea3a729239d203b6c}\newline

\par
\noindent
    Eléments concernant le budget de la commune de Brocas


\vspace{0.5cm}
\needspace{12\baselineskip}
\subsection*{Commune de Brocas : Climatologie
}\index{brocas}\index{climat}\index{landes}\index{meteo}\index{nouvelle!aquitaine}
  \begin{wrapfigure}{r}{2.5cm}
    \centering
    \qrcode[nolink]{https://data.gouv.fr/dataset/584cc8e5c751df4a31c0bb7e}
  \end{wrapfigure}

Licence : \textbf{Open Data Commons Open Database License (ODbL)
}\newline
Créé le : 2016-12-11\newline
Modifié le : 2017-01-07\newline
De 2016-01-01 à 2016-12-31\newline
Mise à jour : annuelle\newline
Popularité : 1 réutilisation,  0 suivi\newline
Mots-clé : \emph{brocas, climat, landes, meteo, nouvelle-aquitaine
}\newline
Permalien : \url{https://data.gouv.fr/dataset/584cc8e5c751df4a31c0bb7e}\newline

\par
\noindent
    Éléments concernant la climatologie pour la commune de Brocas


\vspace{0.5cm}
\needspace{12\baselineskip}
\subsection*{Commune de Brocas : Comptes de la collectivité
}
  \begin{wrapfigure}{r}{2.5cm}
    \centering
    \qrcode[nolink]{https://data.gouv.fr/dataset/547897aec751df7ff04484b3}
  \end{wrapfigure}

Licence : \textbf{Licence Ouverte
}\newline
Créé le : 2014-11-28\newline
Modifié le : 2015-11-04\newline
De 2000-01-01 à 2013-01-01\newline
Granularité : à la commune\newline
Mise à jour : annuelle\newline
Popularité : 1 réutilisation,  1 suivi\newline
Mots-clé : \emph{aucun
}\newline
Permalien : \url{https://data.gouv.fr/dataset/547897aec751df7ff04484b3}\newline

\par
\noindent
    Chiffres Clés, Fonctionnement, Investissement, Fiscalité,
Autofinancement, Endettement (source Ministère des Finances)


\vspace{0.5cm}
\needspace{12\baselineskip}
\subsection*{Commune de Brocas : Démarche open data
}\index{aquitaine}\index{brocas}\index{codification}\index{fichiers}\index{landes}\index{opendata}
  \begin{wrapfigure}{r}{2.5cm}
    \centering
    \qrcode[nolink]{https://data.gouv.fr/dataset/5369911ba3a729239d203b8c}
  \end{wrapfigure}

Licence : \textbf{Open Data Commons Open Database License (ODbL)
}\newline
Créé le : 2013-12-23\newline
Modifié le : 2016-12-29\newline
De 2013-04-07 à 2020-12-31\newline
Granularité : à la commune\newline
Mise à jour : ponctuelle\newline
Popularité : 1 réutilisation,  5 suivis\newline
Mots-clé : \emph{aquitaine, brocas, codification, fichiers, landes, opendata
}\newline
Permalien : \url{https://data.gouv.fr/dataset/5369911ba3a729239d203b8c}\newline

\par
\noindent
    Soucieuse d'efficacité de l'action publique, la commune de Brocas (800
habitants) a choisi l'open data comme support de transparence et
d'efficience dans la vie publique et politique.

L'open data permet aujourd'hui de faciliter les échanges entre la
population et les élus au travers d'un groupe de travail commun.

Quant aux habitants, ils ont compris que l'open data pouvait leur rendre
des services, mettre leur territoire en avant les associer au devenir de
leur village.

Retrouver le reportage de ``Paroles d'Élus'' décrivant la démarche de la
commune :

\href{http://opendata.brocas.fr/content/reportage-lopen-data-au-coeur-des-collectivit\%C3\%A9s}{Le
reportage ``Paroles d'Elus''}


\vspace{0.5cm}
\needspace{12\baselineskip}
\subsection*{Commune de Brocas : Ecole
}\index{aquitaine}\index{brocas}\index{ecole}\index{effectifs}\index{landes}\index{rpi}
  \begin{wrapfigure}{r}{2.5cm}
    \centering
    \qrcode[nolink]{https://data.gouv.fr/dataset/53699113a3a729239d203b78}
  \end{wrapfigure}

Licence : \textbf{Open Data Commons Open Database License (ODbL)
}\newline
Créé le : 2013-12-22\newline
Modifié le : 2015-09-07\newline
De 2003-01-01 à 2013-12-31\newline
Granularité : au canton\newline
Mise à jour : annuelle\newline
Popularité : 1 réutilisation,  1 suivi\newline
Mots-clé : \emph{aquitaine, brocas, ecole, effectifs, landes, rpi
}\newline
Permalien : \url{https://data.gouv.fr/dataset/53699113a3a729239d203b78}\newline

\par
\noindent
    Eléments concernant l'école pour la commune de Brocas (Compétence :
Communauté de Communes du Pays d'Albret)


\vspace{0.5cm}
\needspace{12\baselineskip}
\subsection*{Commune de Brocas : Emploi
}
  \begin{wrapfigure}{r}{2.5cm}
    \centering
    \qrcode[nolink]{https://data.gouv.fr/dataset/547899f2c751df04ae4484b3}
  \end{wrapfigure}

Licence : \textbf{Licence Ouverte
}\newline
Créé le : 2014-11-28\newline
Modifié le : 2016-12-29\newline
De 1999-01-01 à 2011-01-01\newline
Granularité : à la commune\newline
Mise à jour : annuelle\newline
Popularité : 5 réutilisations,  1 suivi\newline
Mots-clé : \emph{aucun
}\newline
Permalien : \url{https://data.gouv.fr/dataset/547899f2c751df04ae4484b3}\newline

\par
\noindent
    Chiffres clés Emploi - Population active (source INSEE)


\vspace{0.5cm}
\needspace{12\baselineskip}
\subsection*{Commune de Brocas : Etat civil
}\index{aquitaine}\index{brocas}\index{deces}\index{etat!civil}\index{landes}\index{mariages}\index{naissances}
  \begin{wrapfigure}{r}{2.5cm}
    \centering
    \qrcode[nolink]{https://data.gouv.fr/dataset/53699116a3a729239d203b7f}
  \end{wrapfigure}

Licence : \textbf{Open Data Commons Open Database License (ODbL)
}\newline
Créé le : 2013-12-23\newline
Modifié le : 2016-12-06\newline
De 2008-01-01 à 2015-12-31\newline
Granularité : à la commune\newline
Mise à jour : annuelle\newline
Popularité : 1 réutilisation,  1 suivi\newline
Mots-clé : \emph{aquitaine, brocas, deces, etat-civil, landes, mariages, naissances
}\newline
Permalien : \url{https://data.gouv.fr/dataset/53699116a3a729239d203b7f}\newline

\par
\noindent
    Eléments concernant l'Etat Civil pour la commune de Brocas


\vspace{0.5cm}
\needspace{12\baselineskip}
\subsection*{Commune de Brocas : Gestion forestière
}\index{aquitaine}\index{brocas}\index{forestiere}\index{gestion}\index{landes}
  \begin{wrapfigure}{r}{2.5cm}
    \centering
    \qrcode[nolink]{https://data.gouv.fr/dataset/53699117a3a729239d203b82}
  \end{wrapfigure}

Licence : \textbf{Open Data Commons Open Database License (ODbL)
}\newline
Créé le : 2013-12-23\newline
Modifié le : 2015-06-04\newline
De 2009-01-01 à 2014-12-31\newline
Granularité : à la commune\newline
Mise à jour : ponctuelle\newline
Popularité : 3 réutilisations,  1 suivi\newline
Mots-clé : \emph{aquitaine, brocas, forestiere, gestion, landes
}\newline
Permalien : \url{https://data.gouv.fr/dataset/53699117a3a729239d203b82}\newline

\par
\noindent
    Éléments concernant la gestion forestière de la commune de Brocas


\vspace{0.5cm}
\needspace{12\baselineskip}
\subsection*{Commune de Brocas : Histoire
}\index{aquitaine}\index{brocas}\index{histoire}\index{histoire!locale}\index{landes}
  \begin{wrapfigure}{r}{2.5cm}
    \centering
    \qrcode[nolink]{https://data.gouv.fr/dataset/536c3f30a3a72933d8d1b396}
  \end{wrapfigure}

Licence : \textbf{Open Data Commons Open Database License (ODbL)
}\newline
Créé le : 2013-12-23\newline
Modifié le : 2016-12-27\newline
De 1790-01-01 à 2014-03-30\newline
Granularité : à la commune\newline
Mise à jour : ponctuelle\newline
Popularité : 1 réutilisation,  1 suivi\newline
Mots-clé : \emph{aquitaine, brocas, histoire, histoire-locale, landes
}\newline
Permalien : \url{https://data.gouv.fr/dataset/536c3f30a3a72933d8d1b396}\newline

\par
\noindent
    Eléments concernant l'histoire de la commune de Brocas


\vspace{0.5cm}
\needspace{12\baselineskip}
\subsection*{Commune de Brocas : Impôts
}\index{aquitaine}\index{brocas}\index{impots}\index{landes}\index{taxe}
  \begin{wrapfigure}{r}{2.5cm}
    \centering
    \qrcode[nolink]{https://data.gouv.fr/dataset/5369911aa3a729239d203b89}
  \end{wrapfigure}

Licence : \textbf{Open Data Commons Open Database License (ODbL)
}\newline
Créé le : 2013-12-23\newline
Modifié le : 2016-12-15\newline
De 2000-01-01 à 2015-12-31\newline
Granularité : à la commune\newline
Mise à jour : annuelle\newline
Popularité : 1 réutilisation,  1 suivi\newline
Mots-clé : \emph{aquitaine, brocas, impots, landes, taxe
}\newline
Permalien : \url{https://data.gouv.fr/dataset/5369911aa3a729239d203b89}\newline

\par
\noindent
    Eléments concernant les impôts pour la commune de Brocas


\vspace{0.5cm}
\needspace{12\baselineskip}
\subsection*{Commune de Brocas : Logement
}
  \begin{wrapfigure}{r}{2.5cm}
    \centering
    \qrcode[nolink]{https://data.gouv.fr/dataset/5478987cc751df02f14484b3}
  \end{wrapfigure}

Licence : \textbf{Licence Ouverte
}\newline
Créé le : 2014-11-28\newline
Modifié le : 2016-12-29\newline
De 1968-01-01 à 2011-01-01\newline
Granularité : à la commune\newline
Mise à jour : annuelle\newline
Popularité : 4 réutilisations,  1 suivi\newline
Mots-clé : \emph{aucun
}\newline
Permalien : \url{https://data.gouv.fr/dataset/5478987cc751df02f14484b3}\newline

\par
\noindent
    Chiffres clés Logement (source INSEE)


\vspace{0.5cm}
\needspace{12\baselineskip}
\subsection*{Commune de Brocas : Population
}
  \begin{wrapfigure}{r}{2.5cm}
    \centering
    \qrcode[nolink]{https://data.gouv.fr/dataset/54789c1ec751df04a24484b4}
  \end{wrapfigure}

Licence : \textbf{Licence Ouverte
}\newline
Créé le : 2014-11-28\newline
Modifié le : 2016-12-29\newline
De 1968-01-01 à 2011-01-01\newline
Granularité : à la commune\newline
Mise à jour : annuelle\newline
Popularité : 4 réutilisations,  1 suivi\newline
Mots-clé : \emph{aucun
}\newline
Permalien : \url{https://data.gouv.fr/dataset/54789c1ec751df04a24484b4}\newline

\par
\noindent
    Chiffres clés Évolution et structure de la population (données INSEE)


\vspace{0.5cm}
\needspace{12\baselineskip}
\subsection*{Commune de Brocas : Revenu moyen
}\index{aquitaine}\index{brocas}\index{economie}\index{landes}\index{revenu}
  \begin{wrapfigure}{r}{2.5cm}
    \centering
    \qrcode[nolink]{https://data.gouv.fr/dataset/53699114a3a729239d203b7a}
  \end{wrapfigure}

Licence : \textbf{Open Data Commons Open Database License (ODbL)
}\newline
Créé le : 2014-01-26\newline
Modifié le : 2016-12-27\newline
De 2011-01-01 à 2011-12-31\newline
Granularité : à l'EPCI\newline
Mise à jour : annuelle\newline
Popularité : 1 réutilisation,  1 suivi\newline
Mots-clé : \emph{aquitaine, brocas, economie, landes, revenu
}\newline
Permalien : \url{https://data.gouv.fr/dataset/53699114a3a729239d203b7a}\newline

\par
\noindent
    Éléments concernant le revenu moyen


\vspace{0.5cm}
\needspace{12\baselineskip}
\subsection*{Commune de Brocas : Vie municipale
}\index{aquitaine}\index{brocas}\index{conseil!municipal}\index{conseiller!municipal}\index{landes}\index{municipalite}\index{vie!municipale}
  \begin{wrapfigure}{r}{2.5cm}
    \centering
    \qrcode[nolink]{https://data.gouv.fr/dataset/5369911da3a729239d203b90}
  \end{wrapfigure}

Licence : \textbf{Open Data Commons Open Database License (ODbL)
}\newline
Créé le : 2013-12-23\newline
Modifié le : 2015-07-20\newline
De 2008-01-01 à 2020-12-31\newline
Granularité : à la commune\newline
Mise à jour : ponctuelle\newline
Popularité : 1 réutilisation,  1 suivi\newline
Mots-clé : \emph{aquitaine, brocas, conseil-municipal, conseiller-municipal, landes, municipalite, vie-municipale
}\newline
Permalien : \url{https://data.gouv.fr/dataset/5369911da3a729239d203b90}\newline

\par
\noindent
    Éléments concernant la vie municipale (2008-2020) pour la commune de
Brocas


\vspace{0.5cm}
\needspace{3\baselineskip} \rule{4cm}{0.25pt}\newline\textbf{Aussi disponible du même producteur :}\begin{itemize}
\item \href{https://data.gouv.fr/dataset/5369910ba3a729239d203b66}{Commune de Brocas : Accidentologie}
\item \href{https://data.gouv.fr/dataset/5369910ca3a729239d203b68}{Commune de Brocas : Associations}
\item \href{https://data.gouv.fr/dataset/58715c2f88ee380ecc0bfefe}{Commune de Brocas : Base adresses}
\item \href{https://data.gouv.fr/dataset/5369910da3a729239d203b6a}{Commune de Brocas : Biodiversité}
\item \href{https://data.gouv.fr/dataset/53699111a3a729239d203b73}{Commune de Brocas : Cartographie}
\item \href{https://data.gouv.fr/dataset/53699112a3a729239d203b77}{Commune de Brocas : Eau}
\item \href{https://data.gouv.fr/dataset/58621002c751df3c332b7154}{Commune de Brocas : Economie}
\item \href{https://data.gouv.fr/dataset/53699114a3a729239d203b7b}{ Commune de Brocas : Elections}
\item \href{https://data.gouv.fr/dataset/53699115a3a729239d203b7e}{Commune de Brocas : Etablissements}
\item \href{https://data.gouv.fr/dataset/58647eb588ee387af23f4e5d}{Commune de Brocas : Formation}
\item \href{https://data.gouv.fr/dataset/53699118a3a729239d203b84}{Commune de Brocas : Gites communaux}
\item \href{https://data.gouv.fr/dataset/53699119a3a729239d203b87}{Commune de Brocas : Hydrographie}
\item \href{https://data.gouv.fr/dataset/5369911ba3a729239d203b8b}{Commune de Brocas : Memento}
\item \href{https://data.gouv.fr/dataset/54789429c751df7c7c4484b3}{Commune de Brocas : Résultats des élections dans la 1ère circonscription des Landes}
\item \href{https://data.gouv.fr/dataset/58715fb188ee380ec60bfefe}{Commune de Brocas : Résultats sportifs}
\item \href{https://data.gouv.fr/dataset/5369911ca3a729239d203b8f}{Commune de Brocas : Sport}
\item \href{https://data.gouv.fr/dataset/58521651c751df1f27c0bb7e}{Commune de Brocas : Téléphonie mobile}
\item \href{https://data.gouv.fr/dataset/5872188c88ee382b2b0bfefe}{Commune de Brocas : Très Haut Débit }
\item \href{https://data.gouv.fr/dataset/551581e0c751df39deb33e55}{Commune de Brocas : Urbanisme}
\end{itemize}

\clearpage
\section{Conseil départemental du Cantal}


\begin{center}
  \includegraphics[width=3cm]{images/orga/88_f5b43ac9d84f51840233b458564bc5-100.png}
\end{center}


Présidé par Bruno Faure, le Conseil départemental du Cantal s'est fixé
comme principal objectif d'agir sur les services aux Cantaliens, en les
rendant toujours plus accessibles, pour confirmer son engagement d'être
″chaque jour à vos côtés″. La diversité et la richesse des actions
réalisées au quotidien par la Collectivité témoignent de la volonté
permanente d'innover, d'oser et de préparer un horizon porteur d'espoir.


\vspace{0.5cm}

\needspace{12\baselineskip}
\subsection*{Collèges du Cantal
}
  \begin{wrapfigure}{r}{2.5cm}
    \centering
    \qrcode[nolink]{https://data.gouv.fr/dataset/536990cca3a729239d203ac6}
  \end{wrapfigure}

Licence : \textbf{Licence Ouverte
}\newline
Créé le : 2013-07-08\newline
Modifié le : 2015-11-29\newline
De 2012-11-05 à 2013-11-04\newline
Mise à jour : annuelle\newline
Popularité : 2 réutilisations,  0 suivi\newline
Mots-clé : \emph{aucun
}\newline
Permalien : \url{https://data.gouv.fr/dataset/536990cca3a729239d203ac6}\newline

\par
\noindent
    informations diverses ( commune,adresse, nom proviseur, statut..)


\vspace{0.5cm}
\needspace{3\baselineskip} \rule{4cm}{0.25pt}\newline\textbf{Aussi disponible du même producteur :}\begin{itemize}
\item \href{https://data.gouv.fr/dataset/53699238a3a729239d203e81}{Défibrillateur Semi-Automatique}
\item \href{https://data.gouv.fr/dataset/536998c4a3a729239d205047}{Données routières du département}
\item \href{https://data.gouv.fr/dataset/539f0000a3a72946de9ebe34}{Etat des subventions versées}
\item \href{https://data.gouv.fr/dataset/536998eba3a729239d2050c3}{Liste des Collèges du Cantal}
\end{itemize}

\clearpage
\section{Conseil général de l'Oise}


\begin{center}
  \includegraphics[width=3cm]{images/orga/e0_2264cdb0d84db3ba8609300078694a-100.jpg}
\end{center}
\needspace{12\baselineskip}
\subsection*{Règlement de réutilisation des données publiques
}\index{archives!publiques}\index{donnees!publiques}\index{reutilisation}
  \begin{wrapfigure}{r}{2.5cm}
    \centering
    \qrcode[nolink]{https://data.gouv.fr/dataset/53699edea3a729239d205fbb}
  \end{wrapfigure}

Licence : \textbf{Licence Ouverte
}\newline
Créé le : 2013-12-02\newline
Modifié le : 2016-03-11\newline
Popularité : 1 réutilisation,  0 suivi\newline
Mots-clé : \emph{archives-publiques, donnees-publiques, reutilisation
}\newline
Permalien : \url{https://data.gouv.fr/dataset/53699edea3a729239d205fbb}\newline

\par
\noindent
    Règlement, licences et tarifs de réutilisation des informations
publiques conservées par les archives départementales de l'Oise


\vspace{0.5cm}
\needspace{3\baselineskip} \rule{4cm}{0.25pt}\newline\textbf{Aussi disponible du même producteur :}\begin{itemize}
\item \href{https://data.gouv.fr/dataset/53698e83a3a729239d2034b1}{Agenda 21}
\item \href{https://data.gouv.fr/dataset/5383c0cea3a729162bf4f026}{Analyse des évolutions sociologiques, des conduites et représentations de l'Oise}
\item \href{https://data.gouv.fr/dataset/53698f1ea3a729239d203662}{Atlas scolaire}
\item \href{https://data.gouv.fr/dataset/53698fa0a3a729239d2037c4}{Bilan social 2011}
\item \href{https://data.gouv.fr/dataset/53698fc5a3a729239d203824}{Budget par chapitres et comptes par fonction}
\item \href{https://data.gouv.fr/dataset/53698fc5a3a729239d203825}{Budget par chapitres et comptes par nature}
\item \href{https://data.gouv.fr/dataset/53698fd9a3a729239d203856}{Budget primitif}
\item \href{https://data.gouv.fr/dataset/53698fd9a3a729239d203857}{Budget primitif - Rapports chapeaux par mission}
\item \href{https://data.gouv.fr/dataset/53699026a3a729239d20391e}{Carte de catégories de voies}
\item \href{https://data.gouv.fr/dataset/53699054a3a729239d203998}{CDDC synthèse}
\item \href{https://data.gouv.fr/dataset/53699193a3a729239d203cc6}{Consommation de chauffage des collèges de 2009 à 2012}
\item \href{https://data.gouv.fr/dataset/5369921da3a729239d203e33}{Débat d'orientations budgétaires}
\item \href{https://data.gouv.fr/dataset/5369922aa3a729239d203e52}{Décision modificative n\degree{} 1}
\item \href{https://data.gouv.fr/dataset/5369922aa3a729239d203e53}{Décision modificative n\degree{} 2}
\item \href{https://data.gouv.fr/dataset/5383c54ca3a729162bf4f02f}{Décision modificative n\degree{} 3}
\item \href{https://data.gouv.fr/dataset/5369932ca3a729239d204105}{Données du rapport développement durable 2012}
\item \href{https://data.gouv.fr/dataset/53699331a3a729239d204114}{Données liées à l'Agenda 21}
\item \href{https://data.gouv.fr/dataset/53699342a3a729239d204149}{DOVH}
\item \href{https://data.gouv.fr/dataset/536997a6a3a729239d204d4f}{La lettre du SMABT}
\item \href{https://data.gouv.fr/dataset/536998dba3a729239d205099}{Liste des bâtiments du département avec les principales données}
\item \href{https://data.gouv.fr/dataset/53699982a3a729239d205283}{L'Oise et ses cantons}
\item \href{https://data.gouv.fr/dataset/53699c99a3a729239d2059f9}{Plan de mobilité durable (PMD)}
\item \href{https://data.gouv.fr/dataset/53699c9aa3a729239d2059fa}{Plan départemental de l'habitat}
\item \href{https://data.gouv.fr/dataset/53699c9aa3a729239d2059fb}{Plan départemental de l'habitat - table ronde du 7 février 2013}
\item \href{https://data.gouv.fr/dataset/53699cbda3a729239d205a4e}{Plan opérationnel d'actions touristiques de la "Destination Oise"}
\item \href{https://data.gouv.fr/dataset/53699d03a3a729239d205b12}{Politique d'exploitation}
\item \href{https://data.gouv.fr/dataset/53699e72a3a729239d205e9a}{PTLI sur le territoire de Creil-Clermont}
\item \href{https://data.gouv.fr/dataset/53699e93a3a729239d205ef0}{Rapport annuel d'activités de la CATER}
\item \href{https://data.gouv.fr/dataset/53699e94a3a729239d205ef1}{Rapport annuel d'activités du SATEP}
\item \href{https://data.gouv.fr/dataset/53699e94a3a729239d205ef2}{Rapport annuel d'activités du SATESE}
\item \href{https://data.gouv.fr/dataset/53699e95a3a729239d205ef5}{Rapport d'activités de la direction des ADO}
\item \href{https://data.gouv.fr/dataset/53699e96a3a729239d205ef6}{Rapport de fouilles}
\item \href{https://data.gouv.fr/dataset/53699e97a3a729239d205efa}{Rapport développement durable}
\item \href{https://data.gouv.fr/dataset/5387fcb1a3a7291cb36754dc}{Rapports d'activités du Conseil général}
\item \href{https://data.gouv.fr/dataset/53699edda3a729239d205fb8}{Règlement de l'aide aux communes}
\item \href{https://data.gouv.fr/dataset/53699edea3a729239d205fb9}{Règlement de la salle de lecture}
\item \href{https://data.gouv.fr/dataset/53699edfa3a729239d205fbc}{Règlement des sanctions dans les transports scolaires}
\item \href{https://data.gouv.fr/dataset/53699edfa3a729239d205fbd}{Règlement de transport des élèves handicapés}
\item \href{https://data.gouv.fr/dataset/53699ee0a3a729239d205fbf}{Règlement du service OMTA}
\item \href{https://data.gouv.fr/dataset/53699ee0a3a729239d205fc0}{Règlement équipe pluridisciplinaire RSA}
\item \href{https://data.gouv.fr/dataset/53699ee1a3a729239d205fc1}{Règlement FDS}
\item \href{https://data.gouv.fr/dataset/53699ee1a3a729239d205fc2}{Règlement intérieur FDSL}
\item \href{https://data.gouv.fr/dataset/5387fdd3a3a7291cb36754e1}{Routes départementales - Grands projets 2013-2014}
\item \href{https://data.gouv.fr/dataset/53699fb4a3a729239d2061be}{Schéma départemental de l'autonomie des personnes 2012-2017}
\item \href{https://data.gouv.fr/dataset/53699fb4a3a729239d2061bf}{Schéma départemental des circulations douces (SDCD)}
\item \href{https://data.gouv.fr/dataset/53699fb5a3a729239d2061c2}{Schéma départemental enfance famille "Prévention-protection" 2007-2014}
\item \href{https://data.gouv.fr/dataset/53699fb7a3a729239d2061c6}{Schéma des ENS}
\item \href{https://data.gouv.fr/dataset/53699fb6a3a729239d2061c5}{Schéma des ENS - cartographie}
\item \href{https://data.gouv.fr/dataset/53699fb8a3a729239d2061c8}{Schéma directeur d'accessibilité du réseau de transport}
\item \href{https://data.gouv.fr/dataset/5369a195a3a729239d206639}{Taux d'emploi de travailleurs handicapés}
\item et 0 autres jeux de données\end{itemize}

\clearpage
\section{Cour des comptes}


\begin{center}
  \includegraphics[width=3cm]{images/orga/ac_83f761a23f4263be8b3401b01625e8-100.png}
\end{center}


Les juridictions financières participent à la démarche française de
gouvernement ouvert en publiant des données sur leurs activités et leurs
travaux. Cette démarche est cohérente avec les articles 14 et 15 de la
Déclaration des droits de l'homme et du citoyen, qui sont au fondement
des missions de la Cour des comptes : dans une démocratie, le citoyen a
le droit de constater, par lui-même ou par ses représentants, la
nécessité de la contribution publique, de la consentir librement, d'en
suivre l'emploi, et d'en déterminer la quotité, l'assiette, le
recouvrement et la durée. Il a le droit de demander compte à tout agent
public de son administration.

Les juridictions financières ont pris l'initiative, dès 2014, d'ouvrir
des premiers jeux de données publiques et vont approfondir cette
démarche. Elles veilleront, à l'avenir, à intéresser davantage le
citoyen à leurs travaux, sous réserve des précautions inhérentes au
fonctionnement d'une juridiction indépendante, notamment le respect des
secrets protégés par la loi.

La Cour des comptes s'engage ainsi à mettre à disposition régulièrement
les jeux de données suivants :

\begin{itemize}
\item
  les données budgétaires fondant l'analyse de l'exécution du budget de
  l'État ;
\item
  des données sur la comptabilité générale de l'État ;
\item
  à chaque fois que cela sera possible, les données fondant les enquêtes
  thématiques de la Cour ;
\item
  les données fondant les travaux concernant les finances locales ;
\item
  certaines données d'activité des juridictions financières, notamment
  la mise à jour de la liste des publications de la Cour et des moyens
  des juridictions financières.
\end{itemize}

La Cour des comptes travaille également, en lien avec les services du
Premier ministre, à la mise en place d'un portail de données, répertorié
sur le portail du Gouvernement (data.gouv.fr), afin de systématiser une
stratégie de gestion des données et la démarche d'ouverture des
informations publiques.

Les données publiques présentes sur cet espace sont celles des
juridictions financières et organismes associés, c'est-à-dire :

\begin{itemize}
\item
  la Cour des comptes ;
\item
  les chambres régionales et territoriales des comptes ;
\item
  la Cour de discipline budgétaire et financière ;
\item
  le Conseil des prélèvements obligatoires ;
\item
  le Haut Conseil pour les finances publiques.
\end{itemize}

Pour une information à jour sur la démarche des juridictions
financières, veuillez consulter le
\href{https://www.ccomptes.fr/Nos-activites/Donnees-publiques}{site des
juridictions financières}.


\vspace{0.5cm}

\needspace{12\baselineskip}
\subsection*{Arrêts de la Cour de discipline budgétaire et financière (1954-2015)
}\index{cour!de!discipline!budgetaire!et}\index{cour!des!comptes}\index{juridictions!financieres}\index{jurisprudence}
  \begin{wrapfigure}{r}{2.5cm}
    \centering
    \qrcode[nolink]{https://data.gouv.fr/dataset/5748310188ee38023ed1b934}
  \end{wrapfigure}

Licence : \textbf{Open Data Commons Open Database License (ODbL)
}\newline
Créé le : 2016-05-27\newline
Modifié le : 2016-05-27\newline
Mise à jour : annuelle\newline
Popularité : 1 réutilisation,  0 suivi\newline
Mots-clé : \emph{cour-de-discipline-budgetaire-et, cour-des-comptes, juridictions-financieres, jurisprudence
}\newline
Permalien : \url{https://data.gouv.fr/dataset/5748310188ee38023ed1b934}\newline

\par
\noindent
    Dans le cadre de sa \#DataSession des 27 et 28 mai 2016, les
juridictions financières ont décidé d'ouvrir en open data le texte
intégral de leur jurisprudence. Le format retenu est le format XML
(voire du HTML directement), ce qui permettra d'inclure des méta-données
utiles aux réutilisateurs. Dans un premier temps, les décisions de la
Cour seront disponibles au format Word mais nous travaillons à leur
conversion.

Les présents jeux de données concernent les arrêts de la Cour de
discipline budgétaire et financière publiés au Journal officiel de la
République française depuis 1954.

Merci de nous signaler toute erreur d'anonymisation que vous pourriez
détecter, afin que le jeu de données soit corrigé.

Nb : les anonymisations concernent les cas de relaxe et non ceux où la
publication de la décision faisait partie de la sanction décidée par la
Cour.


\vspace{0.5cm}
\needspace{12\baselineskip}
\subsection*{Jugements anonymisés des chambres régionales et territoriales des
comptes (2015)
}\index{chambres!regionales!des!comptes}\index{chambres!territoriales!des!compt}\index{collectivites!locales}\index{collectivites!territoriales}\index{crc}\index{crtc}\index{ctc}\index{jugements}\index{juridictions!financieres}
  \begin{wrapfigure}{r}{2.5cm}
    \centering
    \qrcode[nolink]{https://data.gouv.fr/dataset/5746f8ca88ee382b03d1b934}
  \end{wrapfigure}

Licence : \textbf{Open Data Commons Open Database License (ODbL)
}\newline
Créé le : 2016-05-26\newline
Modifié le : 2016-05-27\newline
Mise à jour : annuelle\newline
Popularité : 1 réutilisation,  0 suivi\newline
Mots-clé : \emph{chambres-regionales-des-comptes, chambres-territoriales-des-compt, collectivites-locales, collectivites-territoriales, crc, crtc, ctc, jugements, juridictions-financieres
}\newline
Permalien : \url{https://data.gouv.fr/dataset/5746f8ca88ee382b03d1b934}\newline

\par
\noindent
    Dans le cadre de sa \#DataSession des 27 et 28 mai 2016, les
juridictions financières ont décidé d'ouvrir en open data le texte
intégral de leur jurisprudence. Le format retenu est le format XML
(voire du HTML directement), ce qui permettra d'inclure des méta-données
utiles aux réutilisateurs. Dans un premier temps, les décisions des
chambres régionales et territoriales seront disponibles au format Word
mais nous travaillons à leur conversion en XML ou HTML.

Les présents jeux de données concernent les jugements des chambres
régionales et territoriales pour 2015.

Merci de nous signaler toute erreur d'anonymisation que vous pourriez
détecter, afin que le jeu de données soit corrigé.


\vspace{0.5cm}
\needspace{12\baselineskip}
\subsection*{Jurisprudence anonymisée de la Cour des comptes (2006-2008 et 2010-2015)
}\index{arrets}\index{comptabilite!publique}\index{comptable}\index{cour!des!comptes}\index{debet}\index{gestion!de!fait}\index{juridictions!financieres}\index{jurisprudence}\index{jurisprudences}
  \begin{wrapfigure}{r}{2.5cm}
    \centering
    \qrcode[nolink]{https://data.gouv.fr/dataset/5746d29e88ee385f41d1b934}
  \end{wrapfigure}

Licence : \textbf{Open Data Commons Open Database License (ODbL)
}\newline
Créé le : 2016-05-26\newline
Modifié le : 2016-05-28\newline
Mise à jour : annuelle\newline
Popularité : 2 réutilisations,  0 suivi\newline
Mots-clé : \emph{arrets, comptabilite-publique, comptable, cour-des-comptes, debet, gestion-de-fait, juridictions-financieres, jurisprudence, jurisprudences
}\newline
Permalien : \url{https://data.gouv.fr/dataset/5746d29e88ee385f41d1b934}\newline

\par
\noindent
    Dans le cadre de sa \#DataSession des 27 et 28 mai 2016, les
juridictions financières ont décidé d'ouvrir en open data le texte
intégral de leur jurisprudence. Le format retenu est le format XML
(voire du HTML directement), ce qui permettra d'inclure des méta-données
utiles aux réutilisateurs. Dans un premier temps, les décisions de la
Cour seront disponibles au format Word mais nous travaillons à leur
conversion.

Les présents jeux de données concernent les arrêts de la Cour des
comptes de 2006 à 2008 et de 2010 à 2015.

Merci de nous signaler toute erreur d'anonymisation que vous pourriez
détecter, afin que le jeu de données soit corrigé.


\vspace{0.5cm}
\needspace{12\baselineskip}
\subsection*{La Poste : une transformation à accélérer
}
  \begin{wrapfigure}{r}{2.5cm}
    \centering
    \qrcode[nolink]{https://data.gouv.fr/dataset/58cbdc6cc751df4aa3a12f3a}
  \end{wrapfigure}

Licence : \textbf{Open Data Commons Open Database License (ODbL)
}\newline
Créé le : 2017-03-17\newline
Modifié le : 2017-09-04\newline
Popularité : 1 réutilisation,  0 suivi\newline
Mots-clé : \emph{aucun
}\newline
Permalien : \url{https://data.gouv.fr/dataset/58cbdc6cc751df4aa3a12f3a}\newline

\par
\noindent
    Le groupe La Poste, deuxième employeur public de France avec plus de 250
000 salariés, exerce ses activités dans des domaines divers (banque,
courrier, colis, téléphonie mobile\ldots{}), y compris au moyen de
filiales en France et à l'étranger. La Poste incarne un service public
présent sur tout le territoire grâce à deux réseaux majeurs : celui des
points de contact (plus de 17 000, dont 9 149 bureaux de poste) et la
distribution à domicile (plus de 72 000 facteurs distribuant le courrier
dans 39 millions de boîtes aux lettres, six jours par semaine). La Poste
est aujourd'hui confrontée à la chute continue des volumes du courrier,
qui ampute son chiffre d'affaires de 500 M\euro{} chaque année. À ce
défi s'ajoute pour l'entreprise la nécessité de réussir sa mutation
numérique, de répondre à la croissance du e-commerce et, pour La Banque
Postale, de faire face à des taux d'intérêt bas. Si La Poste a su
évoluer et se montrer résistante, des fragilités et des risques exogènes
persistent. Elle doit donc, pour perdurer, accélérer le rythme de sa
transformation.

Ce rapport est accessible sur
\href{https://www.ccomptes.fr/fr/publications/la-poste-une-transformation-accelerer}{le
site de la Cour}.

Les fichiers publiés correspondent aux données ayant servi de base à
l'élaboration du rapport.


\vspace{0.5cm}
\needspace{12\baselineskip}
\subsection*{Le coût du lycée
}\index{education!nationale}\index{enseignement}\index{lycees}\index{performance}
  \begin{wrapfigure}{r}{2.5cm}
    \centering
    \qrcode[nolink]{https://data.gouv.fr/dataset/562501bec751df09b9bd3534}
  \end{wrapfigure}

Licence : \textbf{Open Data Commons Open Database License (ODbL)
}\newline
Créé le : 2015-10-19\newline
Modifié le : 2017-09-11\newline
Popularité : 1 réutilisation,  0 suivi\newline
Mots-clé : \emph{education-nationale, enseignement, lycees, performance
}\newline
Permalien : \url{https://data.gouv.fr/dataset/562501bec751df09b9bd3534}\newline

\par
\noindent
    La Cour des comptes a rendu public, le 29 septembre 2015, un rapport sur
le coût du lycée, dans le prolongement de ses rapports de 2010
(L'éducation nationale face à la réussite de tous les élèves), 2013
(Gérer les enseignants autrement) et 2015 (Le suivi individualisé des
élèves). Créé en 1802 pour forger l'élite de la Nation, le lycée conduit
désormais 80 \% d'une classe d'âge au niveau du baccalauréat. Ce défi
quantitatif a été relevé. Pourtant, alors que le coût moyen d'un lycéen
français est 38 \% plus élevé que celui des lycéens des autres pays de
l'OCDE, au plan qualitatif les résultats en France en termes de réussite
dans les études post-bac ou d'insertion sur le marché du travail des
bacheliers professionnels ressortent comme très moyens. La Cour, après
avoir analysé les composantes de ce coût, détaillé par voie, par série,
par discipline, identifie plusieurs leviers susceptibles d'en assurer la
maîtrise, pour financer des réformes nécessaires à l'amélioration de la
performance d'ensemble du système éducatif.

Ce rapport est accessible sur le
\href{https://www.ccomptes.fr/fr/publications/le-cout-du-lycee}{site de
la Cour}.

Les fichiers publiés correspondent aux données ayant servi de base à
l'élaboration du rapport, qui seul engage les juridictions financières.
La mise à disposition de ces données complète l'offre d'information à
destination des citoyens.


\vspace{0.5cm}
\needspace{12\baselineskip}
\subsection*{Rapport d'observations définitives des chambres régionales et
territoriales des comptes (2015)
}\index{collectivites!locales}\index{collectivites!territoriales}\index{juridiction}\index{juridictions!financieres}\index{observations!definitives}\index{rod}\index{texte!integral}
  \begin{wrapfigure}{r}{2.5cm}
    \centering
    \qrcode[nolink]{https://data.gouv.fr/dataset/5746ce2a88ee385696d1b934}
  \end{wrapfigure}

Licence : \textbf{Open Data Commons Open Database License (ODbL)
}\newline
Créé le : 2016-05-26\newline
Modifié le : 2017-06-22\newline
Granularité : à la commune\newline
Mise à jour : annuelle\newline
Popularité : 1 réutilisation,  0 suivi\newline
Mots-clé : \emph{collectivites-locales, collectivites-territoriales, juridiction, juridictions-financieres, observations-definitives, rod, texte-integral
}\newline
Permalien : \url{https://data.gouv.fr/dataset/5746ce2a88ee385696d1b934}\newline

\par
\noindent
    Dans le cadre de sa \#DataSession des 27 et 28 mai 2016, les
juridictions financières ont décidé d'ouvrira en open data le texte
intégral de leurs rapports publics. Le format retenu est le format XML
(voire du HTML directement), ce qui permettra d'inclure des méta-données
utiles aux réutilisateurs.

Les présents jeux de données concernent les rapports d'observations
définitives des chambres régionales et territoriales de 2015.


\vspace{0.5cm}
\needspace{12\baselineskip}
\subsection*{Rapports publiés par la Cour des comptes
}\index{cour!des!comptes}\index{juridictions!financieres}\index{rapport!public}\index{rapports!administratifs}\index{rapports!publics}\index{rapports!publics!thematiques}\index{referes}
  \begin{wrapfigure}{r}{2.5cm}
    \centering
    \qrcode[nolink]{https://data.gouv.fr/dataset/57470e8688ee38574dd1b934}
  \end{wrapfigure}

Licence : \textbf{Open Data Commons Open Database License (ODbL)
}\newline
Créé le : 2016-05-26\newline
Modifié le : 2017-06-22\newline
Granularité : au pays\newline
Mise à jour : annuelle\newline
Popularité : 1 réutilisation,  2 suivis\newline
Mots-clé : \emph{cour-des-comptes, juridictions-financieres, rapport-public, rapports-administratifs, rapports-publics, rapports-publics-thematiques, referes
}\newline
Permalien : \url{https://data.gouv.fr/dataset/57470e8688ee38574dd1b934}\newline

\par
\noindent
    Dans le cadre de sa \#DataSession des 27 et 28 mai 2016, les
juridictions financières ont décidé d'ouvrir en open data le texte
intégral de leurs rapports publics. Le format retenu est le format XML
(voire du HTML directement), ce qui permettra d'inclure des méta-données
utiles aux réutilisateurs.

Le présent jeu de données concerne les rapports publiés par la Cour des
comptes.

Il a été mis à jour avec le texte intégral des rapports publics de 2016
pour la \#DataSession sur le thème de la ``transparence de l'action
publique'' des 23 et 24 juin 2017.


\vspace{0.5cm}
\needspace{3\baselineskip} \rule{4cm}{0.25pt}\newline\textbf{Aussi disponible du même producteur :}\begin{itemize}
\item \href{https://data.gouv.fr/dataset/59e875d788ee381d2da961ed}{APB et accès à l’enseignement supérieur : un dispositif contesté à réformer}
\item \href{https://data.gouv.fr/dataset/5c50595a8b4c4107224c06ab}{Approche méthodologique des coûts de la justice}
\item \href{https://data.gouv.fr/dataset/5afe9138c751df5e01834696}{Arrêts de la Cour de discipline budgétaire et financière (2016 , 2017, semestre 1 2018)}
\item \href{https://data.gouv.fr/dataset/575fa9bb88ee3819e2640391}{Association française contre les myopathies - Téléthon}
\item \href{https://data.gouv.fr/dataset/58cfe013c751df1ce87cd034}{Bpifrance : une mise en place réussie, un développement à stabiliser, des perspectives financières à consolider}
\item \href{https://data.gouv.fr/dataset/555df4eac751df5c8dc98e0f}{Budget de l'Etat - Exercice 2012}
\item \href{https://data.gouv.fr/dataset/555df757c751df5c8dc98e10}{Budget de l'Etat - Exercice 2013}
\item \href{https://data.gouv.fr/dataset/55673898c751df5f9ee5726a}{Budget de l'Etat - Exercice 2014}
\item \href{https://data.gouv.fr/dataset/574706f7c751df425b8cc4b3}{Budget de l’État - Exercice 2015 (résultats et gestion)}
\item \href{https://data.gouv.fr/dataset/5b053d50c751df6a1ab662d1}{Certification des comptes 2017 de l’État}
\item \href{https://data.gouv.fr/dataset/5b0bc41cc751df43dc1e357a}{Certification des comptes 2017 du régime général de sécurité sociale}
\item \href{https://data.gouv.fr/dataset/5747f43d88ee380e9cd1b934}{Certification des comptes de l’État pour l’exercice 2015}
\item \href{https://data.gouv.fr/dataset/595254a688ee384d7843fbbf}{Certification des comptes de l’État pour l’exercice 2016}
\item \href{https://data.gouv.fr/dataset/584995dc88ee3803d2c65bb3}{Certification des comptes du régime général de sécurité sociale 2015}
\item \href{https://data.gouv.fr/dataset/593aae24c751df53fe6bff93}{Certification des comptes du régime général de sécurité sociale - exercice 2016}
\item \href{https://data.gouv.fr/dataset/555df4c4c751df592ec98e0f}{Compte général de l’État (2006-2014)}
\item \href{https://data.gouv.fr/dataset/58cbe146c751df532e39026f}{Concours financiers de l’État et disparités de dépenses des communes et de leurs groupements }
\item \href{https://data.gouv.fr/dataset/5369936ba3a729239d2041b1}{Effectifs de la Cour des comptes, par type de grade ou d'affectation (2011-2017)}
\item \href{https://data.gouv.fr/dataset/5369936fa3a729239d2041b9}{Effectifs des chambres régionales et territoriales des comptes, par chambre et type d'affectation (2012-2017)}
\item \href{https://data.gouv.fr/dataset/58ca963c88ee383ed88e01a9}{Fondation assistance aux animaux}
\item \href{https://data.gouv.fr/dataset/58ca9d6388ee384bc8953db1}{France Télévisions : mieux gérer l’entreprise, accélérer les réformes}
\item \href{https://data.gouv.fr/dataset/59d5f60dc751df1749f8db87}{Gérer les enseignants autrement : une réforme qui reste à faire}
\item \href{https://data.gouv.fr/dataset/5afe926cc751df5fd7484e70}{Jugements anonymisés des chambres régionales et territoriales des comptes (2016, 2017 et semestre 1 2018)}
\item \href{https://data.gouv.fr/dataset/5afe827cc751df460ad848ca}{Jurisprudence anonymisée de la Cour des comptes (2009, 2016, 2017 et semestre 1 2018)}
\item \href{https://data.gouv.fr/dataset/5c04f2348b4c410c0d494ac3}{La Banque de France}
\item \href{https://data.gouv.fr/dataset/5c7fcca38b4c4146e0003040}{La Caisse de garantie du logement locatif social}
\item \href{https://data.gouv.fr/dataset/578e3d1888ee382aca7d97c2}{La carte des syndicats intercommunaux : une rationalisation à poursuivre}
\item \href{https://data.gouv.fr/dataset/58ca99c088ee38458062cb30}{L’accès des jeunes à l’emploi : construire des parcours, adapter les aides}
\item \href{https://data.gouv.fr/dataset/56cd89f888ee38256cfa79cf}{La comptabilité générale de l’État, dix ans après : une nouvelle étape à engager}
\item \href{https://data.gouv.fr/dataset/5ad5c81488ee3817019e1d23}{La coopération européenne en matière d’armement}
\item \href{https://data.gouv.fr/dataset/569654c988ee3856c79b47e0}{La départementalisation de Mayotte}
\item \href{https://data.gouv.fr/dataset/5c7032c28b4c414a35a13a5c}{La dette des entités publiques : périmètre et risques}
\item \href{https://data.gouv.fr/dataset/5b2a7e8cc751df7715220f09}{La DGFiP, dix ans après la fusion}
\item \href{https://data.gouv.fr/dataset/5bfbbd56634f413f20b017ac}{La fondation Action enfance}
\item \href{https://data.gouv.fr/dataset/58ca980c88ee384187763afc}{La Fondation Notre-Dame}
\item \href{https://data.gouv.fr/dataset/5b3cfbc288ee38634b0e53c1}{La formation des demandeurs d’emploi}
\item \href{https://data.gouv.fr/dataset/55f05771c751df19f81f92b1}{La masse salariale de l’État : enjeux et leviers}
\item \href{https://data.gouv.fr/dataset/58cbf76288ee381723fd5ba2}{La police technique et scientifique}
\item \href{https://data.gouv.fr/dataset/5a708d9688ee3860bece6c21}{La politique en direction des personnes présentant des troubles du spectre de l’autisme}
\item \href{https://data.gouv.fr/dataset/5a315927c751df1c29a2f49a}{La politique immobilière du ministère de la justice}
\item \href{https://data.gouv.fr/dataset/56f50d8388ee38603ac352f7}{La prévention des conflits d’intérêts en matière d’expertise sanitaire}
\item \href{https://data.gouv.fr/dataset/5c52d0f18b4c412c61ceec6f}{La prise en charge financière des victimes du terrorisme}
\item \href{https://data.gouv.fr/dataset/57446483c751df500e8cc4b3}{La protection judiciaire de la jeunesse}
\item \href{https://data.gouv.fr/dataset/5b0414fcc751df118f9be5f9}{La qualité des comptes des administrations publiques}
\item \href{https://data.gouv.fr/dataset/58cbdd93c751df4cbce8b363}{La régulation des jeux d’argent et de hasard }
\item \href{https://data.gouv.fr/dataset/55ffccfdc751df07b649df41}{La sécurité sociale 2015}
\item \href{https://data.gouv.fr/dataset/595650b288ee3811143059dd}{La situation et les perspectives des finances publiques}
\item \href{https://data.gouv.fr/dataset/5b338249c751df6e4f986e12}{La situation et les perspectives des finances publiques}
\item \href{https://data.gouv.fr/dataset/58306906c751df14a8c0bb7e}{La situation et les perspectives des finances publiques}
\item \href{https://data.gouv.fr/dataset/5a1e860ec751df45b1db6c1f}{L'avenir de l'assurance maladie}
\item et 67 autres jeux de données\end{itemize}

\clearpage
\section{DatARA - Données publiques ouvertes en Auvergne-Rhône-Alpes}


\begin{center}
  \includegraphics[width=3cm]{images/orga/97_c00f6f4f3749a796e2f4e83fea7139-100.png}
\end{center}


DatARA : une plateforme pilotée par la Préfecture de la région
Auvergne-Rhône-Alpes.

Cette infrastructure est né de la \textbf{collaboration entre les
services de l'État et les collectivités territoriales} en
Auvergne-Rhône-Alpes. Ces acteurs publics ont souhaité \textbf{renforcer
leur coordination dans la collecte et la mutualisation de données
publiques ouvertes géographiques et non géographiques}, en favorisant
les échanges d'information et d'expérience, ainsi que la coopération
entre tous les partenaires publics.

Cette coordination est motivée par une double aspiration :

\begin{itemize}

\item
  améliorer \textbf{la connaissance commune du territoire de la région
  Auvergne-Rhône-Alpes}, afin de guider les choix politiques ;
\item
  promouvoir \textbf{une approche territoriale de l'action publique}
  afin d'éviter le cloisonnement des politiques publiques.
\end{itemize}

Ce dispositif partenarial est à l'usage des services de L'État et de la
Région, des établissements publics et des collectivités
d'Auvergne-Rhône-Alpes mais aussi des citoyens et des entreprises.


\vspace{0.5cm}

\needspace{12\baselineskip}
\subsection*{Inventaire du réseau de forêts potentiellement en évolution naturelle -
Rhône-Alpes
}\index{biota}\index{donnees!ouvertes}\index{draaf!auvergne!rhone!alpes}\index{foret}\index{foret!milieu!vegetal}\index{grand!public}\index{habitats!et!biotopes}\index{inventaire!nature!biodiversite}\index{passerelle!inspire}\index{rhone!alpes}
  \begin{wrapfigure}{r}{2.5cm}
    \centering
    \qrcode[nolink]{https://data.gouv.fr/dataset/557f1b45c751df61371d4c28}
  \end{wrapfigure}

Licence : \textbf{Licence Ouverte version 2.0
}\newline
Créé le : 2015-06-15\newline
Modifié le : 2019-01-17\newline
Popularité : 1 réutilisation,  0 suivi\newline
Mots-clé : \emph{biota, donnees-ouvertes, draaf-auvergne-rhone-alpes, foret, foret-milieu-vegetal, grand-public, habitats-et-biotopes, inventaire-nature-biodiversite, passerelle-inspire, rhone-alpes
}\newline
Permalien : \url{https://data.gouv.fr/dataset/557f1b45c751df61371d4c28}\newline

\par
\noindent
    Une forêt en évolution naturelle est constituée de forêts spontanées
(même jeunes). Cette forêt n'a pas fait l'objet de travaux forestiers ou
de coupes depuis plusieurs décennies et représente un type d'habitat
forestier autochtone. Ont été retenues les forêts d'au moins 1 ha. De
manière facultative, la forêt présente des arbres sénescents, du bois
mort au sol ou debout.

L'inventaire a été concentré d'une part sur des forêts bénéficiant déjà
d'un statut de protection et d'autre part dans les massifs forestiers
situées dans des zones difficiles d'exploitation (pentes fortes).

L'inventaire a été mené en 2010 par le service régional forêt, bois et
énergie de la direction régionale de l'alimentation, de l'agriculture et
de la forêt de Rhône-Alpes.

\textbf{Origine}

Collecte auprès de l'ONF des sites potentiellement en libre évolution ou
déjà en libre évolution. Il s'agit notamment des séries d'intérêt
écologique et des îlots de sénescence.

Collecte auprès des gestionnaires de zones à protection
environnementales - Natura 2000, réserves, parcs nationaux et régionaux
- des sites potentiellement en libre évolution ou déjà en libre
évolution.

Production par la DRAAF des zones boisées dans les secteurs de pente de
plus de 100\% (croisement entre les forêts de l'occupation BDCarto et la
couche des zones de pentes produite par la DRAAF).

Assemblage des données et qualification de chacun des objets.

\textbf{Organisations partenaires}

Direction Régionale de l'Alimentation de l'Agriculture et de la Forêt
d'Auvergne-Rhône-Alpes (DRAAF Auvergne-Rhône-Alpes), Direction Régionale
de l'Alimentation de l'Agriculture et de la Forêt d'Auvergne-Rhône-Alpes
(Direction Régionale de l'Alimentation de l'Agriculture et de la Forêt
d'Auvergne-Rhône-Alpes (DRAAF Auvergne-Rhône-Alpes))

\textbf{Liens annexes}

\begin{itemize}

\item
  \href{https://catalogue.datara.gouv.fr/rss/atomfeed/atomdataset/4505a00e-019c-4042-8973-dcab968dce24}{Téléchargement
  direct des données}
\end{itemize}

➞
\href{https://geo.data.gouv.fr/fr/datasets/7245ede1a19a707c0653feda4dd13ed013a657a9}{Consulter
cette fiche sur geo.data.gouv.fr}


\vspace{0.5cm}
\needspace{3\baselineskip} \rule{4cm}{0.25pt}\newline\textbf{Aussi disponible du même producteur :}\begin{itemize}
\item \href{https://data.gouv.fr/dataset/557ee076c751df5c1a1d4bfd}{Aire géographique d'une appellation d'origine protégée de type fromage - Auvergne-Rhône-Alpes}
\item \href{https://data.gouv.fr/dataset/557f1af788ee38061cf4b5a5}{Bassins de production viticole - Rhône-Alpes}
\item \href{https://data.gouv.fr/dataset/5874a76cc751df6fa041ed71}{Carroyage des dalles des points LIDAR 2014 territoire français Grand Genève}
\item \href{https://data.gouv.fr/dataset/5874a76c88ee3805680bfefe}{Carroyage des dalles MNT et MNE LIDAR 2014 territoire français Grand Genève}
\item \href{https://data.gouv.fr/dataset/5874a76c88ee38159a0bfefe}{Carroyage des Orthophotos LIDAR 2014 territoire français Grand Genève}
\item \href{https://data.gouv.fr/dataset/557f1af8c751df61371d4c09}{Chaufferies collectives publiques au bois - Rhône-Alpes}
\item \href{https://data.gouv.fr/dataset/557f1b20c751df42501d4c03}{CLPA :les ouvrages linéaires de Rhône-Alpes}
\item \href{https://data.gouv.fr/dataset/557f1b1ec751df62871d4c06}{CLPA :les ouvrages ponctuels de Rhône-Alpes}
\item \href{https://data.gouv.fr/dataset/557f1b1fc751df62871d4c08}{CLPA :les ouvrages zonaux de Rhône-Alpes}
\item \href{https://data.gouv.fr/dataset/557f1b5088ee3838bef4b5a3}{CLPA : photo-interprétation (lignes) des phénomènes d'avalanche de Rhône-Alpes}
\item \href{https://data.gouv.fr/dataset/557f1b5088ee383533f4b5da}{CLPA : photo-interprétation (zones) des phénomènes d'avalanche de Rhône-Alpes}
\item \href{https://data.gouv.fr/dataset/557f1b1e88ee383533f4b5c0}{CLPA : Témoignages d’avalanche (lignes) de Rhône-Alpes}
\item \href{https://data.gouv.fr/dataset/557f1b3bc751df62871d4c1b}{CLPA : Témoignages d'avalanche (zones) de Rhône-Alpes}
\item \href{https://data.gouv.fr/dataset/557f1b46c751df4d181d4c13}{Collectivités signataires de la charte régionale d'entretien des espaces publics - Auvergne-Rhône-Alpes}
\item \href{https://data.gouv.fr/dataset/59b65ff9c751df226163eba1}{Contrats de développement - Rhône-Alpes}
\item \href{https://data.gouv.fr/dataset/59b66730c751df226163eba2}{Données piscicoles en cours d'Eau - espèces par station (codes- nombre)- région Rhône-Alpes}
\item \href{https://data.gouv.fr/dataset/59b6672e88ee3822be30746f}{Données piscicoles en Plans d'Eau - espèces par opération - région Rhône-Alpes}
\item \href{https://data.gouv.fr/dataset/59b6673088ee3822d1db49c5}{Données piscicoles en Plans d'Eau - région Rhône-Alpes}
\item \href{https://data.gouv.fr/dataset/56f2730bc751df05dee0c748}{Eléments engagés dans une mesure agro-environnementale de conversion à l'agriculture biologique, 2007 - Rhône-Alpes}
\item \href{https://data.gouv.fr/dataset/56f2730688ee386925f07b1a}{Eléments engagés dans une mesure agro-environnementale de conversion à l'agriculture biologique, 2008 - Rhône-Alpes}
\item \href{https://data.gouv.fr/dataset/56f27307c751df21e6e0c73a}{Eléments engagés dans une mesure agro-environnementale de conversion à l'agriculture biologique, 2009 - Rhône-Alpes}
\item \href{https://data.gouv.fr/dataset/56f27307c751df21e6e0c73b}{Eléments engagés dans une mesure agro-environnementale de conversion à l'agriculture biologique, 2010 - Rhône-Alpes}
\item \href{https://data.gouv.fr/dataset/56f27308c751df05dee0c745}{Eléments engagés dans une mesure agro-environnementale de conversion à l'agriculture biologique, 2011 - Rhône-Alpes}
\item \href{https://data.gouv.fr/dataset/5a81a53d88ee386130b8f56b}{Eléments engagés dans une mesure agro-environnementale de conversion à l'agriculture biologique, 2012 - Rhône-Alpes}
\item \href{https://data.gouv.fr/dataset/557f1b48c751df61371d4c2b}{Eléments engagés dans une mesure agro-environnementale de conversion à l'agriculture biologique, 2013 - Rhône-Alpes}
\item \href{https://data.gouv.fr/dataset/56f2730c88ee38738af07ae9}{Eléments engagés dans une mesure agro-environnementale rotationnelle pour la campagne 2010 - Rhône-Alpes}
\item \href{https://data.gouv.fr/dataset/56f27308c751df05dee0c746}{Eléments engagés dans une mesure agro-environnementale rotationnelle pour la campagne 2011 - Rhône-Alpes}
\item \href{https://data.gouv.fr/dataset/56f27303c751df05dee0c744}{Eléments engagés dans une mesure agro-environnementale rotationnelle pour la campagne 2012 - Rhône-Alpes}
\item \href{https://data.gouv.fr/dataset/557f1b4688ee381cdcf4b5bf}{Eléments engagés dans une mesure agro-environnementale rotationnelle pour la campagne 2013 - Rhône-Alpes}
\item \href{https://data.gouv.fr/dataset/56f2730cc751df05dee0c749}{Eléments linéaires engagés dans une mesure agro-environnementale territorialisée pour la campagne 2008 - Rhône-Alpes}
\item \href{https://data.gouv.fr/dataset/56f2730688ee386925f07b1b}{Eléments linéaires engagés dans une mesure agro-environnementale territorialisée pour la campagne 2009 - Rhône-Alpes}
\item \href{https://data.gouv.fr/dataset/56f2730b88ee386925f07b1d}{Eléments linéaires engagés dans une mesure agro-environnementale territorialisée pour la campagne 2010 - Rhône-Alpes}
\item \href{https://data.gouv.fr/dataset/56f2730bc751df2ff5e0c747}{Eléments linéaires engagés dans une mesure agro-environnementale territorialisée pour la campagne 2011 - Rhône-Alpes}
\item \href{https://data.gouv.fr/dataset/56f2730388ee38631ff07b09}{Eléments linéaires engagés dans une mesure agro-environnementale territorialisée pour la campagne 2012 - Rhône-Alpes}
\item \href{https://data.gouv.fr/dataset/557f1b47c751df61371d4c2a}{Eléments linéaires engagés dans une mesure agro-environnementale territorialisée pour la campagne 2013 - Rhône-Alpes}
\item \href{https://data.gouv.fr/dataset/56f27306c751df2ff5e0c744}{Eléments ponctuels engagés dans une mesure agro-environnementale territorialisée pour la campagne 2008 - Rhône-Alpes}
\item \href{https://data.gouv.fr/dataset/56f2730b88ee382d53f07b0b}{Eléments ponctuels engagés dans une mesure agro-environnementale territorialisée pour la campagne 2009 - Rhône-Alpes}
\item \href{https://data.gouv.fr/dataset/56f2730788ee384cd9f07b04}{Eléments ponctuels engagés dans une mesure agro-environnementale territorialisée pour la campagne 2010 - Rhône-Alpes}
\item \href{https://data.gouv.fr/dataset/56f2730bc751df2ff5e0c748}{Eléments ponctuels engagés dans une mesure agro-environnementale territorialisée pour la campagne 2011 - Rhône-Alpes}
\item \href{https://data.gouv.fr/dataset/56f2730388ee38738af07ae7}{Eléments ponctuels engagés dans une mesure agro-environnementale territorialisée pour la campagne 2012 - Rhône-Alpes}
\item \href{https://data.gouv.fr/dataset/557f1b47c751df366a1d4c08}{Eléments ponctuels engagés dans une mesure agro-environnementale territorialisée pour la campagne 2013 - Rhône-Alpes}
\item \href{https://data.gouv.fr/dataset/56f27309c751df05dee0c747}{Eléments surfaciques engagés dans une mesure agro-environnementale territorialisée pour la campagne 2007 - Rhône-Alpes}
\item \href{https://data.gouv.fr/dataset/56f2730988ee38631ff07b0a}{Eléments surfaciques engagés dans une mesure agro-environnementale territorialisée pour la campagne 2008 - Rhône-Alpes}
\item \href{https://data.gouv.fr/dataset/56f2730a88ee38631ff07b0b}{Eléments surfaciques engagés dans une mesure agro-environnementale territorialisée pour la campagne 2009 - Rhône-Alpes}
\item \href{https://data.gouv.fr/dataset/56f27307c751df2ff5e0c745}{Eléments surfaciques engagés dans une mesure agro-environnementale territorialisée pour la campagne 2010 - Rhône-Alpes}
\item \href{https://data.gouv.fr/dataset/56f2730ac751df2ff5e0c746}{Eléments surfaciques engagés dans une mesure agro-environnementale territorialisée pour la campagne 2011 - Rhône-Alpes}
\item \href{https://data.gouv.fr/dataset/56f2730d88ee38738af07aea}{Eléments surfaciques engagés dans une mesure agro-environnementale territorialisée pour la campagne 2012 - Rhône-Alpes}
\item \href{https://data.gouv.fr/dataset/557f1b4788ee386706f4b5a5}{Eléments surfaciques engagés dans une mesure agro-environnementale territorialisée pour la campagne 2013 - Rhône-Alpes}
\item \href{https://data.gouv.fr/dataset/59b66730c751df2417daa080}{Enjeux avifaune et lignes électriques (électrocution et percussion) en Rhône-Alpes}
\item \href{https://data.gouv.fr/dataset/557f089ec751df46011d4bfd}{Établissement d'équarissage - Rhône-Alpes}
\item et 96 autres jeux de données\end{itemize}

\clearpage
\section{Département de Saône-et-Loire}


\begin{center}
  \includegraphics[width=3cm]{images/orga/08_73fed58a8b4a18ba92a1749fc01efb-100.png}
\end{center}


\textbf{Intensément Saône-et-Loire}

Vaste territoire géographique, la Saône-et-Loire n'en demeure pas moins
un territoire riche de ses nombreux atouts. Gastronomie, culture ou
encore patrimoine architectural sont autant d'invitations aux voyages.
Si hier, la Saône-et-Loire n'était qu'une terre de passage pour de
nombreux vacanciers désirant rejoindre le Sud de la France, désormais
notre département est une destination touristique à part entière.

Facile d'accès à 1 h 35 de Paris et 45 minutes de Lyon en TGV, la
Saône-et-Loire s'épanouit entre deux cours d'eau, la Saône et la Loire,
offrant de nombreux espaces naturels et activités nautiques (canoë,
kayak, croisière\ldots{}). Les itinéraires cyclables et pédestres,
comprenant près de 300 km de Voies verte et bleue, enrichissent toujours
plus l'offre de déplacements doux. La Saône-et-Loire est reconnue pour
sa gastronomie autour du vin, de la viande charolaise, des poulardes de
Bresse et de nombreuses AOC. Avec un patrimoine culturel riche et varié,
le département traverse les époques de la Préhistoire avec la Roche de
Solutré, de l'antiquité avec les remparts gallo-romain d'Autun, du Moyen
Âge avec l'abbaye de Cluny et autant d'autres patrimoines à découvrir.
\textbf{Le Conseil départemental en chiffres :}

\begin{itemize}
\item
  58 élus
\item
  29 cantons
\item
  2 200 agents
\item
  120 métiers \textbf{Priorités :}
\item
  \textbf{Stratégies territoriales :} Déploiement du Très haut débit et
  de la fibre optique, routes, transport, eau, agriculture, le
  Département s'efforce de proposer des réponses adaptées aux enjeux
  spécifiques à chaque territoire.
\item
  \textbf{Solidarités :} Familles, enfants, personnes âgées, handicapées
  ou en difficulté sociale, le Département de Saône-et-Loire est aux
  côtés de tous les habitants à chaque étape de leur vie.
\item
  \textbf{Collèges :} Équipement informatique et numérique, travaux,
  sécurisation face au risque d'attentat terroriste, développement des
  circuits courts, le Département s'attache à offrir aux collégiens les
  meilleurs conditions d'enseignement et d'éducation possibles.
\item
  \textbf{Attractivité :} Culture, lecture publique, conservation ou
  valorisation du patrimoine et tourisme, le Département est un acteur
  incontournable et participe au quotidien à l'amélioration du cadre de
  vie des Saône-et-Loiriens.
\end{itemize}

\href{http://www.saoneetloire71.fr/}{www.saoneetloire71.fr}


\vspace{0.5cm}

\needspace{12\baselineskip}
\subsection*{Aide à l'amélioration de l'habitat
}\index{aide}\index{chauffage}\index{habitat}\index{isolation}\index{menuiserie}\index{sanitaire}\index{social}\index{toiture}
  \begin{wrapfigure}{r}{2.5cm}
    \centering
    \qrcode[nolink]{https://data.gouv.fr/dataset/53698eafa3a729239d203522}
  \end{wrapfigure}

Licence : \textbf{Licence Ouverte
}\newline
Créé le : 2013-10-17\newline
Modifié le : 2018-02-05\newline
De 1992-01-01 à 2009-12-31\newline
Granularité : à la commune\newline
Mise à jour : irrégulière\newline
Popularité : 1 réutilisation,  0 suivi\newline
Mots-clé : \emph{aide, chauffage, habitat, isolation, menuiserie, sanitaire, social, toiture
}\newline
Permalien : \url{https://data.gouv.fr/dataset/53698eafa3a729239d203522}\newline

\par
\noindent
    Aide fournie pour l'amélioration de l'habitat : chauffage, toiture,
isolation, sanitaires, menuiseries.


\vspace{0.5cm}
\needspace{12\baselineskip}
\subsection*{Budget du cabinet du Président
}\index{budget}\index{cabinet}\index{president}\index{spl}\index{transparence}
  \begin{wrapfigure}{r}{2.5cm}
    \centering
    \qrcode[nolink]{https://data.gouv.fr/dataset/53698fc3a3a729239d20381e}
  \end{wrapfigure}

Licence : \textbf{Licence Ouverte
}\newline
Créé le : 2013-10-17\newline
Modifié le : 2018-02-05\newline
De 2001-01-01 à 2012-12-31\newline
Granularité : au département\newline
Mise à jour : irrégulière\newline
Popularité : 1 réutilisation,  1 suivi\newline
Mots-clé : \emph{budget, cabinet, president, spl, transparence
}\newline
Permalien : \url{https://data.gouv.fr/dataset/53698fc3a3a729239d20381e}\newline

\par
\noindent
    Le Cabinet œuvre auprès et sous l'autorité du Président du Conseil
départemental. Ses principales missions sont de : - garantir des
réponses adaptées à toutes les sollicitations en direction du Président
et des Vice-Présidents - informer le Président et les Vice-Présidents de
l'évolution des dossiers en cours - préparer les interventions et
déplacements du Président et des Vice-Présidents - assurer le lien entre
les élus et l'administration


\vspace{0.5cm}
\needspace{12\baselineskip}
\subsection*{Données des marchés publics (expérimentation OpenDataLocale) du
Département de Saône-et-Loire
}\index{commande!publique}\index{donnees!essentielles}\index{marche!notifie}\index{marche!public}\index{marches!publics}
  \begin{wrapfigure}{r}{2.5cm}
    \centering
    \qrcode[nolink]{https://data.gouv.fr/dataset/59f2031588ee380aace0cd90}
  \end{wrapfigure}

Licence : \textbf{Licence Ouverte
}\newline
Créé le : 2017-10-26\newline
Modifié le : 2018-02-05\newline
De 2017-01-01 à 2018-01-11\newline
Granularité : au département\newline
Mise à jour : ponctuelle\newline
Popularité : 1 réutilisation,  0 suivi\newline
Mots-clé : \emph{commande-publique, donnees-essentielles, marche-notifie, marche-public, marches-publics
}\newline
Permalien : \url{https://data.gouv.fr/dataset/59f2031588ee380aace0cd90}\newline

\par
\noindent
    Vous êtes sur le point de consulter les données des contrats relatifs
aux marchés passés sur le profil acheteur e-bourgogne-franche-comté par
le Département de Saône-et-Loire depuis le début 2017.

\textbf{La mise à disposition de ces données s'opère dans le cadre d'une
expérimentation, OpendataLocale, ayant pour objectif d'anticiper
l'échéance règlementaire du 1er octobre 2018} relative aux données
essentielles des marchés publics. Il doit être noté à ce titre qu'en
attendant la mise à jour de la plateforme, actuellement la saisie des
champs « contrat » n'est pas obligatoire et dépend des pratiques
internes de chaque organisme.

Ainsi, \textbf{il manque encore certaines informations} (à titre
d'exemple : le lieu d'exécution) pour être totalement conforme à
l'article 107 du décret relatif aux marchés publics. De même,
\textbf{tous les contrats des expérimentateurs pour la période
concernées n'y sont pas encore référencés}. \textbf{Ces informations ne
sauront donc pas être interprétées comme une représentation fidèle de la
commande publique, au niveau local comme au niveau régional}. En cas
d'interrogation, vous pourrez vous rapprocher de l'organisme producteur
des données pour obtenir des explications complémentaires.

Cependant, cette expérimentation vise précisément à identifier et à
anticiper les besoins en termes d'évolution des pratiques de saisie et
permet également de communiquer avant l'ouverture des données relatives
aux marchés publics. En attendant l'arrivée des données essentielles
(Octobre 2018), cette extraction sera mise à jour mensuellement pour
intégrer d'éventuelles corrections et de nouveaux contrats conclus.


\vspace{0.5cm}
\needspace{12\baselineskip}
\subsection*{Liste des déchèteries et des ressourceries en Saône-et-Loire
}\index{dechet}\index{decheterie}\index{environnement}\index{selectif}\index{tri}
  \begin{wrapfigure}{r}{2.5cm}
    \centering
    \qrcode[nolink]{https://data.gouv.fr/dataset/536998efa3a729239d2050cf}
  \end{wrapfigure}

Licence : \textbf{Licence Ouverte
}\newline
Créé le : 2013-12-18\newline
Modifié le : 2018-02-06\newline
De 2012-01-01 à 2012-12-31\newline
Granularité : à la commune\newline
Mise à jour : irrégulière\newline
Popularité : 1 réutilisation,  1 suivi\newline
Mots-clé : \emph{dechet, decheterie, environnement, selectif, tri
}\newline
Permalien : \url{https://data.gouv.fr/dataset/536998efa3a729239d2050cf}\newline

\par
\noindent
    En Saône-et-Loire, les ménages assurent en partie le tri des différents
déchets ne pouvant pas être pris en charge par les autres collectes,
notamment par le ramassage des Ordures Ménagères, du fait de leur
encombrement ou de leur toxicité (encombrants, ferraille, déchets
dangereux\ldots{}). Pour cela 71 déchèteries ont été construites par les
collectivités territoriales et ce dès 1987, pour assurer la collecte de
ces déchets, tant pour les particuliers que pour les petites entreprises
artisanales et les professionnels du bâtiment.

D'autre part 3 ressourceries, lieux associatifs permettant de donner une
deuxième vie aux objets et autres appareils ménagers ont également vu le
jour.


\vspace{0.5cm}
\needspace{12\baselineskip}
\subsection*{Patrimoine Culturel
}\index{culturel}\index{musee}\index{parc}\index{patrimoine}\index{site}\index{tourisme}
  \begin{wrapfigure}{r}{2.5cm}
    \centering
    \qrcode[nolink]{https://data.gouv.fr/dataset/53699bdba3a729239d205835}
  \end{wrapfigure}

Licence : \textbf{Licence Ouverte
}\newline
Créé le : 2013-10-17\newline
Modifié le : 2018-04-09\newline
Mise à jour : irrégulière\newline
Popularité : 1 réutilisation,  1 suivi\newline
Mots-clé : \emph{culturel, musee, parc, patrimoine, site, tourisme
}\newline
Permalien : \url{https://data.gouv.fr/dataset/53699bdba3a729239d205835}\newline

\par
\noindent
    Patrimoine culturel en Saône et Loire


\vspace{0.5cm}
\needspace{3\baselineskip} \rule{4cm}{0.25pt}\newline\textbf{Aussi disponible du même producteur :}\begin{itemize}
\item \href{https://data.gouv.fr/dataset/53698e6ca3a729239d20346a}{Activités de loisirs}
\item \href{https://data.gouv.fr/dataset/53698eb1a3a729239d203528}{Aide à l'électrification rurale}
\item \href{https://data.gouv.fr/dataset/53698eb5a3a729239d203531}{Aide aux communes}
\item \href{https://data.gouv.fr/dataset/53698ed3a3a729239d203592}{Allocation personnalisée à l'autonomie}
\item \href{https://data.gouv.fr/dataset/53698ee7a3a729239d2035d0}{Appels reçus au 18 et 112}
\item \href{https://data.gouv.fr/dataset/53d045c9a3a72970fff91f10}{Bailleurs sociaux}
\item \href{https://data.gouv.fr/dataset/53698f3aa3a729239d2036a8}{Balades Vertes}
\item \href{https://data.gouv.fr/dataset/53698f4ba3a729239d2036db}{Barrières de dégel}
\item \href{https://data.gouv.fr/dataset/5ab13210c751df1ce93eac75}{Budget annexe de l'EHPAD de Mervans}
\item \href{https://data.gouv.fr/dataset/5ab131b2c751df1cd8694604}{Budget annexe du Centre de Santé Départemental}
\item \href{https://data.gouv.fr/dataset/5ab13277c751df1cfb01f9d8}{Budget annexe du Centre Équestre}
\item \href{https://data.gouv.fr/dataset/5ab132e5c751df1edbdaa3ef}{Budget annexe du Laboratoire Départemental d'Analyses 71}
\item \href{https://data.gouv.fr/dataset/5ab130e3c751df16bd5d1733}{Budget annexe du Réseau d'Initiative Publique - THD}
\item \href{https://data.gouv.fr/dataset/53698fc2a3a729239d20381b}{Budget de la direction de la communication}
\item \href{https://data.gouv.fr/dataset/5aafe61b88ee386df56ff34e}{Budget Principal du Département}
\item \href{https://data.gouv.fr/dataset/5369903fa3a729239d203966}{Catalogue de la bibliothèque des AD71}
\item \href{https://data.gouv.fr/dataset/53699a59a3a729239d205482}{Catégorisation des routes départementales}
\item \href{https://data.gouv.fr/dataset/53d04714a3a72970fff91f15}{Centres de planification et d'éducation familiale}
\item \href{https://data.gouv.fr/dataset/53699064a3a729239d2039c2}{Chambres d'hôtes}
\item \href{https://data.gouv.fr/dataset/5369907da3a729239d2039fd}{Chèques habitat durable}
\item \href{https://data.gouv.fr/dataset/5369910aa3a729239d203b63}{Commissions Locales d'Insertion (CLI)}
\item \href{https://data.gouv.fr/dataset/53699132a3a729239d203bcb}{Compostage domestique des déchets}
\item \href{https://data.gouv.fr/dataset/5369923ea3a729239d203e90}{Dégustation des produits du terroir}
\item \href{https://data.gouv.fr/dataset/536992d4a3a729239d204018}{Détail de la collecte sélective des déchets}
\item \href{https://data.gouv.fr/dataset/536992eda3a729239d204058}{Discothèque}
\item \href{https://data.gouv.fr/dataset/5a9d7413c751df550b523e21}{Données financières}
\item \href{https://data.gouv.fr/dataset/5c8266978b4c4120a968bd91}{Données mises à disposition par les Archives départementales de Saône-et-Loire}
\item \href{https://data.gouv.fr/dataset/5a60afb9c751df1c2e6072d6}{Ecoles de musique}
\item \href{https://data.gouv.fr/dataset/5369936fa3a729239d2041ba}{Effectifs des collèges}
\item \href{https://data.gouv.fr/dataset/53699373a3a729239d2041c3}{Effectifs des ULIS (ex UPI)}
\item \href{https://data.gouv.fr/dataset/53699515a3a729239d204613}{Euro J+}
\item \href{https://data.gouv.fr/dataset/5369957ea3a729239d204734}{Fiches des collèges}
\item \href{https://data.gouv.fr/dataset/536995aca3a729239d2047b2}{Fond d'aide aux jeunes}
\item \href{https://data.gouv.fr/dataset/536995ada3a729239d2047b7}{Fond de solidarité au logement}
\item \href{https://data.gouv.fr/dataset/536995c8a3a729239d2047ff}{Fournisseurs du Département à partir de 2004}
\item \href{https://data.gouv.fr/dataset/536995c8a3a729239d2047fe}{Fournisseurs du Département de 2001 à 2003}
\item \href{https://data.gouv.fr/dataset/536995c9a3a729239d204802}{Frais de personnel}
\item \href{https://data.gouv.fr/dataset/536995eaa3a729239d20485c}{Galeries d'art}
\item \href{https://data.gouv.fr/dataset/536995faa3a729239d204892}{Gîtes et meublés}
\item \href{https://data.gouv.fr/dataset/5369960fa3a729239d2048c7}{Hébergements (2011)}
\item \href{https://data.gouv.fr/dataset/53699610a3a729239d2048c9}{Hébergements collectifs}
\item \href{https://data.gouv.fr/dataset/53699611a3a729239d2048cb}{Hébergements innovants}
\item \href{https://data.gouv.fr/dataset/53699638a3a729239d204930}{Hotellerie plein air}
\item \href{https://data.gouv.fr/dataset/5369963ba3a729239d204938}{Hôtels}
\item \href{https://data.gouv.fr/dataset/5c517ae4634f4159cce0daaa}{Implantations territoriales de l’action sociale}
\item \href{https://data.gouv.fr/dataset/53699735a3a729239d204c27}{Interventions des pompiers}
\item \href{https://data.gouv.fr/dataset/53699843a3a729239d204eef}{Les festivals en Saône-et-Loire}
\item \href{https://data.gouv.fr/dataset/53699923a3a729239d205172}{Liste des livres}
\item \href{https://data.gouv.fr/dataset/536999cfa3a729239d205330}{Maturation du raisin - Données parcellaires}
\item \href{https://data.gouv.fr/dataset/536999cfa3a729239d205331}{Maturation du raisin - Moyennes par cépage}
\item et 18 autres jeux de données\end{itemize}

\clearpage
\section{Département des Côtes d'Armor}


\begin{center}
  \includegraphics[width=3cm]{images/orga/2015-01-09_0dea259a0e95466da06dbf06a4569f56_datarmor_logo-100.png}
\end{center}


Le conseil départemental des Côtes-d'Armor est la collectivité
territoriale du département français des Côtes-d'Armor. Son siège se
trouve à Saint-Brieuc.


\vspace{0.5cm}

\needspace{12\baselineskip}
\subsection*{Archives Départementales des Côtes d'Armor: inventaire de la collection
d'affiches anciennes accessible en ligne
}\index{affiches}\index{archives}\index{collection}
  \begin{wrapfigure}{r}{2.5cm}
    \centering
    \qrcode[nolink]{https://data.gouv.fr/dataset/57e39916c751df7b6479df74}
  \end{wrapfigure}

Licence : \textbf{Licence Ouverte
}\newline
Créé le : 2016-09-22\newline
Modifié le : 2016-12-13\newline
De 2016-12-13 à 2016-12-13\newline
Mise à jour : annuelle\newline
Popularité : 1 réutilisation,  0 suivi\newline
Mots-clé : \emph{affiches, archives, collection
}\newline
Permalien : \url{https://data.gouv.fr/dataset/57e39916c751df7b6479df74}\newline

\par
\noindent
    Dans le cadre de leur mission de conservation, de communication et de
valorisation du patrimoine écrit dont elles ont la charge, les Archives
départementales des Côtes-d'Armor conduisent depuis plusieurs années des
opérations de numérisation. Certains types de documents ont ainsi été
numérisés (notamment : plans, cartes postales, photographies, affiches,
état civil, listes nominatives de recensement de population, registres
matricules militaires) mais les 5 millions d'images consultables en
ligne ne représentent, à ce jour, qu'une partie des fonds conservés par
les Archives départementales.

En complément d'un espace de consultation multimédia aménagé dans la
salle de lecture, cette volonté de large diffusion est illustrée par la
mise à disposition d'une téléconsultation gratuite via Internet.

C'est autour du projet de cette ``salle de lecture virtuelle'' que
seront développées dans les prochaines années ces ``archives en ligne''
: images de documents d'archives mais également instruments de recherche
qui constituent en quelque sorte les clés d'accès aux fonds conservés.


\vspace{0.5cm}
\needspace{12\baselineskip}
\subsection*{Archives Départementales des Côtes d'Armor: inventaire de la collection
de cartes postales accessible en ligne
}\index{archives}\index{cartes!postales}\index{collection}
  \begin{wrapfigure}{r}{2.5cm}
    \centering
    \qrcode[nolink]{https://data.gouv.fr/dataset/57e39947c751df799079df72}
  \end{wrapfigure}

Licence : \textbf{Licence Ouverte
}\newline
Créé le : 2016-09-22\newline
Modifié le : 2016-12-13\newline
De 2016-12-13 à 2016-12-13\newline
Mise à jour : annuelle\newline
Popularité : 1 réutilisation,  0 suivi\newline
Mots-clé : \emph{archives, cartes-postales, collection
}\newline
Permalien : \url{https://data.gouv.fr/dataset/57e39947c751df799079df72}\newline

\par
\noindent
    Dans le cadre de leur mission de conservation, de communication et de
valorisation du patrimoine écrit dont elles ont la charge, les Archives
départementales des Côtes-d'Armor conduisent depuis plusieurs années des
opérations de numérisation. Certains types de documents ont ainsi été
numérisés (notamment : plans, cartes postales, photographies, affiches,
état civil, listes nominatives de recensement de population, registres
matricules militaires) mais les 5 millions d'images consultables en
ligne ne représentent, à ce jour, qu'une partie des fonds conservés par
les Archives départementales.

En complément d'un espace de consultation multimédia aménagé dans la
salle de lecture, cette volonté de large diffusion est illustrée par la
mise à disposition d'une téléconsultation gratuite via Internet.

C'est autour du projet de cette ``salle de lecture virtuelle'' que
seront développées dans les prochaines années ces ``archives en ligne''
: images de documents d'archives mais également instruments de recherche
qui constituent en quelque sorte les clés d'accès aux fonds conservés.


\vspace{0.5cm}
\needspace{12\baselineskip}
\subsection*{Fibre optique: noeuds de la montée en débit dans le département des
Côtes d'Armor
}\index{debit}\index{fibre!optique}\index{internet}\index{noeud}\index{telecommunication}
  \begin{wrapfigure}{r}{2.5cm}
    \centering
    \qrcode[nolink]{https://data.gouv.fr/dataset/56a8a4ac88ee38068e2efbf5}
  \end{wrapfigure}

Licence : \textbf{Licence Ouverte
}\newline
Créé le : 2016-01-27\newline
Modifié le : 2016-12-13\newline
De 2016-12-13 à 2016-12-13\newline
Mise à jour : annuelle\newline
Popularité : 1 réutilisation,  0 suivi\newline
Mots-clé : \emph{debit, fibre-optique, internet, noeud, telecommunication
}\newline
Permalien : \url{https://data.gouv.fr/dataset/56a8a4ac88ee38068e2efbf5}\newline

\par
\noindent
    Montée en débit de la fibre optique sur le département des Côtes
d'Armor. Géolocalisation et descriptif d'un local technique ou d'une
chambre de télécommunication.


\vspace{0.5cm}
\needspace{3\baselineskip} \rule{4cm}{0.25pt}\newline\textbf{Aussi disponible du même producteur :}\begin{itemize}
\item \href{https://data.gouv.fr/dataset/56eab1cf88ee3871d789c52c}{Aides du Conseil Départemental des Côtes d'Armor aux Comités Sportifs Départementaux}
\item \href{https://data.gouv.fr/dataset/5672b20088ee38329faf0bf4}{Aires de covoiturages des Côtes d'Armor}
\item \href{https://data.gouv.fr/dataset/56b1bb5888ee38694a81d141}{Antennes du Conseil Départemental des Côtes d'Armor}
\item \href{https://data.gouv.fr/dataset/574e9db0c751df44cf535dd5}{Arbres remarquables des Côtes d'Armor}
\item \href{https://data.gouv.fr/dataset/5757e8f6c751df24b7ac31a0}{Archives Départementales des Côtes d'Armor: bilan des entrées de documents}
\item \href{https://data.gouv.fr/dataset/5756ce19c751df73d9ac31a0}{Archives Départementales des Côtes d'Armor: évolution des services rendus au public}
\item \href{https://data.gouv.fr/dataset/5720cae288ee3830b0a19f12}{Archives Départementales des Côtes d'Armor: expositions}
\item \href{https://data.gouv.fr/dataset/5756ce3988ee380f5a640391}{Archives Départementales des Côtes d'Armor: fréquentation du service éducatif}
\item \href{https://data.gouv.fr/dataset/5720cf7cc751df022efcca0d}{Archives Départementales des Côtes d'Armor: fréquentation par les établissements scolaires}
\item \href{https://data.gouv.fr/dataset/57e3995dc751df0e3579df72}{Archives Départementales des Côtes d'Armor: inventaire de la collection de presse ancienne accessible en ligne}
\item \href{https://data.gouv.fr/dataset/57f7abb4c751df352c79df72}{Archives Départementales des Côtes d'Armor: inventaire de la collection des listes nominatives du recensement de population accessible en ligne}
\item \href{https://data.gouv.fr/dataset/57f7bb28c751df50dc79df72}{Archives Départementales des Côtes d'Armor: inventaire de la collection des rapports et délibérations du Conseil Général accessible en ligne}
\item \href{https://data.gouv.fr/dataset/57f7b24fc751df401e79df72}{Archives Départementales des Côtes d'Armor: inventaire de la collection des registres de recensement militaire accessible en ligne}
\item \href{https://data.gouv.fr/dataset/57fb9c6588ee385b495ff490}{Archives Départementales des Côtes d'Armor: Inventaire de la collection des registres d'état civil accessible en ligne}
\item \href{https://data.gouv.fr/dataset/57fbaa8988ee3875b35ff490}{Archives Départementales des Côtes d'Armor: inventaire de la collection des registres paroissiaux accessible en ligne}
\item \href{https://data.gouv.fr/dataset/57e3998788ee383d5b5ff490}{Archives Départementales des Côtes d'Armor: inventaire des collections de photographies accessibles en ligne}
\item \href{https://data.gouv.fr/dataset/5822ed7ec751df7f71c0bb7e}{Associations et structures Sportives accueillant les personnes en situation de handicap dans les Côtes d'Armor}
\item \href{https://data.gouv.fr/dataset/582dbacdc751df5570c0bb7e}{Bibliothèque Départementale des Côtes d'Armor: acquisition livres Adulte par genre}
\item \href{https://data.gouv.fr/dataset/584140e188ee380fc7c65bb3}{Bibliothèque Départementale des Côtes d'Armor: acquisition livres Jeunesse par genre}
\item \href{https://data.gouv.fr/dataset/57f4f18bc751df48e779df72}{Bibliothèque Départementale des Côtes d'Armor: circuit des navettes}
\item \href{https://data.gouv.fr/dataset/58414022c751df2e53c0bb7e}{Bibliothèque Départementale des Côtes d'Armor: collections par type de documents (Exemplaires)}
\item \href{https://data.gouv.fr/dataset/582d898dc751df3e8ac0bb7e}{Bibliothèque Départementale des Côtes d'Armor: collections par type de documents (Notices)}
\item \href{https://data.gouv.fr/dataset/584141a0c751df2f18c0bb7e}{Bibliothèque Départementale des Côtes d'Armor: création des exemplaires par type de documents}
\item \href{https://data.gouv.fr/dataset/5841428988ee381088c65bb3}{Bibliothèque Départementale des Côtes d'Armor: création des notices par type de documents}
\item \href{https://data.gouv.fr/dataset/582dad88c751df4ed1c0bb7e}{Bibliothèque Départementale des Côtes d'Armor: documents désaffectés}
\item \href{https://data.gouv.fr/dataset/582dabcfc751df4df7c0bb7e}{Bibliothèque Départementale des Côtes d'Armor: documents facturés aux bibliothèques}
\item \href{https://data.gouv.fr/dataset/582d920ac751df4256c0bb7e}{Bibliothèque Départementale des Côtes d'Armor: Etat de la collection des DVD (Exemplaires)}
\item \href{https://data.gouv.fr/dataset/582d92b0c751df4292c0bb7e}{Bibliothèque Départementale des Côtes d'Armor: Etat de la collection des DVD (Notices)}
\item \href{https://data.gouv.fr/dataset/582d901cc751df417ac0bb7e}{Bibliothèque Départementale des Côtes d'Armor: Etat de la collection des livres (Exemplaires)}
\item \href{https://data.gouv.fr/dataset/582d910f88ee385a47c65bb3}{Bibliothèque Départementale des Côtes d'Armor: Etat de la collection des livres (Notices)}
\item \href{https://data.gouv.fr/dataset/57e2424588ee382d135ff491}{Bibliothèque Départementale des Côtes d'Armor: état de la collection des livres numériques (ebook)}
\item \href{https://data.gouv.fr/dataset/57e242bac751df7d0a79df73}{Bibliothèque Départementale des Côtes d'Armor: étude comparative des collections et emprunts pour les bibliothèques du 22, 29 et 35}
\item \href{https://data.gouv.fr/dataset/582db45388ee385f37c65bb3}{Bibliothèque Départementale des Côtes d'Armor: prêt de livres Adulte par genre}
\item \href{https://data.gouv.fr/dataset/582db7d488ee385f37c65bb4}{Bibliothèque Départementale des Côtes d'Armor: prêt de livres Jeunesse par genre}
\item \href{https://data.gouv.fr/dataset/58357845c751df2bc7c0bb7e}{Bibliothèque Départementale des Côtes d'Armor: prêt des documents par commune}
\item \href{https://data.gouv.fr/dataset/582dafb488ee385ce8c65bba}{Bibliothèque Départementale des Côtes d'Armor: prêt par type de documents}
\item \href{https://data.gouv.fr/dataset/582da8e388ee385ce8c65bb9}{Bibliothèque Départementale des Côtes d'Armor: réservation de documents par niveau de bibliothèque}
\item \href{https://data.gouv.fr/dataset/582da6a188ee385ce8c65bb7}{Bibliothèque Départementale des Côtes d'Armor: réservation de documents pour la jeunesse}
\item \href{https://data.gouv.fr/dataset/582d985ec751df4525c0bb7e}{Bibliothèque Départementale des Côtes d'Armor: réservation de documents pour les Adultes}
\item \href{https://data.gouv.fr/dataset/57e2428288ee382d135ff492}{Bibliothèque Départementale des Côtes d'Armor: ressource numérique/ formation en ligne/ connexion par catalogue}
\item \href{https://data.gouv.fr/dataset/57e2429888ee382d325ff490}{Bibliothèque Départementale des Côtes d'Armor: ressource numérique/ formation en ligne/ connexion par cours}
\item \href{https://data.gouv.fr/dataset/57e24265c751df7d0a79df72}{Bibliothèque Départementale des Côtes d'Armor: ressource numérique/ formation en ligne / connexion par jour}
\item \href{https://data.gouv.fr/dataset/582d7b3ec751df37afc0bb7e}{Bibliothèque Départementales des Côtes d'Armor: Statistiques de l'activité}
\item \href{https://data.gouv.fr/dataset/5a1d46e7c751df151abbb304}{Bibliothèques départementales des Côtes d'Armor}
\item \href{https://data.gouv.fr/dataset/56a799c888ee387c68822417}{Budget principal du Conseil Départemental des Côtes d'Armor}
\item \href{https://data.gouv.fr/dataset/573dc9de88ee386598d1b934}{Camélias du Domaine de la Roche Jagu}
\item \href{https://data.gouv.fr/dataset/56a23dafc751df4324ade713}{Classification du réseau routier départemental (Réseau A / B) des Côtes d'Armor}
\item \href{https://data.gouv.fr/dataset/59f6e678c751df2e37801526}{Collèges des Côtes d'Armor}
\item \href{https://data.gouv.fr/dataset/5672c20f88ee383443af0bf8}{Collèges des Côtes d'Armor}
\item \href{https://data.gouv.fr/dataset/5672c257c751df0404c664be}{Communes comprenant des Zones de préemption Espace Naturel Sensible (ENS) dans les Côtes d'Armor}
\item et 67 autres jeux de données\end{itemize}

\clearpage
\section{Département des Hautes-Alpes}


\begin{center}
  \includegraphics[width=3cm]{images/orga/b3_3c12a19dfb4eedb21f99f36b12a7c8-100.png}
\end{center}


Le conseil départemental des Hautes-Alpes est l'assemblée délibérante du
département français des Hautes-Alpes, collectivité territoriale
décentralisée. Son siège se trouve à Gap, place Saint-Arnoux, dans un
bâtiment moderne. (Source : Wikipedia, 2018)


\vspace{0.5cm}

\needspace{12\baselineskip}
\subsection*{Localisation des défibrillateurs appartenant au département des Hautes
Alpes
}\index{donnees!ouvertes}\index{health}\index{passerelle!inspire}\index{sante!et!securite!des!personnes}\index{sante!humaine}
  \begin{wrapfigure}{r}{2.5cm}
    \centering
    \qrcode[nolink]{https://data.gouv.fr/dataset/5acb4dc2c751df71629a1e28}
  \end{wrapfigure}

Licence : \textbf{Licence Ouverte version 2.0
}\newline
Créé le : 2018-04-09\newline
Modifié le : 2019-02-08\newline
Popularité : 1 réutilisation,  0 suivi\newline
Mots-clé : \emph{donnees-ouvertes, health, passerelle-inspire, sante-et-securite-des-personnes, sante-humaine
}\newline
Permalien : \url{https://data.gouv.fr/dataset/5acb4dc2c751df71629a1e28}\newline

\par
\noindent
    Défibrillateurs du département des Hautes Alpes

\textbf{Origine}

Saisie SIG par agent du département avec localisation à partir de
l'ORTHO et de Google Map

\textbf{Organisations partenaires}

Département 05

\textbf{Liens annexes}

\begin{itemize}

\item
  \href{http://www.crige-paca.org/geoportail/geocatalogue.html?id_lot_donnee_carto=852}{geocatalogue.html}
\end{itemize}

➞
\href{https://geo.data.gouv.fr/fr/datasets/eb467e211b24d2a9760029e6c85edd7b5632451a}{Consulter
cette fiche sur geo.data.gouv.fr}


\vspace{0.5cm}
\needspace{3\baselineskip} \rule{4cm}{0.25pt}\newline\textbf{Aussi disponible du même producteur :}\begin{itemize}
\item \href{https://data.gouv.fr/dataset/5acb4d7988ee384359719663}{Classement catégoriel des routes départementales des Hautes-Alpes}
\item \href{https://data.gouv.fr/dataset/5acb4d7988ee383dc6573486}{Découpage cantonal du Département des Hautes-Alpes suite aux élections de 2012}
\item \href{https://data.gouv.fr/dataset/5acb4d6ec751df732afacd20}{Découpage cantonal du Département des Hautes-Alpes suite aux élections de 2015}
\item \href{https://data.gouv.fr/dataset/5acb4da1c751df713983a9a4}{Découpage communal des Hautes-Alpes au 1er janvier 2018}
\item \href{https://data.gouv.fr/dataset/5acb4d6cc751df713983a9a2}{Découpage communal des Hautes-Alpes au 1er juillet 2017}
\item \href{https://data.gouv.fr/dataset/5acb4d6b88ee384359719661}{Découpage inter-communal du Département des Hautes-Alpes au 1er janvier 2014}
\item \href{https://data.gouv.fr/dataset/5acb4d6f88ee384359719662}{Découpage inter-communal du Département des Hautes-Alpes au 1er janvier 2017}
\item \href{https://data.gouv.fr/dataset/5acb4d9688ee38419beb260b}{Découpage territorial des Antennes Techniques départementales des Hautes-Alpes}
\item \href{https://data.gouv.fr/dataset/5acb4d6b88ee384361f0a544}{Identification des communes composant les relais d'assistantes maternelles des Hautes-Alpes}
\item \href{https://data.gouv.fr/dataset/5acb4d8688ee384359719665}{Limites des agences des maisons des solidarités au 1er juillet 2017 dans le département des Hautes-Alpes}
\item \href{https://data.gouv.fr/dataset/5acb4dbac751df713983a9a6}{Limites des Pays au 1er janvier 2015 du territoire GéoMAS}
\item \href{https://data.gouv.fr/dataset/5acb4dad88ee3832e343e0a1}{Limites d'exploitation des Antennes Techniques hors saison hivernale sur les routes départementales des Hautes-Alpes}
\item \href{https://data.gouv.fr/dataset/5acb4da888ee38419beb260c}{Limites d'exploitation des Antennes Techniques sur les routes départementales durant la saison hivernale dans les Hautes-Alpes}
\item \href{https://data.gouv.fr/dataset/5acb4d8088ee384359719664}{Localisation de l'offre de services pour les personnes âgées dans le Département des Hautes-Alpes}
\item \href{https://data.gouv.fr/dataset/5acb4d7fc751df71629a1e25}{Localisation des aérodrômes dans le Département des Hautes-Alpes}
\item \href{https://data.gouv.fr/dataset/5acb4d92c751df74f0602256}{Localisation des Antennes Techniques et des Centres Techniques dans le département des Hautes-Alpes}
\item \href{https://data.gouv.fr/dataset/5acb4d6bc751df71629a1e23}{Localisation des bibliothèques du Département des Hautes-Alpes}
\item \href{https://data.gouv.fr/dataset/5acb4db0c751df68429efb34}{Localisation des bornes des itinéraires vélos du département des Hautes-Alpes}
\item \href{https://data.gouv.fr/dataset/5acb4dba88ee38419beb260d}{Localisation des caméras de surveillance vidéo-routières sur les routes départementales des Hautes-Alpes}
\item \href{https://data.gouv.fr/dataset/5acb4d79c751df71629a1e24}{Localisation des circuits de déneigement sur les routes départementales des Hautes-Alpes}
\item \href{https://data.gouv.fr/dataset/5acb4d9e88ee384361f0a547}{Localisation des circuits de salage-sablage sur les routes départementales des Hautes-Alpes}
\item \href{https://data.gouv.fr/dataset/5acb4da7c751df732afacd25}{Localisation des circuits des patrouilles effectuées durant la saison hivernale sur les routes départementales des Hautes-Alpes}
\item \href{https://data.gouv.fr/dataset/5acb4d6a88ee383dc6573484}{Localisation des collèges et lycées du Département des Hautes-Alpes}
\item \href{https://data.gouv.fr/dataset/5acb4d7588ee383dc6573485}{Localisation des cols routiers sur les routes nationales et départementales des Hautes-Alpes}
\item \href{https://data.gouv.fr/dataset/5acb4d8788ee384361f0a545}{Localisation des compteurs routiers du département des Hautes-Alpes}
\item \href{https://data.gouv.fr/dataset/5acb4d8088ee38419beb2608}{Localisation des établissements/partenaires pour enfants handicapés dans le Département des Hautes-Alpes}
\item \href{https://data.gouv.fr/dataset/5acb4d80c751df732afacd22}{Localisation des établissements pour personnes âgées dans le Département des Hautes-Alpes}
\item \href{https://data.gouv.fr/dataset/5acb4dc388ee384361f0a549}{Localisation des établissements pour personnes handicapées dans le département des Hautes-Alpes}
\item \href{https://data.gouv.fr/dataset/5acb4da7c751df71629a1e26}{Localisation des maisons des solidarités au 1er juillet 2017 dans le département des Hautes-Alpes}
\item \href{https://data.gouv.fr/dataset/5acb4d8088ee383dc6573487}{Localisation des Panneaux à Messages Variables (PMV) sur les routes départementales des Hautes-Alpes}
\item \href{https://data.gouv.fr/dataset/5acb4d80c751df732afacd21}{Localisation des permanences puéricultrices, des consultations sage-femme et nourrissons proposées par le Département des Hautes-Alpes}
\item \href{https://data.gouv.fr/dataset/5acb4d9288ee38419beb260a}{Localisation des permanences sociales proposées par le Département des Hautes-Alpes}
\item \href{https://data.gouv.fr/dataset/5acb4db8c751df76906a69c9}{Localisation des plantations d'alignement situées en accotement des routes départementales des Hautes-Alpes}
\item \href{https://data.gouv.fr/dataset/5acb4dadc751df71629a1e27}{Localisation des plaquettes Points de Repères (PR) dans le département des Hautes-Alpes}
\item \href{https://data.gouv.fr/dataset/5acb4db988ee384506ec53c2}{Localisation des points de visio-conférence du Département des Hautes-Alpes}
\item \href{https://data.gouv.fr/dataset/5acb4da7c751df68429efb33}{Localisation des Postes d'Appel d'Urgence (PAU) dans le département des Hautes-Alpes}
\item \href{https://data.gouv.fr/dataset/5acb4da7c751df713983a9a5}{Localisation des relais d'assistantes maternelles mis en place dans le département des Hautes-Alpes}
\item \href{https://data.gouv.fr/dataset/5acb4da988ee384506ec53c0}{Localisation des stations météo-routières dans le département des Hautes-Alpes}
\item \href{https://data.gouv.fr/dataset/5acb4dba88ee383dc657348a}{Localisation des structures d'accueil permanentes et saisonnières des jeunes enfants dans les Hautes-Alpes}
\item \href{https://data.gouv.fr/dataset/5acb4d6fc751df68429efb32}{Localisation des zones à risques en période hivernale sur les routes départementales des Hautes-alpes}
\item \href{https://data.gouv.fr/dataset/5acb4d93c751df713983a9a3}{Matérialisation des itinéraires vélos du département des Hautes-Alpes}
\item \href{https://data.gouv.fr/dataset/5acb4d6e88ee38419beb2607}{Mesure de l'altitude des plaquettes Points de Repères (PR) sur les routes départementales des Hautes-Alpes}
\item \href{https://data.gouv.fr/dataset/5acb4da788ee3832e343e0a0}{Niveaux de service adoptés par rapport à la viabilité hivernale sur les routes départementales (RD) des Hautes-Alpes}
\item \href{https://data.gouv.fr/dataset/5acb4dad88ee384506ec53c1}{Recensement des passages à niveau ayant une intersection avec le réseau routier départemental des Hautes-Alpes}
\item \href{https://data.gouv.fr/dataset/5acb4db188ee384361f0a548}{Recensement des points de restriction de circulation du réseau routier départemental des Hautes-Alpes}
\item \href{https://data.gouv.fr/dataset/5acb4d9688ee384361f0a546}{Recensement des restrictions de circulation du réseau routier départemental des Hautes-Alpes}
\item \href{https://data.gouv.fr/dataset/5acb4d8c88ee38419beb2609}{Recensement des routes départementales fermées en hiver dans le département des Hautes-Alpes}
\item \href{https://data.gouv.fr/dataset/5acb4d8cc751df74f0602255}{Recensement des surlargeurs cyclables présentes sur la voirie départementale des Hautes-Alpes}
\item \href{https://data.gouv.fr/dataset/5acb4d96c751df732afacd24}{Recensement des travaux effectués sur les chaussées dans le département des Hautes-Alpes}
\item \href{https://data.gouv.fr/dataset/5acb4da988ee383dc6573489}{Recensement des tunnels présents sur les routes départementales des Hautes-Alpes}
\item et 4 autres jeux de données\end{itemize}

\clearpage
\section{Direction interministérielle du numérique et du système d'information et de communication de l'Etat}


\begin{center}
  \includegraphics[width=3cm]{images/orga/58_1d2f3b9edc45428304e3ac3d1dd5fa-100.jpg}
\end{center}


La Direction interministérielle du numérique et du système d'information
et de communication de l'Etat (DINSIC) est un service du Premier
ministre en charge de la performance du SI unifié de l'Etat et de la
transformation numérique de l'action publique.

Elle promeut les méthodes d'innovation du monde du numérique, le recours
aux « data sciences », la diffusion des approches agiles, l'ouverture
des données publiques et le Gouvernement ouvert.

Elle veille également à ce que le SI unifié de l'Etat concoure de
manière cohérente à simplifier les relations entre les usagers et les
administrations.

Elle organise et anime la concertation nécessaire à l'évolution: - d'une
\href{https://references.modernisation.gouv.fr/mareva2-cest-quoi}{méthode
interministérielle d'analyse de la valeur des projets} - des
référentiels généraux
d'\href{http://references.modernisation.gouv.fr/interoperabilite}{interopérabilité}et
d'\href{http://references.modernisation.gouv.fr/accessibilite-numerique}{accessibilité},
- des modèles de données de référence et des modèles d'échange notamment
pour un \href{http://etatplateforme.modernisation.gouv.fr/}{État
plateforme}, - en liaison avec l'Agence nationale de la sécurité des
systèmes d'information (\href{https://www.ssi.gouv.fr/}{ANSSI}), du
\href{http://references.modernisation.gouv.fr/securite}{référentiel
général de sécurité}. Elle contribue également, avec la direction des
achats de l'État
(\href{http://www.economie.gouv.fr/dae/presentation}{DAE}), à définir
les règles et procédures applicables pour les achats concourant au SI.

Elle organise et pilote la conception et la mise en œuvre des opérations
de mutualisation entre administrations pour allier efficience et
maîtrise budgétaire
(\href{http://www.modernisation.gouv.fr/ladministration-change-avec-le-numerique/par-son-systeme-dinformation/les-trois-grandes-missions-du-reseau-interministeriel-de-letat}{réseau
interministériel de l'Etat}, téléphonie sécurisée,
\href{http://www.modernisation.gouv.fr/ladministration-change-avec-le-numerique/par-son-systeme-dinformation/l-etat-passe-un-marche-cloud-pour-favoriser-les-chantiers-agiles-de-l-administration}{cloud
hybride de l'Etat},
\href{https://franceconnect.gouv.fr/}{FranceConnect}).

Elle contribue, avec la direction générale de l'administration et de la
fonction publique, à adapter la gestion des ressources humaines des
administrations dans les métiers du numérique.

La DINSIC comporte aujourd'hui quatre structures : - le service
Performance des services numériques ; - le service à compétence
nationale (SCN) du réseau interministériel de l'État (RIE) ; - la
mission Etalab, qui héberge aussi l'équipe de l'administrateur général
des données ; - la mission Incubateur de services numériques.

\emph{Pour suivre l'actualité de la DINSIC et de ses projets,
rendez-vous sur le \href{http://www.modernisation.gouv.fr/}{portail de
la modernisation de l'action publique}.}


\vspace{0.5cm}

\needspace{12\baselineskip}
\subsection*{Panorama des grands projets SI de l'État
}\index{administration}\index{dinsic}\index{numerique}\index{service!public}\index{services!de!l!etat}\index{sgmap}\index{systeme!d!information}\index{tableau!de!bord}
  \begin{wrapfigure}{r}{2.5cm}
    \centering
    \qrcode[nolink]{https://data.gouv.fr/dataset/582e1e2888ee386b7ac65bb4}
  \end{wrapfigure}

Licence : \textbf{Licence Ouverte
}\newline
Créé le : 2016-11-17\newline
Modifié le : 2019-01-23\newline
Granularité : au pays\newline
Mise à jour : semestrielle\newline
Popularité : 5 réutilisations,  3 suivis\newline
Mots-clé : \emph{administration, dinsic, numerique, service-public, services-de-l-etat, sgmap, systeme-d-information, tableau-de-bord
}\newline
Permalien : \url{https://data.gouv.fr/dataset/582e1e2888ee386b7ac65bb4}\newline

\par
\noindent
    Issu d'une collaboration étroite entre la DINSIC et les ministères, le
\textbf{panorama des grands projets SI de l'État} recense et décrit les
principaux projets informatiques au sein de l'Etat.

Il apporte à ce titre : - une vision partagée des principaux projets
informatiques en cours ; - un levier pour valoriser ces projets ou pour
partager les difficultés ; - le témoignage d'une transformation
numérique en marche : dématérialisation, relation usagers, identité
numérique, solutions collaboratives, mobilité, big data\ldots{}

Sur cette base, les ministères et les services du Premier ministre
identifient et engagent les actions nécessaires au succès de ces
projets.

Chacun de ces projets est décrit en dix points : nom, descriptif,
ministère porteur, date de début, durée prévisionnelle, phase en cours,
coût estimatif, caractère interministériel, zone fonctionnelle
principale en relation avec le plan d'occupation des sols.

Le panorama des grands projets SI est disponible sur
\href{http://www.modernisation.gouv.fr/ladministration-change-avec-le-numerique/par-son-systeme-dinformation/panorama-des-grands-projets-si-letat-renforce-ses-capacites-de-pilotage-et-fait-oeuvre-de-transparence}{le
portail de la modernisation de l'action publique}.

Tout citoyen dispose ainsi d'une vue sur les principaux efforts
consentis par l'Etat pour moderniser son système d'information et
engager une transformation numérique.


\vspace{0.5cm}
\needspace{3\baselineskip} \rule{4cm}{0.25pt}\newline\textbf{Aussi disponible du même producteur :}\begin{itemize}
\item \href{https://data.gouv.fr/dataset/5ad74db988ee38033f37f23e}{Publication des avis DINSIC (articles 3 et 7)}
\item \href{https://data.gouv.fr/dataset/5796275a88ee3831d77d97c2}{Référentiel des métiers et compétences des Systèmes d’information et de communication (SIC) }
\end{itemize}

\clearpage
\section{DREAL Auvergne-Rhône-Alpes}


\begin{center}
  \includegraphics[width=3cm]{images/orga/a2_6487055a3946578c74370f02c5003e-100.png}
\end{center}


DREAL Auvergne-Rhône-Alpes

Données publiées via la \href{https://inspire.data.gouv.fr}{passerelle
INSPIRE}


\vspace{0.5cm}

\needspace{12\baselineskip}
\subsection*{Barrage : Commune soumise à un risque de rupture de barrage an
Auvergne-Rhône-Alpes
}\index{donnees!ouvertes}\index{dreal!auvergne!rhone!alpes}\index{gestion!de!crise}\index{grand!public}\index{passerelle!inspire}\index{planning!cadastre}\index{rhone!alpes}\index{risque!technologique}\index{specification!nationale}\index{zones!a!risque!naturel}
  \begin{wrapfigure}{r}{2.5cm}
    \centering
    \qrcode[nolink]{https://data.gouv.fr/dataset/557f1b2888ee383533f4b5c4}
  \end{wrapfigure}

Licence : \textbf{Licence Ouverte version 2.0
}\newline
Créé le : 2015-06-15\newline
Modifié le : 2019-02-08\newline
Popularité : 1 réutilisation,  0 suivi\newline
Mots-clé : \emph{donnees-ouvertes, dreal-auvergne-rhone-alpes, gestion-de-crise, grand-public, passerelle-inspire, planning-cadastre, rhone-alpes, risque-technologique, specification-nationale, zones-a-risque-naturel
}\newline
Permalien : \url{https://data.gouv.fr/dataset/557f1b2888ee383533f4b5c4}\newline

\par
\noindent
    Donnée à la commune extraite de la base GASPAR du ministère de
l'Ecologie. Les contours communaux sont extraits de la BDCARTO de l'IGN

Utilisations potentielles : gestion de crise

\textbf{Origine}

liste des communes extraite de Gaspar et geocodage sur les contours de
communes de la BDCARTO via le code INSEE

\textbf{Organisations partenaires}

Direction Régionale de l'Environnement de l'Aménagement et du Logement
d'Auvergne-Rhône-Alpes (DREAL Auvergne-Rhône-Alpes)

\textbf{Liens annexes}

\begin{itemize}

\item
  \href{https://catalogue.datara.gouv.fr/rss/atomfeed/atomdataset/e30694f9-9471-4a02-afcf-e71618a40edf}{Téléchargement
  direct des données}
\end{itemize}

➞
\href{https://geo.data.gouv.fr/fr/datasets/d515c58cf7b8b9894553281e934b8c34bf4471d0}{Consulter
cette fiche sur geo.data.gouv.fr}


\vspace{0.5cm}
\needspace{3\baselineskip} \rule{4cm}{0.25pt}\newline\textbf{Aussi disponible du même producteur :}\begin{itemize}
\item \href{https://data.gouv.fr/dataset/557f1b0d88ee381cdcf4b5ad}{Aléa sismicité à la commune GEOFLA sur Auvergne-Rhône-Alpes}
\item \href{https://data.gouv.fr/dataset/559c08a8c751df6d14390bda}{Aléas miniers en Auvergne}
\item \href{https://data.gouv.fr/dataset/5882a3a988ee3839309b81a5}{Aléas Miniers en Rhône-Alpes, tout type d'aléa}
\item \href{https://data.gouv.fr/dataset/56f2731988ee386925f07b23}{Ancien inventaire des paysages remarquables de Rhône-Alpes (donnée historique ponctuelle)}
\item \href{https://data.gouv.fr/dataset/56f27319c751df05dee0c74f}{Ancien inventaire des paysages remarquables de Rhône-Alpes (donnée historique surfacique)}
\item \href{https://data.gouv.fr/dataset/557f1b18c751df4d181d4c0b}{Ancien inventaire régional des tourbières (bassins) de Rhône-Alpes (valeur historique)}
\item \href{https://data.gouv.fr/dataset/557f1b1588ee383533f4b5bb}{Ancien inventaire régional des tourbières (sites) de Rhône-Alpes (valeur historique)}
\item \href{https://data.gouv.fr/dataset/557f1b19c751df42501d4c02}{Arrêté Préfectoral de Protection de Biotope de Auvergne-Rhône-Alpes}
\item \href{https://data.gouv.fr/dataset/557f1b1f88ee381cdcf4b5b4}{Barrage : Localisation des Grands Barrages soumis à PPI sur Auvergne-Rhône-Alpes}
\item \href{https://data.gouv.fr/dataset/557f1b39c751df42501d4c07}{Barrages linéaires classés en Rhône-Alpes}
\item \href{https://data.gouv.fr/dataset/557f1b3788ee383533f4b5cc}{Barrages ponctuels classés de Rhône-Alpes}
\item \href{https://data.gouv.fr/dataset/5882a14ac751df3f9cae0a68}{Base de données Occupation du Sol à l'échelle COMmunale (OSCOM) - Conseil Départemental de la Drôme (CD 26)}
\item \href{https://data.gouv.fr/dataset/5882a14b88ee382cd89b81a6}{Base de données Occupation du Sol à l'échelle COMmunale (OSCOM) - Département de Haute-Loire}
\item \href{https://data.gouv.fr/dataset/5882a14a88ee382cd89b81a5}{Base de données Occupation du Sol à l'échelle COMmunale (OSCOM) - Département de Haute-Savoie}
\item \href{https://data.gouv.fr/dataset/5882a14ac751df18c7ae0a84}{Base de données Occupation du Sol à l'échelle COMmunale (OSCOM) - Département de l'Ain}
\item \href{https://data.gouv.fr/dataset/5882a14ac751df187eae0a71}{Base de données Occupation du Sol à l'échelle COMmunale (OSCOM) - Département de l'Allier}
\item \href{https://data.gouv.fr/dataset/5882a14a88ee3835d89b81a5}{Base de données Occupation du Sol à l'échelle COMmunale (OSCOM) - Département de la Loire}
\item \href{https://data.gouv.fr/dataset/5882a14a88ee3835ca9b81a5}{Base de données Occupation du Sol à l'échelle COMmunale (OSCOM) - Département de l'Ardèche}
\item \href{https://data.gouv.fr/dataset/5882a14a88ee3835ca9b81a4}{Base de données Occupation du Sol à l'échelle COMmunale (OSCOM) - Département de l'Isère}
\item \href{https://data.gouv.fr/dataset/5882a14bc751df3f9cae0a69}{Base de données Occupation du Sol à l'échelle COMmunale (OSCOM) - Département de Savoie}
\item \href{https://data.gouv.fr/dataset/5882a14a88ee3835c79b81a6}{Base de données Occupation du Sol à l'échelle COMmunale (OSCOM) - Département du Cantal}
\item \href{https://data.gouv.fr/dataset/5882a14ac751df1891ae0a7c}{Base de données Occupation du Sol à l'échelle COMmunale (OSCOM) - Département du Puy-de-Dôme}
\item \href{https://data.gouv.fr/dataset/5882a14a88ee3835c79b81a5}{Base de données Occupation du Sol à l'échelle COMmunale (OSCOM) - Département du Rhône}
\item \href{https://data.gouv.fr/dataset/557f1b3a88ee381809f4b5bb}{Catégorie picsicole des cours d'eau de Rhône-Alpes}
\item \href{https://data.gouv.fr/dataset/557f1b10c751df366a1d4c04}{Catégorie piscicole des plans d'eau de Rhône-Alpes}
\item \href{https://data.gouv.fr/dataset/5882a3a8c751df1889ae0a7f}{Commune ayant fait l'objet d'une étude d'aléa minier de Rhône-Alpes}
\item \href{https://data.gouv.fr/dataset/5a81a5e8c751df132b16eaf8}{Communes soumises à un PPRM en Auvergne-Rhône-Alpes}
\item \href{https://data.gouv.fr/dataset/557f1b0ec751df366a1d4c03}{Cours d'eau Classement L432-6 de Rhône-Alpes}
\item \href{https://data.gouv.fr/dataset/557f1b20c751df62871d4c09}{Cours d'eau Classement Liste 1 sur Rhône-Méditerranée}
\item \href{https://data.gouv.fr/dataset/557f1b24c751df61371d4c1a}{Cours d'eau Classement Liste 2 sur Rhône-Méditerranée}
\item \href{https://data.gouv.fr/dataset/557f1b0e88ee38061cf4b5a6}{Cours d'eau Classement loi 1919 de Rhône-Alpes}
\item \href{https://data.gouv.fr/dataset/557f1b0cc751df62871d4c02}{Cours d'eau principaux de Auvergne-Rhône-Alpes}
\item \href{https://data.gouv.fr/dataset/559c08b288ee3858d4764f77}{Déchetteries en Auvergne}
\item \href{https://data.gouv.fr/dataset/5882a3a888ee38396d9b81a4}{Désordres Miniers en Rhône-Alpes (ponctuel)}
\item \href{https://data.gouv.fr/dataset/5882a3a8c751df3f9cae0a6b}{Désordres Miniers en Rhône-Alpes (surfacique)}
\item \href{https://data.gouv.fr/dataset/557f1b38c751df62871d4c19}{Digues constituées en système de protection en Rhône-Alpes}
\item \href{https://data.gouv.fr/dataset/59b65ffa88ee3802372cd34c}{Directive cadre Inondation : Territoires à Risque Important de Auvergne-Rhône-Alpes}
\item \href{https://data.gouv.fr/dataset/557f1b5188ee383533f4b5db}{Directive cadre Inondation : Territoires à Risque Important du Bassin Rhône Méditerranée}
\item \href{https://data.gouv.fr/dataset/557f1b18c751df62871d4c04}{Directive cadre Inondation : Zones inondables des 18 TRI de Auvergne-Rhône-Alpes}
\item \href{https://data.gouv.fr/dataset/59b65ff3c751df2468708171}{Directive cadre Inondation : Zones inondables des 18 TRI de Auvergne-Rhône-Alpes - vue sur scenario faible}
\item \href{https://data.gouv.fr/dataset/59b6600488ee38025667eb50}{Directive cadre Inondation : Zones inondables des 18 TRI de Auvergne-Rhône-Alpes - vue sur scenario fort}
\item \href{https://data.gouv.fr/dataset/59b65ff688ee380222f5e7cd}{Directive cadre Inondation : Zones inondables des 18 TRI de Auvergne-Rhône-Alpes - vue sur scenario moyen}
\item \href{https://data.gouv.fr/dataset/557f1b39c751df62871d4c1a}{Domaine Public Fluvial de Rhône-Alpes}
\item \href{https://data.gouv.fr/dataset/559c089ac751df6d14390bd4}{Données produites par le SIG Directive inondation du territoire à risque d’inondation de Clermont-Ferrand - Riom - Iso classe de hauteur relative à une crue centennale}
\item \href{https://data.gouv.fr/dataset/559c08b2c751df6ead390bda}{Données produites par le SIG Directive inondation du territoire à risque d’inondation de Clermont-Ferrand - Riom - Iso classe de hauteur relative à une crue millénale}
\item \href{https://data.gouv.fr/dataset/559c08a3c751df4853390bed}{Données produites par le SIG Directive inondation du territoire à risque d’inondation de Montluçon - Carte de synthèse tout type d'aléas et de risque}
\item \href{https://data.gouv.fr/dataset/559c089ec751df4853390beb}{Données produites par le SIG Directive inondation du territoire à risque d’inondation de Montluçon - Enjeux}
\item \href{https://data.gouv.fr/dataset/559c089e88ee385c1e764f75}{Données produites par le SIG Directive inondation du territoire à risque d’inondation de Montluçon - Iso classe de hauteur relative à une crue centennale}
\item \href{https://data.gouv.fr/dataset/559c089e88ee3858d4764f6f}{Données produites par le SIG Directive inondation du territoire à risque d’inondation de Montluçon - Iso classe de hauteur relative à une crue millénale}
\item \href{https://data.gouv.fr/dataset/559c08a788ee3858d4764f73}{Données produites par le SIG Directive inondation du territoire à risque d’inondation de Montluçon - Iso classe de hauteur relative à une crue trentennale}
\item et 162 autres jeux de données\end{itemize}

\clearpage
\section{DREAL Bretagne}


\begin{center}
  \includegraphics[width=3cm]{images/orga/2015-07-24_b551348e86d94f6e8ef12dc7842481bc_logo_fr_dreal-100.png}
\end{center}


Direction Régionale de l'Environnement, de l'Aménagement et du Logement
de Bretagne


\vspace{0.5cm}

\needspace{12\baselineskip}
\subsection*{Campings du littoral breton
}\index{batiments}\index{bretagne}\index{camping}\index{donnees!ouvertes}\index{hebergement}\index{mer!et!littoral}\index{passerelle!inspire}\index{structure}
  \begin{wrapfigure}{r}{2.5cm}
    \centering
    \qrcode[nolink]{https://data.gouv.fr/dataset/55b235f5c751df525e10ccb8}
  \end{wrapfigure}

Licence : \textbf{Licence Ouverte version 2.0
}\newline
Créé le : 2015-07-24\newline
Modifié le : 2019-02-08\newline
Popularité : 2 réutilisations,  0 suivi\newline
Mots-clé : \emph{batiments, bretagne, camping, donnees-ouvertes, hebergement, mer-et-littoral, passerelle-inspire, structure
}\newline
Permalien : \url{https://data.gouv.fr/dataset/55b235f5c751df525e10ccb8}\newline

\par
\noindent
    Identification et saisie des campings dans la zone littorale des quatre
département bretons (22,29,35,56)

\textbf{Origine}

Identification par photointerprétation sur l'orthophotoplan littoral
2000 de l'IGN des campings sur la zone littorale, saisie manuelle des
polygones indiquant les aires des parkings. Le travail a été réalisé
sous ArcGis 9.1. Les aires saisies ont été jointes spatiallement avec le
thème lieux\_dits\_surface pour affecter un lieu-dit, une commune et un
code Insee. Le résultat a été converti en table MapInfo avec FME de Safe
software. Les points ont été numérisés sur l'orthophotoplan et placés à
l'entrée de l'aire de camping d'après l'interprétation du technicien de
saisie.

\textbf{Organisations partenaires}

DREAL Bretagne, NASCA Géosystèmes NASCA

\textbf{Liens annexes}

\begin{itemize}

\item
  \href{https://geobretagne.fr/sviewer/?title=Camping\%20en\%20Bretagne\&layers=dreal_b:Camping_littoral}{Visualeur
  simple}
\end{itemize}

➞
\href{https://geo.data.gouv.fr/fr/datasets/6365d9c2128ba2df3816ce1784371b9dafb9fdce}{Consulter
cette fiche sur geo.data.gouv.fr}


\vspace{0.5cm}
\needspace{12\baselineskip}
\subsection*{Ecoles de conduite en Bretagne
}\index{auto!ecole}\index{bretagne}\index{donnees!ouvertes}\index{ecole!de!conduite}\index{passerelle!inspire}\index{service!public}\index{society}
  \begin{wrapfigure}{r}{2.5cm}
    \centering
    \qrcode[nolink]{https://data.gouv.fr/dataset/55b235f2c751df55f510ccb8}
  \end{wrapfigure}

Licence : \textbf{Licence Ouverte version 2.0
}\newline
Créé le : 2015-07-24\newline
Modifié le : 2019-02-08\newline
Popularité : 1 réutilisation,  0 suivi\newline
Mots-clé : \emph{auto-ecole, bretagne, donnees-ouvertes, ecole-de-conduite, passerelle-inspire, service-public, society
}\newline
Permalien : \url{https://data.gouv.fr/dataset/55b235f2c751df55f510ccb8}\newline

\par
\noindent
    Localisation des écoles de conduite en Bretagne

\textbf{Origine}

Localisation des écoles de conduite réalisée à partir des données
adresses des centres fournies par les DDTM et agrégées par la DREAL.
Attention la base adresse ne permet pas toujours d'avoir une
localisation précise des écoles.

\textbf{Organisations partenaires}

DREAL Bretagne

➞
\href{https://geo.data.gouv.fr/fr/datasets/d8c77c0f623419ae56dabbd263ab61d2e8d5d44a}{Consulter
cette fiche sur geo.data.gouv.fr}


\vspace{0.5cm}
\needspace{12\baselineskip}
\subsection*{Etablissements scolaires du premier et du second degré - Métropole
}\index{bretagne}\index{donnees!ouvertes}\index{ecoles}\index{etablissement!denseignement}\index{etablissement!public}\index{metropole}\index{passerelle!inspire}\index{primere}\index{secondaire}\index{society}
  \begin{wrapfigure}{r}{2.5cm}
    \centering
    \qrcode[nolink]{https://data.gouv.fr/dataset/55b235f288ee3814da3ca28c}
  \end{wrapfigure}

Licence : \textbf{Licence Ouverte version 2.0
}\newline
Créé le : 2015-07-24\newline
Modifié le : 2019-02-08\newline
Popularité : 1 réutilisation,  1 suivi\newline
Mots-clé : \emph{bretagne, donnees-ouvertes, ecoles, etablissement-denseignement, etablissement-public, metropole, passerelle-inspire, primere, secondaire, society
}\newline
Permalien : \url{https://data.gouv.fr/dataset/55b235f288ee3814da3ca28c}\newline

\par
\noindent
    Ce jeu de données provient d'un service public
certifié.\url{https://www.data.gouv.fr/fr/datasets/adresse-et-geolocalisation-des-etablissements-denseignement-du-premier-et-second-degres/}Etablissements
d'enseignement des premier et second degrés et structures
administratives de l'éducation du ministère de l'éducation nationale, de
l'enseignement supérieur et de la recherche. Liste géolocalisée des
établissements d'enseignement des premier et second degrés, des
structures administratives de l'éducation du ministère de l'éducation
nationale. Secteurs public et privé.

\textbf{Origine}

Liste géolocalisée des établissements d'enseignement des premier et
second degrés, des structures administratives de l'éducation du
ministère de l'éducation nationale. Pour la géolocalisation, les données
avaient été extraites de la base centrale des établissements (BCE) en
octobre 2011 et la géolocalisation a été effectuée par l'IGN en novembre
2011. Les coordonnées X et Y sont fournies par l'IGN selon les
référentiels suivants~(système de référence - projection utilisée)

\textbf{Organisations partenaires}

DREAL Bretagne, Rectorat Bretagne

\textbf{Liens annexes}

\begin{itemize}

\item
  \href{https://data.education.gouv.fr/explore/dataset/fr-en-adresse-et-geolocalisation-etablissements-premier-et-second-degre/download?format=shp\&timezone=Europe/Berlin\&use_labels_for_header=true}{Adresse
  et géolocalisation des établissements d'enseignement du premier et
  second degrés}
\item
  \href{https://data.education.gouv.fr/explore/dataset/fr-en-adresse-et-geolocalisation-etablissements-premier-et-second-degre/export/?disjunctive.nature_uai\&disjunctive.nature_uai_libe\&disjunctive.code_departement\&disjunctive.code_region\&disjunctive.code_academie}{data.education.gouv.fr}
\end{itemize}

➞
\href{https://geo.data.gouv.fr/fr/datasets/783d184140d5eb3f36e44865dafcfa8fa78cdde9}{Consulter
cette fiche sur geo.data.gouv.fr}


\vspace{0.5cm}
\needspace{12\baselineskip}
\subsection*{Zonage ABC en Bretagne
}\index{abc}\index{aides!fiscales}\index{b!1!b!2}\index{besoins!en!logement}\index{boundaries}\index{bretagne}\index{donnees!ouvertes}\index{habitat}\index{habitation}\index{investissement!locatif}\index{investisseurs}\index{logement!locatif}\index{marche!de!lhabitat}\index{passerelle!inspire}\index{plafond!de!loyer}\index{tension}\index{zonage}
  \begin{wrapfigure}{r}{2.5cm}
    \centering
    \qrcode[nolink]{https://data.gouv.fr/dataset/55b235ff88ee3817593ca28f}
  \end{wrapfigure}

Licence : \textbf{Licence Ouverte version 2.0
}\newline
Créé le : 2015-07-24\newline
Modifié le : 2019-02-08\newline
Popularité : 1 réutilisation,  0 suivi\newline
Mots-clé : \emph{abc, aides-fiscales, b-1-b-2, besoins-en-logement, boundaries, bretagne, donnees-ouvertes, habitat, habitation, investissement-locatif, investisseurs, logement-locatif, marche-de-lhabitat, passerelle-inspire, plafond-de-loyer, tension, zonage
}\newline
Permalien : \url{https://data.gouv.fr/dataset/55b235ff88ee3817593ca28f}\newline

\par
\noindent
    (Mise à jour 27/11/2018 : prise en compte des nouveaux périmètres des
communes au 01 janvier 2018.

Le zonage A / B / C est applicable à compter du 1er octobre 2014 pour
certains dispositifs (notamment le dispositif d'investissement locatif
intermédiaire et le prêt à taux zéro). L'objectif est de favoriser
l'investissement locatif, l'accession à la propriété et la construction
de logements.

Un zonage identifiant la tension d'un territoire

Le zonage A / B / C a été créé en 2003 dans le cadre du dispositif
d'investissement locatif dit « Robien ». Il a été révisé depuis en 2006
et 2009. Le critère de classement dans une des zones est la tension du
marché immobilier local. En matière de logement, la tension d'un marché
immobilier local est définie par le niveau d'adéquation sur un
territoire entre la demande de logements et l`offre de logements
disponibles. Une zone est dite « tendue » si l'offre de logements
disponibles n'est pas suffisante pour couvrir la demande (en termes de
volume et de prix). A contrario, une zone est détendue si l'offre de
logements est suffisante pour couvrir les besoins en demande logements.
Le zonage A / B / C caractérise la tension du marché du logement en
découpant le territoire en 5 zones, de la plus tendue (A bis) à la plus
détendue (zone C).

Avant la révision du zonage au 01/10/2014, les différentes zones étaient
composées de la manière suivante :

• Zone A bis : comprend Paris et 29 communes de la petite couronne ; •
Zone A : comprend la partie agglomérée de l'Île-de-France, la Côte
d'Azur et la partie française de l'agglomération genevoise ; • Zone B1 :
comprend les agglomérations de plus de 250.000 habitants, la grande
couronne parisienne, quelques villes chères comme Annecy, Bayonne,
Cluses, Chambéry, Saint-Malo ou La Rochelle, les départements
d'Outre-Mer, la Corse et les autres îles non reliées au continent ; •
Zone B2 : comprend les autres communes de plus de 50 000 habitants et
les franges de zone B1 ; • Zone C : reste du territoire.

Cette révision était très attendue dans le secteur de l'immobilier

Depuis début 2013, le Ministère du Logement et de l'Égalité des
Territoires a engagé une révision de ce zonage pour tenir compte des
évolutions des dynamiques territoriales. L'objectif était de s'adapter
le plus finement possible aux réalités locales du marché immobilier. Une
concertation avec les partenaires locaux impliqués dans la politique du
logement a été menée début 2014 via les Préfets pour aboutir à un projet
de zonage qui a fait l'objet d'un premier arbitrage par le Premier
Ministre fin juin 2014. Cet arbitrage a de nouveau été transmis aux
Préfets pour avis avant de fixer définitivement le classement des
communes. L'arrêté révisant le zonage A / B / C a alors été publié.

Les dispositifs concernés par une entrée en vigueur du zonage A / B / C
au 1er octobre 2014

Le zonage A / B / C permet d'identifier les zones tendues. Il est
utilisé pour moduler les dispositifs financiers d'aide à l'accession à
la propriété et à la location. Plusieurs dispositifs utilisent ce zonage
pour déterminer l'éligibilité des territoires aux aides ou moduler leurs
paramètres :

• Dispositif d'investissement locatif intermédiaire destiné aux
particuliers : Pour ouvrir droit au dispositif, les logements acquis ou
construits doivent se situer en zones A (y compris A bis) et B1. Les
logements situés dans une commune de zone B2 peuvent également être
éligibles au dispositif, sous réserve que celle-ci ait reçu un agrément
du Préfet de région. Les logements situés en zone C ne peuvent pas
bénéficier du dispositif. Les nombreux reclassements de communes dans
une zone plus tendue, prévus par la révision, permettront à une plus
grande partie du territoire, et in fine de ménages locataires, de
bénéficier de ce dispositif. • PTZ (Prêt à taux zéro) : Les conditions
de ressources permettant de bénéficier du PTZ, ainsi que son montant,
dépendent de la zone où se situe l'achat immobilier à financer. Plus la
zone où est situé le logement est tendue, plus les plafonds de
ressources pour en bénéficier sont élevés et plus le montant du prêt (en
pourcentage de la valeur du bien) est élevé. La révision du zonage A / B
/ C s'articule avec le renforcement du PTZ en zones moyennement ou peu
tendues, qui aura pour effet de limiter l'impact du déclassement en
matière d'accession. Logement intermédiaire, dispositif destiné aux
acteurs institutionnels : Le régime de TVA à 10 \% au bénéfice du
logement locatif intermédiaire (accompagné d'une exonération de TFPB)
s'applique intégralement en zones A bis, A, et B1. Un reclassement de B2
à B1 permettra de développer du logement intermédiaire sur le territoire
des communes concernées.

Les dispositifs concernés par une entrée en vigueur du zonage A / B / C
postérieures au 1er octobre 2014

Pour certains dispositifs, l'entrée en vigueur du nouveau zonage est
postérieure au 1er octobre 2014. Ainsi, le nouveau zonage sera effectif
au 1er janvier 2015 concernant le bénéfice des aides de l'Agence
nationale de l'habitat, le « Borloo ancien », le prêt locatif
intermédiaire, la TVA réduite en zone ANRU, les dispositifs liés à la
promotion HLM et l'appréciation des plafonds de ressources pour les
nouveaux logements intermédiaires détenus par les organismes HLM dans le
cadre de leur service d'intérêt économique général. Il sera enfin
applicable pour les agréments de prêt social de location-accession au
1er février 2015.

\textbf{Origine}

Publication du zonage et des communes éligibles par zone par arrêté
ministériel, applicable à partir du 1er octobre 2014. Le référentiel
géographique pour la construction de la donnée est la BdCarto - Limites
de Communes

\textbf{Organisations partenaires}

DREAL Bretagne

➞
\href{https://geo.data.gouv.fr/fr/datasets/480758a52617f65a3c8a3b88c38d4921850fd3a5}{Consulter
cette fiche sur geo.data.gouv.fr}


\vspace{0.5cm}
\needspace{3\baselineskip} \rule{4cm}{0.25pt}\newline\textbf{Aussi disponible du même producteur :}\begin{itemize}
\item \href{https://data.gouv.fr/dataset/55b235f388ee38213a3ca288}{Accidents mortels en Bretagne en 2011}
\item \href{https://data.gouv.fr/dataset/55b235f0c751df5eb910ccba}{Agences d'urbanisme en Bretagne au 01-01-2018}
\item \href{https://data.gouv.fr/dataset/5a09abcdc751df308a7a7380}{Agences pôle emploi en Bretagne}
\item \href{https://data.gouv.fr/dataset/56ec2b04c751df5240cc7149}{Analyse du niveau de connexion entre milieux naturels en Bretagne (schéma régional de cohérence écologique de Bretagne)}
\item \href{https://data.gouv.fr/dataset/5a09abc6c751df308a7a737f}{Appels à candidatures « dynamisme des centres-villes » et « dynamisme des bourgs ruraux}
\item \href{https://data.gouv.fr/dataset/55b23608c751df285210ccc0}{Arrêtés préfectoraux de biotope en Bretagne}
\item \href{https://data.gouv.fr/dataset/55b235fdc751df5eb910ccbd}{Artificialisation des sols en Bretagne par SCOT 2011-2014}
\item \href{https://data.gouv.fr/dataset/5a09abcc88ee383f43ef7e0c}{Atlas du trait de côte des côtes d'accumulation des départements d'Ille-et-Vilaine, des Côtes d'Armor et du Finistère}
\item \href{https://data.gouv.fr/dataset/55b23600c751df6d6d10ccbb}{Avis de l'Ae sur projets soumis à étude d'impact en Bretagne}
\item \href{https://data.gouv.fr/dataset/55b2360488ee38213a3ca28b}{Bassins versants du Grand Projet 5 en Bretagne (2007-2013)}
\item \href{https://data.gouv.fr/dataset/55b23604c751df55f510ccbd}{Cadastre conchylicole de 2007 Bretagne}
\item \href{https://data.gouv.fr/dataset/584f1fc088ee3801efc65bb3}{Campings - Hébergements touristiques classés en métropole}
\item \href{https://data.gouv.fr/dataset/55b235f688ee38115b3ca28d}{Classement des communes du zonage B2 en Bretagne}
\item \href{https://data.gouv.fr/dataset/5883d37688ee3810b49b8203}{Communes France métropole: limites}
\item \href{https://data.gouv.fr/dataset/55b235f688ee3814da3ca28e}{Communes littorales en Bretagne au titre de la loi}
\item \href{https://data.gouv.fr/dataset/584f1fa388ee3801dec65bb3}{Construction neuve 2014 (SITADEL) en Bretagne en date réelle}
\item \href{https://data.gouv.fr/dataset/5a09ac10c751df30a55ef1e5}{Construction neuve 2015 (SITADEL) en Bretagne en date réelle}
\item \href{https://data.gouv.fr/dataset/5a09abd3c751df30a64217b3}{Construction neuve sur 5 ans 2013 - 2017 (SITADEL) en Bretagne en date de prise en compte}
\item \href{https://data.gouv.fr/dataset/5a09abf188ee3842de23cc6b}{Contrats territoriaux sur le bassin Loire-Bretagne}
\item \href{https://data.gouv.fr/dataset/55b235fdc751df55f510ccbb}{Contrats urbains de cohésion sociale (Cucs) en Bretagne}
\item \href{https://data.gouv.fr/dataset/584f1fbac751df3612c0bb7e}{Corridors écologiques régionaux du schéma régional de cohérence écologique de Bretagne}
\item \href{https://data.gouv.fr/dataset/56ec2b0988ee382820e1a629}{Corridors écologiques régionaux du schéma régional de cohérence écologique de Bretagne}
\item \href{https://data.gouv.fr/dataset/584f1fb688ee3801eac65bb3}{Cours d’eau de la trame verte et bleue régionale (schéma régional de cohérence écologique) de Bretagne}
\item \href{https://data.gouv.fr/dataset/56ec2af8c751df53e6cc7143}{Cours d’eau de la trame verte et bleue régionale (schéma régional de cohérence écologique) de Bretagne}
\item \href{https://data.gouv.fr/dataset/55b235e7c751df5eb910ccb7}{Délégataires de compétence sur les aides au logements (ANAH) en Bretagne}
\item \href{https://data.gouv.fr/dataset/5a09abc8c751df309c2a128d}{Digues de protection contre les inondations et les submersions en Bretagne}
\item \href{https://data.gouv.fr/dataset/5a09abd0c751df309bf95a69}{Donnée sur les évènements de la mobilité en Bretagne}
\item \href{https://data.gouv.fr/dataset/55f281ab88ee383ebaa46ec2}{Ecole de formation aux professions de la santé en Bretagne}
\item \href{https://data.gouv.fr/dataset/55f281abc751df532e1f92b1}{Ecole de formation aux professions sociales en Bretagne}
\item \href{https://data.gouv.fr/dataset/56ec2afec751df53efcc7147}{Edition Cas par Cas en Bretagne}
\item \href{https://data.gouv.fr/dataset/584f1f9cc751df3601c0bb7e}{Éléments de fracture et d’obstacles à la circulation des espèces du schéma régional de cohérence écologique de Bretagne}
\item \href{https://data.gouv.fr/dataset/56ec2b06c751df53e0cc7141}{Éléments de fracture et d’obstacles à la circulation des espèces du schéma régional de cohérence écologique de Bretagne}
\item \href{https://data.gouv.fr/dataset/55b2360788ee3815a33ca28f}{Emprises des Plans de Prévention des Risques Inondations et Littoraux en Bretagne}
\item \href{https://data.gouv.fr/dataset/56ec2b0488ee38265ee1a627}{Enquête sur la commercialisation des logements neufs (ECLN)}
\item \href{https://data.gouv.fr/dataset/55b235ef88ee3817593ca28c}{Eoliennes implantations en Bretagne}
\item \href{https://data.gouv.fr/dataset/55b2360c88ee3814da3ca296}{EPCI en Bretagne au 01/01/2019}
\item \href{https://data.gouv.fr/dataset/5a09a87ec751df2416a4d7b5}{Estuaires de Bretagne}
\item \href{https://data.gouv.fr/dataset/584f1fb8c751df360fc0bb7e}{Etat d'avancement des PLUi, PLUiH et PLUiHD en Bretagne}
\item \href{https://data.gouv.fr/dataset/55b235f888ee3817593ca28e}{Etat d'avancement des PLU - POS - CC et RNU en Bretagne}
\item \href{https://data.gouv.fr/dataset/55b23604c751df5eb910ccc1}{Evaluation environnementale Cas par Cas en Bretagne}
\item \href{https://data.gouv.fr/dataset/5a09abef88ee3842de23cc6a}{Festivals en Bretagne}
\item \href{https://data.gouv.fr/dataset/5a09abeac751df308a7a7382}{fichiers fonciers DREAL - Unités foncieres en Bretagne}
\item \href{https://data.gouv.fr/dataset/55b235e788ee38115b3ca288}{Gares maritimes de Bretagne et Loire Atlantique}
\item \href{https://data.gouv.fr/dataset/56ec2af8c751df2813cc7173}{Grands ensembles de perméabilité identifiés dans le cadre du schéma régional de cohérence écologique de Bretagne}
\item \href{https://data.gouv.fr/dataset/55b235fcc751df55f510ccba}{Hôtels du littoral breton en 2007}
\item \href{https://data.gouv.fr/dataset/55b2360d88ee3814da3ca297}{Hydrométrie - supervision du réseau et situation hydrologique en Bretagne}
\item \href{https://data.gouv.fr/dataset/5883d37888ee38062d9b81a8}{ICPE (industries et élevages) soumises à autorisation (A) ou enregistrement (E) en fonctionnement en Bretagne}
\item \href{https://data.gouv.fr/dataset/55b23607c751df5eb910ccc2}{Installations classées au titre de la protection de l'Environnement (ICPE) en Bretagne \_ en cours de dépublication mars 2019}
\item \href{https://data.gouv.fr/dataset/55b23604c751df6d6d10ccbc}{Installations classées soumises à déclaration sous la rubrique n\degree{} 1136 (emploi et stockage d'ammoniac) en Bretagne (situation en 2007)}
\item \href{https://data.gouv.fr/dataset/55b23609c751df6d6d10ccbe}{Installations industrielles soumises à la Directive IPPC en Bretagne - 2007}
\item et 83 autres jeux de données\end{itemize}

\clearpage
\section{Education Nationale}


\begin{center}
  \includegraphics[width=3cm]{images/orga/84_3194d831264f769fa817e58813d413-100.png}
\end{center}


Le site www.education.gouv.fr propose des informations sur le
fonctionnement du système éducatif, tous les niveaux d'enseignement, de
la maternelle à la terminale, les programmes, les diplômes, les
formations, les filières, etc.

Il donne des informations actualisées quotidiennement sur le ministère,
sur les réformes engagées et la politique ministérielle (conférences de
presse, communiqués de presse, rapports, etc.).

On y trouve également des informations pratiques et des services
(bourses, aides, modalités d'inscription, concours) et des
renseignements sur les carrières de l'éducation, des ressources et des
outils pédagogiques. Cela représente plus de 30 000 pages et documents.


\vspace{0.5cm}

\needspace{12\baselineskip}
\subsection*{Diplôme national du brevet par établissement
}
  \begin{wrapfigure}{r}{2.5cm}
    \centering
    \qrcode[nolink]{https://data.gouv.fr/dataset/5459fc50c751df5aa09b6045}
  \end{wrapfigure}

Licence : \textbf{Licence Ouverte
}\newline
Créé le : 2014-11-05\newline
Modifié le : 2018-12-18\newline
De 2012-09-01 à 2014-08-31\newline
Mise à jour : annuelle\newline
Popularité : 1 réutilisation,  4 suivis\newline
Mots-clé : \emph{aucun
}\newline
Permalien : \url{https://data.gouv.fr/dataset/5459fc50c751df5aa09b6045}\newline

\par
\noindent
    Données statistiques : résultats au diplôme national du brevet par
établissement


\vspace{0.5cm}
\needspace{12\baselineskip}
\subsection*{Indicateurs de résultat des lycées d'enseignement général et
technologique
}\index{enseignement!general}\index{enseignement!technologique}\index{indicateur!de!valeur!ajoutee}\index{lycee}
  \begin{wrapfigure}{r}{2.5cm}
    \centering
    \qrcode[nolink]{https://data.gouv.fr/dataset/53699691a3a729239d204a87}
  \end{wrapfigure}

Licence : \textbf{Licence Ouverte
}\newline
Créé le : 2013-08-16\newline
Modifié le : 2017-03-22\newline
De 2011-09-01 à 2014-08-31\newline
Granularité : au point d'intérêt\newline
Mise à jour : annuelle\newline
Popularité : 10 réutilisations,  14 suivis\newline
Mots-clé : \emph{enseignement-general, enseignement-technologique, indicateur-de-valeur-ajoutee, lycee
}\newline
Permalien : \url{https://data.gouv.fr/dataset/53699691a3a729239d204a87}\newline

\par
\noindent
    données statistiques : les indicateurs de valeur ajoutée des lycées
permettent d'évaluer l'action propre de chaque lycée. Ils sont établis à
partir des résultats des élèves au baccalauréat et de leur parcours
scolaire dans l'établissement. Les lycées d'enseignement général et
technologique, publics et privés sous contrat, sont concernés. , Il ne
s'agit aucunement d'un classement mais d'un regard croisé sur les trois
indicateurs et les « valeurs ajoutées » correspondantes.


\vspace{0.5cm}
\needspace{12\baselineskip}
\subsection*{Indicateurs de résultat des lycées d'enseignement professionnel
}\index{enseignement!professionnel}\index{indicateur!de}\index{indicateur!de!valeur!ajoute}\index{lycee}
  \begin{wrapfigure}{r}{2.5cm}
    \centering
    \qrcode[nolink]{https://data.gouv.fr/dataset/53699692a3a729239d204a8c}
  \end{wrapfigure}

Licence : \textbf{Licence Ouverte
}\newline
Créé le : 2013-08-16\newline
Modifié le : 2017-03-22\newline
De 2011-09-01 à 2014-08-31\newline
Granularité : au point d'intérêt\newline
Mise à jour : annuelle\newline
Popularité : 3 réutilisations,  8 suivis\newline
Mots-clé : \emph{enseignement-professionnel, indicateur-de, indicateur-de-valeur-ajoute, lycee
}\newline
Permalien : \url{https://data.gouv.fr/dataset/53699692a3a729239d204a8c}\newline

\par
\noindent
    données statistiques : les indicateurs de valeur ajoutée des lycées
permettent d'évaluer l'action propre de chaque lycée. Ils sont établis à
partir des résultats des élèves au baccalauréat et de leur parcours
scolaire dans l'établissement. Les lycées professionnels, publics et
privés sous contrat, sont concernés. , Il ne s'agit aucunement d'un
classement mais d'un regard croisé sur les trois indicateurs et les «
valeurs ajoutées » correspondantes.


\vspace{0.5cm}
\needspace{12\baselineskip}
\subsection*{La formation continue dans l'enseignement supérieur
}
  \begin{wrapfigure}{r}{2.5cm}
    \centering
    \qrcode[nolink]{https://data.gouv.fr/dataset/5369979fa3a729239d204d3c}
  \end{wrapfigure}

Licence : \textbf{Licence Ouverte
}\newline
Créé le : 2013-08-16\newline
Modifié le : 2016-01-20\newline
De 2008-01-01 à 2012-12-31\newline
Mise à jour : annuelle\newline
Popularité : 1 réutilisation,  0 suivi\newline
Mots-clé : \emph{aucun
}\newline
Permalien : \url{https://data.gouv.fr/dataset/5369979fa3a729239d204d3c}\newline

\par
\noindent
    Tableaux statistiques : répartition du nombre de stagiaires et
heures-stagiaires par type de dispositif; les ressources de la formation
continue dans l'enseignement supérieur.


\vspace{0.5cm}
\needspace{12\baselineskip}
\subsection*{La réussite au baccalauréat selon la série
}
  \begin{wrapfigure}{r}{2.5cm}
    \centering
    \qrcode[nolink]{https://data.gouv.fr/dataset/536997bfa3a729239d204d8e}
  \end{wrapfigure}

Licence : \textbf{Licence Ouverte
}\newline
Créé le : 2013-08-27\newline
Modifié le : 2016-01-28\newline
De 1995-01-01 à 2013-08-31\newline
Popularité : 1 réutilisation,  1 suivi\newline
Mots-clé : \emph{aucun
}\newline
Permalien : \url{https://data.gouv.fr/dataset/536997bfa3a729239d204d8e}\newline

\par
\noindent
    Tableaux statistiques : évolution de la réussite au baccalauréat depuis
1995 ; évolution des taux de réussite au baccalauréat selon la filière
depuis 1995.


\vspace{0.5cm}
\needspace{12\baselineskip}
\subsection*{Le diplôme national du brevet
}
  \begin{wrapfigure}{r}{2.5cm}
    \centering
    \qrcode[nolink]{https://data.gouv.fr/dataset/536997d2a3a729239d204dbd}
  \end{wrapfigure}

Licence : \textbf{Licence Ouverte
}\newline
Créé le : 2013-08-16\newline
Modifié le : 2016-03-12\newline
De 2009-09-01 à 2014-08-31\newline
Mise à jour : annuelle\newline
Popularité : 3 réutilisations,  6 suivis\newline
Mots-clé : \emph{aucun
}\newline
Permalien : \url{https://data.gouv.fr/dataset/536997d2a3a729239d204dbd}\newline

\par
\noindent
    Tableaux statistiques : taux de réussite au diplôme national du brevet
selon le sexe et la série et résultats au diplôme national du brevet par
académie.


\vspace{0.5cm}
\needspace{12\baselineskip}
\subsection*{Le point sur les 60 000 créations de postes dans l'éducation entre 2012
et 2015
}\index{education}\index{enseignants}\index{enseignement!general}\index{enseignement!superieur}\index{recrutement}
  \begin{wrapfigure}{r}{2.5cm}
    \centering
    \qrcode[nolink]{https://data.gouv.fr/dataset/55e560be88ee3863e2a46ec1}
  \end{wrapfigure}

Licence : \textbf{Licence Ouverte
}\newline
Créé le : 2015-09-01\newline
Modifié le : 2016-01-27\newline
De 2012-09-01 à 2015-09-01\newline
Popularité : 1 réutilisation,  1 suivi\newline
Mots-clé : \emph{education, enseignants, enseignement-general, enseignement-superieur, recrutement
}\newline
Permalien : \url{https://data.gouv.fr/dataset/55e560be88ee3863e2a46ec1}\newline

\par
\noindent
    État des lieux des créations de postes dans l'éducation de 2012 à 2015.
Note aux rédactions diffusée et publiée le
26/08/2015\url{http://www.education.gouv.fr/cid92139/le-point-sur-les-60-000-creations-de-postes-dans-l-education.html}En
savoir plus : Les créations d'emplois depuis 2012 s'inscrivent dans les
objectifs fixés par la loi d'orientation et de programmation sur la
refondation de l'École de la République du 8 juillet 2013. - Consulter
la fiche ``60 000 postes dans l'éducation : le gouvernement tient ses
engagements !'' - extrait du dossier de présentation de l'année scolaire
2015-2016\url{http://cache.media.education.gouv.fr/file/DP_rentree/26/3/2015_rentreescolaire_fiche_01_456263.pdf}-
Consulter le dossier de présentation de l'année scolaire
2015-2016\url{http://www.education.gouv.fr/cid92069/annee-scolaire-2015-2016.html}


\vspace{0.5cm}
\needspace{12\baselineskip}
\subsection*{Les classes préparatoires aux grandes écoles (CPGE)
}
  \begin{wrapfigure}{r}{2.5cm}
    \centering
    \qrcode[nolink]{https://data.gouv.fr/dataset/536997fca3a729239d204e2f}
  \end{wrapfigure}

Licence : \textbf{Licence Ouverte
}\newline
Créé le : 2013-08-27\newline
Modifié le : 2016-03-16\newline
De 2001-09-01 à 2014-08-31\newline
Granularité : au pays\newline
Popularité : 1 réutilisation,  1 suivi\newline
Mots-clé : \emph{aucun
}\newline
Permalien : \url{https://data.gouv.fr/dataset/536997fca3a729239d204e2f}\newline

\par
\noindent
    Tableaux statistiques : évolution des effectifs d'étudiants en classe
préparatoire aux grandes écoles (CPGE) ;effectifs d'étudiants en CPGE
selon l'année et le sexe ; origine scolaire des étudiants entrant en
première année de CPGE.


\vspace{0.5cm}
\needspace{12\baselineskip}
\subsection*{Les diplômes du BTS : présentation générale
}
  \begin{wrapfigure}{r}{2.5cm}
    \centering
    \qrcode[nolink]{https://data.gouv.fr/dataset/53699818a3a729239d204e75}
  \end{wrapfigure}

Licence : \textbf{Licence Ouverte
}\newline
Créé le : 2013-08-16\newline
Modifié le : 2015-11-01\newline
De 2008-09-01 à 2013-08-31\newline
Mise à jour : annuelle\newline
Popularité : 1 réutilisation,  0 suivi\newline
Mots-clé : \emph{aucun
}\newline
Permalien : \url{https://data.gouv.fr/dataset/53699818a3a729239d204e75}\newline

\par
\noindent
    Tableaux statistiques : les candidats au BTS selon la spécialité ;
réussite au BTS selon le mode de formation ; réussite au BTS selon le
diplôme initial ; évolution du nombre de BTS délivrés


\vspace{0.5cm}
\needspace{12\baselineskip}
\subsection*{Le surpoids et obésité en classe de troisième - actualisation 2012
}
  \begin{wrapfigure}{r}{2.5cm}
    \centering
    \qrcode[nolink]{https://data.gouv.fr/dataset/53699889a3a729239d204fb2}
  \end{wrapfigure}

Licence : \textbf{Licence Ouverte
}\newline
Créé le : 2013-07-08\newline
Modifié le : 2016-02-23\newline
De 2003-09-01 à 2004-08-31\newline
Mise à jour : annuelle\newline
Popularité : 1 réutilisation,  0 suivi\newline
Mots-clé : \emph{aucun
}\newline
Permalien : \url{https://data.gouv.fr/dataset/53699889a3a729239d204fb2}\newline

\par
\noindent
    Tableaux statistiques : prévalence du surpoids et de l'obésité des
adolescents en classe de troisième selon la catégorie
socioprofessionnelle du père en 2003-2004 ; surpoids et obésité chez les
adolescents en classe de troisième en ZEP et hors ZEP en 2003-2004 ;
prévalence du surpoids et de l'obésité des adolescents en classe de
troisième selon la ZEAT.


\vspace{0.5cm}
\needspace{12\baselineskip}
\subsection*{Répartition des 6 639 créations de postes d'enseignants pour la rentrée
scolaire 2016
}\index{education}\index{enseignants}\index{rentree!scolaire!2016}\index{repartition!de!postes}
  \begin{wrapfigure}{r}{2.5cm}
    \centering
    \qrcode[nolink]{https://data.gouv.fr/dataset/5693c9c4c751df20e6c664bd}
  \end{wrapfigure}

Licence : \textbf{Licence Ouverte
}\newline
Créé le : 2016-01-11\newline
Modifié le : 2016-03-16\newline
Mise à jour : ponctuelle\newline
Popularité : 1 réutilisation,  1 suivi\newline
Mots-clé : \emph{education, enseignants, rentree-scolaire-2016, repartition-de-postes
}\newline
Permalien : \url{https://data.gouv.fr/dataset/5693c9c4c751df20e6c664bd}\newline

\par
\noindent
    \textbf{Communiqué de presse diffusé le 09 décembre 2015}

Najat Vallaud-Belkacem, ministre de l'Éducation nationale, de
l'Enseignement supérieur et de la Recherche, présente la répartition
académique des 6 639 créations de postes d'enseignants pour la rentrée
2016, qui va donner aux académies des marges de manœuvre sans précédent
au service de la réussite de tous les élèves.

www.education.gouv.fr/cid96488/repartition-des-6-639-creations-de-postes-d-enseignants-pour-la-rentree-scolaire-2016.html


\vspace{0.5cm}
\needspace{12\baselineskip}
\subsection*{Réseau d'éducation prioritaire des collèges
}\index{education}\index{education!prioritaire}\index{reseau!education!prioritaire}\index{reussite!scolaire}
  \begin{wrapfigure}{r}{2.5cm}
    \centering
    \qrcode[nolink]{https://data.gouv.fr/dataset/5492067cc751df406e04805a}
  \end{wrapfigure}

Licence : \textbf{Licence Ouverte
}\newline
Créé le : 2014-12-17\newline
Modifié le : 2016-03-07\newline
De 2014-12-16 à 2018-12-17\newline
Granularité : au point d'intérêt\newline
Mise à jour : annuelle\newline
Popularité : 2 réutilisations,  0 suivi\newline
Mots-clé : \emph{education, education-prioritaire, reseau-education-prioritaire, reussite-scolaire
}\newline
Permalien : \url{https://data.gouv.fr/dataset/5492067cc751df406e04805a}\newline

\par
\noindent
    La politique d'éducation prioritaire a pour objectif de corriger
l'impact des inégalités sociales et économiques sur la réussite scolaire
par un renforcement de l'action pédagogique et éducative dans les écoles
et établissements des territoires qui rencontrent les plus grandes
difficultés sociales. La loi d'orientation et de programmation pour la
refondation de l'École de la République en a défini l'objectif : ramener
à moins de 10\% les écarts de réussite scolaire entre les élèves de
l'éducation prioritaire et les autres élèves de France. La refondation
de la politique d'éducation prioritaire est engagée. Le site national
rénové de l'éducation prioritaire accompagne cette refondation.

Ce fichier présente les réseaux d'éducation prioritaires des collèges.


\vspace{0.5cm}
\needspace{12\baselineskip}
\subsection*{Séries chronologiques Education
}
  \begin{wrapfigure}{r}{2.5cm}
    \centering
    \qrcode[nolink]{https://data.gouv.fr/dataset/54aa91f8c751df41a204805e}
  \end{wrapfigure}

Licence : \textbf{Licence Ouverte
}\newline
Créé le : 2015-01-05\newline
Modifié le : 2018-06-12\newline
De 1996-09-01 à 2015-08-31\newline
Mise à jour : annuelle\newline
Popularité : 2 réutilisations,  0 suivi\newline
Mots-clé : \emph{aucun
}\newline
Permalien : \url{https://data.gouv.fr/dataset/54aa91f8c751df41a204805e}\newline

\par
\noindent
    Calculées sur des périodes historiques longues et actualisées avec les
données 2013, les séries chronologiques proposent des indicateurs
relatifs à l'enseignement scolaire, l'enseignement supérieur et
l'apprentissage. Elles permettent de suivre les évolutions dans le temps
du système éducatif.


\vspace{0.5cm}
\needspace{3\baselineskip} \rule{4cm}{0.25pt}\newline\textbf{Aussi disponible du même producteur :}\begin{itemize}
\item \href{https://data.gouv.fr/dataset/5889d040a3a72974cbf0d5b3}{Académies Education Prioritaire}
\item \href{https://data.gouv.fr/dataset/5889d042a3a72974cbf0d5b8}{Adresse et géolocalisation des établissements d'enseignement du premier et second degrés}
\item \href{https://data.gouv.fr/dataset/5c0a0c4b9ce2e7190ebea585}{Agenda du ministre de l’Éducation Nationale et de la Jeunesse}
\item \href{https://data.gouv.fr/dataset/5c89f9c106e3e7132a4d34c9}{Agenda du secrétaire d'État auprès du ministre de l'Éducation nationale et de la Jeunesse}
\item \href{https://data.gouv.fr/dataset/5889d03fa3a72974cbf0d5b1}{Annuaire de l'éducation}
\item \href{https://data.gouv.fr/dataset/5c5e80bc06e3e7122e232bba}{APB Voeux de poursuite d'étude et admissions}
\item \href{https://data.gouv.fr/dataset/536990cca3a729239d203ac7}{Collèges et lycées : cycle d'enseignement et classes}
\item \href{https://data.gouv.fr/dataset/536990cda3a729239d203ac8}{Collèges et lycées par académie}
\item \href{https://data.gouv.fr/dataset/536990cda3a729239d203ac9}{Collèges et lycées selon le cycle d'enseignement}
\item \href{https://data.gouv.fr/dataset/536990cea3a729239d203aca}{Collèges et lycées selon le type d'établissement}
\item \href{https://data.gouv.fr/dataset/536990cea3a729239d203acb}{Collèges et lycées : types d'établissement et classes - actualisation 2011}
\item \href{https://data.gouv.fr/dataset/536992e2a3a729239d20403c}{Devenir des élèves  après leur entrée en cours préparatoire}
\item \href{https://data.gouv.fr/dataset/5c3ead5e06e3e74fe24cc62c}{Diplôme national du brevet par établissement}
\item \href{https://data.gouv.fr/dataset/5a5d7e4cb595082381ecba32}{Diplômes préparés dans les établissements du secondaire et formations intermédiaires}
\item \href{https://data.gouv.fr/dataset/5889d043a3a72974cbf0d5ba}{Écoles éducation prioritaire}
\item \href{https://data.gouv.fr/dataset/5889d042a3a72974c1f0d608}{Écoles et collèges numériques}
\item \href{https://data.gouv.fr/dataset/5889d03fa3a72974c1f0d600}{Effectifs d'élèves des écoles du premier degré public et privé sous tutelle du ministère en charge de l'éducation nationale}
\item \href{https://data.gouv.fr/dataset/5889d041a3a72974c1f0d604}{Effectifs d'élèves des établissements du second degré public et privé sous tutelle du ministère en charge de l'éducation nationale}
\item \href{https://data.gouv.fr/dataset/561cb68788ee381202628ef9}{Effectifs des écoles du premier degré public et privé sous tutelle du ministère en charge de l'éducation nationale}
\item \href{https://data.gouv.fr/dataset/561cb77788ee3818f3628ef9}{Effectifs des établissements du second degré public et privé sous tutelle du ministère en charge de l'éducation nationale}
\item \href{https://data.gouv.fr/dataset/5889d040a3a72974cbf0d5b4}{Effectifs des personnels des écoles du 1er degré et des établissements du 2nd degré}
\item \href{https://data.gouv.fr/dataset/5889d044a3a72974cbf0d5bd}{Effectifs d’étudiants en CPGE par année et par sexe}
\item \href{https://data.gouv.fr/dataset/5c783e4106e3e72998fe3786}{Enquête ETIC 1er degré}
\item \href{https://data.gouv.fr/dataset/5bf78cd506e3e74fba1a69b7}{Enquête ETIC 1er degré}
\item \href{https://data.gouv.fr/dataset/5bf78cd506e3e74fba1a69b6}{Enquête ETIC 2nd degré}
\item \href{https://data.gouv.fr/dataset/5c783e419ce2e733621da1ae}{Enquête ETIC 2nd degré}
\item \href{https://data.gouv.fr/dataset/53699448a3a729239d2043e7}{"Enseignement commun Enseignement facultatif Enseignement d’exploration de la classe de seconde Programme d'enseignement D’ÉDUCATION PHYSIQUE ET SPORTIVE POUR LES LYCEES D’ENSEIGNEMENT GENERAL ET TECHNOLOGIQUE}
\item \href{https://data.gouv.fr/dataset/53699448a3a729239d2043e8}{Enseignement d’exploration Programme d'enseignement DE BIOTECHNOLOGIES EN CLASSE DE SECONDE GÉNÉRALE ET TECHNOLOGIQUE}
\item \href{https://data.gouv.fr/dataset/53699449a3a729239d2043e9}{Enseignement d'exploration, PROGRAMME D'ENSEIGNEMENT DE CRÉATION ET ACTIVITÉS ARTISTIQUES EN CLASSE DE SECONDE GÉNÉRALE ET TECHNOLOGIQUE, Arts visuels, Arts du son, Arts du spectacle, Patrimoines}
\item \href{https://data.gouv.fr/dataset/53699449a3a729239d2043ea}{Enseignement d'exploration, PROGRAMME D'ENSEIGNEMENT DE CRÉATION ET CULTURE DESIGN EN CLASSE DE SECONDE GÉNÉRALE ET TECHNOLOGIQUE}
\item \href{https://data.gouv.fr/dataset/5369944aa3a729239d2043eb}{Enseignement d’exploration Programme d'enseignement DE CRÉATION ET INNOVATION TECHNOLOGIQUES EN CLASSE DE SECONDE GÉNÉRALE ET TECHNOLOGIQUE}
\item \href{https://data.gouv.fr/dataset/5369944aa3a729239d2043ec}{Enseignement d'exploration, PROGRAMME D'ENSEIGNEMENT DE LITTÉRATURE ET SOCIÉTÉ EN CLASSE DE SECONDE GÉNÉRALE ET TECHNOLOGIQUE}
\item \href{https://data.gouv.fr/dataset/5369944aa3a729239d2043ed}{Enseignement d’exploration Programme d'enseignement DE MÉTHODES ET PRATIQUES SCIENTIFIQUES EN CLASSE DE SECONDE GÉNÉRALE ET TECHNOLOGIQUE}
\item \href{https://data.gouv.fr/dataset/5369944ba3a729239d2043ee}{Enseignement d’exploration Programme d'enseignement DE SANTE ET SOCIAL EN CLASSE DE SECONDE GÉNÉRALE ET TECHNOLOGIQUE}
\item \href{https://data.gouv.fr/dataset/5369944ba3a729239d2043ef}{Enseignement d'exploration, PROGRAMME D'ENSEIGNEMENT DE SCIENCES DE L’INGENIEUR EN CLASSE DE SECONDE GÉNÉRALE ET TECHNOLOGIQUE}
\item \href{https://data.gouv.fr/dataset/5369944ba3a729239d2043f0}{Enseignement d’exploration Programme d'enseignement DE SCIENCES ÉCONOMIQUES ET SOCIALES EN CLASSE DE SECONDE GÉNÉRALE ET TECHNOLOGIQUE}
\item \href{https://data.gouv.fr/dataset/5369944ca3a729239d2043f1}{Enseignement d’exploration Programme d'enseignement DE SCIENCES ET LABORATOIRE EN CLASSE DE SECONDE GÉNÉRALE ET TECHNOLOGIQUE}
\item \href{https://data.gouv.fr/dataset/5369944ca3a729239d2043f2}{Enseignement d’exploration Programme d'enseignement DES PRINCIPES FONDAMENTAUX DE L’ÉCONOMIE ET DE LA GESTION EN CLASSE DE SECONDE GÉNÉRALE ET TECHNOLOGIQUE}
\item \href{https://data.gouv.fr/dataset/5369944da3a729239d2043f3}{Enseignement facultatif, PROGRAMME D'ENSEIGNEMENT D’ARTS EN CLASSE DE SECONDE GÉNÉRALE ET TECHNOLOGIQUE}
\item \href{https://data.gouv.fr/dataset/5369944ea3a729239d2043f5}{Enseignements communs, FRANÇAIS, CLASSES DE SECONDE GÉNÉRALE ET TECHNOLOGIQUE ET PREMIÈRE GÉNÉRALE, LITTÉRATURE, CLASSE DE PREMIÈRE SÉRIE L}
\item \href{https://data.gouv.fr/dataset/5369944ea3a729239d2043f6}{Enseignements communs, PROGRAMME D'ENSEIGNEMENT D'ÉDUCATION CIVIQUE, JURIDIQUE ET SOCIALE EN}
\item \href{https://data.gouv.fr/dataset/5369944fa3a729239d2043f7}{Enseignements communs, Programme d'enseignement de mathématiques pour la classe de seconde - Année scolaire 2009-2010}
\item \href{https://data.gouv.fr/dataset/53699450a3a729239d2043f8}{Enseignements communs, programme d'enseignement d'histoire-géographie des classes de première de la voie générale}
\item \href{https://data.gouv.fr/dataset/53699450a3a729239d2043f9}{Enseignements obligatoire et de spécialité en série L - Enseignement facultatif toutes séries - Programme d'enseignement artistique du cycle terminal - Préambule général}
\item \href{https://data.gouv.fr/dataset/58357002c751df1c96c0bb7e}{ETIC 2016}
\item \href{https://data.gouv.fr/dataset/5889d041a3a72974cbf0d5b5}{Étude TIMMS : heures annuelles d'enseignement}
\item \href{https://data.gouv.fr/dataset/5889d040a3a72974c1f0d603}{Étude TIMMS : proportion d'élèves par niveau}
\item \href{https://data.gouv.fr/dataset/56bb084d88ee3803344b6d5b}{EVALuENT 1D 2015							}
\item \href{https://data.gouv.fr/dataset/568f992d88ee380e01af0bf5}{EVALuENT 2D 2012}
\item \href{https://data.gouv.fr/dataset/568f9bba88ee387cf9af0bf4}{EVALuENT 2D 2014}
\item et 308 autres jeux de données\end{itemize}

\clearpage
\section{Établissement public du château, du musée et du domaine national de Versailles}


\begin{center}
  \includegraphics[width=3cm]{images/orga/29_111900211d429fa4a09f69808b5c55-100.png}
\end{center}


Le château, le musée et le domaine national de Versailles constituent
depuis 1995 un établissement public à caractère administratif, doté
d'une autonomie de gestion administrative et financière. Il est placé
sous la tutelle du ministère de la culture et de la communication et du
ministère en charge du budget.


\vspace{0.5cm}

\needspace{12\baselineskip}
\subsection*{Agenda des événements du Château de Versailles
}\index{culture}
  \begin{wrapfigure}{r}{2.5cm}
    \centering
    \qrcode[nolink]{https://data.gouv.fr/dataset/53698e89a3a729239d2034c0}
  \end{wrapfigure}

Licence : \textbf{Licence Ouverte
}\newline
Créé le : 2013-10-20\newline
Modifié le : 2016-08-04\newline
Mise à jour : annuelle\newline
Popularité : 1 réutilisation,  1 suivi\newline
Mots-clé : \emph{culture
}\newline
Permalien : \url{https://data.gouv.fr/dataset/53698e89a3a729239d2034c0}\newline

\par
\noindent
    Liste des événements (expositions, spectacles, visites commentées, vie
scientifique\ldots{}) du Château de Versailles d'octobre 2013 à mars
2014.


\vspace{0.5cm}

\clearpage
\section{Etalab}


\begin{center}
  \includegraphics[width=3cm]{images/orga/2015-01-12_8ebc26cca2394e089a8b065fe0ae2659_etalab-100.jpg}
\end{center}


La politique d'ouverture et de partage des données publiques (``Open
Data'') est pilotée, sous l'autorité du Premier ministre, par la mission
Etalab, dirigée par Mme Laure Lucchesi
(\href{https://www.etalab.gouv.fr/qui-sommes-nous}{voir détails sur le
site d'Etalab})

Les missions et le rôle d'Etalab pour la transformation numérique de
l'action publique

\textbf{La mission Etalab coordonne la politique d'ouverture et de
partage des données publiques (« open data ») :}

\begin{itemize}

\item
  Etalab coordonne les actions des administrations de l'Etat et leur
  apporte son appui pour faciliter la diffusion et la réutilisation de
  leurs informations publiques. Elle contribue à leur conception et
  coordonne leur mise en œuvre interministérielle
\item
  Elle développe et anime la plateforme d'open data www.data.gouv.fr
  destinée à rassembler et à mettre à disposition librement l'ensemble
  des informations publiques de l'Etat, de ses établissements publics
  et, si elles le souhaitent, des collectivités territoriales et des
  personnes de droit public ou de droit privé chargées d'une mission de
  service public.
\item
  Elle contribue, avec les administrations de l'Etat, à l'ouverture des
  données publiques et à la promotion des sciences des données.
\end{itemize}

\textbf{Etalab contribue aux \href{http://agd.data.gouv.fr/}{missions de
l'Administrateur général des données (AGD)}, fixées par le décret
n\degree{} 2014-1050 du 16 septembre 2014}

Appuyé par l'équipe d'Etalab, l'AGD coordonne l'action des
administrations en matière d'inventaire, de gouvernance, de production,
de circulation et d'exploitation des données par les administrations. Il
organise, dans le respect de la protection des données personnelles et
des secrets protégés par la loi, la meilleure exploitation de ces
données et leur plus large circulation, notamment aux fins d'évaluation
des politiques publiques, d'amélioration et de transparence de l'action
publique et de stimulation de la recherche et de l'innovation.

\textbf{La mission contribue également à la mise en oeuvre des
\href{https://www.etalab.gouv.fr/gouvernement-ouvert}{principes de «
gouvernement ouvert »}: transparence de l'action publique, consultation
et concertation avec la société civile, participation citoyenne,
innovation ouverte\ldots{}}

Etalab coordonne notamment, en lien avec le Ministère des Affaires
étrangères et du Développement international, l'action de la France au
sein du Partenariat pour un gouvernement ouvert (« PGO »). Elle anime
l'élaboration et le suivi du Plan d'action national pour une action
publique transparente et collaborative développé dans le cadre du PGO,
et accompagne les administrations dans leur ouverture.

Etalab contribue également au développement de la
\href{http://openfisca.org/fr/}{plateforme OpenFisca,}moteur ouvert de
simulation du système socio-fiscal français.

Etalab anime le
\href{https://www.etalab.gouv.fr/entrepreneurs-dinteret-general}{programme
Entrepreneurs d'Intérêt Général.}

\begin{center}\rule{0.5\linewidth}{\linethickness}\end{center}


\vspace{0.5cm}

\needspace{12\baselineskip}
\subsection*{Cartographie des bases de données publiques en santé
}\index{sante}
  \begin{wrapfigure}{r}{2.5cm}
    \centering
    \qrcode[nolink]{https://data.gouv.fr/dataset/53699037a3a729239d203949}
  \end{wrapfigure}

Licence : \textbf{Licence Ouverte
}\newline
Créé le : 2014-03-27\newline
Modifié le : 2016-03-16\newline
De 2014-01-01 à 2014-12-31\newline
Granularité : au pays\newline
Mise à jour : ponctuelle\newline
Popularité : 1 réutilisation,  10 suivis\newline
Mots-clé : \emph{sante
}\newline
Permalien : \url{https://data.gouv.fr/dataset/53699037a3a729239d203949}\newline

\par
\noindent
    \textbf{Date de la dernière mise à jour : 6 mai 2014}

Dans le cadre du \textbf{débat thématique sur l'ouverture des données
publiques de santé }lancé par le Ministère des Affaires Sociales et de
la Santé en novembre 2013, la mission Etalab a réalisé un travail de
recensement le plus complet possible des bases et jeux de données
publiques existants dans le domaine de la santé, et \textbf{publie
aujourd'hui cette cartographie}. Plus de 260 bases ou jeux de données
ont été recensés.

Chaque base de données identifiée a fait l'objet d'une évaluation de son
``niveau d'ouverture'' actuel, au regard de 4 critères: la liberté
d'accès (qui a accès aux données ?), le coût d'accès (les données
sont-elles disponibles gratuitement ?), le format de mise à disposition
(les données sont-elles proposées dans des formats facilitant la
réutilisation ?), les conditions juridiques de la réutilisation (la
réutilisation des données est-elle explicitement autorisée ?)

\textbf{Deux niveaux de granularité }sont également identifiés pour
chaque base ou jeux de données : le niveau granulaire (données au niveau
le plus fin qu'il est possible d'obtenir en fonction de l'origine de la
donnée et du système de collecte), et le niveau agrégé (données obtenues
en regroupant des données granulaires selon une ou plusieurs
caractéristiques communes).

Nous vous encourageons fortement à consulter le \textbf{guide de lecture
de la cartographie}, publié ci-après en tant que ressource associée au
jeu de données. Ce document présente notamment les deux niveaux de
typologie utilisés.

Si vous constatez une erreur ou une omission dans le fichier, nous vous
remercions de nous le signaler via l'icône rouge ci-dessous.


\vspace{0.5cm}
\needspace{12\baselineskip}
\subsection*{Données essentielles de la commande publique transmises via le PES
Marché
}
  \begin{wrapfigure}{r}{2.5cm}
    \centering
    \qrcode[nolink]{https://data.gouv.fr/dataset/5bd0b6fd8b4c413d0801dc57}
  \end{wrapfigure}

Licence : \textbf{Licence Ouverte
}\newline
Créé le : 2018-10-24\newline
Modifié le : 2018-11-07\newline
De 2018-10-01 à 2030-10-31\newline
Granularité : au pays\newline
Popularité : 1 réutilisation,  1 suivi\newline
Mots-clé : \emph{aucun
}\newline
Permalien : \url{https://data.gouv.fr/dataset/5bd0b6fd8b4c413d0801dc57}\newline

\par
\noindent
    Dans le cadre de la transparence des marchés publics, la Direction
Générale des Finances Publiques (DGFiP) du Ministère des Finances a
proposé à tous les acheteurs soumis à la comptabilité publique de faire
remonter leurs données essentielles de la commande publique via le
\href{https://www.collectivites-locales.gouv.fr/protocole-dechange-standard-pes-0}{PES
Marché}, afin de faciliter leur centralisation sur data.gouv.fr.

Nous avons transformé les données collectées par la DGFiP pour les
rendre accessibles à tous et toutes en accord avec
\href{https://www.data.gouv.fr/fr/datasets/referentiel-de-donnees-marches-publics/}{les
formats XML et JSON réglementaires} définis par
\href{https://www.legifrance.gouv.fr/eli/arrete/2017/4/14/ECFM1637256A/jo/texte}{l'arrêté
du 14 avril 2017} et modifiés par
\href{https://www.legifrance.gouv.fr/affichTexte.do?cidTexte=JORFTEXT000037282994\&dateTexte=\&categorieLien=id}{l'arrêté
du 27 juillet 2018}. Le script de transformation est disponible
\href{https://github.com/etalab/format-commande-publique/tree/dgfip/scripts}{sur
le dépôt Github dédié}.

Vous ne trouverez pas dans ce jeu de données les données sur les marchés
de l'État : elles seront publiées par
\href{https://www.data.gouv.fr/fr/organizations/agence-pour-linformatique-financiere-de-letat/}{l'AIFE}
via d'autres jeux de données sur data.gouv.fr.

Pour davantage de détails sur le dispositif de remontée des données
essentielles de la commande publique vers data.gouv.fr, nous vous
invitons à lire
\href{https://www.data.gouv.fr/fr/posts/le-point-sur-les-donnees-essentielles-de-la-commande-publique/}{notre
article} consacré au sujet.


\vspace{0.5cm}
\needspace{12\baselineskip}
\subsection*{Fichier consolidé des Bornes de Recharge pour Véhicules Électriques
}\index{borne!de!recharge}\index{bornes}\index{bornes!de!recharge}\index{electrique}\index{geolocalisation}\index{irve}\index{localisation}\index{recharge}\index{transport}\index{vehicule}\index{vehicule!electrique}\index{vehicules}
  \begin{wrapfigure}{r}{2.5cm}
    \centering
    \qrcode[nolink]{https://data.gouv.fr/dataset/5448d3e0c751df01f85d0572}
  \end{wrapfigure}

Licence : \textbf{Licence Ouverte
}\newline
Créé le : 2014-10-23\newline
Modifié le : 2019-03-08\newline
Granularité : au point d'intérêt\newline
Mise à jour : mensuelle\newline
Popularité : 6 réutilisations,  15 suivis\newline
Mots-clé : \emph{borne-de-recharge, bornes, bornes-de-recharge, electrique, geolocalisation, irve, localisation, recharge, transport, vehicule, vehicule-electrique, vehicules
}\newline
Permalien : \url{https://data.gouv.fr/dataset/5448d3e0c751df01f85d0572}\newline

\par
\noindent
    Contexte

Depuis le
\href{https://www.legifrance.gouv.fr/affichTexte.do?cidTexte=JORFTEXT000033860620\&dateTexte=\&categorieLien=id}{décret
n\degree{} 2017-26 du 12 janvier 2017}, les données relatives à la
localisation géographique et aux caractéristiques techniques des
stations et des points de recharge de véhicules électriques ouverts au
public doivent être publiées sur data.gouv.fr. Le format des données est
défini par
\href{https://www.legifrance.gouv.fr/affichTexte.do?cidTexte=JORFTEXT000033860733\&categorieLien=id}{l'arrêté
du 12 janvier 2017}. Etalab réalise une consolidation régulière des
fichiers déposés sur data.gouv.fr avec le tag \texttt{irve}.
Consolidation à partir de 2018

Depuis septembre 2018, les données sont consolidées de manière
automatique par Etalab. Le code source est
\href{https://github.com/etalab/schema.data.gouv.fr/blob/master/irve/aggregration/irve.ipynb}{disponible
ici}.

Les fichiers sources doivent être au format \texttt{csv} et respecter le
format défini par
\href{https://www.legifrance.gouv.fr/affichTexte.do?cidTexte=JORFTEXT000033860733\&categorieLien=id}{l'arrêté
du 12 janvier 2017} pour apparaitre dans la version consolidée,
notamment sur les colonnes \texttt{id\_pdc} et \texttt{date\_maj} qui
servent de pivot. Plus d'information sur le format attendu sont
\href{https://www.data.gouv.fr/fr/datasets/fichier-exemple-stations-de-recharge-de-vehicules-electriques/}{disponibles
ici}. Consolidation de 2014 à 2016

Ce fichier est une version consolidée des sources suivantes:
\href{https://www.data.gouv.fr/fr/datasets/caracteristiques-et-localisation-des-stations-de-recharge-supercharger-tesla/}{Stations
Tesla},
\href{https://www.data.gouv.fr/fr/datasets/localisation-des-bornes-de-recharge-dediees-aux-vehicules-electriques-sur-le-territoire-de-rennes-rm/}{Bornes
de la Métropole de Rennes},
\href{https://www.data.gouv.fr/fr/datasets/infrastructure-de-recharge-pour-vehicules-electriques-legers-reseau-renault/}{Bornes
dans les Concessions Renault},
\href{https://www.data.gouv.fr/fr/datasets/stations-et-espaces-autolib-de-la-metropole-parisienne-prs/}{Bornes
Autolib'},
\href{https://www.data.gouv.fr/fr/datasets/caracteristiques-des-points-de-charge-pour-vehicules-electriques-plus-de-bornes-ouverts-au-public/}{Plus
de Bornes, opérateur en Provence},
\href{https://www.data.gouv.fr/fr/datasets/infrastructures-de-recharge-pour-vehicules-electriques/}{Compagnie
Nationale du Rhône},
\href{https://www.data.gouv.fr/fr/datasets/localisation-des-bornes-de-recharge-pour-vehicules-electriques-dans-les-magasins-e-leclerc/}{Magasins
E.Leclerc}

Données ajoutées en décembre 2014:
\href{https://www.data.gouv.fr/fr/datasets/infrastructures-de-recharge-pour-vehicules-electriques-vincipark/}{Vincipark/Sodetrel},
\href{http://smartdata.grandlyon.com/equipements/station-autopartage/}{Grand
Lyon}, Morbihan Energies

Données ajoutées en octobre 2015: Magasins AUCHAN, Concessions NISSAN,
Réseau ALTERBASE, SyDEV, Freshmile, EFFIA

Données ajoutées en mai 2016: SDE18, SDE24, SDE28, SDE32, MOVeasy, Seine
Aval, SIEML, SDESM, Vienne

Elle sert de support à la carte disponible sur
\url{https://www.data.gouv.fr/fr/reuses/carte-des-bornes-de-recharge-pour-vehicules-electriques/}{]}(https://www.data.gouv.fr/fr/reuses/carte-des-bornes-de-recharge-pour-vehicules-electriques/{]}(https://www.data.gouv.fr/fr/reuses/carte-des-bornes-de-recharge-pour-vehicules-electriques/))
La structure du fichier consolidé reprend celle disponible sur
\url{https://www.data.gouv.fr/fr/datasets/fichier-exemple-stations-de-recharge-de-vehicules-electriques/}{]}(https://www.data.gouv.fr/fr/datasets/fichier-exemple-stations-de-recharge-de-vehicules-electriques/{]}(https://www.data.gouv.fr/fr/datasets/fichier-exemple-stations-de-recharge-de-vehicules-electriques/))avec
l'ajout d'une colonne indiquant l'origine des données.


\vspace{0.5cm}
\needspace{12\baselineskip}
\subsection*{Inventaire des dépôts de code source des organismes publics
}\index{code!source}\index{france}\index{gouvernement}\index{logiciel}\index{organisme!public}
  \begin{wrapfigure}{r}{2.5cm}
    \centering
    \qrcode[nolink]{https://data.gouv.fr/dataset/5bc0d6e5634f4142b4581861}
  \end{wrapfigure}

Licence : \textbf{Licence Ouverte version 2.0
}\newline
Créé le : 2018-10-12\newline
Modifié le : 2019-01-29\newline
Popularité : 1 réutilisation,  2 suivis\newline
Mots-clé : \emph{code-source, france, gouvernement, logiciel, organisme-public
}\newline
Permalien : \url{https://data.gouv.fr/dataset/5bc0d6e5634f4142b4581861}\newline

\par
\noindent
    Ce jeu de données propose des informations sur la liste des dépôts de
code source accessibles depuis les comptes d'organisation des organismes
publics. Collecte des données Ce jeu de données n'est pas exhaustif : la
liste des comptes d'organisation n'est pas complète. Ne sont pour
l'instant traités que les comptes présents sur GitHub. Les comptes
présents sur d'autres plates-formes (par exemple GitLab) pourront être
intégrés dans un second temps.

Le dépôt GitHub présentant la démarche pour la collecte des données se
trouve \href{https://github.com/etalab/data-codes-sources-fr}{ici}.
Ajouter votre organisation Pour ajouter le compte d'organisation de
votre organisme, vous pouvez proposer de modifier
\href{https://github.com/DISIC/politique-de-contribution-open-source/blob/master/OrgAccounts}{le
fichier} pertinent dans le dépôt de la DINSIC relatif à la Politique de
Contribution Open Source de l'État. Schémas des données Une description
des différents champs des données ainsi que les schémas des données au
format Table Schema sont disponibles
\href{https://github.com/etalab/data-codes-sources-fr\#description-des-donn\%C3\%A9es}{sur
GitHub}


\vspace{0.5cm}
\needspace{12\baselineskip}
\subsection*{Lauréats du concours Dataconnexions
}\index{concours}\index{dataconnexions}\index{datavisualisation}\index{etalab}\index{open!data}
  \begin{wrapfigure}{r}{2.5cm}
    \centering
    \qrcode[nolink]{https://data.gouv.fr/dataset/577b8b6b88ee38382d0ddd2c}
  \end{wrapfigure}

Licence : \textbf{Licence Ouverte
}\newline
Créé le : 2016-07-05\newline
Modifié le : 2016-07-05\newline
Mise à jour : annuelle\newline
Popularité : 1 réutilisation,  2 suivis\newline
Mots-clé : \emph{concours, dataconnexions, datavisualisation, etalab, open-data
}\newline
Permalien : \url{https://data.gouv.fr/dataset/577b8b6b88ee38382d0ddd2c}\newline

\par
\noindent
    Ce jeu de données compile les projets finalistes et lauréats des
différentes éditions du concours Dataconnexions, organisé régulièrement
par la mission Etalab.

Le concours Dataconnexions récompense les meilleurs services,
applications ou data-visualisations réutilisant au moins un jeu de
données publiques disponible sur le portail data.gouv.fr.

Pour en savoir plus :\url{https://www.etalab.gouv.fr/dataconnexions}
Structure du jeu de données : Année / Édition / Date de la cérémonie /
Projet / Résultat / Catégorie / Site du projet / Producteur / SIRET /
URL Producteur / Partenaires du projet / Contact / Description du projet


\vspace{0.5cm}
\needspace{12\baselineskip}
\subsection*{Ouverture des données publiques du logement : rapport et cartographie
}\index{anil}\index{cartographie}\index{cimap}\index{habitat}\index{logement}\index{logements}\index{rapport}
  \begin{wrapfigure}{r}{2.5cm}
    \centering
    \qrcode[nolink]{https://data.gouv.fr/dataset/560e91c2c751df29ca756894}
  \end{wrapfigure}

Licence : \textbf{Licence Ouverte
}\newline
Créé le : 2015-10-02\newline
Modifié le : 2016-03-02\newline
Granularité : au pays\newline
Mise à jour : ponctuelle\newline
Popularité : 1 réutilisation,  1 suivi\newline
Mots-clé : \emph{anil, cartographie, cimap, habitat, logement, logements, rapport
}\newline
Permalien : \url{https://data.gouv.fr/dataset/560e91c2c751df29ca756894}\newline

\par
\noindent
    Dans le cadre de la politique du gouvernement en faveur de l'ouverture
des données publiques, il a été décidé de lancer des débats sur quatre
thématiques prioritaires parmi lesquelles figure le logement. Le Conseil
national de l'habitat (CNH) s'est vu confier le rôle d'animation des
échanges, en lien avec la mission Etalab. Le groupe de travail, composé
de membres du CNH et d'acteurs de la communauté de l'Open Data, présente
dans son rapport les principaux enjeux de l'ouverture des données
publiques en matière de logement ainsi que des propositions
d'améliorations. La cartographie des données logement est issue des
travaux menés par ce groupe.


\vspace{0.5cm}
\needspace{12\baselineskip}
\subsection*{Statistiques d'usage de l'API de géocodage adresse.data.gouv.fr
}\index{api}\index{audience}\index{geocodage}\index{statistiques}
  \begin{wrapfigure}{r}{2.5cm}
    \centering
    \qrcode[nolink]{https://data.gouv.fr/dataset/5a2e6517c751df0874ef6bc7}
  \end{wrapfigure}

Licence : \textbf{Licence Ouverte
}\newline
Créé le : 2017-12-11\newline
Modifié le : 2017-12-11\newline
Mise à jour : quotienne\newline
Popularité : 1 réutilisation,  0 suivi\newline
Mots-clé : \emph{api, audience, geocodage, statistiques
}\newline
Permalien : \url{https://data.gouv.fr/dataset/5a2e6517c751df0874ef6bc7}\newline

\par
\noindent
    Ces statistiques détaillent l'usage de l'API de géocodage
adresse.data.gouv.fr
\href{https://adresse.data.gouv.fr/data/stats/stats-api.html}{}


\vspace{0.5cm}
\needspace{12\baselineskip}
\subsection*{Tickets de support logiciels libres (interministériel)
}\index{logiciel}\index{support}
  \begin{wrapfigure}{r}{2.5cm}
    \centering
    \qrcode[nolink]{https://data.gouv.fr/dataset/56b09e02c751df0962c30481}
  \end{wrapfigure}

Licence : \textbf{Licence Ouverte
}\newline
Créé le : 2016-02-02\newline
Modifié le : 2016-02-08\newline
Granularité : au pays\newline
Mise à jour : ponctuelle\newline
Popularité : 3 réutilisations,  2 suivis\newline
Mots-clé : \emph{logiciel, support
}\newline
Permalien : \url{https://data.gouv.fr/dataset/56b09e02c751df0962c30481}\newline

\par
\noindent
    La liste des tickets ouverts dans le cadre du
\href{https://forum.code.gouv2.fr/t/accord-cadre-interministeriel-support-des-logiciels-libres-2011/58}{marché
interministériel de support logiciels libres} entre fin 2012 et 2015.


\vspace{0.5cm}
\needspace{3\baselineskip} \rule{4cm}{0.25pt}\newline\textbf{Aussi disponible du même producteur :}\begin{itemize}
\item \href{https://data.gouv.fr/dataset/5b7accfb8b4c4134620351a2}{Avis rendu par l'administrateur général des données }
\item \href{https://data.gouv.fr/dataset/5a9ac6b9c751df4caed2b133}{Codes postaux}
\item \href{https://data.gouv.fr/dataset/5b7acf238b4c413846e91709}{Conclusion de l’expérimentation sur la pseudonymisation des décisions de justice menée par l’équipe de l’AGD}
\item \href{https://data.gouv.fr/dataset/5908a1ffc751df25f3cf5045}{Contributions à la consultation « Plan d’action gouvernement ouvert 2017-2019 : idéation »}
\item \href{https://data.gouv.fr/dataset/5be5b66a634f4148a8b2e466}{Données du hackathon sur les données essentielles de la commande publiques reformatées et agrégées}
\item \href{https://data.gouv.fr/dataset/5bbf2bf28b4c4175c716d3b1}{Données du suivi du Service Public de la Donnée}
\item \href{https://data.gouv.fr/dataset/5715ef5fc751df11b82015f8}{GeoLogos}
\item \href{https://data.gouv.fr/dataset/554210a9c751df2666a7b26c}{GeoZones}
\item \href{https://data.gouv.fr/dataset/595a8c26c751df565555ac60}{Nombre d'embauches par code APE et code ROME}
\item \href{https://data.gouv.fr/dataset/5b1e37a188ee3849c6b7df9b}{Plan d'action national 2018-2020 de la France pour le Partenariat pour un gouvernement ouvert}
\item \href{https://data.gouv.fr/dataset/58f9bfc4c751df756bc6fae2}{Rapport sur les conditions d'ouverture du système Admission Post-Bac}
\item \href{https://data.gouv.fr/dataset/5458f03bc751df46e6225d2e}{Recensement indicatif des données publiques issues des services publics de l'Etat}
\item \href{https://data.gouv.fr/dataset/56cc6d6988ee385864fa79d0}{Référentiel de données marchés publics }
\item \href{https://data.gouv.fr/dataset/57f0b8f088ee3808545ff490}{Réponses à la consultation sur la mise en oeuvre du Service public de la donnée}
\item \href{https://data.gouv.fr/dataset/5a22644f88ee3848529af925}{Statistiques de consultation de data.gouv.fr}
\item \href{https://data.gouv.fr/dataset/5b337e17c751df67939724bd}{Structure du fichier de déclaration de profil d'acheteur}
\end{itemize}

\clearpage
\section{FranceAgriMer - Établissement national des produits de l'agriculture et de la mer}


\begin{center}
  \includegraphics[width=3cm]{images/orga/cd_6d9c448556476ca76e8c4fcf0221a3-100.jpg}
\end{center}


Intermédiaire entre les filières et l'État, FranceAgriMer organise leurs
échanges. Etablissement national des produits de l'agriculture et de la
mer, exerce ses missions pour le compte de l'État, en lien avec le
ministère de l'Agriculture, de l'Agroalimentaire et de la Forêt. Ces
missions consistent principalement à favoriser la concertation au sein
des filières de l'agriculture et de la forêt, à assurer la connaissance
et l'organisation des marchés, ainsi qu'à gérer des aides publiques
nationales et communautaires. FranceAgriMer met en œuvre les politiques
publiques de régulation des marchés, assure une veille économique qui
renforce l'efficacité des filières et favorise leur organisation. Grâce
au dialogue permanent qu'il suscite auprès des filières, l'établissement
constitue le lieu privilégié des échanges entre ces dernières et les
pouvoirs publics, se posant comme un partenaire incontournable de
concertation et d'arbitrage.

Né le 1er avril 2009 de la fusion de cinq offices agricoles (Ofimer,
Office de l'élevage, ONIGC, Onippam et Viniflhor), FranceAgriMer,
établissement national des produits de l'agriculture et de la mer sous
tutelle de l'État, a été créé par l'ordonnance n\degree{}2009-325 du 25
mars 2009, dans le cadre de la révision générale des politiques
publiques.


\vspace{0.5cm}

\needspace{12\baselineskip}
\subsection*{COTATIONS DES OVINS DE BOUCHERIE
}\index{boucherie}\index{cotations}\index{ovins}
  \begin{wrapfigure}{r}{2.5cm}
    \centering
    \qrcode[nolink]{https://data.gouv.fr/dataset/5685422088ee38415daf0bf7}
  \end{wrapfigure}

Licence : \textbf{Licence Ouverte
}\newline
Créé le : 2015-12-31\newline
Modifié le : 2018-08-22\newline
Mise à jour : hebdomadaire\newline
Popularité : 1 réutilisation,  0 suivi\newline
Mots-clé : \emph{boucherie, cotations, ovins
}\newline
Permalien : \url{https://data.gouv.fr/dataset/5685422088ee38415daf0bf7}\newline

\par
\noindent
    cotations régionales et nationales des agneaux et brebis de boucherie,
en euro par kilo carcasse, frais d'approche inclus prix moyens pondérés
des agneaux et brebis de boucherie


\vspace{0.5cm}
\needspace{12\baselineskip}
\subsection*{COTATIONS GROS BOVINS VIFS
}\index{bovins}\index{cotations}\index{vifs}
  \begin{wrapfigure}{r}{2.5cm}
    \centering
    \qrcode[nolink]{https://data.gouv.fr/dataset/56853d5b88ee3821daaf0bfc}
  \end{wrapfigure}

Licence : \textbf{Licence Ouverte
}\newline
Créé le : 2015-12-31\newline
Modifié le : 2018-08-22\newline
Mise à jour : hebdomadaire\newline
Popularité : 1 réutilisation,  0 suivi\newline
Mots-clé : \emph{bovins, cotations, vifs
}\newline
Permalien : \url{https://data.gouv.fr/dataset/56853d5b88ee3821daaf0bfc}\newline

\par
\noindent
    cotations des gros bovins vifs sur les 10 marchés de référence
interprofessionnels et nationales


\vspace{0.5cm}
\needspace{3\baselineskip} \rule{4cm}{0.25pt}\newline\textbf{Aussi disponible du même producteur :}\begin{itemize}
\item \href{https://data.gouv.fr/dataset/5bd2d7cd8b4c41221db63fd8}{ANALYSES DE LA QUALITÉ DU BLÉ TENDRE (PAR RÉGION et NATIONAL) DE 2004 À 2016 }
\item \href{https://data.gouv.fr/dataset/5bf3d289634f415c8c2fa4f6}{BILAN (NATIONAL ET DEPARTEMENT,  PAR VARIETE) DES PLANTES A PARFUMS, AROMATIQUES ET MEDICINALES}
\item \href{https://data.gouv.fr/dataset/5685428188ee384aa3af0bfa}{COTATIONS DES EQUIDES}
\item \href{https://data.gouv.fr/dataset/5685411988ee386109af0bf5}{COTATIONS DES VEAUX DE 8 JOURS A 4 SEMAINES}
\item \href{https://data.gouv.fr/dataset/5685402a88ee386109af0bf4}{COTATIONS DES VEAUX DE BOUCHERIE}
\item \href{https://data.gouv.fr/dataset/591eeae988ee386bb1adb3dd}{COTATIONS DU RÉSEAU DES NOUVELLES DES MARCHÉS}
\item \href{https://data.gouv.fr/dataset/56853c8188ee381058af0bf5}{COTATIONS GROS BOVINS MAIGRES}
\item \href{https://data.gouv.fr/dataset/5bd6df4b634f415bc94d9f30}{Fabrication d'aliments pour animaux - HISTORIQUE DES INCORPORATIONS ET STOCKS BIOLOGIQUES DEPUIS 2000 }
\item \href{https://data.gouv.fr/dataset/5bd3107d8b4c41715b2cd080}{HISTORIQUE DE LA CONSTATATION NATIONALE DES PRIX MOYENS D'ACHAT EN VRAC COMMUNIQUE A LA COMMISSION EUROPENNE}
\item \href{https://data.gouv.fr/dataset/5bd30a468b4c41685e327e1a}{HISTORIQUE DES PRIX ANNUELS MOYENS PAYÉS AUX PRODUCTEURS – CÉRÉALES ET OLÉOPROTÉAGINEUX BIO À PARTIR DE 2005/2006}
\item \href{https://data.gouv.fr/dataset/5bd300c98b4c415b5ad7561d}{HISTORIQUE DES PRIX MOYENS (MENSUELS ET TRIMESTRIELS) PAYÉS AUX PRODUCTEURS DEPUIS 2005 - CÉRÉALES ET OLÉOPROTÉAGINEUX }
\item \href{https://data.gouv.fr/dataset/5bd30e9a8b4c416f188cc800}{HISTORIQUE DES PRIX MOYENS VRAC VSIG ET IGP (DEPUIS 2000)}
\item \href{https://data.gouv.fr/dataset/5bc9d7e88b4c410cd7d779aa}{historique national sur la collecte bio de grandes cultures et oléo-protéagineux depuis 2000}
\item \href{https://data.gouv.fr/dataset/5bd06828634f417e26b2454b}{Historique (régional, national et départemental) sur les stocks de grandes cultures et oléo-protéagineux depuis 2000}
\item \href{https://data.gouv.fr/dataset/5bc991ed8b4c41165aa5a6d1}{Historique régional sur la collecte mensuelle des grandes cultures et oléo-protéagineux bio depuis 2000}
\item \href{https://data.gouv.fr/dataset/5bd1c58a8b4c41795e8a75cd}{Historique régional sur les stocks de grandes cultures et oléo-protéagineux bio depuis 2000}
\item \href{https://data.gouv.fr/dataset/5bd2da1b8b4c4126548ecc83}{Prix des céréales françaises (mise à jour quotidienne)}
\item \href{https://data.gouv.fr/dataset/5b8ea0c9634f41466a5ad1aa}{Récapitulatif des cotations viandes (bovins, veaux, ovins, équidés, porcs, caprins)}
\item \href{https://data.gouv.fr/dataset/5bd2bfa18b4c417f20e18d65}{RÉPARTION NATIONALE DU BLÉ TENDRE - 1993 / 2013}
\item \href{https://data.gouv.fr/dataset/5b8ea31d634f4149ef2b566f}{Séries chronologiques cotations viandes (bovins, veaux, ovins, équidés, porcs, caprins)}
\item \href{https://data.gouv.fr/dataset/5bd821a88b4c41223873b470}{Séries chronologiques sur les surfaces déclarées à l'ASP au titre de la PAC}
\end{itemize}

\clearpage
\section{Grand Paris Seine Ouest}


\begin{center}
  \includegraphics[width=3cm]{images/orga/7d_4ddbd22071486a8e59b0364f995fa5-100.jpg}
\end{center}


Grand Paris Seine Ouest (GPSO) est l'un des 12 Territoires de la
Métropole du Grand Paris. L'Etablissement public territorial compte plus
de 314 000 habitants et est recouvert à 39\% de forêts et d'espaces
verts. 8 villes composent GPSO :
\href{https://www.data.gouv.fr/fr/organizations/ville-de-boulogne-billancourt/}{Boulogne-Billancourt},
\href{https://www.data.gouv.fr/fr/organizations/commune-de-chaville/}{Chaville},
\href{https://www.data.gouv.fr/fr/organizations/ville-d-issy-les-moulineaux/}{Issy-les-Moulineaux},
\href{https://www.data.gouv.fr/fr/organizations/ville-de-marnes-la-coquette/}{Marnes-la-Coquette},
\href{https://www.data.gouv.fr/fr/organizations/ville-de-meudon/}{Meudon},
\href{https://www.data.gouv.fr/fr/organizations/sevres/}{Sèvres},
\href{https://www.data.gouv.fr/fr/organizations/ville-de-vanves/}{Vanves},
\href{https://www.data.gouv.fr/fr/organizations/mairie-de-ville-davray/}{Ville-d'Avray}.

Territoire innovant, durable et numérique, GPSO a validé son contrat de
développement territorial (CDT) avec l'État dans le cadre de la loi sur
le Grand Paris, le 13 novembre 2013. Avec 22 000 entreprises et
commerces et 166 000 emplois, GPSO constitue un pôle économique
incontournable et attractif pour des acteurs économiques tels que
Coca-Cola, le laboratoire Roche ou L'Équipe. GPSO a en charge les
compétences du développement économique, de l'aménagement du territoire,
de l'équilibre social de l'habitat, de la voirie et du stationnement
d'intérêt communautaire, de la protection et de la mise en valeur de
l'environnement et du cadre de vie, des équipements culturels et
sportifs d'intérêt communautaire et de l'assainissement, assurées grâce
à l'implication de plus de 1 000 agents.


\vspace{0.5cm}

\needspace{12\baselineskip}
\subsection*{Agenda et événements
}\index{agenda}\index{evenement}\index{gpso}\index{programme}\index{rendez!vous}
  \begin{wrapfigure}{r}{2.5cm}
    \centering
    \qrcode[nolink]{https://data.gouv.fr/dataset/54abfff1c751df5fec04805b}
  \end{wrapfigure}

Licence : \textbf{Licence Ouverte
}\newline
Créé le : 2015-01-06\newline
Modifié le : 2018-01-15\newline
Granularité : à l'EPCI\newline
Popularité : 1 réutilisation,  0 suivi\newline
Mots-clé : \emph{agenda, evenement, gpso, programme, rendez-vous
}\newline
Permalien : \url{https://data.gouv.fr/dataset/54abfff1c751df5fec04805b}\newline

\par
\noindent
    Ce jeu de données au format csv correspond à l'agenda en ligne de
l'Établissement Public Grand Paris Seine Ouest (GPSO) regroupant les
villes de Boulogne-Billancourt, Chaville, Issy-les-Moulineaux,
Marnes-la-Coquette, Meudon, Sèvres, Vanves et Ville-d'Avray, et contient
les événements et actualités datées de GPSO et de ses structures
(Conservatoires, Maison de la nature, Seine Ouest Entreprise et Emploi,
Agence locale de l'énergie). Ce fichier est mis à jour automatiquement.


\vspace{0.5cm}
\needspace{12\baselineskip}
\subsection*{Parcs et jardins
}\index{aire!de!jeux}\index{fontaine}\index{fontaine!a!eau}\index{gpso}\index{jardins}\index{lpo}\index{parcs}\index{pique!nique}
  \begin{wrapfigure}{r}{2.5cm}
    \centering
    \qrcode[nolink]{https://data.gouv.fr/dataset/545a25b1c751df15ee9b6045}
  \end{wrapfigure}

Licence : \textbf{Licence Ouverte
}\newline
Créé le : 2014-11-05\newline
Modifié le : 2018-01-15\newline
Granularité : à l'EPCI\newline
Mise à jour : ponctuelle\newline
Popularité : 2 réutilisations,  0 suivi\newline
Mots-clé : \emph{aire-de-jeux, fontaine, fontaine-a-eau, gpso, jardins, lpo, parcs, pique-nique
}\newline
Permalien : \url{https://data.gouv.fr/dataset/545a25b1c751df15ee9b6045}\newline

\par
\noindent
    Donnée géographique des parcs et jardins (localisation, adresse ,
horaires d'ouverture et de fermeture, paysagiste, concepteur, activités
et équipements grand public (toilettes, aire de jeux, tables de
pique-nique, espace canin, bornes fontaines), nichoirs « Refuge LPO »,
gardien) sur le territoire de l'Établissement Public Grand Paris Seine
Ouest (GPSO) regroupant les villes de Boulogne-Billancourt, Chaville,
Issy-les-Moulineaux, Marnes-la-Coquette, Meudon, Sèvres, Vanves et
Ville-d'Avray.


\vspace{0.5cm}
\needspace{12\baselineskip}
\subsection*{Plan Local d'Urbanisme de Chaville approuvé le 05/04/2012 et
dernièrement mis à jour le 13/07/2018
}\index{gpso}\index{planification}\index{plu}\index{urbanisme}
  \begin{wrapfigure}{r}{2.5cm}
    \centering
    \qrcode[nolink]{https://data.gouv.fr/dataset/59e84bd388ee385f78fa0041}
  \end{wrapfigure}

Licence : \textbf{Licence Ouverte
}\newline
Créé le : 2017-10-19\newline
Modifié le : 2019-02-04\newline
Granularité : à l'EPCI\newline
Mise à jour : ponctuelle\newline
Popularité : 1 réutilisation,  0 suivi\newline
Mots-clé : \emph{gpso, planification, plu, urbanisme
}\newline
Permalien : \url{https://data.gouv.fr/dataset/59e84bd388ee385f78fa0041}\newline

\par
\noindent
    Le Plan Local d'Urbanisme (PLU) est le principal document d'urbanisme de
planification de l'urbanisme au niveau communal ou intercommunal.

\href{https://www.geoportail-urbanisme.gouv.fr/map/\#tile=1\&lon=2.2207159949789106\&lat=48.818556737378685\&zoom=13}{Accès
au GéoPortail de l'Urbanisme pour consulter le PLU de Chaville.}


\vspace{0.5cm}
\needspace{12\baselineskip}
\subsection*{Point adresse
}\index{gpso}\index{point!adresse}
  \begin{wrapfigure}{r}{2.5cm}
    \centering
    \qrcode[nolink]{https://data.gouv.fr/dataset/55716fe9c751df588de5726a}
  \end{wrapfigure}

Licence : \textbf{Licence Ouverte
}\newline
Créé le : 2015-06-05\newline
Modifié le : 2018-01-15\newline
Granularité : à l'EPCI\newline
Mise à jour : ponctuelle\newline
Popularité : 1 réutilisation,  1 suivi\newline
Mots-clé : \emph{gpso, point-adresse
}\newline
Permalien : \url{https://data.gouv.fr/dataset/55716fe9c751df588de5726a}\newline

\par
\noindent
    Donnée géographique des points des adresses consolidées avec le Conseil
Départemental des Hauts de Seine (CD92) sur le territoire de
l'Établissement Public Grand Paris Seine Ouest (GPSO) regroupant les
villes de Boulogne-Billancourt, Chaville, Issy-les-Moulineaux,
Marnes-la-Coquette, Meudon, Sèvres, Vanves et Ville-d'Avray.


\vspace{0.5cm}
\needspace{12\baselineskip}
\subsection*{Points d'apport volontaire verre
}\index{borne}\index{collecte}\index{gpso}\index{pav}\index{verre}
  \begin{wrapfigure}{r}{2.5cm}
    \centering
    \qrcode[nolink]{https://data.gouv.fr/dataset/54dc89f2c751df675e467389}
  \end{wrapfigure}

Licence : \textbf{Licence Ouverte
}\newline
Créé le : 2015-02-12\newline
Modifié le : 2018-11-14\newline
Granularité : à l'EPCI\newline
Mise à jour : ponctuelle\newline
Popularité : 2 réutilisations,  0 suivi\newline
Mots-clé : \emph{borne, collecte, gpso, pav, verre
}\newline
Permalien : \url{https://data.gouv.fr/dataset/54dc89f2c751df675e467389}\newline

\par
\noindent
    Donnée géographique des points d'apport volontaire pour la collecte du
verre sur le territoire de l'Établissement Public Grand Paris Seine
Ouest (GPSO) regroupant les villes de Boulogne-Billancourt, Chaville,
Issy-les-Moulineaux, Marnes-la-Coquette, Meudon, Sèvres, Vanves et
Ville-d'Avray.


\vspace{0.5cm}
\needspace{12\baselineskip}
\subsection*{Stationnement PMR (Personne à mobilité réduite)
}\index{gpso}\index{handicap}\index{pmr}\index{stationnement}
  \begin{wrapfigure}{r}{2.5cm}
    \centering
    \qrcode[nolink]{https://data.gouv.fr/dataset/54dc8944c751df65b1467389}
  \end{wrapfigure}

Licence : \textbf{Licence Ouverte
}\newline
Créé le : 2015-02-12\newline
Modifié le : 2019-02-08\newline
Granularité : à l'EPCI\newline
Mise à jour : ponctuelle\newline
Popularité : 1 réutilisation,  0 suivi\newline
Mots-clé : \emph{gpso, handicap, pmr, stationnement
}\newline
Permalien : \url{https://data.gouv.fr/dataset/54dc8944c751df65b1467389}\newline

\par
\noindent
    Donnée géographique des stationnements PMR (Personne à mobilité réduite)
sur le territoire de l'Établissement Public Grand Paris Seine Ouest
(GPSO) regroupant les villes de Boulogne-Billancourt, Chaville,
Issy-les-Moulineaux, Marnes-la-Coquette, Meudon, Sèvres, Vanves et
Ville-d'Avray.


\vspace{0.5cm}
\needspace{12\baselineskip}
\subsection*{Thermographie de 2010
}\index{cartographie}\index{chaleur}\index{deperdition}\index{gpso}\index{habitat}\index{thermographie}
  \begin{wrapfigure}{r}{2.5cm}
    \centering
    \qrcode[nolink]{https://data.gouv.fr/dataset/54dddc9cc751df591e467389}
  \end{wrapfigure}

Licence : \textbf{Licence Ouverte
}\newline
Créé le : 2015-02-13\newline
Modifié le : 2018-01-15\newline
Granularité : à l'EPCI\newline
Mise à jour : ponctuelle\newline
Popularité : 1 réutilisation,  1 suivi\newline
Mots-clé : \emph{cartographie, chaleur, deperdition, gpso, habitat, thermographie
}\newline
Permalien : \url{https://data.gouv.fr/dataset/54dddc9cc751df591e467389}\newline

\par
\noindent
    Donnée géographique de 2010 contenant des informations sur la
thermographie (déperditions de chaleur des habitations) sur le
territoire de l'Établissement Public Grand Paris Seine Ouest (GPSO)
regroupant les villes de Boulogne-Billancourt, Chaville,
Issy-les-Moulineaux, Marnes-la-Coquette, Meudon, Sèvres, Vanves et
Ville-d'Avray.


\vspace{0.5cm}
\needspace{3\baselineskip} \rule{4cm}{0.25pt}\newline\textbf{Aussi disponible du même producteur :}\begin{itemize}
\item \href{https://data.gouv.fr/dataset/5a58cef888ee3849dadcf8cc}{Aires de jeux}
\item \href{https://data.gouv.fr/dataset/59dcd6e588ee3803548eac5c}{Aires de stationnement pour 2 roues}
\item \href{https://data.gouv.fr/dataset/59dcd55688ee387c8d02d39b}{Aménagements cyclables}
\item \href{https://data.gouv.fr/dataset/5a58cace88ee3844258cb5d4}{Arrêts de bus}
\item \href{https://data.gouv.fr/dataset/5a38d7efc751df08275fedaf}{Arrêts de bus aux normes PMR}
\item \href{https://data.gouv.fr/dataset/5a58cfe088ee384bc2c06665}{Bacs à sel}
\item \href{https://data.gouv.fr/dataset/564f32c7c751df2612aad370}{Budget primitif 2015 de la Communauté d'agglomération Grand Paris Seine Ouest}
\item \href{https://data.gouv.fr/dataset/57f670c888ee382e2e5ff490}{Budget primitif 2016}
\item \href{https://data.gouv.fr/dataset/59086857c751df3a3f3ff894}{Budget primitif 2017}
\item \href{https://data.gouv.fr/dataset/54dc88bcc751df63fb467389}{Canisites, distributeurs et dépositaires de sacs canins}
\item \href{https://data.gouv.fr/dataset/584029c588ee38040bc65bb3}{Carrefours équipés de kits sonores}
\item \href{https://data.gouv.fr/dataset/5840261d88ee387c0ac65bb3}{Cheminements accessibles aux PMR}
\item \href{https://data.gouv.fr/dataset/57f66f4788ee382be55ff490}{Compte administratif 2015}
\item \href{https://data.gouv.fr/dataset/596f400288ee3856b03d5651}{Compte administratif 2016}
\item \href{https://data.gouv.fr/dataset/564f3494c751df2612aad371}{Compte administratif de Grand Paris Seine Ouest 2014}
\item \href{https://data.gouv.fr/dataset/5a5c51a788ee3830ecc1824b}{Concepteurs réputés de parcs et jardins}
\item \href{https://data.gouv.fr/dataset/545a2842c751df14309b6045}{Conservatoires}
\item \href{https://data.gouv.fr/dataset/54ddd5d6c751df498d467389}{Corbeilles de rue}
\item \href{https://data.gouv.fr/dataset/54dc8b59c751df6933467389}{Déchèteries fixes}
\item \href{https://data.gouv.fr/dataset/545a27d5c751df15ee9b6046}{Déchèteries mobiles }
\item \href{https://data.gouv.fr/dataset/54dc8bd7c751df675e46738c}{Enclos à sapins}
\item \href{https://data.gouv.fr/dataset/5885c52d88ee3846e69b81a4}{Entreprises "boostées"}
\item \href{https://data.gouv.fr/dataset/54dc8ae4c751df675e46738b}{Entreprises de plus de 20 salariés en 2015}
\item \href{https://data.gouv.fr/dataset/54dc8284c751df5135467389}{Equipements sportifs}
\item \href{https://data.gouv.fr/dataset/545a2927c751df1e6b9b6045}{Espaces de travail collaboratif ou de co-working }
\item \href{https://data.gouv.fr/dataset/5a58d07788ee384d5e9999ae}{Fontaines}
\item \href{https://data.gouv.fr/dataset/5a5c4fcf88ee382f174f9100}{Illuminations de Noël}
\item \href{https://data.gouv.fr/dataset/5b46281cc751df618a03d0ea}{Les arbres}
\item \href{https://data.gouv.fr/dataset/5b0fd3e188ee3836b6da8842}{Liste des subventions  aux associations 2017}
\item \href{https://data.gouv.fr/dataset/5bf82a638b4c413ad55b101e}{Marchés publics}
\item \href{https://data.gouv.fr/dataset/5a58d03588ee384b7e1b101f}{Mobilier d'affiche libre}
\item \href{https://data.gouv.fr/dataset/54ddd925c751df498d46738a}{Opération Habitat Qualité}
\item \href{https://data.gouv.fr/dataset/5a5c4f2488ee382d08908b26}{Ouvrages d'art}
\item \href{https://data.gouv.fr/dataset/545a28b1c751df19749b6045}{Permanences des conseillers info-énergie}
\item \href{https://data.gouv.fr/dataset/5c5809918b4c4121ddbbd46c}{Plan Local d'Urbanisme de Boulogne-Billancourt approuvé le 08/04/2004 et dernièrement révisé le 19/12/2019}
\item \href{https://data.gouv.fr/dataset/59e84d0388ee385294de0187}{Plan Local d'Urbanisme de Marnes-la-Coquette approuvé le 09/02/2011 et dernièrement modifié le 21/12/2017}
\item \href{https://data.gouv.fr/dataset/59e84a3788ee385f78fa0040}{Plan Local d'Urbanisme de Sèvres révisé et approuvé le 18/12/2015 et dernièrement mis à jour le 13/07/2018}
\item \href{https://data.gouv.fr/dataset/59e84e7888ee386a39fd6250}{Plan Local d'Urbanisme de Vanves approuvé le 22/06/2011 et dernièrement modifié le 15/12/2015}
\item \href{https://data.gouv.fr/dataset/59e84f5b88ee38685d8cf1f0}{Plan Local d'Urbanisme de Ville-d'Avray approuvé le 18/12/2013 et dernièrement mis à jour le 13/07/2018}
\item \href{https://data.gouv.fr/dataset/59e8482388ee385f78fa003f}{Plan Local d'Urbanisme d'Issy-les-Moulineaux approuvé le 17/12/2015 et dernièrement mis à jour le 04/01/2018}
\item \href{https://data.gouv.fr/dataset/54dc8743c751df608c467389}{Points d'apport volontaire textile}
\item \href{https://data.gouv.fr/dataset/54dddc19c751df5241467389}{Potentiel solaire de 2010}
\item \href{https://data.gouv.fr/dataset/5a5c4e3788ee382b343dd0ec}{Quartiers}
\item \href{https://data.gouv.fr/dataset/5a5c539188ee383473efad1f}{Réalisations de travaux en 2015}
\item \href{https://data.gouv.fr/dataset/54dc8a62c751df675e46738a}{Secteurs de collecte des encombrants}
\item \href{https://data.gouv.fr/dataset/54dc8c4ac751df6765467389}{Secteurs de collecte des ordures ménagères}
\item \href{https://data.gouv.fr/dataset/54dc87e8c751df608c46738a}{Secteurs de collecte sélective}
\item \href{https://data.gouv.fr/dataset/5b2a4f0588ee38393a34f545}{Travaux associés à des Avis Riverains sur le territoire de GPSO}
\item \href{https://data.gouv.fr/dataset/5b7684048b4c4173fa502c66}{Voies dont la collecte des déchets s'effectue à l'extrémité de la voie}
\item \href{https://data.gouv.fr/dataset/59dcd65988ee387c7faa1aca}{Zones apaisées}
\item et 1 autres jeux de données\end{itemize}

\clearpage
\section{Grand Poitiers Open Data}


\begin{center}
  \includegraphics[width=3cm]{images/orga/b9_066a0cc577427b89fff91fafd44b4b-100.jpg}
\end{center}


Entre Paris et Bordeaux, à 1h30 seulement de la capitale française,
l'agglomération de Grand Poitiers regroupe 40 communes au dynamisme
économique, touristique et économique reconnu.\\
Prisée des entreprises et choisie par plus de 25 000 étudiants chaque
année, l'agglomération de Grand Poitiers a su garder néanmoins cette
dimension humaine, qui manque parfois aux grandes métropoles. La qualité
de vie et le bien être de ses habitants, le développement culturel et
l'attractivité économique sont au cœur de toutes ses actions. Avec ses 2
000 ans d'histoire et le parc d'attraction le plus futuriste d'Europe,
Poitiers et son agglomération vous accueilleront avec charme et
caractère. Les trésors de l'art roman européen à Poitiers, le
Futuroscope, les cités médiévales de Lusignan et Chauvigny, le site
Gallo-Romain de Sanxay, le Golf et la base de loisirs de Saint-Cyr, les
Géants du Ciel, le musée Sainte Croix de Poitiers et sa collection
Camille Claudel\ldots{} La palette touristique est large !

Grand Poitiers : territoire d'innovation et de recherche De la French
Tech dédiée à l'Edutainment au Biopole santé, Grand Poitiers est un pôle
de référence de la haute technologie au cœur de la Nouvelle Aquitaine.

Grand Poitiers : un tissu économique riche et diversifié de 16 300
entreprises - 27 zones économiques - Plus de 88 000 salariés - 65 \%
d'emplois dans les services - 44 établissements d'enseignement supérieur
et 900 enseignants chercheurs

\textbf{\href{https://data.grandpoitiers.fr/pages/accueil/}{Accéder à la
plateforme locale de données publiques Grand Poitiers Open Data}}

\href{http://un-nouveau-grand-poitiers.fr/c__290_1150__Grand_Poitiers_Open_Data.html}{En
savoir plus sur Grand Poitiers Open Data}

\href{https://infogeo.grandpoitiers.fr/geoportal/catalog/main/home.page}{Accès
aux données du Portail de l'information géographique de Grand Poitiers}


\vspace{0.5cm}

\needspace{12\baselineskip}
\subsection*{Environnement - Défibrillateurs - Grand Poitiers (13 communes) Données
métiers
}\index{appareil}\index{defibrillateurs}\index{donnees!ouvertes}\index{health}\index{passerelle!inspire}\index{sante}\index{secours}\index{systeme!de!secours}
  \begin{wrapfigure}{r}{2.5cm}
    \centering
    \qrcode[nolink]{https://data.gouv.fr/dataset/5513078cc751df0cfbcea12a}
  \end{wrapfigure}

Licence : \textbf{Licence Ouverte version 2.0
}\newline
Créé le : 2015-03-25\newline
Modifié le : 2019-02-08\newline
Popularité : 1 réutilisation,  0 suivi\newline
Mots-clé : \emph{appareil, defibrillateurs, donnees-ouvertes, health, passerelle-inspire, sante, secours, systeme-de-secours
}\newline
Permalien : \url{https://data.gouv.fr/dataset/5513078cc751df0cfbcea12a}\newline

\par
\noindent
    Description du parc de défibrilateurs gérés par la collectivité (Grand
Poitiers 13 communes - Ville de Poitiers - CCAS de Poitiers)

\textbf{Origine}

Compilations et illustrations cartographiques des données relatives aux
défibrillateurs gérés par la collectivité (Grand Poitiers - Ville de
Poitiers - CCAS de Poitiers). La mise à jour à lieu au fil de l'eau. Les
données issues d'autres entités (Département, Région\ldots{}) présentes
sur Grand Poitiers sont intégrées au fil de l'eau.

\textbf{Organisations partenaires}

Grand Poitiers, Grand Poitiers - Direction Hygiène Publique - Qualité
Environnementale

➞
\href{https://geo.data.gouv.fr/fr/datasets/2b0f913041394a9384c6d9d9e79d064100f3d67a}{Consulter
cette fiche sur geo.data.gouv.fr}


\vspace{0.5cm}
\needspace{12\baselineskip}
\subsection*{Mobilité - Stationnement vélos - Grand Poitiers Données métiers
}\index{abris}\index{appuis}\index{donnees!ouvertes}\index{parc!de!stationnement}\index{passerelle!inspire}\index{reseaux!de!transport}\index{transportation}\index{velos}
  \begin{wrapfigure}{r}{2.5cm}
    \centering
    \qrcode[nolink]{https://data.gouv.fr/dataset/5513079e88ee3857e27c596e}
  \end{wrapfigure}

Licence : \textbf{Licence Ouverte version 2.0
}\newline
Créé le : 2015-03-25\newline
Modifié le : 2019-02-08\newline
Popularité : 1 réutilisation,  0 suivi\newline
Mots-clé : \emph{abris, appuis, donnees-ouvertes, parc-de-stationnement, passerelle-inspire, reseaux-de-transport, transportation, velos
}\newline
Permalien : \url{https://data.gouv.fr/dataset/5513079e88ee3857e27c596e}\newline

\par
\noindent
    Ces données, à l'échelle de l'ancien territoire de Grand Poitiers (soit
13 communes) concerne l'implantation des points de stationnement vélos
(Appuis et abris vélos) avec mention du nombre de places de
stationnement.

A terme ces données concerneront l'ensemble du territoire de la
communauté urbaine de Poitiers (soit 40 communes).

\textbf{Origine}

Les données sont issues de relevés terrains et de recencements effectués
dans chacune des communes

\textbf{Organisations partenaires}

Grand Poitiers, Grand Poitiers - Direction Mobilité

➞
\href{https://geo.data.gouv.fr/fr/datasets/7a937b52b97f5327d751eefe0fc2ac2f15c16f08}{Consulter
cette fiche sur geo.data.gouv.fr}


\vspace{0.5cm}
\needspace{12\baselineskip}
\subsection*{Propreté - Bornes à verre - Grand Poitiers Données métiers
}\index{borne!a!verre}\index{collecte}\index{colonnes!enterrees}\index{dechets}\index{donnees!ouvertes}\index{environment}\index{passerelle!inspire}\index{proprete}\index{recyclage!des!dechets}\index{tri!des!dechets}\index{verre}\index{verre!usage}
  \begin{wrapfigure}{r}{2.5cm}
    \centering
    \qrcode[nolink]{https://data.gouv.fr/dataset/5513060988ee3851297c596c}
  \end{wrapfigure}

Licence : \textbf{Licence Ouverte version 2.0
}\newline
Créé le : 2015-03-25\newline
Modifié le : 2019-02-08\newline
Popularité : 1 réutilisation,  0 suivi\newline
Mots-clé : \emph{borne-a-verre, collecte, colonnes-enterrees, dechets, donnees-ouvertes, environment, passerelle-inspire, proprete, recyclage-des-dechets, tri-des-dechets, verre, verre-usage
}\newline
Permalien : \url{https://data.gouv.fr/dataset/5513060988ee3851297c596c}\newline

\par
\noindent
    Localisation de l'ensemble des bornes à verres sur GRAND POITIERS. Mise
à jour à la date du 22/09/2014.

\textbf{Origine}

Bornes à verre: situation géographique (adresse), numéro attribué à la
borne, volume, fréquence de collecte.

\textbf{Organisations partenaires}

Grand Poitiers, Grand Poitiers - Direction Déchets Propreté

➞
\href{https://geo.data.gouv.fr/fr/datasets/565ef2e0c5241147d92a03882aa726849bf49af7}{Consulter
cette fiche sur geo.data.gouv.fr}


\vspace{0.5cm}
\needspace{12\baselineskip}
\subsection*{Référentiel géographique - Adresses - Grand Poitiers Données de
référence
}\index{adresses}\index{boites!aux!lettres}\index{code!postale}\index{donnees!ouvertes}\index{numero!voies}\index{passerelle!inspire}\index{rue}\index{society}
  \begin{wrapfigure}{r}{2.5cm}
    \centering
    \qrcode[nolink]{https://data.gouv.fr/dataset/5513078dc751df0cfbcea12b}
  \end{wrapfigure}

Licence : \textbf{Licence Ouverte version 2.0
}\newline
Créé le : 2015-03-25\newline
Modifié le : 2019-02-08\newline
Popularité : 1 réutilisation,  0 suivi\newline
Mots-clé : \emph{adresses, boites-aux-lettres, code-postale, donnees-ouvertes, numero-voies, passerelle-inspire, rue, society
}\newline
Permalien : \url{https://data.gouv.fr/dataset/5513078dc751df0cfbcea12b}\newline

\par
\noindent
    Actuellement ce jeu de données renseigne uniquement les adresses sur
Poitiers mais s'étendra progressivement à Grand Poitiers. Il est la base
des points d'adresses d'immeubles ou de terrains et de leur adresses
normalisée. Son utilisation se tourne soit vers la constitution du RIL
(Répertoire Immeubles Localisé), soit vers un géoréférencement de toutes
données comportant une adresse pour une étude thématique. L'echelle
1/100 est conseillée pour les plans de détail.

\textbf{Origine}

Plan schématique établi a partir du fichier RIL de l'INSEE et d'enquêtes
terrain.

\textbf{Organisations partenaires}

Grand Poitiers, Grand Poitiers - DEPP - CA Valorisation des Données

➞
\href{https://geo.data.gouv.fr/fr/datasets/48b6287b156c122b577d0747486c06a7c7ebcc07}{Consulter
cette fiche sur geo.data.gouv.fr}


\vspace{0.5cm}
\needspace{12\baselineskip}
\subsection*{Référentiel géographique - Equipements publics - Grand Poitiers Données
de référence
}\index{administration}\index{adresses}\index{batiments}\index{college}\index{culture}\index{donnees!ouvertes}\index{ecole}\index{elementaire}\index{enfance}\index{environnement}\index{equipements!culturels}\index{equipements!sportifs}\index{etablissement!public}\index{maternelle}\index{passerelle!inspire}\index{patrimoine}\index{primaire}\index{sante}\index{scolaire}\index{social}\index{sport}\index{structure}\index{toilettes!publiques}\index{transport}
  \begin{wrapfigure}{r}{2.5cm}
    \centering
    \qrcode[nolink]{https://data.gouv.fr/dataset/5513078f88ee3856407c596d}
  \end{wrapfigure}

Licence : \textbf{Licence Ouverte version 2.0
}\newline
Créé le : 2015-03-25\newline
Modifié le : 2019-02-08\newline
Popularité : 1 réutilisation,  0 suivi\newline
Mots-clé : \emph{administration, adresses, batiments, college, culture, donnees-ouvertes, ecole, elementaire, enfance, environnement, equipements-culturels, equipements-sportifs, etablissement-public, maternelle, passerelle-inspire, patrimoine, primaire, sante, scolaire, social, sport, structure, toilettes-publiques, transport
}\newline
Permalien : \url{https://data.gouv.fr/dataset/5513078f88ee3856407c596d}\newline

\par
\noindent
    Jeu de données présente les équipements publics recevant ou non du
public de Grand Poitiers. Organisé par thème, il est principalement
utilisé pour des autorisations d'ouvertures de locaux ou d'études
thématiques destinées à de l'information grand public.

\textbf{Origine}

A partir d'informations livrées par les services, de plans cadastraux au
2000ème et de recherche de géolocalisation.

\textbf{Organisations partenaires}

Grand Poitiers, Grand Poitiers - Direction Communication

➞
\href{https://geo.data.gouv.fr/fr/datasets/1b2775d31e698adc4846b36322ba2d790df93db1}{Consulter
cette fiche sur geo.data.gouv.fr}


\vspace{0.5cm}
\needspace{12\baselineskip}
\subsection*{Référentiel géographique - Levés topographiques - Grand Poitiers Données
de référence
}\index{altitude}\index{chantiers}\index{donnees!ouvertes}\index{foliotages}\index{leves}\index{passerelle!inspire}\index{topographie}\index{utilities!communication}
  \begin{wrapfigure}{r}{2.5cm}
    \centering
    \qrcode[nolink]{https://data.gouv.fr/dataset/55968cb688ee3808594a22ea}
  \end{wrapfigure}

Licence : \textbf{Licence Ouverte version 2.0
}\newline
Créé le : 2015-07-03\newline
Modifié le : 2019-02-08\newline
Popularité : 1 réutilisation,  0 suivi\newline
Mots-clé : \emph{altitude, chantiers, donnees-ouvertes, foliotages, leves, passerelle-inspire, topographie, utilities-communication
}\newline
Permalien : \url{https://data.gouv.fr/dataset/55968cb688ee3808594a22ea}\newline

\par
\noindent
    Il s'agit de la localisation de tous les levés topographiques avant
projets exécutés depuis 1998. Cette base de donnée permet la gestion des
levés topographiques sur le territoires de Grand Poitiers. (nom de la
société, date de commande, date de livraison, service demandeur, montant
de la prestation, commune, \ldots{}.) Les levés topos servent
essentiellement aux services techniques pour créer des projets. Derrière
chaque surfaces d'emprise du levé, se trouve un levé topographique avant
projet.

\textbf{Origine}

Il s'agit de plans topographiques au 200ème issus de levés de géomètre.
Représentations des détails topo, des réseaux, du mobilier urbains, de
la voirie de la signalisation routière

\textbf{Organisations partenaires}

Grand Poitiers Communauté urbaine, Grand Poitiers - DEPP - CA
Valorisation des Données

➞
\href{https://geo.data.gouv.fr/fr/datasets/c4ec6c2577ef93c38f249d117beb5fd95d197534}{Consulter
cette fiche sur geo.data.gouv.fr}


\vspace{0.5cm}
\needspace{12\baselineskip}
\subsection*{Réseaux - Point de couverture wifi - Grand Poitiers Données métiers
}\index{borne}\index{donnees!ouvertes}\index{fournisseur!de!services!internet}\index{hotspot}\index{internet}\index{passerelle!inspire}\index{publicwifipoitiers}\index{reseau}\index{reseau!electronique!dinformation}\index{utilities!communication}\index{wifi}
  \begin{wrapfigure}{r}{2.5cm}
    \centering
    \qrcode[nolink]{https://data.gouv.fr/dataset/5513078b88ee3856407c596c}
  \end{wrapfigure}

Licence : \textbf{Licence Ouverte version 2.0
}\newline
Créé le : 2015-03-25\newline
Modifié le : 2019-02-08\newline
Popularité : 1 réutilisation,  0 suivi\newline
Mots-clé : \emph{borne, donnees-ouvertes, fournisseur-de-services-internet, hotspot, internet, passerelle-inspire, publicwifipoitiers, reseau, reseau-electronique-dinformation, utilities-communication, wifi
}\newline
Permalien : \url{https://data.gouv.fr/dataset/5513078b88ee3856407c596c}\newline

\par
\noindent
    Positionnement des bornes wifi publiques du service PublicWifiPoitiers
de Grand Poitiers. Sa mise à jour est effectuée au fur et à mesure des
installations et d'arrêt des bornes. La 1ère publication correspond à la
mise en service d'un 1er lot cohérent de bornes.

\textbf{Origine}

Les données ont été rentrées manuellement à partir du positionnement
connu vis à vis du cadastre. Les points ne sont pas géoréférencé par GPS
et sont donc positionnés au bâtiment près.

\textbf{Organisations partenaires}

Grand Poitiers, Grand Poitiers - DEPP - CA Valorisation des données

➞
\href{https://geo.data.gouv.fr/fr/datasets/52ed6f1f9ed94a50d977b85d9e16e428e713b842}{Consulter
cette fiche sur geo.data.gouv.fr}


\vspace{0.5cm}
\needspace{3\baselineskip} \rule{4cm}{0.25pt}\newline\textbf{Aussi disponible du même producteur :}\begin{itemize}
\item \href{https://data.gouv.fr/dataset/5b4c72b6b595087b22d496c9}{Agenda culturel-Programmation annuelle depuis 2007 de l'Espace Mendès France}
\item \href{https://data.gouv.fr/dataset/5bc97bdf9ce2e77b62d601ad}{Budget 2018 - CCAS de Poitiers}
\item \href{https://data.gouv.fr/dataset/5bc97c0d9ce2e77b62d601ae}{Budget Primitif 2018 - Grand Poitiers}
\item \href{https://data.gouv.fr/dataset/5bc97bb706e3e73e298c0eb9}{Budget primitif 2018 - Ville de Poitiers}
\item \href{https://data.gouv.fr/dataset/59377b4ba3a72968e18772ef}{Citoyenneté - 1er tour Législatives- 1ère circonscription Poitiers 2017 - résultats}
\item \href{https://data.gouv.fr/dataset/59377b4aa3a72968e2877326}{Citoyenneté - 1er tour Législatives- 2ème circonscription Poitiers 2017 - résultats}
\item \href{https://data.gouv.fr/dataset/59474ceca3a72968e287b2f6}{Citoyenneté - 2ème tour Législatives- 1ère circonscription Poitiers 2017 - résultats}
\item \href{https://data.gouv.fr/dataset/59474ceda3a72968e187b424}{Citoyenneté - 2ème tour Législatives- 2ème circonscription Poitiers 2017 - résultats}
\item \href{https://data.gouv.fr/dataset/593e1304a3a72968e2878d6b}{Citoyenneté - 2ème tour Législatives-2ème circonscription-Poitiers 2017 - résultats – Datavisualisation}
\item \href{https://data.gouv.fr/dataset/590fed6aa3a7290a52372d92}{Citoyenneté - 2eme tour Présidentielle Poitiers 2017 - résultats}
\item \href{https://data.gouv.fr/dataset/593a67ae88ee385f709c587f}{Citoyenneté - Bureaux de vote - Grand Poitiers Données de référence}
\item \href{https://data.gouv.fr/dataset/551a477888ee387f653fa225}{Citoyenneté - Découpage des bureaux de vote - Grand Poitiers Données de référence}
\item \href{https://data.gouv.fr/dataset/551a477ac751df46bd4b1077}{Citoyenneté- Découpage des sites de vote - Grand Poitiers Données de référence}
\item \href{https://data.gouv.fr/dataset/55130617c751df78f8cea129}{Citoyenneté - Quartiers politiques Poitiers - Grand Poitiers Données de référence}
\item \href{https://data.gouv.fr/dataset/55969107c751df26c5a453bc}{Citoyenneté - Reconnaissance des enfants à Poitiers}
\item \href{https://data.gouv.fr/dataset/591564e0c751df0d4ed32b0c}{Citoyenneté - Sites de vote - Grand Poitiers Données de référence}
\item \href{https://data.gouv.fr/dataset/58ef3af5a3a7294c24603eab}{Citoyenneté - Subventions directes attribuées aux associations -2017- Grand Poitiers}
\item \href{https://data.gouv.fr/dataset/58ef3af1a3a7294c24603eaa}{Citoyenneté - Subventions directes attribuées aux associations -2017- Ville de Poitiers}
\item \href{https://data.gouv.fr/dataset/58ef2cfba3a7293d4ac4e183}{Citoyenneté - Subventions indirectes attribuées aux associations - Grand Poitiers}
\item \href{https://data.gouv.fr/dataset/58ef3bcfa3a7294c24603eaf}{Citoyenneté - Subventions indirectes attribuées aux associations - Ville de Poitiers}
\item \href{https://data.gouv.fr/dataset/58ef3bcba3a7294c24603eae}{Culture - Fréquentation des musées}
\item \href{https://data.gouv.fr/dataset/55130795c751df0cfbcea12d}{Découpages administratifs - Aire urbaine Grand Poitiers - Grand Poitiers Données de référence}
\item \href{https://data.gouv.fr/dataset/58becc0e88ee38787319b4d2}{Découpages administratifs - Anciens cantons de Poitiers - Grand Poitiers Données de référence}
\item \href{https://data.gouv.fr/dataset/58becc0dc751df6182998c27}{Découpages administratifs - Anciens cantons Grand Poitiers - Grand Poitiers Données de référence}
\item \href{https://data.gouv.fr/dataset/55130792c751df0620cea130}{Découpages administratifs - Cantons Grand Poitiers - Grand Poitiers Données de référence}
\item \href{https://data.gouv.fr/dataset/551307aa88ee3851107c596d}{Découpages administratifs - Cantons Poitiers - Grand Poitiers Données de référence}
\item \href{https://data.gouv.fr/dataset/58c973ee88ee38511f1a89a5}{Découpages administratifs - Communauté Urbaine GP - 40 communes - Grand Poitiers Données de référence}
\item \href{https://data.gouv.fr/dataset/588f417188ee3871e69b81a4}{Découpages administratifs - Communauté Urbaine GP - 42 communes - Grand Poitiers Données de référence}
\item \href{https://data.gouv.fr/dataset/55130794c751df0cfbcea12c}{Découpages administratifs - Limites Communales GP - Grand Poitiers Données de référence}
\item \href{https://data.gouv.fr/dataset/551307ad88ee383e8c7c596c}{Découpages administratifs - Quartiers Iris Grand Poitiers - Grand Poitiers Données de référence}
\item \href{https://data.gouv.fr/dataset/5513079c88ee3856407c596e}{Découpages administratifs - Quartiers Poitiers - Grand Poitiers Données de référence}
\item \href{https://data.gouv.fr/dataset/55968cb388ee380f554a22ea}{Eau - Bassin d'orage - Grand Poitiers Données métiers}
\item \href{https://data.gouv.fr/dataset/5a7ea010b595082373b66414}{Eau - branchements pour les particuliers}
\item \href{https://data.gouv.fr/dataset/559654ffc751df4fd2a453ba}{Eau - Captage eau potable - Grand Poitiers Données métiers}
\item \href{https://data.gouv.fr/dataset/5596550288ee38296e4a22ea}{Eau - Equipements eau pluviale - Grand Poitiers Données métiers}
\item \href{https://data.gouv.fr/dataset/5596550488ee38296e4a22eb}{Eau - Equipements eau usée - Grand Poitiers Données métiers}
\item \href{https://data.gouv.fr/dataset/5596550888ee382cf44a22eb}{Eau - Réseau de distribution d'eau potable - Grand Poitiers Données métiers}
\item \href{https://data.gouv.fr/dataset/5596920ac751df26c5a453bf}{Eau - Réseaux eaux usées et eaux pluviales - Grand Poitiers Données métiers}
\item \href{https://data.gouv.fr/dataset/5596550bc751df4fd2a453bb}{Eau - Réservoir d'eau potable - Grand Poitiers Données métiers}
\item \href{https://data.gouv.fr/dataset/5513061388ee3851107c596b}{Economie - Pôles économiques - Grand Poitiers Données métiers}
\item \href{https://data.gouv.fr/dataset/55130615c751df0620cea12d}{Economie - ZAE - Grand Poitiers Données métiers}
\item \href{https://data.gouv.fr/dataset/551307b1c751df0cfbcea12f}{Education - Ecoles - Grand Poitiers Données métiers}
\item \href{https://data.gouv.fr/dataset/551183b1c751df294c882845}{Education-Effectifs des écoles publiques élémentaires-Poitiers}
\item \href{https://data.gouv.fr/dataset/5513078388ee3857e27c596b}{Education - Secteurs scolaires - Grand Poitiers Données métiers}
\item \href{https://data.gouv.fr/dataset/55965282c751df3026a453bb}{Environnement- Activité de capture des animaux errants sur poitiers}
\item \href{https://data.gouv.fr/dataset/5596550d88ee382cf44a22ec}{Environnement - Balades Nature - Grand Poitiers (13 communes) Données métiers}
\item \href{https://data.gouv.fr/dataset/5596550f88ee382cf44a22ed}{Environnement - Circuit Ville Nature - Grand Poitiers Données métiers}
\item \href{https://data.gouv.fr/dataset/58d39b1dc751df20b0113e00}{Environnement - classement sonore des infrastructures de transports terrestres - Grand Poitiers (13 communes) Données métiers}
\item \href{https://data.gouv.fr/dataset/55130786c751df0cfbcea129}{Environnement - Distributeurs Toutounets - Grand Poitiers (13 communes) Données métiers}
\item \href{https://data.gouv.fr/dataset/5596572688ee38296e4a22ee}{Environnement - Inventaire des habitats naturels - Grand Poitiers (13 communes) Données métiers}
\item et 44 autres jeux de données\end{itemize}

\clearpage
\section{Haute Autorité de santé (HAS)}


\begin{center}
  \includegraphics[width=3cm]{images/orga/fd_a753642943453eaa9cc5ae003c0edd-100.jpg}
\end{center}


La HAS est une autorité publique indépendante qui contribue à la
régulation du système de santé par la qualité. Elle exerce ses missions
dans les champs de l'évaluation des produits de santé, des pratiques
professionnelles, de l'organisation des soins et de la santé publique.

Ces missions sont définies aux articles 161-37 et suivants du code de la
sécurité sociale. Elles peuvent être regroupées en deux activités
principales : Évaluation et recommandation, et accréditation et
certification.

La HAS conçoit et met également à disposition des acteurs de santé des
outils, guides et méthodes afin d'améliorer leur prise en charge ou la
mise en œuvre de leurs projets.

En savoir plus : www.has-sante.fr


\vspace{0.5cm}

\needspace{12\baselineskip}
\subsection*{Evaluation des médicaments
}\index{evaluation}\index{medicament}\index{medicaments}\index{sante}\index{social}\index{soins}
  \begin{wrapfigure}{r}{2.5cm}
    \centering
    \qrcode[nolink]{https://data.gouv.fr/dataset/53be7179a3a72910f2860b9d}
  \end{wrapfigure}

Licence : \textbf{Licence Ouverte
}\newline
Créé le : 2014-07-10\newline
Modifié le : 2019-02-26\newline
De 2019-01-01 à 2019-12-31\newline
Granularité : au pays\newline
Mise à jour : mensuelle\newline
Popularité : 1 réutilisation,  1 suivi\newline
Mots-clé : \emph{evaluation, medicament, medicaments, sante, social, soins
}\newline
Permalien : \url{https://data.gouv.fr/dataset/53be7179a3a72910f2860b9d}\newline

\par
\noindent
    Le Service Médical Rendu et l'Amélioration du Service Médical Rendu sont
des notions permettant à la Haute autorité de santé (HAS) de proposer
l'inscription d'un médicament sur la liste des spécialités remboursables
en ville et/ou hôpital. Aussi, seuls les médicaments pour lesquels le
laboratoire qui les commercialise souhaite obtenir leur inscription sur
cette liste feront l'objet d'une évaluation du SMR et de l'ASMR. Par
ailleurs, les médicaments génériques ne font pas systématiquement
l'objet d'une évaluation du SMR et de l'ASMR. Seules les évaluations les
plus récentes sont présentées. Les évaluations antérieures restent
disponibles sous forme d'avis textuels sur le site de la HAS
(http://www.has-sante.fr/portail/jcms/r\_1500918/fr/les-avis-sur-les-medicaments)
ou encore sur demande auprès de ses services.


\vspace{0.5cm}
\needspace{12\baselineskip}
\subsection*{Indicateurs de qualité et de sécurité des soins - Infections associées
aux soins - recueil 2016
}\index{ias}\index{qualite!des!soins}\index{sante}\index{securite!des!soins}\index{soins}
  \begin{wrapfigure}{r}{2.5cm}
    \centering
    \qrcode[nolink]{https://data.gouv.fr/dataset/53bfa27aa3a7293e19309ef7}
  \end{wrapfigure}

Licence : \textbf{Licence Ouverte
}\newline
Créé le : 2014-07-10\newline
Modifié le : 2018-01-22\newline
De 2015-01-01 à 2015-12-31\newline
Granularité : au pays\newline
Mise à jour : annuelle\newline
Popularité : 1 réutilisation,  3 suivis\newline
Mots-clé : \emph{ias, qualite-des-soins, sante, securite-des-soins, soins
}\newline
Permalien : \url{https://data.gouv.fr/dataset/53bfa27aa3a7293e19309ef7}\newline

\par
\noindent
    Les données mises à disposition correspondent aux campagnes où la Haute
Autorité de Santé avait la maîtrise du cahier des charges de la campagne
du thème « Infections Associées aux Soins ».


\vspace{0.5cm}
\needspace{12\baselineskip}
\subsection*{Indicateurs de qualité et de sécurité des soins - Infections associées
aux soins - recueil 2018
}\index{dossier!patient}\index{ias}\index{qualite!des!soins}\index{sante}\index{securite!des!soins}
  \begin{wrapfigure}{r}{2.5cm}
    \centering
    \qrcode[nolink]{https://data.gouv.fr/dataset/5c59832e634f4152e06cd68a}
  \end{wrapfigure}

Licence : \textbf{Licence Ouverte version 2.0
}\newline
Créé le : 2019-02-05\newline
Modifié le : 2019-02-05\newline
De 2017-01-01 à 2017-12-31\newline
Mise à jour : annuelle\newline
Popularité : 1 réutilisation,  0 suivi\newline
Mots-clé : \emph{dossier-patient, ias, qualite-des-soins, sante, securite-des-soins
}\newline
Permalien : \url{https://data.gouv.fr/dataset/5c59832e634f4152e06cd68a}\newline

\par
\noindent
    La HAS développe et valide, avec les professionnels, des indicateurs de
qualité et de sécurité des soins permettant la comparaison
inter-établissements, utilisés par les établissements comme outils
d'amélioration de la qualité. Elle est engagée avec le ministère chargé
de la santé, depuis 2008, dans la mise en œuvre du recueil de ces
indicateurs.

Les indicateurs de qualité et de sécurité des soins concernés par cette
partie sont recueillis à partir d'un questionnaire
établissement.\url{https://www.has-sante.fr/portail/jcms/c_970481/fr/iqss-recueils-des-indicateurs-de-qualite-et-de-securite-des-soins}


\vspace{0.5cm}
\needspace{12\baselineskip}
\subsection*{Médecins accrédités par la HAS - 2017-2018
}\index{accreditation}\index{medecins}\index{risque}\index{sante}\index{sante!sociale}\index{soin}
  \begin{wrapfigure}{r}{2.5cm}
    \centering
    \qrcode[nolink]{https://data.gouv.fr/dataset/54940289c751df3c1904805b}
  \end{wrapfigure}

Licence : \textbf{Licence Ouverte
}\newline
Créé le : 2014-12-19\newline
Modifié le : 2019-01-30\newline
De 2017-01-01 à 2018-12-31\newline
Granularité : au pays\newline
Popularité : 4 réutilisations,  0 suivi\newline
Mots-clé : \emph{accreditation, medecins, risque, sante, sante-sociale, soin
}\newline
Permalien : \url{https://data.gouv.fr/dataset/54940289c751df3c1904805b}\newline

\par
\noindent
    En application de l'article L. 1414-3-2 du Code de la santé publique,
l'accréditation des médecins est une démarche volontaire. C'est une
démarche de gestion des risques fondée sur les programmes élaborés par
les organismes agréés pour l'accréditation (OA-Accréditation) dont
objectif est d'améliorer la qualité des pratiques professionnelles, de
réduire le nombre et de limiter les conséquences des événements
indésirables associés aux soins au bénéfice de la sécurité du patient.
Elle concerne le s médecins exerçant une spécialité ou une activité dite
«à risques» en établissement de santé décret n\degree{} 2006-909 du 21
juillet 2006. : gynécologie-obstétrique, anesthésie-réanimation,
chirurgie, spécialités interventionnelles, échographie obstétricale,
réanimation ou de soins intensifs. Pour plus d'information sur
l'accréditation
:\url{http://www.has-sante.fr/portail/jcms/c_428381/fr/mieux-connaitre-laccreditation}


\vspace{0.5cm}
\needspace{3\baselineskip} \rule{4cm}{0.25pt}\newline\textbf{Aussi disponible du même producteur :}\begin{itemize}
\item \href{https://data.gouv.fr/dataset/53bf9b88a3a7293e19309ef4}{Certification des établissements de santé 2016-2018}
\item \href{https://data.gouv.fr/dataset/5c472453634f411836452d44}{Certification des établissements de santé 2019}
\item \href{https://data.gouv.fr/dataset/54943ee0c751df114904805a}{Evaluation des dispositifs médicaux 2018}
\item \href{https://data.gouv.fr/dataset/5c3771e88b4c41227752c983}{Evaluation des dispositifs médicaux 2019}
\item \href{https://data.gouv.fr/dataset/5c376ffd8b4c4118c1b5efca}{Evaluation des médicaments 2018}
\item \href{https://data.gouv.fr/dataset/5a5df02688ee384ac8000430}{Indicateurs de qualité et de sécurité des soins - Infections associées aux soins - recueil 2017}
\item \href{https://data.gouv.fr/dataset/59492ebe88ee3849e2345d0b}{Indicateurs de qualité et de sécurité des soins - Mesure de la satisfaction (dispositif e-satis) - recueil 2016}
\item \href{https://data.gouv.fr/dataset/5a5dec2e88ee384519952fad}{Indicateurs de qualité et de sécurité des soins - Mesure de la satisfaction (dispositif e-satis) - recueil 2017}
\item \href{https://data.gouv.fr/dataset/5c598485634f415bb1e35b95}{Indicateurs de qualité et de sécurité des soins - Mesure de la satisfaction (dispositif e-Satis) - recueil 2018}
\item \href{https://data.gouv.fr/dataset/5a00405bc751df6d2deedc3d}{Indicateurs de qualité et de sécurité des soins - source dossier patient - recueil 2014}
\item \href{https://data.gouv.fr/dataset/5a005242c751df0419e3e6f1}{Indicateurs de qualité et de sécurité des soins - source dossier patient - recueil 2015}
\item \href{https://data.gouv.fr/dataset/5949316f88ee385372f8f456}{Indicateurs de qualité et de sécurité des soins - source dossier patient - recueil 2016}
\item \href{https://data.gouv.fr/dataset/5a5dee0588ee384519952fae}{Indicateurs de qualité et de sécurité des soins - source dossier patient - recueil 2017}
\item \href{https://data.gouv.fr/dataset/5c597f69634f414f578f495d}{Indicateurs de qualité et de sécurité des soins - source dossier patient - recueil 2018}
\item \href{https://data.gouv.fr/dataset/5493f0a8c751df0e0004805a}{Logiciels d'aide à la prescription certifiés}
\item \href{https://data.gouv.fr/dataset/5c5163fd8b4c410709285d70}{Médecins accrédités par la HAS 2019}
\end{itemize}

\clearpage
\section{Haute Autorité pour la diffusion des œuvres et la protection des droits sur internet (HADOPI)}


\begin{center}
  \includegraphics[width=3cm]{images/orga/35_1865a37a834c6f8ea65961a1d1c966-100.jpg}
\end{center}


La Haute Autorité pour la diffusion des œuvres et la protection des
droits sur internet (Hadopi), est une autorité publique indépendante
créée par la loi du 12 juin 2009.


\vspace{0.5cm}

\needspace{12\baselineskip}
\subsection*{Rapport annuel 2011-2012 de l'Hadopi : annexe indicateurs
}\index{culture}
  \begin{wrapfigure}{r}{2.5cm}
    \centering
    \qrcode[nolink]{https://data.gouv.fr/dataset/53699e93a3a729239d205eee}
  \end{wrapfigure}

Licence : \textbf{Licence Ouverte
}\newline
Créé le : 2013-07-08\newline
Modifié le : 2016-08-04\newline
Mise à jour : ponctuelle\newline
Popularité : 1 réutilisation,  1 suivi\newline
Mots-clé : \emph{culture
}\newline
Permalien : \url{https://data.gouv.fr/dataset/53699e93a3a729239d205eee}\newline

\par
\noindent
    Données issues d'études et utilisées dans le cadre des indicateurs que
la Haute Autorité suit.


\vspace{0.5cm}
\needspace{12\baselineskip}
\subsection*{Rapport annuel 2012-2013 de l'Hadopi : annexe indicateurs
}\index{culture}
  \begin{wrapfigure}{r}{2.5cm}
    \centering
    \qrcode[nolink]{https://data.gouv.fr/dataset/53699e93a3a729239d205eef}
  \end{wrapfigure}

Licence : \textbf{Licence Ouverte
}\newline
Créé le : 2013-10-20\newline
Modifié le : 2016-08-04\newline
Mise à jour : ponctuelle\newline
Popularité : 1 réutilisation,  0 suivi\newline
Mots-clé : \emph{culture
}\newline
Permalien : \url{https://data.gouv.fr/dataset/53699e93a3a729239d205eef}\newline

\par
\noindent
    Données issues d'études et utilisées dans le cadre des indicateurs que
la Haute Autorité suit.


\vspace{0.5cm}
\needspace{3\baselineskip} \rule{4cm}{0.25pt}\newline\textbf{Aussi disponible du même producteur :}\begin{itemize}
\item \href{https://data.gouv.fr/dataset/539a5d03a3a7293bc2728370}{Baromètre de l'offre légale de biens culturels dématérialisés}
\item \href{https://data.gouv.fr/dataset/59cb63c7c751df02bde82986}{Baromètre de l'offre légale de biens culturels dématérialisés - janvier 2017}
\item \href{https://data.gouv.fr/dataset/540aedc5a3a7297a99d24efa}{Baromètre de l'offre légale de biens culturels dématérialisés - mars 2014}
\item \href{https://data.gouv.fr/dataset/540aedc7a3a7297a99d24efb}{Baromètre de l'offre légale de biens culturels dématérialisés - septembre 2013}
\item \href{https://data.gouv.fr/dataset/59d20768c751df6eafcce5d3}{Baromètre usages biens culturels sur Internet - avril/mai 2017}
\item \href{https://data.gouv.fr/dataset/53698f48a3a729239d2036d4}{Baromètre usages biens culturels sur Internet - mai 2013}
\item \href{https://data.gouv.fr/dataset/540aedd1a3a7297a99d24efc}{Baromètre usages biens culturels sur Internet - mai 2014}
\item \href{https://data.gouv.fr/dataset/53698f49a3a729239d2036d5}{Baromètre usages biens culturels sur Internet - octobre 2012}
\item \href{https://data.gouv.fr/dataset/53698f49a3a729239d2036d6}{Baromètre usages biens culturels sur Internet - octobre 2013}
\item \href{https://data.gouv.fr/dataset/53698f91a3a729239d2037a3}{Biens culturels et usages d’internet : pratiques et perceptions des internautes français - janvier 2011}
\item \href{https://data.gouv.fr/dataset/53698f92a3a729239d2037a4}{Biens culturels et usages d’internet : pratiques et perceptions des internautes français - mai 2011}
\item \href{https://data.gouv.fr/dataset/59dce63188ee381933849ffe}{Étude des perceptions et usages du livre numérique - juin/juillet 2014}
\item \href{https://data.gouv.fr/dataset/5369950aa3a729239d2045fa}{Etude sur le jeu vidéo protégé - juillet 2013}
\item \href{https://data.gouv.fr/dataset/558abb1ec751df542ba453b9}{Etude sur les jeux vidéo - octobre 2014}
\item \href{https://data.gouv.fr/dataset/59cb6bc0c751df0bf0aafdea}{Étude sur les risques encourus sur les sites illicites - janvier/février 2017}
\item \href{https://data.gouv.fr/dataset/5ae0909cc751df0b8d8353c6}{Étude sur les stratégies d’accès aux œuvres dématérialisées - Mai 2018}
\item \href{https://data.gouv.fr/dataset/53699794a3a729239d204d1f}{Label PUR : données des plateformes labellisées, janvier 2013}
\item \href{https://data.gouv.fr/dataset/53699794a3a729239d204d20}{Label PUR : données des plateformes labellisées, juin 2013}
\item \href{https://data.gouv.fr/dataset/5447b9a5c751df61e45d0572}{Métadonnées offre VOD et SVOD}
\item \href{https://data.gouv.fr/dataset/53699b22a3a729239d20566d}{Offrelégale.fr : liste et description des plateformes publiées, décembre 2013}
\item \href{https://data.gouv.fr/dataset/539a6b68a3a7293bc272838b}{Offrelégale.fr : liste et description des plateformes publiées, juin 2014}
\item \href{https://data.gouv.fr/dataset/5b62c9ec634f412cde0ae7f2}{Tableau de suivi statistique de la réponse graduée}
\end{itemize}

\clearpage
\section{Haute Autorité pour la transparence de la vie publique}


\begin{center}
  \includegraphics[width=3cm]{images/orga/2015-01-15_ad713daf74694ab6a2c498d0d2500590_HATVP_logo-100.png}
\end{center}


Créée par les lois du 11 octobre 2013 relatives à la transparence de la
vie publique, la Haute Autorité pour la transparence de la vie publique
(HATVP) est une autorité administrative indépendante (AAI) chargée de
promouvoir la probité des responsables publics.

A ce titre, elle reçoit et contrôle les déclarations de situation
patrimoniale et d'intérêts des 8 000 plus hauts responsables publics
(membres du Gouvernement et du Parlement, grands élus locaux,
collaborateurs du président de la République, des ministres et des
présidents des assemblées ou dirigeants d'organismes publics).

Elle peut également être consultée par les élus sur des questions de
déontologie relatives à l'exercice de leur fonction et émettre des
recommandations à la demande du Premier ministre ou de sa propre
initiative sur toute question relative à la prévention des conflits
d'intérêts.

Elle publie un rapport annuel remis au président de la République, au
Premier ministre et au Parlement et peut formuler des recommandations
pour l'application de la législation en matière notamment de relations
avec les représentants d'intérêts.


\vspace{0.5cm}

\needspace{12\baselineskip}
\subsection*{Liste des déclarations et appréciations publiées par la Haute Autorité
de la vie publique
}\index{deputes}\index{elus}\index{gouvernement}\index{interets}\index{patrimoine}\index{senateurs}\index{transparence}
  \begin{wrapfigure}{r}{2.5cm}
    \centering
    \qrcode[nolink]{https://data.gouv.fr/dataset/53ae8c38a3a729709f56d510}
  \end{wrapfigure}

Licence : \textbf{Licence Ouverte
}\newline
Créé le : 2014-06-28\newline
Modifié le : 2016-02-10\newline
Granularité : au pays\newline
Mise à jour : ponctuelle\newline
Popularité : 1 réutilisation,  1 suivi\newline
Mots-clé : \emph{deputes, elus, gouvernement, interets, patrimoine, senateurs, transparence
}\newline
Permalien : \url{https://data.gouv.fr/dataset/53ae8c38a3a729709f56d510}\newline

\par
\noindent
    Les lois du 11 octobre 2013 relatives à la transparence de la vie
publique prévoient que les déclarations publiées par la Haute Autorité
pour la transparence de la vie publique sont librement réutilisables.
Afin de faciliter leur réutilisation, la Haute Autorité pour la
transparence de la vie publique met à la disposition du public, au
format CSV, la liste des déclarations et appréciations publiées sur son
site Internet, les déclarations n'étant pour l'instant disponibles qu'en
format PDF. Ces documents sont publiés sous la licence ouverte Etalab.


\vspace{0.5cm}
\needspace{3\baselineskip} \rule{4cm}{0.25pt}\newline\textbf{Aussi disponible du même producteur :}\begin{itemize}
\item \href{https://data.gouv.fr/dataset/5979b06088ee380e9896013c}{Contenu des déclarations publiées après le 1er juillet 2017, au format XML}
\item \href{https://data.gouv.fr/dataset/5a2adc8688ee382ed328b586}{Données du répertoire des représentants d'intérêts}
\end{itemize}

\clearpage
\section{IAU-ÎdF}


\begin{center}
  \includegraphics[width=3cm]{images/orga/3c_3bc8ca39b14c05a41938c65d4dfc2e-100.jpg}
\end{center}
\needspace{12\baselineskip}
\subsection*{Zones franches urbaines
}\index{vie!urbaine}
  \begin{wrapfigure}{r}{2.5cm}
    \centering
    \qrcode[nolink]{https://data.gouv.fr/dataset/5369a3a3a3a729239d206aea}
  \end{wrapfigure}

Licence : \textbf{Open Data Commons Open Database License (ODbL)
}\newline
Créé le : 2013-09-14\newline
Modifié le : 2016-01-16\newline
Popularité : 1 réutilisation,  0 suivi\newline
Mots-clé : \emph{vie-urbaine
}\newline
Permalien : \url{https://data.gouv.fr/dataset/5369a3a3a3a729239d206aea}\newline

\par
\noindent
    \begin{verbatim}
 Les ZFU intègrent le périmètre d'une ZRU, mais s'élargissent à des espaces limitrophes. Elles correspondent aux quartiers de plus de 10 000 habitants présentant les caractéristiques les plus dégradées en termes de chômage, de qualification professionnelle ou de ressources des communes concernées. </p> <p>  (Source : Périmètre régional d'intervention foncière) </p>
\end{verbatim}


\vspace{0.5cm}
\needspace{12\baselineskip}
\subsection*{ZUS, ZRU et ZFU
}\index{geolocalisation}\index{politique!de!la!ville}\index{population}\index{vie!urbaine}
  \begin{wrapfigure}{r}{2.5cm}
    \centering
    \qrcode[nolink]{https://data.gouv.fr/dataset/5369a3a8a3a729239d206af6}
  \end{wrapfigure}

Licence : \textbf{Open Data Commons Open Database License (ODbL)
}\newline
Créé le : 2013-09-14\newline
Modifié le : 2016-03-04\newline
Popularité : 1 réutilisation,  0 suivi\newline
Mots-clé : \emph{geolocalisation, politique-de-la-ville, population, vie-urbaine
}\newline
Permalien : \url{https://data.gouv.fr/dataset/5369a3a8a3a729239d206af6}\newline

\par
\noindent
    \begin{verbatim}
  Périmètres des "Zones Urbaines Sensibles" (ZUS), des "Zones de Redynamisation Urbaines" (ZRU) et des "Zones Franches Urbaines" (ZFU). </p> <p>      Cette géographie prioritaire a été mise en oeuvre par le "Pacte de relance pour la ville" (loi du 14 novembre 1996 mise en vigueur au 1er janvier 1997) selon le principe de la discrimination territoriale positive. Au moyen de mesures dérogatoires dans le domaine fiscal et social, l'objectif était de relancer l'activité économique et l'emploi dans les territoires sélectionnés. </p> <p>     Des critères majoritèrement statistiques définissent un zonage à trois niveaux emboîtés, au sein duquel est appliqué un ensemble de mesures hiérarchisées. </p> <p>     (Source : Périmètre régional d'intervention foncière) </p>
\end{verbatim}


\vspace{0.5cm}
\needspace{3\baselineskip} \rule{4cm}{0.25pt}\newline\textbf{Aussi disponible du même producteur :}\begin{itemize}
\item \href{https://data.gouv.fr/dataset/53698f5ca3a729239d20371a}{Bases de loisirs et de plein air régionales - emprise}
\item \href{https://data.gouv.fr/dataset/53698f5fa3a729239d203722}{Bassins versants élémentaires}
\item \href{https://data.gouv.fr/dataset/53699055a3a729239d20399b}{Ceinture verte : limite de la zone d'étude}
\item \href{https://data.gouv.fr/dataset/53699057a3a729239d2039a1}{Centres de formation des apprentis localisés à l'adresse du site de formation}
\item \href{https://data.gouv.fr/dataset/536991a8a3a729239d203cfb}{Construction neuve}
\item \href{https://data.gouv.fr/dataset/5369920aa3a729239d203dfe}{Crues de références}
\item \href{https://data.gouv.fr/dataset/5369920ba3a729239d203e00}{Crues de type 1910-1995-1970, crues exceptionnelles}
\item \href{https://data.gouv.fr/dataset/53699323a3a729239d2040e8}{Données communales sur la densité de population}
\item \href{https://data.gouv.fr/dataset/536994c1a3a729239d204517}{Espaces agricoles de la région île-de-France inscrits sur la CDGT du Sdrif arrêté en 2012}
\item \href{https://data.gouv.fr/dataset/536995ffa3a729239d20489e}{Grand Projet de renouvellement urbain (GPRU)}
\item \href{https://data.gouv.fr/dataset/536995ffa3a729239d2048a0}{Grand Projet de Ville (GPV)}
\item \href{https://data.gouv.fr/dataset/53699606a3a729239d2048af}{Groupements à fiscalité propre - les communes}
\item \href{https://data.gouv.fr/dataset/53699758a3a729239d204c83}{IUT (Instituts Universitaires de Technologie)}
\item \href{https://data.gouv.fr/dataset/536997e3a3a729239d204deb}{Le réseau routier magistral existant de la région île-de-France inscrit sur la CDGT du Sdrif arrêté en 2012}
\item \href{https://data.gouv.fr/dataset/536997e3a3a729239d204ded}{Le réseau routier principal existant de la région île-de-France inscrit sur la CDGT du Sdrif arrêté en 2012}
\item \href{https://data.gouv.fr/dataset/536997eda3a729239d204e06}{Les aérodromes de la région île-de-France inscrits au Sdrif arrêté en 2012}
\item \href{https://data.gouv.fr/dataset/536997f7a3a729239d204e20}{Les bois, espaces naturels et espaces verts et de loisirs existants de la région île-de-France inscrits sur la CDGT du Sdrif arrêté en 2012}
\item \href{https://data.gouv.fr/dataset/53699800a3a729239d204e38}{Les communes}
\item \href{https://data.gouv.fr/dataset/5369981ea3a729239d204e88}{Les écoles de formation supérieure non universitaires}
\item \href{https://data.gouv.fr/dataset/5369983aa3a729239d204ed4}{Les espaces d'urbanisation cartographiés de la région île-de-France inscrits du Sdrif arrêté en 2012}
\item \href{https://data.gouv.fr/dataset/5369983ba3a729239d204ed6}{Les espaces urbanisés de la région île-de-France localisés sur la CDGT du Sdrif arrêté en 2012}
\item \href{https://data.gouv.fr/dataset/5369983fa3a729239d204ee3}{Les établissements universitaires et établissements d'enseignement général supérieur privé}
\item \href{https://data.gouv.fr/dataset/53699846a3a729239d204ef5}{Les fronts urbains localisés sur la CDGT du Sdrif arrêté en 2012}
\item \href{https://data.gouv.fr/dataset/53699847a3a729239d204ef8}{Les grands aéroports de la région île-de-France localisés sur la CDGT du Sdrif arrêté en 2012}
\item \href{https://data.gouv.fr/dataset/5369984ea3a729239d204f0d}{Les liaisons vertes}
\item \href{https://data.gouv.fr/dataset/53699854a3a729239d204f1d}{Les lycées}
\item \href{https://data.gouv.fr/dataset/5369985aa3a729239d204f2a}{Les milieux naturels franciliens (ECOMOS)}
\item \href{https://data.gouv.fr/dataset/53699868a3a729239d204f4e}{Les périmètres (2km) autour des gares de la région île-de-France localisés sur la CDGT du Sdrif arrêté en 2012}
\item \href{https://data.gouv.fr/dataset/5369986aa3a729239d204f54}{Les polarités de la région île-de-France localisées sur la CDGT du Sdrif arrêté en 2012}
\item \href{https://data.gouv.fr/dataset/53699876a3a729239d204f7f}{Les projets d'espaces verts ou d'espaces de loisirs de la région île-de-France inscrits sur la CDGT du Sdrif arrêté en 2012}
\item \href{https://data.gouv.fr/dataset/53699877a3a729239d204f81}{Les projets de voies routières de la région île-de-France localisés sur la CDGT du Sdrif arrêté en 2012}
\item \href{https://data.gouv.fr/dataset/5369987ba3a729239d204f8c}{Les sections des continuités écologiques, coupures d'urbanisation, liaisons agricoles ou liaisons vertes localisées sur la CDGT du Sdrif arrêté en 2012}
\item \href{https://data.gouv.fr/dataset/5369987ea3a729239d204f95}{Les sites logistiques multimodaux de la région île-de-France localisés sur la CDGT du Sdrif arrêté en 2012}
\item \href{https://data.gouv.fr/dataset/53699881a3a729239d204f9b}{Les stations de chemin de fer existantes et/ou en projet de la région île-de-France localisées sur la CDGT du Sdrif arrêté en 2012}
\item \href{https://data.gouv.fr/dataset/53699887a3a729239d204fac}{Les transports en commun existants de la région Île-de-France de la région île-de-France inscrit sur la CDGT du Sdrif arrêté en 2012}
\item \href{https://data.gouv.fr/dataset/53699888a3a729239d204fb0}{Les unités paysagères}
\item \href{https://data.gouv.fr/dataset/53699b7fa3a729239d205754}{Parcs Naturels Régionaux (PNR)}
\item \href{https://data.gouv.fr/dataset/53699c10a3a729239d2058a9}{Pépinières d'entreprises}
\item \href{https://data.gouv.fr/dataset/53699d00a3a729239d205b0c}{Pôles logistique , les principales zones logistiques}
\item \href{https://data.gouv.fr/dataset/53699d1aa3a729239d205b56}{Population active selon les catégories socio professionelles}
\item \href{https://data.gouv.fr/dataset/53699e66a3a729239d205e83}{Projets de transport de la région île-de-France inscrits sur la CDGT du Sdrif arrêté en 2012}
\item \href{https://data.gouv.fr/dataset/53699f0ea3a729239d206033}{Réseau des continuités écologiques aquatiques pour les espèces plus inféodées aux berges}
\item \href{https://data.gouv.fr/dataset/53699f11a3a729239d206040}{réseau régional schématique des connexions écologiques de zones humides}
\item \href{https://data.gouv.fr/dataset/53699fb6a3a729239d2061c4}{Schéma des continuités écologiques relatives au déplacement de la grande faune}
\item \href{https://data.gouv.fr/dataset/53699fb8a3a729239d2061ca}{Schéma régional de synthèse des connexions écologiques}
\item \href{https://data.gouv.fr/dataset/53699fb9a3a729239d2061cc}{Schéma régional de synthèse des connexions écologiques}
\item \href{https://data.gouv.fr/dataset/5369a18ea3a729239d206628}{Taux de logements sociaux (SRU)}
\item \href{https://data.gouv.fr/dataset/5369a259a3a729239d206809}{Trame régionale des continuités herbacées}
\item \href{https://data.gouv.fr/dataset/5369a259a3a729239d20680b}{Trame régionale des corridors boisés}
\item \href{https://data.gouv.fr/dataset/5369a312a3a729239d206999}{Type de franchissement étudié conjointement aux liaisons vertes.}
\item et 0 autres jeux de données\end{itemize}

\clearpage
\section{Images \&{} Réseaux}


\begin{center}
  \includegraphics[width=3cm]{images/orga/17_f3a9357fae402eab0a5e26204ec7ee-100.jpg}
\end{center}


Labellisé en 2005 pôle de compétitivité à vocation mondiale, Images \&
Réseaux accompagne les porteurs d'innovation dans leurs projets et
active la rencontre des innovateurs, des industriels et des financeurs
pour porter collectivement des technologies, des usages et des marchés.
Images \& Réseaux a élaboré sa feuille de route stratégique en cohérence
avec ses grands objectifs marchés, et les axes de recherche et de
développement souhaités pour accompagner ses membres dans leur
développement et en cohérence avec les stratégies régionales
d'innovation et de spécialisation intelligente élaborée par ses deux
régions d'appartenance. Ces objectifs ambitieux se déclinent aussi grâce
à un réseau fort sur le territoire de la Bretagne, des Pays de la Loire,
de partenaires du Pôle, les professionnels au service du développement
économique régional et au-delà via l'implication des adhérents
(industriels, TPE, PME et Grands Groupes, mais aussi les acteurs de la
recherche, de la formation)


\vspace{0.5cm}

\needspace{12\baselineskip}
\subsection*{Projets de R\&D labellisés par Images \& Réseaux
}\index{collaboratif}\index{innovation}\index{numerique}\index{pole!de!competitivite}\index{projets}\index{recherche!et!developpement}
  \begin{wrapfigure}{r}{2.5cm}
    \centering
    \qrcode[nolink]{https://data.gouv.fr/dataset/5447bbe1c751df546b5d0572}
  \end{wrapfigure}

Licence : \textbf{Licence Ouverte
}\newline
Créé le : 2014-10-22\newline
Modifié le : 2015-10-03\newline
De 2005-01-01 à 2014-10-22\newline
Granularité : à la région\newline
Mise à jour : hebdomadaire\newline
Popularité : 1 réutilisation,  0 suivi\newline
Mots-clé : \emph{collaboratif, innovation, numerique, pole-de-competitivite, projets, recherche-et-developpement
}\newline
Permalien : \url{https://data.gouv.fr/dataset/5447bbe1c751df546b5d0572}\newline

\par
\noindent
    Les projets collaboratifs de R\&D sont, depuis quasiment 10 ans, au cœur
des missions des pôles de compétitivité. Et aujourd'hui encore plus,
alors que nous entamons le virage de la pahse 3.0 des pôles, avec une
volonté qui anime tout l'écosystème de l'innovation : transformer la
R\&D en produit et réussir leur mise sur le marché. Pour la première
fois depuis sa création, Images \& Réseaux se concentre sur les projets
collaboratifs innovants labellisés et terminés au 31 décembre 2013 et
sort son premier livre des retombées. Côté chiffres on parle de 556
projets collaboratifs innovants labellisés depuis sa création et 260
projets financés 688 M\euro{} investis en R\&D depuis 2006

Le jeu de données vous propose une vision sur les projets de R\&D
labellisés et financés depuis la création du pôle avec leurs appels à
projets, les thématiques concernés, \ldots{}


\vspace{0.5cm}
\needspace{3\baselineskip} \rule{4cm}{0.25pt}\newline\textbf{Aussi disponible du même producteur :}\begin{itemize}
\item \href{https://data.gouv.fr/dataset/5447ba21c751df547d5d0572}{Membres Images \&{} Réseaux}
\end{itemize}

\clearpage
\section{Incubateur de Services Numériques}


\begin{center}
  \includegraphics[width=3cm]{images/orga/c4_56bb28fe944bba98bb7b778071fa1f-100.png}
\end{center}


Découvrez nos réalisations sur \href{http://beta.gouv.fr}{beta.gouv.fr}
!


\vspace{0.5cm}

\needspace{12\baselineskip}
\subsection*{Caractéristiques techniques des sites
}\index{securite}\index{sites}\index{sites!web}
  \begin{wrapfigure}{r}{2.5cm}
    \centering
    \qrcode[nolink]{https://data.gouv.fr/dataset/5805f1e2c751df2bb879df72}
  \end{wrapfigure}

Licence : \textbf{Licence Ouverte
}\newline
Créé le : 2016-10-18\newline
Modifié le : 2018-03-13\newline
Mise à jour : quotienne\newline
Popularité : 1 réutilisation,  1 suivi\newline
Mots-clé : \emph{securite, sites, sites-web
}\newline
Permalien : \url{https://data.gouv.fr/dataset/5805f1e2c751df2bb879df72}\newline

\par
\noindent
    Ces données permettent d'estimer la sécurité et la performance des sites
web de l'administration. Il est utilisé comme support pour
\href{https://verif.site/}{https://verif.site}{]}(https://verif.site{]}(https://verif.site/))pour
suivre les services proposés par \href{http://beta.gouv.fr}{l'incubateur
de services numériques}.


\vspace{0.5cm}
\needspace{3\baselineskip} \rule{4cm}{0.25pt}\newline\textbf{Aussi disponible du même producteur :}\begin{itemize}
\item \href{https://data.gouv.fr/dataset/5c7929508b4c4150ff5452bc}{Améliorer la prise de rendez-vous dans les services sociaux des départements - Appel à manifestation d'intérêt}
\item \href{https://data.gouv.fr/dataset/5b5ec5c188ee3842dbe6ba19}{Conventions de partenariat}
\item \href{https://data.gouv.fr/dataset/5b6952a88b4c4119b83b67f7}{Dossier de consultation des entreprises pour l'accompagnement de la DINSIC dans le développement et le design de services publics numériques en mode agile}
\item \href{https://data.gouv.fr/dataset/5b71a9ea8b4c414c31b9a5bc}{Dossier de consultation des entreprises pour l'accompagnement du SGMAP dans le déploiement de Start-up d’État et la réalisation de développements de services numériques en mode Agile}
\item \href{https://data.gouv.fr/dataset/5bcdf23f634f410181adf541}{Test d'intégration pour l'enrichissement de l'annuaire de service-public.fr}
\end{itemize}

\clearpage
\section{Infogreffe }


\begin{center}
  \includegraphics[width=3cm]{images/orga/46_ccf8461acc41bfad4be5d5bc6a9767-100.jpg}
\end{center}


Créé il y a plus de 30 ans, Infogreffe est un service public qui permet
aux greffiers des Tribunaux de commerce d'assurer la plus large
diffusion de l'information légale des entreprises. Cette diffusion des
actes officiels qui certifient l'activité économique des entreprises,
quelle que soit leur taille, est essentielle pour la sécurité des
affaires et du commerce. C'est aussi un modèle reconnu en Europe.

Infogreffe a récemment mis en ligne le portail www.datainfogreffe.fr,
qui a pour objectif d'augmenter le potentiel de création économique des
entreprises, en mettant à leur disposition des données dans des formats
ouverts et facilement réutilisables.


\vspace{0.5cm}

\needspace{12\baselineskip}
\subsection*{Chiffres Clés 2013/2012/2011
}\index{chiffre!daffaires}\index{chiffres!cles}\index{comptes!annuels}\index{comptes!sociaux}\index{effectif}\index{resultat!dexercice}
  \begin{wrapfigure}{r}{2.5cm}
    \centering
    \qrcode[nolink]{https://data.gouv.fr/dataset/5624b5a3b5950871ed4bbb3a}
  \end{wrapfigure}

Licence : \textbf{Licence Ouverte
}\newline
Créé le : 2015-10-19\newline
Modifié le : 2016-04-10\newline
Popularité : 1 réutilisation,  0 suivi\newline
Mots-clé : \emph{chiffre-daffaires, chiffres-cles, comptes-annuels, comptes-sociaux, effectif, resultat-dexercice
}\newline
Permalien : \url{https://data.gouv.fr/dataset/5624b5a3b5950871ed4bbb3a}\newline

\par
\noindent
    Les chiffres clés des sociétés commerciales~ayant déposé leurs comptes
annuels pour les exercices 2011,~2012 et~2013.


\vspace{0.5cm}
\needspace{12\baselineskip}
\subsection*{Chiffres Clés -- 2014/2013/2012
}\index{chiffre!daffaires}\index{chiffres!cles}\index{comptes!annuels}\index{comptes!sociaux}\index{effectif}\index{resultat!dexercice}
  \begin{wrapfigure}{r}{2.5cm}
    \centering
    \qrcode[nolink]{https://data.gouv.fr/dataset/57092f4ea3a72944fcc6ab3d}
  \end{wrapfigure}

Licence : \textbf{Licence Ouverte
}\newline
Créé le : 2016-04-09\newline
Modifié le : 2018-03-08\newline
Popularité : 2 réutilisations,  1 suivi\newline
Mots-clé : \emph{chiffre-daffaires, chiffres-cles, comptes-annuels, comptes-sociaux, effectif, resultat-dexercice
}\newline
Permalien : \url{https://data.gouv.fr/dataset/57092f4ea3a72944fcc6ab3d}\newline

\par
\noindent
    Les chiffres clés des sociétés commerciales~ayant déposé leurs comptes
annuels pour les exercices 2012,~2013 et 2014


\vspace{0.5cm}
\needspace{12\baselineskip}
\subsection*{Entreprises immatriculées en 2016
}\index{creation}\index{entreprises}\index{immatriculation}\index{immatriculations}\index{societes}
  \begin{wrapfigure}{r}{2.5cm}
    \centering
    \qrcode[nolink]{https://data.gouv.fr/dataset/57092f4aa3a72944fcc6ab37}
  \end{wrapfigure}

Licence : \textbf{Licence Ouverte
}\newline
Créé le : 2016-04-09\newline
Modifié le : 2016-04-10\newline
Popularité : 2 réutilisations,  1 suivi\newline
Mots-clé : \emph{creation, entreprises, immatriculation, immatriculations, societes
}\newline
Permalien : \url{https://data.gouv.fr/dataset/57092f4aa3a72944fcc6ab37}\newline

\par
\noindent
    \textbf{RCS - Liste des entreprises immatriculées en 2016}

\begin{center}\rule{0.5\linewidth}{\linethickness}\end{center}

Liste des sociétés commerciales~immatriculées au registre du commerce et
des sociétés en~2016.


\vspace{0.5cm}
\needspace{12\baselineskip}
\subsection*{Statistiques Immatriculations en 2015
}\index{greffes}\index{immatriculations}\index{statistiques}
  \begin{wrapfigure}{r}{2.5cm}
    \centering
    \qrcode[nolink]{https://data.gouv.fr/dataset/5624ae8aa3a72922e4a9a7fd}
  \end{wrapfigure}

Licence : \textbf{Licence Ouverte
}\newline
Créé le : 2015-10-19\newline
Modifié le : 2016-04-10\newline
De 2015-01-01 à 2015-08-31\newline
Granularité : au point d'intérêt\newline
Popularité : 1 réutilisation,  0 suivi\newline
Mots-clé : \emph{greffes, immatriculations, statistiques
}\newline
Permalien : \url{https://data.gouv.fr/dataset/5624ae8aa3a72922e4a9a7fd}\newline

\par
\noindent
    \textbf{RCS -~Statistiques des immatriculations en 2015}

\begin{center}\rule{0.5\linewidth}{\linethickness}\end{center}

Il s'agit du nombre de formalités de création d'entreprises
individuelles (commerçants)~et de sociétés commerciales enregistrées au
registre du commerce et des sociétés en 2015


\vspace{0.5cm}
\needspace{3\baselineskip} \rule{4cm}{0.25pt}\newline\textbf{Aussi disponible du même producteur :}\begin{itemize}
\item \href{https://data.gouv.fr/dataset/5624ae8ab595080b1989caa3}{Statistiques Jugements de procédures collectives 2015}
\item \href{https://data.gouv.fr/dataset/5624b5a3a3a72928ca32aa4c}{Statistiques Radiations en 2015}
\end{itemize}

\clearpage
\section{Institut français de recherche pour l'exploitation de la mer (IFREMER)}


\begin{center}
  \includegraphics[width=3cm]{images/orga/2015-03-02_d03d3b7d7ad244c09f9eff7f2cab5d55_couleur-100.png}
\end{center}


L'Ifremer contribue, par ses travaux et expertises, à la connaissance
des océans et de leurs ressources, à la surveillance du milieu marin et
du littoral et au développement durable des activités maritimes. À ces
fins, il conçoit et met en œuvre des outils d'observation,
d'expérimentation et de surveillance, et gère des bases de données
océanographiques.

Créé en 1984, l'Ifremer est un établissement public à caractère
industriel et commercial (EPIC), placé sous la tutelle conjointe des
ministères de l'Enseignement supérieur et de la Recherche et de
l'Écologie, du Développement durable et de l'Énergie.

L'Ifremer travaille en réseau avec la communauté scientifique française,
mais aussi des organismes partenaires dans de nombreux pays. La
coopération est centrée sur des grands programmes internationaux, sur
l'Outre-mer et sur quelques pays-cibles (Etats-Unis, Canada, Japon,
Chine, Australie, Russie), et sur une politique méditerranéenne
associant l'Europe à la rive Sud de la Méditerranée.

\emph{Des missions de recherche, d'expertise et d'agence de moyens }:

\begin{itemize}

\item
  Une recherche finalisée afin de répondre aux questions sociétales
  actuelles (effets du changement climatique, biodiversité marine,
  prévention des pollutions, qualité des produits de la mer\ldots{}).
\item
  La surveillance des mers et du littoral, en soutien à la politique
  publique de gestion du milieu et des ressources.
\item
  Le développement, la gestion et la mise à disposition de grandes
  infrastructures de recherche (flotte, moyens de calcul, centre de
  données, moyens d'essais, structures expérimentales) à la disposition
  de la communauté scientifique nationale et européenne.
\end{itemize}


\vspace{0.5cm}

\needspace{12\baselineskip}
\subsection*{Cartographie des principaux peuplements sublittoraux du secteur de
Plogoff (Source Belsher T., Hamon D., Guillaumont B., 1980-1981) -
Echelle 1/25000
}\index{benthos}\index{biota}\index{bretagne}\index{donnees!ouvertes}\index{golfe!de!gascogne}\index{habitat}\index{milieu!biologique!habitats}\index{open!data}\index{passerelle!inspire}\index{peuplement!biologique}\index{plogoff}\index{recherche}\index{repartition!des!especes}
  \begin{wrapfigure}{r}{2.5cm}
    \centering
    \qrcode[nolink]{https://data.gouv.fr/dataset/566a8c3ac751df077ec664c0}
  \end{wrapfigure}

Licence : \textbf{Licence Ouverte version 2.0
}\newline
Créé le : 2015-12-11\newline
Modifié le : 2019-02-08\newline
Popularité : 1 réutilisation,  0 suivi\newline
Mots-clé : \emph{benthos, biota, bretagne, donnees-ouvertes, golfe-de-gascogne, habitat, milieu-biologique-habitats, open-data, passerelle-inspire, peuplement-biologique, plogoff, recherche, repartition-des-especes
}\newline
Permalien : \url{https://data.gouv.fr/dataset/566a8c3ac751df077ec664c0}\newline

\par
\noindent
    Cette cartographie numérique des principaux peuplements sublittoraux du
secteur de Plogoff a été réalisée en 2010 d'après l'étude menée à
l'Ifremer par Belsher T., Hamon D. et Guillaumont G. et publiée en 1987.
Pour cette étude, des photographies sous-marines associées à des
prélèvements par plongées ont permis de caractériser et de quantifier
les principales composantes biosédimentaires. Ces prospections terrain
ont été réalisées entre juin 1979 et juin 1980. Neuf ensembles
biosédimentaires ont pu être distingués. Cette carte a fait l'objet en
2010 d'une numérisation dans le cadre du SINP Mer (Système d'Information
sur la Nature et les Paysages, volet Mer), avec le soutien du REBENT
(REseau de surveillance de la flore et de la faune BENThique marine).

\textbf{Origine}

SOURCE : - Belsher T., Hamon D., Guillaumont B., 1987, Phytobenthos et
zoobenthos in Etude écologique de projet, site de Plogoff, février
1980-juin 1981. Volume 2 : le domaine benthique. Rapport Ifremer pour
EDF : DERO-87.04-EL, 196 p.

\textbf{Organisations partenaires}

Ifremer

\textbf{Liens annexes}

\begin{itemize}

\item
  \href{http://www.rebent.org/cartographie/index.php}{index.php}
\item
  \href{http://www.rebent.org/fr/documentation/conditions-d-utilisation-des-documents.php}{conditions-d-utilisation-des-documents.php}
\item
  \href{http://www.searchmesh.net/Default.aspx?page=1616}{Default.aspx}
\item
  \href{http://www.rebent.org/fr/le-rebent/projet-europeen-mesh/guide-mesh.php}{guide-mesh.php}
\end{itemize}

➞
\href{https://geo.data.gouv.fr/fr/datasets/6dbbcebb28cf0a562f75f6b9172884db55454c61}{Consulter
cette fiche sur geo.data.gouv.fr}


\vspace{0.5cm}
\needspace{3\baselineskip} \rule{4cm}{0.25pt}\newline\textbf{Aussi disponible du même producteur :}\begin{itemize}
\item \href{https://data.gouv.fr/dataset/595e3b75c751df2b857f9fb2}{Abondance d'arnoglosses (Arnoglossus sp.) observée lors des campagnes scientifiques sur la période 2005 – 2010}
\item \href{https://data.gouv.fr/dataset/595e3b85c751df2620c36254}{Abondance de bar (Dicentrarchus labrax) observée lors des campagnes scientifiques sur la période 2005 – 2009}
\item \href{https://data.gouv.fr/dataset/595e3b7dc751df2b857f9fb7}{Abondance de baudroies (Lophius sp.) observée lors des campagnes scientifiques sur la période 2005 – 2010}
\item \href{https://data.gouv.fr/dataset/595e3b7b88ee381a30976f04}{Abondance de bulot (Buccinum undatum) observée lors des campagnes scientifiques sur la période 2005 – 2010}
\item \href{https://data.gouv.fr/dataset/595e3b77c751df25ee79baed}{Abondance de callionymes (Callionymus sp.) observée lors des campagnes scientifiques sur la période 2005 – 2010}
\item \href{https://data.gouv.fr/dataset/595e3b7988ee381af10e9af4}{Abondance de cardines (Lepidorhombus sp.) observée lors des campagnes scientifiques sur la période 2006 – 2009}
\item \href{https://data.gouv.fr/dataset/595e3b83c751df2b857f9fbb}{Abondance de céteau (Dicologlossa cuneata) observée lors des campagnes scientifiques sur la période 2007 – 2010}
\item \href{https://data.gouv.fr/dataset/595e3b79c751df2b857f9fb4}{Abondance de coquille Saint-Jacques (Pecten maximus) observée lors des campagnes scientifiques sur la période 2005 – 2011}
\item \href{https://data.gouv.fr/dataset/595e3b7cc751df2b857f9fb6}{Abondance de galathée (Munida bamffia) observée lors des campagnes scientifiques sur la période 2006 – 2010}
\item \href{https://data.gouv.fr/dataset/595e3b7d88ee381af10e9af6}{Abondance d'églefin (Melanogrammus aeglefinus) observée lors des campagnes scientifiques sur la période 2005 – 2010}
\item \href{https://data.gouv.fr/dataset/595e3b7e88ee3812b934cfc3}{Abondance de grande roussette (Scyliorhinus stellaris) observée lors des campagnes scientifiques sur la période 2005 – 2009}
\item \href{https://data.gouv.fr/dataset/595e3b79c751df25ee79baee}{Abondance de grande vive (Trachinus draco) observée lors des campagnes scientifiques sur la période 2005 – 2010}
\item \href{https://data.gouv.fr/dataset/595e3b7a88ee381af10e9af5}{Abondance de griset (Spondyliosoma cantharus) observée lors des campagnes scientifiques sur la période 2005 – 2009}
\item \href{https://data.gouv.fr/dataset/595e3b89c751df25ee79baf0}{Abondance de grondins (Triglidae) observée lors des campagnes scientifiques sur la période 2005 – 2010}
\item \href{https://data.gouv.fr/dataset/595e3b88c751df2b9f48439b}{Abondance de juvéniles de barbue (Scophthalmus rhombus) observée lors des campagnes scientifiques sur la période 2005 – 2009}
\item \href{https://data.gouv.fr/dataset/595e3b84c751df2b986b6444}{Abondance de juvéniles de bar (Dicentrarchus labrax) observée lors des campagnes scientifiques sur la période 1998 – 2009}
\item \href{https://data.gouv.fr/dataset/595e3b7688ee381af10e9af2}{Abondance de juvéniles de céteau (Dicologlossa cuneata) observée lors des campagnes scientifiques sur la période 2000 – 2009}
\item \href{https://data.gouv.fr/dataset/595e3b7588ee3812b934cfc0}{Abondance de juvéniles de flet (Platichthys fletus) observée lors des campagnes scientifiques sur la période 1998 – 2009}
\item \href{https://data.gouv.fr/dataset/595e3b7688ee3812b934cfc1}{Abondance de juvéniles de griset (Spondyliosoma cantharus) observée lors des campagnes scientifiques sur la période 2000 – 2009}
\item \href{https://data.gouv.fr/dataset/595e3b7bc751df2b857f9fb5}{Abondance de juvéniles de langoustine (Nephrops norvegicus) observée lors des campagnes scientifiques sur la période 2006 – 2009}
\item \href{https://data.gouv.fr/dataset/595e3b8f88ee3812b934cfc5}{Abondance de juvéniles de maigre (Argyrosomus regius) observée lors des campagnes scientifiques sur la période 2000 – 2006}
\item \href{https://data.gouv.fr/dataset/595e3b79c751df2620c36253}{Abondance de juvéniles de merlan (Merlangius merlangus) observée lors des campagnes scientifiques sur la période 1998 – 2009}
\item \href{https://data.gouv.fr/dataset/595e3b8d88ee381af10e9af8}{Abondance de juvéniles de merlu (Merluccius merluccius) observée lors des campagnes scientifiques sur la période 2000 – 2009}
\item \href{https://data.gouv.fr/dataset/595e3b7fc751df2b857f9fb8}{Abondance de juvéniles de plie (Pleuronectes platessa) observée lors des campagnes scientifiques sur la période 1998 – 2009}
\item \href{https://data.gouv.fr/dataset/595e3b7f88ee381af10e9af7}{Abondance de juvéniles de rouget barbet (Mullus surmuletus) observée lors des campagnes scientifiques sur la période 2000 – 2009}
\item \href{https://data.gouv.fr/dataset/595e3b8ec751df2b986b6447}{Abondance de juvéniles de sole commune (Solea solea) observée lors des campagnes scientifiques sur la période 1998 – 2009}
\item \href{https://data.gouv.fr/dataset/595e3ba3c751df2620c36258}{Abondance de juvéniles de tacaud (Trisopterus luscus) observée lors des campagnes scientifiques sur la période 1998 – 2009}
\item \href{https://data.gouv.fr/dataset/595e3b87c751df2b857f9fbd}{Abondance de juvéniles de turbot (Psetta maxima) observée lors des campagnes scientifiques sur la période 2005 – 2009}
\item \href{https://data.gouv.fr/dataset/595e3b90c751df25ee79baf1}{Abondance de langoustine (Nephrops norvegicus) observée lors des campagnes scientifiques sur la période 2006 – 2010}
\item \href{https://data.gouv.fr/dataset/595e3b88c751df2b857f9fbe}{Abondance de limandes (Limanda sp.) observée lors des campagnes scientifiques sur la période 2005 – 2009}
\item \href{https://data.gouv.fr/dataset/595e3bb3c751df2b986b644c}{Abondance de maigre (Argyrosomus regius) observée lors des campagnes scientifiques sur la période 2005 – 2009}
\item \href{https://data.gouv.fr/dataset/595e3b8fc751df2620c36256}{Abondance de merlan bleu (Micromesistius poutassou) observée lors des campagnes scientifiques sur la période 2005 – 2010}
\item \href{https://data.gouv.fr/dataset/595e3b86c751df2b9f48439a}{Abondance de merlan (Merlangius merlangus) observée lors des campagnes scientifiques sur la période 2005 – 2010}
\item \href{https://data.gouv.fr/dataset/595e3b9b88ee381a30976f08}{Abondance de merlu (Merluccius merluccius) observée lors des campagnes scientifiques sur la période 2005 – 2010}
\item \href{https://data.gouv.fr/dataset/595e3b9988ee381af10e9af9}{Abondance de morue (Gadus morhua) observée lors des campagnes scientifiques sur la période 2005 – 2009}
\item \href{https://data.gouv.fr/dataset/595e3b8e88ee3816907771e2}{Abondance de motelles (Lotinae) observée lors des campagnes scientifiques sur la période 2005 – 2010}
\item \href{https://data.gouv.fr/dataset/595e3b93c751df2b9f48439c}{Abondance de petite roussette (Scyliorhinus canicula) observée lors des campagnes scientifiques sur la période 2005 – 2010}
\item \href{https://data.gouv.fr/dataset/595e3b8d88ee381a30976f06}{Abondance de petite sole jaune (Buglossidium luteum) observée lors des campagnes scientifiques sur la période 2005 – 2010}
\item \href{https://data.gouv.fr/dataset/595e3ba6c751df2620c36259}{Abondance de petite vive (Echiichthys vipera) observée lors des campagnes scientifiques sur la période 2005 – 2010}
\item \href{https://data.gouv.fr/dataset/595e3b9a88ee381a30976f07}{Abondance de petit tacaud (Trisopterus minutus) observée lors des campagnes scientifiques sur la période 2005 – 2010}
\item \href{https://data.gouv.fr/dataset/595e3b9d88ee3816907771e4}{Abondance de pétoncle noir (Aequipecten opercularis) observée lors des campagnes scientifiques sur la période 2006 – 2010}
\item \href{https://data.gouv.fr/dataset/595e3ba2c751df2620c36257}{Abondance de plie (Pleuronectes platessa) observée lors des campagnes scientifiques sur la période 2005 – 2010}
\item \href{https://data.gouv.fr/dataset/595e3b9a88ee3816907771e3}{Abondance de poulpe blanc (Eledone cirrhosa) observée lors des campagnes scientifiques sur la période 2005 – 2010}
\item \href{https://data.gouv.fr/dataset/595e3b90c751df2b857f9fbf}{Abondance de raie bouclée (Raja clavata) observée lors des campagnes scientifiques sur la période 2005 – 2009}
\item \href{https://data.gouv.fr/dataset/595e3bb1c751df2b986b644b}{Abondance de raie douce (Raja montagui) observée lors des campagnes scientifiques sur la période 2005 – 2010}
\item \href{https://data.gouv.fr/dataset/595e3b92c751df2b857f9fc0}{Abondance de raie fleurie (Leucoraja naevus) observée lors des campagnes scientifiques sur la période 2005 – 2009}
\item \href{https://data.gouv.fr/dataset/595e3b9b88ee3812b934cfc6}{Abondance de rouget barbet (Mullus surmuletus) observée lors des campagnes scientifiques sur la période 2005 – 2010}
\item \href{https://data.gouv.fr/dataset/595e3b9d88ee381af10e9afa}{Abondance de saint-pierre (Zeus faber) observée lors des campagnes scientifiques sur la période 2005 – 2010}
\item \href{https://data.gouv.fr/dataset/595e3ba2c751df25ee79baf2}{Abondance de seiche commune (Sepia officinalis) observée lors des campagnes scientifiques sur la période 2005 – 2010}
\item \href{https://data.gouv.fr/dataset/595e3ba888ee3812b934cfc7}{Abondance de seiche élégante (Sepia elegansis) observée lors des campagnes scientifiques sur la période 2005 – 2010}
\item et 196 autres jeux de données\end{itemize}

\clearpage
\section{Institut français du cheval et de l'équitation (IFCE)}


\begin{center}
  \includegraphics[width=3cm]{images/orga/13_874a211e4e4a6992e6ae424000cb15-100.jpg}
\end{center}


L'Institut français du cheval et de l'équitation, établissement public à
caractère administratif placé sous la tutelle des ministères chargés des
sports et de l'agriculture, est l'opérateur public unique pour
accompagner la professionnalisation de la filière équine. Ses actions,
proposées sous ses marques Haras nationaux et Cadre noir - Saumur,
s'exercent au profit de la profession, des collectivités territoriales,
de l'Etat et de tous les publics concernés par le cheval et
l'équitation. Elles se déploient sur tout le territoire dans un contexte
d'engouement du public pour le cheval.


\vspace{0.5cm}

\needspace{12\baselineskip}
\subsection*{Fichier des équidés
}\index{cheval}
  \begin{wrapfigure}{r}{2.5cm}
    \centering
    \qrcode[nolink]{https://data.gouv.fr/dataset/547ca9d7c751df6d344484b3}
  \end{wrapfigure}

Licence : \textbf{Licence Ouverte
}\newline
Créé le : 2014-12-01\newline
Modifié le : 2018-12-05\newline
Mise à jour : annuelle\newline
Popularité : 1 réutilisation,  3 suivis\newline
Mots-clé : \emph{cheval
}\newline
Permalien : \url{https://data.gouv.fr/dataset/547ca9d7c751df6d344484b3}\newline

\par
\noindent
    Le département SIRE (Système d'Information Répertoriant les Equidés) de
l'Institut Français du Cheval et de l'Equitation produit un fichier
contenant les données publiques des équidés immatriculés en France
depuis 1976 soit environ 3 millions de lignes.

Structure du fichier:

Race du cheval (15 caractères alphabétiques maxi) : Libellé de la race
du cheval

Sexe du cheval (1 caractère alphabétique) : M: Mâle, F:Femelle, H~:
Hongre

Robe du cheval (25 caractères alphabétiques maxi) : Libellé de la robe
du cheval

Date de naissance du cheval (8 caractères alphabétiques JJ/MM/AAAA)

Pays de naissance du cheval (15 caractères alphabétiques maxi) : Libellé
du pays de naissance

Nom du cheval (25 caractères alphabétiques maxi)

Destiné à la consommation humaine (1 caractère alphabétique) : O: Oui,
N: Non

Date de mort du cheval (8 caractères alphabétiques JJ/MM/AAAA)


\vspace{0.5cm}
\needspace{3\baselineskip} \rule{4cm}{0.25pt}\newline\textbf{Aussi disponible du même producteur :}\begin{itemize}
\item \href{https://data.gouv.fr/dataset/53699a7da3a729239d2054da}{Nombre de juments saillies en France}
\item \href{https://data.gouv.fr/dataset/53699a8aa3a729239d2054f7}{Nombre de naissances d'équidés en France}
\item \href{https://data.gouv.fr/dataset/53699ac4a3a729239d205587}{Nombre d'étalons en activité en France}
\end{itemize}

\clearpage
\section{Institut national de la jeunesse et de l’éducation populaire}


\begin{center}
  \includegraphics[width=3cm]{images/orga/fb_64d9fbda6b43d6b26903eb4113a3d3-100.jpg}
\end{center}


L'Institut national de la jeunesse et de l'éducation populaire (INJEP)
est un service à compétence nationale du ministère de l'Éducation
nationale.

Observatoire producteur de connaissances, l'Institut national de la
jeunesse et de l'éducation populaire (INJEP) est un centre de ressources
et d'expertise sur les questions de jeunesse et les politiques qui lui
sont dédiées, sur l'éducation populaire, la vie associative et le sport.

Sa mission : contribuer à améliorer la connaissance dans ces domaines
par la production de statistiques et d'analyses, l'observation,
l'expérimentation et l'évaluation. Son ambition : partager cette
connaissance avec tous les acteurs et éclairer la décision publique.


\vspace{0.5cm}

\needspace{12\baselineskip}
\subsection*{Données géocodées issues du recensement des licences et clubs auprès des
fédérations sportives agréées par le ministère chargé des sports
}\index{club}\index{clubs}\index{federation}\index{federations}\index{licence}\index{licences}\index{recensement}\index{sport}\index{sportifs}\index{sportives}
  \begin{wrapfigure}{r}{2.5cm}
    \centering
    \qrcode[nolink]{https://data.gouv.fr/dataset/53699ebba3a729239d205f58}
  \end{wrapfigure}

Licence : \textbf{Licence Ouverte
}\newline
Créé le : 2013-10-13\newline
Modifié le : 2018-01-31\newline
De 2011-01-01 à 2014-12-31\newline
Mise à jour : annuelle\newline
Popularité : 10 réutilisations,  13 suivis\newline
Mots-clé : \emph{club, clubs, federation, federations, licence, licences, recensement, sport, sportifs, sportives
}\newline
Permalien : \url{https://data.gouv.fr/dataset/53699ebba3a729239d205f58}\newline

\par
\noindent
    Le recensement annuel des licences auprès des fédérations sportives
agréées par le ministère en charge des sports permet de mesurer le
niveau et l'évolution dans le temps de la pratique sportive encadrée.
Ces statistiques fournissent un éclairage pour les politiques publiques
de développement du sport, tant au niveau national que territorial.
\textbf{Il s'agit d'un recensement au lieu d'habitation de la personne
et non au lieu de pratique.}

Les données issues du recensement sont dans un second temps
\textbf{géocodées par l'Insee}, afin de pouvoir communiquer ces fichiers
au niveau communal.

Les données ne sont pas disponibles pour l'ensemble des fédérations. Un
certain nombre d'entre elles ne disposaient pas de données totalement
géo\_localisables à la commune permettant une exploitation exhaustive.
Les données géocodées ont donc été traitées afin de pouvoir communiquer
une \textbf{\emph{estimation}} du nombre de licences par commune et par
fédération.


\vspace{0.5cm}

\clearpage
\section{Institut national de la propriété industrielle (INPI)}


\begin{center}
  \includegraphics[width=3cm]{images/orga/0a_277d25e3174a8db89dad2b67d42331-100.png}
\end{center}


L'INPI est un établissement public, entièrement autofinancé, placé sous
la tutelle du ministère de l'Économie, de l'Industrie et du Numérique.
Il participe activement à l'élaboration et à la mise en œuvre des
politiques publiques dans le domaine de la propriété industrielle et de
la lutte anti-contrefaçon. Il délivre les brevets, marques, dessins et
modèles et donne accès à toute l'information sur la propriété
industrielle et les entreprises.

A ce titre, l'INPI diffuse ses données relatives aux marques, brevets,
dessins et modèles et les met à disposition dans des formats ouverts
pour un usage interne ou de réutilisation.

L'Institut diffuse également les données brutes statistiques contenues
dans ses études.

L'INPI souhaite ainsi favoriser l'innovation et la création de services
à valeur ajoutée en permettant à des tiers, et notamment aux
développeurs et entrepreneurs, d'exploiter un fonds couvrant les données
relatives à environ 4,4 millions de titres de propriété industrielle.~


\vspace{0.5cm}

\needspace{12\baselineskip}
\subsection*{Les dépôts de dessins et modèles français
}\index{deposant}\index{depot}\index{dessins!et!modeles}\index{dessins!et!modeles!deposes}\index{economie}\index{innovation}\index{inpi}\index{procedure!classique}\index{procedure!simplifiee}\index{propriete!industrielle}\index{voie!nationale}
  \begin{wrapfigure}{r}{2.5cm}
    \centering
    \qrcode[nolink]{https://data.gouv.fr/dataset/542bb4b188ee3821be5f13c3}
  \end{wrapfigure}

Licence : \textbf{Licence Ouverte
}\newline
Créé le : 2014-10-01\newline
Modifié le : 2016-03-07\newline
De 2009-01-01 à 2013-12-31\newline
Granularité : au pays\newline
Mise à jour : annuelle\newline
Popularité : 1 réutilisation,  0 suivi\newline
Mots-clé : \emph{deposant, depot, dessins-et-modeles, dessins-et-modeles-deposes, economie, innovation, inpi, procedure-classique, procedure-simplifiee, propriete-industrielle, voie-nationale
}\newline
Permalien : \url{https://data.gouv.fr/dataset/542bb4b188ee3821be5f13c3}\newline

\par
\noindent
    Le jeu de données met à disposition les dépôts de dessins et modèles
français depuis 2009.


\vspace{0.5cm}
\needspace{12\baselineskip}
\subsection*{Renouvellements de marques nationales
}\index{economie}\index{innovation}\index{inpi}\index{marque}\index{propriete!industrielle}\index{renouvellement}\index{voie!nationale}
  \begin{wrapfigure}{r}{2.5cm}
    \centering
    \qrcode[nolink]{https://data.gouv.fr/dataset/542baf8e88ee3821bf5f13c3}
  \end{wrapfigure}

Licence : \textbf{Licence Ouverte
}\newline
Créé le : 2014-10-01\newline
Modifié le : 2016-03-07\newline
De 2009-01-01 à 2013-12-31\newline
Granularité : au pays\newline
Mise à jour : annuelle\newline
Popularité : 1 réutilisation,  0 suivi\newline
Mots-clé : \emph{economie, innovation, inpi, marque, propriete-industrielle, renouvellement, voie-nationale
}\newline
Permalien : \url{https://data.gouv.fr/dataset/542baf8e88ee3821bf5f13c3}\newline

\par
\noindent
    Le jeu de données met à disposition les comptages des déclarations de
renouvellements de marques nationales depuis 2009.


\vspace{0.5cm}
\needspace{3\baselineskip} \rule{4cm}{0.25pt}\newline\textbf{Aussi disponible du même producteur :}\begin{itemize}
\item \href{https://data.gouv.fr/dataset/542ac6db88ee3856881605a0}{Brevets délivrés par la voie nationale }
\item \href{https://data.gouv.fr/dataset/542ac06988ee38568716059e}{Certificats d'utilité}
\item \href{https://data.gouv.fr/dataset/5964c402c751df0671f823d0}{Comptes annuels déposés auprès des RCS}
\item \href{https://data.gouv.fr/dataset/542ac79788ee3856881605a1}{Délais de délivrance des brevets délivrés }
\item \href{https://data.gouv.fr/dataset/542a644e88ee382b8e22e2d7}{Dépôts de brevets par la voie nationale}
\item \href{https://data.gouv.fr/dataset/542a6bc388ee382b8f22e2da}{Dépôts de brevets sous priorité}
\item \href{https://data.gouv.fr/dataset/542ac31c88ee38568716059f}{Dépôts de certificats complémentaires de protection}
\item \href{https://data.gouv.fr/dataset/542c074f88ee386c41cb3922}{Dépôts et délivrances de brevets en France depuis 1791}
\item \href{https://data.gouv.fr/dataset/542a68ee88ee382b8f22e2d9}{Dépôts, publications, délivrances de brevets par la voie nationale}
\item \href{https://data.gouv.fr/dataset/542c02e488ee386c40cb3922}{Dessins et modèles}
\item \href{https://data.gouv.fr/dataset/59785a7888ee385cb8d806af}{Immatriculations, modifications, radiations des sociétés}
\item \href{https://data.gouv.fr/dataset/542bb3ac88ee3821bd5f13c4}{Les principaux déposants de marques, voie nationale}
\item \href{https://data.gouv.fr/dataset/542ac8cb88ee3856871605a0}{Nombre d'annuités payées à l'INPI }
\item \href{https://data.gouv.fr/dataset/542aca9588ee38568916059f}{Nombre d'annuités payées à l'INPI par origine du paiement}
\item \href{https://data.gouv.fr/dataset/542acd3e88ee3856871605a2}{Oppositions aux marques nationales}
\item \href{https://data.gouv.fr/dataset/542ac4b388ee38568516059e}{Origine des dépôts de certificats complémentaires de protection}
\item \href{https://data.gouv.fr/dataset/542acc5788ee3856871605a1}{Premiers dépôts de marques par la voie nationale }
\item \href{https://data.gouv.fr/dataset/542ac1b688ee38568616059e}{Principales origines des dépôts de certificats d'utilité}
\item \href{https://data.gouv.fr/dataset/542bb5c588ee3841553d12fe}{Principales origines des dessins et modèles par la voie nationale }
\item \href{https://data.gouv.fr/dataset/542bb9f088ee3845dcff90ab}{Principaux déposants des dessins et modèles par la voie nationale (hors dépôts simplifiés) }
\item \href{https://data.gouv.fr/dataset/542a724a88ee382b8f22e2dc}{Principaux déposants par la voie nationale selon le nombre de demandes de brevets publiées}
\item \href{https://data.gouv.fr/dataset/542bb0a688ee3821c05f13c4}{Répartition des dépôts publiés de marques nationales par groupes de classes. }
\item \href{https://data.gouv.fr/dataset/542bb2b088ee3821bf5f13c4}{Répartition par classes des dépôts de marques publiés par la voie nationale}
\item \href{https://data.gouv.fr/dataset/542bb8ef88ee3845dbff90ab}{Répartition par classes des dessins et modèles déposés par la voie nationale}
\item \href{https://data.gouv.fr/dataset/542ac5ba88ee38568616059f}{Répartition par domaines technologiques des demandes de brevets publiées}
\end{itemize}

\clearpage
\section{Institut National de la Statistique et des Etudes Economiques (Insee)}


\begin{center}
  \includegraphics[width=3cm]{images/orga/db_fbfd745ae543f6856ed59e3d44eb71-100.jpg}
\end{center}


L'Institut national de la statistique et des études économiques (Insee)
collecte, produit, analyse et diffuse des informations sur l'économie et
la société françaises. Ces informations intéressent les pouvoirs
publics, les administrations, les entreprises, les chercheurs, les
médias, les enseignants, les étudiants, les particuliers. Elles leur
permettent d'enrichir leurs connaissances, d'effectuer des études, de
faire des prévisions et de prendre des décisions. Pour satisfaire ses
utilisateurs, l'Insee est à l'écoute de leurs besoins et oriente ses
travaux en conséquence en poursuivant un objectif principal : éclairer
le débat économique et social.


\vspace{0.5cm}

\needspace{12\baselineskip}
\subsection*{Activité emploi et chômage - enquête emploi en continu - fichiers détail
}\index{activite}\index{chomage}\index{emploi}\index{inactivite}
  \begin{wrapfigure}{r}{2.5cm}
    \centering
    \qrcode[nolink]{https://data.gouv.fr/dataset/53699434a3a729239d2043b3}
  \end{wrapfigure}

Licence : \textbf{Licence Ouverte
}\newline
Créé le : 2013-09-05\newline
Modifié le : 2019-01-16\newline
De 2003-01-01 à 2016-12-31\newline
Granularité : au pays\newline
Mise à jour : annuelle\newline
Popularité : 1 réutilisation,  4 suivis\newline
Mots-clé : \emph{activite, chomage, emploi, inactivite
}\newline
Permalien : \url{https://data.gouv.fr/dataset/53699434a3a729239d2043b3}\newline

\par
\noindent
    Ce jeu de données met à disposition des fichiers détail anonymisés de
l'enquête emploi en continu. L'enquête emploi en continu vise à observer
à la fois de manière structurelle et conjoncturelle la situation des
personnes sur le marché du travail. Elle s'inscrit dans le cadre des
enquêtes ``Forces de travail'' défini par l'Union européenne (``Labour
Force Survey'').

C'est la seule source fournissant une mesure des concepts d'activité,
chômage, emploi et inactivité tels qu'ils sont définis par le Bureau
international du travail (BIT).

Depuis 2014, les départements d'outre-mer (Guadeloupe, Martinique,
Guyane, La Réunion, à l'exception de Mayotte) ont intégré le dispositif
de l'Enquête Emploi en continu. Les questions portent sur l'emploi, le
chômage, la formation, l'origine sociale, la situation un an auparavant,
et la situation principale mensuelle sur les douze derniers mois.

Les fichiers téléchargeables sur le site de l'Insee sont aux formats
Dbase et Beyond.


\vspace{0.5cm}
\needspace{12\baselineskip}
\subsection*{Base Sirene des entreprises et de leurs établissements (SIREN,
SIRET){[}fin le 30 avril 2019{]}
}\index{associations}\index{companies}\index{entreprises}\index{etablissements}\index{immatriculation}\index{register}\index{registre}\index{siren}\index{sirene}\index{siret}
  \begin{wrapfigure}{r}{2.5cm}
    \centering
    \qrcode[nolink]{https://data.gouv.fr/dataset/5862206588ee38254d3f4e5e}
  \end{wrapfigure}

Licence : \textbf{Licence Ouverte
}\newline
Créé le : 2016-12-27\newline
Modifié le : 2019-03-16\newline
Mise à jour : quotienne\newline
Popularité : 43 réutilisations,  66 suivis\newline
Mots-clé : \emph{associations, companies, entreprises, etablissements, immatriculation, register, registre, siren, sirene, siret
}\newline
Permalien : \url{https://data.gouv.fr/dataset/5862206588ee38254d3f4e5e}\newline

\par
\noindent
    \texttt{Suite\ à\ plusieurs\ demandes\ émanant\ d\textquotesingle{}utilisateurs\ des\ fichiers\ Sirene,\ l\textquotesingle{}Insee\ a\ décidé\ de\ prolonger\ \ de\ trois\ mois\ la\ mise\ à\ disposition\ des\ fichiers\ à\ l\textquotesingle{}ancien\ format\ (dessins\ L2\ et\ XL2\ :\ fichiers\ stocks,\ mises\ à\ jour\ mensuelles\ et\ quotidiennes).\ Cette\ offre\ prendra\ donc\ fin\ le\ 30\ avril\ 2019.\ La\ mise\ en\ concordance\ semestrielle\ ne\ se\ fera\ pas\ sur\ ce\ jeu\ de\ données.}ATTENTION
! Ce jeu de données est remplacé par un
\href{https://www.data.gouv.fr/fr/datasets/base-sirene-des-entreprises-et-de-leurs-etablissements-siren-siret/}{nouveau
jeu de données} et prendra fin le \textbf{30 avril 2019} \emph{(au lieu
du 31 janvier 2019)}.

\textbf{La base Sirene contenant des données à caractère personnel,
l'Insee attire votre attention sur les obligations légales qui en
découlent} : - Le traitement de ces données relève des obligations de
déclaration de la Loi 78-17 du 6 janvier 1978 modifiée, dite Loi
CNIL~:\url{https://www.cnil.fr/fr/loi-78-17-du-6-janvier-1978-modifiee}-
Selon votre usage du jeu de données, il est de votre responsabilité de
tenir compte du statut de diffusion le plus récent de chaque personne
physique. En effet, l'article A123-96 du code de commerce dispose que~:
``Toute personne physique peut demander soit directement lors de ses
formalités de création ou de modification, soit par lettre adressée au
directeur général de l'Institut national de la statistique et des études
économiques, que les informations du répertoire la concernant ne
puissent être utilisées par des tiers autres que les organismes
habilités au titre de l'article R. 123-224 ou les administrations, à des
fins de prospection, notamment commerciale.''

\textbf{Fichiers} : Trois types de fichiers compactés (format ZIP) sont
mis à disposition (1 fichier stock et 2 fichiers de mises à jour
mensuelles ou quotidiennes, France entière). Chaque fichier compacté
(Format ZIP) contient un fichier de données en format CSV.

Le fichier stock et le fichier des mises à jour mensuelles sont déposés
le premier jour de chaque mois avec les données (dessin de fichier L2)
au dernier jour du mois précédent. Le fichier des mises à jour
mensuelles étant au même dessin que le fichier stock, vous pouvez faire
le choix chaque mois, d'annuler et remplacer le fichier stock par le
nouveau fichier stock ou de l'actualiser avec le fichier des mises à
jour mensuelles. Tous les mouvements relatifs à un établissement au
cours du mois sont agrégés, rendant l'information facilement utilisable.

Le fichier des mises à jour quotidiennes est déposé dans la nuit suivant
la journée de gestion qu'il concerne (dessin de fichier XL2).

\emph{\textbf{NB}~: Les fichiers sont déposés un peu plus tardivement
début janvier et début juillet en raison d'opérations de mise en
concordance semestrielle de la base de diffusion.}

Le site de l'Insee www.sirene.fr fournit des informations sur le contenu
de ces bases, ainsi que la
\href{http://sirene.fr/sirene/public/static/documentation}{documentation}
associée et une \href{http://sirene.fr/sirene/public/faq}{Foire Aux
Questions} pour vous aider, comprenant plus de 50 questions-réponses.

Pour toute demande de création, de modification ou de changement
concernant votre situation administrative, nous vous invitons à
contacter le Centre de formalités des entreprises dont vous dépendez :
\url{https://www.service-public.fr/professionnels-entreprises/vosdroits/F24023}.{]}(https://www.service-public.fr/professionnels-entreprises/vosdroits/F24023{]}(https://www.service-public.fr/professionnels-entreprises/vosdroits/F24023).)


\vspace{0.5cm}
\needspace{12\baselineskip}
\subsection*{Code Officiel Géographique (COG)
}\index{code!geographique}\index{decoupages!administratifs}
  \begin{wrapfigure}{r}{2.5cm}
    \centering
    \qrcode[nolink]{https://data.gouv.fr/dataset/58c984b088ee386cdb1261f3}
  \end{wrapfigure}

Licence : \textbf{Licence Ouverte
}\newline
Créé le : 2017-03-15\newline
Modifié le : 2018-05-28\newline
Granularité : à la commune\newline
Mise à jour : annuelle\newline
Popularité : 8 réutilisations,  18 suivis\newline
Mots-clé : \emph{code-geographique, decoupages-administratifs
}\newline
Permalien : \url{https://data.gouv.fr/dataset/58c984b088ee386cdb1261f3}\newline

\par
\noindent
    Le code officiel géographique rassemble les codes et libellés des
communes, des cantons, des arrondissements, des départements, des
régions, des collectivités d'outre-mer et des pays et territoires
étrangers au 1er janvier de chaque année.


\vspace{0.5cm}
\needspace{12\baselineskip}
\subsection*{Données carroyées à 200 m sur la population
}
  \begin{wrapfigure}{r}{2.5cm}
    \centering
    \qrcode[nolink]{https://data.gouv.fr/dataset/5369931ca3a729239d2040d1}
  \end{wrapfigure}

Licence : \textbf{Licence Ouverte
}\newline
Créé le : 2013-12-17\newline
Modifié le : 2019-01-15\newline
De 2010-12-31 à 2011-01-01\newline
Popularité : 11 réutilisations,  10 suivis\newline
Mots-clé : \emph{aucun
}\newline
Permalien : \url{https://data.gouv.fr/dataset/5369931ca3a729239d2040d1}\newline

\par
\noindent
    Cette base comprend 18 variables sur la structure par âge des individus,
les caractéristiques des ménages (locataire/propriétaire, etc.) et les
revenus au 31 décembre 2010. Afin de respecter la règle de diffusion des
données sur les revenus fiscaux des ménages, aucune information
statistique (à l'exception du nombre total d'individus) n'est diffusée
sur des carreaux de moins de 11 ménages. Ces carreaux de faibles
effectifs sont donc regroupés en rectangles de taille plus importante et
satisfaisant à cette règle des 11 ménages minimum. Par ailleurs, un
certain nombre de variables considérées comme « à risque » ont été
traitées afin que tout risque de rupture de confidentialité soit évité.
L'utilisation correcte de ces données carroyées impose une lecture
attentive de la documentation complète sur les données carroyées à 200
mètres.


\vspace{0.5cm}
\needspace{12\baselineskip}
\subsection*{La parité entre les femmes et les hommes
}
  \begin{wrapfigure}{r}{2.5cm}
    \centering
    \qrcode[nolink]{https://data.gouv.fr/dataset/536997aaa3a729239d204d5a}
  \end{wrapfigure}

Licence : \textbf{Licence Ouverte
}\newline
Créé le : 2013-09-30\newline
Modifié le : 2016-02-22\newline
De 1947-01-01 à 2012-12-31\newline
Granularité : au pays\newline
Popularité : 2 réutilisations,  1 suivi\newline
Mots-clé : \emph{aucun
}\newline
Permalien : \url{https://data.gouv.fr/dataset/536997aaa3a729239d204d5a}\newline

\par
\noindent
    Cette série de données référence l'édition 2012 de « Femmes et hommes -
Regards sur la parité », accompagnée de nombreux tableaux
téléchargeables.

Cette édition apporte un éclairage particulier sur les inégalités au
moment de la retraite aujourd'hui et pour les générations actuellement
en âge de travailler, la répartition des tâches domestiques entre les
hommes et les femmes et les bénéficiaires du complément de libre choix
d'activité.

De multiples aspects de la parité femmes-hommes sont décryptés au
travers de 44 tableaux répartis en 6 grands thèmes : Population-santé,
Education-formation, Travail-emploi, Revenus-niveaux de vie, Conditions
de vie, Pouvoir. Ces tableaux se présentent pour la plupart, sous forme
de séries longues et agrégées.


\vspace{0.5cm}
\needspace{12\baselineskip}
\subsection*{Le revenu salarial des femmes reste inférieur à celui des hommes
}\index{ecart!de!salaire}
  \begin{wrapfigure}{r}{2.5cm}
    \centering
    \qrcode[nolink]{https://data.gouv.fr/dataset/536997e5a3a729239d204df0}
  \end{wrapfigure}

Licence : \textbf{Licence Ouverte
}\newline
Créé le : 2013-09-12\newline
Modifié le : 2016-03-15\newline
De 1995-01-01 à 2010-12-31\newline
Granularité : au pays\newline
Popularité : 2 réutilisations,  0 suivi\newline
Mots-clé : \emph{ecart-de-salaire
}\newline
Permalien : \url{https://data.gouv.fr/dataset/536997e5a3a729239d204df0}\newline

\par
\noindent
    Cette publication Insee Références de mars 2013 fait le point sur les
écarts de salaires entre les femmes et les hommes par grand secteur
d'activité, la proportion de femmes dans ces secteurs ou par catégorie
socio-professionnelle, les évolutions dans le temps\ldots{}

Le fichier de données téléchargeable au format xls, présente les données
des tableaux et des graphiques de la publication, enrichies
éventuellement par des données complémentaires


\vspace{0.5cm}
\needspace{12\baselineskip}
\subsection*{Population active et taux d'activité selon le sexe et l'âge
}
  \begin{wrapfigure}{r}{2.5cm}
    \centering
    \qrcode[nolink]{https://data.gouv.fr/dataset/53699d11a3a729239d205b3e}
  \end{wrapfigure}

Licence : \textbf{Licence Ouverte
}\newline
Créé le : 2013-09-05\newline
Modifié le : 2016-03-07\newline
De 1997-01-01 à 2012-12-31\newline
Granularité : au pays\newline
Popularité : 2 réutilisations,  2 suivis\newline
Mots-clé : \emph{aucun
}\newline
Permalien : \url{https://data.gouv.fr/dataset/53699d11a3a729239d205b3e}\newline

\par
\noindent
    Les taux d'activité des hommes et des femmes entre 1975 et 2012 sont
présentés par tranche d'âge.


\vspace{0.5cm}
\needspace{3\baselineskip} \rule{4cm}{0.25pt}\newline\textbf{Aussi disponible du même producteur :}\begin{itemize}
\item \href{https://data.gouv.fr/dataset/53698e6aa3a729239d203466}{Activité productrice des entreprises}
\item \href{https://data.gouv.fr/dataset/53698ecfa3a729239d203587}{Allocataires de la prestation de compensation du handicap (PCH) et de l'allocation compensatrice pour tierce personne (ACTP)}
\item \href{https://data.gouv.fr/dataset/53699185a3a729239d203ca3}{Conditions de vie, société}
\item \href{https://data.gouv.fr/dataset/536992d0a3a729239d20400d}{Description des emplois privés et publics et des salaires}
\item \href{https://data.gouv.fr/dataset/53699351a3a729239d204170}{Échanges extérieurs}
\item \href{https://data.gouv.fr/dataset/53699437a3a729239d2043bb}{Enquêtes de conjoncture}
\item \href{https://data.gouv.fr/dataset/5369957ba3a729239d20472c}{Femmes et hommes face à la violence}
\item \href{https://data.gouv.fr/dataset/53699717a3a729239d204bdc}{Innovations par les entreprises des industries agroalimentaires - Résultats ventilés par tranches d’effectifs salariés}
\item \href{https://data.gouv.fr/dataset/536997caa3a729239d204dab}{La vie associative  }
\item \href{https://data.gouv.fr/dataset/5369a052a3a729239d206338}{Statistiques annuelles d'entreprises : Esane}
\item \href{https://data.gouv.fr/dataset/55e4129788ee386899a46ec1}{Transports}
\end{itemize}

\clearpage
\section{Institut National de l'Information Géographique et Forestière}


\begin{center}
  \includegraphics[width=3cm]{images/orga/1b_e4985396724faf9f6e1122baa7b65c-100.png}
\end{center}


L'IGN est l'opérateur public de référence pour l'information
géographique et forestière. L'IGN a pour vocation de décrire, d'un point
de vue géométrique et physique, la surface du territoire national et
l'occupation de son sol, d'élaborer et de mettre à jour l'inventaire
permanent des ressources forestières nationales.

Producteur et diffuseur de référentiels faisant autorité, de données
géographiques multithématiques, il est également fournisseur de services
d'utilisation des données.

L'IGN intervient en appui pour contribuer à l'analyse des territoires,
faciliter la mise en oeuvre des projets d'aménagement et de
développement durables comme l'application des réglementations.

Les activités :

\begin{itemize}

\item
  L'observation et la mesure de la Terre
\item
  La description du territoire
\item
  La diffusion de l'information géographique
\item
  La recherche, l'innovation et l'enseignement
\item
  Le conseil et l'expertise
\end{itemize}


\vspace{0.5cm}

\needspace{12\baselineskip}
\subsection*{BD ORTHO® 50 cm
}\index{bd!ortho}\index{ortho}\index{orthophoto}\index{orthophotographie}\index{orthophotographie!aerienne}\index{orthophotos}\index{rge}
  \begin{wrapfigure}{r}{2.5cm}
    \centering
    \qrcode[nolink]{https://data.gouv.fr/dataset/594b8736c751df067609049c}
  \end{wrapfigure}

Licence : \textbf{Licence Ouverte
}\newline
Créé le : 2017-06-22\newline
Modifié le : 2018-07-20\newline
Granularité : au département\newline
Popularité : 3 réutilisations,  0 suivi\newline
Mots-clé : \emph{bd-ortho, ortho, orthophoto, orthophotographie, orthophotographie-aerienne, orthophotos, rge
}\newline
Permalien : \url{https://data.gouv.fr/dataset/594b8736c751df067609049c}\newline

\par
\noindent
    L'orthophotographie départementale de l'IGN, l'outil numérique de
référence des collectivités et des ministères, pour mettre en valeur le
territoire, enrichir la visualisation de vos données et de vos projets.

La BD ORTHO® 50 cm de plusieurs départements est téléchargeable
gratuitement, selon les termes de la ``licence ouverte'' version 1.0.


\vspace{0.5cm}
\needspace{12\baselineskip}
\subsection*{BD ORTHO® 5 m - orthophotographie de la France par département
}\index{image}\index{imagerie!aerienne}\index{orthophotographie}\index{photo!aerienne}\index{referentiel}\index{rge}
  \begin{wrapfigure}{r}{2.5cm}
    \centering
    \qrcode[nolink]{https://data.gouv.fr/dataset/53698f82a3a729239d20377a}
  \end{wrapfigure}

Licence : \textbf{Licence Ouverte
}\newline
Créé le : 2013-10-07\newline
Modifié le : 2016-10-21\newline
De 2013-01-01 à 2013-12-31\newline
Granularité : au département\newline
Mise à jour : mensuelle\newline
Popularité : 2 réutilisations,  1 suivi\newline
Mots-clé : \emph{image, imagerie-aerienne, orthophotographie, photo-aerienne, referentiel, rge
}\newline
Permalien : \url{https://data.gouv.fr/dataset/53698f82a3a729239d20377a}\newline

\par
\noindent
    La BD ORTHO® 5m est l'orthophotographie départementale de l'IGN à une
résolution de 5 mètres, disponible par départements.

La BD ORTHO® 5m est un outil numérique de référence des collectivités et
des ministères, pour mettre en valeur le territoire, enrichir la
visualisation de vos données et de vos projets.


\vspace{0.5cm}
\needspace{12\baselineskip}
\subsection*{Données brutes de l'inventaire forestier
}\index{arbre}\index{botanique}\index{dendrometrie}\index{flore}\index{foret}\index{ifn}\index{ign}\index{inventaire!forestier!national}\index{pedologie}\index{repartition!especes}\index{ressources!forestieres}\index{sol}\index{tarif!de!cubage}\index{usage!des!sols}
  \begin{wrapfigure}{r}{2.5cm}
    \centering
    \qrcode[nolink]{https://data.gouv.fr/dataset/5369931ba3a729239d2040cc}
  \end{wrapfigure}

Licence : \textbf{Licence Ouverte
}\newline
Créé le : 2013-11-28\newline
Modifié le : 2016-10-03\newline
Granularité : au point d'intérêt\newline
Mise à jour : annuelle\newline
Popularité : 1 réutilisation,  6 suivis\newline
Mots-clé : \emph{arbre, botanique, dendrometrie, flore, foret, ifn, ign, inventaire-forestier-national, pedologie, repartition-especes, ressources-forestieres, sol, tarif-de-cubage, usage-des-sols
}\newline
Permalien : \url{https://data.gouv.fr/dataset/5369931ba3a729239d2040cc}\newline

\par
\noindent
    Les données brutes de l'inventaire forestier correspondent à l'ensemble
des données collectées en forêt (y compris en peupleraie) sur le
territoire métropolitain par les agents forestiers de terrain de l'IGN.
Ces données portent sur les caractéristiques des placettes d'inventaire
(8000 par an), les mesures et observations sur les arbres (60 000 par
an), les données éco-floristiques. Les coordonnées géographiques des
placettes sont fournies au kilomètre près.


\vspace{0.5cm}
\needspace{12\baselineskip}
\subsection*{Fond cartographique FranceRaster IGN/Esri
}\index{carte}\index{cartographie}\index{esri}\index{ign}
  \begin{wrapfigure}{r}{2.5cm}
    \centering
    \qrcode[nolink]{https://data.gouv.fr/dataset/536995aca3a729239d2047b0}
  \end{wrapfigure}

Licence : \textbf{Licence Ouverte
}\newline
Créé le : 2013-12-03\newline
Modifié le : 2018-01-05\newline
De 2013-08-01 à 2018-01-05\newline
Granularité : au pays\newline
Mise à jour : irrégulière\newline
Popularité : 1 réutilisation,  3 suivis\newline
Mots-clé : \emph{carte, cartographie, esri, ign
}\newline
Permalien : \url{https://data.gouv.fr/dataset/536995aca3a729239d2047b0}\newline

\par
\noindent
    FranceRaster® est une série d'images géoréférencées couvrant la France
Métropolitaine et les DOM sur 12 échelles différentes. De cartographie
homogène, elle est produite avec les bases de données vecteur de l'IGN
les plus adaptées à chaque échelle.

Selon les échelles, FranceRaster® permet la visualisation des thèmes
réseau routier et ferré, bâti, hydrographie, végétation, adresses, sens
de circulation, toponymie\ldots{}

Six échelles, du 1:100.000 au 1:8.000.000, sont disponibles sous Licence
ouverte.

Les échelles concernées sont : 1:100.000 ; 1:250.000 avec ombrage ;
1:500.000 ; 1:1.000.000 avec ombrage ; 1:2.000.000 ; 1:4.000.000 ;
1:8.000.000

FranceRaster est une coédition IGN/Esri.


\vspace{0.5cm}
\needspace{12\baselineskip}
\subsection*{Registre parcellaire graphique (RPG) : contours des parcelles et îlots
culturaux et leur groupe de cultures majoritaire
}\index{agriculture}\index{aides}\index{europe}\index{politique!agricole!commune}\index{registre!parcellaire!graphique}
  \begin{wrapfigure}{r}{2.5cm}
    \centering
    \qrcode[nolink]{https://data.gouv.fr/dataset/58d8d8a0c751df17537c66be}
  \end{wrapfigure}

Licence : \textbf{Licence Ouverte
}\newline
Créé le : 2017-03-27\newline
Modifié le : 2018-11-13\newline
De 2013-01-01 à 2017-12-31\newline
Mise à jour : annuelle\newline
Popularité : 2 réutilisations,  7 suivis\newline
Mots-clé : \emph{agriculture, aides, europe, politique-agricole-commune, registre-parcellaire-graphique
}\newline
Permalien : \url{https://data.gouv.fr/dataset/58d8d8a0c751df17537c66be}\newline

\par
\noindent
    Le registre parcellaire graphique est une base de données géographiques
servant de référence à l'instruction des aides de la politique agricole
commune (PAC). La version anonymisée diffusée ici dans le cadre du
service public de mise à disposition des données de référence contient
les données graphiques des parcelles (depuis 2015) et îlots (éditions
2014 et antérieures). munis de leur culture principale. Ces données sont
produites par l'agence de services et de paiement (ASP) depuis 2007.

La réutilisation du RPG est gratuite pour tous les usages, y compris
commerciaux, selon les termes de la ``licence ouverte'' version 1.0.

Les éditions du Registre parcellaire graphique antérieures à 2013 (2010,
2011 et 2012) sur data.gouv.fr sont disponibles sur
\href{https://www.data.gouv.fr/fr/organizations/agence-de-services-et-de-paiement-asp/}{la
page de l'Agence de service et de paiement}

Les données anonymes du RPG sont millésimées et contiennent des
parcelles et îlots correspondant à ceux déclarés pour la campagne N dans
leur situation connue et arrêtée par l'administration, en général au 1er
janvier de l'année N+1. Ces données couvrent l'ensemble du territoire
français hors Mayotte (y compris les collectivités d'outre-mer de
Saint-Barthélemy et de Saint-Martin).

Format : shapefile

\textbf{Projections disponibles :}

Dans les systèmes géodésiques légaux :

\begin{itemize}

\item
  En métropole : (RGF93) projection Lambert-93
\item
  En outre-mer : (système légal) projections UTM
\end{itemize}

\textbf{Découpages disponibles :}

Le RPG est disponible en téléchargement : - France entière (à compter de
l'édition 2015) - par région (à compter de l'édition 2015) - France
entière par région (de l'édition 2013 à l'édition 2014)

\textbf{API d'accès :}

Le RPG France entière est disponible en web service via les géoservices
du Géoportail - services de consultation au standard WMS et WMTS -
services vecteur au standard WFS.

L'URL pour l'obtention d'une clé d'accès est indiquée ci-dessous. La
documentation d'utilisation des géoservices est accessible
sur\url{https://geoservices.ign.fr/}


\vspace{0.5cm}
\needspace{12\baselineskip}
\subsection*{ROUTE 500
}\index{cartographie}\index{petite!echelle}\index{route}\index{routes}
  \begin{wrapfigure}{r}{2.5cm}
    \centering
    \qrcode[nolink]{https://data.gouv.fr/dataset/53699f65a3a729239d20610a}
  \end{wrapfigure}

Licence : \textbf{Licence Ouverte
}\newline
Créé le : 2013-07-08\newline
Modifié le : 2016-10-21\newline
Mise à jour : annuelle\newline
Popularité : 1 réutilisation,  1 suivi\newline
Mots-clé : \emph{cartographie, petite-echelle, route, routes
}\newline
Permalien : \url{https://data.gouv.fr/dataset/53699f65a3a729239d20610a}\newline

\par
\noindent
    ROUTE 500® est la base de données routières décrivant 500 000 km de
routes du réseau classé (autoroutes, nationales, départementales) et des
éléments d'habillage à des échelles nationales et régionales. ROUTE 500®
ne couvre que le France métropolitaine.

ROUTE 500® permet de situer toute information thématique, d'analyser des
données statistiques et de gérer des déplacements routiers.


\vspace{0.5cm}
\needspace{3\baselineskip} \rule{4cm}{0.25pt}\newline\textbf{Aussi disponible du même producteur :}\begin{itemize}
\item \href{https://data.gouv.fr/dataset/568d01ebc751df2668c664be}{Découpage des cantons pour les élections départementales de mars 2015}
\item \href{https://data.gouv.fr/dataset/5602a60a88ee3820b55b59ee}{Donnée écologiques inventaire forestier}
\item \href{https://data.gouv.fr/dataset/53699337a3a729239d20412d}{Données sur les arbres vivants en peupleraie}
\item \href{https://data.gouv.fr/dataset/570f906fc751df02fcac9326}{EuroGlobalMap - données topographiques au 1/1 000 000 couvrant 45 pays et territoires en Europe}
\item \href{https://data.gouv.fr/dataset/594b8af5c751df7a79788a3a}{ORTHO HR®}
\item \href{https://data.gouv.fr/dataset/56cb3d3388ee382495fa79d0}{Répertoire de données de référence géodésiques et des codelists ISO 19115}
\item \href{https://data.gouv.fr/dataset/53699f65a3a729239d206109}{ROUTE 120}
\end{itemize}

\clearpage
\section{Institut national de l'origine et de la qualité (INAO)}


\begin{center}
  \includegraphics[width=3cm]{images/orga/e4_a2fe7b52ad475892124bb80fb77a0d-100.png}
\end{center}


L'Institut national de l'origine et de la qualité est un établissement
public administratif, doté de la personnalité civile, sous tutelle du
Ministère de l'agriculture et de la pêche.

Par la Loi d'orientation agricole du 5 janvier 2006, l'INAO est chargé
de la mise en œuvre de la politique française relative aux produits sous
signes officiels d'identification de l'origine et de la qualité :
appellation d'origine ; IGP ; label rouge ; STG et agriculture
biologique. L'Institut dont le siège est à Paris, s'appuie sur 8 unités
territoriales couvrant l'ensemble du territoire métropolitain.

Les agents de l'INAO (260 environ) accompagnent les producteurs dans
leurs démarches pour l'obtention d'un signe officiel de l'origine et de
la qualité. Après obtention du signe, ils poursuivent cet
accompagnement, notamment dans le cadre de leur mission de contrôle,
tout au long de la vie du produit. De manière générale, ils préparent et
mettent en œuvre les décisions des instances de l'INAO


\vspace{0.5cm}

\needspace{12\baselineskip}
\subsection*{Aire géographique des IGP
}\index{aire!geographique}\index{igp}\index{indication!geographique}
  \begin{wrapfigure}{r}{2.5cm}
    \centering
    \qrcode[nolink]{https://data.gouv.fr/dataset/53698ec7a3a729239d20356a}
  \end{wrapfigure}

Licence : \textbf{Licence Ouverte
}\newline
Créé le : 2013-07-08\newline
Modifié le : 2019-02-21\newline
De 2013-12-13 à 2014-05-20\newline
Granularité : à la commune\newline
Mise à jour : ponctuelle\newline
Popularité : 1 réutilisation,  7 suivis\newline
Mots-clé : \emph{aire-geographique, igp, indication-geographique
}\newline
Permalien : \url{https://data.gouv.fr/dataset/53698ec7a3a729239d20356a}\newline

\par
\noindent
    Aire géographique des indications géographiques protégées (IGP). Le
fichier liste pour chaque commune, identifiée par son département, son
nom et son code INSEE, les aires géographiques des IGP présentes sur la
commune.


\vspace{0.5cm}
\needspace{12\baselineskip}
\subsection*{Aires et produits AOC/AOP et IGP
}\index{aire!geographique}\index{aoc}\index{aop}\index{appellation!d!origine}\index{igp}\index{indication!geographique}\index{produit}
  \begin{wrapfigure}{r}{2.5cm}
    \centering
    \qrcode[nolink]{https://data.gouv.fr/dataset/53698ecba3a729239d203577}
  \end{wrapfigure}

Licence : \textbf{Licence Ouverte
}\newline
Créé le : 2013-07-08\newline
Modifié le : 2019-02-21\newline
De 2013-12-13 à 2013-12-14\newline
Granularité : au pays\newline
Mise à jour : ponctuelle\newline
Popularité : 2 réutilisations,  8 suivis\newline
Mots-clé : \emph{aire-geographique, aoc, aop, appellation-d-origine, igp, indication-geographique, produit
}\newline
Permalien : \url{https://data.gouv.fr/dataset/53698ecba3a729239d203577}\newline

\par
\noindent
    Ce fichier liste les aires géographiques des appellations d'origine
contrôlée (AOC)/protégée (AOP) et des indications géographiques
protégées (IGP) et, pour chaque aire géographique, les produits qui lui
sont rattachés.


\vspace{0.5cm}
\needspace{12\baselineskip}
\subsection*{Aires géographiques des AOC/AOP
}\index{aire!geographique}\index{aoc}\index{aop}\index{appellation!d!origine}
  \begin{wrapfigure}{r}{2.5cm}
    \centering
    \qrcode[nolink]{https://data.gouv.fr/dataset/53698ecca3a729239d203579}
  \end{wrapfigure}

Licence : \textbf{Licence Ouverte
}\newline
Créé le : 2013-07-08\newline
Modifié le : 2019-02-21\newline
De 2013-12-13 à 2014-05-20\newline
Granularité : à la commune\newline
Mise à jour : ponctuelle\newline
Popularité : 10 réutilisations,  19 suivis\newline
Mots-clé : \emph{aire-geographique, aoc, aop, appellation-d-origine
}\newline
Permalien : \url{https://data.gouv.fr/dataset/53698ecca3a729239d203579}\newline

\par
\noindent
    Aires géographiques des appellations d'origine contrôlées
(AOC)/protégées (AOP). Le fichier liste pour chaque commune, identifiée
par son département, son nom et son code INSEE, les aires géographiques
des appellations AOC/AOP qui se situent sur la commune


\vspace{0.5cm}
\needspace{12\baselineskip}
\subsection*{Délimitation Parcellaire des AOC Viticoles de l'INAO
}\index{agriculture}\index{aoc}\index{aop}\index{aop!vins}\index{appellation!d!origine!controlee}\index{appellation!d!origine!protegee}\index{appellation!dorigine}\index{economie}\index{igp}\index{inao}\index{origine}\index{viticole}
  \begin{wrapfigure}{r}{2.5cm}
    \centering
    \qrcode[nolink]{https://data.gouv.fr/dataset/5aaf6b7ec751df67b0d7a87f}
  \end{wrapfigure}

Licence : \textbf{Licence Ouverte
}\newline
Créé le : 2018-03-19\newline
Modifié le : 2018-03-19\newline
De 2010-01-01 à 2018-03-19\newline
Mise à jour : trimestrielle\newline
Popularité : 4 réutilisations,  8 suivis\newline
Mots-clé : \emph{agriculture, aoc, aop, aop-vins, appellation-d-origine-controlee, appellation-d-origine-protegee, appellation-dorigine, economie, igp, inao, origine, viticole
}\newline
Permalien : \url{https://data.gouv.fr/dataset/5aaf6b7ec751df67b0d7a87f}\newline

\par
\noindent
    \textbf{Attention} \textbf{: Les données communiquées ici ne le sont
qu'à titre informatif. Les délimitations parcellaires officielles sont
celles consultables sur les plans déposés en mairie ou auprès des
services de l'INAO. A noter également que l'ensemble des délimitations
parcellaires des AOC viticoles ne sont pas encore représentées, le
travail de dématérialisation étant en cours.}

Selon les règlements européens 510/2006 du 6 mars 2006 et 1234/2007, le
cahier des charges des appellations définit l'aire géographique des
produits enregistrés en AOP ou en IGP. Au sein de cette aire est incluse
une aire parcellaire correspondant à l'aire de production de la matière
première. L'aire parcellaire délimitée correspondant à une délimitation
reposant sur les limites administratives du cadastre (les parcelles) et
dont le maillage suffisamment fin permet de tenir compte de variations
très localisées des éléments du milieu physique. Cette délimitation est
utilisée essentiellement pour les AOP et IGP viticoles et correspond
dans ce cas à l'aire de production de la matière première. Quelques
autres produits en AOP concernent d'autres productions agricoles telles
que des huiles d'olives, des noix, des oignons\ldots{}

\textbf{Nos données sont également disponibles sur le Géoportail opéré
par L'IGN} :
\url{https://www.geoportail.gouv.fr/donnees/inao-aoc-viticoles-parcellaire}


\vspace{0.5cm}
\needspace{3\baselineskip} \rule{4cm}{0.25pt}\newline\textbf{Aussi disponible du même producteur :}\begin{itemize}
\item \href{https://data.gouv.fr/dataset/5aa7ef8ec751df2a42a0587a}{SIQO Publiés}
\end{itemize}

\clearpage
\section{IRDES \&{} ECOSANTE.FR : Institut de Recherche et Documentation en Economie de la Santé}


\begin{center}
  \includegraphics[width=3cm]{images/orga/cb_3785c208c44d75b9e2f1d43054ea5f-100.jpg}
\end{center}


Producteur de données et d'analyses en économie de la santé, l'Irdes a
pour objectif de contribuer à nourrir la réflexion de tous ceux qui
s'intéressent à l'avenir du système de santé. Multidisciplinaire,
l'équipe de l'Irdes observe et analyse l'évolution des comportements des
consommateurs et des producteurs de soins à la fois sous l'angle
médical, économique, géographique\ldots{}

Les publications, sont toutes en ligne dans leur version intégrale sur
le site \href{http://www.irdes.fr}{\textbf{www.irdes.fr}}

L'Irdes produit également
\href{http://www.ecosante.fr}{\textbf{www.ecosante.fr}} Les bases de
données Eco-Santé rassemblent des séries statistiques dans le domaine
sanitaire et social. Mises à jour en continu et accessibles
gratuitement, elles ont pour objectif de mettre à disposition ces séries
de données tout en permettant à l'utilisateur un gain de temps dans la
recherche et la présentation d'informations.

Contact : \href{mailto:ecosante@irdes.fr}{\nolinkurl{ecosante@irdes.fr}}


\vspace{0.5cm}

\needspace{12\baselineskip}
\subsection*{Activité des médecins libéraux (France uniquement)
}\index{consultation}\index{medecin!liberal}\index{snir}\index{visite}
  \begin{wrapfigure}{r}{2.5cm}
    \centering
    \qrcode[nolink]{https://data.gouv.fr/dataset/53698e66a3a729239d20345c}
  \end{wrapfigure}

Licence : \textbf{Licence Ouverte
}\newline
Créé le : 2014-04-14\newline
Modifié le : 2016-03-16\newline
De 1980-01-01 à 2011-12-31\newline
Granularité : au pays\newline
Mise à jour : annuelle\newline
Popularité : 1 réutilisation,  0 suivi\newline
Mots-clé : \emph{consultation, medecin-liberal, snir, visite
}\newline
Permalien : \url{https://data.gouv.fr/dataset/53698e66a3a729239d20345c}\newline

\par
\noindent
    Export : CSV, HTML et XLS Source :
\href{http://www.ecosante.fr/}{www.ecosante.fr} Irdes d'après données
Snir


\vspace{0.5cm}
\needspace{12\baselineskip}
\subsection*{Capacités d'accueil et personnels des établissements de santé
}\index{departement}\index{equipement!lit}\index{etablissement!sante}\index{personnel!administratif}\index{personnel!hospitalier}\index{praticien!hospitalier}\index{region}\index{sae}
  \begin{wrapfigure}{r}{2.5cm}
    \centering
    \qrcode[nolink]{https://data.gouv.fr/dataset/5369900fa3a729239d2038e5}
  \end{wrapfigure}

Licence : \textbf{Licence Ouverte
}\newline
Créé le : 2013-12-16\newline
Modifié le : 2016-01-26\newline
De 1995-01-01 à 2012-12-31\newline
Granularité : au département\newline
Mise à jour : annuelle\newline
Popularité : 2 réutilisations,  0 suivi\newline
Mots-clé : \emph{departement, equipement-lit, etablissement-sante, personnel-administratif, personnel-hospitalier, praticien-hospitalier, region, sae
}\newline
Permalien : \url{https://data.gouv.fr/dataset/5369900fa3a729239d2038e5}\newline

\par
\noindent
    Export : CSV, HTML et XLS Source : www.ecosante.fr Irdes d'après données
SAE, Drees


\vspace{0.5cm}
\needspace{12\baselineskip}
\subsection*{Décès domiciliés par tranche d'âge
}\index{age}\index{deces}\index{departement}\index{insee}\index{region}\index{sexe}
  \begin{wrapfigure}{r}{2.5cm}
    \centering
    \qrcode[nolink]{https://data.gouv.fr/dataset/5369921fa3a729239d203e38}
  \end{wrapfigure}

Licence : \textbf{Licence Ouverte
}\newline
Créé le : 2013-12-16\newline
Modifié le : 2016-02-26\newline
De 1988-01-01 à 2010-12-31\newline
Granularité : au département\newline
Mise à jour : annuelle\newline
Popularité : 1 réutilisation,  0 suivi\newline
Mots-clé : \emph{age, deces, departement, insee, region, sexe
}\newline
Permalien : \url{https://data.gouv.fr/dataset/5369921fa3a729239d203e38}\newline

\par
\noindent
    Export : CSV, HTML et XLS\\
Source : \href{http://www.ecosante.fr/}{www.ecosante.fr} Irdes d'après
données Insee


\vspace{0.5cm}
\needspace{12\baselineskip}
\subsection*{Décès par cause
}\index{cause!deces}\index{deces}\index{femme}\index{homme}\index{mortalite}\index{sexe}
  \begin{wrapfigure}{r}{2.5cm}
    \centering
    \qrcode[nolink]{https://data.gouv.fr/dataset/53699222a3a729239d203e40}
  \end{wrapfigure}

Licence : \textbf{Licence Ouverte
}\newline
Créé le : 2013-12-16\newline
Modifié le : 2016-03-13\newline
De 1979-01-01 à 2011-12-31\newline
Granularité : au pays\newline
Mise à jour : annuelle\newline
Popularité : 2 réutilisations,  0 suivi\newline
Mots-clé : \emph{cause-deces, deces, femme, homme, mortalite, sexe
}\newline
Permalien : \url{https://data.gouv.fr/dataset/53699222a3a729239d203e40}\newline

\par
\noindent
    Export : CSV, HTML et XLS\\
Source : \href{http://www.ecosante.fr/}{www.ecosante.fr} Irdes d'après
données CépiDc Inserm


\vspace{0.5cm}
\needspace{12\baselineskip}
\subsection*{Démographie des chirurgiens-dentistes
}\index{adeli}\index{chirurgien!dentiste}\index{demographie!medicale}\index{densite!medicale}\index{departement}\index{effectif}\index{exercice!liberal}\index{exercice!salarie}\index{region}
  \begin{wrapfigure}{r}{2.5cm}
    \centering
    \qrcode[nolink]{https://data.gouv.fr/dataset/53699260a3a729239d203ee8}
  \end{wrapfigure}

Licence : \textbf{Licence Ouverte
}\newline
Créé le : 2013-12-16\newline
Modifié le : 2016-03-13\newline
De 1988-01-01 à 2011-12-31\newline
Granularité : au département\newline
Mise à jour : annuelle\newline
Popularité : 1 réutilisation,  0 suivi\newline
Mots-clé : \emph{adeli, chirurgien-dentiste, demographie-medicale, densite-medicale, departement, effectif, exercice-liberal, exercice-salarie, region
}\newline
Permalien : \url{https://data.gouv.fr/dataset/53699260a3a729239d203ee8}\newline

\par
\noindent
    Export : CSV, HTML et XLS\\
Source : \href{http://www.ecosante.fr/}{www.ecosante.fr} Irdes d'après
données Adeli, Drees


\vspace{0.5cm}
\needspace{12\baselineskip}
\subsection*{Densité des gynécologues et obstétriciens
}\index{asip!sante}\index{densite}\index{femme}\index{gynecologue}\index{obstetricien}\index{rpps}
  \begin{wrapfigure}{r}{2.5cm}
    \centering
    \qrcode[nolink]{https://data.gouv.fr/dataset/53699269a3a729239d203f07}
  \end{wrapfigure}

Licence : \textbf{Licence Ouverte
}\newline
Créé le : 2014-05-02\newline
Modifié le : 2016-02-18\newline
De 2012-01-01 à 2013-12-31\newline
Granularité : au département\newline
Mise à jour : annuelle\newline
Popularité : 1 réutilisation,  0 suivi\newline
Mots-clé : \emph{asip-sante, densite, femme, gynecologue, obstetricien, rpps
}\newline
Permalien : \url{https://data.gouv.fr/dataset/53699269a3a729239d203f07}\newline

\par
\noindent
    Export : CSV, HTML et XLS Source :
\href{http://www.ecosante.fr/}{www.ecosante.fr} Irdes d'après données
Drees/Asip-Santé, répertoire partagé des professionnels de santé (RPPS)


\vspace{0.5cm}
\needspace{12\baselineskip}
\subsection*{Distances d'accès aux médecins libéraux
}\index{acces!soin}\index{decile}\index{departement}\index{medecin!generaliste}\index{medecin!liberal}\index{pediatre}\index{profession!sante}\index{region}\index{temps!acces}
  \begin{wrapfigure}{r}{2.5cm}
    \centering
    \qrcode[nolink]{https://data.gouv.fr/dataset/536992f4a3a729239d204069}
  \end{wrapfigure}

Licence : \textbf{Licence Ouverte
}\newline
Créé le : 2013-12-16\newline
Modifié le : 2015-10-20\newline
De 2010-01-01 à 2010-12-31\newline
Granularité : au département\newline
Mise à jour : ponctuelle\newline
Popularité : 2 réutilisations,  0 suivi\newline
Mots-clé : \emph{acces-soin, decile, departement, medecin-generaliste, medecin-liberal, pediatre, profession-sante, region, temps-acces
}\newline
Permalien : \url{https://data.gouv.fr/dataset/536992f4a3a729239d204069}\newline

\par
\noindent
    Distances d'accès aux généralistes libéraux en temps (minutes) Export :
CSV, HTML et XLS Source :
\href{http://www.ecosante.fr/}{www.ecosante.fr} Irdes d'après données
Cnamts, Insee, Irdes Les calculs des distances ont été effectués avec le
logiciel de calcul des distances routières ``Distancier Odomatrix -
INRA, UMR1041 CESAER'' développé par l'Inra en collaboration avec le
Certu et l'Insee.


\vspace{0.5cm}
\needspace{12\baselineskip}
\subsection*{Emploi salarié, non salarié et par secteur
}\index{departement}\index{effectif!non!salarie}\index{effectif!salarie}\index{emploi}\index{insee}\index{region}
  \begin{wrapfigure}{r}{2.5cm}
    \centering
    \qrcode[nolink]{https://data.gouv.fr/dataset/5369940ba3a729239d204351}
  \end{wrapfigure}

Licence : \textbf{Licence Ouverte
}\newline
Créé le : 2013-12-16\newline
Modifié le : 2016-03-16\newline
De 1989-01-01 à 2011-12-31\newline
Granularité : au département\newline
Mise à jour : annuelle\newline
Popularité : 1 réutilisation,  0 suivi\newline
Mots-clé : \emph{departement, effectif-non-salarie, effectif-salarie, emploi, insee, region
}\newline
Permalien : \url{https://data.gouv.fr/dataset/5369940ba3a729239d204351}\newline

\par
\noindent
    Export : CSV, HTML et XLS\\
Source : \href{http://www.ecosante.fr/}{www.ecosante.fr} Irdes d'après
données Insee


\vspace{0.5cm}
\needspace{12\baselineskip}
\subsection*{Financement de la dépense de soins par la Sécurité sociale
}\index{autres!biens!medicaux}\index{comptes!de!la!sante}\index{consommation!soins}\index{hopital}\index{medicament}\index{prise!en!charge}\index{securite!sociale}\index{soins!de!ville}\index{transport!malade}
  \begin{wrapfigure}{r}{2.5cm}
    \centering
    \qrcode[nolink]{https://data.gouv.fr/dataset/5416c5b3a3a72937ecd41f9c}
  \end{wrapfigure}

Licence : \textbf{Licence Ouverte
}\newline
Créé le : 2014-09-15\newline
Modifié le : 2016-03-03\newline
De 2006-01-01 à 2013-12-31\newline
Granularité : au pays\newline
Mise à jour : annuelle\newline
Popularité : 1 réutilisation,  0 suivi\newline
Mots-clé : \emph{autres-biens-medicaux, comptes-de-la-sante, consommation-soins, hopital, medicament, prise-en-charge, securite-sociale, soins-de-ville, transport-malade
}\newline
Permalien : \url{https://data.gouv.fr/dataset/5416c5b3a3a72937ecd41f9c}\newline

\par
\noindent
    Export : CSV, HTML et XLS Source :
\href{http://www.ecosante.fr/}{www.ecosante.fr} Irdes d'après données
Comptes de la santé de la Drees


\vspace{0.5cm}
\needspace{12\baselineskip}
\subsection*{La densité des sages-femmes selon le mode d'exercice
}\index{activite!liberale}\index{activite!mixte}\index{activite!salariee}\index{asip!sante}\index{mode!exercice}\index{rpps}\index{sage!femme}
  \begin{wrapfigure}{r}{2.5cm}
    \centering
    \qrcode[nolink]{https://data.gouv.fr/dataset/537893d4a3a7295dd332d9e1}
  \end{wrapfigure}

Licence : \textbf{Licence Ouverte
}\newline
Créé le : 2014-05-12\newline
Modifié le : 2016-02-08\newline
De 2011-01-01 à 2014-12-31\newline
Granularité : au département\newline
Mise à jour : annuelle\newline
Popularité : 2 réutilisations,  0 suivi\newline
Mots-clé : \emph{activite-liberale, activite-mixte, activite-salariee, asip-sante, mode-exercice, rpps, sage-femme
}\newline
Permalien : \url{https://data.gouv.fr/dataset/537893d4a3a7295dd332d9e1}\newline

\par
\noindent
    Export : CSV, HTML et XLS Source :
\href{http://www.ecosante.fr/}{www.ecosante.fr} Irdes d'après données
Drees/Asip-Santé, répertoire partagé des professionnels de santé (RPPS)


\vspace{0.5cm}
\needspace{12\baselineskip}
\subsection*{Le Salaire Minimum Interprofessionnel de Croissance SMIC
}\index{economie}\index{insee}\index{salaire}\index{salaire!minimum}
  \begin{wrapfigure}{r}{2.5cm}
    \centering
    \qrcode[nolink]{https://data.gouv.fr/dataset/536997efa3a729239d204e0a}
  \end{wrapfigure}

Licence : \textbf{Licence Ouverte
}\newline
Créé le : 2013-12-16\newline
Modifié le : 2016-03-05\newline
De 1950-01-01 à 2012-12-31\newline
Granularité : au pays\newline
Mise à jour : annuelle\newline
Popularité : 1 réutilisation,  1 suivi\newline
Mots-clé : \emph{economie, insee, salaire, salaire-minimum
}\newline
Permalien : \url{https://data.gouv.fr/dataset/536997efa3a729239d204e0a}\newline

\par
\noindent
    Export : CSV, HTML et XLS\\
Source : \href{http://www.ecosante.fr/}{www.ecosante.fr} Irdes d'après
données Insee


\vspace{0.5cm}
\needspace{12\baselineskip}
\subsection*{Les pharmacies
}\index{cnop}\index{departement}\index{officine!privee}\index{pharmacie!officine}\index{region}
  \begin{wrapfigure}{r}{2.5cm}
    \centering
    \qrcode[nolink]{https://data.gouv.fr/dataset/53699c78a3a729239d2059a9}
  \end{wrapfigure}

Licence : \textbf{Licence Ouverte
}\newline
Créé le : 2013-12-16\newline
Modifié le : 2016-03-13\newline
De 1984-01-01 à 2014-12-31\newline
Granularité : au département\newline
Mise à jour : annuelle\newline
Popularité : 1 réutilisation,  0 suivi\newline
Mots-clé : \emph{cnop, departement, officine-privee, pharmacie-officine, region
}\newline
Permalien : \url{https://data.gouv.fr/dataset/53699c78a3a729239d2059a9}\newline

\par
\noindent
    Export : CSV, HTML et XLS\\
Source : \href{http://www.ecosante.fr/}{www.ecosante.fr} Irdes d'après
données Conseil national de l'Ordre des pharmaciens


\vspace{0.5cm}
\needspace{12\baselineskip}
\subsection*{Mortalité
}\index{cause!deces}\index{deces}\index{departement}\index{insee}\index{mortalite}\index{mortalite!infantile}
  \begin{wrapfigure}{r}{2.5cm}
    \centering
    \qrcode[nolink]{https://data.gouv.fr/dataset/53699a1ea3a729239d2053f0}
  \end{wrapfigure}

Licence : \textbf{Licence Ouverte
}\newline
Créé le : 2013-12-13\newline
Modifié le : 2016-02-29\newline
De 1990-01-01 à 2013-12-31\newline
Granularité : au département\newline
Mise à jour : annuelle\newline
Popularité : 1 réutilisation,  1 suivi\newline
Mots-clé : \emph{cause-deces, deces, departement, insee, mortalite, mortalite-infantile
}\newline
Permalien : \url{https://data.gouv.fr/dataset/53699a1ea3a729239d2053f0}\newline

\par
\noindent
    Export : CSV, HTML et XLS\\
Source : \href{http://www.ecosante.fr/}{www.ecosante.fr} Irdes d'après
données Insee ; CépiDc de l'Inserm (Cim10)


\vspace{0.5cm}
\needspace{12\baselineskip}
\subsection*{Séjours d'hospitalisation partielle de MCO
}\index{age}\index{chirurgie}\index{court!sejour}\index{departement}\index{hospitalisation!partielle}\index{medecine}\index{obstetrique}\index{pmsi}\index{region}\index{sejour!hospitalier}
  \begin{wrapfigure}{r}{2.5cm}
    \centering
    \qrcode[nolink]{https://data.gouv.fr/dataset/53699fcba3a729239d2061f5}
  \end{wrapfigure}

Licence : \textbf{Licence Ouverte
}\newline
Créé le : 2013-12-18\newline
Modifié le : 2016-01-04\newline
De 2000-01-01 à 2013-12-31\newline
Granularité : au département\newline
Mise à jour : annuelle\newline
Popularité : 2 réutilisations,  1 suivi\newline
Mots-clé : \emph{age, chirurgie, court-sejour, departement, hospitalisation-partielle, medecine, obstetrique, pmsi, region, sejour-hospitalier
}\newline
Permalien : \url{https://data.gouv.fr/dataset/53699fcba3a729239d2061f5}\newline

\par
\noindent
    Export : CSV, HTML et XLS\\
Source : \href{http://www.ecosante.fr/}{www.ecosante.fr} Irdes d'après
données PMSI-MCO, Drees

L'hospitalisation partielle est une hospitalisation inférieure à 24
heures


\vspace{0.5cm}
\needspace{12\baselineskip}
\subsection*{Soldes par branche du Régime général
}\index{assurance!maladie}\index{assurance!maternite}\index{assurance!vieillesse}\index{prestation!familiale}\index{prestation!securite!sociale}\index{regime!general}
  \begin{wrapfigure}{r}{2.5cm}
    \centering
    \qrcode[nolink]{https://data.gouv.fr/dataset/5369a01da3a729239d2062b6}
  \end{wrapfigure}

Licence : \textbf{Licence Ouverte
}\newline
Créé le : 2013-12-16\newline
Modifié le : 2015-07-24\newline
De 1977-01-01 à 2014-12-31\newline
Granularité : au pays\newline
Mise à jour : annuelle\newline
Popularité : 1 réutilisation,  1 suivi\newline
Mots-clé : \emph{assurance-maladie, assurance-maternite, assurance-vieillesse, prestation-familiale, prestation-securite-sociale, regime-general
}\newline
Permalien : \url{https://data.gouv.fr/dataset/5369a01da3a729239d2062b6}\newline

\par
\noindent
    Export : CSV, HTML et XLS Source : www.ecosante.fr Irdes d'après données
Comptes de la Sécurité sociale


\vspace{0.5cm}
\needspace{3\baselineskip} \rule{4cm}{0.25pt}\newline\textbf{Aussi disponible du même producteur :}\begin{itemize}
\item \href{https://data.gouv.fr/dataset/539a5bd5a3a7293bc272836d}{Accueil de l'enfance et la jeunesse handicapées}
\item \href{https://data.gouv.fr/dataset/53698e53a3a729239d20342b}{Accueil enfance et jeunesse handicapées}
\item \href{https://data.gouv.fr/dataset/53698e67a3a729239d20345d}{Activité des médecins libéraux actifs à part entière APE}
\item \href{https://data.gouv.fr/dataset/53698e7ea3a729239d2034a5}{Affections de longue durée ALD}
\item \href{https://data.gouv.fr/dataset/53698f86a3a729239d203786}{Bénéficiaires de la CMU par région et département}
\item \href{https://data.gouv.fr/dataset/53c859bea3a72933e780ee50}{Bénéficiaires de l'aide sociale à l'hébergement et places d'hébergement pour personnes âgées}
\item \href{https://data.gouv.fr/dataset/53bdd02fa3a7292f66115a0e}{Bénéficiaires de l'Allocation Personnalisée d'Autonomie (APA) pour personnes âgées dépendantes}
\item \href{https://data.gouv.fr/dataset/53698f88a3a729239d20378d}{Bénéficiaires du RSA ou revenu de solidarité active par département et région}
\item \href{https://data.gouv.fr/dataset/5369900fa3a729239d2038e3}{Capacités d'accueil des centres hospitaliers régionaux}
\item \href{https://data.gouv.fr/dataset/5369900fa3a729239d2038e4}{Capacités d'accueil des établissements de santé}
\item \href{https://data.gouv.fr/dataset/5369919ba3a729239d203cd9}{Consommation de tabac}
\item \href{https://data.gouv.fr/dataset/536991aea3a729239d203d0b}{Consultations et visites de spécialistes libéraux}
\item \href{https://data.gouv.fr/dataset/53699223a3a729239d203e43}{Décès par Sida, VIH}
\item \href{https://data.gouv.fr/dataset/53788e2fa3a7295dd332d9d1}{Démographie des médecins au 1er janvier}
\item \href{https://data.gouv.fr/dataset/53699261a3a729239d203eef}{Démographie des professions de santé}
\item \href{https://data.gouv.fr/dataset/53699262a3a729239d203ef3}{Démographie des professions de santé libérales}
\item \href{https://data.gouv.fr/dataset/53788e38a3a7295dd332d9d2}{Densité des médecins au 1er janvier}
\item \href{https://data.gouv.fr/dataset/5369926aa3a729239d203f09}{Densité des professionnels paramédicaux libéraux}
\item \href{https://data.gouv.fr/dataset/53699273a3a729239d203f22}{Dépenses d'aide sociale départementale pour les personnes handicapées}
\item \href{https://data.gouv.fr/dataset/53d380d2a3a7290427013693}{Distance d'accès à plusieurs spécialités médicales}
\item \href{https://data.gouv.fr/dataset/536992f4a3a729239d20406a}{Distances d'accès aux professionnels de santé}
\item \href{https://data.gouv.fr/dataset/53788f51a3a7295dd332d9dc}{Effectif de médecins par catégorie}
\item \href{https://data.gouv.fr/dataset/536994b1a3a729239d2044ed}{Equipement en services pour jeunes handicapés}
\item \href{https://data.gouv.fr/dataset/536994b1a3a729239d2044ec}{Equipement en soins infirmiers à domicile ou SSIAD}
\item \href{https://data.gouv.fr/dataset/536994e9a3a729239d204577}{Espérance de vie à la naissance et à différents âges}
\item \href{https://data.gouv.fr/dataset/536994f6a3a729239d204597}{Etablissements pour personnes âgées et pour adultes handicapés}
\item \href{https://data.gouv.fr/dataset/5417e259a3a7294af4493de2}{Financement de la dépense de soins par les ménages}
\item \href{https://data.gouv.fr/dataset/5416c5b3a3a72937ecd41f9d}{Financement de la dépense de soins par les organismes complémentaires }
\item \href{https://data.gouv.fr/dataset/536995b8a3a729239d2047d6}{Formation aux professions de santé}
\item \href{https://data.gouv.fr/dataset/53699604a3a729239d2048a9}{Greffes : Causes de non-prélèvements}
\item \href{https://data.gouv.fr/dataset/53699604a3a729239d2048aa}{Greffes : Les donneurs}
\item \href{https://data.gouv.fr/dataset/53699604a3a729239d2048ab}{Greffes par type d'organe}
\item \href{https://data.gouv.fr/dataset/5369968ca3a729239d204a79}{Indicateurs démographiques, indice de vieillissement}
\item \href{https://data.gouv.fr/dataset/5369968da3a729239d204a7b}{Indicateurs démographiques, Population}
\item \href{https://data.gouv.fr/dataset/5369978fa3a729239d204d10}{Journées d'hospitalisation complète en MCO }
\item \href{https://data.gouv.fr/dataset/53699798a3a729239d204d2b}{La comptabilité nationale}
\item \href{https://data.gouv.fr/dataset/536997d6a3a729239d204dcc}{Le marché officinal par classe thérapeutique}
\item \href{https://data.gouv.fr/dataset/536997e4a3a729239d204def}{Le revenu disponible brut des ménages}
\item \href{https://data.gouv.fr/dataset/55ae16b488ee38725d3ca289}{Le revenu disponible brut régional des ménages}
\item \href{https://data.gouv.fr/dataset/536997eaa3a729239d204dfd}{Les accidents de la circulation}
\item \href{https://data.gouv.fr/dataset/536997eba3a729239d204e00}{Les accidents de trajet }
\item \href{https://data.gouv.fr/dataset/536997eaa3a729239d204dff}{Les accidents du travail}
\item \href{https://data.gouv.fr/dataset/54940113c751df38af04805a}{Les alternatives à l'hospitalisation complète}
\item \href{https://data.gouv.fr/dataset/53699804a3a729239d204e43}{Les comptes nationaux de la santé}
\item \href{https://data.gouv.fr/dataset/5387f353a3a7291cb36754ab}{Les dépenses de prévention collective}
\item \href{https://data.gouv.fr/dataset/53699805a3a729239d204e46}{Les dépenses de prévention individuelle}
\item \href{https://data.gouv.fr/dataset/53699813a3a729239d204e6a}{Les dépenses de santé}
\item \href{https://data.gouv.fr/dataset/53699814a3a729239d204e6b}{Les dépenses de soins de longue durée aux personnes âgées}
\item \href{https://data.gouv.fr/dataset/53dfd3a3a3a729110ca8d36d}{Les emplois et les ressources de la protection sociale}
\item \href{https://data.gouv.fr/dataset/53699855a3a729239d204f1e}{Les maladies à déclaration obligatoire}
\item et 33 autres jeux de données\end{itemize}

\clearpage
\section{La Poste}


\begin{center}
  \includegraphics[width=3cm]{images/orga/95_5d400cd1d24e40aa2d6e970b77ac93-100.jpg}
\end{center}


Société anonyme à capitaux 100 \% publics depuis le 1er mars 2010, Le
Groupe La Poste est un modèle original de groupe multi-métiers structuré
autour de quatre activités : le Courrier, le Colis/Express, La Banque
Postale et l'Enseigne La Poste.


\vspace{0.5cm}

\needspace{12\baselineskip}
\subsection*{Liste des points de contact du réseau postal français - Horaires,
équipements et services associés
}\index{accessibilite}\index{agence}\index{amenagement!du!territoire}\index{automate}\index{batiment}\index{billet}\index{bureau!de!poste}\index{changeur!de!monnaie}\index{commercant}\index{distributeur}\index{equipement}\index{fermeture}\index{handicap}\index{heure!limite!de!depot}\index{horaires}\index{la!poste}\index{logement}\index{moneo}\index{ouverture}\index{photocopie}\index{pmr}\index{relais}\index{rse}\index{service}\index{timbre}\index{urbanisme}
  \begin{wrapfigure}{r}{2.5cm}
    \centering
    \qrcode[nolink]{https://data.gouv.fr/dataset/53699940a3a729239d2051d4}
  \end{wrapfigure}

Licence : \textbf{Open Data Commons Open Database License (ODbL)
}\newline
Créé le : 2013-07-08\newline
Modifié le : 2016-03-13\newline
Granularité : au point d'intérêt\newline
Mise à jour : mensuelle\newline
Popularité : 15 réutilisations,  17 suivis\newline
Mots-clé : \emph{accessibilite, agence, amenagement-du-territoire, automate, batiment, billet, bureau-de-poste, changeur-de-monnaie, commercant, distributeur, equipement, fermeture, handicap, heure-limite-de-depot, horaires, la-poste, logement, moneo, ouverture, photocopie, pmr, relais, rse, service, timbre, urbanisme
}\newline
Permalien : \url{https://data.gouv.fr/dataset/53699940a3a729239d2051d4}\newline

\par
\noindent
    Ce jeu de données propose 5 ressources, issues directement du portail
\href{http://datanova.legroupe.laposte.fr/page/accueil/}{http://datanova.legroupe.laposte.fr}{]}(http://datanova.legroupe.laposte.fr{]}(http://datanova.legroupe.laposte.fr/page/accueil/)):
- Liste des bureaux de poste, agences postales et relais poste - Liste
des services disponibles en bureaux de poste, agences postales et relais
poste - Calendrier d'ouverture des bureaux de poste, agences postales et
relais poste - Liste des automates en bureaux de poste - Liste des
bureaux de poste et agences postales accessibles aux personnes
handicapées

Vous découvrirez des formats de fichiers supplémentaires, des outils de
visualisation et des API sur
\url{http://datanova.legroupe.laposte.fr}.{]}(http://datanova.legroupe.laposte.fr{]}(http://datanova.legroupe.laposte.fr).)


\vspace{0.5cm}

\clearpage
\section{Les Lilas}


\begin{center}
  \includegraphics[width=3cm]{images/orga/b0_c56aee31ef40f2a2d9c64d61e4904d-100.jpg}
\end{center}


Compte officiel de la Mairie des Lilas (Seine-Saint Denis)


\vspace{0.5cm}

\needspace{12\baselineskip}
\subsection*{Comptes de la collectivité
}
  \begin{wrapfigure}{r}{2.5cm}
    \centering
    \qrcode[nolink]{https://data.gouv.fr/dataset/5369914da3a729239d203c13}
  \end{wrapfigure}

Licence : \textbf{Licence Ouverte
}\newline
Créé le : 2014-02-26\newline
Modifié le : 2015-09-11\newline
De 2000-01-01 à 2012-12-31\newline
Granularité : à la commune\newline
Mise à jour : annuelle\newline
Popularité : 1 réutilisation,  0 suivi\newline
Mots-clé : \emph{aucun
}\newline
Permalien : \url{https://data.gouv.fr/dataset/5369914da3a729239d203c13}\newline

\par
\noindent
    Chiffres Clés, Fonctionnement, Investissement, Fiscalité,
Autofinancement, Endettement (source Ministère des Finances)


\vspace{0.5cm}
\needspace{12\baselineskip}
\subsection*{Population
}
  \begin{wrapfigure}{r}{2.5cm}
    \centering
    \qrcode[nolink]{https://data.gouv.fr/dataset/53699d0ea3a729239d205b2d}
  \end{wrapfigure}

Licence : \textbf{Licence Ouverte
}\newline
Créé le : 2014-02-26\newline
Modifié le : 2015-12-08\newline
De 1968-01-01 à 2011-12-31\newline
Granularité : à la commune\newline
Mise à jour : annuelle\newline
Popularité : 1 réutilisation,  0 suivi\newline
Mots-clé : \emph{aucun
}\newline
Permalien : \url{https://data.gouv.fr/dataset/53699d0ea3a729239d205b2d}\newline

\par
\noindent
    Chiffres clés Évolution et structure de la population (données INSEE)


\vspace{0.5cm}
\needspace{3\baselineskip} \rule{4cm}{0.25pt}\newline\textbf{Aussi disponible du même producteur :}\begin{itemize}
\item \href{https://data.gouv.fr/dataset/536993cfa3a729239d2042c1}{Emploi}
\item \href{https://data.gouv.fr/dataset/53699976a3a729239d205267}{Logement}
\item \href{https://data.gouv.fr/dataset/539a702da3a7293bc2728394}{Résultats des élections dans la 9e circonscription de Seine-Saint-Denis (Les Lilas)}
\item \href{https://data.gouv.fr/dataset/54900377c751df51436accbb}{Zonages des Politiques de la Ville}
\end{itemize}

\clearpage
\section{le.taxi}


\begin{center}
  \includegraphics[width=3cm]{images/orga/70_f1beabe8fb4d13a6f6dcec6fdb5eb7-100.png}
\end{center}


Un clic, un taxi


\vspace{0.5cm}

\needspace{12\baselineskip}
\subsection*{Zones Uniques de Prises en Charge des taxis (ZUPC)
}\index{administratif}\index{administration}\index{administration!locale}\index{codes!insee}\index{commune}\index{contour}\index{decoupage}\index{geo}\index{geographie}\index{geolocalisation}\index{limite}\index{limite!administrative}\index{openstreetmap}\index{osm}\index{prefecture}\index{taxi}\index{taxis}\index{transport}
  \begin{wrapfigure}{r}{2.5cm}
    \centering
    \qrcode[nolink]{https://data.gouv.fr/dataset/562a5619c751df25dcbd3536}
  \end{wrapfigure}

Licence : \textbf{Licence Ouverte
}\newline
Créé le : 2015-10-23\newline
Modifié le : 2016-03-16\newline
De 2015-10-23 à 2016-06-05\newline
Mise à jour : quotienne\newline
Popularité : 1 réutilisation,  2 suivis\newline
Mots-clé : \emph{administratif, administration, administration-locale, codes-insee, commune, contour, decoupage, geo, geographie, geolocalisation, limite, limite-administrative, openstreetmap, osm, prefecture, taxi, taxis, transport
}\newline
Permalien : \url{https://data.gouv.fr/dataset/562a5619c751df25dcbd3536}\newline

\par
\noindent
    Ce fichier décrit les zones de prises en charges des taxis. Il doit être
couplé au fichier de contours des communes.


\vspace{0.5cm}

\clearpage
\section{Lorient Agglomération}


\begin{center}
  \includegraphics[width=3cm]{images/orga/9a_5182a77f854bf3b51897a94b4fdf22-100.jpg}
\end{center}


Lorient Agglomération regroupe 25 communes en Bretagne Sud : Brandérion,
Bubry, Calan, Caudan, Cléguer, Gâvres, Gestel, Groix, Guidel, Hennebont,
Inguiniel, Inzinzac-Lochrist, Lanester, Languidic, Lanvaudan,
Larmor-Plage, Locmiquélic, Lorient, Plœmeur, Plouay, Pont-Scorff,
Port-Louis, Quéven, Quistinic, Riantec.

Troisième agglomération bretonne, Lorient Agglomération met au coeur de
son projet ses 204 500 habitants pour en faire un territoire de vie,
dynamique et durable, ouvert sur le monde. Par délibération en date du 3
février 2015, le conseil communautaire a adopté à l'unanimité
l'engagement de Lorient Agglomération dans une démarche d'ouverture de
ses données publiques.

Accès au catalogue géographique de Lorient Agglomération
:\url{http://geolorient.isogeo.com/}


\vspace{0.5cm}

\needspace{12\baselineskip}
\subsection*{Liaisons maritimes
}
  \begin{wrapfigure}{r}{2.5cm}
    \centering
    \qrcode[nolink]{https://data.gouv.fr/dataset/55b8d97088ee3831743ca28a}
  \end{wrapfigure}

Licence : \textbf{Licence Ouverte
}\newline
Créé le : 2015-07-29\newline
Modifié le : 2016-02-11\newline
Granularité : à l'EPCI\newline
Mise à jour : annuelle\newline
Popularité : 1 réutilisation,  0 suivi\newline
Mots-clé : \emph{aucun
}\newline
Permalien : \url{https://data.gouv.fr/dataset/55b8d97088ee3831743ca28a}\newline

\par
\noindent
    Comptabilisation des passagers de l'ensemble des liaisons transrade de
Lorient Agglomération (Locmiquélic, Riantec, Port-Louis depuis 1997 et
Gâvres depuis 2002).

Statistiques des traversées. Ce fichier comporte 3 « tableaux de données
» : • le nombre de passagers depuis 2002 • l'évolution du nombre de
passagers en pourcentage depuis 2002 • Le résultat en condensé du nombre
de passagers depuis 1997


\vspace{0.5cm}
\needspace{3\baselineskip} \rule{4cm}{0.25pt}\newline\textbf{Aussi disponible du même producteur :}\begin{itemize}
\item \href{https://data.gouv.fr/dataset/5a09aa61c751df2416a4d7b8}{Accès à la mer de Lorient Agglomération}
\item \href{https://data.gouv.fr/dataset/55c21ae888ee381ab8a46ec1}{Accueil gens du voyage - tarifs 2015 et 2016}
\item \href{https://data.gouv.fr/dataset/55c2238488ee38253ca46ec1}{Accueil gens du voyage - taux occupation terrains 2014 et 2015}
\item \href{https://data.gouv.fr/dataset/56d45c99c751df3e10abb90c}{Aide installation des jeunes agriculteurs - Années 2013 et 2014}
\item \href{https://data.gouv.fr/dataset/56dd4cc3c751df511a1ffcce}{ASSAINISSEMENT  COLLECTIF :  comptes administratifs, budget primitif, état de la dette, consommations énergetiques des ouvrages}
\item \href{https://data.gouv.fr/dataset/56dd8ae4c751df21301ffccd}{ASSAINISSEMENT  NON COLLECTIF :  comptes administratifs, budget primitif}
\item \href{https://data.gouv.fr/dataset/55b8cf2088ee382df93ca289}{Autobus}
\item \href{https://data.gouv.fr/dataset/56deaf9988ee381095cfd996}{BP 2015 Subventions et concours attribués aux associations  (fonctionnement et investissement)}
\item \href{https://data.gouv.fr/dataset/56dea6ce88ee380984cfd996}{BUDGET EAU: comptes administratifs, budget primitif, état de la dette, consommation energétique des ouvrages}
\item \href{https://data.gouv.fr/dataset/56dda7dfc751df47d51ffccd}{BUDGET PRINCIPAL : comptes administratifs, budget primitif, état de la dette}
\item \href{https://data.gouv.fr/dataset/56deac8a88ee381b1acfd996}{CA 2014 Subventions attribuées aux associations fonctionnement et investissement }
\item \href{https://data.gouv.fr/dataset/5a09aa6588ee383d5c485dd2}{Cadastre de le Commune de Locmiquélic}
\item \href{https://data.gouv.fr/dataset/5a09aa83c751df2cf9c0fabd}{Cadastre graphique de Gâvres}
\item \href{https://data.gouv.fr/dataset/5a09aa66c751df2798e2ff2d}{Cadastre graphique de Groix}
\item \href{https://data.gouv.fr/dataset/5a09aa85c751df2cf9c0fabe}{Cadastre graphique de la Commune de Bubry}
\item \href{https://data.gouv.fr/dataset/5a09aa6dc751df2416a4d7ba}{Cadastre graphique de la Commune de Caudan}
\item \href{https://data.gouv.fr/dataset/5a09aa95c751df2cf9c0fac1}{Cadastre graphique de la Commune de Cléguer}
\item \href{https://data.gouv.fr/dataset/5a09aa6788ee383d5c485dd4}{Cadastre graphique de la Commune de Gestel}
\item \href{https://data.gouv.fr/dataset/5a09aa89c751df2798e2ff30}{Cadastre graphique de la Commune de Guidel}
\item \href{https://data.gouv.fr/dataset/5a09aa68c751df2416a4d7b9}{Cadastre graphique de la Commune de Languidic}
\item \href{https://data.gouv.fr/dataset/5a09aa6f88ee383f2f603768}{Cadastre graphique de la Commune de Lanvaudan}
\item \href{https://data.gouv.fr/dataset/5a09aa8b88ee383d5c485dd8}{Cadastre graphique de la Commune de Larmor-Plage}
\item \href{https://data.gouv.fr/dataset/5a09aa6d88ee383b626c5ea3}{Cadastre graphique de la Commune de Plouay}
\item \href{https://data.gouv.fr/dataset/5a09aa9188ee383b626c5ea8}{Cadastre graphique de la Commune de Port-louis}
\item \href{https://data.gouv.fr/dataset/5a09aa9ac751df2798e2ff33}{Cadastre graphique de la Commune de Quistinic}
\item \href{https://data.gouv.fr/dataset/5a09aa6d88ee383f2f603767}{Cadastre graphique de la Commune de Riantec}
\item \href{https://data.gouv.fr/dataset/5a09aa6588ee383d5c485dd3}{Cadastre graphique de la Commune d'Inguiniel}
\item \href{https://data.gouv.fr/dataset/5a09aa84c751df2ecca46396}{Cadastre graphique d'Inzinzac-Lochrist}
\item \href{https://data.gouv.fr/dataset/5a09aa6c88ee383d5c485dd5}{Cartographie de la bathymétrie de la rade de Lorient}
\item \href{https://data.gouv.fr/dataset/5a09aa8dc751df2789ef7b4b}{Cartographie des extrémités des voies de Lorient Agglomération}
\item \href{https://data.gouv.fr/dataset/56d94580c751df1ed41ffcce}{Cartographie des habitats naturels terrestre de Lorient Agglomération}
\item \href{https://data.gouv.fr/dataset/5a09aa8dc751df2ecca46398}{Cartographie des itinéraires de randonnées de Lorient Agglomération}
\item \href{https://data.gouv.fr/dataset/5a09aa7288ee383b626c5ea4}{Cartographie des points adresse de Lorient Agglomération}
\item \href{https://data.gouv.fr/dataset/5a09aa9688ee383d65fb103e}{Cartographie des tronçons des voies de Lorient Agglomération}
\item \href{https://data.gouv.fr/dataset/5a09aa70c751df2798e2ff2e}{Cartographie et étude du suivi morphologique des plages de Lorient Agglomération}
\item \href{https://data.gouv.fr/dataset/55c0938688ee386f9ba46ec1}{Chenil et fourrière animale}
\item \href{https://data.gouv.fr/dataset/56337dd488ee383c66531575}{Compte Administratif 2013 dépenses du BUDGET ASSAINISSEMENT COLLECTIF}
\item \href{https://data.gouv.fr/dataset/56337fbc88ee383729531576}{Compte Administratif 2013 dépenses du BUDGET ASSAINISSEMENT NON COLLECTIF}
\item \href{https://data.gouv.fr/dataset/5629f2eb88ee382c6812613d}{Compte Administratif 2013 dépenses du BUDGET EAU}
\item \href{https://data.gouv.fr/dataset/5633745688ee380d24531576}{Compte Administratif 2013 dépenses du BUDGET PARC D’ACTIVITES ECONOMIQUES}
\item \href{https://data.gouv.fr/dataset/5629f6e788ee383df612613d}{Compte Administratif 2013 dépenses du BUDGET PORTS DE PLAISANCE}
\item \href{https://data.gouv.fr/dataset/56029c4ec751df17d48dabee}{Compte Administratif 2013 Dépenses du BUDGET PRINCIPAL}
\item \href{https://data.gouv.fr/dataset/5629ee3c88ee3828e912613c}{Compte Administratif 2013 dépenses du BUDGET TRANSPORTS URBAIN}
\item \href{https://data.gouv.fr/dataset/5633850a88ee383c66531577}{Compte administratif 2013  - dépenses et recettes de tous les budgets}
\item \href{https://data.gouv.fr/dataset/5603acc688ee382f7f5b59ee}{Compte Administratif 2013 liste des codes fonctionnels BUDGET PRINCIPAL}
\item \href{https://data.gouv.fr/dataset/5602b107c751df40308dabee}{Compte Administratif 2013 recettes budget PRINCIPAL}
\item \href{https://data.gouv.fr/dataset/56337ece88ee383729531575}{Compte Administratif 2013 Recettes du BUDGET ASSAINISSEMENT COLLECTIF}
\item \href{https://data.gouv.fr/dataset/563380d688ee3831dc531576}{Compte Administratif 2013 Recettes du BUDGET ASSAINISSEMENT NON COLLECTIF}
\item \href{https://data.gouv.fr/dataset/5629f5ba88ee38256b12613c}{Compte Administratif 2013 Recettes du BUDGET EAU}
\item \href{https://data.gouv.fr/dataset/5633767788ee382053531575}{Compte Administratif 2013 Recettes du BUDGET PARC D’ACTIVITES ECONOMIQUES}
\item et 57 autres jeux de données\end{itemize}

\clearpage
\section{Mairie de Boé}


\begin{center}
  \includegraphics[width=3cm]{images/orga/a2_8616d48dc8442db455aa8865b88bf2-100.jpg}
\end{center}


Boé est une ville de France du Sud de l'Europe. Elle est une commune
située dans le département du Lot-et-Garonne (dans la région
Nouvelle-Aquitaine). La ville de Boé appartient au canton d'
Agen-Sud-Est et à l'arrondissement d'Agen. Coordonnées géographiques:
44\degree{}10'29.87``N - 0\degree{}38'38.80''E Altitude: 5,66 km. La
population de Boé compte 5 674 habitants au 1er janvier 2017. Les
habitants de Boé s'appellent les Boétiens.


\vspace{0.5cm}

\needspace{12\baselineskip}
\subsection*{Défibrillateurs
}\index{accident}\index{cardiaque}\index{defibrillateurs}\index{sante}\index{secours}\index{securite}
  \begin{wrapfigure}{r}{2.5cm}
    \centering
    \qrcode[nolink]{https://data.gouv.fr/dataset/59706b4ac751df3cbe8bfd34}
  \end{wrapfigure}

Licence : \textbf{Licence Ouverte
}\newline
Créé le : 2017-07-20\newline
Modifié le : 2017-10-12\newline
Granularité : au point d'intérêt\newline
Mise à jour : annuelle\newline
Popularité : 1 réutilisation,  0 suivi\newline
Mots-clé : \emph{accident, cardiaque, defibrillateurs, sante, secours, securite
}\newline
Permalien : \url{https://data.gouv.fr/dataset/59706b4ac751df3cbe8bfd34}\newline

\par
\noindent
    Liste des défibrillateurs présents sur la commune de Boé


\vspace{0.5cm}
\needspace{3\baselineskip} \rule{4cm}{0.25pt}\newline\textbf{Aussi disponible du même producteur :}\begin{itemize}
\item \href{https://data.gouv.fr/dataset/5af94f4188ee3855336341e2}{Bornes de recharge pour véhicule électrique}
\item \href{https://data.gouv.fr/dataset/596ca590c751df248b567668}{Budget Primitif 2017}
\item \href{https://data.gouv.fr/dataset/5af9337b88ee382a302194ee}{Budget primitif 2018}
\item \href{https://data.gouv.fr/dataset/596dc10dc751df6cd6bac8f9}{Bureaux de vote }
\item \href{https://data.gouv.fr/dataset/596da74ac751df42e15f44fb}{Compte Administratif 2016}
\item \href{https://data.gouv.fr/dataset/5af9345c88ee3814ac81dabb}{Compte administratif 2017}
\item \href{https://data.gouv.fr/dataset/596df7b488ee384a6742278f}{Dépenses de fonctionnement par service 2016}
\item \href{https://data.gouv.fr/dataset/5af9361688ee382fa10306d5}{Dépenses de fonctionnement par service 2017}
\item \href{https://data.gouv.fr/dataset/596dfcbf88ee384fda684ea0}{Dépenses d'investissement par service 2016}
\item \href{https://data.gouv.fr/dataset/5af9355488ee382de737618e}{Dépenses d'investissement par service 2017}
\item \href{https://data.gouv.fr/dataset/59705e30c751df26d334169f}{Écoles}
\item \href{https://data.gouv.fr/dataset/59709e49c751df0f0e8d3ca7}{Résultats du 1er tour des élections législatives 2017 }
\item \href{https://data.gouv.fr/dataset/5970b9bdc751df3b7f78d33a}{ Résultats du 1er tour des élections présidentielles 2017 }
\item \href{https://data.gouv.fr/dataset/59709f78c751df07a7397879}{Résultats du 2ème tour des élections législatives 2017 }
\item \href{https://data.gouv.fr/dataset/5970ba7ec751df3d80926ec6}{Résultats du 2ème tour des élections présidentielles 2017 }
\item \href{https://data.gouv.fr/dataset/59705982c751df2131ece1b7}{Subventions aux associations 2015}
\item \href{https://data.gouv.fr/dataset/59705644c751df1d96b19601}{Subventions aux associations 2016}
\item \href{https://data.gouv.fr/dataset/5970c858c751df537f0c5aed}{Subventions aux associations 2017}
\item \href{https://data.gouv.fr/dataset/5af9547088ee385b0e2e72ab}{Subventions aux associations 2018}
\end{itemize}

\clearpage
\section{Mairie de Cugnaux}


\begin{center}
  \includegraphics[width=3cm]{images/orga/24_cd612cd6cb4bbbb65fa876d7ff8328-100.jpg}
\end{center}


Cugnaux (en occitan Cunhaus) est une commune française située dans le
département de la Haute-Garonne, en région Occitanie.


\vspace{0.5cm}

\needspace{12\baselineskip}
\subsection*{Défibrillateurs
}\index{secours}\index{secours!d!urgence}\index{urgence}
  \begin{wrapfigure}{r}{2.5cm}
    \centering
    \qrcode[nolink]{https://data.gouv.fr/dataset/58d284c0c751df3538a279f1}
  \end{wrapfigure}

Licence : \textbf{Licence Ouverte
}\newline
Créé le : 2017-03-22\newline
Modifié le : 2017-03-23\newline
De 2017-03-22 à 2018-03-22\newline
Granularité : à la commune\newline
Popularité : 1 réutilisation,  0 suivi\newline
Mots-clé : \emph{secours, secours-d-urgence, urgence
}\newline
Permalien : \url{https://data.gouv.fr/dataset/58d284c0c751df3538a279f1}\newline

\par
\noindent
    Liste des défibrillateurs installées dans les services de la mairie.
Cette base de données est gérée par la mairie de Cugnaux.


\vspace{0.5cm}

\clearpage
\section{Mairie de Paris}


\begin{center}
  \includegraphics[width=3cm]{images/orga/b2_0e83ec3699415a87640e97c51cb006-100.png}
\end{center}


La mairie de Paris


\vspace{0.5cm}

\needspace{12\baselineskip}
\subsection*{Accessibilité des équipements de la Ville de Paris
}\index{equipements}
  \begin{wrapfigure}{r}{2.5cm}
    \centering
    \qrcode[nolink]{https://data.gouv.fr/dataset/53698e49a3a729239d20340f}
  \end{wrapfigure}

Licence : \textbf{Open Data Commons Open Database License (ODbL)
}\newline
Créé le : 2013-12-05\newline
Modifié le : 2016-03-16\newline
Popularité : 3 réutilisations,  1 suivi\newline
Mots-clé : \emph{equipements
}\newline
Permalien : \url{https://data.gouv.fr/dataset/53698e49a3a729239d20340f}\newline

\par
\noindent
    Ce jeu de données recense les équipements accessibles de la Ville.

Il détaille :\\
- le type d'équipement\\
- le nom de l'équipement\\
- l'adresse complète\\
- le type de handicap

Consultez la notice en pièce jointe, définissant les niveaux
d'accéssibilité de chaque équipement.


\vspace{0.5cm}
\needspace{12\baselineskip}
\subsection*{Evenements à Paris - Cibul
}\index{cibul}\index{concert}\index{evenements}\index{spectacle}\index{theatre}
  \begin{wrapfigure}{r}{2.5cm}
    \centering
    \qrcode[nolink]{https://data.gouv.fr/dataset/53c267fda3a7292df9d3196b}
  \end{wrapfigure}

Licence : \textbf{Open Data Commons Open Database License (ODbL)
}\newline
Créé le : 2014-07-11\newline
Modifié le : 2016-02-12\newline
Popularité : 1 réutilisation,  1 suivi\newline
Mots-clé : \emph{cibul, concert, evenements, spectacle, theatre
}\newline
Permalien : \url{https://data.gouv.fr/dataset/53c267fda3a7292df9d3196b}\newline

\par
\noindent
    Cibul est une plateforme d'agendas sociaux, géolocalisés et Open Data.

Les données de ce dataset sont extraites de l'API de
\url{http://developers.cibul.net/\%3E} Le jeu de données est mis à jour
tous les jours à 08h00


\vspace{0.5cm}
\needspace{12\baselineskip}
\subsection*{Les 1000 titres les plus recherchés au catalogue des bibliothèques de
prêt
}\index{bibliotheques}\index{documents}\index{dvd}\index{livres}\index{pret}\index{recherche}\index{references}
  \begin{wrapfigure}{r}{2.5cm}
    \centering
    \qrcode[nolink]{https://data.gouv.fr/dataset/53c26b06a3a7292df9d3196c}
  \end{wrapfigure}

Licence : \textbf{Open Data Commons Open Database License (ODbL)
}\newline
Créé le : 2014-07-11\newline
Modifié le : 2016-03-06\newline
Popularité : 1 réutilisation,  2 suivis\newline
Mots-clé : \emph{bibliotheques, documents, dvd, livres, pret, recherche, references
}\newline
Permalien : \url{https://data.gouv.fr/dataset/53c26b06a3a7292df9d3196c}\newline

\par
\noindent
    Les 1000 références les plus recherchées en 2014 au sein des
bibliothèques municipales de la ville de Paris.

La mise de ce jeu de données est semestrielle.

En pièce jointe, vous trouverez le même fichier au format xls contenant
les liens vers les oeuvres.


\vspace{0.5cm}
\needspace{12\baselineskip}
\subsection*{Les 1000 titres les plus réservés dans les bibliothèques de prêt
}\index{bibliotheques}\index{documents}\index{dvd}\index{livres}\index{pret}\index{references}\index{reservation}
  \begin{wrapfigure}{r}{2.5cm}
    \centering
    \qrcode[nolink]{https://data.gouv.fr/dataset/53c26b08a3a7292df9d3196d}
  \end{wrapfigure}

Licence : \textbf{Open Data Commons Open Database License (ODbL)
}\newline
Créé le : 2014-07-11\newline
Modifié le : 2016-03-02\newline
Popularité : 1 réutilisation,  4 suivis\newline
Mots-clé : \emph{bibliotheques, documents, dvd, livres, pret, references, reservation
}\newline
Permalien : \url{https://data.gouv.fr/dataset/53c26b08a3a7292df9d3196d}\newline

\par
\noindent
    Les 1000 références les plus réservées en 2014 au sein des bibliothèques
municipales de la ville de Paris.

La mise de ce jeu de données est semestrielle.

En pièce jointe, vous trouverez le même fichier au format xls contenant
les liens vers les oeuvres.


\vspace{0.5cm}
\needspace{12\baselineskip}
\subsection*{Les titres les plus prêtés dans les bibliothèques de prêt
}\index{bibliotheques}\index{documents}\index{dvd}\index{livres}\index{pret}\index{references}
  \begin{wrapfigure}{r}{2.5cm}
    \centering
    \qrcode[nolink]{https://data.gouv.fr/dataset/53c26bbba3a7292df9d3196e}
  \end{wrapfigure}

Licence : \textbf{Open Data Commons Open Database License (ODbL)
}\newline
Créé le : 2014-07-11\newline
Modifié le : 2015-10-13\newline
Popularité : 1 réutilisation,  3 suivis\newline
Mots-clé : \emph{bibliotheques, documents, dvd, livres, pret, references
}\newline
Permalien : \url{https://data.gouv.fr/dataset/53c26bbba3a7292df9d3196e}\newline

\par
\noindent
    Les 1000 références les plus empruntées en 2014 au sein des
bibliothèques municipales de la ville de Paris.~

\textbf{Attention, les nombres d'exemplaires ne sont pas indiqués, mais
jouent un rôle important. Ainsi les titres jeunesse sont présents en
nombre d'exemplaires très élevé, ce qui explique la prédominance de ces
titres dans la liste.}

La mise de ce jeu de données est semestrielle.

En pièce jointe, vous trouverez le même fichier au format xls contenant
les liens vers les oeuvres.


\vspace{0.5cm}
\needspace{12\baselineskip}
\subsection*{Lieux de tournage de films (long métrage)
}\index{culture}\index{film}\index{long!metrage}\index{tournage}
  \begin{wrapfigure}{r}{2.5cm}
    \centering
    \qrcode[nolink]{https://data.gouv.fr/dataset/536998b7a3a729239d205025}
  \end{wrapfigure}

Licence : \textbf{Licence Ouverte
}\newline
Créé le : 2013-09-14\newline
Modifié le : 2016-08-02\newline
Popularité : 2 réutilisations,  0 suivi\newline
Mots-clé : \emph{culture, film, long-metrage, tournage
}\newline
Permalien : \url{https://data.gouv.fr/dataset/536998b7a3a729239d205025}\newline

\par
\noindent
    Liste des lieux de tournages extérieurs de long métrage à Paris de 2002
à 2010. Avertissement : certains films ont pu changer de nom entre le
moment des tournages et leurs sorties en salle.


\vspace{0.5cm}
\needspace{12\baselineskip}
\subsection*{Lieux de tournage de films (long métrage)
}\index{cinema}\index{paris}
  \begin{wrapfigure}{r}{2.5cm}
    \centering
    \qrcode[nolink]{https://data.gouv.fr/dataset/536998b8a3a729239d205027}
  \end{wrapfigure}

Licence : \textbf{Open Data Commons Open Database License (ODbL)
}\newline
Créé le : 2013-12-05\newline
Modifié le : 2016-03-14\newline
Popularité : 3 réutilisations,  3 suivis\newline
Mots-clé : \emph{cinema, paris
}\newline
Permalien : \url{https://data.gouv.fr/dataset/536998b8a3a729239d205027}\newline

\par
\noindent
    Liste des lieux de tournages extérieurs de long métrage à Paris de 2002
à 2010. Avertissement : certains films ont pu changer de nom entre le
moment des tournages et leurs sorties en salle.


\vspace{0.5cm}
\needspace{12\baselineskip}
\subsection*{Liste des associations parisiennes
}\index{annuaire}\index{association}
  \begin{wrapfigure}{r}{2.5cm}
    \centering
    \qrcode[nolink]{https://data.gouv.fr/dataset/536998d3a3a729239d205080}
  \end{wrapfigure}

Licence : \textbf{Open Data Commons Open Database License (ODbL)
}\newline
Créé le : 2013-12-05\newline
Modifié le : 2016-03-16\newline
Popularité : 2 réutilisations,  2 suivis\newline
Mots-clé : \emph{annuaire, association
}\newline
Permalien : \url{https://data.gouv.fr/dataset/536998d3a3a729239d205080}\newline

\par
\noindent
    Liste des associations en relation avec la municipalité parisienne. Ce
jeu de données comprend le nom statutaire de l'association, son code
postal et sa ville d'enregistrement, ses secteurs d'activité, son
domaine d'activité, ses publics, son secteur géographique d'activité et
l'identifiant de sa fiche si elle est présente dans l'annuaire parisien
des associations accessible via l'adresse
:\url{http://w35-associations.apps.paris.fr/searchasso/jsp/site/Portal.jsp?page=searchasso\&id=\%5BIDASSO\%5D}Attention
: - Le fichier dispose déjà du caractère ``dollar'' comme séparateur. -
Les mêmes lignes semblent parfois se répéter plusieurs fois mais cette
répétition est justifiée par une information potentiellement différente
dans chaque ligne. Par exemple si l'association déclare dédier son
activité au public des jeunes et au public des adultes, elle sera
répétée deux fois, une fois par public. Idem pour son champs d'activité,
son secteur d'activité, son secteur géographique. Nous allons publier
cette précision dans la fiche pour éviter les malentendus.


\vspace{0.5cm}
\needspace{12\baselineskip}
\subsection*{Liste des cafés à un euro
}\index{bar}\index{bistrot}\index{cafe}\index{euro}\index{snack}\index{terrasse}
  \begin{wrapfigure}{r}{2.5cm}
    \centering
    \qrcode[nolink]{https://data.gouv.fr/dataset/536998e0a3a729239d2050a4}
  \end{wrapfigure}

Licence : \textbf{Open Data Commons Open Database License (ODbL)
}\newline
Créé le : 2014-04-23\newline
Modifié le : 2016-03-16\newline
Popularité : 2 réutilisations,  3 suivis\newline
Mots-clé : \emph{bar, bistrot, cafe, euro, snack, terrasse
}\newline
Permalien : \url{https://data.gouv.fr/dataset/536998e0a3a729239d2050a4}\newline

\par
\noindent
    La liste des cafés à un euro de Paris.

Ce jeu de données est totalement crowdsourcé. N'hésitez pas à participer
à l'enrichissement de ce jeu de données ici:
\url{https://docs.google.com/spreadsheet/viewform?formkey=dGNHWl83NzJvUm9USk1NZVkzWENRUUE6MQ\%3E}


\vspace{0.5cm}
\needspace{12\baselineskip}
\subsection*{Liste des jardins partagés à Paris
}\index{jardins}\index{partages}\index{plantes}
  \begin{wrapfigure}{r}{2.5cm}
    \centering
    \qrcode[nolink]{https://data.gouv.fr/dataset/5369991fa3a729239d205167}
  \end{wrapfigure}

Licence : \textbf{Open Data Commons Open Database License (ODbL)
}\newline
Créé le : 2014-04-23\newline
Modifié le : 2016-03-06\newline
Popularité : 1 réutilisation,  4 suivis\newline
Mots-clé : \emph{jardins, partages, plantes
}\newline
Permalien : \url{https://data.gouv.fr/dataset/5369991fa3a729239d205167}\newline

\par
\noindent
    Jeu de données listant l'ensemble des jardins partagés de Paris.


\vspace{0.5cm}
\needspace{12\baselineskip}
\subsection*{Liste des marchés de quartier
}\index{marche}
  \begin{wrapfigure}{r}{2.5cm}
    \centering
    \qrcode[nolink]{https://data.gouv.fr/dataset/5369992ca3a729239d205188}
  \end{wrapfigure}

Licence : \textbf{Open Data Commons Open Database License (ODbL)
}\newline
Créé le : 2013-12-05\newline
Modifié le : 2016-03-13\newline
Popularité : 1 réutilisation,  2 suivis\newline
Mots-clé : \emph{marche
}\newline
Permalien : \url{https://data.gouv.fr/dataset/5369992ca3a729239d205188}\newline

\par
\noindent
    Liste des marchés de quartier en 2012 avec : • Adresse du site • Jours
et horaires d'ouverture (ils sont fixes) • Type du marché : alimentaires
couverts, alimentaires découverts, fleurs, timbres, création, puces,
biologique


\vspace{0.5cm}
\needspace{12\baselineskip}
\subsection*{Liste des marchés de travaux passés par la Ville de Paris
}\index{marche}\index{travaux}\index{ville}
  \begin{wrapfigure}{r}{2.5cm}
    \centering
    \qrcode[nolink]{https://data.gouv.fr/dataset/5369992fa3a729239d205191}
  \end{wrapfigure}

Licence : \textbf{Open Data Commons Open Database License (ODbL)
}\newline
Créé le : 2014-04-23\newline
Modifié le : 2016-02-28\newline
Popularité : 1 réutilisation,  0 suivi\newline
Mots-clé : \emph{marche, travaux, ville
}\newline
Permalien : \url{https://data.gouv.fr/dataset/5369992fa3a729239d205191}\newline

\par
\noindent
    Le jeu de données comprend les marchés des travaux passés par la Ville
de Paris, de 2009 à 2013.

Ce jeu de données comprend:

\begin{itemize}
\item
  l'année
\item
  le n\degree{} national de marché
\item
  le nom du titulaire
\item
  le code postal
\item
  l'objet du marché
\item
  le montant
\item
  la date de notif
\end{itemize}


\vspace{0.5cm}
\needspace{12\baselineskip}
\subsection*{Liste des marchés de travaux passés par le Département de Paris
}\index{departement}\index{marche}\index{travaux}
  \begin{wrapfigure}{r}{2.5cm}
    \centering
    \qrcode[nolink]{https://data.gouv.fr/dataset/53699930a3a729239d205193}
  \end{wrapfigure}

Licence : \textbf{Open Data Commons Open Database License (ODbL)
}\newline
Créé le : 2014-04-23\newline
Modifié le : 2015-12-23\newline
Popularité : 1 réutilisation,  0 suivi\newline
Mots-clé : \emph{departement, marche, travaux
}\newline
Permalien : \url{https://data.gouv.fr/dataset/53699930a3a729239d205193}\newline

\par
\noindent
    Le jeu de données comprend les marchés de travaux passés par le
Département de Paris, de 2009 à 2013.

Ce jeu de données comprend:

\begin{itemize}
\item
  l'année
\item
  le n\degree{} national de marché
\item
  le nom du titulaire
\item
  le code postal
\item
  l'objet du marché
\item
  le montant
\item
  la date de notif
\end{itemize}


\vspace{0.5cm}
\needspace{12\baselineskip}
\subsection*{Liste des sites des hotspots Paris WiFi
}\index{hotspot}\index{internet}\index{navigation}\index{web}\index{wifi}
  \begin{wrapfigure}{r}{2.5cm}
    \centering
    \qrcode[nolink]{https://data.gouv.fr/dataset/53699952a3a729239d205209}
  \end{wrapfigure}

Licence : \textbf{Open Data Commons Open Database License (ODbL)
}\newline
Créé le : 2013-12-05\newline
Modifié le : 2016-03-11\newline
Popularité : 4 réutilisations,  2 suivis\newline
Mots-clé : \emph{hotspot, internet, navigation, web, wifi
}\newline
Permalien : \url{https://data.gouv.fr/dataset/53699952a3a729239d205209}\newline

\par
\noindent
    Liste des sites municipaux disposant d'un hotspot Paris WiFi permettant
une connexion à Internet offerte pendant 2h.


\vspace{0.5cm}
\needspace{12\baselineskip}
\subsection*{Mobiliers de distribution d'eau potable et non potable - Données
géographiques
}\index{eau!potable}\index{equipements}\index{mobilier!urbain}\index{paris}
  \begin{wrapfigure}{r}{2.5cm}
    \centering
    \qrcode[nolink]{https://data.gouv.fr/dataset/536999f0a3a729239d20537d}
  \end{wrapfigure}

Licence : \textbf{Open Data Commons Open Database License (ODbL)
}\newline
Créé le : 2013-12-05\newline
Modifié le : 2016-03-08\newline
Popularité : 1 réutilisation,  0 suivi\newline
Mots-clé : \emph{eau-potable, equipements, mobilier-urbain, paris
}\newline
Permalien : \url{https://data.gouv.fr/dataset/536999f0a3a729239d20537d}\newline

\par
\noindent
    Données de type vecteur décrivant divers points de distribution d'eau
potable ou non potable : fontaine wallace, borne fontaine, vanne,
bouche\ldots{}\\[2\baselineskip]\textbf{Exemple(s) de réutilisation(s)
des données :\\
} - Site web et App iPhone
\href{http://www.eaupen.net\%20/}{http://www.eaupen.net}{]}(http://www.eaupen.net{]}(http://www.eaupen.net\%20/))


\vspace{0.5cm}
\needspace{12\baselineskip}
\subsection*{Mobiliers des transports en commun - Données géographiques
}\index{equipements}\index{mobilier!urbain}\index{paris}\index{transports!en!commun}
  \begin{wrapfigure}{r}{2.5cm}
    \centering
    \qrcode[nolink]{https://data.gouv.fr/dataset/536999f3a3a729239d205383}
  \end{wrapfigure}

Licence : \textbf{Open Data Commons Open Database License (ODbL)
}\newline
Créé le : 2013-12-05\newline
Modifié le : 2016-03-13\newline
Popularité : 1 réutilisation,  1 suivi\newline
Mots-clé : \emph{equipements, mobilier-urbain, paris, transports-en-commun
}\newline
Permalien : \url{https://data.gouv.fr/dataset/536999f3a3a729239d205383}\newline

\par
\noindent
    Données de type vecteur, décrivant les mobiliers liés aux transports en
commun, notamment bus\\[2\baselineskip]\textbf{Exemple(s) de
réutilisation(s) des données:\\
} - Application Android ``VuzZz''
\href{https://play.google.com/store/apps/details?id=com.vuzzz.android\&feature=search_result\#?t=W251bGwsMSwyLDEsImNvbS52dXp6ei5hbmRyb2lkIl0.}{Lien
Google Play}


\vspace{0.5cm}
\needspace{12\baselineskip}
\subsection*{Postes publics des bibliothèques
}\index{bibliotheques}\index{catalogues!en!ligne}\index{culture}\index{ordinateurs}\index{postes!publics}
  \begin{wrapfigure}{r}{2.5cm}
    \centering
    \qrcode[nolink]{https://data.gouv.fr/dataset/59cc758ba3a72921191db6c4}
  \end{wrapfigure}

Licence : \textbf{Open Data Commons Open Database License (ODbL)
}\newline
Créé le : 2017-09-28\newline
Modifié le : 2019-03-17\newline
Popularité : 1 réutilisation,  0 suivi\newline
Mots-clé : \emph{bibliotheques, catalogues-en-ligne, culture, ordinateurs, postes-publics
}\newline
Permalien : \url{https://data.gouv.fr/dataset/59cc758ba3a72921191db6c4}\newline

\par
\noindent
    Liste, au 25 juillet 2018 des postes publics des bibliothèques et leur
répartition par type : consultation du catalogue, postes adultes, postes
jeunesse, postes chercheur\ldots{} Ces données sont susceptibles de
varier.


\vspace{0.5cm}
\needspace{12\baselineskip}
\subsection*{Postes publics des bibliothèques de prêt
}\index{bibliotheques}\index{catalogues!en!ligne}\index{culture}\index{ordinateurs}\index{postes!publics}\index{pret}
  \begin{wrapfigure}{r}{2.5cm}
    \centering
    \qrcode[nolink]{https://data.gouv.fr/dataset/59590d24a3a7291dcf9c8028}
  \end{wrapfigure}

Licence : \textbf{Open Data Commons Open Database License (ODbL)
}\newline
Créé le : 2017-07-02\newline
Modifié le : 2018-07-03\newline
Popularité : 1 réutilisation,  0 suivi\newline
Mots-clé : \emph{bibliotheques, catalogues-en-ligne, culture, ordinateurs, postes-publics, pret
}\newline
Permalien : \url{https://data.gouv.fr/dataset/59590d24a3a7291dcf9c8028}\newline

\par
\noindent
    Liste, au 25 juillet 2017 des postes publics des bibliothèques de prêt
et leur répartition par type : consultation du catalogue, postes
adultes, postes jeunesse\ldots{} Ces données sont susceptibles de
varier.


\vspace{0.5cm}
\needspace{12\baselineskip}
\subsection*{Stations et espaces AutoLib de la métropole parisienne
}\index{autolib}\index{bornes}\index{espaces}\index{irve}\index{stations}\index{transport}
  \begin{wrapfigure}{r}{2.5cm}
    \centering
    \qrcode[nolink]{https://data.gouv.fr/dataset/5369a041a3a729239d20630e}
  \end{wrapfigure}

Licence : \textbf{Open Data Commons Open Database License (ODbL)
}\newline
Créé le : 2014-04-23\newline
Modifié le : 2017-07-26\newline
Popularité : 1 réutilisation,  4 suivis\newline
Mots-clé : \emph{autolib, bornes, espaces, irve, stations, transport
}\newline
Permalien : \url{https://data.gouv.fr/dataset/5369a041a3a729239d20630e}\newline

\par
\noindent
    Ce jeu de données comprend:~

\begin{itemize}
\item
  l'Identifiant DSP
\item
  le nom de la station
\item
  le Code postal Ville
\item
  Latitude \& Longitude
\item
  le type de station
\item
  l'emplacement
\item
  les places~Autolib'~
\item
  les places ``Recharge tiers''
\item
  le nombre total de places
\end{itemize}


\vspace{0.5cm}
\needspace{12\baselineskip}
\subsection*{Stations Vélib - Disponibilités en temps réel
}\index{stations}\index{velib}\index{velo}
  \begin{wrapfigure}{r}{2.5cm}
    \centering
    \qrcode[nolink]{https://data.gouv.fr/dataset/53c2740ba3a7292df9d31973}
  \end{wrapfigure}

Licence : \textbf{Licence Ouverte
}\newline
Créé le : 2014-07-11\newline
Modifié le : 2016-03-09\newline
Popularité : 3 réutilisations,  1 suivi\newline
Mots-clé : \emph{stations, velib, velo
}\newline
Permalien : \url{https://data.gouv.fr/dataset/53c2740ba3a7292df9d31973}\newline

\par
\noindent
    Ce jeu de données donne les disponibilités en temps réel des stations
Vélib. La mise à jour est faite à la minute.

Pour avoir accès aux données Velib en direct,~accédez à l'API velib à
\url{https://developer.jcdecaux.com/}


\vspace{0.5cm}
\needspace{12\baselineskip}
\subsection*{Tonnages des déchets bacs verts
}\index{environnement}
  \begin{wrapfigure}{r}{2.5cm}
    \centering
    \qrcode[nolink]{https://data.gouv.fr/dataset/5369a238a3a729239d2067b9}
  \end{wrapfigure}

Licence : \textbf{Licence Ouverte
}\newline
Créé le : 2013-09-14\newline
Modifié le : 2015-07-28\newline
De 2011-01-01 à 2011-12-31\newline
Granularité : à la commune\newline
Popularité : 1 réutilisation,  0 suivi\newline
Mots-clé : \emph{environnement
}\newline
Permalien : \url{https://data.gouv.fr/dataset/5369a238a3a729239d2067b9}\newline

\par
\noindent
    Tonnages des déchets bacs verts collectés par arrondissement et par
mois.


\vspace{0.5cm}
\needspace{12\baselineskip}
\subsection*{Trottoirs des rues de Paris
}\index{pieton}\index{rue}\index{trottoir}\index{voie}
  \begin{wrapfigure}{r}{2.5cm}
    \centering
    \qrcode[nolink]{https://data.gouv.fr/dataset/5369a30da3a729239d20698f}
  \end{wrapfigure}

Licence : \textbf{Open Data Commons Open Database License (ODbL)
}\newline
Créé le : 2014-04-23\newline
Modifié le : 2016-03-16\newline
Popularité : 1 réutilisation,  2 suivis\newline
Mots-clé : \emph{pieton, rue, trottoir, voie
}\newline
Permalien : \url{https://data.gouv.fr/dataset/5369a30da3a729239d20698f}\newline

\par
\noindent
    Données de type vecteur décrivant les trottoirs parisiens. Exemple(s) de
réutilisation(s) des données : - Cartographie
sur\url{http://demo.3liz.fr/opendataparis}


\vspace{0.5cm}
\needspace{12\baselineskip}
\subsection*{Volumes bâtis - Données géographiques
}\index{construction}\index{paris}
  \begin{wrapfigure}{r}{2.5cm}
    \centering
    \qrcode[nolink]{https://data.gouv.fr/dataset/5369a384a3a729239d206aa9}
  \end{wrapfigure}

Licence : \textbf{Open Data Commons Open Database License (ODbL)
}\newline
Créé le : 2013-12-05\newline
Modifié le : 2016-03-13\newline
Popularité : 1 réutilisation,  0 suivi\newline
Mots-clé : \emph{construction, paris
}\newline
Permalien : \url{https://data.gouv.fr/dataset/5369a384a3a729239d206aa9}\newline

\par
\noindent
    Données de type vecteur qui décrivent les bâtiments représentés sur le
plan parcellaire


\vspace{0.5cm}
\needspace{3\baselineskip} \rule{4cm}{0.25pt}\newline\textbf{Aussi disponible du même producteur :}\begin{itemize}
\item \href{https://data.gouv.fr/dataset/53698e0da3a729239d203379}{1000 références les plus empruntées en 2011}
\item \href{https://data.gouv.fr/dataset/53698e0da3a729239d20337a}{1000 titres les plus empruntés par bibliothèque 2011}
\item \href{https://data.gouv.fr/dataset/59590d1ba3a7291dd09c7ffb}{Accessibilité des équipements de la Ville de Paris}
\item \href{https://data.gouv.fr/dataset/59590d4fa3a7291dd09c8022}{Accidentologie}
\item \href{https://data.gouv.fr/dataset/53698e7aa3a729239d203495}{Adresse des panneaux d'affichages des Conseils de Quartier}
\item \href{https://data.gouv.fr/dataset/53698e7ba3a729239d20349c}{Adresses des panneaux d'affichage des Conseils de Quartier}
\item \href{https://data.gouv.fr/dataset/5a7a9331a3a72906378d9a8b}{Aide Paris Forfait Famille}
\item \href{https://data.gouv.fr/dataset/5a7a9314b595087c8309c628}{Aide Paris Pass Famille}
\item \href{https://data.gouv.fr/dataset/5a87d126a3a72976e0ba013e}{Allocation soutien parents enfants handicapés (ASPEH)}
\item \href{https://data.gouv.fr/dataset/53698ee0a3a729239d2035ba}{Annuaire Immobilier de l'Enseignement Superieur}
\item \href{https://data.gouv.fr/dataset/53698f01a3a729239d203610}{Arbres d'alignement - Données géographiques}
\item \href{https://data.gouv.fr/dataset/53698f00a3a729239d20360f}{Arbres d'alignement - Données géographiques}
\item \href{https://data.gouv.fr/dataset/53698f06a3a729239d203620}{Arbres remarquables - Données Géographiques}
\item \href{https://data.gouv.fr/dataset/53698f0ca3a729239d203630}{Arrêtés municipaux d'insalubrité}
\item \href{https://data.gouv.fr/dataset/53698f0ca3a729239d20362f}{Arrêtés municipaux d'insalubrité - 2010}
\item \href{https://data.gouv.fr/dataset/53698f0ba3a729239d20362e}{Arrêtés municipaux d'insalubrité - 2010}
\item \href{https://data.gouv.fr/dataset/53698f22a3a729239d20366b}{Attractions foraines sur l'espace public}
\item \href{https://data.gouv.fr/dataset/53698f21a3a729239d203669}{Attractions foraines sur l'espace public}
\item \href{https://data.gouv.fr/dataset/53698f20a3a729239d203667}{Attractions foraines sur l'espace public en 2011}
\item \href{https://data.gouv.fr/dataset/53698f21a3a729239d203668}{Attractions foraines sur l'espace public en 2011}
\item \href{https://data.gouv.fr/dataset/53698f22a3a729239d20366a}{Attractions foraines sur l'espace public parisien}
\item \href{https://data.gouv.fr/dataset/595f1746a3a7296408d6a3e2}{Autolib - Disponibilité temps réel}
\item \href{https://data.gouv.fr/dataset/53698f26a3a729239d203676}{Autorisation d'urbanisme}
\item \href{https://data.gouv.fr/dataset/59a8db80a3a7294680fd9e37}{Autorisations d'urbanisme récentes (v2)}
\item \href{https://data.gouv.fr/dataset/53698f60a3a729239d203725}{Bâti - Données géographiques}
\item \href{https://data.gouv.fr/dataset/5abb3856b595081333b71ba0}{Bornes Belib - Disponibilité en temps réel}
\item \href{https://data.gouv.fr/dataset/53698faba3a729239d2037e0}{Bornes et Arceaux de Stationnement à Paris}
\item \href{https://data.gouv.fr/dataset/53698faba3a729239d2037e1}{Bornes et Arceaux de Stationnement à Paris}
\item \href{https://data.gouv.fr/dataset/59590cf9a3a7291dd09c7fde}{Bornes et Arceaux de Stationnement à Paris}
\item \href{https://data.gouv.fr/dataset/595b1f20a3a7291dd09c8559}{Budgets Votés AP (Autorisations de Programmes)}
\item \href{https://data.gouv.fr/dataset/595b1f22a3a7291dcf9c8654}{Budgets Votés Etats Spéciaux d'Arrondissements}
\item \href{https://data.gouv.fr/dataset/5c5124c99ce2e7624b275b6c}{Budgets votés Etats Spéciaux d’Arrondissements (M57)}
\item \href{https://data.gouv.fr/dataset/536998dea3a729239d2050a0}{Bureaux de vote}
\item \href{https://data.gouv.fr/dataset/5bee436d9ce2e7206d830ff1}{CASVP Demandeurs en Résidence Appartement et Résidence Service par tranche d 'âge}
\item \href{https://data.gouv.fr/dataset/5baf373d9ce2e752a360eb0a}{CASVP - Nombre de bénéficiaires en situation de handicap sur les principales aides du CASVP}
\item \href{https://data.gouv.fr/dataset/5bd7e9f606e3e754538c0eba}{CASVP - Nombre de bénéficiaires Paris Logement Familles}
\item \href{https://data.gouv.fr/dataset/53699041a3a729239d203969}{Catalogue des cours municipaux d'adultes}
\item \href{https://data.gouv.fr/dataset/53699040a3a729239d203968}{Catalogue des cours municipaux d'adultes}
\item \href{https://data.gouv.fr/dataset/53699074a3a729239d2039e7}{Cheminement d'assainissement}
\item \href{https://data.gouv.fr/dataset/53699074a3a729239d2039e6}{Cheminement d'assainissement}
\item \href{https://data.gouv.fr/dataset/53c26323a3a7292df9d3196a}{Collections des bibliothèques et fonds spécialisés et patrimoniaux}
\item \href{https://data.gouv.fr/dataset/59590d02a3a7291dd09c7fe6}{Comptes Administratifs Principaux (Ville-Département)}
\item \href{https://data.gouv.fr/dataset/5369917ba3a729239d203c8a}{Concessions dans les jardins}
\item \href{https://data.gouv.fr/dataset/5369917ba3a729239d203c89}{Concessions dans les jardins}
\item \href{https://data.gouv.fr/dataset/536991aca3a729239d203d05}{Consultations des centres de santé}
\item \href{https://data.gouv.fr/dataset/536991aaa3a729239d203d01}{Consultations des centres de santé à Paris}
\item \href{https://data.gouv.fr/dataset/536991aba3a729239d203d04}{Consultations des centres de santé à Paris}
\item \href{https://data.gouv.fr/dataset/536991aba3a729239d203d03}{Consultations des centres de santé à Paris}
\item \href{https://data.gouv.fr/dataset/536992d2a3a729239d204014}{Détail du Bâti}
\item \href{https://data.gouv.fr/dataset/536992d3a3a729239d204015}{Détail du Bâti}
\item et 202 autres jeux de données\end{itemize}

\clearpage
\section{Mairie de Toulouse}


\begin{center}
  \includegraphics[width=3cm]{images/orga/32_f8ac4e146c47b2b9a01a488709cff6-100.png}
\end{center}
\needspace{12\baselineskip}
\subsection*{Bureaux de vote 2012
}\index{bureau}\index{canton}\index{circonscription}\index{decoupage}\index{election}\index{localisation}\index{public}\index{vote}
  \begin{wrapfigure}{r}{2.5cm}
    \centering
    \qrcode[nolink]{https://data.gouv.fr/dataset/53698fdea3a729239d203865}
  \end{wrapfigure}

Licence : \textbf{Open Data Commons Open Database License (ODbL)
}\newline
Créé le : 2013-11-18\newline
Modifié le : 2016-01-08\newline
Mise à jour : annuelle\newline
Popularité : 1 réutilisation,  0 suivi\newline
Mots-clé : \emph{bureau, canton, circonscription, decoupage, election, localisation, public, vote
}\newline
Permalien : \url{https://data.gouv.fr/dataset/53698fdea3a729239d203865}\newline

\par
\noindent
    Emplacements des bureaux de vote en 2012


\vspace{0.5cm}
\needspace{12\baselineskip}
\subsection*{Crèches
}\index{collective}\index{creche}\index{enfant}\index{equipement}\index{garderie}\index{halte}\index{petite!enfance}\index{public}
  \begin{wrapfigure}{r}{2.5cm}
    \centering
    \qrcode[nolink]{https://data.gouv.fr/dataset/536991ffa3a729239d203de2}
  \end{wrapfigure}

Licence : \textbf{Open Data Commons Open Database License (ODbL)
}\newline
Créé le : 2013-11-18\newline
Modifié le : 2016-01-15\newline
Mise à jour : hebdomadaire\newline
Popularité : 1 réutilisation,  0 suivi\newline
Mots-clé : \emph{collective, creche, enfant, equipement, garderie, halte, petite-enfance, public
}\newline
Permalien : \url{https://data.gouv.fr/dataset/536991ffa3a729239d203de2}\newline

\par
\noindent
    Localisation des crèches sur le territoire de Toulouse


\vspace{0.5cm}
\needspace{12\baselineskip}
\subsection*{Gymnases
}\index{gymnases}\index{sport}
  \begin{wrapfigure}{r}{2.5cm}
    \centering
    \qrcode[nolink]{https://data.gouv.fr/dataset/5369960ba3a729239d2048bd}
  \end{wrapfigure}

Licence : \textbf{Open Data Commons Open Database License (ODbL)
}\newline
Créé le : 2013-11-18\newline
Modifié le : 2014-11-05\newline
Mise à jour : hebdomadaire\newline
Popularité : 1 réutilisation,  0 suivi\newline
Mots-clé : \emph{gymnases, sport
}\newline
Permalien : \url{https://data.gouv.fr/dataset/5369960ba3a729239d2048bd}\newline

\par
\noindent
    Localisation des gymnases gérés par le service des Sports sur la commune
de Toulouse


\vspace{0.5cm}
\needspace{12\baselineskip}
\subsection*{Musée
}\index{abattoirs}\index{arts}\index{augustins}\index{culture}\index{equipement}\index{exposition}\index{jacobins}\index{loisir}\index{musee}\index{peinture}\index{public}\index{raymond}\index{sculpture}\index{tourisme}
  \begin{wrapfigure}{r}{2.5cm}
    \centering
    \qrcode[nolink]{https://data.gouv.fr/dataset/53699a45a3a729239d205450}
  \end{wrapfigure}

Licence : \textbf{Open Data Commons Open Database License (ODbL)
}\newline
Créé le : 2013-11-18\newline
Modifié le : 2015-12-08\newline
Mise à jour : annuelle\newline
Popularité : 1 réutilisation,  2 suivis\newline
Mots-clé : \emph{abattoirs, arts, augustins, culture, equipement, exposition, jacobins, loisir, musee, peinture, public, raymond, sculpture, tourisme
}\newline
Permalien : \url{https://data.gouv.fr/dataset/53699a45a3a729239d205450}\newline

\par
\noindent
    Localisation des Musées sur le territoire de Toulouse


\vspace{0.5cm}
\needspace{12\baselineskip}
\subsection*{Parcellaire de 1680
}\index{1680}\index{archive}\index{cadastre}\index{capitoulat}\index{histoire}\index{moulon}\index{municipale}\index{parcelle}\index{toulouse}\index{ville}
  \begin{wrapfigure}{r}{2.5cm}
    \centering
    \qrcode[nolink]{https://data.gouv.fr/dataset/53699b76a3a729239d20573d}
  \end{wrapfigure}

Licence : \textbf{Open Data Commons Open Database License (ODbL)
}\newline
Créé le : 2013-11-18\newline
Modifié le : 2015-03-15\newline
Mise à jour : annuelle\newline
Popularité : 1 réutilisation,  0 suivi\newline
Mots-clé : \emph{1680, archive, cadastre, capitoulat, histoire, moulon, municipale, parcelle, toulouse, ville
}\newline
Permalien : \url{https://data.gouv.fr/dataset/53699b76a3a729239d20573d}\newline

\par
\noindent
    Parcellaire historique de Toulouse autour de 1680


\vspace{0.5cm}
\needspace{12\baselineskip}
\subsection*{Théâtre
}\index{capitole}\index{culture}\index{equipement}\index{loisir}\index{musique}\index{opera}\index{piece}\index{public}\index{spectacle}\index{theatre}
  \begin{wrapfigure}{r}{2.5cm}
    \centering
    \qrcode[nolink]{https://data.gouv.fr/dataset/5369a228a3a729239d20678a}
  \end{wrapfigure}

Licence : \textbf{Open Data Commons Open Database License (ODbL)
}\newline
Créé le : 2013-11-18\newline
Modifié le : 2016-03-12\newline
Mise à jour : annuelle\newline
Popularité : 1 réutilisation,  1 suivi\newline
Mots-clé : \emph{capitole, culture, equipement, loisir, musique, opera, piece, public, spectacle, theatre
}\newline
Permalien : \url{https://data.gouv.fr/dataset/5369a228a3a729239d20678a}\newline

\par
\noindent
    Localisation des théâtre sur le territoire de Toulouse


\vspace{0.5cm}
\needspace{3\baselineskip} \rule{4cm}{0.25pt}\newline\textbf{Aussi disponible du même producteur :}\begin{itemize}
\item \href{https://data.gouv.fr/dataset/53698fada3a729239d2037e6}{Boulodromes}
\item \href{https://data.gouv.fr/dataset/53698fd6a3a729239d203850}{Budget Primitif commune de Toulouse 2013}
\item \href{https://data.gouv.fr/dataset/53698fdaa3a729239d203858}{Budget Primitif Ville de Toulouse}
\item \href{https://data.gouv.fr/dataset/53698fdfa3a729239d203866}{Bureaux de vote découpage}
\item \href{https://data.gouv.fr/dataset/53699008a3a729239d2038d3}{Cantons électoraux Haute Garonne}
\item \href{https://data.gouv.fr/dataset/53699041a3a729239d20396a}{Catalogue des données - Dataset}
\item \href{https://data.gouv.fr/dataset/5369905ba3a729239d2039ab}{Centres de vaccination}
\item \href{https://data.gouv.fr/dataset/5369905fa3a729239d2039b3}{Centres sociaux}
\item \href{https://data.gouv.fr/dataset/536990b2a3a729239d203a83}{Circonscriptions électorales Haute Garonne}
\item \href{https://data.gouv.fr/dataset/536990c1a3a729239d203aab}{Clubs du 3ème age}
\item \href{https://data.gouv.fr/dataset/5369913da3a729239d203beb}{Compte administratif 2012 de la Ville de Toulouse}
\item \href{https://data.gouv.fr/dataset/536991e0a3a729239d203d95}{Courts de tennis}
\item \href{https://data.gouv.fr/dataset/53699239a3a729239d203e84}{Défibrillateurs}
\item \href{https://data.gouv.fr/dataset/53699358a3a729239d204180}{Ecoles élémentaires publiques}
\item \href{https://data.gouv.fr/dataset/53699358a3a729239d204181}{Ecoles maternelles publiques}
\item \href{https://data.gouv.fr/dataset/5369937ca3a729239d2041e2}{Election présidentielle 2012 1er et  2 ème tours a Toulouse (Résultats)}
\item \href{https://data.gouv.fr/dataset/5369937ea3a729239d2041ea}{Elections au parlement européen a Toulouse (Résultats)}
\item \href{https://data.gouv.fr/dataset/53699382a3a729239d2041f5}{Elections cantonales 2008 1er et  2 ème tours a Toulouse (Résultats)}
\item \href{https://data.gouv.fr/dataset/53699384a3a729239d2041f9}{Elections cantonales 2011 1er et  2 ème tours a Toulouse (Résultats)}
\item \href{https://data.gouv.fr/dataset/5369938fa3a729239d204215}{Elections législatives 2012 1er e t2eme tours a Toulouse (Résultats)}
\item \href{https://data.gouv.fr/dataset/53699392a3a729239d20421d}{Elections municipales 2001 1er et  2 ème tours a Toulouse (Résultats)}
\item \href{https://data.gouv.fr/dataset/53699393a3a729239d204222}{Elections municipales 2008 1er et  2 ème tours a Toulouse (Résultats)}
\item \href{https://data.gouv.fr/dataset/536993a6a3a729239d20425a}{Elections régionales 2010 1er et  2 ème tours a Toulouse (Résultats)}
\item \href{https://data.gouv.fr/dataset/536994f8a3a729239d20459b}{Etablissements Recevant du Public}
\item \href{https://data.gouv.fr/dataset/53699725a3a729239d204bff}{Installations sportives de la ville de Toulouse}
\item \href{https://data.gouv.fr/dataset/5369975aa3a729239d204c89}{Jardins partagés}
\item \href{https://data.gouv.fr/dataset/5369998aa3a729239d205297}{Ludothèques}
\item \href{https://data.gouv.fr/dataset/536999a7a3a729239d2052d0}{Mairies annexes}
\item \href{https://data.gouv.fr/dataset/536999bca3a729239d205304}{Marchés publics 2012 Ville de Toulouse}
\item \href{https://data.gouv.fr/dataset/536999d3a3a729239d205339}{Médiatheques, Bibliothèques et Bibliobus}
\item \href{https://data.gouv.fr/dataset/53699af6a3a729239d2055fd}{Nomenclature des Voies}
\item \href{https://data.gouv.fr/dataset/53699b6ba3a729239d205725}{Panneaux d'expression libre}
\item \href{https://data.gouv.fr/dataset/53699b76a3a729239d20573e}{Parcellaire de1830}
\item \href{https://data.gouv.fr/dataset/53699bd8a3a729239d20582d}{Patinoires}
\item \href{https://data.gouv.fr/dataset/53699c8fa3a729239d2059e1}{Piscines municipales}
\item \href{https://data.gouv.fr/dataset/53699d9ca3a729239d205c95}{prénoms declares a l'Etat civil de Toulouse de 2003 a 2012}
\item \href{https://data.gouv.fr/dataset/53699e7fa3a729239d205eba}{Quartiers de proximité}
\item \href{https://data.gouv.fr/dataset/53699f26a3a729239d206073}{Restaurants séniors}
\item \href{https://data.gouv.fr/dataset/53699fc3a3a729239d2061e3}{Secteurs de proximité}
\item \href{https://data.gouv.fr/dataset/5369a007a3a729239d206288}{Skateparks}
\item \href{https://data.gouv.fr/dataset/5369a034a3a729239d2062ee}{Stades}
\item \href{https://data.gouv.fr/dataset/5369a039a3a729239d2062fc}{Stations d'auto partage}
\end{itemize}

\clearpage
\section{Météo-France}


\begin{center}
  \includegraphics[width=3cm]{images/orga/44_3f1d6b9eac45b2bca7ae9b2a80d812-100.jpg}
\end{center}


\textbf{Qui sommes-nous ?} Météo-France, service météorologique et
climatique national, est un établissement public de l'Etat. Il joue un
rôle opérationnel en matière de sécurité météorologique des personnes et
des biens, d'appui météorologique aux forces armées et, en tant que
prestataire désigné par l'Etat, de fournisseur de services
météorologiques à la navigation aérienne.

Météo-France joue également un rôle de premier plan dans l'étude du
climat et de ses évolutions en produisant des scénarios de changement
climatique et met son expertise au service des politiques d'adaptation.

Ces activités opérationnelles s'adossent à une activité de recherche sur
le comportement de l'atmosphère et du système climatique et de formation
et de diffusion de la connaissance en sciences météorologiques.

Météo-France maîtrise ainsi une chaîne de métiers qui va de la recherche
à la prévision du temps et au développement de services climatiques, en
passant par le développement d'infrastructures d'observation, de
transmission de données et de calcul intensif.

Pour en savoir plus :
\href{http://www.meteofrance.fr}{www.meteofrance.fr}

\textbf{Les données publiques de Météo-France :} Météo-France met à la
disposition de tous les utilisateurs les données produites dans le cadre
de ses missions de service public dans son
\href{https://donneespubliques.meteofrance.fr/}{Portail des données
publiques}.

La réutilisation de certaines données est gratuite, pour d'autres elle
se fait contre paiement de redevances, permettant aux utilisateurs
d'apporter une juste contribution aux frais d'infrastructure, de
collecte, de transformation et de mise à disposition des données
supportés par Météo-France. \emph{Les données ci-dessous sont
gratuites.}


\vspace{0.5cm}

\needspace{12\baselineskip}
\subsection*{Diffusion ``temps réel'' de la vigilance météorologique en métropole
}\index{meteorologie}\index{securite!publique}\index{vigilance}
  \begin{wrapfigure}{r}{2.5cm}
    \centering
    \qrcode[nolink]{https://data.gouv.fr/dataset/57ecdd1988ee3845045ff490}
  \end{wrapfigure}

Licence : \textbf{Licence Ouverte
}\newline
Créé le : 2016-09-29\newline
Modifié le : 2016-09-29\newline
Granularité : au département\newline
Popularité : 2 réutilisations,  4 suivis\newline
Mots-clé : \emph{meteorologie, securite-publique, vigilance
}\newline
Permalien : \url{https://data.gouv.fr/dataset/57ecdd1988ee3845045ff490}\newline

\par
\noindent
    L'information mise à disposition de l'usager est constituée d'un fichier
unique zippé qui est mis à jour chaque minute et qui contient l'ensemble
des fichiers constituant le flux vigilance dont on trouvera le détail à
la suite. Le flux vigilance est constitué par :\\
• Les fichiers des données de l'état de la vigilance au format XML • Les
cartes de vigilance au format image • L'ensemble des bulletins vigilance
départementaux au format XML • L'ensemble des bulletins vigilance
avalanche départementaux au format XML • Les bulletins vigilance
nationale au format XML Chacun des produits est diffusé aussi souvent
que l'exige la situation météorologique.


\vspace{0.5cm}
\needspace{12\baselineskip}
\subsection*{Données d'observations du réseau nivo-météorologique
}\index{meteo}\index{neige}\index{temperatures}\index{vent}
  \begin{wrapfigure}{r}{2.5cm}
    \centering
    \qrcode[nolink]{https://data.gouv.fr/dataset/546a0fd8c751df054d6c8d07}
  \end{wrapfigure}

Licence : \textbf{Licence Ouverte
}\newline
Créé le : 2014-11-17\newline
Modifié le : 2016-03-10\newline
Popularité : 1 réutilisation,  0 suivi\newline
Mots-clé : \emph{meteo, neige, temperatures, vent
}\newline
Permalien : \url{https://data.gouv.fr/dataset/546a0fd8c751df054d6c8d07}\newline

\par
\noindent
    Données d'observations météorologiques issues des stations de montagne
opérées par les partenaires conventionnés avec Météo-France pour la
surveillance du manteau neigeux en période hivernale. Elles sont
disponibles depuis décembre 2010.

Paramètres : direction et force du vent, température de l'air, humidité,
hauteurs de neige et divers paramètres de caractérisation de la neige
accumulée au sol.

Metropole (Montagne) - Fréquence : 12 h - Format : ASCII


\vspace{0.5cm}
\needspace{12\baselineskip}
\subsection*{Données du modèle atmosphérique AROME à aire limitée à haute résolution
}\index{analyse}\index{humidite}\index{meteo}\index{modelisation!numerique}\index{nebulosite}\index{precipitations}\index{previsions!meteo}\index{rayonnement}\index{temperature}\index{vent}
  \begin{wrapfigure}{r}{2.5cm}
    \centering
    \qrcode[nolink]{https://data.gouv.fr/dataset/55b0f09388ee3838d73ca288}
  \end{wrapfigure}

Licence : \textbf{Licence Ouverte
}\newline
Créé le : 2015-07-23\newline
Modifié le : 2016-09-13\newline
Popularité : 3 réutilisations,  1 suivi\newline
Mots-clé : \emph{analyse, humidite, meteo, modelisation-numerique, nebulosite, precipitations, previsions-meteo, rayonnement, temperature, vent
}\newline
Permalien : \url{https://data.gouv.fr/dataset/55b0f09388ee3838d73ca288}\newline

\par
\noindent
    Champs d'analyse et de prévisions en points de grilles, issus du modèle
atmosphérique Arome sur la métropole.

Choix des paramètres, niveaux, échéances jusqu'à H+36h ou 42h.

Format : GRIB V2 - Résolution : 0,025\degree{} (env. 2,5 km) et
0.01\degree{} (en 1.3km)

Les données de Météo-France concernées ici sont exploitables et
interopérables en conformité avec les standards de l'OGC (Open
Geospatial Consortium) et la directive INSPIRE


\vspace{0.5cm}
\needspace{12\baselineskip}
\subsection*{Données du modèle atmosphérique global ARPEGE
}\index{analyses}\index{humidite}\index{meteo}\index{modelisation!numerique}\index{nebulosite}\index{precipitations}\index{previsions!meteo}\index{rayonnement}\index{temperature}\index{vent}
  \begin{wrapfigure}{r}{2.5cm}
    \centering
    \qrcode[nolink]{https://data.gouv.fr/dataset/55b0f44088ee3838d13ca28a}
  \end{wrapfigure}

Licence : \textbf{Licence Ouverte
}\newline
Créé le : 2015-07-23\newline
Modifié le : 2016-09-13\newline
Popularité : 4 réutilisations,  0 suivi\newline
Mots-clé : \emph{analyses, humidite, meteo, modelisation-numerique, nebulosite, precipitations, previsions-meteo, rayonnement, temperature, vent
}\newline
Permalien : \url{https://data.gouv.fr/dataset/55b0f44088ee3838d13ca28a}\newline

\par
\noindent
    Champs d'analyse et de prévision en points de grille régulière sur
différents domaines géographiques issus du modèle de prévision
atmosphérique global français Arpège.

Paramètres, niveaux, échéances et sous-domaines paramétrables selon
diverses résolutions et échéances jusqu'à H+102 h maximum.

Monde - Fréquence : 6 h - Format : GRIB V2 - Résolution : 0,5\degree{}
sur le monde/0,1\degree{} sur l'Europe

Les données de Météo-France concernées ici sont exploitables et
interopérables en conformité avec les standards de l'OGC (Open
Geospatial Consortium) et la directive INSPIRE


\vspace{0.5cm}
\needspace{12\baselineskip}
\subsection*{Indices mensuels de précipitations et nombre de jours de précipitations
issus du modèle Aladin-Climat
}\index{aladin}\index{cnrm}\index{france!metropolitaine}\index{mensuel}\index{precipitations}\index{projections}\index{reference}
  \begin{wrapfigure}{r}{2.5cm}
    \centering
    \qrcode[nolink]{https://data.gouv.fr/dataset/560c266488ee3807bc83905e}
  \end{wrapfigure}

Licence : \textbf{Licence Ouverte
}\newline
Créé le : 2015-09-30\newline
Modifié le : 2016-03-08\newline
De 1976-01-01 à 2005-12-31\newline
Granularité : au point d'intérêt\newline
Mise à jour : ponctuelle\newline
Popularité : 1 réutilisation,  0 suivi\newline
Mots-clé : \emph{aladin, cnrm, france-metropolitaine, mensuel, precipitations, projections, reference
}\newline
Permalien : \url{https://data.gouv.fr/dataset/560c266488ee3807bc83905e}\newline

\par
\noindent
    Fourniture, au format texte, des \textbf{moyennes des différents indices
mensuels de précipitations et de nombre de jours de précipitations,} sur
les 8602 points terrestres de la France métropolitaine \textbf{sur la
période de référence (1976-2005)} et pour les scenarios RCP
(Representative Concentration Pathway) \textbf{2,6 / 4,5 / 8.5}.
Moyennes calculées à partir de données corrigées du CNRM (Centre
National de Recherches Météorologiques).


\vspace{0.5cm}
\needspace{12\baselineskip}
\subsection*{Prévisions d'ensemble du modèle météorologique global ARPEGE sur la zone
Large Molène pour un mois
}\index{meteo}\index{molene}\index{previsions}
  \begin{wrapfigure}{r}{2.5cm}
    \centering
    \qrcode[nolink]{https://data.gouv.fr/dataset/539f2ec1a3a729478718a7f1}
  \end{wrapfigure}

Licence : \textbf{Licence Ouverte
}\newline
Créé le : 2014-06-12\newline
Modifié le : 2016-03-13\newline
De 2014-01-01 à 2014-01-31\newline
Granularité : à la région\newline
Mise à jour : ponctuelle\newline
Popularité : 1 réutilisation,  0 suivi\newline
Mots-clé : \emph{meteo, molene, previsions
}\newline
Permalien : \url{https://data.gouv.fr/dataset/539f2ec1a3a729478718a7f1}\newline

\par
\noindent
    Prévisions d'ensemble (35 membres) du modèle météorologique global
ARPEGE sur la zone 47\degree{}N-49\degree{}N, 2\degree{}W-6\degree{}W,
pour les 31 runs de 18h UTC du mois de janvier 2014, à résolution
horizontale de 25 km. Ce jeu de données est mis à disposition pour
l'opération Archipel Molène.


\vspace{0.5cm}
\needspace{12\baselineskip}
\subsection*{Radiosondages d'altitude sous forme d'émagrammes
}\index{meteo}\index{meteorologie}\index{pression}\index{radiosondage}
  \begin{wrapfigure}{r}{2.5cm}
    \centering
    \qrcode[nolink]{https://data.gouv.fr/dataset/53699e86a3a729239d205ecc}
  \end{wrapfigure}

Licence : \textbf{Licence Ouverte
}\newline
Créé le : 2013-08-16\newline
Modifié le : 2015-11-15\newline
Granularité : au point d'intérêt\newline
Popularité : 1 réutilisation,  0 suivi\newline
Mots-clé : \emph{meteo, meteorologie, pression, radiosondage
}\newline
Permalien : \url{https://data.gouv.fr/dataset/53699e86a3a729239d205ecc}\newline

\par
\noindent
    Emagrammes des 5 stations de France métropolitaine et 9 stations
d'Outre-mer, sous forme de graphes au format PDF, pour les 15 derniers
jours glissants.

Données disponibles pour:

\begin{itemize}

\item
  Trappes
\item
  Brest-Guipavas
\item
  Ajaccio
\item
  Nîmes-Courbessac
\item
  Bordeaux-Mérignac
\item
  Dumont-d'Urville (Terres australes et antarctiques françaises)
\item
  Kerguelen (Terres australes et antarctiques françaises)
\item
  Cayenne-Rochambeau (Guyane)
\item
  Hiva-Oa (îles Marquises, Polynésie française)
\item
  Tahiti-Faa'a (Polynésie française)
\item
  Rapa (îles Australes, Polynésie française)
\item
  Nouméa (Nouvelle-Calédonie)
\item
  Le Raizet (Guadeloupe)
\item
  Gillot (île de la Réunion)
\end{itemize}

\emph{Couverture géographique : Métropole et Outre-mer}

\emph{Fréquence : 1 à 2 fois par jour (00h et 12h UTC)}

\emph{Format : PDF}


\vspace{0.5cm}
\needspace{3\baselineskip} \rule{4cm}{0.25pt}\newline\textbf{Aussi disponible du même producteur :}\begin{itemize}
\item \href{https://data.gouv.fr/dataset/539ef88ea3a72946de9ebe31}{Analyses du modèle de surcotes sur la zone Large Molène pour un mois}
\item \href{https://data.gouv.fr/dataset/53698ed9a3a729239d2035a3}{Animation Cartes modèles atmosphériques}
\item \href{https://data.gouv.fr/dataset/53698fdca3a729239d20385e}{Bulletin climatique mensuel régional}
\item \href{https://data.gouv.fr/dataset/53698fdca3a729239d20385f}{Bulletin climatique mensuel sur la France}
\item \href{https://data.gouv.fr/dataset/53698fdda3a729239d203860}{Bulletin climatique quotidien sur la France}
\item \href{https://data.gouv.fr/dataset/5b768e8b634f414cabe6f443}{Catalogue des archives du climat conservées à la Direction de la Climatologie et des Services Climatiques à Toulouse}
\item \href{https://data.gouv.fr/dataset/5b758ce3634f413761a379d6}{Catalogue des archives du climat conservées dans des centres de la Direction InterRégionale Antilles-Guyane}
\item \href{https://data.gouv.fr/dataset/5b727d4c8b4c41176f720d28}{Catalogue des archives du climat conservées dans des centres de la Direction InterRégionale Centre-Est}
\item \href{https://data.gouv.fr/dataset/5b729e678b4c415026090df4}{Catalogue des archives du climat conservées dans des centres de la Direction InterRégionale Ile de France-Centre}
\item \href{https://data.gouv.fr/dataset/5b718aed8b4c411c36677020}{Catalogue des archives du climat conservées dans des centres de la Direction InterRégionale Nord}
\item \href{https://data.gouv.fr/dataset/5b726ae88b4c417d0313c288}{Catalogue des archives du climat conservées dans des centres de la Direction InterRégionale Nord-Est}
\item \href{https://data.gouv.fr/dataset/5b758e2f634f413cca43980b}{Catalogue des archives du climat conservées dans des centres de la Direction InterRégionale Océan Indien (La Réunion)}
\item \href{https://data.gouv.fr/dataset/5b7532338b4c415e9f025959}{Catalogue des archives du climat conservées dans des centres de la Direction InterRégionale Ouest}
\item \href{https://data.gouv.fr/dataset/5b75675a8b4c413397bc138c}{Catalogue des archives du climat conservées dans des centres de la Direction InterRégionale Sud-Est}
\item \href{https://data.gouv.fr/dataset/5b7582328b4c415ebd98fc7c}{Catalogue des archives du climat conservées dans des centres de la Direction InterRégionale Sud-Ouest}
\item \href{https://data.gouv.fr/dataset/5b7679e8634f412ed6d322f7}{Catalogue des archives du climat conservées dans le Centre des Archives Intermédiaires de Trappes (CAIT)}
\item \href{https://data.gouv.fr/dataset/5b7bbf598b4c412fb2ff2282}{Documents d'archives du climat numérisés : observations météorologiques de surface et d'altitude relevées en Martinique, Guadeloupe et Guyane }
\item \href{https://data.gouv.fr/dataset/5b7ec60c8b4c410775324e55}{Documents d'archives du climat numérisés : sonsages aérologiques (observations en altitude) de France métropolitaine}
\item \href{https://data.gouv.fr/dataset/5b87f1568b4c417e46a85e4a}{Documents d'archives du climat numérisés : tableaux climatologiques mensuels, résumant les observations météoroloqiques quotidiennes de France métropolitaine }
\item \href{https://data.gouv.fr/dataset/58dded5688ee385381896373}{Données de modèle atmosphérique à aire limitée à haute résolution sur l'Outre-mer}
\item \href{https://data.gouv.fr/dataset/539a61c9a3a7293bc2728378}{Données de réflectivité et lames d'eau radar sur la zone Large Molène sur un mois}
\item \href{https://data.gouv.fr/dataset/55682fc4c751df5acfe57269}{Données d'observation en mer (bouées et bateaux)}
\item \href{https://data.gouv.fr/dataset/546a0db0c751df7b9f6c8d07}{Données historiques des cartes et bulletins de Vigilance}
\item \href{https://data.gouv.fr/dataset/539a61d5a3a7293bc2728379}{Données horaires des 55 stations terrestres de la zone Large Molène sur un mois}
\item \href{https://data.gouv.fr/dataset/539efde3a3a72946de9ebe32}{Données satellites sur la zone Large Molène sur un mois}
\item \href{https://data.gouv.fr/dataset/581c344488ee387704c65bb3}{\#{}HackRisques Episode de vent fort sur la région Marseillaise du 27/10/2012 au 28/10/2012 }
\item \href{https://data.gouv.fr/dataset/581b55a288ee384157c65bb6}{\#{}HackRisques Fortes pluies sur le Languedoc-Roussillon du 26/11 au 30/11/2014}
\item \href{https://data.gouv.fr/dataset/581b3e6988ee384157c65bb4}{\#{}HackRisques Précipitations horaires sur l'évènement Seine- Loing du 27/05 au 04/06/2016}
\item \href{https://data.gouv.fr/dataset/560c21a588ee38636883905e}{Indices annuels d'anomalie de précipitations et d'anomalies de nombre de jours de précipitations issus du modèle Aladin-Climat }
\item \href{https://data.gouv.fr/dataset/560c120d88ee386c5183905c}{Indices annuels d'anomalie de température et d'anomalies de nombre de jours de température issus du modèle Aladin-Climat}
\item \href{https://data.gouv.fr/dataset/560c27c188ee3807bc83905f}{Indices annuels de précipitations et nombre de jours de précipitations  issus du modèle Aladin-Climat}
\item \href{https://data.gouv.fr/dataset/560c249c88ee386c51839066}{Indices annuels de température et nombre de jours de température issus du modèle Aladin-Climat}
\item \href{https://data.gouv.fr/dataset/560c20df88ee38636883905d}{Indices mensuels d'anomalie de précipitations et d'anomalies de nombre de jours de précipitations issus du modèle Aladin-Climat }
\item \href{https://data.gouv.fr/dataset/560c112c88ee38719e83905b}{Indices mensuels d'anomalie de température et d'anomalies de nombre de jours de température issus du modèle Aladin-Climat }
\item \href{https://data.gouv.fr/dataset/560c0b5c88ee38636883905a}{Indices mensuels d'anomalie de température et d'anomalies de nombre de jours de température issus du modèle Aladin-Climat pour le scénario RCP 2.6}
\item \href{https://data.gouv.fr/dataset/560c0e4288ee38719e83905a}{Indices mensuels d'anomalie de température et d'anomalies de nombre de jours de température issus du modèle Aladin-Climat pour le scénario RCP 4.5}
\item \href{https://data.gouv.fr/dataset/560c23dc88ee38636883905f}{Indices mensuels de température et nombre de jours de température issus du modèle Aladin-Climat}
\item \href{https://data.gouv.fr/dataset/560c226788ee38520083905f}{Indices saisonniers d'anomalie de précipitations et d'anomalies de nombre de jours de précipitations issus du modèle Aladin-Climat}
\item \href{https://data.gouv.fr/dataset/560c12f088ee386c5183905d}{Indices saisonniers d'anomalie de température et d'anomalies de nombre de jours de température issus du modèle Aladin-Climat }
\item \href{https://data.gouv.fr/dataset/560c287788ee3807bc839060}{Indices saisonniers de précipitations et nombre de jours de précipitations  issus du modèle Aladin-Climat}
\item \href{https://data.gouv.fr/dataset/560c257888ee3807bc83905d}{Indices saisonniers de température et nombre de jours de température issus du modèle Aladin-Climat}
\item \href{https://data.gouv.fr/dataset/54a12162c751df720a04805a}{Métadonnées des stations terrestres de Météo-France}
\item \href{https://data.gouv.fr/dataset/5b714efe8b4c4136058be7f1}{Plan de classement des archives du climat}
\item \href{https://data.gouv.fr/dataset/539a6e67a3a7293bc272838f}{Prévisions du modèle de chimie-transport MOCAGE sur la zone Large Molène pour un mois}
\item \href{https://data.gouv.fr/dataset/539f2ec2a3a729478718a7f2}{Prévisions du modèle de vagues MFWAM sur la zone Large Molène pour un mois}
\item \href{https://data.gouv.fr/dataset/539f2f8ba3a729478718a7f3}{Projections climatiques sur la zone Large Molène sur un mois}
\end{itemize}

\clearpage
\section{Métropole Européenne de Lille }


\begin{center}
  \includegraphics[width=3cm]{images/orga/ae_7ec709120947b49d45ced43ae069d1-100.png}
\end{center}


La Métropole Européenne de Lille est un établissement public de
coopération intercommunale instauré par la loi du 31 décembre 1966 dans
l'objectif de remédier aux inconvénients résultant du morcellement des
communes dans les grandes agglomérations.

L'institution s'est lancée dans l'open data le 18 novembre 2016, via son
site\url{http://opendata.lillemetropole.fr} Suivez-nous sur twitter
@MEL\_OpenData pour ne manquer aucune nouvelle donnée ouverte.


\vspace{0.5cm}

\needspace{12\baselineskip}
\subsection*{Localisation des arrêts de bus et tram + GTFS + Pictogrammes du réseau
\textbar{} Transpole -- copie
}\index{arret}\index{charte!graphique}\index{deplacements}\index{gtfs}\index{logos}\index{pictogrammes}\index{station}\index{transports}
  \begin{wrapfigure}{r}{2.5cm}
    \centering
    \qrcode[nolink]{https://data.gouv.fr/dataset/5a9a5742b595084b5c84c001}
  \end{wrapfigure}

Licence : \textbf{Licence Ouverte
}\newline
Créé le : 2018-03-03\newline
Modifié le : 2018-03-08\newline
Popularité : 1 réutilisation,  0 suivi\newline
Mots-clé : \emph{arret, charte-graphique, deplacements, gtfs, logos, pictogrammes, station, transports
}\newline
Permalien : \url{https://data.gouv.fr/dataset/5a9a5742b595084b5c84c001}\newline

\par
\noindent
    Localisation des arrêts issue du fichier GTFS.\\
Les données issues du fichier GTFS sont des données théoriques.

Le fichier GTFS est régulièrement mis à jour en pièce jointe.

A propos du format
GTFS~:~\url{https://fr.wikipedia.org/wiki/General_Transit_Feed_Specification\%3E}
\url{https://developers.google.com/transit/gtfs/reference/\%3E}


\vspace{0.5cm}
\needspace{12\baselineskip}
\subsection*{Magasins et boutiques
}\index{boutiques}\index{business}\index{commerces}\index{developpement!economique}\index{economie}\index{emploi}\index{entreprise}\index{horaires}\index{magasins}\index{pme}
  \begin{wrapfigure}{r}{2.5cm}
    \centering
    \qrcode[nolink]{https://data.gouv.fr/dataset/599678f8a3a729783d0f6c65}
  \end{wrapfigure}

Licence : \textbf{Open Data Commons Open Database License (ODbL)
}\newline
Créé le : 2017-08-18\newline
Modifié le : 2018-12-03\newline
Popularité : 1 réutilisation,  1 suivi\newline
Mots-clé : \emph{boutiques, business, commerces, developpement-economique, economie, emploi, entreprise, horaires, magasins, pme
}\newline
Permalien : \url{https://data.gouv.fr/dataset/599678f8a3a729783d0f6c65}\newline

\par
\noindent
    Localisation des magasins sur le territoire.\\
Source : OpenStreetMap - des données peuvent manquer.


\vspace{0.5cm}
\needspace{3\baselineskip} \rule{4cm}{0.25pt}\newline\textbf{Aussi disponible du même producteur :}\begin{itemize}
\item \href{https://data.gouv.fr/dataset/59951608a3a729783e0f6c4a}{Aires de jeux}
\item \href{https://data.gouv.fr/dataset/5995162fa3a729783e0f6c4c}{Aires de pique-nique}
\item \href{https://data.gouv.fr/dataset/592505c2a3a7296494ca38da}{Arrêts de tramway}
\item \href{https://data.gouv.fr/dataset/59951607b595086a4f475640}{Banques et distributeurs de billets}
\item \href{https://data.gouv.fr/dataset/58b69b47a3a7295a7d8251cf}{Base SIRENE - Répertoire des entreprises de la MEL}
\item \href{https://data.gouv.fr/dataset/5995162db595086a4f475641}{Boites postales}
\item \href{https://data.gouv.fr/dataset/599678d6a3a729783d0f6c64}{Bornes podotactiles}
\item \href{https://data.gouv.fr/dataset/5c020ab49ce2e77e5fbea585}{Consommation électrique annuelle à la maille commune}
\item \href{https://data.gouv.fr/dataset/58b69b3aa3a7295a7d8251c6}{Consommation électrique par secteur d'activité}
\item \href{https://data.gouv.fr/dataset/596ffa2bb595082d11c49697}{Data.e}
\item \href{https://data.gouv.fr/dataset/58b69b4ca3a7295a7d8251d2}{Equipements et services de l'enseignement supérieur par commune dans le Nord en 2014}
\item \href{https://data.gouv.fr/dataset/58b69b46a3a7295a7c82511d}{Etablissements cinématographiques en 2015}
\item \href{https://data.gouv.fr/dataset/58b69b36a3a7295a7d8251c3}{Evolution et structure de la population - France 2012 - Catégories socio-professionnelles}
\item \href{https://data.gouv.fr/dataset/58b69b3da3a7295a7d8251c8}{Evolution et structure de la population - France 2012 - migrations résidentielles}
\item \href{https://data.gouv.fr/dataset/596ff9e2a3a72957dff834b0}{Feux d'artifices et festivités pour la fête nationale 2017}
\item \href{https://data.gouv.fr/dataset/5be51c1e06e3e7269377c6e1}{Ilévia Abris à vélos}
\item \href{https://data.gouv.fr/dataset/5bdaee7206e3e722988c0eba}{Ilévia Accessibilité PMR}
\item \href{https://data.gouv.fr/dataset/5bdaee7006e3e722988c0eb9}{Ilévia Bornes d'informations voyageurs}
\item \href{https://data.gouv.fr/dataset/5be6627106e3e77a0e77c6e2}{Ilévia Charte graphique couleurs des lignes}
\item \href{https://data.gouv.fr/dataset/5c11dbba06e3e72fd004ccef}{Ilévia Commerces en station}
\item \href{https://data.gouv.fr/dataset/5c11dc359ce2e762bbbea585}{Ilévia Distributeurs de titres}
\item \href{https://data.gouv.fr/dataset/5bdaefa806e3e7240f8c0eb9}{Ilévia - Gamme tarifaire}
\item \href{https://data.gouv.fr/dataset/5be3c2ab06e3e7389a77c6e1}{Ilévia Mobilier aux points d'arrêt}
\item \href{https://data.gouv.fr/dataset/5bdaef1f06e3e722a58c0eb9}{Ilévia Parc bus}
\item \href{https://data.gouv.fr/dataset/5be296199ce2e73e7b830ff0}{Ilévia Parking relais}
\item \href{https://data.gouv.fr/dataset/5c0f385706e3e761cb04ccef}{Ilévia Perturbations du réseau}
\item \href{https://data.gouv.fr/dataset/5c0f39259ce2e73cadbea585}{Ilévia - Tracés des lignes de bus}
\item \href{https://data.gouv.fr/dataset/58b69b36a3a7295a7c825112}{Indices mensuels de retard des bus}
\item \href{https://data.gouv.fr/dataset/58b69b44a3a7295a7d8251cd}{Liste des bus du réseau Transpole en service}
\item \href{https://data.gouv.fr/dataset/5a90e4d4b5950803ef84c001}{Liste des prénoms des nouveaux-nés entre 2014 et 2017 - Ville de Tourcoing}
\item \href{https://data.gouv.fr/dataset/58b69b41a3a7295a7c825119}{Localisation des cimetières}
\item \href{https://data.gouv.fr/dataset/591fc02fa3a7296494ca2994}{Localisation des hopitaux}
\item \href{https://data.gouv.fr/dataset/58b69b4da3a7295a7d8251d3}{Localisation des installations sportives}
\item \href{https://data.gouv.fr/dataset/58b69b57a3a7295a7c825128}{Localisation des parkings}
\item \href{https://data.gouv.fr/dataset/58b69b2aa3a7295a7d8251bb}{Localisation des stations de métro}
\item \href{https://data.gouv.fr/dataset/5923b477a3a7296495ca3575}{Mairies}
\item \href{https://data.gouv.fr/dataset/58b69b38a3a7295a7c825113}{Naissances par commune, département et région de 2003 à 2013}
\item \href{https://data.gouv.fr/dataset/58b69b20a3a7295a7d8251b4}{Nombre de crémations par mois depuis 1981}
\item \href{https://data.gouv.fr/dataset/59001b9ca3a7296bcc3cd4c7}{Nombre de passages de vélos en 2016}
\item \href{https://data.gouv.fr/dataset/5995162ca3a729783e0f6c4b}{Passages piétons}
\item \href{https://data.gouv.fr/dataset/59951609b595086a504756ef}{Pharmacies}
\item \href{https://data.gouv.fr/dataset/5a8d62e3b595082bf984c001}{Points d'arrêts Ouibus dans la Métropole de Lille}
\item \href{https://data.gouv.fr/dataset/5995160aa3a729783d0f6bf2}{Points de vue - tourisme}
\item \href{https://data.gouv.fr/dataset/5a8d62efa3a7294c0cb44b85}{Production électrique par filière sur le territoire de la MEL}
\item \href{https://data.gouv.fr/dataset/58f04a8da3a7293d4ac4e1ba}{Qualité des eaux superficielles vis-à-vis des nitrates en 2013 et 2014}
\item \href{https://data.gouv.fr/dataset/58b7990ba3a7293affefb38d}{Registre français des émission polluantes - Etablissements}
\item \href{https://data.gouv.fr/dataset/58b69b28a3a7295a7d8251b9}{Régularité mensuelle des TGV arrivant à Lille}
\item \href{https://data.gouv.fr/dataset/58b69b47a3a7295a7c82511e}{Résumé statistique communes, départements et régions France - 2012, 2013, 2014}
\item \href{https://data.gouv.fr/dataset/5995162ba3a729783d0f6bf3}{Rues pavées}
\item \href{https://data.gouv.fr/dataset/5995162ea3a729783d0f6bf4}{Stationnements PMR}
\item et 0 autres jeux de données\end{itemize}

\clearpage
\section{Metz Métropole}


\begin{center}
  \includegraphics[width=3cm]{images/orga/e2_4c0d8652d6455aaae8bc6b24662ac8-100.png}
\end{center}


Metz Métropole


\vspace{0.5cm}

\needspace{12\baselineskip}
\subsection*{Transport - Données GTFS
}\index{gtfs}\index{le!met}\index{metz!metropole}\index{tamm}\index{territoires!et!transport}\index{urbain}
  \begin{wrapfigure}{r}{2.5cm}
    \centering
    \qrcode[nolink]{https://data.gouv.fr/dataset/546609c1c751df1a6f6c8d07}
  \end{wrapfigure}

Licence : \textbf{Licence Ouverte
}\newline
Créé le : 2014-11-14\newline
Modifié le : 2018-10-10\newline
Mise à jour : semestrielle\newline
Popularité : 1 réutilisation,  8 suivis\newline
Mots-clé : \emph{gtfs, le-met, metz-metropole, tamm, territoires-et-transport, urbain
}\newline
Permalien : \url{https://data.gouv.fr/dataset/546609c1c751df1a6f6c8d07}\newline

\par
\noindent
    Données GTFS


\vspace{0.5cm}
\needspace{12\baselineskip}
\subsection*{Voirie - Adresses
}\index{adresses}\index{batiments}\index{metz!metropole}\index{numeros}\index{rues}\index{territoires!et!transports}\index{voirie}
  \begin{wrapfigure}{r}{2.5cm}
    \centering
    \qrcode[nolink]{https://data.gouv.fr/dataset/542ab42f88ee382b8f22e2e4}
  \end{wrapfigure}

Licence : \textbf{Open Data Commons Open Database License (ODbL)
}\newline
Créé le : 2014-09-30\newline
Modifié le : 2017-06-20\newline
Popularité : 1 réutilisation,  0 suivi\newline
Mots-clé : \emph{adresses, batiments, metz-metropole, numeros, rues, territoires-et-transports, voirie
}\newline
Permalien : \url{https://data.gouv.fr/dataset/542ab42f88ee382b8f22e2e4}\newline

\par
\noindent
    Numéros d'adresses. Jointure possible avec la couche des emprises de
voies: champs ``\_voie``.


\vspace{0.5cm}
\needspace{3\baselineskip} \rule{4cm}{0.25pt}\newline\textbf{Aussi disponible du même producteur :}\begin{itemize}
\item \href{https://data.gouv.fr/dataset/5432b3a088ee38109064e606}{Arbres d'alignement}
\item \href{https://data.gouv.fr/dataset/5432b42788ee38109164e603}{Arbres remarquables}
\item \href{https://data.gouv.fr/dataset/5432b50b88ee38108f64e606}{Archéologie préventive - Chantiers}
\item \href{https://data.gouv.fr/dataset/542a9f4d88ee382b9022e2e3}{Balades - Grande randonnée}
\item \href{https://data.gouv.fr/dataset/542e8d3888ee387d53138110}{Cartes postales historiques}
\item \href{https://data.gouv.fr/dataset/5432b07288ee38108f64e605}{Cimetières - Élément ponctuel}
\item \href{https://data.gouv.fr/dataset/5432b0e788ee38109264e603}{Cimetières - Élément surfacique}
\item \href{https://data.gouv.fr/dataset/5432b25588ee38109264e604}{Cimetières - Emplacements}
\item \href{https://data.gouv.fr/dataset/5432af8788ee38109064e603}{Cimetières - Emprise}
\item \href{https://data.gouv.fr/dataset/5432b2c388ee38109064e604}{Cimetières - Rangées}
\item \href{https://data.gouv.fr/dataset/5432b31688ee38109064e605}{Cimetières - Sections}
\item \href{https://data.gouv.fr/dataset/542ac07288ee38568816059e}{Equipements culturels}
\item \href{https://data.gouv.fr/dataset/5429728c88ee380329a59161}{Equipements - Hôpitaux}
\item \href{https://data.gouv.fr/dataset/542968b388ee38032aa5915d}{Equipements - Lieux de culte}
\item \href{https://data.gouv.fr/dataset/5429743288ee380328a5915b}{Equipements - Mairies}
\item \href{https://data.gouv.fr/dataset/5429753988ee380327a59166}{Equipements - Police}
\item \href{https://data.gouv.fr/dataset/542976c288ee380329a59162}{Equipements - Poste}
\item \href{https://data.gouv.fr/dataset/54323f0988ee38259e396b3f}{Equipements sportifs}
\item \href{https://data.gouv.fr/dataset/542e937088ee387d54138112}{Forts}
\item \href{https://data.gouv.fr/dataset/54292a0a88ee380329a5915a}{Habillage - Bâti schématique}
\item \href{https://data.gouv.fr/dataset/542aa9e388ee382b8c22e2e2}{Habillage - Hydrographie linéaire}
\item \href{https://data.gouv.fr/dataset/542aa8cf88ee382b8e22e2df}{Habillage - Hydrographie surfacique}
\item \href{https://data.gouv.fr/dataset/542aa77388ee382b8e22e2de}{Habillage - Verdure schématique}
\item \href{https://data.gouv.fr/dataset/542aab2f88ee382b8d22e2e2}{Habillage - Voies ferrées}
\item \href{https://data.gouv.fr/dataset/542aa1f188ee382b8e22e2dd}{Personnes à mobilité réduite - Parkings}
\item \href{https://data.gouv.fr/dataset/5432630788ee3825a2396b45}{Plans Locaux d'Urbanisme - Alignements, implantations}
\item \href{https://data.gouv.fr/dataset/54325dec88ee38259e396b44}{Plans Locaux d'Urbanisme - Emplacements réservés}
\item \href{https://data.gouv.fr/dataset/5432603388ee38259e396b45}{Plans Locaux d'Urbanisme - Espaces boisés classés}
\item \href{https://data.gouv.fr/dataset/5432614388ee3825a0396b45}{Plans Locaux d'Urbanisme - Plantations à réaliser}
\item \href{https://data.gouv.fr/dataset/5432625088ee38259e396b46}{Plans Locaux d'Urbanisme - Prescriptions ponctuelles}
\item \href{https://data.gouv.fr/dataset/5432ac6d88ee38108f64e604}{Plans Locaux d'Urbanisme - Secteur sauvegardé}
\item \href{https://data.gouv.fr/dataset/5432abb188ee38108f64e603}{Plans Locaux d'Urbanisme - ZAC}
\item \href{https://data.gouv.fr/dataset/54325d0888ee38259f396b3f}{Plans Locaux d'Urbanisme - Zonages}
\item \href{https://data.gouv.fr/dataset/542e9cd188ee387d54138113}{Transport - Arrêts de bus}
\item \href{https://data.gouv.fr/dataset/542ea64e88ee387d55138112}{Transport - Arrêts schématiques METTIS}
\item \href{https://data.gouv.fr/dataset/5432459988ee3825a2396b40}{Transport - lignes de bus}
\item \href{https://data.gouv.fr/dataset/5432479988ee38259e396b40}{Transport - Parcellaire METTIS}
\item \href{https://data.gouv.fr/dataset/5432575488ee3825a2396b43}{Transport - Pistes cyclables}
\item \href{https://data.gouv.fr/dataset/5432596688ee3825a1396b41}{Transport - Stationnement vélo}
\item \href{https://data.gouv.fr/dataset/5432580d88ee3825a2396b44}{Transport - Stations de location de vélo}
\item \href{https://data.gouv.fr/dataset/54324aff88ee3825a0396b42}{Transport - Stations METTIS physiques}
\item \href{https://data.gouv.fr/dataset/54324bce88ee3825a0396b43}{Transport - Tracé schématique METTIS}
\item \href{https://data.gouv.fr/dataset/5432495488ee3825a2396b41}{Transport - Voies en site propre METTIS}
\item \href{https://data.gouv.fr/dataset/542aad6a88ee382b8f22e2e1}{Voirie - Arcs de voies}
\item \href{https://data.gouv.fr/dataset/542ab0a788ee382b8f22e2e2}{Voirie - Libellés de voies}
\item \href{https://data.gouv.fr/dataset/542d6ba488ee38615c8561ad}{Voirie - Rues structurées pour localisation}
\item \href{https://data.gouv.fr/dataset/542ab9b388ee382b8f22e2e6}{Voirie - Sens de circulation}
\end{itemize}

\clearpage
\section{Ministère de la Cohésion des territoires}


\begin{center}
  \includegraphics[width=3cm]{images/orga/10_47c4b217ff4a258df1bc6468f1bd4e-100.png}
\end{center}


Le Ministère de la Cohésion des territoires est en charge des politiques
du logement, de la ville et de l'aménagement des territoires.


\vspace{0.5cm}

\needspace{12\baselineskip}
\subsection*{Base des permis de construire (Sitadel)
}\index{locaux}\index{logements}\index{permis!de!construire}\index{permisdeconstruire}\index{sdes}\index{sitadel}
  \begin{wrapfigure}{r}{2.5cm}
    \centering
    \qrcode[nolink]{https://data.gouv.fr/dataset/5a5f4f6c88ee387da4d252a3}
  \end{wrapfigure}

Licence : \textbf{Licence Ouverte
}\newline
Créé le : 2018-01-17\newline
Modifié le : 2019-03-16\newline
De 2017-12-01 à 2018-01-31\newline
Granularité : au point d'intérêt\newline
Mise à jour : mensuelle\newline
Popularité : 1 réutilisation,  4 suivis\newline
Mots-clé : \emph{locaux, logements, permis-de-construire, permisdeconstruire, sdes, sitadel
}\newline
Permalien : \url{https://data.gouv.fr/dataset/5a5f4f6c88ee387da4d252a3}\newline

\par
\noindent
    Tout pétitionnaire projetant une construction neuve ou la transformation
d'une construction nécessitant le dépôt d'un permis de construire, doit
remplir un formulaire relatif à son projet et le transmettre à la mairie
de la commune de localisation des travaux. Le projet de permis est
traité par les services instructeurs (État, collectivité territoriale)
dont relève la commune. Les informations du formulaire alimentent
l'application Sitadel via les centres instructeurs. Les mouvements
relatifs à la vie du permis (dépôts, autorisations, annulations,
modificatifs, mises en chantier, achèvements des travaux) sont transmis
mensuellement au service de la donnée et des études statistiques (SDES)
qui exploite les données à des fins statistiques.

À partir de janvier 2018, le Service de la donnée et des études
statistiques (SDES) met chaque mois à disposition les événements
concernant les projets de construction (dépôt de la demande de permis,
autorisation et début des travaux) reçus dans la base Sitadel depuis la
précédente mise à disposition.

Afin de garantir la protection des personnes physiques, le champ de
diffusion est restreint aux permis de construire déposés par les
personnes morales.

Chaque mois selon le calendrier de parution des résultats de la
construction (Sitadel), deux listes sont diffusées : l'une relative aux
permis destinés à la construction de logements et l'autre à la
construction de locaux non résidentiels.

Seules les listes diffusées au cours des 24 derniers mois sont
conservées en ligne.


\vspace{0.5cm}
\needspace{12\baselineskip}
\subsection*{Présence dans la commune d'une Zone franche urbaine - ZFU
}
  \begin{wrapfigure}{r}{2.5cm}
    \centering
    \qrcode[nolink]{https://data.gouv.fr/dataset/53699d9fa3a729239d205c9c}
  \end{wrapfigure}

Licence : \textbf{Licence Ouverte
}\newline
Créé le : 2013-07-08\newline
Modifié le : 2015-10-30\newline
De 2009-01-01 à 2013-12-31\newline
Mise à jour : annuelle\newline
Popularité : 1 réutilisation,  0 suivi\newline
Mots-clé : \emph{aucun
}\newline
Permalien : \url{https://data.gouv.fr/dataset/53699d9fa3a729239d205c9c}\newline

\par
\noindent
    Les zones urbaines sensibles (ZUS) sont des territoires infra-urbains
définis par les pouvoirs publics pour être la cible prioritaire de la
politique de la ville, en fonction des considérations locales liées aux
difficultés que connaissent les habitants de ces territoires.


\vspace{0.5cm}
\needspace{12\baselineskip}
\subsection*{Présence dans la commune d'une Zone urbaine sensible - ZUS
}
  \begin{wrapfigure}{r}{2.5cm}
    \centering
    \qrcode[nolink]{https://data.gouv.fr/dataset/53699d9fa3a729239d205c9d}
  \end{wrapfigure}

Licence : \textbf{Licence Ouverte
}\newline
Créé le : 2013-07-08\newline
Modifié le : 2016-01-01\newline
De 2009-01-01 à 2013-12-31\newline
Mise à jour : annuelle\newline
Popularité : 1 réutilisation,  0 suivi\newline
Mots-clé : \emph{aucun
}\newline
Permalien : \url{https://data.gouv.fr/dataset/53699d9fa3a729239d205c9d}\newline

\par
\noindent
    Les zones urbaines sensibles (ZUS) sont des territoires infra-urbains
définis par les pouvoirs publics pour être la cible prioritaire de la
politique de la ville, en fonction des considérations locales liées aux
difficultés que connaissent les habitants de ces territoires.


\vspace{0.5cm}
\needspace{3\baselineskip} \rule{4cm}{0.25pt}\newline\textbf{Aussi disponible du même producteur :}\begin{itemize}
\item \href{https://data.gouv.fr/dataset/58edf4d088ee38604b3b69c7}{Base de données éco-PTZ (éco-prêt à taux zéro)}
\item \href{https://data.gouv.fr/dataset/58edf72f88ee387741f8d383}{Base de données PAS (prêt à l’accession sociale).}
\item \href{https://data.gouv.fr/dataset/58ede97088ee385916953c07}{Base de données PTZ (prêts à taux zéro)}
\item \href{https://data.gouv.fr/dataset/588fb50dc751df5c03ae0a65}{Base des ÉcoQuartiers }
\item \href{https://data.gouv.fr/dataset/5b45bd4d88ee380cf27d2c9a}{Compte satellite du logement}
\item \href{https://data.gouv.fr/dataset/536991b9a3a729239d203d2c}{Contrat urbain de cohésion sociale - CUCS}
\item \href{https://data.gouv.fr/dataset/57066d30c751df29c2dafbba}{Demande de logement social}
\item \href{https://data.gouv.fr/dataset/5ad9b848c751df02acbf05c2}{Etat d'avancement des DU au 31/12/2016 Source SuDocUH à Mayotte}
\item \href{https://data.gouv.fr/dataset/5ad9b848c751df0293dc994e}{Etat d'avancement des DU au 31/12/2016 Source SuDocUH en Guadeloupe}
\item \href{https://data.gouv.fr/dataset/5ad9b847c751df028a4c9090}{Etat d'avancement des DU au 31/12/2016 Source SuDocUH en Guyane}
\item \href{https://data.gouv.fr/dataset/5ad9b847c751df7f7e4e8940}{Etat d'avancement des DU au 31/12/2016 Source SuDocUH en Martinique}
\item \href{https://data.gouv.fr/dataset/5ad9b84888ee3852ce7c3e43}{Etat d'avancement des DU au 31/12/2016 Source SuDocUH sur la métropole}
\item \href{https://data.gouv.fr/dataset/5b04387188ee3816f4886adb}{Etat d'avancement des SCOT au 31/12/2017 Source SuDocUH en Martinique (composition communale)}
\item \href{https://data.gouv.fr/dataset/5af993b6c751df6e1a8c165d}{Liste des communes situés en zone B2 ou C, éligibles aux dispositifs d'investissement locatif "Pinel" et "Duflot" ayant obtenu un agrément préfectoral}
\item \href{https://data.gouv.fr/dataset/559c029088ee381a50764f5d}{Massif forestier à risque feu de forêt en Charente.}
\item \href{https://data.gouv.fr/dataset/5bc5c7eb634f41547efbc0c0}{Observatoire des performances énergétiques}
\end{itemize}

\clearpage
\section{Ministère de la Culture}


\begin{center}
  \includegraphics[width=3cm]{images/orga/cf_f1ce1cb584470c8b83a9839dd9c83e-100.jpg}
\end{center}


Le Ministère de la culture a pour mission de rendre accessible au plus
grand nombre les œuvres culturelles mondiales et celles de la France.

A ce titre, il conduit la politique de sauvegarde, de protection et de
mise en valeur du patrimoine culturel dans toutes ses composantes, il
favorise la création des œuvres de l'art et de l'esprit et le
développement des pratiques et de l'éducation artistique et culturelle
sur l'ensemble du territoire. Le Ministère de la culture veille au
développement des industries créatives et culturelles. Il contribue au
développement des nouvelles technologies de diffusion et de valorisation
du patrimoine culturel matériel et immatériel.

Les données publiques en matière de culture et de communication sont
accessibles sur une plateforme dédiée
\url{https://data.culture.gouv.fr}synchronisée avec la plateforme
www.data.gouv.fr. Cette initiative s'inscrit dans le cadre de
l'engagement du Ministère de la culture en faveur de l'ouverture et du
partage des données publiques, ainsi que dans le développement d'une
économie numérique culturelle.


\vspace{0.5cm}

\needspace{12\baselineskip}
\subsection*{Agenda de l'offre culturelle
}\index{agenda}\index{communication}\index{culture}\index{evenement!culturel}
  \begin{wrapfigure}{r}{2.5cm}
    \centering
    \qrcode[nolink]{https://data.gouv.fr/dataset/53698e87a3a729239d2034b9}
  \end{wrapfigure}

Licence : \textbf{Licence Ouverte
}\newline
Créé le : 2013-07-08\newline
Modifié le : 2018-10-15\newline
De 2011-01-01 à 2015-07-24\newline
Granularité : à la commune\newline
Mise à jour : semestrielle\newline
Popularité : 7 réutilisations,  5 suivis\newline
Mots-clé : \emph{agenda, communication, culture, evenement-culturel
}\newline
Permalien : \url{https://data.gouv.fr/dataset/53698e87a3a729239d2034b9}\newline

\par
\noindent
    Liste des évènements culturels et des organismes producteurs
d'événements en France et des grandes manifestations en France et à
l'étranger. Ce jeu de données n'est plus mis à jour.


\vspace{0.5cm}
\needspace{12\baselineskip}
\subsection*{Aides à la presse : classement des titres et groupes de presse aidés
}\index{aide}\index{aides}\index{aides!a!la!presse}\index{aides!a!la!presse!ecrite}\index{aides!la!presse!ecrite}\index{culture}\index{montant!des!aides}\index{presse}
  \begin{wrapfigure}{r}{2.5cm}
    \centering
    \qrcode[nolink]{https://data.gouv.fr/dataset/53699a19a3a729239d2053df}
  \end{wrapfigure}

Licence : \textbf{Licence Ouverte
}\newline
Créé le : 2013-12-12\newline
Modifié le : 2019-02-12\newline
Mise à jour : annuelle\newline
Popularité : 6 réutilisations,  5 suivis\newline
Mots-clé : \emph{aide, aides, aides-a-la-presse, aides-a-la-presse-ecrite, aides-la-presse-ecrite, culture, montant-des-aides, presse
}\newline
Permalien : \url{https://data.gouv.fr/dataset/53699a19a3a729239d2053df}\newline

\par
\noindent
    Dans un objectif de transparence, le ministère de la culture publie la
liste des titres et, à partir de 2016, des principaux groupes de presse,
ayant bénéficié d'aides directes et indirectes. La colonne « Total des
aides » cumule les montants : - des aides directes, perçues par le titre
et conservées dans ses comptes ; - des aides à la filière, perçues par
le titre mais qui sont ensuite reversées à sa messagerie (aide à la
distribution de la presse quotidienne nationale) ; - des aides à la
modernisation sociale, bénéficiant à d'anciens salariés du titre.

Une notice pour chaque année, jointe aux tableaux, explique la façon
dont ceux-ci ont été réalisés et comment ils se lisent.


\vspace{0.5cm}
\needspace{12\baselineskip}
\subsection*{Archives publiques en France
}\index{archive}\index{archives}\index{archives!publiques}\index{budget}\index{communes}\index{communication}\index{conservation}\index{culture}\index{departement}\index{enseignement}\index{lecteurs}\index{public}\index{region}\index{service!public}\index{statistiques}\index{villes}
  \begin{wrapfigure}{r}{2.5cm}
    \centering
    \qrcode[nolink]{https://data.gouv.fr/dataset/567bba4dc751df4361c664bc}
  \end{wrapfigure}

Licence : \textbf{Licence Ouverte
}\newline
Créé le : 2015-12-24\newline
Modifié le : 2018-06-15\newline
De 2010-01-01 à 2015-12-31\newline
Granularité : au point d'intérêt\newline
Mise à jour : annuelle\newline
Popularité : 2 réutilisations,  1 suivi\newline
Mots-clé : \emph{archive, archives, archives-publiques, budget, communes, communication, conservation, culture, departement, enseignement, lecteurs, public, region, service-public, statistiques, villes
}\newline
Permalien : \url{https://data.gouv.fr/dataset/567bba4dc751df4361c664bc}\newline

\par
\noindent
    Cette page présente l'ensemble des données statistiques recueillies par
le ministère de la Culture (service interministériel des Archives de
France) auprès des services publics d'archives.

Chaque année, au titre du contrôle scientifique et technique qu'elles
exercent (article R 212-56 du Code du patrimoine), garant de
l'uniformité des pratiques et de leur conformité avec la législation,
les Archives de France lancent une enquête statistique auprès du réseau
de services d'archives. Ce réseau correspond au maillage territorial
français et se compose de cent-un services d'Archives départementales,
de vingt-six services d'Archives régionales (jusqu'au 1er janvier 2016),
de près de sept cents services d'Archives communales et intercommunales,
d'une quarantaine de service d'aide à l'archivage dans les centres de
gestion et d'une centaine de services d'archives dans des établissements
publics

Chacun est invité à remplir la grille qui lui est envoyée puis à la
transmettre aux Archives de France, qui, à leur tour, analysent et
vérifient les chiffres obtenus. Le ministère de la Culture et de la
Communication reçoit ainsi plus de six cents réponses, qui sont
présentées dans les tableaux disponibles ci-dessous Les indicateurs
demandés par le ministère sont multiples et illustrent toutes les
activités des services d'Archives, de leur fonctionnement (budget,
personnel, bâtiment), à la valorisation culturelle et éducative
(expositions, nombre de visiteurs), en passant par la collecte
d'archives, leur traitement, leur numérisation et leur mise en ligne sur
des sites internet.

Outre les tableaux présentés par type de services d'Archives, les
Archives de France publient également une brochure annuelle, \emph{Des
Archives en France}, qui présente une analyse détaillée de certains
chiffres, agrémentée d'illustrations.

Ces publications sont disponibles, par année, sur le site internet des
Archives de France~:\url{https://www.francearchives.fr/article/37979}


\vspace{0.5cm}
\needspace{12\baselineskip}
\subsection*{Données statistiques des services publics d' Archives régionales
}\index{archivage!electronique}\index{archive}\index{archives}\index{archives!publiques}\index{archives!regionales}\index{batiment}\index{budget}\index{classement}\index{collecte}\index{communication}\index{conservation}\index{culture}\index{enseignement}\index{insturments!de!recherche}\index{lecteurs}\index{microfilm}\index{numerisation}\index{personnel}\index{public}\index{region}\index{restauration}\index{statistiques}
  \begin{wrapfigure}{r}{2.5cm}
    \centering
    \qrcode[nolink]{https://data.gouv.fr/dataset/567982bcc751df2e01c664bf}
  \end{wrapfigure}

Licence : \textbf{Licence Ouverte
}\newline
Créé le : 2015-12-22\newline
Modifié le : 2017-02-01\newline
De 2010-01-01 à 2015-12-31\newline
Granularité : à la région\newline
Mise à jour : annuelle\newline
Popularité : 1 réutilisation,  0 suivi\newline
Mots-clé : \emph{archivage-electronique, archive, archives, archives-publiques, archives-regionales, batiment, budget, classement, collecte, communication, conservation, culture, enseignement, insturments-de-recherche, lecteurs, microfilm, numerisation, personnel, public, region, restauration, statistiques
}\newline
Permalien : \url{https://data.gouv.fr/dataset/567982bcc751df2e01c664bf}\newline

\par
\noindent
    Les données statistiques des services d'Archives régionales présentent
les principaux indicateurs relatifs à ces services de 2010 à 2014 :
budget, personnel, bâtiment, collecte d'archives, conservation,
numérisation et mise en ligne, activités culturelles et éducatives.


\vspace{0.5cm}
\needspace{12\baselineskip}
\subsection*{Egalité hommes-femmes dans la culture et la communication
}\index{culture}\index{culture!et!communication}\index{egalite!homme!femme}\index{egalite!hommes!femmes}\index{homme!femme}\index{hommes!femmes}\index{postes!de!direction}\index{remuneration}\index{ressources!humaines}
  \begin{wrapfigure}{r}{2.5cm}
    \centering
    \qrcode[nolink]{https://data.gouv.fr/dataset/5369995fa3a729239d20522b}
  \end{wrapfigure}

Licence : \textbf{Licence Ouverte
}\newline
Créé le : 2013-11-28\newline
Modifié le : 2016-03-16\newline
Mise à jour : annuelle\newline
Popularité : 1 réutilisation,  2 suivis\newline
Mots-clé : \emph{culture, culture-et-communication, egalite-homme-femme, egalite-hommes-femmes, homme-femme, hommes-femmes, postes-de-direction, remuneration, ressources-humaines
}\newline
Permalien : \url{https://data.gouv.fr/dataset/5369995fa3a729239d20522b}\newline

\par
\noindent
    Etat des lieux de l'observatoire de l'égalité entre les femmes et les
hommes dans le champ de la culture et de la communication . Elle est le
fruit d'un travail statistique qui porte sur l'administration du
ministère de la culture et de la communication comme sur les
institutions culturelles et les médias, et qui sera poursuivi chaque
année, afin de faire apparaître les progrès et les difficultés, et le
cas échéant de réorienter l'action.

Vous trouverez dans ce document des tableaux : - Indicateurs-clés pour
le suivi de l'égalité hommes-femmes dans la culture et la communication
- Repères sur l'égalité hommes-femmes dans la culture et la
communication - Postes de direction et ressources humaines - Accès aux
moyens de production - Programmation (Part des femmes dans la production
et diffusion des spectacles ; Acquisitions des fonds d'art contemporain
; Expositions des fonds régionaux d'art contemporain et centres d'art ;
Audiovisuel public) - Rémunérations


\vspace{0.5cm}
\needspace{12\baselineskip}
\subsection*{Journées Européennes du Patrimoine
}\index{culture}\index{jep}\index{journees!europeennes!du!patrimoi}\index{journees!patrimoine}\index{patrimoine!architectural}\index{patrimoine!culturel}
  \begin{wrapfigure}{r}{2.5cm}
    \centering
    \qrcode[nolink]{https://data.gouv.fr/dataset/5369978fa3a729239d204d11}
  \end{wrapfigure}

Licence : \textbf{Licence Ouverte
}\newline
Créé le : 2013-07-08\newline
Modifié le : 2016-09-19\newline
Granularité : au point d'intérêt\newline
Mise à jour : annuelle\newline
Popularité : 6 réutilisations,  8 suivis\newline
Mots-clé : \emph{culture, jep, journees-europeennes-du-patrimoi, journees-patrimoine, patrimoine-architectural, patrimoine-culturel
}\newline
Permalien : \url{https://data.gouv.fr/dataset/5369978fa3a729239d204d11}\newline

\par
\noindent
    Programme des éditions des Journées Européennes du Patrimoine (JEP).

Pour télécharger les données régionales dans de multiples formats (RSS,
CSV, XLSX, JSON, PDF, iCal, ICS, Google Calendar), cliquez ici
:\url{http://journeesdupatrimoine.culturecommunication.gouv.fr/Communication/Telecharger-les-donnees-Open-Data}


\vspace{0.5cm}
\needspace{12\baselineskip}
\subsection*{Liste des objets mobiliers propriété publique classés au titre des
Monuments Historiques
}\index{culture}\index{objet!classe}\index{objet!mobilier}
  \begin{wrapfigure}{r}{2.5cm}
    \centering
    \qrcode[nolink]{https://data.gouv.fr/dataset/536c47d9a3a72933d8d1b3b2}
  \end{wrapfigure}

Licence : \textbf{Licence Ouverte
}\newline
Créé le : 2013-10-13\newline
Modifié le : 2016-09-12\newline
Granularité : à la commune\newline
Mise à jour : annuelle\newline
Popularité : 2 réutilisations,  0 suivi\newline
Mots-clé : \emph{culture, objet-classe, objet-mobilier
}\newline
Permalien : \url{https://data.gouv.fr/dataset/536c47d9a3a72933d8d1b3b2}\newline

\par
\noindent
    Cet extrait de la base Palissy concerne les seuls objets mobiliers
classés propriété de l'Administration. La base Palissy contient des
informations sur des objets d'une grande diversité technique et
historique tels que vitraux, meubles, tableaux, sculptures, orfèvrerie,
dessins, monnaies, bronzes d'art, etc. de la Préhistoire au 20e siècle.


\vspace{0.5cm}
\needspace{12\baselineskip}
\subsection*{Nuit des Musées : programme national de la 11ème édition le 16 mai 2015
}\index{culture}\index{evenement}\index{musees}\index{nuit!des!musees}
  \begin{wrapfigure}{r}{2.5cm}
    \centering
    \qrcode[nolink]{https://data.gouv.fr/dataset/5550cfd9c751df388e190c78}
  \end{wrapfigure}

Licence : \textbf{Licence Ouverte
}\newline
Créé le : 2015-05-11\newline
Modifié le : 2015-12-06\newline
De 2015-05-16 à 2015-05-16\newline
Granularité : à la commune\newline
Mise à jour : annuelle\newline
Popularité : 2 réutilisations,  0 suivi\newline
Mots-clé : \emph{culture, evenement, musees, nuit-des-musees
}\newline
Permalien : \url{https://data.gouv.fr/dataset/5550cfd9c751df388e190c78}\newline

\par
\noindent
    Pour sa 11ème édition, la Nuit européenne des musées propose cette année
l'ouverture gratuite de plus de 1300 musées en France, et près de 3400
en Europe. Liste des données : Nom du lieu, Adresse, Pays, Région,
Latitude, Longitude, téléphone, site internet, Courriel, Facebook,
Twitter, Autres réseaux sociaux, Accès, Accès handicapés, Label
Tourisme, Thèmes, Titre de l'animation, Type de visite ou d'animation,
Dates, Horaires, Conditions d'accès, Accessibilité handicapés, Thème de
la visite.


\vspace{0.5cm}
\needspace{12\baselineskip}
\subsection*{Thésaurus de la désignation des objets mobiliers
}\index{culture}\index{inventaire}\index{objets!mobiliers}\index{thesaurus}\index{vocabulaire}
  \begin{wrapfigure}{r}{2.5cm}
    \centering
    \qrcode[nolink]{https://data.gouv.fr/dataset/54748d26c751df0555c2acbd}
  \end{wrapfigure}

Licence : \textbf{Licence Ouverte
}\newline
Créé le : 2014-11-25\newline
Modifié le : 2016-03-08\newline
Granularité : au point d'intérêt\newline
Mise à jour : ponctuelle\newline
Popularité : 2 réutilisations,  1 suivi\newline
Mots-clé : \emph{culture, inventaire, objets-mobiliers, thesaurus, vocabulaire
}\newline
Permalien : \url{https://data.gouv.fr/dataset/54748d26c751df0555c2acbd}\newline

\par
\noindent
    Le Thésaurus de la désignation des objets mobiliers constitue la
réédition, revue et complétée, de l'outil élaboré par l'Inventaire
général des monuments et richesses artistiques de la France et édité en
2001 par les éditions du patrimoine, dans la collection Documents \&
méthodes. Treize ans plus tard, il a paru nécessaire de reprendre
l'ouvrage pour, d'une part, introduire les concepts que les nouveaux
domaines de recherche ont fait émerger, et d'autre part, en faire un
véritable outil documentaire numérique.


\vspace{0.5cm}
\needspace{3\baselineskip} \rule{4cm}{0.25pt}\newline\textbf{Aussi disponible du même producteur :}\begin{itemize}
\item \href{https://data.gouv.fr/dataset/5890bf78a3a72974c1f0dc8f}{Adresses des bibliothèques publiques}
\item \href{https://data.gouv.fr/dataset/565db6f2c751df2d32aad370}{Aides à la presse}
\item \href{https://data.gouv.fr/dataset/5ac44e05b5950877d5c04a3e}{Année européenne du patrimoine culturel 2018 (1e édition)}
\item \href{https://data.gouv.fr/dataset/53698edca3a729239d2035ac}{Annexe au projet de loi de finances pour 2012. Effort financier de l’État dans le domaine de la culture et de la communication}
\item \href{https://data.gouv.fr/dataset/5a3be6cc88ee3874634d5e1d}{Archives nationales}
\item \href{https://data.gouv.fr/dataset/5890bf74a3a72974cbf0dc8b}{Archives publiques en France : données statistiques}
\item \href{https://data.gouv.fr/dataset/536c3e46a3a72933d8d1b394}{Catalogue des collections numérisées - Base Patrimoine Numérique}
\item \href{https://data.gouv.fr/dataset/5a4e847cb59508409d014056}{CCFr : Répertoire des bibliothèques}
\item \href{https://data.gouv.fr/dataset/5a4e84d2a3a7293f8d17f221}{CCFr : Répertoire des fonds}
\item \href{https://data.gouv.fr/dataset/5a4e84aaa3a7293f8d17f220}{CCFr : Répertoire des manuscrits littéraires français du XXème siècle (Palme)}
\item \href{https://data.gouv.fr/dataset/5ac44de3b5950874a8c04a49}{Collections des musées de France : extrait de la base Joconde}
\item \href{https://data.gouv.fr/dataset/5b435ff2c751df675059dde9}{Collections des musées de France : extrait de la base Joconde (en format XML)}
\item \href{https://data.gouv.fr/dataset/59592252a3a7291dcf9c81d6}{Corpus d'archives numérisées des Archives nationales}
\item \href{https://data.gouv.fr/dataset/567ade27c751df747ec664bc}{Données statistiques des services d’aide à l’archivage dans les centres de gestion}
\item \href{https://data.gouv.fr/dataset/567bb7dbc751df7f0ec664bc}{Données statistiques des services d’archives des établissements publics}
\item \href{https://data.gouv.fr/dataset/56798aeac751df67fec664bd}{Données statistiques des services publics d’archives départementales}
\item \href{https://data.gouv.fr/dataset/5890bf77a3a72974cbf0dc92}{Etablissements cinématographiques en 2016}
\item \href{https://data.gouv.fr/dataset/59a78bf4a3a7294680fd9dc2}{Etablissements cinématographiques en 2016}
\item \href{https://data.gouv.fr/dataset/59592250a3a7291dcf9c81d5}{Fête de la musique 2017}
\item \href{https://data.gouv.fr/dataset/5af120f7a3a7295e54c41a1a}{Fête de la musique 2017}
\item \href{https://data.gouv.fr/dataset/5c6290bd8b4c41056741c019}{Fonds de soutien à l'émergence et à l'innovation dans la presse}
\item \href{https://data.gouv.fr/dataset/5890bf77a3a72974c1f0dc8e}{Fréquentation dans les salles de cinéma}
\item \href{https://data.gouv.fr/dataset/5890bf74a3a72974cbf0dc8c}{Galaxie des offres éditoriales de culture.fr}
\item \href{https://data.gouv.fr/dataset/5890bf79a3a72974cbf0dc96}{Images du Palais-Royal }
\item \href{https://data.gouv.fr/dataset/5890bf76a3a72974cbf0dc8f}{Institutions culturelles productrices de ressources pédagogiques en ligne}
\item \href{https://data.gouv.fr/dataset/5af120deb595087cfabcde7d}{Journées Européenes du Patrimoine - JEP 2017}
\item \href{https://data.gouv.fr/dataset/596ff9dfa3a72957e0f834b3}{Journées Européenes du Patrimoine - JEP 2017}
\item \href{https://data.gouv.fr/dataset/5890bf76a3a72974cbf0dc90}{Journées Européennes du Patrimoine - JEP 2016}
\item \href{https://data.gouv.fr/dataset/5b228739c751df5132e7fa55}{Les  données du hackathon des Archives nationales}
\item \href{https://data.gouv.fr/dataset/5aab4167a3a7291ed12c89f7}{Les entrées d'archives aux Archives nationales}
\item \href{https://data.gouv.fr/dataset/59592252a3a7291dd09c8177}{Les entrées d'archives en 2016 aux Archives nationales}
\item \href{https://data.gouv.fr/dataset/536998f0a3a729239d2050d1}{Liste des diplômes de l'enseignement supérieur Culture}
\item \href{https://data.gouv.fr/dataset/5af1210ab595080488bcde6b}{Liste des établissements de l'enseignement supérieur culture}
\item \href{https://data.gouv.fr/dataset/5890bf72a3a72974cbf0dc88}{Liste des établissements de l'enseignement supérieur culture}
\item \href{https://data.gouv.fr/dataset/53699915a3a729239d20514a}{Liste des Etablissements Publics}
\item \href{https://data.gouv.fr/dataset/549946f4c751df6a78048060}{Liste des films en première exclusivité}
\item \href{https://data.gouv.fr/dataset/5890bf76a3a72974c1f0dc8a}{Liste des objets mobiliers propriété publique classés au titre des Monuments Historiques}
\item \href{https://data.gouv.fr/dataset/5890bf73a3a72974c1f0dc85}{Liste des organismes publics culturels géolocalisés}
\item \href{https://data.gouv.fr/dataset/5bc6cf3a9ce2e72581422d3f}{Liste des services de presse en ligne reconnus}
\item \href{https://data.gouv.fr/dataset/59592252a3a7291dcf9c81d7}{Littoral Occitanie : liste des immeubles protégés}
\item \href{https://data.gouv.fr/dataset/5ac44dfdb5950877d5c04a3d}{Nuit de la lecture 2018}
\item \href{https://data.gouv.fr/dataset/5890bf75a3a72974cbf0dc8e}{Nuit des musées 2016}
\item \href{https://data.gouv.fr/dataset/59592252a3a7291dd09c8176}{Nuit des musées 2017}
\item \href{https://data.gouv.fr/dataset/591c039988ee3851bec4cbf0}{Nuit des musées 2017}
\item \href{https://data.gouv.fr/dataset/5890bf75a3a72974c1f0dc89}{Opérations archéologiques dans la région Pays de la Loire}
\item \href{https://data.gouv.fr/dataset/5890bf76a3a72974c1f0dc8b}{Photographies : Fonds de la guerre 14-18}
\item \href{https://data.gouv.fr/dataset/5890bf77a3a72974cbf0dc93}{Photographies série « Monuments historiques » de 1851 à 1914}
\item \href{https://data.gouv.fr/dataset/5890bf72a3a72974c1f0dc83}{Pratiques culturelles des français - Evolution 1973-2008}
\item \href{https://data.gouv.fr/dataset/5890bf79a3a72974cbf0dc97}{Principales structures culturelles dans la région Pays de la Loire}
\item \href{https://data.gouv.fr/dataset/5890bf79a3a72974c1f0dc91}{Recensement des circulaires de tri et de sélection des archives publiques}
\item et 7 autres jeux de données\end{itemize}

\clearpage
\section{Ministère de l'Agriculture et de l'Alimentation}


\begin{center}
  \includegraphics[width=3cm]{images/orga/e6_47661bd0b94c4eb5a95dc1ba348630-100.png}
\end{center}


Le ministre de l'agriculture et de l'alimentation prépare et met en
œuvre la politique du Gouvernement dans le domaine de l'agriculture, de
la forêt et du bois.

Il prépare et met en œuvre la politique de l'alimentation en liaison
avec le ministre de l'économie, des finances et du commerce extérieur et
le ministre des affaires sociales et de la santé.


\vspace{0.5cm}

\needspace{12\baselineskip}
\subsection*{Agreste- Cartographie des données du recensement agricole
}\index{agriculture}\index{recensement}
  \begin{wrapfigure}{r}{2.5cm}
    \centering
    \qrcode[nolink]{https://data.gouv.fr/dataset/53699038a3a729239d20394f}
  \end{wrapfigure}

Licence : \textbf{Licence Ouverte
}\newline
Créé le : 2014-02-04\newline
Modifié le : 2016-03-16\newline
De 2010-01-01 à 2010-12-31\newline
Granularité : à la commune\newline
Mise à jour : ponctuelle\newline
Popularité : 1 réutilisation,  3 suivis\newline
Mots-clé : \emph{agriculture, recensement
}\newline
Permalien : \url{https://data.gouv.fr/dataset/53699038a3a729239d20394f}\newline

\par
\noindent
    Résultats du recensement agricole 2010 : Des cartes par communes,
canton, département ou régions permettent d'accéder à des données de
cadrage sur les exploitations agricoles et des données sur l'emploi
agricole. Ces données peuvent être visualisées sous forme de tableaux.


\vspace{0.5cm}
\needspace{12\baselineskip}
\subsection*{Agreste - Données communales - Les principaux résultats des recensements
agricoles 2010, 2000 et 1988 par commune
}
  \begin{wrapfigure}{r}{2.5cm}
    \centering
    \qrcode[nolink]{https://data.gouv.fr/dataset/5369931da3a729239d2040d8}
  \end{wrapfigure}

Licence : \textbf{Licence Ouverte
}\newline
Créé le : 2013-07-08\newline
Modifié le : 2016-02-25\newline
De 1988-01-01 à 2010-12-31\newline
Mise à jour : ponctuelle\newline
Popularité : 2 réutilisations,  0 suivi\newline
Mots-clé : \emph{aucun
}\newline
Permalien : \url{https://data.gouv.fr/dataset/5369931da3a729239d2040d8}\newline

\par
\noindent
    Nombre d'exploitations, travail superficie agricole utilisée, superficie
en terres labourables, en cultures permanentes, superficie toujours en
herbe, cheptel, orientation technico-économique de la commune.


\vspace{0.5cm}
\needspace{12\baselineskip}
\subsection*{Agreste - Les principaux résultats des recensements agricoles 2010, 2000
et 1988 par département et canton
}
  \begin{wrapfigure}{r}{2.5cm}
    \centering
    \qrcode[nolink]{https://data.gouv.fr/dataset/53699870a3a729239d204f6b}
  \end{wrapfigure}

Licence : \textbf{Licence Ouverte
}\newline
Créé le : 2013-07-08\newline
Modifié le : 2015-10-30\newline
De 1988-01-01 à 2010-12-31\newline
Mise à jour : ponctuelle\newline
Popularité : 1 réutilisation,  0 suivi\newline
Mots-clé : \emph{aucun
}\newline
Permalien : \url{https://data.gouv.fr/dataset/53699870a3a729239d204f6b}\newline

\par
\noindent
    Nombre d'exploitations, travail superficie agricole utilisée, superficie
en terres labourables, en cultures permanentes, superficie toujours en
herbe, cheptel, orientation technico-économique de la commune


\vspace{0.5cm}
\needspace{12\baselineskip}
\subsection*{Communes de montagne
}
  \begin{wrapfigure}{r}{2.5cm}
    \centering
    \qrcode[nolink]{https://data.gouv.fr/dataset/5369911ea3a729239d203b94}
  \end{wrapfigure}

Licence : \textbf{Licence Ouverte
}\newline
Créé le : 2013-07-08\newline
Modifié le : 2016-03-07\newline
De 2011-01-01 à 2011-12-31\newline
Mise à jour : annuelle\newline
Popularité : 3 réutilisations,  0 suivi\newline
Mots-clé : \emph{aucun
}\newline
Permalien : \url{https://data.gouv.fr/dataset/5369911ea3a729239d203b94}\newline

\par
\noindent
    Le classement des communes en zone de montagne repose sur les
dispositions du règlement n\degree{}1257/1999 du Conseil du 17 mai 1999
concernant le soutien au développement rural et plus particulièrement
sur son article 18 pour la montagne, et la directive 76/401/CEE du
Conseil du 6 avril 1976 (détermination précise des critères pour le
classement en France en zone de montagne).La zone de montagne est
définie, par l'article 18 du règlement 1257/99, comme se caractérisant
par des handicaps liés à l'altitude, à la pente, et/ou au climat, qui
ont pour effet de restreindre de façon conséquente les possibilités
d'utilisation des terres et d'augmenter de manière générale le coût de
tous les travaux.Cette liste de communes zones de montagne sert
notamment au calcul de la dotation globale de fonctionnement des
communes par la DGCL.En France, deux délimitations officielles et
administratives des montagnes se superposent. Les zones dites de
montagne d'une part (elles relèvent d'une approche sectorielle dédiée en
priorité à l'agriculture au titre de la reconnaissance et de la
compensation des handicaps naturels) et d'autre part des massifs
construits pour promouvoir l'auto-développement des territoires de
montagne.Le massif englobe, non seulement les zones de montagne, mais
aussi les zones qui leur sont immédiatement contigües : piémonts, voire
plaines si ces dernières assurent la continuité du massif.La notion de
massif est une approche uniquement française, permettant d'avoir une
entité administrative compétente pour mener à bien la politique de la
montagne.Cette notion de massif est à différencier de la notion de
montagne.


\vspace{0.5cm}
\needspace{12\baselineskip}
\subsection*{Liste des abattoirs de volailles, de lapins (et apparenté) agréés CE /
Meat from poultry and lagomorphs
}
  \begin{wrapfigure}{r}{2.5cm}
    \centering
    \qrcode[nolink]{https://data.gouv.fr/dataset/536998d1a3a729239d205079}
  \end{wrapfigure}

Licence : \textbf{Licence Ouverte
}\newline
Créé le : 2013-07-08\newline
Modifié le : 2015-07-11\newline
Mise à jour : bi-hebdomadaire\newline
Popularité : 1 réutilisation,  2 suivis\newline
Mots-clé : \emph{aucun
}\newline
Permalien : \url{https://data.gouv.fr/dataset/536998d1a3a729239d205079}\newline

\par
\noindent
    en cours d'élaboration par le bureau métier correspondant


\vspace{0.5cm}
\needspace{12\baselineskip}
\subsection*{Liste des ateliers de decoupe de viandes de volailles, de lapins (et
apparentes) agrees CE
}\index{atelier!de!da!c!coupe}\index{etablissements!agra!c!a!c!s!ce}\index{sa!c!curita!c!sanitaire}\index{volailles!et!lapins}
  \begin{wrapfigure}{r}{2.5cm}
    \centering
    \qrcode[nolink]{https://data.gouv.fr/dataset/55f797c2a3a72951efb9838b}
  \end{wrapfigure}

Licence : \textbf{Licence Ouverte
}\newline
Créé le : 2015-09-15\newline
Modifié le : 2019-03-17\newline
Mise à jour : quotienne\newline
Popularité : 1 réutilisation,  0 suivi\newline
Mots-clé : \emph{atelier-de-da-c-coupe, etablissements-agra-c-a-c-s-ce, sa-c-curita-c-sanitaire, volailles-et-lapins
}\newline
Permalien : \url{https://data.gouv.fr/dataset/55f797c2a3a72951efb9838b}\newline

\par
\noindent
    La liste de tous les établissements agréés CE conformément au
règlement(CE) nÂ\degree{}853/2004 est disponible sur le site du
Ministère Ã~ l'adresse
:\url{http://agriculture.gouv.fr/liste-des-etablissements-agrees-ce}


\vspace{0.5cm}
\needspace{12\baselineskip}
\subsection*{Liste des ateliers de découpe de viandes de volailles, de lapins (et
apparentés) agréés CE / Meat from poultry and lagomorphs cutting plant
}
  \begin{wrapfigure}{r}{2.5cm}
    \centering
    \qrcode[nolink]{https://data.gouv.fr/dataset/536998d6a3a729239d205088}
  \end{wrapfigure}

Licence : \textbf{Licence Ouverte
}\newline
Créé le : 2013-07-08\newline
Modifié le : 2015-07-02\newline
Mise à jour : bi-hebdomadaire\newline
Popularité : 1 réutilisation,  2 suivis\newline
Mots-clé : \emph{aucun
}\newline
Permalien : \url{https://data.gouv.fr/dataset/536998d6a3a729239d205088}\newline

\par
\noindent
    en cours d'élaboration par le bureau métier correspondant


\vspace{0.5cm}
\needspace{12\baselineskip}
\subsection*{Liste des massifs forestiers classés en forêts de protection
}
  \begin{wrapfigure}{r}{2.5cm}
    \centering
    \qrcode[nolink]{https://data.gouv.fr/dataset/53699932a3a729239d20519b}
  \end{wrapfigure}

Licence : \textbf{Licence Ouverte
}\newline
Créé le : 2013-07-08\newline
Modifié le : 2016-03-10\newline
Mise à jour : annuelle\newline
Popularité : 1 réutilisation,  1 suivi\newline
Mots-clé : \emph{aucun
}\newline
Permalien : \url{https://data.gouv.fr/dataset/53699932a3a729239d20519b}\newline

\par
\noindent
    Liste des forêts classées sous un régime de protection : données
départementales , libellés, surfaces et statut (public, privé)


\vspace{0.5cm}
\needspace{12\baselineskip}
\subsection*{Résultats aux examens de l'enseignement agricole
}\index{examagri}\index{resultats!bac}\index{resultats!bep}\index{resultats!bts}\index{resultats!cap}\index{resultats!examens}
  \begin{wrapfigure}{r}{2.5cm}
    \centering
    \qrcode[nolink]{https://data.gouv.fr/dataset/596c8c8988ee387e51f2c2f7}
  \end{wrapfigure}

Licence : \textbf{Licence Ouverte
}\newline
Créé le : 2017-07-17\newline
Modifié le : 2017-07-24\newline
Granularité : au point d'intérêt\newline
Popularité : 1 réutilisation,  1 suivi\newline
Mots-clé : \emph{examagri, resultats-bac, resultats-bep, resultats-bts, resultats-cap, resultats-examens
}\newline
Permalien : \url{https://data.gouv.fr/dataset/596c8c8988ee387e51f2c2f7}\newline

\par
\noindent
    Dans le respect des données personnelles des candidats, la mission des
examens de la direction générale de l'enseignement et de la recherche du
ministère en charge de l'agriculture met à disposition, dès leur
publication, l'ensemble des résultats aux examens de l'enseignement
agricole.

Ces résultats sont consultables sur le site dédié
:\url{https://ensagri.agriculture.gouv.fr/arpent-resultats/} Nous
attirons votre attention sur les obligations légales qui découlent de la
réutilisation de ce jeu de données. Le traitement de ces données relève
des obligations de déclaration de la Loi 78-17 du 6 janvier 1978
modifiée, dîte Loi CNIL
:\url{https://www.cnil.fr/fr/loi-78-17-du-6-janvier-1978-modifiee}


\vspace{0.5cm}
\needspace{3\baselineskip} \rule{4cm}{0.25pt}\newline\textbf{Aussi disponible du même producteur :}\begin{itemize}
\item \href{https://data.gouv.fr/dataset/59a02c13c751df05192164a8}{Adresses des unités administratives immatriculées (UAI) du SI de l'enseignement agricole}
\item \href{https://data.gouv.fr/dataset/543f9879c751df200c380179}{Agreste - Bâtiments d’élevage bovin}
\item \href{https://data.gouv.fr/dataset/543f9b3cc751df168b0c9bd2}{Agreste - Bâtiments d’élevage porcin}
\item \href{https://data.gouv.fr/dataset/53698f9da3a729239d2037bd}{Agreste - Bilans d'approvisionnement}
\item \href{https://data.gouv.fr/dataset/53699187a3a729239d203ca7}{Agreste - Conjoncture agricole : le bulletin mensuel}
\item \href{https://data.gouv.fr/dataset/543fbb73c751df168b0c9bd4}{Agreste - Consommation directe et production d’énergie des EDT et des Cuma}
\item \href{https://data.gouv.fr/dataset/536991a0a3a729239d203ce7}{Agreste - Consommations d’énergie dans les IAA et les scieries}
\item \href{https://data.gouv.fr/dataset/536992c2a3a729239d203fe9}{Agreste - Dépenses pour protéger l’environnement dans les industries agroalimentaires}
\item \href{https://data.gouv.fr/dataset/5369931da3a729239d2040d9}{Agreste - Données communales - Main-d’œuvre permanente dans les exploitations agricoles - Champ : ensemble des exploitations - recensements agricoles 2010 et 2000}
\item \href{https://data.gouv.fr/dataset/5369931ea3a729239d2040da}{Agreste - Données communales - Main-d’œuvre permanente dans les exploitations agricoles - Champ : moyennes et grandes exploitations - recensements agricoles 2010 et 2000}
\item \href{https://data.gouv.fr/dataset/5369931ea3a729239d2040db}{Agreste - Données communales – Nombre d'exploitations agricoles et superficies par âge du chef ou du premier coexploitant – champ : ensemble des exploitations - recensements agricoles 2010 et 2000}
\item \href{https://data.gouv.fr/dataset/5369931ea3a729239d2040dc}{Agreste - Données communales – Nombre d'exploitations agricoles et superficies par âge du chef ou du premier coexploitant – champ : moyennes et grandes exploitations - recensements agricoles 2010 et 2000}
\item \href{https://data.gouv.fr/dataset/5369931fa3a729239d2040dd}{Agreste - Données communales – Nombre d'exploitations agricoles et superficies par orientation technico-économique – Champ : ensemble des exploitations - recensements agricoles 2010 et 2000}
\item \href{https://data.gouv.fr/dataset/5369931fa3a729239d2040de}{Agreste - Données communales – Nombre d'exploitations agricoles et superficies par orientation technico-économique – Champ : moyennes et grandes exploitations - recensements agricoles 2010 et 2000}
\item \href{https://data.gouv.fr/dataset/53699320a3a729239d2040df}{Agreste - Données communales - Nombre d'exploitations agricoles et superficies par statut juridique - Champ : ensemble des exploitations - recensements agricoles 2010 et 2000}
\item \href{https://data.gouv.fr/dataset/53699320a3a729239d2040e0}{Agreste - Données communales - Nombre d'exploitations agricoles et superficies par statut juridique - Champ : moyennes et grandes exploitations - recensements agricoles 2010 et 2000}
\item \href{https://data.gouv.fr/dataset/53699321a3a729239d2040e1}{Agreste - Données communales - Nombre d'exploitations agricoles par succession - Champ : ensemble des exploitations - recensements agricoles 2010 et 2000}
\item \href{https://data.gouv.fr/dataset/53699321a3a729239d2040e2}{Agreste - Données communales - Nombre d'exploitations agricoles par succession - Champ : moyennes et grandes exploitations - recensements agricoles 2010 et 2000}
\item \href{https://data.gouv.fr/dataset/53699321a3a729239d2040e3}{Agreste - Données communales - Principales cultures - nombre d’exploitations en cultivant et superficie – Champ : ensemble des exploitations - recensements agricoles 2010 et 2000}
\item \href{https://data.gouv.fr/dataset/53699322a3a729239d2040e4}{Agreste - Données communales - Principales cultures - nombre d’exploitations en cultivant et superficie – Champ : moyennes et grandes exploitations - recensements agricoles 2010 et 2000}
\item \href{https://data.gouv.fr/dataset/53699325a3a729239d2040eb}{Agreste - Données de cadrage sur l'agriculture : surfaces, productions, rendements, effectifs animaux}
\item \href{https://data.gouv.fr/dataset/5369932da3a729239d204107}{Agreste - Données économiques agricoles : les comptes de l'agriculture}
\item \href{https://data.gouv.fr/dataset/53699429a3a729239d204398}{Agreste- Enquête annuelle laitière}
\item \href{https://data.gouv.fr/dataset/543fb86cc751df168e0c9bd3}{Agreste - Enquête eau et assainissement}
\item \href{https://data.gouv.fr/dataset/544a58ebc751df42579bbff9}{Agreste - Enquête Structure 2007}
\item \href{https://data.gouv.fr/dataset/55816d8bc751df3d4b1d4bfc}{Agreste - Enquête structure 2013 }
\item \href{https://data.gouv.fr/dataset/53699443a3a729239d2043da}{Agreste - Enquête sur la production de déchets non dangereux dans l’industrie}
\item \href{https://data.gouv.fr/dataset/53699443a3a729239d2043db}{Agreste - Enquête sur la structure de la forêt privée}
\item \href{https://data.gouv.fr/dataset/53699445a3a729239d2043de}{Agreste - Enquête sur les signes officiels de la qualité et de l’origine}
\item \href{https://data.gouv.fr/dataset/543f769cc751df200a380172}{Agreste - Enquête sur les structures de la production légumière }
\item \href{https://data.gouv.fr/dataset/560a5c7a88ee38595483905a}{Agreste - Entreprises agroalimentaires - Echanges extérieurs}
\item \href{https://data.gouv.fr/dataset/5369954ea3a729239d2046a5}{Agreste - Exploitation forestière, main d'oeuvre permanente}
\item \href{https://data.gouv.fr/dataset/5369954ea3a729239d2046a6}{Agreste - Exploitation forestière, nombre d'entreprises par région, département}
\item \href{https://data.gouv.fr/dataset/56b84de0c751df1f499f9dea}{Agreste - Graphagri}
\item \href{https://data.gouv.fr/dataset/53699715a3a729239d204bd8}{Agreste - Innovation - Entreprises agroalimentaires}
\item \href{https://data.gouv.fr/dataset/5369973ca3a729239d204c3c}{Agreste - Inventaire des vergers}
\item \href{https://data.gouv.fr/dataset/544a1f9cc751df69c59bbff8}{Agreste - Les entreprises d'exploitations forestières et entreprises de sciage, de rabotage et d’imprégnation du bois}
\item \href{https://data.gouv.fr/dataset/53699ad5a3a729239d2055b2}{Agreste - Nombre d' exploitations agricoles aux recensements de l'agriculture 2010 et 2000}
\item \href{https://data.gouv.fr/dataset/543f844dc751df200b380178}{Agreste - Pratiques culturales dans la viticulture}
\item \href{https://data.gouv.fr/dataset/54b8d87ac751df4f445fa5a3}{AGRESTE - Pratiques culturales en arboriculture}
\item \href{https://data.gouv.fr/dataset/53699ddaa3a729239d205d2f}{Agreste - Prix des terres et prés}
\item \href{https://data.gouv.fr/dataset/544a246dc751df69c59bbff9}{Agreste - Prix du bois}
\item \href{https://data.gouv.fr/dataset/53699dfda3a729239d205d84}{Agreste - Production commercialisée des IAA}
\item \href{https://data.gouv.fr/dataset/543fc4a3c751df168e0c9bd5}{Agreste - Recensement de la pisciculture et élevage de crustacés dans les Dom}
\item \href{https://data.gouv.fr/dataset/543fc6e6c751df168c0c9bd5}{Agreste  - Recensement de la pisciculture marine}
\item \href{https://data.gouv.fr/dataset/543fc9e1c751df168f0c9bdb}{Agreste - Recensement de la salmoniculture}
\item \href{https://data.gouv.fr/dataset/543f7c9ac751df200d380174}{Agreste - Recensement de l’horticulture}
\item \href{https://data.gouv.fr/dataset/53699ec4a3a729239d205f71}{Agreste - Récolte de bois et production de sciages}
\item \href{https://data.gouv.fr/dataset/53699f0fa3a729239d206035}{Agreste - Réseau d'information comptable agricole France (RICA France) }
\item \href{https://data.gouv.fr/dataset/5369a204a3a729239d20673c}{Agreste - Technologies de l’information et de la communication - entreprises des IAA}
\item et 223 autres jeux de données\end{itemize}

\clearpage
\section{Ministère de la Justice}


\begin{center}
  \includegraphics[width=3cm]{images/orga/6f_aef3c56db140d09a9e421758047fd7-100.jpg}
\end{center}


La Justice en France est administrée par un ministère, nommé aussi
Chancellerie, dont le titulaire est le garde des Sceaux, ministre de la
Justice.


\vspace{0.5cm}

\needspace{12\baselineskip}
\subsection*{Données géocodées des structures de la Justice
}
  \begin{wrapfigure}{r}{2.5cm}
    \centering
    \qrcode[nolink]{https://data.gouv.fr/dataset/5369932fa3a729239d20410c}
  \end{wrapfigure}

Licence : \textbf{Licence Ouverte
}\newline
Créé le : 2013-07-08\newline
Modifié le : 2018-12-04\newline
De 2011-04-11 à 2011-10-26\newline
Mise à jour : mensuelle\newline
Popularité : 1 réutilisation,  5 suivis\newline
Mots-clé : \emph{aucun
}\newline
Permalien : \url{https://data.gouv.fr/dataset/5369932fa3a729239d20410c}\newline

\par
\noindent
    Liste de l'ensemble des juridictions françaises: ensemble des données
géocodées, horaires, coordonnées + Couverture géographique --- Les
juridictions compétentes, commune par commune - sous la catégorie
`'juridictions'`sont référencées 1 082 adresses, dont les hautes
juridictions et les juridictions de l'ordre judiciaire et
administratif.-les hautes juridictions {[}Cour de Cassation et Conseil
d'Etat{]} ; -les juridictions de l'ordre judiciaire {[}cours d'appel et
chambres détachées de la Cour d'appel ; tribunaux de grande instance et
sections détachées du tribunal de première instance ; tribunaux
d'instance ; conseils de prud'hommes ; tribunaux de commerce, tribunaux
de grande instance à compétence commerciale, tribunaux mixtes de
commerce ; tribunaux pour enfants{]}-les juridictions de l'ordre
administratif {[}cours d'appel administratives et tribunaux
administratifs{]}sous la catégorie Administration pénitentiaire, sont
référencées 293 adresses, dont les établissements pénitentiaires, les
services d'insertion et de probation et des directions interrégionales
des services pénitentiaires.-Les établissements pénitentiaires (centres
de détention ; centres pénitentiaires ; maisons centrales, maisons
d'arrêt et établissements pour mineurs)-Les services d'insertion et de
probation-Les directions interrégionales des services pénitentiaires-
Sous la catégorie Protection Judiciaires de la Jeunesse, sont
référencées 79 adresses, dont les directions interrégionales et
territoriales de la protection judiciaire de la jeunessesous la
catégorie Ecoles et pôles de formation, sont référencées 16 adresses,
soit les écoles du ministère de la Justice et des libertés et les pôles
de formation de la protection judiciaire de la jeunesse.sous la
catégorie Accès au droit, sont référencées toutes les structures Justice
hors milieu pénitentiaire, lieux hors Justice types mairies, centres
médicaux-sociaux, etc. et associations spécialisés dans l'accès au droit
et faisant parti des dispositifs mis en place et coordonnés par les
Conseils Départementaux d'Accès au Droit, dont :-les Conseils
Départementaux d'accès au droit ;-les structures Justice assurant une
présence judiciaire de proximité : les maisons de justice et du droit et
antennes de justice ; -les antennes juridiques et de médiation ; -les
points d'accès au droit généralistes et spécialisés ;-les relais d'accès
au droit généralistes et spécialisés ; -les permanences d'information et
d'orientation juridiques Les données sont extraites de la base Francine
( qui recense l'ensemble des données géocodées des structures de la
Justice), dont la consultation est disponible en ligne sur
l'espace'`Justice en région'' www.annuaires.justice.gouv.fr et via
l'application Mobidroits


\vspace{0.5cm}
\needspace{12\baselineskip}
\subsection*{Les PACS depuis leur création (statistiques)
}\index{couple}\index{justice}\index{pacs}
  \begin{wrapfigure}{r}{2.5cm}
    \centering
    \qrcode[nolink]{https://data.gouv.fr/dataset/53699867a3a729239d204f4b}
  \end{wrapfigure}

Licence : \textbf{Licence Ouverte
}\newline
Créé le : 2013-12-06\newline
Modifié le : 2016-10-12\newline
Granularité : au département\newline
Mise à jour : trimestrielle\newline
Popularité : 1 réutilisation,  4 suivis\newline
Mots-clé : \emph{couple, justice, pacs
}\newline
Permalien : \url{https://data.gouv.fr/dataset/53699867a3a729239d204f4b}\newline

\par
\noindent
    Statistiques sur les PACS (pactes civils de solidarité) conclus ou
dissous en juridiction ou chez un notaire selon le type de PACS, le lieu
d'enregistrement, l'âge des partenaires\ldots{}(statistique publique)


\vspace{0.5cm}
\needspace{12\baselineskip}
\subsection*{Les statistiques sur les divorces
}\index{divorce}\index{divorces}\index{ruptures!d!union}\index{statistiques}
  \begin{wrapfigure}{r}{2.5cm}
    \centering
    \qrcode[nolink]{https://data.gouv.fr/dataset/53699309a3a729239d20409f}
  \end{wrapfigure}

Licence : \textbf{Licence Ouverte
}\newline
Créé le : 2013-07-08\newline
Modifié le : 2016-10-12\newline
Granularité : au département\newline
Mise à jour : annuelle\newline
Popularité : 2 réutilisations,  2 suivis\newline
Mots-clé : \emph{divorce, divorces, ruptures-d-union, statistiques
}\newline
Permalien : \url{https://data.gouv.fr/dataset/53699309a3a729239d20409f}\newline

\par
\noindent
    Depuis juillet 2014, tableaux détaillés sur les ruptures d'union depuis
2010 et tableau historique depuis 1976 en plus du nombre de divorces
prononcés par an, par tribunal de grande instance et département depuis
1999. ``Statistique publique''.


\vspace{0.5cm}
\needspace{12\baselineskip}
\subsection*{Statistique mensuelle de la population écrouée et détenue en France
}\index{amenagement!de!peine}\index{bracelet!electronique}\index{detention}\index{ecrou}\index{liberation!conditionnelle}\index{placement!exterieur}\index{prison}\index{prisonnier}\index{semi!liberte}\index{stock}
  \begin{wrapfigure}{r}{2.5cm}
    \centering
    \qrcode[nolink]{https://data.gouv.fr/dataset/5369a048a3a729239d206322}
  \end{wrapfigure}

Licence : \textbf{Licence Ouverte
}\newline
Créé le : 2013-08-21\newline
Modifié le : 2019-03-07\newline
Mise à jour : mensuelle\newline
Popularité : 2 réutilisations,  2 suivis\newline
Mots-clé : \emph{amenagement-de-peine, bracelet-electronique, detention, ecrou, liberation-conditionnelle, placement-exterieur, prison, prisonnier, semi-liberte, stock
}\newline
Permalien : \url{https://data.gouv.fr/dataset/5369a048a3a729239d206322}\newline

\par
\noindent
    Cette statistique contient des données sur la population écrouée au sein
des établissements pénitentiaires en France. Elle permet de distinguer
le nombre de personnes écrouées détenues (détenu.e.s au sein des
établissements pénitentiaires) du nombre de personnes écrouées non
détenues (placement extérieur non hébergé et placement sous surveillance
électronique).

Données nationales au 1er jour de chaque mois (population répartie selon
le statut (détenu.e.s ou non), le sexe, la catégorie pénale et l'âge
(mineurs/majeurs)).

Suite à un regroupement avec la ``Statistique mensuelle des personnes
écrouées en aménagements de peine'', depuis mai 2015, les tableaux
concernant les aménagements de peine ont été rajoutés à ce jeu de
données.


\vspace{0.5cm}
\needspace{3\baselineskip} \rule{4cm}{0.25pt}\newline\textbf{Aussi disponible du même producteur :}\begin{itemize}
\item \href{https://data.gouv.fr/dataset/53699072a3a729239d2039e1}{Charte de nommage des données publiques}
\item \href{https://data.gouv.fr/dataset/5369932ea3a729239d204108}{Données géocodées des conseils des prud'hommes}
\item \href{https://data.gouv.fr/dataset/5369932ea3a729239d20410a}{Données géocodées des cours d'appel}
\item \href{https://data.gouv.fr/dataset/5369932ea3a729239d20410b}{Données géocodées des juridictions commerciales}
\item \href{https://data.gouv.fr/dataset/58498d76c751df22afc0bb7e}{Indicateurs pénaux trimestriels}
\item \href{https://data.gouv.fr/dataset/539a67b7a3a7293bc2728384}{Les condamnations (statistiques à partir du casier judiciaire national)}
\item \href{https://data.gouv.fr/dataset/53699882a3a729239d204f9f}{Les statistiques par juridiction}
\item \href{https://data.gouv.fr/dataset/53699891a3a729239d204fc8}{Lexique mots clés de la Justice}
\item \href{https://data.gouv.fr/dataset/536998cfa3a729239d20506e}{liste des 150 infractions les plus fréquentes dans les condamnations pénales pour l'année 2010}
\item \href{https://data.gouv.fr/dataset/5369a049a3a729239d206325}{Statistique mensuelle des personnes écrouées en aménagements de peine}
\item \href{https://data.gouv.fr/dataset/5369a0eaa3a729239d206496}{Statistiques trimestrielles de la population prise en charge en milieu fermé}
\item \href{https://data.gouv.fr/dataset/5a01771bc751df5b3e60abea}{Statistique trimestrielle du milieu ouvert (Hors écrou)}
\item \href{https://data.gouv.fr/dataset/5369a307a3a729239d206984}{Tribunaux de grande instance et sections détachées du tribunal de première instance}
\item \href{https://data.gouv.fr/dataset/5369a307a3a729239d206985}{Tribunaux d’instance}
\item \href{https://data.gouv.fr/dataset/5369a307a3a729239d206986}{Tribunaux pour enfants}
\end{itemize}

\clearpage
\section{Ministère de la Transition écologique et solidaire}


\begin{center}
  \includegraphics[width=3cm]{images/orga/e7_31d2ab370142c6b6a5611b55768dd6-100.jpeg}
\end{center}


Le ministère de la Transition écologique et solidaire doit concilier les
urgences environnementales du présent et préparer l'avenir pour les
générations futures.

Le ministère intervient à plusieurs niveaux et notamment au travers des
directions régionales de l'environnement, de l'aménagement et du
logement (DREAL) qui agissent de concert avec les directions
départementales interministérielles, en particulier les directions
départementales des territoires (DDT), pour déployer les politiques
publiques du ministère au plan local. Le ministère assure également la
tutelle de plus d'une soixantaine d'établissements publics, dont
Météo-France, l'IGN, les agences de l'eau, les parcs naturels\ldots{}


\vspace{0.5cm}

\needspace{12\baselineskip}
\subsection*{Aéroports français coordonnées géographiques
}
  \begin{wrapfigure}{r}{2.5cm}
    \centering
    \qrcode[nolink]{https://data.gouv.fr/dataset/53698e7ea3a729239d2034a2}
  \end{wrapfigure}

Licence : \textbf{Licence Ouverte
}\newline
Créé le : 2013-07-08\newline
Modifié le : 2016-03-15\newline
De 2011-01-01 à 2011-01-01\newline
Granularité : au point d'intérêt\newline
Mise à jour : ponctuelle\newline
Popularité : 3 réutilisations,  4 suivis\newline
Mots-clé : \emph{aucun
}\newline
Permalien : \url{https://data.gouv.fr/dataset/53698e7ea3a729239d2034a2}\newline

\par
\noindent
    Coordonnées géographiques des aéroports français recevant un trafic
commercial significatif soit \textgreater{}2000 passagers , région et
ville desservieLes coordonnées sont fournies sous le système WGS84.


\vspace{0.5cm}
\needspace{12\baselineskip}
\subsection*{Assiette des droits de mutation immobiliers
}\index{droits}\index{enregistrement}\index{immobilier}\index{logement}\index{mutation}
  \begin{wrapfigure}{r}{2.5cm}
    \centering
    \qrcode[nolink]{https://data.gouv.fr/dataset/53698f17a3a729239d20364d}
  \end{wrapfigure}

Licence : \textbf{Licence Ouverte
}\newline
Créé le : 2013-12-21\newline
Modifié le : 2016-03-11\newline
Granularité : au département\newline
Mise à jour : mensuelle\newline
Popularité : 1 réutilisation,  1 suivi\newline
Mots-clé : \emph{droits, enregistrement, immobilier, logement, mutation
}\newline
Permalien : \url{https://data.gouv.fr/dataset/53698f17a3a729239d20364d}\newline

\par
\noindent
    Assiette des droits de mutation immobiliers (parfois improprement
appelés
\href{http://www.cgedd.developpement-durable.gouv.fr/frais-de-notaire-et-droits-de-a1414.html}{``frais
de notaire''}), par mois, département et type de droit, calculée à
partir du produit des droits d'enregistrement perçus par la Direction
Générale des Finances Publiques et publiée par le CGEDD (J. Friggit)
dans son dossier sur le
\href{http://www.cgedd.developpement-durable.gouv.fr/prix-immobilier-evolution-1200-a1048.html}{prix
de l'immobilier sur le long terme}. Cette série supporte notamment ces
\href{http://www.cgedd.fr/prix-immobilier-friggit.doc}{graphiques sur
l'évolution du marché immobilier résidentiel}.

\href{http://www.cgedd.developpement-durable.gouv.fr/droits-de-mutation-immobiliers-a1013.html}{Méthode
de calcul}

\href{http://www.cgedd.developpement-durable.gouv.fr/effet-du-raccourcissement-du-delai-a1189.html}{Précision
importante}


\vspace{0.5cm}
\needspace{12\baselineskip}
\subsection*{Base nationale de Gestion ASsistée des Procédures Administratives
relatives aux Risques (GASPAR)
}\index{necmergitur}
  \begin{wrapfigure}{r}{2.5cm}
    \centering
    \qrcode[nolink]{https://data.gouv.fr/dataset/536995eea3a729239d20486b}
  \end{wrapfigure}

Licence : \textbf{Licence Ouverte
}\newline
Créé le : 2013-07-08\newline
Modifié le : 2018-03-27\newline
Granularité : à la commune\newline
Mise à jour : quotienne\newline
Popularité : 3 réutilisations,  1 suivi\newline
Mots-clé : \emph{necmergitur
}\newline
Permalien : \url{https://data.gouv.fr/dataset/536995eea3a729239d20486b}\newline

\par
\noindent
    Recensement de toutes les procédures réalisées au titre de la prévention
des risques : Plan de prévention des risques, Atlas de zone
inondables\ldots{}


\vspace{0.5cm}
\needspace{12\baselineskip}
\subsection*{Bornage du réseau routier national
}\index{bornage}\index{point!de!repere}\index{reseau!routier!national}
  \begin{wrapfigure}{r}{2.5cm}
    \centering
    \qrcode[nolink]{https://data.gouv.fr/dataset/57a83c3dc751df5b90bb5dd5}
  \end{wrapfigure}

Licence : \textbf{Licence Ouverte
}\newline
Créé le : 2016-08-08\newline
Modifié le : 2018-08-01\newline
Mise à jour : annuelle\newline
Popularité : 1 réutilisation,  2 suivis\newline
Mots-clé : \emph{bornage, point-de-repere, reseau-routier-national
}\newline
Permalien : \url{https://data.gouv.fr/dataset/57a83c3dc751df5b90bb5dd5}\newline

\par
\noindent
    Ce jeu de données précise le bornage du réseau routier national. Il est
compatible avec les recommandations de la directive INSPIRE, et peut
s'inscrire dans ce cadre dans la rubrique ``Réseau de transport
routier''

Outre des coordonnées (X,Y,Z) Lambert93, les données contiennent les
attributs suivants : - Route = nom de la route - pr = Point de Repère
routier - depPr = département où se situe le PR - concessionPr = indique
si le PR se trouve sur une route concédée (C) ou non concédée (N) - abs
= abscisse, ou distance au PR (en mètres) - cote = pour les chaussées
séparées, indique si le PR se trouve sur la chaussée droite (D) ou
gauche (G). la valeur ``I'' correspond aux chaussées uniques

\emph{Nota : ces données sont également disponibles aux formats Mid/Mif
et Shp dans le jeu de données ``Liaison du réseau routier national''}


\vspace{0.5cm}
\needspace{12\baselineskip}
\subsection*{Entrepôt d'indicateurs et de données sur l'environnement («~Eider~»)
}\index{changement!climatique}\index{climat}\index{construction}\index{developpement!durable}\index{energie}\index{environnement}\index{logement}\index{soes}\index{ssp}\index{transport}
  \begin{wrapfigure}{r}{2.5cm}
    \centering
    \qrcode[nolink]{https://data.gouv.fr/dataset/5369945ba3a729239d204413}
  \end{wrapfigure}

Licence : \textbf{Licence Ouverte
}\newline
Créé le : 2013-12-12\newline
Modifié le : 2016-08-29\newline
Granularité : au département\newline
Mise à jour : annuelle\newline
Popularité : 4 réutilisations,  3 suivis\newline
Mots-clé : \emph{changement-climatique, climat, construction, developpement-durable, energie, environnement, logement, soes, ssp, transport
}\newline
Permalien : \url{https://data.gouv.fr/dataset/5369945ba3a729239d204413}\newline

\par
\noindent
    Développement durable - source «~Eider~» : - Le but de l'outil
\href{http://www.stats.environnement.developpement-durable.gouv.fr/Eider/}{Eider}
est de répondre au 1er pilier de la convention de Aarhus~: «~améliorer
l'information environnementale délivrée par les autorités publiques,
vis-à-vis des principales données environnementales~». Son objectif
consiste à diffuser auprès du grand public sur le web des données
statistiques sur l'environnement, produites ou collectées par le
service. - Les données, le plus souvent annuelles, sont disponibles à un
échelon agrégé à la région et au département, sous forme de tableaux
détaillés, de séries longues et de portraits régionaux. Leur périmètre
est la France entière ou métropolitaine. - L'outil permet ainsi une
approche multi-thématique sur les domaines~suivants : agriculture, air
et émissions atmosphériques, déchets, eau, état de la flore de la faune
et des écosystèmes terrestres et marins, forêt, littoral, nuisances
sonores, occupation du territoire et paysages, sols, énergie, logement,
transport, emplois environnementaux, territoire et population, économie
et société, radioactivité, risques naturels et technologiques, tourisme.
- Cet outil exploite les technologies web et est accessible avec un
navigateur. -
\href{http://www.statistiques.developpement-durable.gouv.fr/}{service de
l'observation et des statistiques (SOeS)} - système statistique public -
ministère de l'environnement, de l'énergie et de la mer (MEEM)


\vspace{0.5cm}
\needspace{12\baselineskip}
\subsection*{Gestionnaires du réseau routier national
}\index{gestionnaire}\index{reseau!routier!national}\index{rrn}
  \begin{wrapfigure}{r}{2.5cm}
    \centering
    \qrcode[nolink]{https://data.gouv.fr/dataset/57a84262c751df5b85bb5dd4}
  \end{wrapfigure}

Licence : \textbf{Licence Ouverte
}\newline
Créé le : 2016-08-08\newline
Modifié le : 2018-08-01\newline
Mise à jour : annuelle\newline
Popularité : 1 réutilisation,  2 suivis\newline
Mots-clé : \emph{gestionnaire, reseau-routier-national, rrn
}\newline
Permalien : \url{https://data.gouv.fr/dataset/57a84262c751df5b85bb5dd4}\newline

\par
\noindent
    Ce jeu de données précise le nom du gestionnaire de chaque section du
réseau routier national. Il répond aux recommandations de la directive
INSPIRE et peut s'inscrire dans la rubrique ``Réseau de transport
routier - Autorités en charge de la maintenance''.

Chaque section de route est définie par: - son nom (route) - sa longueur
- ses coordonnées de début (D) et de fin (F) exprimées en coordonnées
(X, Y, Z) ou par un système de repérage routier composé de 4 attributs :
- pr = Point de repère routier - depPr = département où se situe le PR -
concessionPr = indique si le PR se trouve sur une section concédée (C)
ou non (N) - abs = abscisse ou distance (en mètres) séparant le point du
PR auquel il se rattache - cote = précise si le PR se trouve sur une
chaussée séparée droite (D) ou gauche (G) ou sur une route à chaussée
unique (I) - concession = indique si la section est concédée (C) ou non
(N) - le nom du gestionnaire


\vspace{0.5cm}
\needspace{12\baselineskip}
\subsection*{Liste des initiatives géolocalisées issues du site
votreenergiepourlafrance.fr
}\index{citoyens}\index{climat}\index{energie}\index{france}\index{initiatives}\index{participation}\index{territoires}\index{transition!energetique}\index{votreenergiepourlafrance}
  \begin{wrapfigure}{r}{2.5cm}
    \centering
    \qrcode[nolink]{https://data.gouv.fr/dataset/563234a688ee384a75531575}
  \end{wrapfigure}

Licence : \textbf{Licence Ouverte
}\newline
Créé le : 2015-10-29\newline
Modifié le : 2016-03-03\newline
De 2015-01-01 à 2015-10-23\newline
Granularité : au point d'intérêt\newline
Mise à jour : ponctuelle\newline
Popularité : 2 réutilisations,  2 suivis\newline
Mots-clé : \emph{citoyens, climat, energie, france, initiatives, participation, territoires, transition-energetique, votreenergiepourlafrance
}\newline
Permalien : \url{https://data.gouv.fr/dataset/563234a688ee384a75531575}\newline

\par
\noindent
    Dans tous les territoires, de nombreux acteurs (entreprises,
collectivités territoriales, associations, particuliers\ldots{}) sont
déjà engagés dans la transition énergétique. Il est important que chacun
puisse échanger sur ses pratiques et partager sur ses initiatives.

Dans le cadre du projet de loi de programmation pour la transition
énergétique, la parole vous est donnée par l'intermédiaire d'un site
participatif : www.votreenergiepourlafrance.fr

La liste détaillée de ces initiatives est maintenant accessible en open
data et comprend: - l'ID de l'initiative - le titre - le contenu - l'url
vers le post - le thème - les données de localisation (latitude,
longitude, pays, adresse, ville) - image


\vspace{0.5cm}
\needspace{12\baselineskip}
\subsection*{Pesticides dans les eaux souterraines
}\index{eaux!souterraines}\index{pesticides}\index{polluants}\index{pollution}\index{sdes}
  \begin{wrapfigure}{r}{2.5cm}
    \centering
    \qrcode[nolink]{https://data.gouv.fr/dataset/594c298ec751df76726294d9}
  \end{wrapfigure}

Licence : \textbf{Licence Ouverte
}\newline
Créé le : 2017-06-22\newline
Modifié le : 2018-08-16\newline
Granularité : au point d'intérêt\newline
Mise à jour : annuelle\newline
Popularité : 3 réutilisations,  0 suivi\newline
Mots-clé : \emph{eaux-souterraines, pesticides, polluants, pollution, sdes
}\newline
Permalien : \url{https://data.gouv.fr/dataset/594c298ec751df76726294d9}\newline

\par
\noindent
    En réponse aux engagements pris lors du Sommet de la Terre de Rio de
1992, la France a fait de l'information environnementale un axe
prioritaire. Le ministère de l'environnement diffuse régulièrement des
informations sur les enjeux liés à la transition écologique, et en
particulier sur l'état des milieux et de la biodiversité et sur
l'exposition aux risques et nuisances. Ces informations portent
notamment sur la contamination des milieux aquatiques, dont la pollution
des eaux souterraines par les pesticides.

Des informations sur la contamination des eaux souterraines par les
pesticides sont régulièrement diffusées. Elles s'appuient sur
l'exploitation des données provenant du réseau de surveillance des
nappes souterraines qui comprend près de 2 200 stations de mesures
réparties sur le territoire français (métropole et outremer).

Le terme « pesticides » (dénommés également « produits phytosanitaires »
ou « produits phytopharmaceutiques ») est un terme générique qui
rassemble notamment les insecticides, les fongicides et les herbicides.
Ils sont majoritairement utilisés en agriculture pour la protection des
récoltes mais également pour l'entretien des jardins (collectivités
locales, particuliers) ou des infrastructures de transports. Les
pesticides peuvent avoir des Concours Data-visualisation des données
Pesticides dans les eaux souterraines -- Règlement -- 15 décembre 2016 3
/ 14 effets toxiques aigus et/ou chroniques tant sur les écosystèmes,
notamment aquatiques, que sur l'homme. Près de 600 pesticides différents
sont recherchés dans les différents échantillons d'eau prélevés dans le
cadre du suivi de la qualité des eaux souterraines. Il en résulte un
volume très important de données. Les substances suivies dans les eaux
souterraines sont les substances actives des produits commercialisés, ou
leurs résidus de dégradation (métabolites).

Une des spécificités du suivi des pesticides dans les eaux souterraines
réside dans le fait que le sous-sol est très souvent constitué d'une
superposition de nappes d'eau souterraine plus ou moins indépendantes
les unes des autres. Un des enjeux consiste à avoir un aperçu de l'état
qualitatif de l'ensemble de ces masses d'eau dont certaines sont
mobilisées pour la production d'eau potable.

Les ressources suivantes sont proposées : - Caractéristiques des
stations de mesures - Caractéristiques des différents pesticides
recherchés : herbicides, insecticides, fongicides, molécules mère,
métabolites, date éventuelle d'interdiction d'usage - Caractéristiques
des différentes masses d'eau souterraines suivies - Résultats des
exploitations des mesures effectuées pour chaque station pour les
millésimes 2007 à 2012 : nombre de mesures par an, pesticides
recherchés, concentrations moyenne annuelle, positionnement par rapport
aux normes en vigueur, etc. - Résultats des exploitations des mesures
effectuées pour chaque station pour les millésimes 2007 à 2012 :
concentrations totales en pesticides et nombre de pesticides par
stations


\vspace{0.5cm}
\needspace{12\baselineskip}
\subsection*{Prix de l'immobilier d'habitation à Paris depuis 1840
}\index{immobilier}\index{paris}\index{prix}
  \begin{wrapfigure}{r}{2.5cm}
    \centering
    \qrcode[nolink]{https://data.gouv.fr/dataset/536c4c9ea3a72933d8d1b3b7}
  \end{wrapfigure}

Licence : \textbf{Licence Ouverte
}\newline
Créé le : 2013-12-19\newline
Modifié le : 2016-01-21\newline
De 1840-01-01 à 2012-12-31\newline
Granularité : au département\newline
Mise à jour : annuelle\newline
Popularité : 1 réutilisation,  0 suivi\newline
Mots-clé : \emph{immobilier, paris, prix
}\newline
Permalien : \url{https://data.gouv.fr/dataset/536c4c9ea3a72933d8d1b3b7}\newline

\par
\noindent
    Cet indice du prix des logements à Paris depuis 1840 a été reconstitué
par J. Friggit (CGEDD) en chaînant l'indice de Gaston Duon sur
1840-1944, un indice reconstitué par une méthode de ventes répétées à
partir des bases de données notariales de 1944 à 1999 et l'indice
Notaires-INSEE à partir de 1999. La méthodologie est décrite au § 1.2.
de la note
``\href{http://www.cgedd.developpement-durable.gouv.fr/IMG/doc/house-price-index-Paris-and-others-secular_cle7fed11.doc}{Comparing
Four Secular Home Price Indices}'' (juin 2008) téléchargeable au § 2.3.
du dossier sur le
\href{http://www.cgedd.developpement-durable.gouv.fr/prix-immobilier-evolution-1200-a1048.html}{prix
de l'immobilier d'habitation sur le long terme} tenu à jour par le
CGEDD. Cet indice supporte notamment ces
\href{http://www.cgedd.fr/prix-immobilier-ministere.htm}{graphiques sur
le prix de l'immobilier d'habitation}.


\vspace{0.5cm}
\needspace{12\baselineskip}
\subsection*{Prix immobilier et revenu dans les agglomérations françaises
}\index{agglomerations}\index{immobilier}\index{prix}\index{revenu}
  \begin{wrapfigure}{r}{2.5cm}
    \centering
    \qrcode[nolink]{https://data.gouv.fr/dataset/53699de7a3a729239d205d4d}
  \end{wrapfigure}

Licence : \textbf{Licence Ouverte
}\newline
Créé le : 2013-12-22\newline
Modifié le : 2016-02-06\newline
De 2006-01-01 à 2006-12-31\newline
Popularité : 1 réutilisation,  0 suivi\newline
Mots-clé : \emph{agglomerations, immobilier, prix, revenu
}\newline
Permalien : \url{https://data.gouv.fr/dataset/53699de7a3a729239d205d4d}\newline

\par
\noindent
    Valeurs numériques du graphique de l'article
\href{http://www.cgedd.developpement-durable.gouv.fr/IMG/doc/prix-immobilier-notaire-agglomeration_cle245eb4.doc}{«
Le lien entre prix des logements et revenu par ménage dans les
principales agglomérations »}, J. Friggit, avril 2010, paru dans la note
de conjoncture des Notaires de France.


\vspace{0.5cm}
\needspace{12\baselineskip}
\subsection*{Schéma de cohérence territoriale - SCOT
}
  \begin{wrapfigure}{r}{2.5cm}
    \centering
    \qrcode[nolink]{https://data.gouv.fr/dataset/53699fb3a3a729239d2061bb}
  \end{wrapfigure}

Licence : \textbf{Licence Ouverte
}\newline
Créé le : 2013-07-08\newline
Modifié le : 2016-03-07\newline
De 2011-01-01 à 2011-12-31\newline
Mise à jour : annuelle\newline
Popularité : 1 réutilisation,  0 suivi\newline
Mots-clé : \emph{aucun
}\newline
Permalien : \url{https://data.gouv.fr/dataset/53699fb3a3a729239d2061bb}\newline

\par
\noindent
    Les schémas de cohérence territoriale (SCOT) succèdent aux schémas
directeurs (SD).Les élus définissent ensemble l'évolution de
l'agglomération et les priorités en matière d'habitat, de commerce, de
zones d'activité, de transports alors que les SD portaient
essentiellement sur la destination des sols sans prendre en compte les
autres politiques au niveau de l'agglomération (urbanisme, logement,
déplacement).Ils seront, par ailleurs, soumis à enquête publique avant
approbation et feront l'objet d'un examen périodique. Leur élaboration
et révision seront simplifiées.


\vspace{0.5cm}
\needspace{12\baselineskip}
\subsection*{Statistiques sur les permis de construire (PC), permis d'aménager (PA)
et déclaration préalable (DP) (base Sitadel)
}\index{amenager}\index{autorise}\index{commence}\index{construire}\index{declaration}\index{local}\index{logement}\index{permis}\index{prealable}\index{projet}\index{sdes}\index{sitadel}\index{ssp}\index{utilisation}
  \begin{wrapfigure}{r}{2.5cm}
    \centering
    \qrcode[nolink]{https://data.gouv.fr/dataset/53699c21a3a729239d2058d4}
  \end{wrapfigure}

Licence : \textbf{Licence Ouverte
}\newline
Créé le : 2013-12-11\newline
Modifié le : 2018-04-04\newline
Granularité : à la commune\newline
Mise à jour : annuelle\newline
Popularité : 1 réutilisation,  5 suivis\newline
Mots-clé : \emph{amenager, autorise, commence, construire, declaration, local, logement, permis, prealable, projet, sdes, sitadel, ssp, utilisation
}\newline
Permalien : \url{https://data.gouv.fr/dataset/53699c21a3a729239d2058d4}\newline

\par
\noindent
    Il s'agit de statistiques agrégées tirées des permis de construire (PC),
des permis d'aménager (PA) et des déclarations préalables (DP).

Ces statistiques sont produites par le
\href{http://www.statistiques.developpement-durable.gouv.fr/}{Service de
la donnée et des études statistiques} (SDES)


\vspace{0.5cm}
\needspace{12\baselineskip}
\subsection*{Trafic aéroport Bordeaux-Mérignac : passagers et mouvements
}
  \begin{wrapfigure}{r}{2.5cm}
    \centering
    \qrcode[nolink]{https://data.gouv.fr/dataset/5369a247a3a729239d2067de}
  \end{wrapfigure}

Licence : \textbf{Licence Ouverte
}\newline
Créé le : 2013-07-08\newline
Modifié le : 2016-01-30\newline
De 1986-01-01 à 2011-12-31\newline
Mise à jour : annuelle\newline
Popularité : 1 réutilisation,  0 suivi\newline
Mots-clé : \emph{aucun
}\newline
Permalien : \url{https://data.gouv.fr/dataset/5369a247a3a729239d2067de}\newline

\par
\noindent
    Transport aérien- serie longue de données de trafic- pour l'aéroport de
Bordeaux-Mérignac


\vspace{0.5cm}
\needspace{12\baselineskip}
\subsection*{Trafic international, répartiton par zones géographiques : passagers
}
  \begin{wrapfigure}{r}{2.5cm}
    \centering
    \qrcode[nolink]{https://data.gouv.fr/dataset/5369a24ea3a729239d2067f0}
  \end{wrapfigure}

Licence : \textbf{Licence Ouverte
}\newline
Créé le : 2013-07-08\newline
Modifié le : 2015-06-16\newline
De 1986-01-01 à 2011-12-31\newline
Mise à jour : annuelle\newline
Popularité : 1 réutilisation,  3 suivis\newline
Mots-clé : \emph{aucun
}\newline
Permalien : \url{https://data.gouv.fr/dataset/5369a24ea3a729239d2067f0}\newline

\par
\noindent
    Transport aérien, série longue de données de trafic. Nombres de
passagers à l'international par année et par zone géographique.


\vspace{0.5cm}
\needspace{3\baselineskip} \rule{4cm}{0.25pt}\newline\textbf{Aussi disponible du même producteur :}\begin{itemize}
\item \href{https://data.gouv.fr/dataset/59a04567c751df2e409fa755}{Accidents impliquant des dispositifs de signalisation lumineuse sur le réseau routier national non concédé}
\item \href{https://data.gouv.fr/dataset/53698f52a3a729239d2036fd}{Base documentaire DTRF (Documentation des Techniques Routières Françaises)}
\item \href{https://data.gouv.fr/dataset/5981d44988ee381cca6e5ccd}{Carrefours du réseau routier national non concédé}
\item \href{https://data.gouv.fr/dataset/57a88228c751df3053bb5dd5}{Classes d'état des chaussées du réseau routier national non concédé}
\item \href{https://data.gouv.fr/dataset/59887d57c751df03005fa2c1}{Classification du réseau routier national}
\item \href{https://data.gouv.fr/dataset/55793883c751df494de57269}{Comptages 2008-2011 Réseau Routier National d'Ile-de-France}
\item \href{https://data.gouv.fr/dataset/57c316e888ee381eb5b627a9}{Dépenses d'entretien et d'exploitation du réseau routier national non concédé}
\item \href{https://data.gouv.fr/dataset/59a03e0bc751df18e1efd4a8}{Echangeurs du réseau routier national concédé}
\item \href{https://data.gouv.fr/dataset/536993b8a3a729239d204287}{Emission CO2 et évolution de l'efficience énergétique par faisceau}
\item \href{https://data.gouv.fr/dataset/57a88850c751df16d7bb5dd4}{Etat général des ouvrages d'art sur le réseau routier national non concédé}
\item \href{https://data.gouv.fr/dataset/57a88dc7c751df4b68bb5dd4}{Fortes pentes du réseau routier national}
\item \href{https://data.gouv.fr/dataset/59a0416ec751df23aa4d2736}{Gares de péage du réseau routier national concédé}
\item \href{https://data.gouv.fr/dataset/57a86d2ec751df1c14bb5dd4}{Hiérarchisation du réseau routier national}
\item \href{https://data.gouv.fr/dataset/59c50320c751df79c0e4dc73}{Indicateurs des objectifs du développement durable (ODD)}
\item \href{https://data.gouv.fr/dataset/5bbe22e28b4c414c5175a28e}{Indices globaux d'état des chaussées du réseau routier national non concédé entre 2015 et 2017}
\item \href{https://data.gouv.fr/dataset/57a87753c751df39e6bb5dd4}{Largeur de routes sur le réseau routier national}
\item \href{https://data.gouv.fr/dataset/53699859a3a729239d204f28}{Les ménages et leur logement depuis 1955 et 1970: quelques résultats extraits des enquêtes logement}
\item \href{https://data.gouv.fr/dataset/57a837e2c751df5b90bb5dd4}{Liaisons du réseau routier national}
\item \href{https://data.gouv.fr/dataset/57a8735dc751df1c14bb5dd5}{Nature des routes du réseau routier national}
\item \href{https://data.gouv.fr/dataset/53699acda3a729239d20559f}{Nombre de ventes de maisons et appartements anciens}
\item \href{https://data.gouv.fr/dataset/53699b0da3a729239d205634}{Obligations de service public}
\item \href{https://data.gouv.fr/dataset/53699b59a3a729239d2056f8}{OSP}
\item \href{https://data.gouv.fr/dataset/57a89139c751df4b68bb5dd5}{Passages à niveau sur le réseau routier national }
\item \href{https://data.gouv.fr/dataset/53699c8fa3a729239d2059e0}{Piézomètrie France entière (ADES)}
\item \href{https://data.gouv.fr/dataset/53699cbba3a729239d205a4b}{Plan local de l'urbanisme - PLU}
\item \href{https://data.gouv.fr/dataset/53699cc2a3a729239d205a5b}{Plan strategique des IIC}
\item \href{https://data.gouv.fr/dataset/536c4c9ea3a72933d8d1b3b6}{Prix de l'immobilier d'habitation à Paris depuis 1200}
\item \href{https://data.gouv.fr/dataset/53699de7a3a729239d205d4e}{Prix immobilier et revenu dans les villes d'Ile-de-France}
\item \href{https://data.gouv.fr/dataset/53699e2aa3a729239d205df3}{Produits Biocides}
\item \href{https://data.gouv.fr/dataset/53699e58a3a729239d205e5e}{Programme local de l'habitat - PLH}
\item \href{https://data.gouv.fr/dataset/53699e7ba3a729239d205eb0}{qualité des eaux souterraines France entière (ADES)}
\item \href{https://data.gouv.fr/dataset/5a5f3b8988ee385e8e9e1c75}{Rapport du Gouvernement au Parlement sur les colonnes montantes d'électricité}
\item \href{https://data.gouv.fr/dataset/579f1f8988ee385456d73ff5}{Routes européennes}
\item \href{https://data.gouv.fr/dataset/5bf5465e634f413c0535ebd4}{Titres de qualification des marins }
\item \href{https://data.gouv.fr/dataset/5369a247a3a729239d2067df}{Trafic aéroport Lyon-Saint-Exupéry : passagers et mouvements}
\item \href{https://data.gouv.fr/dataset/5369a248a3a729239d2067e0}{Trafic aéroport Marseille-Provence : passagers et mouvements}
\item \href{https://data.gouv.fr/dataset/5369a248a3a729239d2067e1}{Trafic aéroport Montpellier-Méditerranée : passagers et mouvements}
\item \href{https://data.gouv.fr/dataset/5369a248a3a729239d2067e2}{Trafic aéroport Nantes-Atlantique : passagers et mouvements}
\item \href{https://data.gouv.fr/dataset/5369a249a3a729239d2067e3}{Trafic aéroport Nice-Côte d'Azur : passagers et mouvements}
\item \href{https://data.gouv.fr/dataset/5369a249a3a729239d2067e4}{Trafic aéroport Paris-CDG : passagers et mouvements}
\item \href{https://data.gouv.fr/dataset/5369a24aa3a729239d2067e5}{Trafic aéroport Paris-Orly : passagers et mouvements}
\item \href{https://data.gouv.fr/dataset/5369a24aa3a729239d2067e6}{Trafic aéroport Strasbourg-Entzheim : passagers et mouvements}
\item \href{https://data.gouv.fr/dataset/5369a24ba3a729239d2067e7}{Trafic aéroport Toulouse-Blagnac : passagers et mouvements}
\item \href{https://data.gouv.fr/dataset/5369a24ca3a729239d2067eb}{Trafic des 10 premières liaisons Paris - Régions : passagers}
\item \href{https://data.gouv.fr/dataset/5369a24da3a729239d2067ec}{Trafic des 10 premières liaisons transversales : passagers}
\item \href{https://data.gouv.fr/dataset/5369a24ea3a729239d2067ee}{Trafic en Passagers -Kilomètres-Transportés}
\item \href{https://data.gouv.fr/dataset/5369a24fa3a729239d2067f2}{Trafic Métropole Intérieur : passagers et mouvements}
\item \href{https://data.gouv.fr/dataset/5369a24fa3a729239d2067f3}{Trafic Métropole - International : passagers et mouvements}
\item \href{https://data.gouv.fr/dataset/5369a250a3a729239d2067f4}{Trafic Métropole - Outre Mer : passagers et mouvements}
\item \href{https://data.gouv.fr/dataset/598892dcc751df3eae8bd6ec}{Trafic moyen journalier annuel sur le réseau routier national}
\item et 7 autres jeux de données\end{itemize}

\clearpage
\section{Ministère de l'Economie, de l'Industrie et du Numérique}


\begin{center}
  \includegraphics[width=3cm]{images/orga/90_e2ebc2d5ea4b1b95639640d46c59db-100.png}
\end{center}


Le ministère de l'Économie, de l'Industrie et du Numérique prépare et
met en œuvre la politique du Gouvernement en matière économique ainsi
qu'en matière d'industrie, de services, de petites et moyennes
entreprises, d'artisanat, de commerce, de postes et communications
électroniques, de suivi et de soutien des activités touristiques,
d'économie numérique et d'innovation.

A ce titre, il définit les mesures propres à promouvoir la croissance et
la compétitivité de l'économie française et à encourager et orienter
l'investissement. Il est responsable de la préparation des scénarios
macroéconomiques pour la France et son environnement international. Il
est compétent pour le financement des entreprises en dette et en fonds
propres, en particulier des petites et moyennes entreprises et des
entreprises de taille intermédiaire. Il est chargé de la promotion et du
développement de l'économie sociale et solidaire. Il est responsable de
la politique en faveur de la création d'entreprises et de la
simplification des formalités leur incombant. Il exerce la tutelle des
établissements des réseaux des chambres de métiers et de l'artisanat et
des chambres de commerce et d'industrie.


\vspace{0.5cm}

\needspace{12\baselineskip}
\subsection*{Consultation sur le projet de loi République numérique
}\index{consultation}\index{loi}
  \begin{wrapfigure}{r}{2.5cm}
    \centering
    \qrcode[nolink]{https://data.gouv.fr/dataset/5668697e88ee381d74af0bf4}
  \end{wrapfigure}

Licence : \textbf{Licence Ouverte
}\newline
Créé le : 2015-12-09\newline
Modifié le : 2016-02-23\newline
De 2015-09-26 à 2015-10-18\newline
Granularité : au pays\newline
Popularité : 4 réutilisations,  2 suivis\newline
Mots-clé : \emph{consultation, loi
}\newline
Permalien : \url{https://data.gouv.fr/dataset/5668697e88ee381d74af0bf4}\newline

\par
\noindent
    Les données proposées sont extraites de la plateforme web
www.republique-numerique.fr utilisée pour la consultation sur le projet
de loi République numérique, présenté par Axelle Lemaire, Secrétaire
d'État chargée du numérique. Cette consultation s'est tenue du 26
septembre au 18 octobre 2015.

Ces données retracent les contributions de 21 330 contributeurs qui ont
voté près de 150 000 fois et déposé plus de 8500 arguments, amendements
et propositions. Quatre types de contributions étaient possibles sur la
plateforme : 1/ propositions (ajout de nouveaux articles), 2/
modification (proposition de modification d'articles), 3/ arguments,
sources, etc (présentés en colonne H), 4/ vote.

La variable ``Auteur'' a été retirée, afin de pseudonymiser complètement
les données. La variable ``Type de profil'' a été conservée afin de
savoir si la contribution ou le vote provenait d'un citoyen, d'une
institution, d'une organisation à but lucratif, d'une organisation à but
non lucratif ou d'un acteur non identifié.

Ce jeu de données est encodé en UTF-8 et les champs sont séparés par des
virgules.


\vspace{0.5cm}
\needspace{12\baselineskip}
\subsection*{Les professions libérales
}\index{emploi}\index{professions!liberales}
  \begin{wrapfigure}{r}{2.5cm}
    \centering
    \qrcode[nolink]{https://data.gouv.fr/dataset/53699871a3a729239d204f6e}
  \end{wrapfigure}

Licence : \textbf{Licence Ouverte
}\newline
Créé le : 2014-01-14\newline
Modifié le : 2017-12-04\newline
De 2011-01-01 à 2011-12-31\newline
Granularité : au département\newline
Mise à jour : annuelle\newline
Popularité : 1 réutilisation,  5 suivis\newline
Mots-clé : \emph{emploi, professions-liberales
}\newline
Permalien : \url{https://data.gouv.fr/dataset/53699871a3a729239d204f6e}\newline

\par
\noindent
    \textbf{Effectif par profession non salariée et par département, par
tranche d'âge, par sexe.}

\emph{Les professions libérales groupent les personnes exerçant à titre
habituel, de manière indépendante et sous leur responsabilité, une
activité de nature généralement civile ayant pour objet d'assurer, dans
l'intérêt du client ou du public, des prestations principalement
intellectuelles, techniques ou de soins mises en oeuvre au moyen de
qualifications professionnelles appropriées et dans le respect de
principes éthiques ou d'une déontologie professionnelle, sans préjudice
des dispositions législatives applicables aux autres formes de travail
indépendant.}


\vspace{0.5cm}
\needspace{12\baselineskip}
\subsection*{Marque d'Etat Tourisme \& Handicap
}\index{marque!d!etat}\index{tourisme!handicap}
  \begin{wrapfigure}{r}{2.5cm}
    \centering
    \qrcode[nolink]{https://data.gouv.fr/dataset/5369983ea3a729239d204ee0}
  \end{wrapfigure}

Licence : \textbf{Licence Ouverte
}\newline
Créé le : 2013-07-08\newline
Modifié le : 2019-03-13\newline
Mise à jour : trimestrielle\newline
Popularité : 2 réutilisations,  6 suivis\newline
Mots-clé : \emph{marque-d-etat, tourisme-handicap
}\newline
Permalien : \url{https://data.gouv.fr/dataset/5369983ea3a729239d204ee0}\newline

\par
\noindent
    Tourisme \& Handicap est l'unique marque d'État attribuée aux
professionnels du tourisme qui œuvrent en faveur de l'accessibilité pour
tous. Elle a pour objectif d'apporter une information objective et
homogène sur l'accessibilité des sites et des équipements touristiques.
Tourisme \& Handicap prend en compte les quatre familles de handicaps
(auditif, mental, moteur et visuel) et vise à développer une offre
touristique adaptée et intégrée à l'offre généraliste. Pour obtenir
cette marque, le prestataire doit s'engager dans une démarche exigeante
(critères précis) et vérifiée tous les 5 ans (visite d'évaluation).

Pour en savoir plus :

\begin{itemize}

\item
  \href{https://www.entreprises.gouv.fr/tourisme-handicap}{Site Tourisme
  \& Handicap}
\item
  \href{https://www.entreprises.gouv.fr/tourisme-handicap/tourisme-handicap-allez-la-ou-envies-vous-portent}{Annuaire
  en ligne des établissements marqués Tourisme \& Handicap}
\item
  \href{http://www.entreprises.gouv.fr/marques-nationales-tourisme/presentation-tourisme-et-handicap}{Site
  des marques nationales du tourisme}
\end{itemize}


\vspace{0.5cm}
\needspace{12\baselineskip}
\subsection*{Périmètre des interventions économiques analysées dans le cadre de la
mission MAP sur les aides aux entreprises
}
  \begin{wrapfigure}{r}{2.5cm}
    \centering
    \qrcode[nolink]{https://data.gouv.fr/dataset/53699c14a3a729239d2058b5}
  \end{wrapfigure}

Licence : \textbf{Licence Ouverte
}\newline
Créé le : 2013-07-08\newline
Modifié le : 2015-05-20\newline
De 2009-01-01 à 2013-12-31\newline
Mise à jour : annuelle\newline
Popularité : 1 réutilisation,  0 suivi\newline
Mots-clé : \emph{aucun
}\newline
Permalien : \url{https://data.gouv.fr/dataset/53699c14a3a729239d2058b5}\newline

\par
\noindent
    Le tableau contient la liste de l'ensemble des interventions étudiées
dans le cadre de la mission avec pour chaque dispositif : des données
budgétaires publiques que l'on retrouve notamment dans les documents
annexés aux projet de loi de finances, des informations budgétaires
permettant de qualifier l'intervention, des informations juridiques sur
les textes régissant la mesure et des précisions traduisant l'analyse et
la classification opérée sur le dispositif dans le cadre de l'étude.

Le premier type d'information est relativement fiable et objectif. Les
deuxième et troisième types sont plus susceptibles de contenir des
erreurs. Le dernier type relève d'une appréciation propre à la mission.


\vspace{0.5cm}
\needspace{12\baselineskip}
\subsection*{Pôles de compétitivité : Données sur les membres adhérents, 2011 à 2015
}\index{competitivite}\index{membres}\index{poles}\index{poles!de!competitivite}
  \begin{wrapfigure}{r}{2.5cm}
    \centering
    \qrcode[nolink]{https://data.gouv.fr/dataset/53699cffa3a729239d205b07}
  \end{wrapfigure}

Licence : \textbf{Licence Ouverte
}\newline
Créé le : 2013-07-08\newline
Modifié le : 2018-10-04\newline
De 2011-01-01 à 2015-12-31\newline
Mise à jour : annuelle\newline
Popularité : 1 réutilisation,  3 suivis\newline
Mots-clé : \emph{competitivite, membres, poles, poles-de-competitivite
}\newline
Permalien : \url{https://data.gouv.fr/dataset/53699cffa3a729239d205b07}\newline

\par
\noindent
    \textbf{Les principales caractéristiques des pôles de compétitivité :
localisation, entreprises et établissements membres, projets.}

\emph{Ces données sont issues de l'enquête annuelle auprès des
gouvernances des pôles de compétitivité puis de leur appariement avec
les bases de données de l'Insee. Ces tableaux décrivent les principales
caractéristiques des entreprises membres pour chacun des pôles de
compétitivité ainsi que pour l'ensemble des pôles (nombre d'entreprises
membres, localisation des établissements des entreprises membres,
répartition PME / ETI / grandes entreprises, taux
d'exportation,\ldots{}).}


\vspace{0.5cm}
\needspace{3\baselineskip} \rule{4cm}{0.25pt}\newline\textbf{Aussi disponible du même producteur :}\begin{itemize}
\item \href{https://data.gouv.fr/dataset/53698eb0a3a729239d203525}{Aide à la réindustrialisation}
\item \href{https://data.gouv.fr/dataset/53699108a3a729239d203b5f}{Commissaires aux comptes des entreprises relevant de l'APE}
\item \href{https://data.gouv.fr/dataset/5369912ca3a729239d203bb8}{Composition des conseils d'administration ou de surveillance des entreprises relevant de l'APE}
\item \href{https://data.gouv.fr/dataset/5369914ca3a729239d203c10}{Comptes Combinés 2012}
\item \href{https://data.gouv.fr/dataset/53699374a3a729239d2041ce}{Effectifs par entreprise relevant de l'APE ventillés par secteurs d'activités (en ETP)}
\item \href{https://data.gouv.fr/dataset/55ffde17c751df563949df42}{Fuel prices in France}
\item \href{https://data.gouv.fr/dataset/53699794a3a729239d204d1e}{La Base Économique des Entreprises Régionales}
\item \href{https://data.gouv.fr/dataset/5761babc88ee382661640391}{Le secteur du tourisme}
\item \href{https://data.gouv.fr/dataset/5369984da3a729239d204f09}{Les jeunes entreprises innovantes}
\item \href{https://data.gouv.fr/dataset/5369983ea3a729239d204edf}{Marque d'Etat Qualité Tourisme™}
\item \href{https://data.gouv.fr/dataset/53699cffa3a729239d205b08}{Pôles de compétitivité : nombre de projets déposés et retenus au Fonds Unique Interministériel (FUI) et leur financement par l’État et les collectivités locales, 2006-2018}
\item \href{https://data.gouv.fr/dataset/5bb61a378b4c4118b11dd650}{Pôles de compétitivité : nombre de projets de recherche et développement et d’innovation (RDI) et aides financières associées, par pôle, 2006-2016}
\item \href{https://data.gouv.fr/dataset/53699d5fa3a729239d205bff}{Portefeuille des participations cotées de l'Etat}
\item \href{https://data.gouv.fr/dataset/53699da4a3a729239d205ca8}{Présidents des conseils d'administrations ou de surveillances des entreprises du périmètre de l'APE}
\item \href{https://data.gouv.fr/dataset/5369a136a3a729239d206551}{Synthèse des comptes 2012 des principales entreprises à participations publiques}
\end{itemize}

\clearpage
\section{Ministère de l'économie et des finances}


\begin{center}
  \includegraphics[width=3cm]{images/orga/68_0660a939e7495d94117e0d9845d6f1-100.png}
\end{center}


Le ministère de l'Economie et des Finances est chargé de mettre en œuvre
la politique du Gouvernement en matière économique, financière,
budgétaire et fiscale. Ses missions recouvrent des champs d'action
variés tels que l'industrie, le commerce, les services, l'innovation
mais aussi la gestion des comptes publics et la définition de la
stratégie des finances publiques (affaires monétaires, impôts,
douanes\ldots{}). Elles sont assurées par les différents services et
directions qui composent le ministère.


\vspace{0.5cm}

\needspace{12\baselineskip}
\subsection*{Adresses des débits de tabac
}\index{adresse}\index{debits!de!tabac}\index{designation!commerciale}\index{dgddi}\index{douane}\index{tabac}
  \begin{wrapfigure}{r}{2.5cm}
    \centering
    \qrcode[nolink]{https://data.gouv.fr/dataset/555b3ec7c751df4633190c7a}
  \end{wrapfigure}

Licence : \textbf{Licence Ouverte
}\newline
Créé le : 2015-05-19\newline
Modifié le : 2018-05-07\newline
De 2015-01-01 à 2017-12-31\newline
Granularité : au point d'intérêt\newline
Mise à jour : ponctuelle\newline
Popularité : 2 réutilisations,  2 suivis\newline
Mots-clé : \emph{adresse, debits-de-tabac, designation-commerciale, dgddi, douane, tabac
}\newline
Permalien : \url{https://data.gouv.fr/dataset/555b3ec7c751df4633190c7a}\newline

\par
\noindent
    Désignation commerciale et adresse postale des débits de tabac en
activité.

\textbf{Contenu (descriptifs des colonnes):} - ID : numéro de ligne -
Enseigne : nom du commerce - Adresse~: indication principale de
l'adresse - Complément : indication permettant une meilleure
localisation (centre commercial, etc.) ou complément de dénomination -
Code postal - Commune - (depuis 2018) Nature du débit de tabac~:
ordinaire permanent / ordinaire saisonnier / spécial


\vspace{0.5cm}
\needspace{12\baselineskip}
\subsection*{Annuaire statistique
}\index{amende}\index{annuaire!statistique}\index{audiovisuel!public}\index{cadastre}\index{cet}\index{contentieux}\index{controle!fiscal}\index{cvae}\index{departement}\index{dgfip}\index{economie}\index{fiscalite}\index{fonciere}\index{impot}\index{impots!locaux}\index{irpp}\index{isf}\index{moyens!de!paiement}\index{recettes}\index{recouvrement}\index{taxe}\index{tva}\index{urbanisme}\index{usage!de!bureau}
  \begin{wrapfigure}{r}{2.5cm}
    \centering
    \qrcode[nolink]{https://data.gouv.fr/dataset/53698ee1a3a729239d2035bc}
  \end{wrapfigure}

Licence : \textbf{Licence Ouverte
}\newline
Créé le : 2014-03-07\newline
Modifié le : 2018-10-08\newline
De 2004-01-01 à 2017-12-31\newline
Granularité : au département\newline
Mise à jour : annuelle\newline
Popularité : 1 réutilisation,  5 suivis\newline
Mots-clé : \emph{amende, annuaire-statistique, audiovisuel-public, cadastre, cet, contentieux, controle-fiscal, cvae, departement, dgfip, economie, fiscalite, fonciere, impot, impots-locaux, irpp, isf, moyens-de-paiement, recettes, recouvrement, taxe, tva, urbanisme, usage-de-bureau
}\newline
Permalien : \url{https://data.gouv.fr/dataset/53698ee1a3a729239d2035bc}\newline

\par
\noindent
    Les ex-Directions générales des impôts et de la comptabilité publique
ont fusionné en 2008 en une direction unique. L'annuaire statistique de
la Direction générale des Finances publiques (DGFiP) retrace l'activité
de cette dernière pour une année civile de référence. Il recense sous
forme de tableaux les principales données relatives :

\begin{itemize}
\item
  aux fiscalités personnelle, professionnelle et directe locale,
\item
  aux moyens de paiement,
\item
  au contrôle fiscal et au contentieux
\item
  aux affaires foncières et domaniales (cadastre, publicité foncière et
  domaine)
\item
  aux amendes.
\end{itemize}

L'atlas fiscal, nouvelle publication de la DGFiP, accompagne et illustre
certains millésimes de l'annuaire statistique. Rédigé à partir des
données de celui-ci, il propose une approche du territoire national à
travers le prisme de la fiscalité.

Retrouvez toutes les publications statistiques de la DGFiP sur le site
\href{http://www.impots.gouv.fr/portal/dgi/public/statistiques?espId=-4\&pageId=statistiques\&sfid=450}{www.impots.gouv.fr}.

Les données fournies par la Direction générale des Finances publiques
sont gratuites et peuvent être utilisées dans les conditions et limites
fixées par la loi n\degree{} 78-753 d


\vspace{0.5cm}
\needspace{12\baselineskip}
\subsection*{Balances comptables des départements
}\index{balances}\index{comptabilite!publique}\index{departements}
  \begin{wrapfigure}{r}{2.5cm}
    \centering
    \qrcode[nolink]{https://data.gouv.fr/dataset/5537bf2bc751df66296dd01b}
  \end{wrapfigure}

Licence : \textbf{Licence Ouverte
}\newline
Créé le : 2015-04-22\newline
Modifié le : 2019-02-18\newline
Granularité : au département\newline
Mise à jour : annuelle\newline
Popularité : 1 réutilisation,  11 suivis\newline
Mots-clé : \emph{balances, comptabilite-publique, departements
}\newline
Permalien : \url{https://data.gouv.fr/dataset/5537bf2bc751df66296dd01b}\newline

\par
\noindent
    Balances des budgets principaux et annexes des départements de 2010 à
2017.


\vspace{0.5cm}
\needspace{12\baselineskip}
\subsection*{Balances comptables des groupements à fiscalité propre
}\index{balances}\index{comptabilite!publique}\index{epci!a!fiscalite!propre}\index{groupements!a!fiscalite!propre}
  \begin{wrapfigure}{r}{2.5cm}
    \centering
    \qrcode[nolink]{https://data.gouv.fr/dataset/55520575c751df74e9190c79}
  \end{wrapfigure}

Licence : \textbf{Licence Ouverte
}\newline
Créé le : 2015-05-12\newline
Modifié le : 2019-02-18\newline
Granularité : à l'EPCI\newline
Mise à jour : annuelle\newline
Popularité : 2 réutilisations,  10 suivis\newline
Mots-clé : \emph{balances, comptabilite-publique, epci-a-fiscalite-propre, groupements-a-fiscalite-propre
}\newline
Permalien : \url{https://data.gouv.fr/dataset/55520575c751df74e9190c79}\newline

\par
\noindent
    Balances des budgets principaux et budgets annexes des groupements à
fiscalité propre de 2010 à 2017. Pour l'exercice 2017, le fichier
comprend les balances comptables des établissements publics de
territoire. En 2016, les balances de ces établissements figurent dans le
fichier des syndicats.


\vspace{0.5cm}
\needspace{12\baselineskip}
\subsection*{Balances comptables des régions
}\index{balances}\index{comptabilite!publique}\index{regions}
  \begin{wrapfigure}{r}{2.5cm}
    \centering
    \qrcode[nolink]{https://data.gouv.fr/dataset/5527d0fec751df579caec475}
  \end{wrapfigure}

Licence : \textbf{Licence Ouverte
}\newline
Créé le : 2015-04-10\newline
Modifié le : 2019-02-18\newline
Granularité : à la région\newline
Mise à jour : annuelle\newline
Popularité : 1 réutilisation,  14 suivis\newline
Mots-clé : \emph{balances, comptabilite-publique, regions
}\newline
Permalien : \url{https://data.gouv.fr/dataset/5527d0fec751df579caec475}\newline

\par
\noindent
    Balances des budgets principaux et budgets annexes des régions de 2010 à
2017


\vspace{0.5cm}
\needspace{12\baselineskip}
\subsection*{Balances comptables des syndicats
}\index{balances}\index{balances!comptables}\index{comptabilite!publique}\index{intercommunalite}\index{syndicats}
  \begin{wrapfigure}{r}{2.5cm}
    \centering
    \qrcode[nolink]{https://data.gouv.fr/dataset/564b42ba88ee384b22e72046}
  \end{wrapfigure}

Licence : \textbf{Licence Ouverte
}\newline
Créé le : 2015-11-17\newline
Modifié le : 2019-02-18\newline
Granularité : à la commune\newline
Mise à jour : annuelle\newline
Popularité : 2 réutilisations,  8 suivis\newline
Mots-clé : \emph{balances, balances-comptables, comptabilite-publique, intercommunalite, syndicats
}\newline
Permalien : \url{https://data.gouv.fr/dataset/564b42ba88ee384b22e72046}\newline

\par
\noindent
    Balances des budgets principaux et budgets annexes des syndicats (SIVU,
SIVOM\ldots{}) de 2010 à 2017. Pour l'exercice 2016, les balances
comptables de établissements publics de territoire figurent dans ce
fichier. En 2017, les balances de ces établissement sont dans le fichier
des balances comptables des groupements à fiscalité propre.


\vspace{0.5cm}
\needspace{12\baselineskip}
\subsection*{Campagne viti-vinicole
}\index{dgddi}\index{douanes}\index{viti!vinicole}\index{viticole}
  \begin{wrapfigure}{r}{2.5cm}
    \centering
    \qrcode[nolink]{https://data.gouv.fr/dataset/555b3f7cc751df45e5190c79}
  \end{wrapfigure}

Licence : \textbf{Licence Ouverte
}\newline
Créé le : 2015-05-19\newline
Modifié le : 2018-04-06\newline
De 2011-08-01 à 2017-07-31\newline
Granularité : au département\newline
Mise à jour : mensuelle\newline
Popularité : 1 réutilisation,  0 suivi\newline
Mots-clé : \emph{dgddi, douanes, viti-vinicole, viticole
}\newline
Permalien : \url{https://data.gouv.fr/dataset/555b3f7cc751df45e5190c79}\newline

\par
\noindent
    Quantités de vins soumises au droit de circulation. Quantités de vins
sorties des chais des récoltants. Quantités détaillées par département.
Données mensuelles : la campagne commence le 1er août de chaque année et
se termine le 31 juillet de l'année suivante.


\vspace{0.5cm}
\needspace{12\baselineskip}
\subsection*{Chiffres de l'Aide Publique au Développement (APD) 2015
}
  \begin{wrapfigure}{r}{2.5cm}
    \centering
    \qrcode[nolink]{https://data.gouv.fr/dataset/57bfe1d088ee38768cb627a9}
  \end{wrapfigure}

Licence : \textbf{Licence Ouverte
}\newline
Créé le : 2016-08-26\newline
Modifié le : 2017-04-03\newline
De 2015-01-01 à 2015-12-31\newline
Mise à jour : annuelle\newline
Popularité : 1 réutilisation,  1 suivi\newline
Mots-clé : \emph{aucun
}\newline
Permalien : \url{https://data.gouv.fr/dataset/57bfe1d088ee38768cb627a9}\newline

\par
\noindent
    Les données préliminaires de la France pour 2015 d'aide publique au
développement (APD) au format CSV.


\vspace{0.5cm}
\needspace{12\baselineskip}
\subsection*{Fichier FANTOIR des voies et lieux-dits
}\index{cadastre}\index{commune}\index{fantoir}\index{lieux!dits}\index{voies}
  \begin{wrapfigure}{r}{2.5cm}
    \centering
    \qrcode[nolink]{https://data.gouv.fr/dataset/53699580a3a729239d204738}
  \end{wrapfigure}

Licence : \textbf{Licence Ouverte
}\newline
Créé le : 2013-11-20\newline
Modifié le : 2019-02-05\newline
Granularité : à la commune\newline
Mise à jour : trimestrielle\newline
Popularité : 11 réutilisations,  38 suivis\newline
Mots-clé : \emph{cadastre, commune, fantoir, lieux-dits, voies
}\newline
Permalien : \url{https://data.gouv.fr/dataset/53699580a3a729239d204738}\newline

\par
\noindent
    Ce fichier répertorie pour chaque commune le nom des lieux-dits et des
voies, y compris celles situées dans les lotissements et les
copropriétés.


\vspace{0.5cm}
\needspace{12\baselineskip}
\subsection*{Impôts locaux : fichier de recensement des éléments d'imposition à la
fiscalité directe locale (REI)
}
  \begin{wrapfigure}{r}{2.5cm}
    \centering
    \qrcode[nolink]{https://data.gouv.fr/dataset/58da72bbc751df38734d8658}
  \end{wrapfigure}

Licence : \textbf{Licence Ouverte
}\newline
Créé le : 2017-03-28\newline
Modifié le : 2018-11-05\newline
De 2013-01-01 à 2015-12-31\newline
Granularité : à la commune\newline
Mise à jour : annuelle\newline
Popularité : 2 réutilisations,  7 suivis\newline
Mots-clé : \emph{aucun
}\newline
Permalien : \url{https://data.gouv.fr/dataset/58da72bbc751df38734d8658}\newline

\par
\noindent
    Le fichier de recensement des éléments d'imposition à la fiscalité
directe locale (REI) est un fichier agrégé au niveau communal. Il
détaille l'ensemble des données de fiscalité directe locale par taxe et
par collectivité bénéficiaire (commune, syndicats et assimilés,
intercommunalité, département, région). Ces données concernent
exclusivement les impositions primitives, c'est-à-dire ne tiennent pas
compte des impositions supplémentaires consécutives à des omissions ou
insuffisances de l'imposition initiale. Ce fichier contient notamment
les informations relatives aux principaux impôts locaux suivants : - la
taxe foncière sur les propriétés non bâties (TFPNB) ; - la taxe foncière
sur les propriétés bâties (TFPB) ; - la taxe d'habitation (TH) ; - la
cotisation foncière des entreprises (CFE) ; - la cotisation sur la
valeur ajoutée des entreprises (CVAE) ; - la taxe spéciale d'équipement
au profit de la région d'Île-de-France et d'établissements publics (TSE)
; - la taxe d'enlèvement des ordures ménagères (TEOM) ; - les
impositions forfaitaires sur les entreprises de réseaux (Ifer) ; - la
taxe sur les surfaces commerciales (Tascom).

Il comprend aussi les informations concernant les taxes annexes au
profit des chambres d'agriculture, de la caisse d'assurance des
accidents agricoles, des chambres de commerce et d'industrie et des
chambres des métiers.


\vspace{0.5cm}
\needspace{12\baselineskip}
\subsection*{Marchés publics conclus recensés sur la plateforme des achats de l'Etat
}\index{achat!public}\index{appel!doffre}\index{attributaires}\index{consultations}\index{dae}\index{direction!achats!etat}\index{donnees!essentielles}\index{marche!public}\index{marches!conclus}\index{marches!publics}\index{marches!publics!gouv!fr}\index{place}\index{profil!acheteur}
  \begin{wrapfigure}{r}{2.5cm}
    \centering
    \qrcode[nolink]{https://data.gouv.fr/dataset/53699931a3a729239d205195}
  \end{wrapfigure}

Licence : \textbf{Licence Ouverte
}\newline
Créé le : 2013-07-08\newline
Modifié le : 2018-11-08\newline
De 2011-01-01 à 2017-12-12\newline
Granularité : au pays\newline
Mise à jour : annuelle\newline
Popularité : 2 réutilisations,  9 suivis\newline
Mots-clé : \emph{achat-public, appel-doffre, attributaires, consultations, dae, direction-achats-etat, donnees-essentielles, marche-public, marches-conclus, marches-publics, marches-publics-gouv-fr, place, profil-acheteur
}\newline
Permalien : \url{https://data.gouv.fr/dataset/53699931a3a729239d205195}\newline

\par
\noindent
    \textbf{Attributaires des marchés de l'État et de ses établissements
publics, des CCI et de l'UGAP, ayant fait l'objet d'une mise en ligne
sur la plateforme des achats de l'Etat (PLACE)(1).}

Cette liste comprend le nom de l'entité publique, l'entité d'achat, le
nom de l'attributaire, la date de notification, le code postal, la
tranche budgétaire en euro HT, la nature de l'achat, l'objet du marché,
le montant du marché.

\emph{(1) Cette liste répond à l'obligation, pour les acheteurs, de
publier les informations relatives aux marchés publics conclus chaque
année (article 133 du code des marchés publics). Chaque ministère ou
pouvoir adjudicateur est cependant responsable de la publication de ses
données, qui peut être faite par différents canaux. La liste publiée ici
n'est donc pas exhaustive, elle ne comprend que les informations
relatives aux marchés conclus que les acheteurs ont choisi de publier
via la PLACE}


\vspace{0.5cm}
\needspace{12\baselineskip}
\subsection*{Nombre de productions audiovisuelles accueillies dans les lieux publics
}\index{sites!publics}\index{tournages}
  \begin{wrapfigure}{r}{2.5cm}
    \centering
    \qrcode[nolink]{https://data.gouv.fr/dataset/53699aa8a3a729239d205544}
  \end{wrapfigure}

Licence : \textbf{Licence Ouverte
}\newline
Créé le : 2014-02-26\newline
Modifié le : 2014-11-05\newline
De 2010-01-01 à 2012-12-31\newline
Mise à jour : annuelle\newline
Popularité : 1 réutilisation,  0 suivi\newline
Mots-clé : \emph{sites-publics, tournages
}\newline
Permalien : \url{https://data.gouv.fr/dataset/53699aa8a3a729239d205544}\newline

\par
\noindent
    Nombre de productions audiovisuelles (cinéma, documentaires,
téléfilms\ldots{}) dans les principaux lieux publics, par secteur
(justice, santé, environnement\ldots{}), référencés par l'Agence du
patrimoine immatériel de l'État (APIE). Ce recensement ne se veut donc
pas exhaustif.

\href{http://www.economie.gouv.fr/apie/espaces-publics}{La démarche
d'ouverture des lieux publics aux tournages} (sur le site de l'APIE)


\vspace{0.5cm}
\needspace{12\baselineskip}
\subsection*{Plan Cadastral Informatisé
}\index{batiments}\index{cadastre}\index{parcelles}
  \begin{wrapfigure}{r}{2.5cm}
    \centering
    \qrcode[nolink]{https://data.gouv.fr/dataset/58e5924b88ee3802ca255566}
  \end{wrapfigure}

Licence : \textbf{Licence Ouverte
}\newline
Créé le : 2017-04-06\newline
Modifié le : 2018-10-31\newline
Mise à jour : trimestrielle\newline
Popularité : 3 réutilisations,  54 suivis\newline
Mots-clé : \emph{batiments, cadastre, parcelles
}\newline
Permalien : \url{https://data.gouv.fr/dataset/58e5924b88ee3802ca255566}\newline

\par
\noindent
    Le plan cadastral est un assemblage d'environ \textbf{600 000 feuilles}
ou planches représentant chacune une section ou une partie d'une section
cadastrale.

Il couvre la France entière, à l'exception de la ville de Strasbourg et
de quelques communes voisines, pour des raisons historiques liée à
l'occupation de l'Alsace-Moselle par l'Allemagne entre 1871 et 1918. PCI
Vecteur et PCI Image

Pour des questions pratiques et techniques, le Plan Cadastral
Informatisé existe sous la forme de \textbf{deux produits
complémentaires} : le PCI Vecteur et le PCI Image.

Le \textbf{PCI Vecteur} regroupe les feuilles qui ont été numérisées et
couvre l'essentiel du territoire.

Le \textbf{PCI Image} regroupe les feuilles qui n'ont été que scannées,
et complète la couverture. Couverture

\textbf{33 682 communes} sont couvertes par le PCI Vecteur, sur un total
de près de 35 400. Les plans des autres communes sont disponibles dans
le PCI Image.

Strasbourg et les communes limitrophes ne sont actuellement pas gérées
au format PCI.

Les collectivités d'outre-mer de Saint-Martin et de Saint-Barthelemy
sont présentes et historiquement intégrées dans le département de la
Guadeloupe (971). Formats disponibles

Les données du PCI Vecteur sont disponibles dans plusieurs formats :

\begin{itemize}

\item
  Format
  \href{https://www.data.gouv.fr/s/resources/plan-cadastral-informatise/20170906-150737/standard_edigeo_2013.pdf}{EDIGÉO}
  en projection Lambert 93 ;
\item
  Format
  \href{https://www.data.gouv.fr/s/resources/plan-cadastral-informatise/20170906-150737/standard_edigeo_2013.pdf}{EDIGÉO}
  en projection Lambert CC 9 zones ;
\item
  Format DXF-PCI en projection Lambert 93 ;
\item
  Format DXF-PCI en projection Lambert CC 9 zones.
\end{itemize}

Les données du PCI Image sont disponibles au format TIFF. Modèle de
données

Chaque commune est subdivisée en sections, elles-mêmes subdivisées en
feuilles (ou planches). Une feuille cadastrale comporte des parcelles,
qui peuvent supporter des bâtiments, ainsi que de nombreux autres objets
d'habillage ou de gestion. Pour plus de précision, veuillez vous
reporter à la documentation du standard
\href{https://www.data.gouv.fr/s/resources/plan-cadastral-informatise/20170906-150737/standard_edigeo_2013.pdf}{EDIGÉO}.
Mise à disposition

Les données sont mises à disposition de deux manières :

\begin{itemize}

\item
  En téléchargement direct à la feuille ou en archive départementale. Ce
  sont ces URL qu'il faut utiliser si vous souhaitez automatiser la
  récupération des données et bénéficier des meilleures performances.
\item
  Via un outil en ligne pour les archives communales. Les données sont
  alors produites à la volée.
\end{itemize}

Les deux modes de mise à disposition sont accessibles sur
\href{https://cadastre.data.gouv.fr/datasets/plan-cadastral-informatise}{cadastre.data.gouv.fr}.
Millésimes disponibles :

\begin{itemize}
\item
  13 février 2017
\item
  14 mai 2017
\item
  6 juillet 2017
\item
  12 octobre 2017
\item
  2 janvier 2018
\item
  3 avril 2018
\item
  29 juin 2018 (plus récent) Voir aussi
\item
  \href{https://cadastre.data.gouv.fr/datasets/cadastre-etalab}{Données
  cadastrales retravaillées par Etalab (formats GeoJSON et Shapefile)}
\item
  \href{https://cadastre.data.gouv.fr/datasets/cadastre-strasbourg}{Données
  cadastrales Eurométropole de Strasbourg}
\end{itemize}


\vspace{0.5cm}
\needspace{12\baselineskip}
\subsection*{PLF2012 Budget général par mission
}
  \begin{wrapfigure}{r}{2.5cm}
    \centering
    \qrcode[nolink]{https://data.gouv.fr/dataset/53699cc9a3a729239d205a6d}
  \end{wrapfigure}

Licence : \textbf{Licence Ouverte
}\newline
Créé le : 2013-07-08\newline
Modifié le : 2015-05-26\newline
Mise à jour : annuelle\newline
Popularité : 1 réutilisation,  0 suivi\newline
Mots-clé : \emph{aucun
}\newline
Permalien : \url{https://data.gouv.fr/dataset/53699cc9a3a729239d205a6d}\newline

\par
\noindent
    Présentation des crédits (AE CP) du budget général du PLF 2012 par
Mission et Titre 2 / hors Titre 2


\vspace{0.5cm}
\needspace{12\baselineskip}
\subsection*{PLF2012 Budget général par Programme
}
  \begin{wrapfigure}{r}{2.5cm}
    \centering
    \qrcode[nolink]{https://data.gouv.fr/dataset/53699cc9a3a729239d205a6e}
  \end{wrapfigure}

Licence : \textbf{Licence Ouverte
}\newline
Créé le : 2013-07-08\newline
Modifié le : 2015-09-11\newline
Mise à jour : annuelle\newline
Popularité : 1 réutilisation,  0 suivi\newline
Mots-clé : \emph{aucun
}\newline
Permalien : \url{https://data.gouv.fr/dataset/53699cc9a3a729239d205a6e}\newline

\par
\noindent
    Présentation des crédits (AE CP) du budget général du PLF 2012 par
Programme et Titre 2 / hors Titre 2


\vspace{0.5cm}
\needspace{12\baselineskip}
\subsection*{PLF 2013 - Budget général par Mission
}
  \begin{wrapfigure}{r}{2.5cm}
    \centering
    \qrcode[nolink]{https://data.gouv.fr/dataset/53699cd4a3a729239d205a8b}
  \end{wrapfigure}

Licence : \textbf{Licence Ouverte
}\newline
Créé le : 2013-07-08\newline
Modifié le : 2016-02-27\newline
De 2013-01-01 à 2013-12-31\newline
Mise à jour : annuelle\newline
Popularité : 2 réutilisations,  2 suivis\newline
Mots-clé : \emph{aucun
}\newline
Permalien : \url{https://data.gouv.fr/dataset/53699cd4a3a729239d205a8b}\newline

\par
\noindent
    Présentation des crédits (AE CP) du budget général du PLF 2013 par
mission et Titre 2 / hors Titre 2


\vspace{0.5cm}
\needspace{12\baselineskip}
\subsection*{PLF2013-Jaune-Données Associations subventionnées
}
  \begin{wrapfigure}{r}{2.5cm}
    \centering
    \qrcode[nolink]{https://data.gouv.fr/dataset/53699cdaa3a729239d205a9a}
  \end{wrapfigure}

Licence : \textbf{Licence Ouverte
}\newline
Créé le : 2013-07-08\newline
Modifié le : 2016-01-22\newline
De 2011-01-01 à 2011-12-31\newline
Mise à jour : annuelle\newline
Popularité : 1 réutilisation,  3 suivis\newline
Mots-clé : \emph{aucun
}\newline
Permalien : \url{https://data.gouv.fr/dataset/53699cdaa3a729239d205a9a}\newline

\par
\noindent
    liste des associations subventionnées par l'Etat avec le montant des
subventions par ministère et programme pour 2011


\vspace{0.5cm}
\needspace{12\baselineskip}
\subsection*{PLF 2014 - Budget général (BG) par destination
}
  \begin{wrapfigure}{r}{2.5cm}
    \centering
    \qrcode[nolink]{https://data.gouv.fr/dataset/53699cdba3a729239d205a9e}
  \end{wrapfigure}

Licence : \textbf{Licence Ouverte
}\newline
Créé le : 2013-10-13\newline
Modifié le : 2015-09-28\newline
De 2014-01-01 à 2014-12-31\newline
Mise à jour : annuelle\newline
Popularité : 1 réutilisation,  0 suivi\newline
Mots-clé : \emph{aucun
}\newline
Permalien : \url{https://data.gouv.fr/dataset/53699cdba3a729239d205a9e}\newline

\par
\noindent
    Présentation des crédits (AE CP) du Budget général (BG) du PLF 2014
suivant la nomenclature par destination


\vspace{0.5cm}
\needspace{12\baselineskip}
\subsection*{PLF 2014 - Budget général (BG) par Mission et Titre
}
  \begin{wrapfigure}{r}{2.5cm}
    \centering
    \qrcode[nolink]{https://data.gouv.fr/dataset/53699cdda3a729239d205aa8}
  \end{wrapfigure}

Licence : \textbf{Licence Ouverte
}\newline
Créé le : 2013-10-13\newline
Modifié le : 2015-09-04\newline
De 2014-01-01 à 2014-12-31\newline
Mise à jour : annuelle\newline
Popularité : 1 réutilisation,  0 suivi\newline
Mots-clé : \emph{aucun
}\newline
Permalien : \url{https://data.gouv.fr/dataset/53699cdda3a729239d205aa8}\newline

\par
\noindent
    Présentation des crédits (AE CP) du budget général du PLF 2014 par
mission / programme et Titre 2 / hors Titre 2


\vspace{0.5cm}
\needspace{12\baselineskip}
\subsection*{PLF - Jaune - Associations subventionnées
}\index{association}\index{associations}\index{subvention}\index{subventions}
  \begin{wrapfigure}{r}{2.5cm}
    \centering
    \qrcode[nolink]{https://data.gouv.fr/dataset/53699ce7a3a729239d205ac3}
  \end{wrapfigure}

Licence : \textbf{Licence Ouverte
}\newline
Créé le : 2013-12-03\newline
Modifié le : 2016-03-01\newline
De 2010-01-01 à 2012-12-31\newline
Granularité : au pays\newline
Mise à jour : annuelle\newline
Popularité : 3 réutilisations,  11 suivis\newline
Mots-clé : \emph{association, associations, subvention, subventions
}\newline
Permalien : \url{https://data.gouv.fr/dataset/53699ce7a3a729239d205ac3}\newline

\par
\noindent
    Liste des associations subventionnées par l'Etat avec le montant des
subventions par ministère et programme.


\vspace{0.5cm}
\needspace{12\baselineskip}
\subsection*{PLR 2017 - projet de loi de règlement pour l'année 2017 - Données de
l'exécution budgétaire
}\index{budget!2017}\index{etat}\index{execution}\index{finances!publiques}
  \begin{wrapfigure}{r}{2.5cm}
    \centering
    \qrcode[nolink]{https://data.gouv.fr/dataset/5b116de2c751df6f660cccab}
  \end{wrapfigure}

Licence : \textbf{Licence Ouverte
}\newline
Créé le : 2018-06-01\newline
Modifié le : 2018-06-01\newline
De 2017-01-01 à 2017-12-31\newline
Mise à jour : annuelle\newline
Popularité : 1 réutilisation,  0 suivi\newline
Mots-clé : \emph{budget-2017, etat, execution, finances-publiques
}\newline
Permalien : \url{https://data.gouv.fr/dataset/5b116de2c751df6f660cccab}\newline

\par
\noindent
    Données de l'exécution telles que publiées dans les rapports annuels de
performance annexés au projet de loi de règlement pour 2017 déposée au
Parlement en mai 2017 comprenant :

\begin{itemize}

\item
  Exécution par mission, programme et titre en autorisation d'engagement
  (AE) pour le budget général (BG), les comptes d'affection spéciale
  (CAS) et les comptes de concours financiers (CCF) ;
\item
  Exécution par mission, programme et titre en crédit de paiement (CP)
  pour le budget général (BG), les comptes d'affection spéciale (CAS) et
  les comptes de concours financiers (CCF) ;
\item
  Exécution par ministère, programme et titre en autorisation
  d'engagement (AE) pour le budget général (BG), les comptes d'affection
  spéciale (CAS) et les comptes de concours financiers (CCF) ;
\item
  Exécution par ministère, programme et titre en crédit de paiement (CP)
  pour le budget général (BG), les comptes d'affection spéciale (CAS) et
  les comptes de concours financiers (CCF) ;
\item
  Exécution par mission et catégorie en AE et CP pour le budget général
  (BG), les comptes d'affection spéciale (CAS) et les comptes de
  concours financiers (CCF).
\item
  Exécution par ministère Titre 2 / hors Titre 2 en AE, CP et ETPT pour
  le budget général et les budgets annexes.
\end{itemize}


\vspace{0.5cm}
\needspace{12\baselineskip}
\subsection*{Projet de loi de finances pour 2017 (PLF 2017)
}\index{etat}\index{finances!publiques}
  \begin{wrapfigure}{r}{2.5cm}
    \centering
    \qrcode[nolink]{https://data.gouv.fr/dataset/5807832bc751df174d79df72}
  \end{wrapfigure}

Licence : \textbf{Licence Ouverte
}\newline
Créé le : 2016-10-19\newline
Modifié le : 2016-10-19\newline
Mise à jour : annuelle\newline
Popularité : 1 réutilisation,  0 suivi\newline
Mots-clé : \emph{etat, finances-publiques
}\newline
Permalien : \url{https://data.gouv.fr/dataset/5807832bc751df174d79df72}\newline

\par
\noindent
    Données issues du projet de loi de finances (PLF) pour 2017 et de ses
annexes, à savoir :

\begin{itemize}

\item
  Nomenclature par destination : Mission / Programme / Action (MPA)
\item
  Recettes fiscales nettes
\item
  Budget général (BG) par mission titre 2 / hors titre 2 (T2 / HT2)
\item
  Budgets annexes (BA) par mission titre 2 / hors titre 2 (T2 / HT2)
\item
  Comptes spéciaux et comptes de commerce (CS) par mission titre 2 /
  hors titre 2 (T2 / HT2)
\item
  Budget général (BG) par ministère titre 2 / hors titre 2 (T2 / HT2)
\item
  Budgets annexes (BA) par ministère titre 2 / hors titre 2 (T2 / HT2)
\item
  Comptes spéciaux et comptes de commerce (CS) par ministère titre 2 /
  hors titre 2 (T2 / HT2)
\item
  Budget général (BG) par mission sur l'axe destination
\item
  Budgets annexes (BA) par mission sur l'axe destination
\item
  Comptes spéciaux et comptes de commerce (CS) par mission sur l'axe
  destination
\item
  Budget général (BG) par mission sur l'axe nature
\item
  Comptes spéciaux et comptes de commerce (CS) par mission sur l'axe
  nature
\item
  Budget général (BG) décompte des emplois équivalent temps plein (ETP)
  par ministère
\item
  Budgets annexes (BA) décompte des emplois équivalent temps plein (ETP)
  par ministère
\end{itemize}


\vspace{0.5cm}
\needspace{12\baselineskip}
\subsection*{Projet de loi de finances pour 2018 (PLF 2018) - Données du PLF et des
annexes Projet Annuel de Performance (PAP)
}\index{budget!2018}\index{etat}\index{finances!publiques}\index{loi!de!finances}
  \begin{wrapfigure}{r}{2.5cm}
    \centering
    \qrcode[nolink]{https://data.gouv.fr/dataset/5a0581aec751df1c68fee715}
  \end{wrapfigure}

Licence : \textbf{Licence Ouverte
}\newline
Créé le : 2017-11-10\newline
Modifié le : 2017-11-10\newline
De 2018-01-01 à 2019-12-31\newline
Mise à jour : annuelle\newline
Popularité : 1 réutilisation,  0 suivi\newline
Mots-clé : \emph{budget-2018, etat, finances-publiques, loi-de-finances
}\newline
Permalien : \url{https://data.gouv.fr/dataset/5a0581aec751df1c68fee715}\newline

\par
\noindent
    Données issues du projet de loi de finances (PLF) pour 2018 et de ses
annexes, à savoir :

\begin{itemize}

\item
  Nomenclature par destination : Mission / Programme / Action (MPA)
\item
  Recettes fiscales nettes
\item
  Budget général (BG) par mission titre 2 / hors titre 2 (T2 / HT2)
\item
  Budgets annexes (BA) par mission titre 2 / hors titre 2 (T2 / HT2)
\item
  Comptes spéciaux et comptes de commerce (CS) par mission titre 2 /
  hors titre 2 (T2 / HT2)
\item
  Budget général (BG) par ministère titre 2 / hors titre 2 (T2 / HT2)
\item
  Budgets annexes (BA) par ministère titre 2 / hors titre 2 (T2 / HT2)
\item
  Comptes spéciaux et comptes de commerce (CS) par ministère titre 2 /
  hors titre 2 (T2 / HT2)
\item
  Budget général (BG) par mission sur l'axe destination
\item
  Budgets annexes (BA) par mission sur l'axe destination
\item
  Comptes spéciaux et comptes de commerce (CS) par mission sur l'axe
  destination
\item
  Budget général (BG) par mission sur l'axe nature
\item
  Comptes spéciaux et comptes de commerce (CS) par mission sur l'axe
  nature
\item
  Budget général (BG) décompte des emplois équivalent temps plein (ETP)
  par ministère
\item
  Budgets annexes (BA) décompte des emplois équivalent temps plein (ETP)
  par ministère
\end{itemize}


\vspace{0.5cm}
\needspace{12\baselineskip}
\subsection*{Statistiques nationales du commerce extérieur
}\index{commerce!exterieur}\index{commerce!international}\index{datadouane}\index{dgddi}\index{douane}\index{echanges!commerciaux}\index{exportation}\index{import!export}\index{importation}
  \begin{wrapfigure}{r}{2.5cm}
    \centering
    \qrcode[nolink]{https://data.gouv.fr/dataset/5369a0b4a3a729239d206418}
  \end{wrapfigure}

Licence : \textbf{Licence Ouverte
}\newline
Créé le : 2014-04-25\newline
Modifié le : 2019-03-08\newline
De 2012-01-01 à 2019-01-31\newline
Granularité : au pays\newline
Mise à jour : mensuelle\newline
Popularité : 5 réutilisations,  10 suivis\newline
Mots-clé : \emph{commerce-exterieur, commerce-international, datadouane, dgddi, douane, echanges-commerciaux, exportation, import-export, importation
}\newline
Permalien : \url{https://data.gouv.fr/dataset/5369a0b4a3a729239d206418}\newline

\par
\noindent
    DOUANE - Statistiques nationales du commerce extérieur sur les 13
derniers mois et données annuelles consolidées (Données Produits/Pays).

L'information sur les échanges de marchandises est collectée sur la base
de déclarations d'échanges de biens (DEB) pour les échanges avec les 27
autres États membres et des déclarations en douane (DAU) pour les
échanges avec les autres pays (nommés pays tiers). Chaque mois, la
collecte statistique porte sur les échanges du mois de référence (mois
de publication) et sur des corrections et enrichissements relatifs aux
mois antérieurs.

\textbf{Source DataDouane} :
\href{http://www.douane.gouv.fr/services/datadouane}{www.douane.gouv.fr/services/datadouane}


\vspace{0.5cm}
\needspace{12\baselineskip}
\subsection*{Statistiques régionales et départementales du commerce extérieur
}\index{commerce!exterieur}\index{commerce!international}\index{departements}\index{dgddi}\index{douane}\index{echanges!commerciaux}\index{exportation}\index{import!export}\index{importation}\index{regions}
  \begin{wrapfigure}{r}{2.5cm}
    \centering
    \qrcode[nolink]{https://data.gouv.fr/dataset/5369a0c7a3a729239d206446}
  \end{wrapfigure}

Licence : \textbf{Licence Ouverte
}\newline
Créé le : 2014-04-25\newline
Modifié le : 2019-03-08\newline
De 2013-01-01 à 2018-12-31\newline
Granularité : au département\newline
Mise à jour : mensuelle\newline
Popularité : 3 réutilisations,  4 suivis\newline
Mots-clé : \emph{commerce-exterieur, commerce-international, departements, dgddi, douane, echanges-commerciaux, exportation, import-export, importation, regions
}\newline
Permalien : \url{https://data.gouv.fr/dataset/5369a0c7a3a729239d206446}\newline

\par
\noindent
    1- Statistiques régionales et départementales du commerce extérieur sur
les 5 derniers trimestres (Données Produits/Pays), mis à jour chaque
mois. 2- Statistiques régionales et départementales du commerce
extérieur annuelles (Données Produits/Pays), mis à jour chaque mois.

L'information sur les échanges de marchandises est collectée sur la base
de déclarations d'échanges de biens (DEB) pour les échanges avec les 27
autres États membres et des déclarations en douane (DAU) pour les
échanges avec les autres pays (nommés pays tiers). Chaque mois, la
collecte statistique porte sur les échanges du mois de référence (mois
de publication) et sur des corrections et enrichissements relatifs aux
mois antérieurs. Les produits sont détaillés selon la Classification de
produit française (CPF4).

\textbf{Source DataDouane} :
\href{http://www.douane.gouv.fr/services/datadouane}{www.douane.gouv.fr/services/datadouane}


\vspace{0.5cm}
\needspace{12\baselineskip}
\subsection*{Ventes de tabacs manufacturés en France métropolitaine
}\index{dgddi}\index{douane}\index{tabac}\index{ventes}
  \begin{wrapfigure}{r}{2.5cm}
    \centering
    \qrcode[nolink]{https://data.gouv.fr/dataset/5c5c21248b4c4114cce91bf8}
  \end{wrapfigure}

Licence : \textbf{Licence Ouverte
}\newline
Créé le : 2019-02-07\newline
Modifié le : 2019-02-07\newline
De 2018-01-01 à 2019-01-31\newline
Mise à jour : mensuelle\newline
Popularité : 1 réutilisation,  0 suivi\newline
Mots-clé : \emph{dgddi, douane, tabac, ventes
}\newline
Permalien : \url{https://data.gouv.fr/dataset/5c5c21248b4c4114cce91bf8}\newline

\par
\noindent
    Données en volume des livraisons mensuelles au réseau des buralistes.


\vspace{0.5cm}
\needspace{3\baselineskip} \rule{4cm}{0.25pt}\newline\textbf{Aussi disponible du même producteur :}\begin{itemize}
\item \href{https://data.gouv.fr/dataset/555b401bc751df4633190c7b}{Adresses des services douaniers ouverts au public}
\item \href{https://data.gouv.fr/dataset/5c6a748d634f413469486d99}{Agrégats comptables des collectivités et des établissements publics locaux}
\item \href{https://data.gouv.fr/dataset/53698f1ea3a729239d20365e}{Atlas fiscal}
\item \href{https://data.gouv.fr/dataset/5bd1bc078b4c41671cefdc4a}{Balances comptables des collectivités et des établissements publics locaux avec la présentation croisée nature-fonction }
\item \href{https://data.gouv.fr/dataset/555b6f18c751df4cc7190c79}{Balances comptables des établissements publics locaux }
\item \href{https://data.gouv.fr/dataset/5bbf48df8b4c41310ef22ca3}{Base documentaire DGFiP}
\item \href{https://data.gouv.fr/dataset/5bb332a58b4c412e6f4179b1}{Base documentaire  DGFIP }
\item \href{https://data.gouv.fr/dataset/53698fe9a3a729239d203880}{CAD 2a: Destination de l'aide publique au développement de la France}
\item \href{https://data.gouv.fr/dataset/53698feba3a729239d203886}{CAD 5: Engagement (ou versements bruts) bilatéraux du secteur public français}
\item \href{https://data.gouv.fr/dataset/53699042a3a729239d20396d}{Catalogue des sites publics ouverts aux evenements}
\item \href{https://data.gouv.fr/dataset/5b30b3e788ee384c1aacbd66}{Cessions immobilières de l’Etat}
\item \href{https://data.gouv.fr/dataset/57bfe2e188ee386dd1b627a9}{Chiffres de l'épargne logement 2016}
\item \href{https://data.gouv.fr/dataset/5ba367a98b4c4105786813f2}{Code source de la taxe d'habitation}
\item \href{https://data.gouv.fr/dataset/5369914ca3a729239d203c0f}{Comptes Combinés 2010}
\item \href{https://data.gouv.fr/dataset/5a5f55a488ee380b7396245a}{Comptes individuels des collectivités}
\item \href{https://data.gouv.fr/dataset/5c66b5fb8b4c411bff4d5a0a}{Comptes individuels des départements et des collectivités territoriales uniques (fichier global)}
\item \href{https://data.gouv.fr/dataset/5c66bf598b4c4134b51758d2}{Comptes individuels des groupements à fiscalité propre (fichier global)}
\item \href{https://data.gouv.fr/dataset/5c669acd8b4c414c13aee04c}{Comptes individuels des régions (fichier global)}
\item \href{https://data.gouv.fr/dataset/5bc704478b4c41621c8575f6}{Déclaration préliminaire d’aide publique au développement pour les flux 2016}
\item \href{https://data.gouv.fr/dataset/5436a05b88ee3842753199b2}{Déclaration préliminaire de l'aide publique au développement (APD)}
\item \href{https://data.gouv.fr/dataset/5ad0afe588ee3861b20f7c0d}{Déclarations nationales de résultats des impôts professionnels BIC/IS, BNC et BA}
\item \href{https://data.gouv.fr/dataset/536992e4a3a729239d204042}{DGDDI - Plafond autorisé d'emploi et Effectifs de référence}
\item \href{https://data.gouv.fr/dataset/5435040488ee384495ba99a6}{DGDDI - Statistiques de consultation des supports numériques de la douane}
\item \href{https://data.gouv.fr/dataset/5a576fda88ee381c9d4e3304}{Documents de filiation informatisés (DFI) des parcelles}
\item \href{https://data.gouv.fr/dataset/55509ec7c751df6a8a190c78}{Données 2014 de l'APD}
\item \href{https://data.gouv.fr/dataset/573db5ef88ee384011d1b934}{Données de consommation d'électricité des ministères}
\item \href{https://data.gouv.fr/dataset/543bf17f88ee3805983c164d}{Données d' exécution budgétaire des collectivités territoriales}
\item \href{https://data.gouv.fr/dataset/55d45bcfc751df5d1f1f92b3}{Effectifs DGCCRF}
\item \href{https://data.gouv.fr/dataset/53699378a3a729239d2041d5}{Effort financier de l'Etat en faveur des PME}
\item \href{https://data.gouv.fr/dataset/5369941da3a729239d204379}{Encours des créances de la France sur les Etats étrangers au 31 décembre 2010}
\item \href{https://data.gouv.fr/dataset/5bfd04788b4c4160bf8a104e}{Encours des créances de la France sur les États étrangers au 31 décembre 2017}
\item \href{https://data.gouv.fr/dataset/53699444a3a729239d2043dc}{Enquête sur les dépenses prévisionnelles}
\item \href{https://data.gouv.fr/dataset/5bb7943e634f41260b2a67d4}{enregistrement des exploitants en alimentation animale}
\item \href{https://data.gouv.fr/dataset/536994a6a3a729239d2044cf}{EPARGNE LOGEMENT Ventilation des prêts épargne-logement accordés suivant leur objet (en euros) }
\item \href{https://data.gouv.fr/dataset/5369953ea3a729239d20467e}{Evolution générale de l'épargne logement}
\item \href{https://data.gouv.fr/dataset/53699541a3a729239d204685}{Exécution 2010 des comptes de concours financiers en AE}
\item \href{https://data.gouv.fr/dataset/53699541a3a729239d204686}{Exécution 2010 des comptes de concours financiers en CP}
\item \href{https://data.gouv.fr/dataset/53699542a3a729239d204687}{Exécution 2010 du budget des comptes d'affectation spéciale en AE}
\item \href{https://data.gouv.fr/dataset/53699542a3a729239d204688}{Exécution 2010 du budget des comptes d'affectation spéciale en CP}
\item \href{https://data.gouv.fr/dataset/53699543a3a729239d204689}{Exécution 2010 du budget général en AE}
\item \href{https://data.gouv.fr/dataset/53699543a3a729239d20468a}{Exécution 2010 du budget général en CP}
\item \href{https://data.gouv.fr/dataset/53699543a3a729239d20468b}{Exécution 2011 des comptes de concours financiers en AE}
\item \href{https://data.gouv.fr/dataset/53699544a3a729239d20468c}{Exécution 2011 des comptes de concours financiers en CP}
\item \href{https://data.gouv.fr/dataset/53699544a3a729239d20468d}{Exécution 2011 du budget des comptes d'affectation spéciale en AE}
\item \href{https://data.gouv.fr/dataset/53699545a3a729239d20468e}{Exécution 2011 du budget des comptes d'affectation spéciale en CP}
\item \href{https://data.gouv.fr/dataset/53699545a3a729239d20468f}{Exécution 2011 du budget général en AE}
\item \href{https://data.gouv.fr/dataset/53699545a3a729239d204690}{Exécution 2011 du budget général en CP}
\item \href{https://data.gouv.fr/dataset/53699546a3a729239d204691}{Exécution 2012 du budget de l'Etat en AE et CP suivant la nomenclature Mission / Programme / Action et Catégorie}
\item \href{https://data.gouv.fr/dataset/53699546a3a729239d204692}{Exécution 2012 du budget de l'Etat en AE suivant la nomenclature Ministère / Programme / Action et Titre}
\item \href{https://data.gouv.fr/dataset/53699546a3a729239d204693}{Exécution 2012 du budget de l'Etat en AE suivant la nomenclature Mission / Programme / Action et Titre}
\item et 223 autres jeux de données\end{itemize}

\clearpage
\section{Ministère de l'Enseignement supérieur, de la Recherche et de l'Innovation}


\begin{center}
  \includegraphics[width=3cm]{images/orga/00_933e61180545a6a4c5afa4f6077183-100.jpg}
\end{center}


Le \href{http://www.enseignementsup-recherche.gouv.fr}{Ministère de
l'Enseignement supérieur, de la Recherche et de l'Innovation} prépare et
met en œuvre la politique du gouvernement relative au développement de
l'enseignement supérieur, de la recherche et de la technologie. Il
définit et suit la politique en matière d'innovation. Il participe à la
promotion des sciences et des technologies et à la diffusion de la
culture scientifique, technologique et industrielle, ainsi qu'à
l'élaboration et à la mise en œuvre de la politique du gouvernement en
faveur du développement et de la diffusion des usages du numérique dans
la société et l'économie.

Les données publiques en matière d'enseignement supérieur, de recherche
et de l'innovation sont accessibles sur une plateforme dédiée
(\url{https://data.enseignementsup-recherche.gouv.fr}){]}(https://data.enseignementsup-recherche.gouv.fr{]}(https://data.enseignementsup-recherche.gouv.fr)))synchronisée
avec la plateforme www.data.gouv.fr. le Ministère propose également un
moteur de recherche des différentes ressources en données (applications,
tableaux de bord, API, publications) sur ses domaines :
\href{https://data.esr.gouv.fr/FR/}{\#dataESR}.

Cette initiative s'inscrit dans le cadre de l'engagement du ministère de
de l'Enseignement supérieur, de la Recherche et de l'innovation en
faveur de l'open data et de sa stratégie numérique pour l'enseignement
supérieur et la recherche.


\vspace{0.5cm}

\needspace{12\baselineskip}
\subsection*{APB Voeux de poursuite d'étude et admissions
}\index{administration}\index{admis}\index{admissibles}\index{apb}\index{boursier}\index{boursiers}\index{bts}\index{capacite}\index{citoyennete}\index{cpge}\index{cpge!integree!a!luniversite}\index{dcg}\index{demande}\index{deust}\index{dut}\index{ecole!de!commerce}\index{education}\index{endo!recrutement}\index{enseignement}\index{enseignement!superieur}\index{etablissement}\index{etablissements}\index{etudiants}\index{finances!publiques}\index{formation}\index{formations}\index{gouvernement}\index{licence}\index{offre}\index{orientation}\index{paces}\index{parcoursup}\index{places}\index{proposition}\index{propositions}\index{rang}\index{recherche}\index{sts}\index{uai}\index{voeu}\index{voeux}
  \begin{wrapfigure}{r}{2.5cm}
    \centering
    \qrcode[nolink]{https://data.gouv.fr/dataset/5bc9493c06e3e7724c8c0eb9}
  \end{wrapfigure}

Licence : \textbf{Licence Ouverte version 2.0
}\newline
Créé le : 2018-10-19\newline
Modifié le : 2019-03-17\newline
Popularité : 1 réutilisation,  0 suivi\newline
Mots-clé : \emph{administration, admis, admissibles, apb, boursier, boursiers, bts, capacite, citoyennete, cpge, cpge-integree-a-luniversite, dcg, demande, deust, dut, ecole-de-commerce, education, endo-recrutement, enseignement, enseignement-superieur, etablissement, etablissements, etudiants, finances-publiques, formation, formations, gouvernement, licence, offre, orientation, paces, parcoursup, places, proposition, propositions, rang, recherche, sts, uai, voeu, voeux
}\newline
Permalien : \url{https://data.gouv.fr/dataset/5bc9493c06e3e7724c8c0eb9}\newline

\par
\noindent
    Ce jeu de données présente les vœux de poursuite d'études et de
réorientation dans l'enseignement supérieur ainsi que les propositions
des établissements pour chaque formation à la fin du processus
d'affectation de la plateforme APB pour les session 2016 et~2017 (du 20
janvier au 30 septembre).

Ces données, par établissement de l'enseignement supérieur, sont tirées
de la plateforme Admission Post Bac (APB) et sont figées à partir de la
date du 30 septembre, date correspondant à la fin du processus
d'affectation d'APB.

\href{https://data.enseignementsup-recherche.gouv.fr/explore/dataset/fr-esr-apb_voeux-et-admissions/data/}{\textbf{Une
datavisualisation~vous permet d'explorer facilement ces données.}} **\\
**

\textbf{Pour en savoir plus sur les données du jeu de données, consultez
la\href{https://data.enseignementsup-recherche.gouv.fr/api/datasets/1.0/fr-esr-apb_voeux-et-admissions/attachments/methodo_opendata_docx/}{documentation
associée}.}

\textbf{Vous pouvez également consulter les publications statistiques
sur\href{https://publication.enseignementsup-recherche.gouv.fr/FR/recherche/apb/}{APB~}et~\href{https://publication.enseignementsup-recherche.gouv.fr/FR/recherche/parcoursup/}{parcoursup}}
\textbf{.}


\vspace{0.5cm}
\needspace{12\baselineskip}
\subsection*{Appels à projets ANR - Projets retenus et participants identifiés
}\index{agence!nationale!de!la!recherche}\index{anr}\index{appels!a!projet}\index{grid}\index{laboratoire}\index{laboratoires}\index{recherche}\index{rnsr}\index{siren}
  \begin{wrapfigure}{r}{2.5cm}
    \centering
    \qrcode[nolink]{https://data.gouv.fr/dataset/577cc4e3c751df47c09901a0}
  \end{wrapfigure}

Licence : \textbf{Licence Ouverte
}\newline
Créé le : 2016-07-06\newline
Modifié le : 2016-07-06\newline
Granularité : au point d'intérêt\newline
Mise à jour : ponctuelle\newline
Popularité : 1 réutilisation,  0 suivi\newline
Mots-clé : \emph{agence-nationale-de-la-recherche, anr, appels-a-projet, grid, laboratoire, laboratoires, recherche, rnsr, siren
}\newline
Permalien : \url{https://data.gouv.fr/dataset/577cc4e3c751df47c09901a0}\newline

\par
\noindent
    Ce jeu de données présente les projets et participants des appels de
l'agence nationale de la recherche (ANR). Les participants français sont
identifiés par leur numéro SIREN ou leur numéro RNSR selon leur nature
et les participants étrangers sont identifiés par leur numéro GRID.


\vspace{0.5cm}
\needspace{12\baselineskip}
\subsection*{Bibliothèques de l'Enseignement supérieur
}\index{administration}\index{annuaire!des!bibliotheques!de!lenseignement!superieur}\index{bibliotheque}\index{bibliotheques}\index{citoyennete}\index{coordonnees!geographiques}\index{culture}\index{education}\index{enseignement}\index{esgbu}\index{finances!publiques}\index{formation}\index{geolocalisation}\index{gouvernement}\index{patrimoine}\index{recherche}\index{reseaux!sociaux}\index{sites!internet}\index{url}
  \begin{wrapfigure}{r}{2.5cm}
    \centering
    \qrcode[nolink]{https://data.gouv.fr/dataset/5b12086db5950870b30303f1}
  \end{wrapfigure}

Licence : \textbf{Licence Ouverte version 2.0
}\newline
Créé le : 2018-06-02\newline
Modifié le : 2019-03-17\newline
Popularité : 1 réutilisation,  0 suivi\newline
Mots-clé : \emph{administration, annuaire-des-bibliotheques-de-lenseignement-superieur, bibliotheque, bibliotheques, citoyennete, coordonnees-geographiques, culture, education, enseignement, esgbu, finances-publiques, formation, geolocalisation, gouvernement, patrimoine, recherche, reseaux-sociaux, sites-internet, url
}\newline
Permalien : \url{https://data.gouv.fr/dataset/5b12086db5950870b30303f1}\newline

\par
\noindent
    Ce jeu de données présentent les informations sur les bibliothèques de
l'enseignement supérieur qui ne sont pas sièges de~bibliothèques de
l'enseignement supérieur.

Ces informations alimentent l'annuaire des bibliothèques de
l'enseignement supérieur :
\url{http://bibliotheques.enseignementsup-recherche.gouv.fr/FR/annuaire/\%3E}


\vspace{0.5cm}
\needspace{12\baselineskip}
\subsection*{Cartographie de l'enseignement supérieur
}\index{atlas}\index{atlas!regional}\index{carte}\index{cartes}\index{enseignement}\index{formation}
  \begin{wrapfigure}{r}{2.5cm}
    \centering
    \qrcode[nolink]{https://data.gouv.fr/dataset/559e98cdc751df4443390bd3}
  \end{wrapfigure}

Licence : \textbf{Licence Ouverte
}\newline
Créé le : 2015-07-09\newline
Modifié le : 2016-01-30\newline
De 2011-09-01 à 2014-08-31\newline
Granularité : au point d'intérêt\newline
Mise à jour : ponctuelle\newline
Popularité : 2 réutilisations,  2 suivis\newline
Mots-clé : \emph{atlas, atlas-regional, carte, cartes, enseignement, formation
}\newline
Permalien : \url{https://data.gouv.fr/dataset/559e98cdc751df4443390bd3}\newline

\par
\noindent
    Cartes issues de la publication statistique "
\href{http://publication.enseignementsup-recherche.gouv.fr/atlas/atlas_regional_des_effectifs_etudiants.php}{Atlas
régional, les effectifs d'étudiants} " éditée par la sous-direction des
systèmes d'information et des études statistiques du Ministère de
l'Éducation nationale, de l'Enseignement supérieur et de la Recherche.

Les données sources de cette publication sont disponibles dans le jeu de
données "
\href{https://www.data.gouv.fr/fr/datasets/effectifs-d-etudiants-inscrits-dans-les-etablissements-et-les-formations-de-l-enseignement-superie-0/}{Effectifs
d'étudiants inscrits dans les établissements et les formations de
l'enseignement supérieur} ``.


\vspace{0.5cm}
\needspace{12\baselineskip}
\subsection*{Contrats signés du programme-cadre de recherche et développement
technologique (7ème PCRDT) de la Commission européenne
}\index{cooperation}\index{developpement}\index{e!corda}\index{europe}\index{financement!europeen}\index{fp7}\index{pcrdt}\index{recherche}
  \begin{wrapfigure}{r}{2.5cm}
    \centering
    \qrcode[nolink]{https://data.gouv.fr/dataset/536991b8a3a729239d203d2a}
  \end{wrapfigure}

Licence : \textbf{Licence Ouverte
}\newline
Créé le : 2014-04-23\newline
Modifié le : 2015-10-23\newline
De 2007-01-01 à 2014-02-21\newline
Popularité : 1 réutilisation,  1 suivi\newline
Mots-clé : \emph{cooperation, developpement, e-corda, europe, financement-europeen, fp7, pcrdt, recherche
}\newline
Permalien : \url{https://data.gouv.fr/dataset/536991b8a3a729239d203d2a}\newline

\par
\noindent
    Ce jeu de données contient le nombre de contrats signés dans le cadre du
7ème PCRDT à la date du 21 février 2014.

Données complémentaires dans \emph{Participations dans les contrats
signés du programme-cadre recherche et développement technologique (7ème
PCRDT) de la Commission européenne}


\vspace{0.5cm}
\needspace{12\baselineskip}
\subsection*{Contrats signés du programme-cadre de recherche et développement
technologique (H2020) de la Commission européenne
}\index{europe}\index{pcrdt}\index{recherche}\index{recherche!et!developpement}
  \begin{wrapfigure}{r}{2.5cm}
    \centering
    \qrcode[nolink]{https://data.gouv.fr/dataset/5660186ac751df37c1aad373}
  \end{wrapfigure}

Licence : \textbf{Licence Ouverte
}\newline
Créé le : 2015-12-03\newline
Modifié le : 2016-01-27\newline
De 2014-01-01 à 2015-07-15\newline
Granularité : au pays\newline
Mise à jour : ponctuelle\newline
Popularité : 1 réutilisation,  1 suivi\newline
Mots-clé : \emph{europe, pcrdt, recherche, recherche-et-developpement
}\newline
Permalien : \url{https://data.gouv.fr/dataset/5660186ac751df37c1aad373}\newline

\par
\noindent
    Ce jeu de données contient le nombre de contrats signés dans le cadre de
H2020 à la date du 15 juillet 2015.

Données complémentaires dans Participations dans les contrats signés du
programme-cadre recherche et développement technologique (H2020) de la
Commission européenne

Ces données sont reprises dans les tableau de bord du site
\href{http://www.horizon2020.gouv.fr/cid91235/donnees-statistiques-horizon-2020.html}{Horizon
2020} du ministère en charge de l'Enseignement supérieur et de la
Recherche.


\vspace{0.5cm}
\needspace{12\baselineskip}
\subsection*{Dictionnaire des compétences - RéFérens III pour la filière des ITRF
}\index{administration}\index{competences}\index{connaissances}\index{dictionnaire}\index{education}\index{emploi}\index{enseignement!superieur}\index{metier}\index{metiers}\index{public}\index{recherche}\index{referens3}\index{referentiel}\index{ressources!humaines}
  \begin{wrapfigure}{r}{2.5cm}
    \centering
    \qrcode[nolink]{https://data.gouv.fr/dataset/58123444c751df4dcac562c5}
  \end{wrapfigure}

Licence : \textbf{Licence Ouverte
}\newline
Créé le : 2016-10-27\newline
Modifié le : 2016-10-27\newline
De 2017-01-01 à 2019-12-31\newline
Mise à jour : ponctuelle\newline
Popularité : 1 réutilisation,  0 suivi\newline
Mots-clé : \emph{administration, competences, connaissances, dictionnaire, education, emploi, enseignement-superieur, metier, metiers, public, recherche, referens3, referentiel, ressources-humaines
}\newline
Permalien : \url{https://data.gouv.fr/dataset/58123444c751df4dcac562c5}\newline

\par
\noindent
    Ce jeu de données reprend les entrées du
\href{http://referens.enseignementsup-recherche.gouv.fr/pages/dictionnaire_competences/}{dictionnaire
des compétences} qui présente d'une manière structurée et transversale
les compétences requises dans une activité professionnelle des filières
ITRF (ingénieurs et personnels techniques de recherche et de formation)
et ITA (ingénieurs et personnels techniques de la recherche) et en
explicite le contenu. La structure de description des compétences
retenue s'appuie sur une approche « classique » en trois catégories de
compétences : savoir, savoir-faire, savoir-être.

Néanmoins, les intitulés des catégories de compétences jugés équivoques
(notamment le concept de « savoir-être ») ont été modifiés. Aussi, les
catégories de compétences sont renommées de la façon suivante : Savoir =
Connaissance Savoir-faire = Compétence opérationnelle Savoir-être =
Compétence comportementale


\vspace{0.5cm}
\needspace{12\baselineskip}
\subsection*{Écoles supérieures du professorat et de l'éducation : Concours
d'enseignement visés
}\index{concours}\index{enseignement}\index{enseignement!superieur}\index{espe}\index{etudiants}
  \begin{wrapfigure}{r}{2.5cm}
    \centering
    \qrcode[nolink]{https://data.gouv.fr/dataset/53699359a3a729239d204183}
  \end{wrapfigure}

Licence : \textbf{Licence Ouverte
}\newline
Créé le : 2014-04-23\newline
Modifié le : 2016-02-07\newline
De 2013-09-01 à 2015-06-30\newline
Popularité : 2 réutilisations,  1 suivi\newline
Mots-clé : \emph{concours, enseignement, enseignement-superieur, espe, etudiants
}\newline
Permalien : \url{https://data.gouv.fr/dataset/53699359a3a729239d204183}\newline

\par
\noindent
    Ce jeu de données permet d'identifier les implantations des~Écoles
supérieures du professorat et de l'éducation préparant aux concours
d'enseignement selon le type de concours visé, la section et l'option.


\vspace{0.5cm}
\needspace{12\baselineskip}
\subsection*{Écoles supérieures du professorat et de l'éducation : Implantations et
parcours MEEF proposés
}\index{enseignement}\index{enseignement!superieur}\index{espe}\index{etudiants}\index{formations}\index{meef}\index{parcours}
  \begin{wrapfigure}{r}{2.5cm}
    \centering
    \qrcode[nolink]{https://data.gouv.fr/dataset/53699359a3a729239d204184}
  \end{wrapfigure}

Licence : \textbf{Licence Ouverte
}\newline
Créé le : 2014-04-23\newline
Modifié le : 2015-11-23\newline
De 2013-09-01 à 2015-06-30\newline
Popularité : 2 réutilisations,  1 suivi\newline
Mots-clé : \emph{enseignement, enseignement-superieur, espe, etudiants, formations, meef, parcours
}\newline
Permalien : \url{https://data.gouv.fr/dataset/53699359a3a729239d204184}\newline

\par
\noindent
    Ce jeu de données permet de localiser les implantations et l'offre de
formation des Écoles supérieures du professorat et de l'éducation.


\vspace{0.5cm}
\needspace{12\baselineskip}
\subsection*{Écoles supérieures du professorat et de l'éducation : Parcours des
masters Métiers de l'enseignement, de l'éducation et de la formation
}\index{enseignement}\index{espe}\index{etudiant}\index{formations}\index{meef}
  \begin{wrapfigure}{r}{2.5cm}
    \centering
    \qrcode[nolink]{https://data.gouv.fr/dataset/5369935aa3a729239d204185}
  \end{wrapfigure}

Licence : \textbf{Licence Ouverte
}\newline
Créé le : 2014-04-23\newline
Modifié le : 2015-11-22\newline
De 2013-09-01 à 2015-06-30\newline
Popularité : 2 réutilisations,  1 suivi\newline
Mots-clé : \emph{enseignement, espe, etudiant, formations, meef
}\newline
Permalien : \url{https://data.gouv.fr/dataset/5369935aa3a729239d204185}\newline

\par
\noindent
    Ce jeu de données permet d'identifier les parcours des masters Métiers
de l'enseignement, de l'éducation et de la formation proposées dans les
implantations des Écoles supérieures du professorat et de l'éducation.


\vspace{0.5cm}
\needspace{12\baselineskip}
\subsection*{Effectifs d'étudiants inscrits dans les établissements et les formations
de l'enseignement supérieur
}\index{academie}\index{atlas}\index{commune}\index{cpge}\index{departement}\index{ens}\index{enseignement}\index{enseignement!superieur}\index{etudiant}\index{etudiants}\index{grand!etablissement}\index{ingenieur}\index{iut}\index{region}\index{superieur}\index{unite!urbaine}\index{universite}
  \begin{wrapfigure}{r}{2.5cm}
    \centering
    \qrcode[nolink]{https://data.gouv.fr/dataset/53699373a3a729239d2041c4}
  \end{wrapfigure}

Licence : \textbf{Licence Ouverte
}\newline
Créé le : 2014-04-23\newline
Modifié le : 2016-03-04\newline
De 2001-09-01 à 2014-08-31\newline
Granularité : à la commune\newline
Mise à jour : annuelle\newline
Popularité : 5 réutilisations,  4 suivis\newline
Mots-clé : \emph{academie, atlas, commune, cpge, departement, ens, enseignement, enseignement-superieur, etudiant, etudiants, grand-etablissement, ingenieur, iut, region, superieur, unite-urbaine, universite
}\newline
Permalien : \url{https://data.gouv.fr/dataset/53699373a3a729239d2041c4}\newline

\par
\noindent
    Ce jeu de données présente les effectifs d'étudiants inscrits dans les
établissements et les formations de l'enseignement supérieur, recensés
pour les années 2001-2002 à 2013-14 dans les systèmes d'information et
enquêtes du ministère de l'Éducation nationale, de l'Enseignement
supérieur et de la Recherche, des ministères en charge de l'Agriculture,
de la Pêche, de la Culture, de la Santé et des Sports.

Il décline les informations à tous les niveaux géographiques, de la
commune jusqu'au national.

Ces données sont diffusées au travers de tableaux et de cartes dans la
publication
``\href{http://publication.enseignementsup-recherche.gouv.fr/atlas/}{Atlas
régional, les effectifs d'étudiants}''.


\vspace{0.5cm}
\needspace{12\baselineskip}
\subsection*{Finalistes et lauréats du concours Ma Thèse en 180 secondes France
}\index{doctorat}\index{education}\index{enseignement!superieur}\index{these}
  \begin{wrapfigure}{r}{2.5cm}
    \centering
    \qrcode[nolink]{https://data.gouv.fr/dataset/577cb6f0c751df2ddb9901a0}
  \end{wrapfigure}

Licence : \textbf{Licence Ouverte
}\newline
Créé le : 2016-07-06\newline
Modifié le : 2016-07-06\newline
Granularité : au point d'intérêt\newline
Mise à jour : annuelle\newline
Popularité : 1 réutilisation,  0 suivi\newline
Mots-clé : \emph{doctorat, education, enseignement-superieur, these
}\newline
Permalien : \url{https://data.gouv.fr/dataset/577cb6f0c751df2ddb9901a0}\newline

\par
\noindent
    Ma thèse en 180 secondes permet aux doctorants de présenter leur sujet
de recherche, en français et en termes simples, à un auditoire profane
et diversifié. Chaque étudiant ou étudiante doit faire, en trois
minutes, un exposé clair, concis et néanmoins convaincant sur son projet
de recherche. Le tout avec l'appui d'une seule diapositive !

Ce concours s'inspire de Three minute thesis (3MT®), conçu à
l'Université du Queensland en Australie. Le concept a été repris en 2012
au Québec par l'Association francophone pour le savoir (Acfas) qui a
souhaité étendre le projet à l'ensemble des pays francophones. En
France, le concours est organisé par la Conférence des présidents
d'université (CPU) et le CNRS.


\vspace{0.5cm}
\needspace{12\baselineskip}
\subsection*{Insertion professionnelle des diplômé.e.s de Diplôme universitaire de
technologie (DUT) en universités et établissements assimilés - données
nationales par disciplines détaillées
}\index{boursier}\index{boursiers}\index{cadre}\index{cdi}\index{diplome}\index{diplomes}\index{donnees!sexuees}\index{dut}\index{emploi}\index{etablissements}\index{insertion}\index{insertion!professionnelle}\index{parite}\index{salaire}\index{statistiques}\index{taux!de!chomage}\index{universite}\index{universites}
  \begin{wrapfigure}{r}{2.5cm}
    \centering
    \qrcode[nolink]{https://data.gouv.fr/dataset/58480c1fc751df560cc0bb7e}
  \end{wrapfigure}

Licence : \textbf{Licence Ouverte
}\newline
Créé le : 2016-12-07\newline
Modifié le : 2016-12-07\newline
De 2013-06-01 à 2015-12-31\newline
Granularité : au point d'intérêt\newline
Mise à jour : annuelle\newline
Popularité : 1 réutilisation,  0 suivi\newline
Mots-clé : \emph{boursier, boursiers, cadre, cdi, diplome, diplomes, donnees-sexuees, dut, emploi, etablissements, insertion, insertion-professionnelle, parite, salaire, statistiques, taux-de-chomage, universite, universites
}\newline
Permalien : \url{https://data.gouv.fr/dataset/58480c1fc751df560cc0bb7e}\newline

\par
\noindent
    Ces données sont basées sur les données collectées dans le cadre de
l'opération nationale de collecte de données sur l'insertion
professionnelle des diplômés du Diplôme universitaire de technologie
(DUT). Cette enquête a été menée en décembre 2015, 18 et 30 mois après
l'obtention de leur diplôme, auprès des diplômés de DUT de la session
2013. Le taux d'insertion est défini comme étant le pourcentage de
diplômés occupant un emploi, quel qu'il soit, sur l'ensemble des
diplômés présents sur le marché du travail. Il est calculé sur les
diplômés de nationalité française, issus de la formation initiale,
entrés immédiatement et durablement sur le marché du travail après
l'obtention de leur diplôme en 2013. L'information collectée sur le
salaire porte sur le salaire net, primes comprises. Les salaires
affichés correspondent aux valeurs médianes sur les emplois à temps
plein. A partir de ces valeurs, on estime un salaire brut annuel, sur la
base d'un taux forfaitaire de passage du net au brut de 1,3 (donnée
moyenne constatée sur les salaires du secteur privé). L'enquête a été
menée par les universités dans le cadre d'une charte dont les
dispositions visent à garantir la comparabilité des résultats entre les
établissements. La coordination d'ensemble et l'exploitation de
l'enquête sont prises en charge par le ministère en charge de
l'Enseignement supérieur et de la Recherche.


\vspace{0.5cm}
\needspace{12\baselineskip}
\subsection*{Insertion professionnelle des diplômé.e.s de Licence professionnelle en
universités et établissements assimilés
}\index{cadre}\index{cdi}\index{diplome}\index{diplomes}\index{emploi}\index{enseignement!superieur}\index{etablissement}\index{etablissements}\index{insertion}\index{insertion!professionnelle}\index{licence!professionnelle}\index{salaire}\index{statistiques}\index{uai}\index{universite}\index{universites}
  \begin{wrapfigure}{r}{2.5cm}
    \centering
    \qrcode[nolink]{https://data.gouv.fr/dataset/584810d9c751df5ed8c0bb7e}
  \end{wrapfigure}

Licence : \textbf{Licence Ouverte
}\newline
Créé le : 2016-12-07\newline
Modifié le : 2016-12-07\newline
Granularité : au point d'intérêt\newline
Mise à jour : annuelle\newline
Popularité : 1 réutilisation,  0 suivi\newline
Mots-clé : \emph{cadre, cdi, diplome, diplomes, emploi, enseignement-superieur, etablissement, etablissements, insertion, insertion-professionnelle, licence-professionnelle, salaire, statistiques, uai, universite, universites
}\newline
Permalien : \url{https://data.gouv.fr/dataset/584810d9c751df5ed8c0bb7e}\newline

\par
\noindent
    Ces informations sont basées sur les données collectées dans le cadre de
l'opération nationale de collecte de données sur l'insertion
professionnelle des diplômés de Licence professionnelle.

Cette enquête a été menée

en décembre 2015, 18 et 30 mois après l'obtention de leur diplôme,
auprès des diplômés de Licence professionnelle de la session 2013. Le
taux d'insertion est défini comme étant le pourcentage de diplômés
occupant un emploi, quel qu'il soit, sur l'ensemble des diplômés
présents sur le marché du travail. Il est calculé sur les diplômés de
nationalité française, issus de la formation initiale, entrés
immédiatement et durablement sur le marché du travail après l'obtention
de leur diplôme.

L'information collectée sur le salaire porte sur le salaire net, primes
comprises. Les salaires affichés correspondent aux valeurs médianes sur
les emplois à temps plein. A partir de ces valeurs, on estime un salaire
brut annuel, sur la base d'un taux forfaitaire de passage du net au brut
de 1,3 (donnée moyenne constatée sur les salaires du secteur privé).

L'enquête a été menée par les universités dans le cadre d'une charte
dont les dispositions visent à garantir la comparabilité des résultats
entre les établissements. La coordination d'ensemble et l'exploitation
de l'enquête sont prises en charge par le ministère en charge de
l'Enseignement supérieur et de la Recherche.


\vspace{0.5cm}
\needspace{12\baselineskip}
\subsection*{Insertion professionnelle des diplômé.e.s de Licence professionnelle en
universités et établissements assimilés - données nationales par
disciplines détaillées
}\index{boursier}\index{boursiers}\index{cadre}\index{cdi}\index{diplome}\index{diplomes}\index{donnees!sexuees}\index{emploi}\index{etablissements}\index{insertion}\index{insertion!professionnelle}\index{licence!professionnelle}\index{parite}\index{salaire}\index{statistiques}\index{taux!de!chomage}\index{universite}\index{universites}
  \begin{wrapfigure}{r}{2.5cm}
    \centering
    \qrcode[nolink]{https://data.gouv.fr/dataset/584814c3c751df65ebc0bb7e}
  \end{wrapfigure}

Licence : \textbf{Licence Ouverte
}\newline
Créé le : 2016-12-07\newline
Modifié le : 2016-12-07\newline
Granularité : au point d'intérêt\newline
Mise à jour : annuelle\newline
Popularité : 1 réutilisation,  0 suivi\newline
Mots-clé : \emph{boursier, boursiers, cadre, cdi, diplome, diplomes, donnees-sexuees, emploi, etablissements, insertion, insertion-professionnelle, licence-professionnelle, parite, salaire, statistiques, taux-de-chomage, universite, universites
}\newline
Permalien : \url{https://data.gouv.fr/dataset/584814c3c751df65ebc0bb7e}\newline

\par
\noindent
    Ces données sont basées sur les données collectées dans le cadre de
l'opération nationale de collecte de données sur l'insertion
professionnelle des diplômés de Licence professionnelle. Cette enquête a
été menée en décembre 2015, 18 et 30 mois après l'obtention de leur
diplôme, auprès des diplômés de Licence professionnelle de la session
2013. Le taux d'insertion est défini comme étant le pourcentage de
diplômés occupant un emploi, quel qu'il soit, sur l'ensemble des
diplômés présents sur le marché du travail. Il est calculé sur les
diplômés de nationalité française, issus de la formation initiale,
entrés immédiatement et durablement sur le marché du travail après
l'obtention de leur diplôme en 2013. L'information collectée sur le
salaire porte sur le salaire net, primes comprises. Les salaires
affichés correspondent aux valeurs médianes sur les emplois à temps
plein. A partir de ces valeurs, on estime un salaire brut annuel, sur la
base d'un taux forfaitaire de passage du net au brut de 1,3 (donnée
moyenne constatée sur les salaires du secteur privé). L'enquête a été
menée par les universités dans le cadre d'une charte dont les
dispositions visent à garantir la comparabilité des résultats entre les
établissements. La coordination d'ensemble et l'exploitation de
l'enquête sont prises en charge par le ministère en charge de
l'Enseignement supérieur et de la Recherche.


\vspace{0.5cm}
\needspace{12\baselineskip}
\subsection*{Insertion professionnelle des diplômés de Master en universités et
établissements assimilés
}\index{insertion}\index{insertion!professionnelle}\index{master}\index{salaire}\index{travail}\index{universite}
  \begin{wrapfigure}{r}{2.5cm}
    \centering
    \qrcode[nolink]{https://data.gouv.fr/dataset/53699721a3a729239d204bf6}
  \end{wrapfigure}

Licence : \textbf{Licence Ouverte
}\newline
Créé le : 2014-04-23\newline
Modifié le : 2016-03-10\newline
De 2010-06-01 à 2013-12-01\newline
Granularité : au point d'intérêt\newline
Mise à jour : annuelle\newline
Popularité : 5 réutilisations,  4 suivis\newline
Mots-clé : \emph{insertion, insertion-professionnelle, master, salaire, travail, universite
}\newline
Permalien : \url{https://data.gouv.fr/dataset/53699721a3a729239d204bf6}\newline

\par
\noindent
    Ces informations sont basées sur les données collectées dans le cadre de
l'opération nationale de collecte de données sur l'insertion
professionnelle des diplômés de Master.

Cette enquête a été menée en décembre 2013, 30 mois après l'obtention de
leur diplôme, auprès de 59 600 diplômés de Master de la session 2011. en
décembre 2012, 30 mois après l'obtention de leur diplôme, auprès de 47
500 diplômés de Master de la session 2010. Le taux d'insertion est
défini comme étant le pourcentage de diplômés occupant un emploi, quel
qu'il soit, sur l'ensemble des diplômés présents sur le marché du
travail. Il est calculé sur les diplômés de nationalité française, issus
de la formation initiale, entrés immédiatement et durablement sur le
marché du travail après l'obtention de leur diplôme en 2011. Les
diplômés vérifiant ces conditions représentent 38 \% de l'ensemble des
diplômés de Master (39 \% pour la session 2010).

L'information collectée sur le salaire porte sur le salaire net, primes
comprises. Les salaires affichés correspondent aux valeurs médianes sur
les emplois à temps plein. A partir de ces valeurs, on estime un salaire
brut annuel, sur la base d'un taux forfaitaire de passage du net au brut
de 1,3 (donnée moyenne constatée sur les salaires du secteur privé).

L'enquête a été menée par les universités dans le cadre d'une charte
dont les dispositions visent à garantir la comparabilité des résultats
entre les établissements. La coordination d'ensemble et l'exploitation
de l'enquête sont prises en charge par le ministère en charge de
l'Enseignement supérieur et de la Recherche.

En 2011, le taux de réponses exploitables sur l'ensemble des universités
est de 71 \% (70 \% pour la session 2010) mais ce taux varie
sensiblement d'une université à l'autre (de 92 \% à 28 \% pour la
session 2011, de 93 \% à 9 \% pour la session 2010). Le taux de réponse
et les effectifs de répondants jouant sur la qualité des données, il a
été décidé de ne pas diffuser les résultats des universités ayant des
effectifs de répondants trop faibles (moins de 30) ou un taux de réponse
inférieur à 30 \% et de signaler par la mention « résultats fragiles »
celles dont le taux de réponse est inférieur à 50 \%.

Sources des données complémentaires :

\% de diplômés boursiers : données observées sur la population de
l'enquête sur l'insertion professionnelle.

Taux de chômage régional : INSEE - 4ème trimestre 2012 pour la session
2010, 4ème trimestre 2013 pour la session 2011.

Salaire net mensuel médian régional : INSEE DADS 2010 pour la session
2010 et INSEE DADS 2011 pour la session 2010 - pour les 25-29 ans
employés à temps plein dans les catégories socioprofessionnelles
``Cadres et professions intellectuelles supérieures'' et " Professions
intermédiaires.

Légende : nd = non disponible (aucun répondant) ns = non significatif
(nombre de répondants inférieur à 30).

\begin{center}\rule{0.5\linewidth}{\linethickness}\end{center}

Libellés des variables des fichiers exportés (pour l'année N) :

nombre\_de\_reponses : Nombre de réponses

taux\_de\_reponse : Taux de réponse

poids\_de\_la\_discipline : Poids de la discipline

emplois\_cadre\_ou\_professions\_intermediaires : Part des emplois de
niveau cadre ou profession intermédiaire

emplois\_stables : Part des emplois stables

emplois\_a\_temps\_plein : Part des emplois à temps plein

salaire\_net\_median\_des\_emplois\_a\_temps\_plein : Salaire net
mensuel médian des emplois à temps plein

salaire\_brut\_annuel\_estime : Salaire brut annuel médian estimé
de\_diplomes\_boursiers : Part des diplômés boursiers dans
l'établissement

taux\_de\_chomage\_regional : Taux de chômage régional (INSEE : 4ème
trimestre N+2)

salaire\_net\_mensuel\_median\_regional : Salaire mensuel net médian des
jeunes de 25 à 29 ans employés à temps plein dans les catégories cadre
et professions intermédiaires (INSEE : DADS N)

taux\_dinsertion : Taux d'insertion

emplois\_cadre : Part des emplois de niveau cadre (Dans certains
secteurs d'activité, les emplois correspondants au diplôme ne sont pas
tous de niveau cadre. L'accès au niveau cadre peut nécessiter une
expérience professionnelle préalable.)

emplois\_exterieurs\_a\_la\_region\_de\_luniversite Part des emplois
situés en dehors de la région de l'établissement (y compris à
l'étranger)

femmes : Part des femmes

Les données des diplômés de 2011 peuvent être consultées sur le site du
\href{http://www.enseignementsup-recherche.gouv.fr/pid24624/taux-insertion-professionnelle-des-diplomes-universite.html}{MENESR}

\href{https://www.data.gouv.fr/fr/datasets/insertion-professionnelle-des-diplomes-de-master-en-universites-et-etablissements-assimiles-donnees-nationales-par-discipline/}{Les
données nationales par discipline fine sont également disponibles.}


\vspace{0.5cm}
\needspace{12\baselineskip}
\subsection*{Insertion professionnelle des diplômés de Master en universités et
établissements assimilés - données nationales par discipline
}\index{emploi}\index{enseignement!superieur}\index{insertion}\index{salaire}\index{travail}
  \begin{wrapfigure}{r}{2.5cm}
    \centering
    \qrcode[nolink]{https://data.gouv.fr/dataset/54ad4ff3c751df50f7de6534}
  \end{wrapfigure}

Licence : \textbf{Licence Ouverte
}\newline
Créé le : 2015-01-07\newline
Modifié le : 2015-12-31\newline
De 2010-06-01 à 2013-12-01\newline
Granularité : au pays\newline
Mise à jour : annuelle\newline
Popularité : 1 réutilisation,  2 suivis\newline
Mots-clé : \emph{emploi, enseignement-superieur, insertion, salaire, travail
}\newline
Permalien : \url{https://data.gouv.fr/dataset/54ad4ff3c751df50f7de6534}\newline

\par
\noindent
    Ces données sont basées sur les données collectées dans le cadre de
l'opération nationale de collecte de données sur l'insertion
professionnelle des diplômés de Master.

Cette enquête a été menée

en décembre 2013, 30 mois après l'obtention de leur diplôme, auprès de
59 600 diplômés de Master de la session 2011. en décembre 2012, 30 mois
après l'obtention de leur diplôme, auprès de 47 500 diplômés de Master
de la session 2010. Le taux d'insertion est défini comme étant le
pourcentage de diplômés occupant un emploi, quel qu'il soit, sur
l'ensemble des diplômés présents sur le marché du travail. Il est
calculé sur les diplômés de nationalité française, issus de la formation
initiale, entrés immédiatement et durablement sur le marché du travail
après l'obtention de leur diplôme en 2011. Les diplômés vérifiant ces
conditions représentent 38 \% de l'ensemble des diplômés de Master (39
\% pour la session 2010). L'information collectée sur le salaire porte
sur le salaire net, primes comprises. Les salaires affichés
correspondent aux valeurs médianes sur les emplois à temps plein. A
partir de ces valeurs, on estime un salaire brut annuel, sur la base
d'un taux forfaitaire de passage du net au brut de 1,3 (donnée moyenne
constatée sur les salaires du secteur privé).

L'enquête a été menée par les universités dans le cadre d'une charte
dont les dispositions visent à garantir la comparabilité des résultats
entre les établissements. La coordination d'ensemble et l'exploitation
de l'enquête sont prises en charge par le ministère en charge de
l'Enseignement supérieur et de la Recherche.

En 2011, le taux de réponses exploitables sur l'ensemble des universités
est de 71 \% (70 \% pour la session 2010) mais ce taux varie
sensiblement d'une université à l'autre (de 92 \% à 28 \% pour la
session 2011, de 93 \% à 9 \% pour la session 2010). Le taux de réponse
et les effectifs de répondants jouant sur la qualité des données, il a
été décidé de ne pas diffuser les résultats des universités ayant des
effectifs de répondants trop faibles (moins de 30) ou un taux de réponse
inférieur à 30 \% et de signaler par la mention « résultats fragiles »
celles dont le taux de réponse est inférieur à 50 \%.

Sources des données complémentaires :

\% de diplômés boursiers : données observées sur la population de
l'enquête sur l'insertion professionnelle.

Taux de chômage régional : INSEE - 4ème trimestre 2012 pour la session
2010, 4ème trimestre 2013 pour la session 2011.

Salaire net mensuel médian régional : INSEE DADS 2010 pour la session
2010 et INSEE DADS 2011 pour la session 2010 - pour les 25-29 ans
employés à temps plein dans les catégories socioprofessionnelles
``Cadres et professions intellectuelles supérieures'' et " Professions
intermédiaires.

Légende : nd = non disponible (aucun répondant) ns = non significatif
(nombre de répondants inférieur à 30).

\begin{center}\rule{0.5\linewidth}{\linethickness}\end{center}

Libellés des variables des fichiers exportés (pour l'année N) :

nombre\_de\_reponses : Nombre de réponses

taux\_de\_reponse : Taux de réponse

poids\_de\_la\_discipline : Poids de la discipline

emplois\_cadre\_ou\_professions\_intermediaires : Part des emplois de
niveau cadre ou profession intermédiaire

emplois\_stables : Part des emplois stables

emplois\_a\_temps\_plein : Part des emplois à temps plein

salaire\_net\_median\_des\_emplois\_a\_temps\_plein : Salaire net
mensuel médian des emplois à temps plein

salaire\_brut\_annuel\_estime : Salaire brut annuel médian estimé

de\_diplomes\_boursiers : Part des diplômés boursiers dans
l'établissement

taux\_de\_chomage\_regional : Taux de chômage régional (INSEE : 4ème
trimestre N+2)

salaire\_net\_mensuel\_median\_regional : Salaire mensuel net médian des
jeunes de 25 à 29 ans employés à temps plein dans les catégories cadre
et professions intermédiaires (INSEE : DADS N)

taux\_dinsertion : Taux d'insertion

emplois\_cadre : Part des emplois de niveau cadre (Dans certains
secteurs d'activité, les emplois correspondants au diplôme ne sont pas
tous de niveau cadre. L'accès au niveau cadre peut nécessiter une
expérience professionnelle préalable.)

emplois\_exterieurs\_a\_la\_region\_de\_luniversite Part des emplois
situés en dehors de la région de l'établissement (y compris à
l'étranger)

femmes : Part des femmes

Les données des diplômés de 2011 peuvent être consultées sur le site du
\href{http://www.enseignementsup-recherche.gouv.fr/pid24624/taux-insertion-professionnelle-des-diplomes-universite.html}{MENESR}

\href{https://www.data.gouv.fr/fr/datasets/insertion-professionnelle-des-diplomes-de-master-en-universites-et-etablissements-assimil-0/}{Les
données par établissement sont également disponibles}.


\vspace{0.5cm}
\needspace{12\baselineskip}
\subsection*{Journées des Arts et de la Culture dans l'Enseignement Supérieur
}\index{2016}\index{art}\index{arts}\index{culture}\index{education}\index{education!artistique!et!culturel}\index{enseignement!superieur}\index{jaces}\index{jaces!2016}
  \begin{wrapfigure}{r}{2.5cm}
    \centering
    \qrcode[nolink]{https://data.gouv.fr/dataset/56eb191e88ee38750689c52e}
  \end{wrapfigure}

Licence : \textbf{Licence Ouverte
}\newline
Créé le : 2016-03-17\newline
Modifié le : 2016-03-17\newline
Granularité : au point d'intérêt\newline
Mise à jour : ponctuelle\newline
Popularité : 1 réutilisation,  1 suivi\newline
Mots-clé : \emph{2016, art, arts, culture, education, education-artistique-et-culturel, enseignement-superieur, jaces, jaces-2016
}\newline
Permalien : \url{https://data.gouv.fr/dataset/56eb191e88ee38750689c52e}\newline

\par
\noindent
    Les Journées des arts et de la culture dans l'enseignement supérieur
(JACES 2016) se déroulent les 29, 30 et 31 mars 2016.

La plupart des universités et certaines écoles de l'enseignement
supérieur participent à cette manifestation nationale en présentant les
réalisations les plus emblématiques de leur production. Les JACES,
vitrines des actions culturelles et artistiques menées dans
l'enseignement supérieur, commencent à être bien identifiées et
deviennent de plus en plus un moment de rendez-vous, de partage et de
rencontres culturelles et artistiques. Le grand public peut ainsi
découvrir la diversité et la qualité des offres culturelles et
artistiques dans l'enseignement supérieur.

Les Journées des arts et de la culture dans l'enseignement supérieur ont
été créées à l'occasion de la signature de la convention cadre
``Université, lieu de culture'' entre le ministère de l'Éducation
nationale, de l'Enseignement supérieur et de la Recherche et le
ministère de la Culture et de la Communication, à Avignon le 12 juillet
2013.

Pour en savoir plus, consultez le site des
\href{http://journee-arts-culture-sup.fr/}{Journées des arts et de la
culture dans l'enseignement supérieur (JACES 2016)}

Partenaires des JACES 2016 : ​AUC, CPU, CDEFI, CGE, CROUS-CNOUS, Radio
Campus


\vspace{0.5cm}
\needspace{12\baselineskip}
\subsection*{Lauréats I-LAB Concours national d'aide à la création d'entreprises de
technologies innovantes
}\index{creation!d!entreprises}\index{entreprises}\index{i!lab}\index{ilab}\index{innovation}\index{siren}
  \begin{wrapfigure}{r}{2.5cm}
    \centering
    \qrcode[nolink]{https://data.gouv.fr/dataset/577cb487c751df29779901a0}
  \end{wrapfigure}

Licence : \textbf{Licence Ouverte
}\newline
Créé le : 2016-07-06\newline
Modifié le : 2016-07-06\newline
Granularité : au point d'intérêt\newline
Mise à jour : annuelle\newline
Popularité : 1 réutilisation,  0 suivi\newline
Mots-clé : \emph{creation-d-entreprises, entreprises, i-lab, ilab, innovation, siren
}\newline
Permalien : \url{https://data.gouv.fr/dataset/577cb487c751df29779901a0}\newline

\par
\noindent
    i-LAB est né de la volonté du ministère de l'Education nationale, de
l'Enseignement supérieur et de la Recherche de renforcer le soutien à la
création d'entreprises innovantes, de mieux accompagner le développement
des start-up et d'encourager l'esprit d'entreprendre, en particulier
auprès des jeunes de l'enseignement supérieur.

Initié en 1999 par le ministère en charge de la Recherche dans le cadre
de la loi sur l'innovation et la recherche, le concours national d'aide
à la création d'entreprises de technologies innovantes, a été reconduit
chaque année, avec pour enjeux :

• de détecter et faire émerger des projets de création d'entreprises
s'appuyant sur des technologies innovantes;

• de favoriser le transfert des résultats de la recherche vers le monde
économique.


\vspace{0.5cm}
\needspace{12\baselineskip}
\subsection*{Les bénéficiaires de la prime d'excellence scientifique
}\index{enseignant!chercheur}\index{enseignement!superieur}\index{recherche}\index{recherche!universitaire}\index{universite}
  \begin{wrapfigure}{r}{2.5cm}
    \centering
    \qrcode[nolink]{https://data.gouv.fr/dataset/536997f6a3a729239d204e1d}
  \end{wrapfigure}

Licence : \textbf{Licence Ouverte
}\newline
Créé le : 2014-04-23\newline
Modifié le : 2016-02-21\newline
De 1993-09-01 à 2012-09-01\newline
Popularité : 2 réutilisations,  3 suivis\newline
Mots-clé : \emph{enseignant-chercheur, enseignement-superieur, recherche, recherche-universitaire, universite
}\newline
Permalien : \url{https://data.gouv.fr/dataset/536997f6a3a729239d204e1d}\newline

\par
\noindent
    Ce jeu de données présente les effectifs des bénéficiaires de la prime
d'excellence scientifique (ex prime d'encadrement doctoral et de
recherche) par sexe, secteur CNU, corps et établissement.


\vspace{0.5cm}
\needspace{12\baselineskip}
\subsection*{Les brevets français à l'INPI et l'OEB
}\index{administration}\index{amenagement!du!territoire}\index{atlas}\index{atlas!des!brevets}\index{batiments}\index{brevets}\index{business}\index{citoyennete}\index{deposants}\index{developpement!economique}\index{domaine!technologique}\index{domaines!emergents}\index{donnees!sexuees}\index{economie}\index{education}\index{emploi}\index{enseignement}\index{entreprises}\index{equipements}\index{eti}\index{finances!publiques}\index{formation}\index{gouvernement}\index{grande!entreprise}\index{innovation}\index{inventeurs}\index{logement}\index{micro}\index{organismes}\index{organismes!de!recherche}\index{patent}\index{pme}\index{pmi}\index{recherche}\index{recherche!publique}\index{recherche!universitaire}\index{siren}\index{universite}\index{universites}\index{urbanisme}
  \begin{wrapfigure}{r}{2.5cm}
    \centering
    \qrcode[nolink]{https://data.gouv.fr/dataset/5892a0cba3a72974c1f0de3c}
  \end{wrapfigure}

Licence : \textbf{Licence Ouverte version 2.0
}\newline
Créé le : 2017-02-02\newline
Modifié le : 2019-03-17\newline
Popularité : 1 réutilisation,  0 suivi\newline
Mots-clé : \emph{administration, amenagement-du-territoire, atlas, atlas-des-brevets, batiments, brevets, business, citoyennete, deposants, developpement-economique, domaine-technologique, domaines-emergents, donnees-sexuees, economie, education, emploi, enseignement, entreprises, equipements, eti, finances-publiques, formation, gouvernement, grande-entreprise, innovation, inventeurs, logement, micro, organismes, organismes-de-recherche, patent, pme, pmi, recherche, recherche-publique, recherche-universitaire, siren, universite, universites, urbanisme
}\newline
Permalien : \url{https://data.gouv.fr/dataset/5892a0cba3a72974c1f0de3c}\newline

\par
\noindent
    Les données brutes sont~fournies par l'INPI et~sont issues d'une
extraction de sa base interne de 2015 enrichie de la base Patstat pour
certains brevets européens. Il s'agit de la base des publications de
demandes de brevets et de délivrances de brevets en France à l'INPI
(voie nationale), et des demandes de brevets européens de l'Office
européen des brevets (OEB) qui entrent en phase nationale française.~La
base de données brevets de l'INPI est disponible dans son intégralité à
des fins de réutilisation dans le cadre d'une licence
(https://www.inpi.fr/fr/services-et-prestations/bases-en-acces-libre)

Les~données par brevet~présentent des dates de procédures, des adresses
françaises~des déposants et inventeurs, la nature des déposants et des
codes de la Classification internationale des brevets (CIB) et des
nomenclatures technologiques liées.~Afin de protéger les données
personnelles, les noms et adresses d'inventeurs et de déposants
(personne physique) ne figurent pas dans ce jeu de donnée.

Chaque brevet est géo-localisés à différents niveaux selon l'adresse des
déposants et des inventeurs~: pays, régions, départements, unités
urbaines, communes.

Ces données sont mobilisées dans
l'\href{http://publication.enseignementsup-recherche.gouv.fr/atlas-brevets/index.php/Accueil/fr}{Atlas
des brevets}

L'objectif de cet atlas est de contribuer à la compréhension \textbf{de
l'activité de recherche et d'innovation en France} au travers d'une
analyse territoriale du nombre de brevets. A partir d'un corpus fiable,
historisé, intégrant des nomenclatures-clés de la technologie, l'atlas
répond à des questions simples sur les publications de demandes de
brevets et délivrances de brevets (quoi, où, combien,qui ?). Il est
prévu une actualisation annuelle de cet atlas.~Les données disponibles
dans ce jeu ouvert n'ont aucune valeur légale.


\vspace{0.5cm}
\needspace{12\baselineskip}
\subsection*{Les étudiants étrangers dans l'enseignement supérieur
}\index{etudiant!etranger}
  \begin{wrapfigure}{r}{2.5cm}
    \centering
    \qrcode[nolink]{https://data.gouv.fr/dataset/53699841a3a729239d204ee9}
  \end{wrapfigure}

Licence : \textbf{Licence Ouverte
}\newline
Créé le : 2013-08-27\newline
Modifié le : 2016-01-25\newline
De 1990-09-01 à 2012-08-31\newline
Granularité : au pays\newline
Mise à jour : annuelle\newline
Popularité : 1 réutilisation,  3 suivis\newline
Mots-clé : \emph{etudiant-etranger
}\newline
Permalien : \url{https://data.gouv.fr/dataset/53699841a3a729239d204ee9}\newline

\par
\noindent
    Tableaux statistiques : évolution du nombre d'étudiants étrangers
inscrits dans l'enseignement supérieur ; répartition des étudiants de
nationalité étrangère dans les universités selon l'origine et le cursus


\vspace{0.5cm}
\needspace{12\baselineskip}
\subsection*{Les rectorats d'academies et vice-rectorats
}\index{academie}\index{administration}\index{education}\index{education!nationale}\index{geolocalisation}\index{localisation}\index{rectorat}\index{reseau!social}\index{wikidata}
  \begin{wrapfigure}{r}{2.5cm}
    \centering
    \qrcode[nolink]{https://data.gouv.fr/dataset/55317ab3c751df259437dddb}
  \end{wrapfigure}

Licence : \textbf{Licence Ouverte
}\newline
Créé le : 2015-04-17\newline
Modifié le : 2016-03-01\newline
De 2015-01-01 à 2015-12-31\newline
Granularité : au point d'intérêt\newline
Mise à jour : annuelle\newline
Popularité : 1 réutilisation,  1 suivi\newline
Mots-clé : \emph{academie, administration, education, education-nationale, geolocalisation, localisation, rectorat, reseau-social, wikidata
}\newline
Permalien : \url{https://data.gouv.fr/dataset/55317ab3c751df259437dddb}\newline

\par
\noindent
    Localisation et coordonnées des rectorats d'académies et des
vice-rectorats.

Situation au 1er janvier 2015.


\vspace{0.5cm}
\needspace{12\baselineskip}
\subsection*{Lexique français - anglais des termes liés à l'enseignement supérieur et
à la recherche en France
}\index{anglais}\index{education}\index{english}\index{enseignement!superieur}\index{enseignement!superieur!recherche}\index{francais}\index{french}\index{higher!education}\index{innovation}\index{lexique}\index{recherche}\index{research}\index{traduction}
  \begin{wrapfigure}{r}{2.5cm}
    \centering
    \qrcode[nolink]{https://data.gouv.fr/dataset/5653a8dc88ee38737fe72046}
  \end{wrapfigure}

Licence : \textbf{Licence Ouverte
}\newline
Créé le : 2015-11-24\newline
Modifié le : 2016-01-25\newline
De 2015-01-01 à 2015-12-31\newline
Mise à jour : ponctuelle\newline
Popularité : 2 réutilisations,  1 suivi\newline
Mots-clé : \emph{anglais, education, english, enseignement-superieur, enseignement-superieur-recherche, francais, french, higher-education, innovation, lexique, recherche, research, traduction
}\newline
Permalien : \url{https://data.gouv.fr/dataset/5653a8dc88ee38737fe72046}\newline

\par
\noindent
    Lexique français - anglais des termes liés à l'enseignement supérieur et
à la recherche en France, principalement employé dans la version
anglaise de la publication statistiques
\href{http://publication.enseignementsup-recherche.gouv.fr/eesr/8/l-etat-de-l-enseignement-superieur-et-de-la-recherche-en-france-8.php}{L'état
de l'Enseignement supérieur et de la Recherche en France n\degree{}8 -
juin 2015} :
\href{http://publication.enseignementsup-recherche.gouv.fr/eesr/8EN/higher-education-and-research-in-france-facts-and-figures-8EN.php}{Higher
education \& research in France, facts and figures 8th edition -
November 2015}


\vspace{0.5cm}
\needspace{12\baselineskip}
\subsection*{Nomenclature domaines émergents des brevets
}\index{atlas!des!brevets}\index{brevet}\index{domaines!emergents}\index{education}\index{enseignement}\index{formation}\index{organismes!de!recherche}\index{patent}\index{recherche}\index{recherche!publique}\index{recherche!universitaire}\index{structures!de!recherche}\index{technologie!emergente}
  \begin{wrapfigure}{r}{2.5cm}
    \centering
    \qrcode[nolink]{https://data.gouv.fr/dataset/5892a0cca3a72974cbf0de2e}
  \end{wrapfigure}

Licence : \textbf{Licence Ouverte
}\newline
Créé le : 2017-02-02\newline
Modifié le : 2017-02-03\newline
Popularité : 1 réutilisation,  0 suivi\newline
Mots-clé : \emph{atlas-des-brevets, brevet, domaines-emergents, education, enseignement, formation, organismes-de-recherche, patent, recherche, recherche-publique, recherche-universitaire, structures-de-recherche, technologie-emergente
}\newline
Permalien : \url{https://data.gouv.fr/dataset/5892a0cca3a72974cbf0de2e}\newline

\par
\noindent
    Pour identifier les domaines technologiques émergents, on utilise les
nomenclatures de diverses sources : l'OEB, l'INPI, Eurostat, l'OMPI et
enfin
l'OCDE\href{file:///C:/Projets/Brevets/Atlas/Atlas\%202016/Atlas\%20des\%20brevets\%20-\%20M\%C3\%A9thodologie\%20et\%20d\%C3\%A9finitions.docx\#_ftn1}{{[}1{]}}
qui a fait un travail d'identification précis des codes de la CIB et a
produit une nomenclature très fine sur l'environnement. On identifie
ainsi les domaines suivants :

\textbf{TIC}

******Biotechnologies**

******Nanotechnologies**

******Environnement **: utilisation de la CIB et de la Classification
coopérative des brevets
(CPC)\href{file:///C:/Projets/Brevets/Atlas/Atlas\%202016/Atlas\%20des\%20brevets\%20-\%20M\%C3\%A9thodologie\%20et\%20d\%C3\%A9finitions.docx\#_ftn2}{{[}2{]}}
. Les codes Y02 concernent l'environnement et le climat et sont répartis
selon 6 grands domaines :

Gestion environnementale générale

Les technologies d'adaptation liées à l'eau

Technologies d'atténuation du changement climatique liées à la
production d'énergie, la transmission et la distribution

Capture, stockage, séquestration ou élimination de gaz à effet de serre

Atténuation des changements climatiques, technologies liées aux
transports

Atténuation des changements climatiques, technologies liées aux
bâtiments

\textbf{Concepts de villes intelligentes, réseaux intelligents}

Les codes Y02 sont récupérés dans la base Patstat

\begin{center}\rule{0.5\linewidth}{\linethickness}\end{center}

\href{file:///C:/Projets/Brevets/Atlas/Atlas\%202016/Atlas\%20des\%20brevets\%20-\%20M\%C3\%A9thodologie\%20et\%20d\%C3\%A9finitions.docx\#_ftnref1}{\emph{\textbf{{[}1{]}}}}\emph{Eurostat
Patent classifications and technology areas
(\href{http://epp.eurostat.ec.europa.eu/cache/ITY_SDDS/Annexes/pat_esms_an4.pdf)}{http://epp.eurostat.ec.europa.eu/cache/ITY\_SDDS/An\ldots{}}}{]}(http://epp.eurostat.ec.europa.eu/cache/ITY\_SDDS/An\ldots{}{]}(http://epp.eurostat.ec.europa.eu/cache/ITY\_SDDS/Annexes/pat\_esms\_an4.pdf))\emph{)\emph{OEB
\href{http://www.epo.org/news-issues/issues/classification/nanotechnology_fr.html}{http://www.epo.org/news-issues/issues/classificati\ldots{}}}{]}(http://www.epo.org/news-issues/issues/classificati\ldots{}{]}(http://www.epo.org/news-issues/issues/classification/nanotechnology\_fr.html)})\emph{OCDE
Identifying technology areas for patents
(\href{http://www.oecd.org/sti/inno/40807441.pdf}{http://www.oecd.org/sti/inno/40807441.pdf)\textgreater{}\_}\textgreater{}})
\emph{OEB
\href{http://documents.epo.org/projects/babylon/eponet.nsf/0/9BA3D3AA8CE3B9BBC125785700578B40/$File/Patentinfo_News_1101_en.pdf}{http://documents.epo.org/projects/babylon/eponet.n\ldots{}}}{]}(http://documents.epo.org/projects/babylon/eponet.n\ldots{}{]}(http://documents.epo.org/projects/babylon/eponet.nsf/0/9BA3D3AA8CE3B9BBC125785700578B40/\$File/Patentinfo\_News\_1101\_en.pdf)\_)

\begin{center}\rule{0.5\linewidth}{\linethickness}\end{center}

\_ OEB :
{[}http://worldwide.espacenet.com/classification?loca\ldots{}{]}(http://worldwide.espacenet.com/classification?locale=fr\_EP{]}(http://worldwide.espacenet.com/classification?loca\ldots{}{]}(http://worldwide.espacenet.com/classification?locale=fr\_EP)!/CPC=Y)\_

\emph{OCDE :
\href{http://www.oecd.org/env/consumption-innovation/env-tech-search-strategies.pdf}{http://www.oecd.org/env/consumption-innovation/env\ldots{}}{]}(http://www.oecd.org/env/consumption-innovation/env\ldots{}{]}(http://www.oecd.org/env/consumption-innovation/env-tech-search-strategies.pdf))(2015)}

\_ WIPO :
\href{http://www.wipo.int/edocs/pubdocs/en/wipo_pub_941_2014.pdf}{http://www.wipo.int/edocs/pubdocs/en/wipo\_pub\_941\_\ldots{}}{]}(http://www.wipo.int/edocs/pubdocs/en/wipo\_pub\_941\_\ldots{}{]}(http://www.wipo.int/edocs/pubdocs/en/wipo\_pub\_941\_2014.pdf))(page
158)\_

\_ INPI : smart\_city\_2014.pdf\_

\href{file:///C:/Projets/Brevets/Atlas/Atlas\%202016/Atlas\%20des\%20brevets\%20-\%20M\%C3\%A9thodologie\%20et\%20d\%C3\%A9finitions.docx\#_ftnref2}{\emph{\textbf{{[}2{]}}}}\_
La CPC est une extension de la CIB et est gérée conjointement par l'OEB
et l'USPTO Elle est divisée en neuf sections, A-H et Y, qui sont à leurs
tours subdivisées en classes, sous-classes, groupes et sous-groupes.
Créée en 2013, la CPC comporte environ 250 000 entrées de
classification, ce qui en fait un système plus précis.\_


\vspace{0.5cm}
\needspace{12\baselineskip}
\subsection*{Nomenclature simplifiée codes CIB des brevets
}\index{atlas!des!brevets}\index{brevet}\index{domaine!technologique}\index{education}\index{enseignement}\index{formation}\index{innovation}\index{organismes!de!recherche}\index{patent}\index{recherche}\index{recherche!publique}\index{recherche!universitaire}\index{regions!2016}\index{structures!de!recherche}
  \begin{wrapfigure}{r}{2.5cm}
    \centering
    \qrcode[nolink]{https://data.gouv.fr/dataset/5892a0cca3a72974c1f0de3d}
  \end{wrapfigure}

Licence : \textbf{Licence Ouverte
}\newline
Créé le : 2017-02-02\newline
Modifié le : 2017-02-03\newline
Popularité : 1 réutilisation,  0 suivi\newline
Mots-clé : \emph{atlas-des-brevets, brevet, domaine-technologique, education, enseignement, formation, innovation, organismes-de-recherche, patent, recherche, recherche-publique, recherche-universitaire, regions-2016, structures-de-recherche
}\newline
Permalien : \url{https://data.gouv.fr/dataset/5892a0cca3a72974c1f0de3d}\newline

\par
\noindent
    Chaque brevet contient des codes de la CIB (Classification
internationale des brevets) attribués par l'office examinateur, qui
indiquent selon une arborescence très détaillée le domaine technique
auquel se rapporte l'invention : la CIB divise la technologie en huit
sections comptant environ \textbf{70 000} subdivisions. C'est un système
hiérarchique de symboles indépendants de la langue pour le classement
des brevets selon les différents domaines technologiques auxquels ils
appartiennent.Cette nomenclature étant difficile à utiliser, la
classification technologique utilisée est la classification simplifiée
établie par l'Organisation mondiale de la propriété intellectuelle
(OMPI) qui regroupe les classes technologiques de la CIB en 5 domaines
technologiques et 35 sous-domaines technologiques qui correspondent au
domaine d'application de l'invention (et non au secteur d'activité
économique).


\vspace{0.5cm}
\needspace{12\baselineskip}
\subsection*{Parcours des bachelières et bacheliers inscrits pour la première fois en
PACES
}\index{bac}\index{bacheliers}\index{cohorte}\index{data1ercycle}\index{datapaces}\index{donnees!sexuees}\index{education}\index{enseignement}\index{formation}\index{formations!de!sante}\index{mention!au!bac}\index{paces}\index{parcours}\index{recherche}\index{reussite}\index{statistiques}
  \begin{wrapfigure}{r}{2.5cm}
    \centering
    \qrcode[nolink]{https://data.gouv.fr/dataset/5a14e8b9a3a7297775073502}
  \end{wrapfigure}

Licence : \textbf{Licence Ouverte version 2.0
}\newline
Créé le : 2017-11-22\newline
Modifié le : 2019-03-17\newline
Popularité : 1 réutilisation,  0 suivi\newline
Mots-clé : \emph{bac, bacheliers, cohorte, data1ercycle, datapaces, donnees-sexuees, education, enseignement, formation, formations-de-sante, mention-au-bac, paces, parcours, recherche, reussite, statistiques
}\newline
Permalien : \url{https://data.gouv.fr/dataset/5a14e8b9a3a7297775073502}\newline

\par
\noindent
    Les indicateurs sont réalisés à partir des données issues du Système
d'Information sur le Suivi de l'Etudiant (SISE), qui recense les
inscrits (SISE Inscriptions). Le champ couvre l'ensemble des universités
françaises (y compris l'université de Lorraine devenue grand
établissement en 2011, l'institut universitaire d'Albi, le CUFR de
Mayotte et l'université de Polynésie française mais non compris
l'université de Nouvelle Calédonie (données non disponibles en
2015-2016)).

Ces indicateurs sont calculés sur la base des inscriptions
administratives, et non d'une présence effective de l'étudiant. Ils
peuvent être de ce fait biaisés négativement. Passage en deuxième année
en un ou deux ans PACES

Le champ des indicateurs est constitué des bacheliers 2014 inscrits en
2014-2015 en Première Année Commune aux Études de Santé (PACES) dans
l'enseignement supérieur public.

Les indicateurs sont calculés en rapportant des effectifs d'étudiants :

Base (dénominateur) : étudiants du champ inscrits en PACES pendant
l'année 2014-2015 ;

Passage en deuxième année (numérateur) : étudiants de la base
(dénominateur) inscrits, quel que soit l'établissement d'accueil, en
deuxième année de premier cycle d'études de médecine (PCEM2),
d'odontologie (dentaire), de pharmacie ou de maïeutique (sage-femme) :\\
pendant l'année 2015-2016 (passage en un an) ;\\
pendant l'année 2016-2017 (passage en deux ans).

Redoublement (numérateur) : étudiants de la base (dénominateur)
réinscrits en PACES pendant l'année 2015-2016, quel que soit
l'établissement d'accueil.


\vspace{0.5cm}
\needspace{12\baselineskip}
\subsection*{Parcours des bachelières et bacheliers inscrits pour la première fois en
PACES (données consolidées)
}\index{bac}\index{bacheliers}\index{cohorte}\index{data1ercycle}\index{datapaces}\index{donnees!sexuees}\index{education}\index{enseignement}\index{formation}\index{formations!de!sante}\index{mention!au!bac}\index{paces}\index{parcours}\index{recherche}\index{reussite}\index{statistiques}
  \begin{wrapfigure}{r}{2.5cm}
    \centering
    \qrcode[nolink]{https://data.gouv.fr/dataset/5a14e8c4b5950840efec2627}
  \end{wrapfigure}

Licence : \textbf{Licence Ouverte version 2.0
}\newline
Créé le : 2017-11-22\newline
Modifié le : 2019-03-17\newline
Popularité : 1 réutilisation,  0 suivi\newline
Mots-clé : \emph{bac, bacheliers, cohorte, data1ercycle, datapaces, donnees-sexuees, education, enseignement, formation, formations-de-sante, mention-au-bac, paces, parcours, recherche, reussite, statistiques
}\newline
Permalien : \url{https://data.gouv.fr/dataset/5a14e8c4b5950840efec2627}\newline

\par
\noindent
    \_\_Ce jeu de données présente les données sur les parcours et réussites
en PACES en présentant tous les croisements possibles sur les variables
suivantes : genre, série ou type de bac, mention et âge au bac. Il
existe
également\href{https://data.enseignementsup-recherche.gouv.fr/explore/dataset/fr-esr-parcours-des-bacheliers-en-paces/}{un
jeu de données présentant ces mêmes données sous forme de~données
micro-agrégés}.

Les indicateurs sont réalisés à partir des données issues du Système
d'Information sur le Suivi de l'Etudiant (SISE), qui recense les
inscrits (SISE Inscriptions). Le champ couvre l'ensemble des universités
françaises (y compris l'université de Lorraine devenue grand
établissement en 2011, l'institut universitaire d'Albi, le CUFR de
Mayotte et l'université de Polynésie française mais non compris
l'université de Nouvelle Calédonie (données non disponibles en
2015-2016)).

Ces indicateurs sont calculés sur la base des inscriptions
administratives, et non d'une présence effective de l'étudiant. Ils
peuvent être de ce fait biaisés négativement. Passage en deuxième année
en un ou deux ans PACES

Le champ des indicateurs est constitué des bacheliers 2014 inscrits en
2014-2015 en Première Année Commune aux Études de Santé (PACES) dans
l'enseignement supérieur public.

Les indicateurs sont calculés en rapportant des effectifs d'étudiants :

\textbf{Base (dénominateur)} : étudiants du champ inscrits en PACES
pendant l'année 2014-2015 ;

\textbf{Passage en deuxième année (numérateur)} : étudiants de la base
(dénominateur) inscrits, quel que soit l'établissement d'accueil, en
deuxième année de premier cycle d'études de médecine (PCEM2),
d'odontologie (dentaire), de pharmacie ou de maïeutique (sage-femme) :\\
pendant l'année 2015-2016 (passage en un an) ;\\
pendant l'année 2016-2017 (passage en deux ans).

\textbf{Redoublement (numérateur)} : étudiants de la base (dénominateur)
réinscrits en PACES pendant l'année 2015-2016, quel que soit
l'établissement d'accueil.


\vspace{0.5cm}
\needspace{12\baselineskip}
\subsection*{Parcours et réussite des bachelières et bacheliers inscrits pour la
première fois en DUT
}\index{bac}\index{bacheliers}\index{cohorte}\index{data1ercycle}\index{datadut}\index{dataiut}\index{diplome!universitaire!de!technologie}\index{donnees!sexuees}\index{dut}\index{education}\index{enseignement}\index{enseignement!superieur}\index{etudiants}\index{formation}\index{formations}\index{iut}\index{mention!au!bac}\index{parcours}\index{recherche}\index{reussite}\index{reussite!dut}\index{statistiques}
  \begin{wrapfigure}{r}{2.5cm}
    \centering
    \qrcode[nolink]{https://data.gouv.fr/dataset/5a260d37a3a7293578c3f71a}
  \end{wrapfigure}

Licence : \textbf{Licence Ouverte version 2.0
}\newline
Créé le : 2017-12-05\newline
Modifié le : 2019-03-17\newline
Popularité : 1 réutilisation,  0 suivi\newline
Mots-clé : \emph{bac, bacheliers, cohorte, data1ercycle, datadut, dataiut, diplome-universitaire-de-technologie, donnees-sexuees, dut, education, enseignement, enseignement-superieur, etudiants, formation, formations, iut, mention-au-bac, parcours, recherche, reussite, reussite-dut, statistiques
}\newline
Permalien : \url{https://data.gouv.fr/dataset/5a260d37a3a7293578c3f71a}\newline

\par
\noindent
    Les indicateurs sont réalisés à partir des données issues du Système
d'Information sur le Suivi de l'Etudiant (SISE), qui recense les
inscrits (SISE Inscriptions) et les diplômés (SISE-Résultats). Le champ
couvre l'ensemble des universités françaises (y compris l'université de
Lorraine devenue grand établissement en 2011, l'institut universitaire
d'Albi, le CUFR de Mayotte et l'université de Polynésie française mais
non compris l'université de Nouvelle Calédonie (données non disponibles
en 2015-2016)).

Ces indicateurs sont calculés sur la base des inscriptions
administratives, et non d'une présence effective de l'étudiant. Ils
peuvent être de ce fait biaisés négativement. Passage en deuxième année
en un ou deux ans DUT

Le champ des indicateurs est constitué des bacheliers 2014 inscrits en
2014-2015 en première année de DUT dans l'enseignement supérieur public.
En sont exclus les étudiants ayant pris une inscription parallèle en
Licence, STS ou CPGE.

Les indicateurs sont calculés en rapportant des effectifs d'étudiants :

\textbf{Base (dénominateur)} : étudiants du champ inscrits en première
année de DUT pendant l'année 2014-2015 ;

\textbf{Passage en deuxième année (numérateur)} : étudiants de la base
(dénominateur) inscrits, quel que soit l'établissement d'accueil, en
deuxième année de DUT :\\
pendant l'année 2015-2016 (passage en un an) ;\\
pendant l'année 2016-2017 (passage en deux ans).

Précision sur les inscriptions multiples : si un étudiant est inscrit
dans plusieurs établissements pendant les années 2015-2016 ou 2016-2017
et si sa situation n'est pas la même dans ces établissements, c'est la
situation la plus favorable qui est retenue (passage en deuxième année
puis redoublement puis réorientation).

\textbf{Redoublement (numérateur)} : étudiants de la base (dénominateur)
réinscrits en première année de DUT~pendant l'année 2015-2016, quel que
soit l'établissement d'accueil. Obtention du DUT en trois ou quatre ans

Le champ des indicateurs est constitué des bacheliers 2013 inscrits en
2013-2014 en première année de DUT dans l'enseignement supérieur public.
En sont exclus les étudiants ayant pris une inscription parallèle en
Licence, STS ou CPGE.

L'indicateur est calculé en rapportant des effectifs d'étudiants :

\textbf{Base (dénominateur)} : étudiants du champ inscrit en première
année de DUT pendant l'année 2013-2014 ;

\textbf{Réussite en 2 ans (numérateur)} : étudiants de la base
(dénominateur) ayant obtenu le DUT à la session 2015, que ce soit ou non
dans l'établissement ou la spécialité de son inscription en première
année ;

\textbf{Réussite en 3 ans (numérateur)} : étudiants de la base
(dénominateur) ayant obtenu le DUT à la session 2016 (sans l'avoir
obtenu à la session 2015), que ce soit ou non dans l'établissement ou la
spécialité de son inscription en première année.


\vspace{0.5cm}
\needspace{12\baselineskip}
\subsection*{Parcours et réussite des bachelières et bacheliers inscrits pour la
première fois en DUT (données consolidées)
}\index{bac}\index{bacheliers}\index{cohorte}\index{data1ercycle}\index{datadut}\index{dataiut}\index{diplome!universitaire!de!technologie}\index{donnees!sexuees}\index{dut}\index{education}\index{enseignement}\index{enseignement!superieur}\index{etudiants}\index{formation}\index{formations}\index{iut}\index{mention!au!bac}\index{parcours}\index{recherche}\index{reussite}\index{reussite!dut}\index{statistiques}
  \begin{wrapfigure}{r}{2.5cm}
    \centering
    \qrcode[nolink]{https://data.gouv.fr/dataset/5a260d5ca3a7293578c3f71b}
  \end{wrapfigure}

Licence : \textbf{Licence Ouverte version 2.0
}\newline
Créé le : 2017-12-05\newline
Modifié le : 2019-03-17\newline
Popularité : 1 réutilisation,  0 suivi\newline
Mots-clé : \emph{bac, bacheliers, cohorte, data1ercycle, datadut, dataiut, diplome-universitaire-de-technologie, donnees-sexuees, dut, education, enseignement, enseignement-superieur, etudiants, formation, formations, iut, mention-au-bac, parcours, recherche, reussite, reussite-dut, statistiques
}\newline
Permalien : \url{https://data.gouv.fr/dataset/5a260d5ca3a7293578c3f71b}\newline

\par
\noindent
    \_\_Ce jeu de données présente les données sur les parcours et réussites
en DUT en présentant tous les croisements possibles sur les variables
suivantes : spécialité, genre, série ou type de bac, mention et âge au
bac. Il existe
également\href{https://data.enseignementsup-recherche.gouv.fr/explore/dataset/fr-esr-parcours-et-reussite-des-bacheliers-en-DUT/}{un
jeu de données présentant ces mêmes données sous forme de~données
micro-agrégés}.

Les indicateurs sont réalisés à partir des données issues du Système
d'Information sur le Suivi de l'Etudiant (SISE), qui recense les
inscrits (SISE Inscriptions) et les diplômés (SISE-Résultats). Le champ
couvre l'ensemble des universités françaises (y compris l'université de
Lorraine devenue grand établissement en 2011, l'institut universitaire
d'Albi, le CUFR de Mayotte et l'université de Polynésie française mais
non compris l'université de Nouvelle Calédonie (données non disponibles
en 2015-2016)).

Ces indicateurs sont calculés sur la base des inscriptions
administratives, et non d'une présence effective de l'étudiant. Ils
peuvent être de ce fait biaisés négativement. Passage en deuxième année
en un ou deux ans DUT

Le champ des indicateurs est constitué des bacheliers 2014 inscrits en
2014-2015 en première année de DUT dans l'enseignement supérieur public.
En sont exclus les étudiants ayant pris une inscription parallèle en
Licence, STS ou CPGE.

Les indicateurs sont calculés en rapportant des effectifs d'étudiants :

\textbf{Base (dénominateur)} : étudiants du champ inscrits en première
année de DUT pendant l'année 2014-2015 ;

\textbf{Passage en deuxième année (numérateur)} : étudiants de la base
(dénominateur) inscrits, quel que soit l'établissement d'accueil, en
deuxième année de DUT :\\
pendant l'année 2015-2016 (passage en un an) ;\\
pendant l'année 2016-2017 (passage en deux ans).

Précision sur les inscriptions multiples : si un étudiant est inscrit
dans plusieurs établissements pendant les années 2015-2016 ou 2016-2017
et si sa situation n'est pas la même dans ces établissements, c'est la
situation la plus favorable qui est retenue (passage en deuxième année
puis redoublement puis réorientation).

\textbf{Redoublement (numérateur)} : étudiants de la base (dénominateur)
réinscrits en première année de DUT~pendant l'année 2015-2016, quel que
soit l'établissement d'accueil. Obtention du DUT en trois ou quatre ans

Le champ des indicateurs est constitué des bacheliers 2013 inscrits en
2013-2014 en première année de DUT dans l'enseignement supérieur public.
En sont exclus les étudiants ayant pris une inscription parallèle en
Licence, STS ou CPGE.

L'indicateur est calculé en rapportant des effectifs d'étudiants :

\textbf{Base (dénominateur)} : étudiants du champ inscrit en première
année de DUT pendant l'année 2013-2014 ;

\textbf{Réussite en 2 ans (numérateur)} : étudiants de la base
(dénominateur) ayant obtenu le DUT à la session 2015, que ce soit ou non
dans l'établissement ou la spécialité de son inscription en première
année ;

\textbf{Réussite en 3 ans (numérateur)} : étudiants de la base
(dénominateur) ayant obtenu le DUT à la session 2016 (sans l'avoir
obtenu à la session 2015), que ce soit ou non dans l'établissement ou la
spécialité de son inscription en première année.


\vspace{0.5cm}
\needspace{12\baselineskip}
\subsection*{Parcours et réussite des bachelières et bacheliers inscrits pour la
première fois en licence
}\index{bac}\index{bacheliers}\index{cohorte}\index{data1ercycle}\index{datalicence}\index{disciplines}\index{donnees!sexuees}\index{education}\index{enseignement}\index{enseignement!superieur}\index{etudiants}\index{formation}\index{formations}\index{grandes!disciplines}\index{licence}\index{mention!au!bac}\index{parcours}\index{recherche}\index{reussite}\index{reussite!licence}\index{secteurs!disciplinaires}\index{statistiques}
  \begin{wrapfigure}{r}{2.5cm}
    \centering
    \qrcode[nolink]{https://data.gouv.fr/dataset/5a14e8cea3a7297775073504}
  \end{wrapfigure}

Licence : \textbf{Licence Ouverte version 2.0
}\newline
Créé le : 2017-11-22\newline
Modifié le : 2019-03-17\newline
Popularité : 1 réutilisation,  0 suivi\newline
Mots-clé : \emph{bac, bacheliers, cohorte, data1ercycle, datalicence, disciplines, donnees-sexuees, education, enseignement, enseignement-superieur, etudiants, formation, formations, grandes-disciplines, licence, mention-au-bac, parcours, recherche, reussite, reussite-licence, secteurs-disciplinaires, statistiques
}\newline
Permalien : \url{https://data.gouv.fr/dataset/5a14e8cea3a7297775073504}\newline

\par
\noindent
    Les indicateurs sont réalisés à partir des données issues du Système
d'Information sur le Suivi de l'Etudiant (SISE), qui recense les
inscrits (SISE Inscriptions) et les diplômés (SISE-Résultats). Le champ
couvre l'ensemble des universités françaises (y compris l'université de
Lorraine devenue grand établissement en 2011, l'institut universitaire
d'Albi, le CUFR de Mayotte et l'université de Polynésie française mais
non compris l'université de Nouvelle Calédonie (données non disponibles
en 2015-2016)).

Ces indicateurs sont calculés sur la base des inscriptions
administratives, et non d'une présence effective de l'étudiant. Ils
peuvent être de ce fait biaisés négativement. Passage en deuxième année
en un ou deux ans Licence

Le champ des indicateurs est constitué des bacheliers 2014 inscrits en
2014-2015 en première année de licence (hors licence professionnelle)
dans l'enseignement supérieur public. En sont exclus les étudiants ayant
pris une inscription parallèle en STS, DUT ou CPGE.

Les indicateurs sont calculés en rapportant des effectifs d'étudiants :

\textbf{Base (dénominateur)} : étudiants du champ inscrits en première
année de licence (hors licence professionnelle) pendant l'année
2014-2015 ;

\textbf{Passage en deuxième année (numérateur)} : étudiants de la base
(dénominateur) inscrits, quel que soit l'établissement d'accueil, en
deuxième année de licence :\\
pendant l'année 2015-2016 (passage en un an) ;\\
pendant l'année 2016-2017 (passage en deux ans).

Précision sur les inscriptions multiples : si un étudiant est inscrit
dans plusieurs établissements pendant les années 2015-2016 ou 2016-2017
et si sa situation n'est pas la même dans ces établissements, c'est la
situation la plus favorable qui est retenue (passage en deuxième année
puis redoublement puis réorientation).

\textbf{Redoublement (numérateur)} : étudiants de la base (dénominateur)
réinscrits en première année de licence pendant l'année 2015-2016, quel
que soit l'établissement d'accueil.

\textbf{Réorientation en DUT (numérateur)} : étudiants de la base
(dénominateur) inscrits en première ou deuxième année de DUT pendant
l'année 2015-2016 (passage en DUT après 1 an) ou pendant l'année
2016-2017 (passage en DUT après 2 ans), quel que soit l'établissement
d'accueil. Obtention de la licence en trois ou quatre ans

Le champ des indicateurs est constitué des bacheliers 2012 inscrits en
2012-2013 en première année de licence dans l'enseignement supérieur
public. En sont exclus les étudiants ayant pris une inscription
parallèle en STS, DUT ou CPGE.

L'indicateur est calculé en rapportant des effectifs d'étudiants :

\textbf{Base (dénominateur)} : étudiants du champ inscrit en première
année de licence pendant l'année 2012-2013 ;

\textbf{Réussite en 3 ans (numérateur)} : étudiants de la base
(dénominateur) ayant obtenu le diplôme de la licence (éventuellement
professionnelle) à la session 2015, que ce soit ou non dans
l'établissement ou la discipline de son inscription en première année ;

\textbf{Réussite en 4 ans (numérateur)} : étudiants de la base
(dénominateur) ayant obtenu le diplôme de la licence (éventuellement
professionnelle) à la session 2016 (sans l'avoir obtenu à la session
2015), que ce soit ou non dans l'établissement ou la discipline de son
inscription en première année.


\vspace{0.5cm}
\needspace{12\baselineskip}
\subsection*{Parcours et réussite des bachelières et bacheliers inscrits pour la
première fois en licence (données consolidées)
}\index{bac}\index{bacheliers}\index{cohorte}\index{data1ercycle}\index{datalicence}\index{disciplines}\index{donnees!sexuees}\index{education}\index{enseignement}\index{enseignement!superieur}\index{etudiants}\index{formation}\index{formations}\index{grandes!disciplines}\index{licence}\index{mention!au!bac}\index{parcours}\index{recherche}\index{reussite}\index{reussite!licence}\index{secteurs!disciplinaires}\index{statistiques}
  \begin{wrapfigure}{r}{2.5cm}
    \centering
    \qrcode[nolink]{https://data.gouv.fr/dataset/5a14e8c3a3a7297775073503}
  \end{wrapfigure}

Licence : \textbf{Licence Ouverte version 2.0
}\newline
Créé le : 2017-11-22\newline
Modifié le : 2019-03-17\newline
Popularité : 1 réutilisation,  0 suivi\newline
Mots-clé : \emph{bac, bacheliers, cohorte, data1ercycle, datalicence, disciplines, donnees-sexuees, education, enseignement, enseignement-superieur, etudiants, formation, formations, grandes-disciplines, licence, mention-au-bac, parcours, recherche, reussite, reussite-licence, secteurs-disciplinaires, statistiques
}\newline
Permalien : \url{https://data.gouv.fr/dataset/5a14e8c3a3a7297775073503}\newline

\par
\noindent
    \_\_Ce jeu de données présente les données sur les parcours et réussites
en Licence en présentant tous les croisements possibles sur les
variables suivantes : grande discipline, discipline, secteur
disciplinaire, genre, série ou type de bac, mention et âge au bac. Il
existe
également\href{https://data.enseignementsup-recherche.gouv.fr/explore/dataset/fr-esr-parcours-et-reussite-des-bacheliers-en-licence/}{un
jeu de données présentant ces mêmes données sous forme de~données
micro-agrégés}.

Les indicateurs sont réalisés à partir des données issues du Système
d'Information sur le Suivi de l'Etudiant (SISE), qui recense les
inscrits (SISE Inscriptions) et les diplômés (SISE-Résultats). Le champ
couvre l'ensemble des universités françaises (y compris l'université de
Lorraine devenue grand établissement en 2011, l'institut universitaire
d'Albi, le CUFR de Mayotte et l'université de Polynésie française mais
non compris l'université de Nouvelle Calédonie (données non disponibles
en 2015-2016)).

Ces indicateurs sont calculés sur la base des inscriptions
administratives, et non d'une présence effective de l'étudiant. Ils
peuvent être de ce fait biaisés négativement. Passage en deuxième année
en un ou deux ans Licence

Le champ des indicateurs est constitué des bacheliers 2014 inscrits en
2014-2015 en première année de licence (hors licence professionnelle)
dans l'enseignement supérieur public. En sont exclus les étudiants ayant
pris une inscription parallèle en STS, DUT ou CPGE.

Les indicateurs sont calculés en rapportant des effectifs d'étudiants :

\textbf{Base (dénominateur)} : étudiants du champ inscrits en première
année de licence (hors licence professionnelle) pendant l'année
2014-2015 ;

\textbf{Passage en deuxième année (numérateur)} : étudiants de la base
(dénominateur) inscrits, quel que soit l'établissement d'accueil, en
deuxième année de licence :\\
pendant l'année 2015-2016 (passage en un an) ;\\
pendant l'année 2016-2017 (passage en deux ans).

Précision sur les inscriptions multiples : si un étudiant est inscrit
dans plusieurs établissements pendant les années 2015-2016 ou 2016-2017
et si sa situation n'est pas la même dans ces établissements, c'est la
situation la plus favorable qui est retenue (passage en deuxième année
puis redoublement puis réorientation).

\textbf{Redoublement (numérateur)} : étudiants de la base (dénominateur)
réinscrits en première année de licence pendant l'année 2015-2016, quel
que soit l'établissement d'accueil.

\textbf{Réorientation en DUT (numérateur)} : étudiants de la base
(dénominateur) inscrits en première ou deuxième année de DUT pendant
l'année 2015-2016 (passage en DUT après 1 an) ou pendant l'année
2016-2017 (passage en DUT après 2 ans), quel que soit l'établissement
d'accueil. Obtention de la licence en trois ou quatre ans

Le champ des indicateurs est constitué des bacheliers 2012 inscrits en
2012-2013 en première année de licence dans l'enseignement supérieur
public. En sont exclus les étudiants ayant pris une inscription
parallèle en STS, DUT ou CPGE.

L'indicateur est calculé en rapportant des effectifs d'étudiants :

\textbf{Base (dénominateur)} : étudiants du champ inscrit en première
année de licence pendant l'année 2012-2013 ;

\textbf{Réussite en 3 ans (numérateur)} : étudiants de la base
(dénominateur) ayant obtenu le diplôme de la licence (éventuellement
professionnelle) à la session 2015, que ce soit ou non dans
l'établissement ou la discipline de son inscription en première année ;

\textbf{Réussite en 4 ans (numérateur)} : étudiants de la base
(dénominateur) ayant obtenu le diplôme de la licence (éventuellement
professionnelle) à la session 2016 (sans l'avoir obtenu à la session
2015), que ce soit ou non dans l'établissement ou la discipline de son
inscription en première année.


\vspace{0.5cm}
\needspace{12\baselineskip}
\subsection*{Participations dans les contrats signés du programme-cadre recherche et
développement technologique (7ème PCRDT) de la Commission européenne
}\index{europe}\index{h2020}\index{pcrdt}\index{recherche}\index{recherche!et!developpement}
  \begin{wrapfigure}{r}{2.5cm}
    \centering
    \qrcode[nolink]{https://data.gouv.fr/dataset/56603b3fc751df601daad375}
  \end{wrapfigure}

Licence : \textbf{Licence Ouverte
}\newline
Créé le : 2015-12-03\newline
Modifié le : 2015-12-03\newline
De 2014-01-01 à 2015-07-15\newline
Granularité : au pays\newline
Mise à jour : ponctuelle\newline
Popularité : 1 réutilisation,  1 suivi\newline
Mots-clé : \emph{europe, h2020, pcrdt, recherche, recherche-et-developpement
}\newline
Permalien : \url{https://data.gouv.fr/dataset/56603b3fc751df601daad375}\newline

\par
\noindent
    Ce jeu de données contient les subventions allouées par la Commission
européenne aux participants au PCRDT à la date du 6 octobre 2014.

Données complémentaires dans Contrats signés du programme-cadre
recherche et développement technologique (7ème PCRDT) de la Commission
européenne


\vspace{0.5cm}
\needspace{12\baselineskip}
\subsection*{Participations dans les contrats signés du programme-cadre recherche et
développement technologique (H2020) de la Commission européenne
}\index{europe}\index{h2020}\index{pcrdt}\index{recherche}\index{recherche!et!developpement}
  \begin{wrapfigure}{r}{2.5cm}
    \centering
    \qrcode[nolink]{https://data.gouv.fr/dataset/566011d9c751df37c1aad371}
  \end{wrapfigure}

Licence : \textbf{Licence Ouverte
}\newline
Créé le : 2015-12-03\newline
Modifié le : 2016-03-24\newline
De 2014-01-01 à 2015-07-15\newline
Granularité : au pays\newline
Mise à jour : ponctuelle\newline
Popularité : 1 réutilisation,  1 suivi\newline
Mots-clé : \emph{europe, h2020, pcrdt, recherche, recherche-et-developpement
}\newline
Permalien : \url{https://data.gouv.fr/dataset/566011d9c751df37c1aad371}\newline

\par
\noindent
    Ce jeu de données contient les subventions allouées par la Commission
européenne aux participants au PCRDT à la date du 15 juillet 2015.

Données complémentaires dans Contrats signés du programme-cadre
recherche et développement technologique (H2020) de la Commission
européenne

Ces données sont reprises dans les tableau de bord du site
\href{http://www.horizon2020.gouv.fr/cid91235/donnees-statistiques-horizon-2020.html}{Horizon
2020} du ministère en charge de l'Enseignement supérieur et de la
Recherche.


\vspace{0.5cm}
\needspace{12\baselineskip}
\subsection*{Principaux établissements d'enseignement supérieur
}\index{comue}\index{ecole}\index{enseignement!superieur}\index{etablissement}\index{formation}\index{geolocalisation}\index{universite}
  \begin{wrapfigure}{r}{2.5cm}
    \centering
    \qrcode[nolink]{https://data.gouv.fr/dataset/53699dbfa3a729239d205ceb}
  \end{wrapfigure}

Licence : \textbf{Licence Ouverte
}\newline
Créé le : 2014-04-23\newline
Modifié le : 2016-03-04\newline
De 2015-01-01 à 2015-01-01\newline
Granularité : au point d'intérêt\newline
Mise à jour : ponctuelle\newline
Popularité : 1 réutilisation,  2 suivis\newline
Mots-clé : \emph{comue, ecole, enseignement-superieur, etablissement, formation, geolocalisation, universite
}\newline
Permalien : \url{https://data.gouv.fr/dataset/53699dbfa3a729239d205ceb}\newline

\par
\noindent
    Caractéristiques des principaux établissements d'enseignement supérieur.
Situation au 1er janvier 2015.


\vspace{0.5cm}
\needspace{12\baselineskip}
\subsection*{Programme de la fête de la Science 2016
}\index{agenda}\index{animation}\index{animations}\index{culture}\index{debat}\index{debats}\index{education}\index{education!artistique!et!culturel}\index{evenement}\index{evenements}\index{exposition}\index{expositions}\index{fete}\index{initiative!locale}\index{initiatives}\index{recherche}\index{science}\index{societe}
  \begin{wrapfigure}{r}{2.5cm}
    \centering
    \qrcode[nolink]{https://data.gouv.fr/dataset/57adea70c751df6e1a97bae5}
  \end{wrapfigure}

Licence : \textbf{Licence Ouverte
}\newline
Créé le : 2016-08-12\newline
Modifié le : 2016-08-12\newline
Granularité : au point d'intérêt\newline
Mise à jour : ponctuelle\newline
Popularité : 2 réutilisations,  0 suivi\newline
Mots-clé : \emph{agenda, animation, animations, culture, debat, debats, education, education-artistique-et-culturel, evenement, evenements, exposition, expositions, fete, initiative-locale, initiatives, recherche, science, societe
}\newline
Permalien : \url{https://data.gouv.fr/dataset/57adea70c751df6e1a97bae5}\newline

\par
\noindent
    Ce jeu de données présente l'ensemble des événements organisés partout
en France pour la 26e édition de la Fête de la science qui aura lieu du
8 au 16 octobre 2016 : animations, expositions, débats et d'initiatives
scientifiques gratuites, inventives et ludiques.

Il alimente le site national de la Fête de la science.


\vspace{0.5cm}
\needspace{12\baselineskip}
\subsection*{Publications scientifiques recensées dans scanR
}\index{archives!ouvertes}\index{hal}\index{prodinra}\index{publications}\index{publications!scientifiques}\index{recherche}\index{scanr}
  \begin{wrapfigure}{r}{2.5cm}
    \centering
    \qrcode[nolink]{https://data.gouv.fr/dataset/577cc85ac751df4e1b9901a0}
  \end{wrapfigure}

Licence : \textbf{Licence Ouverte
}\newline
Créé le : 2016-07-06\newline
Modifié le : 2016-07-06\newline
Mise à jour : ponctuelle\newline
Popularité : 1 réutilisation,  0 suivi\newline
Mots-clé : \emph{archives-ouvertes, hal, prodinra, publications, publications-scientifiques, recherche, scanr
}\newline
Permalien : \url{https://data.gouv.fr/dataset/577cc85ac751df4e1b9901a0}\newline

\par
\noindent
    Ce jeu de données présente les publications scientifiques des structures
de recherche publiques recensées dans l'application scanR. Ces
publications scientifiques sont extraites des archives HAL et ProdInra.


\vspace{0.5cm}
\needspace{12\baselineskip}
\subsection*{Publications statistiques sur l'enseignement supérieur et la recherche
}\index{education}\index{enseignement}\index{enseignement!formation!recherche}\index{enseignement!superieur}\index{innovation}\index{recherche}
  \begin{wrapfigure}{r}{2.5cm}
    \centering
    \qrcode[nolink]{https://data.gouv.fr/dataset/5665ff71c751df71e6c664c2}
  \end{wrapfigure}

Licence : \textbf{Licence Ouverte
}\newline
Créé le : 2015-12-07\newline
Modifié le : 2016-01-18\newline
Granularité : au point d'intérêt\newline
Mise à jour : ponctuelle\newline
Popularité : 1 réutilisation,  0 suivi\newline
Mots-clé : \emph{education, enseignement, enseignement-formation-recherche, enseignement-superieur, innovation, recherche
}\newline
Permalien : \url{https://data.gouv.fr/dataset/5665ff71c751df71e6c664c2}\newline

\par
\noindent
    Liste et lien vers les publications statistiques sur l'enseignement
supérieur et la recherche éditées par les services statistiques
ministériels du ministère de l'Éducation nationale, de l'Enseignement
supérieur et de la Recherche depuis 1999.

Cette liste est utilisé dans le
\href{http://publication.enseignementsup-recherche.gouv.fr/}{moteur de
recherche des publications statistiques de l'Enseignement supérieur et
de la recherche}.


\vspace{0.5cm}
\needspace{12\baselineskip}
\subsection*{Référentiel géographique français, communes, unités urbaines, aires
urbaines, départements, académies, régions
}\index{academie}\index{academies}\index{aires!urbaines}\index{commune}\index{communes}\index{communes!rurales}\index{departements}\index{geographie}\index{geographique}\index{referentiel}\index{region}\index{regions}\index{unites!urbaines}
  \begin{wrapfigure}{r}{2.5cm}
    \centering
    \qrcode[nolink]{https://data.gouv.fr/dataset/56ebf05fc751df7756cc7140}
  \end{wrapfigure}

Licence : \textbf{Licence Ouverte
}\newline
Créé le : 2016-03-18\newline
Modifié le : 2016-03-18\newline
Granularité : à la commune\newline
Mise à jour : ponctuelle\newline
Popularité : 1 réutilisation,  3 suivis\newline
Mots-clé : \emph{academie, academies, aires-urbaines, commune, communes, communes-rurales, departements, geographie, geographique, referentiel, region, regions, unites-urbaines
}\newline
Permalien : \url{https://data.gouv.fr/dataset/56ebf05fc751df7756cc7140}\newline

\par
\noindent
    Ce jeu de données fournit les différents niveaux d'agrégats
géographiques associés aux communes françaises (situation au 1er janvier
2015).

Les champs préfixés \_CODE correspondent aux identifiants géographiques
officiels sans préfixe. Les valeurs des champs préfixes \_ID contiennent
un préfixe pour signifier le niveau géographique associé : R : régions A
: académies D : départements C : communes FD : France détaillée FR :
France regroupée FE : France entière ZE : zone d'emploi UU : unités
urbaines CR : communes rurales AU : aires urbaines UUCR : unités
urbaines et communes rurales AUC : aires urbaines et communes


\vspace{0.5cm}
\needspace{12\baselineskip}
\subsection*{Référentiel métiers RéFérens III pour la filière des ITRF
}\index{administration}\index{emploi}\index{emploi!type}\index{enseignement!superieur}\index{ingenieur}\index{metier}\index{metiers}\index{public}\index{recherche}\index{referens3}\index{referentiel}\index{ressources!humaines}\index{secteur!public}\index{technicien}
  \begin{wrapfigure}{r}{2.5cm}
    \centering
    \qrcode[nolink]{https://data.gouv.fr/dataset/58123657c751df389fc562c8}
  \end{wrapfigure}

Licence : \textbf{Licence Ouverte
}\newline
Créé le : 2016-10-27\newline
Modifié le : 2016-10-27\newline
De 2017-01-01 à 2019-12-31\newline
Mise à jour : ponctuelle\newline
Popularité : 1 réutilisation,  0 suivi\newline
Mots-clé : \emph{administration, emploi, emploi-type, enseignement-superieur, ingenieur, metier, metiers, public, recherche, referens3, referentiel, ressources-humaines, secteur-public, technicien
}\newline
Permalien : \url{https://data.gouv.fr/dataset/58123657c751df389fc562c8}\newline

\par
\noindent
    Ce jeu de données reprend les informations et nomenclatures concernant
les 242 métiers des ingénieurs et personnels techniques de recherche et
de formation (ITRF) et des ingénieurs et personnels techniques de la
recherche (ITA) décrits dans le répertoire des branches d'activités
professionnelles et des emplois-types, dénommé RéFérens (REFérentiel des
Emplois-types de la recherche et de l'Enseignement Supérieur).

Ce jeu de données alimente le
\href{http://referens.enseignementsup-recherche.gouv.fr/}{site dédié à
ce référentiel}.

Il est associé à un autre jeu de données qui présente les
\href{https://www.data.gouv.fr/fr/datasets/dictionnaire-des-competences-referens-iii-pour-la-filiere-des-itrf/}{entrées
du dictionnaire des compétences} qui décrit les connaissances et
compétences liés aux emplois-types de RéFérens III.


\vspace{0.5cm}
\needspace{12\baselineskip}
\subsection*{Répertoire national des structures de recherche
}\index{laboratoire}\index{laboratoires}\index{repertoire!national!des!structur}\index{rnsr}
  \begin{wrapfigure}{r}{2.5cm}
    \centering
    \qrcode[nolink]{https://data.gouv.fr/dataset/577cbccbc751df389c9901a0}
  \end{wrapfigure}

Licence : \textbf{Licence Ouverte
}\newline
Créé le : 2016-07-06\newline
Modifié le : 2016-07-06\newline
Granularité : au point d'intérêt\newline
Mise à jour : ponctuelle\newline
Popularité : 1 réutilisation,  0 suivi\newline
Mots-clé : \emph{laboratoire, laboratoires, repertoire-national-des-structur, rnsr
}\newline
Permalien : \url{https://data.gouv.fr/dataset/577cbccbc751df389c9901a0}\newline

\par
\noindent
    Ce jeu de données présente les structures de recherche publiques,
actives ou inactives, référencées dans le répertoire national des
structures de recherche (RNSR).


\vspace{0.5cm}
\needspace{12\baselineskip}
\subsection*{Répertoire national des structures de recherche - Historique annuel
}\index{laboratoire}\index{laboratoires}\index{recherche}\index{repertoire!national!des!structur}\index{rnsr}\index{structure!de!recherche}
  \begin{wrapfigure}{r}{2.5cm}
    \centering
    \qrcode[nolink]{https://data.gouv.fr/dataset/577cb9ccc751df340c9901a0}
  \end{wrapfigure}

Licence : \textbf{Licence Ouverte
}\newline
Créé le : 2016-07-06\newline
Modifié le : 2016-07-06\newline
Granularité : au point d'intérêt\newline
Mise à jour : ponctuelle\newline
Popularité : 1 réutilisation,  0 suivi\newline
Mots-clé : \emph{laboratoire, laboratoires, recherche, repertoire-national-des-structur, rnsr, structure-de-recherche
}\newline
Permalien : \url{https://data.gouv.fr/dataset/577cb9ccc751df340c9901a0}\newline

\par
\noindent
    Ce jeu de données présente l'historique annuel depuis 1990 des
structures de recherche publiques référencées dans le répertoire
national des structures de recherche (RNSR). Il inclut les changements
potentiels de tutelle au cours de l'existence d'une structure de
recherche.


\vspace{0.5cm}
\needspace{12\baselineskip}
\subsection*{Ressources pédagogiques numériques
}\index{ecole}\index{enseignement}\index{enseignement!formation!recherche}\index{enseignement!superieur}\index{formations}\index{recherche}\index{ressources!pedagogiques}\index{universite}\index{unt}
  \begin{wrapfigure}{r}{2.5cm}
    \centering
    \qrcode[nolink]{https://data.gouv.fr/dataset/55718162c751df73d1e5726d}
  \end{wrapfigure}

Licence : \textbf{Licence Ouverte
}\newline
Créé le : 2015-06-05\newline
Modifié le : 2015-12-01\newline
De 1996-05-22 à 2015-10-22\newline
Mise à jour : ponctuelle\newline
Popularité : 1 réutilisation,  2 suivis\newline
Mots-clé : \emph{ecole, enseignement, enseignement-formation-recherche, enseignement-superieur, formations, recherche, ressources-pedagogiques, universite, unt
}\newline
Permalien : \url{https://data.gouv.fr/dataset/55718162c751df73d1e5726d}\newline

\par
\noindent
    \begin{verbatim}
Indexation des ressources pédagogiques numériques du moteur national&nbsp;<a href=[http://www.sup-numerique.gouv.fr/pid33288/moteur-des-ressources-pedagogiques.html](http://www.sup-numerique.gouv.fr/pid33288/moteur-des-ressources-pedagogiques.html) target="_blank"[http://www.sup-numerique.gouv.fr/pid33288/moteur-des-ressources-pedagogiques.html/</a>](http://www.sup-numerique.gouv.fr/pid33288/moteur-des-ressources-pedagogiques.html/</a>)</p>
<p>
    Ressources proposées par les établissements d'enseignement supérieur et les organismes de recherche français et diffusées par les UNT :
</p>
\end{verbatim}

\begin{verbatim}
<li><a href=[http://www.aunege.fr/](http://www.aunege.fr/) target="_blank">Aunege</a></li>
<li><a href=[http://www.iutenligne.net/](http://www.iutenligne.net/) target="_blank">IUTenLigne</a></li>
<li><a href=[http://www.unf3s.org/](http://www.unf3s.org/) target="_blank">unf3S</a></li>
<li><a href=[http://www.unisciel.fr/](http://www.unisciel.fr/) target="_blank">Unisciel</a></li>
<li><a href=[http://www.unit.eu/fr](http://www.unit.eu/fr) target="_blank">Unit</a></li>
<li><a href=[http://www.unjf.fr/](http://www.unjf.fr/) target="_blank">Unjf</a></li>
<li><a href=[http://www.uoh.fr/front](http://www.uoh.fr/front) target="_blank">UOH</a></li>
<li><a href=[http://www.uved.fr/navigation/accueil.html](http://www.uved.fr/navigation/accueil.html) target="_blank">UVED</a></li>
\end{verbatim}

\begin{verbatim}
<a href=[http://www.uved.fr/navigation/accueil.html](http://www.uved.fr/navigation/accueil.html) target="_blank"></a>
\end{verbatim}

\begin{verbatim}
Et
\end{verbatim}

\begin{verbatim}
<li><a href=[https://www.canal-u.tv/](https://www.canal-u.tv/) target="_blank">Canal-U</a></li>
\end{verbatim}

\begin{verbatim}
<p>
    Moteur basé sur le&nbsp;<a href=[http://www.ori-oai.org/](http://www.ori-oai.org/) target="_blank">projet ORI-OAI</a> et développé en collaboration avec l'<a href=[http://www.univ-valenciennes.fr/](http://www.univ-valenciennes.fr/)>Université de Valenciennes et du Hainaut Cambrésis</a>
</p>
<hr id="horizontalrule">
\end{verbatim}


\vspace{0.5cm}
\needspace{12\baselineskip}
\subsection*{Services de documentation et sièges des bibliothèques de l'Enseignement
supérieur
}\index{administration}\index{annuaire!des!bibliotheques!de!lenseignement!superieur}\index{bibliotheque}\index{bibliotheque!de!depot}\index{bibliotheque!de!grand!etablissement}\index{bibliotheque!decole!francaise!a!letranger}\index{bibliotheque!inter!etablissements}\index{bibliotheque!interuniversitaire}\index{bibliotheques}\index{biu}\index{centre!de!formation!aux!carrieres!des!bibliotheques}\index{citoyennete}\index{coordonnees!geographiques}\index{crfcb}\index{culture}\index{direction}\index{direction!de!la!documentation}\index{education}\index{enseignement}\index{esgbu}\index{etablissement!public!a!caractere!administratif}\index{finances!publiques}\index{formation}\index{geolocalisation}\index{gouvernement}\index{grand!etablissement}\index{grands!etablissements}\index{groupement!dinteret!public}\index{ibliotheque!universitaire}\index{operateur!national}\index{patrimoine}\index{recherche}\index{reseau!documentaire}\index{reseaux!sociaux}\index{scd}\index{service!commun!de!documentation}\index{service!interetablissement!de!cooperation!documentaire}\index{service!interetablissement!de!documentation}\index{sicd}\index{unite!regionale!de!formation!a!linformation!scientifique!et!technique}\index{universite}\index{universites}\index{urfist}\index{url}
  \begin{wrapfigure}{r}{2.5cm}
    \centering
    \qrcode[nolink]{https://data.gouv.fr/dataset/5b120868b5950870b30303f0}
  \end{wrapfigure}

Licence : \textbf{Licence Ouverte version 2.0
}\newline
Créé le : 2018-06-02\newline
Modifié le : 2019-03-17\newline
Popularité : 1 réutilisation,  0 suivi\newline
Mots-clé : \emph{administration, annuaire-des-bibliotheques-de-lenseignement-superieur, bibliotheque, bibliotheque-de-depot, bibliotheque-de-grand-etablissement, bibliotheque-decole-francaise-a-letranger, bibliotheque-inter-etablissements, bibliotheque-interuniversitaire, bibliotheques, biu, centre-de-formation-aux-carrieres-des-bibliotheques, citoyennete, coordonnees-geographiques, crfcb, culture, direction, direction-de-la-documentation, education, enseignement, esgbu, etablissement-public-a-caractere-administratif, finances-publiques, formation, geolocalisation, gouvernement, grand-etablissement, grands-etablissements, groupement-dinteret-public, ibliotheque-universitaire, operateur-national, patrimoine, recherche, reseau-documentaire, reseaux-sociaux, scd, service-commun-de-documentation, service-interetablissement-de-cooperation-documentaire, service-interetablissement-de-documentation, sicd, unite-regionale-de-formation-a-linformation-scientifique-et-technique, universite, universites, urfist, url
}\newline
Permalien : \url{https://data.gouv.fr/dataset/5b120868b5950870b30303f0}\newline

\par
\noindent
    Ce jeu de données présentent les informations sur les services de
documentation et autres sièges des~bibliothèques de l'enseignement
supérieur, les centres de formation aux carrières des bibliothèques
(CRFCB) et~les unités régionales de formation à l'information
scientifique et technique (URFIST).

Ces informations alimentent l'annuaire des bibliothèques de
l'enseignement supérieur
:~\url{http://bibliotheques.enseignementsup-recherche.gouv.fr/FR/annuaire/\%3E}


\vspace{0.5cm}
\needspace{3\baselineskip} \rule{4cm}{0.25pt}\newline\textbf{Aussi disponible du même producteur :}\begin{itemize}
\item \href{https://data.gouv.fr/dataset/5bf8c0f606e3e708781a69b5}{Agenda public des ministres en charge de l'Enseignement supérieur, de la Recherche et de l'Innovation}
\item \href{https://data.gouv.fr/dataset/586dae6ca3a7290df6f4be97}{Appels à projets 7ème PCRDT - Projets retenus et participants identifiés}
\item \href{https://data.gouv.fr/dataset/577cc0a4c751df40e49901a0}{Appels à projets 7ème PCRDT - Projets retenus et participants identifiés}
\item \href{https://data.gouv.fr/dataset/586dae6ca3a7290df5f4be80}{Appels à projets ANR - Projets retenus et participants identifiés}
\item \href{https://data.gouv.fr/dataset/59376c4ca3a72968e287731f}{Appels à projets Horizon 2020 - Projets retenus et participants identifiés}
\item \href{https://data.gouv.fr/dataset/58bf73bfa3a7293afeefbf92}{[archive] Initiatives pour la lutte contre les violences sexistes et sexuelles}
\item \href{https://data.gouv.fr/dataset/586dae78a3a7290df5f4be8d}{Bibliothèques universitaires qui étendent leurs horaires dans le cadre du plan Bibliothèques ouvertes}
\item \href{https://data.gouv.fr/dataset/57db148888ee383cd95ff490}{Budget de recherche et de transfert de  technologie (R\&{}T) des collectivités territoriales }
\item \href{https://data.gouv.fr/dataset/586dae74a3a7290df5f4be8a}{Budget de recherche et de transfert de technologie (R\&{}T) des collectivités territoriales}
\item \href{https://data.gouv.fr/dataset/57db115088ee3802f55ff490}{Candidats et lauréats du prix PEPS - Passion Enseignement et Pédagogie dans le Supérieur}
\item \href{https://data.gouv.fr/dataset/586dae5ea3a7290df6f4be87}{Candidats et lauréats du prix PEPS - Passion Enseignement et Pédagogie dans le Supérieur}
\item \href{https://data.gouv.fr/dataset/586dae61a3a7290df6f4be8c}{Cartographie de l'enseignement supérieur et de la recherche}
\item \href{https://data.gouv.fr/dataset/5bfcb5e106e3e72ddb04ccef}{Cartographie des cellules de lutte contre les violences sexistes et sexuelles}
\item \href{https://data.gouv.fr/dataset/5c67a4a606e3e706e97bb84a}{Composition des regroupements issus de la loi n\degree{} 2013-660 du 22 juillet 2013 relative à l'enseignement supérieur et à la recherche}
\item \href{https://data.gouv.fr/dataset/586dae68a3a7290df6f4be93}{Contrats signés du programme-cadre cadre pour la recherche et l’innovation (H2020) de la Commission européenne}
\item \href{https://data.gouv.fr/dataset/586dae6ea3a7290df5f4be82}{Dates des élections étudiantes aux conseils d'administration des CROUS et localisation des bureaux de vote}
\item \href{https://data.gouv.fr/dataset/586dae6ba3a7290df6f4be96}{Dictionnaire des compétences - RéFérens III pour la filière des ITRF}
\item \href{https://data.gouv.fr/dataset/586dae73a3a7290df5f4be88}{Diplômes délivrés dans les établissements publics sous tutelle du ministère en charge de l'enseignement supérieur et de la recherche}
\item \href{https://data.gouv.fr/dataset/586dae5fa3a7290df6f4be89}{Écoles supérieures du professorat et de l'éducation : Concours d'enseignement visés}
\item \href{https://data.gouv.fr/dataset/586dae5fa3a7290df5f4be73}{Écoles supérieures du professorat et de l'éducation : Implantations et parcours MEEF proposés}
\item \href{https://data.gouv.fr/dataset/586dae60a3a7290df5f4be75}{Écoles supérieures du professorat et de l'éducation : Parcours des  masters Métiers de l'enseignement, de l'éducation et de la formation}
\item \href{https://data.gouv.fr/dataset/586dae72a3a7290df6f4be9e}{Effectifs d'étudiants inscrits dans les établissements et les formations de l'enseignement supérieur}
\item \href{https://data.gouv.fr/dataset/586dae6fa3a7290df6f4be9a}{Effectifs d'étudiants inscrits dans les établissements publics sous tutelle du ministère en charge de l'Enseignement supérieur}
\item \href{https://data.gouv.fr/dataset/56ebf37ac751df73c0cc7140}{English/French glossary on higher education \&{} research in France}
\item \href{https://data.gouv.fr/dataset/586dae74a3a7290df6f4bea0}{Ensemble des lieux de restauration des CROUS}
\item \href{https://data.gouv.fr/dataset/586dae74a3a7290df6f4bea1}{Ensemble des logements proposés aux étudiants par le réseau des CROUS}
\item \href{https://data.gouv.fr/dataset/586dae6aa3a7290df6f4be95}{Établissements publics et privés impliqués dans la recherche et développement}
\item \href{https://data.gouv.fr/dataset/577ccc40c751df56389901a0}{Établissements publics et privés impliqués dans la recherche et le développement}
\item \href{https://data.gouv.fr/dataset/5a6a9bf3b59508044178dc80}{Experts scientifiques ou techniques privés et stylistes agréés CIR}
\item \href{https://data.gouv.fr/dataset/5afbbb86b5950846ea734620}{Feuille de route des infrastructures de recherche 2018}
\item \href{https://data.gouv.fr/dataset/5afbbb9ab5950846ea734621}{Feuille de route des infrastructures de recherche 2018 - Les différentes implantations}
\item \href{https://data.gouv.fr/dataset/586dae69a3a7290df5f4be7e}{Finalistes et lauréats du concours Ma Thèse en 180 secondes France}
\item \href{https://data.gouv.fr/dataset/58759fb5a3a7291596b43543}{Implantations des établissements d'enseignement supérieur publics}
\item \href{https://data.gouv.fr/dataset/586dae70a3a7290df6f4be9b}{Indicateurs des contrats du 7ème PCRDT par pays 2007-2013}
\item \href{https://data.gouv.fr/dataset/53699691a3a729239d204a86}{Indicateurs des contrats du 7ème PCRDT par pays 2007-2013}
\item \href{https://data.gouv.fr/dataset/586dae73a3a7290df6f4be9f}{Indicateurs des contrats H2020 par pays 2014-2020}
\item \href{https://data.gouv.fr/dataset/56601aa9c751df37c1aad374}{Indicateurs des contrats H2020 par pays 2014-2020}
\item \href{https://data.gouv.fr/dataset/586dae79a3a7290df5f4be8e}{Insertion professionnelle des diplômés de Diplôme universitaire de technologie (DUT) en universités et établissements assimilés - données nationales par disciplines détaillées}
\item \href{https://data.gouv.fr/dataset/586dae6ea3a7290df6f4be99}{Insertion professionnelle des diplômés de Licence professionnelle en universités et établissements assimilés}
\item \href{https://data.gouv.fr/dataset/586dae5ea3a7290df5f4be72}{Insertion professionnelle des diplômés de Master en universités et établissements assimilés}
\item \href{https://data.gouv.fr/dataset/586dae65a3a7290df6f4be90}{Insertion professionnelle des diplômés de Master en universités et établissements assimilés - données nationales par disciplines détaillées}
\item \href{https://data.gouv.fr/dataset/586dae70a3a7290df5f4be85}{Institutions partenaires des pôles étudiants pour l'innovation, le transfert et l'entrepreneuriat (PEPITE)}
\item \href{https://data.gouv.fr/dataset/53699725a3a729239d204c00}{Institutions partenaires des pôles étudiants pour l'innovation, le transfert et l'entrepreneuriat (PEPITE)}
\item \href{https://data.gouv.fr/dataset/586dae5da3a7290df5f4be70}{Journées des Arts et de la Culture dans l'Enseignement Supérieur}
\item \href{https://data.gouv.fr/dataset/5845abf4c751df7bc0c0bb7e}{Lauréat-e-s du trophée Les Étoiles de l'Europe}
\item \href{https://data.gouv.fr/dataset/586dae78a3a7290df6f4bea4}{Lauréat-e-s du trophée Les Étoiles de l'Europe}
\item \href{https://data.gouv.fr/dataset/5a98be42b595087d2a84c001}{Lauréates et lauréats du prix PÉPITE-Tremplin pour l’Entrepreneuriat Étudiant}
\item \href{https://data.gouv.fr/dataset/586dae67a3a7290df6f4be92}{Lauréats I-LAB Concours national d'aide à la création d'entreprises de technologies innovantes}
\item \href{https://data.gouv.fr/dataset/5b15fd22a3a7290eb6758880}{Les écoles doctorales - Historique annuel}
\item \href{https://data.gouv.fr/dataset/58bf73d5a3a7293afeefbf93}{Les enseignants non permanents des établissements publics de l'enseignement supérieur}
\item et 49 autres jeux de données\end{itemize}

\clearpage
\section{Ministère de l'Europe et des Affaires Etrangères (MEAE)}


\begin{center}
  \includegraphics[width=3cm]{images/orga/ca_203d4974a84ab8b1f8e88592abeff2-100.jpg}
\end{center}


Le ministère de l'Europe et des Affaires étrangères (MEAE) :

\begin{itemize}
\item
  est au service des près de 2 millions de Français vivant à l'étranger
\item
  conduit l'action diplomatique de la France en Europe et dans le monde
\item
  informe le président de la République et le Premier ministre de la
  situation en Europe et dans le monde
\item
  défend les intérêts politiques et économiques de notre pays sur la
  scène internationale, favorise son rayonnement culturel et
  scientifique et mène des actions de coopération
\end{itemize}


\vspace{0.5cm}

\needspace{12\baselineskip}
\subsection*{Corpus de documents relatif aux négociations commerciales
internationales en cours (TTIP, TiSA et CETA)
}\index{acs}\index{ceta}\index{commerce}\index{commerce!exterieur}\index{diplomatie}\index{europe}\index{geopolitique}\index{international}\index{omc}\index{tisa}\index{ttip}
  \begin{wrapfigure}{r}{2.5cm}
    \centering
    \qrcode[nolink]{https://data.gouv.fr/dataset/54802738c751df2156e127c2}
  \end{wrapfigure}

Licence : \textbf{Licence Ouverte
}\newline
Créé le : 2014-12-04\newline
Modifié le : 2016-02-27\newline
Mise à jour : ponctuelle\newline
Popularité : 1 réutilisation,  0 suivi\newline
Mots-clé : \emph{acs, ceta, commerce, commerce-exterieur, diplomatie, europe, geopolitique, international, omc, tisa, ttip
}\newline
Permalien : \url{https://data.gouv.fr/dataset/54802738c751df2156e127c2}\newline

\par
\noindent
    Cette fiche présente un corpus de documents relatifs aux négociations du
Partenariat Transatlantique de Commerce et d'Investissement (TTIP), de
l'Accord économique et commercial global (CETA) et de l'Accord sur le
commerce des services (TISA).

\begin{itemize}

\item
  Le Partenariat Transatlantique de Commerce et d'Investissement (TTIP)
\end{itemize}

Le 14 juin 2013, le Conseil de l'Union Européenne, qui réunit les chefs
d'Etat et de gouvernement des Etats membres de l'Union, a confié à la
Commission Européenne un mandat pour mener les négociations avec les
États-Unis en vue d'aboutir à un accord transatlantique de commerce et
d'investissements. Plusieurs dénominations désignent ce projet de
Partenariat transatlantique qu'elles soient anglophones (TAFTA - Trans
Atlantic Free Trade agreement ; TTIP - Transatlantic Trade and
Investment Partnership) ou francophones (PTCI - Partenariat
transatlantique sur le commerce et l'investissement). Ces négociations
sont organisées autour de trois piliers : accès aux marchés (biens
agricoles et industriels, services, marchés publics) ; barrières au
commerce de nature non-tarifaire, mesures sanitaires et phytosanitaires
et convergence réglementaire ; règles (propriété intellectuelle et
indications géographiques, énergie et matières premières, concurrence,
règles d'origine, facilitation des échanges, développement durable).

\begin{itemize}

\item
  L'Accord économique et commercial global (CETA)
\end{itemize}

Le 18 octobre 2013 ont pris fin les négociations de l'accord économique
et commercial global (AECG) (Comprehensive Economic and Trade Agreement
-- CETA en anglais) entre l'UE et le Canada. Afin de favoriser le
commerce entre l'Union et le Canada, l'objectif de la négociation a été,
notamment, d'obtenir la reconnaissance de nos indications géographiques,
de permettre l'augmentation de nos exportations agricoles, d'abaisser de
nombreuses barrières au commerce de nature non-tarifaire, de faciliter
des investissements croisés, d'obtenir une meilleure protection de la
propriété intellectuelle. A la demande de chaque partie, un certain
nombre de matières ne sont pas concernées par l'accord, telles que les
services audiovisuels ou les législations de protection de la santé et
du consommateur en matière alimentaire (par exemple : OGM, viande aux
hormones). Le projet de l'Accord a été mis en ligne par la Commission
européenne et va désormais être traduit dans les langues des 28 États
membres de l'Union européenne avant d'être soumis au Conseil et au
Parlement européen, puis aux Parlements nationaux.

\begin{itemize}

\item
  L'Accord sur le Commerce des Services (ACS - TiSA)
\end{itemize}

Le 18 mars 2013, le Conseil de l'Union européenne, qui réunit les
ministres des gouvernements des Etats membres, a confié à la Commission
Européenne un mandat pour mener les négociations relatives à l'Accord
sur le commerce des services (Trade in Services Agreement -- TiSA en
anglais). Ces négociations réunissent 25 membres de l'Organisation
mondiale du commerce (OMC), dont l'Union Européenne, soit 52 pays qui
représentent au total 70 \% du commerce mondial des services. Elles
visent à obtenir l'ouverture des marchés des services encore fermés aux
entreprises européennes et à améliorer les règles relatives au commerce
transnational, qu'elles soient thématiques (mouvement temporaire de
travailleurs, règlementation intérieure) ou sectorielles (services de
transport maritime, services de transport aérien, services de transport
routier, services financiers, services professionnels, services de
télécommunications, commerce électronique et services des technologies
de l'information et de la communication).


\vspace{0.5cm}
\needspace{12\baselineskip}
\subsection*{Elections présidentielles 1995 (1er tour) : vote des Français établis
hors de France
}\index{elections}\index{francais!de!l!etranger}\index{francais!etablis!hors!de!france}\index{vote}
  \begin{wrapfigure}{r}{2.5cm}
    \centering
    \qrcode[nolink]{https://data.gouv.fr/dataset/536993a0a3a729239d20424a}
  \end{wrapfigure}

Licence : \textbf{Licence Ouverte
}\newline
Créé le : 2013-10-13\newline
Modifié le : 2015-08-25\newline
De 1995-04-23 à 1995-04-23\newline
Granularité : au pays\newline
Mise à jour : annuelle\newline
Popularité : 1 réutilisation,  0 suivi\newline
Mots-clé : \emph{elections, francais-de-l-etranger, francais-etablis-hors-de-france, vote
}\newline
Permalien : \url{https://data.gouv.fr/dataset/536993a0a3a729239d20424a}\newline

\par
\noindent
    Ce tableau présente les résultats du vote des Français établis hors de
France, à l'occasion du premier tour des élections présidentielles de
1995.


\vspace{0.5cm}
\needspace{12\baselineskip}
\subsection*{Elections présidentielles 2017 (1er tour) : vote des Français.es
établi.e.s hors de France
}\index{consulaire}\index{election}\index{elections}\index{francais!de!l!etranger}\index{francais!etablis!hors!de!france}\index{presidentielles}\index{vote}
  \begin{wrapfigure}{r}{2.5cm}
    \centering
    \qrcode[nolink]{https://data.gouv.fr/dataset/58ff562ec751df1ef29ae18d}
  \end{wrapfigure}

Licence : \textbf{Licence Ouverte
}\newline
Créé le : 2017-04-25\newline
Modifié le : 2017-06-08\newline
Mise à jour : ponctuelle\newline
Popularité : 1 réutilisation,  2 suivis\newline
Mots-clé : \emph{consulaire, election, elections, francais-de-l-etranger, francais-etablis-hors-de-france, presidentielles, vote
}\newline
Permalien : \url{https://data.gouv.fr/dataset/58ff562ec751df1ef29ae18d}\newline

\par
\noindent
    Retrouvez les résultats détaillés, par pays et par circonscription
consulaire, du vote Français résidant à l'étranger au premier tour de
l'élection présidentielle 2017.


\vspace{0.5cm}
\needspace{12\baselineskip}
\subsection*{Français de l'étranger : inscriptions au registre des Français établis
hors de France (2001-2013)
}\index{ambassade}\index{demarches}\index{election}\index{registre}
  \begin{wrapfigure}{r}{2.5cm}
    \centering
    \qrcode[nolink]{https://data.gouv.fr/dataset/536995caa3a729239d204804}
  \end{wrapfigure}

Licence : \textbf{Licence Ouverte
}\newline
Créé le : 2013-10-13\newline
Modifié le : 2019-01-29\newline
De 2001-01-01 à 2013-12-31\newline
Granularité : au pays\newline
Mise à jour : ponctuelle\newline
Popularité : 4 réutilisations,  6 suivis\newline
Mots-clé : \emph{ambassade, demarches, election, registre
}\newline
Permalien : \url{https://data.gouv.fr/dataset/536995caa3a729239d204804}\newline

\par
\noindent
    Ce jeu de données présente l'état, année par année, et pays par pays,
des inscriptions au registre des Français établis hors de France, de
2001 à 2013. L'inscription au Registre, démarche volontaire ouverte aux
Français résidant à l'étranger, s'effectue auprès du réseau des
ambassades et consulats français. L'inscription au registre n'est pas
obligatoire. On estime le nombre global de Français vivant à l'étranger,
y compris ceux qui ne sont pas inscrits au registre, entre 2 et 2,5
millions.


\vspace{0.5cm}
\needspace{12\baselineskip}
\subsection*{Géolocalisation des ambassades de France
}\index{ambassade}\index{ambassadeur}\index{diplomatie}\index{geolocalisation}
  \begin{wrapfigure}{r}{2.5cm}
    \centering
    \qrcode[nolink]{https://data.gouv.fr/dataset/536995f6a3a729239d204882}
  \end{wrapfigure}

Licence : \textbf{Licence Ouverte
}\newline
Créé le : 2013-07-08\newline
Modifié le : 2016-03-10\newline
Granularité : au point d'intérêt\newline
Popularité : 3 réutilisations,  7 suivis\newline
Mots-clé : \emph{ambassade, ambassadeur, diplomatie, geolocalisation
}\newline
Permalien : \url{https://data.gouv.fr/dataset/536995f6a3a729239d204882}\newline

\par
\noindent
    Ce tableau rassemble les coordonnées géolocalisées des ambassades de
France à l'étranger.


\vspace{0.5cm}
\needspace{12\baselineskip}
\subsection*{L'aide publique au développement de la France
}\index{aide}\index{aide!au!developpement}\index{cooperation}\index{developpement}\index{france}\index{ong}\index{projets}\index{transparence}
  \begin{wrapfigure}{r}{2.5cm}
    \centering
    \qrcode[nolink]{https://data.gouv.fr/dataset/53ae8ac1a3a729709f56d50b}
  \end{wrapfigure}

Licence : \textbf{Licence Ouverte
}\newline
Créé le : 2014-06-24\newline
Modifié le : 2018-09-11\newline
Popularité : 1 réutilisation,  3 suivis\newline
Mots-clé : \emph{aide, aide-au-developpement, cooperation, developpement, france, ong, projets, transparence
}\newline
Permalien : \url{https://data.gouv.fr/dataset/53ae8ac1a3a729709f56d50b}\newline

\par
\noindent
    Les jeux de données portent sur l'aide publique au développement
française. Elles seront systématiquement mises à jour pour prendre en
compte les nouveaux projets de développement financés par le ministère
de l'Europe et des Affaires étrangères et l'Agence française de
développement. Dans un esprit de transparence, ces données respectent le
format et le standard IATI (Initiative internationale pour la
transparence de l'aide). La plupart de ces données sont disponibles sur
le site unique\url{https://afd.opendatasoft.com/page/accueil/}


\vspace{0.5cm}
\needspace{12\baselineskip}
\subsection*{Les Alliances françaises dans le monde
}\index{cooperation!internationale}\index{organisation!culturelle}\index{promotion!des!echanges}
  \begin{wrapfigure}{r}{2.5cm}
    \centering
    \qrcode[nolink]{https://data.gouv.fr/dataset/536997efa3a729239d204e0c}
  \end{wrapfigure}

Licence : \textbf{Licence Ouverte
}\newline
Créé le : 2013-07-08\newline
Modifié le : 2016-02-15\newline
De 2012-01-01 à 2012-12-31\newline
Granularité : au point d'intérêt\newline
Mise à jour : ponctuelle\newline
Popularité : 1 réutilisation,  3 suivis\newline
Mots-clé : \emph{cooperation-internationale, organisation-culturelle, promotion-des-echanges
}\newline
Permalien : \url{https://data.gouv.fr/dataset/536997efa3a729239d204e0c}\newline

\par
\noindent
    Ce jeu de données contient les coordonnées des Alliances françaises dans
le monde : nom, zone géographique, pays (et éventuellement région ou
comté), ville, adresse, code téléphonique, numéro(s) de téléphone,
numéro(s) de télécopie, courriel(s), site(s) internet.


\vspace{0.5cm}
\needspace{12\baselineskip}
\subsection*{Les Espaces CampusFrance dans le monde
}\index{campusfrance}\index{education}\index{enseignement}\index{enseignement!superieur}
  \begin{wrapfigure}{r}{2.5cm}
    \centering
    \qrcode[nolink]{https://data.gouv.fr/dataset/53699839a3a729239d204ed0}
  \end{wrapfigure}

Licence : \textbf{Licence Ouverte
}\newline
Créé le : 2013-07-08\newline
Modifié le : 2016-01-16\newline
De 2012-01-01 à 2012-12-31\newline
Granularité : au pays\newline
Mise à jour : ponctuelle\newline
Popularité : 1 réutilisation,  3 suivis\newline
Mots-clé : \emph{campusfrance, education, enseignement, enseignement-superieur
}\newline
Permalien : \url{https://data.gouv.fr/dataset/53699839a3a729239d204ed0}\newline

\par
\noindent
    Ce jeu de données contient les coordonnées des Espaces CampusFrance dans
le monde : zone géographique, pays, nom, adresse, ville, numéro de
téléphone, numéro de télécopie, site internet.


\vspace{0.5cm}
\needspace{12\baselineskip}
\subsection*{Les Instituts de recherche français à l'étranger (IFRE) dans le monde
}\index{ifre}\index{recherche}
  \begin{wrapfigure}{r}{2.5cm}
    \centering
    \qrcode[nolink]{https://data.gouv.fr/dataset/5369984ba3a729239d204f03}
  \end{wrapfigure}

Licence : \textbf{Licence Ouverte
}\newline
Créé le : 2013-07-08\newline
Modifié le : 2016-03-11\newline
De 2012-01-01 à 2012-12-31\newline
Granularité : au pays\newline
Mise à jour : ponctuelle\newline
Popularité : 1 réutilisation,  1 suivi\newline
Mots-clé : \emph{ifre, recherche
}\newline
Permalien : \url{https://data.gouv.fr/dataset/5369984ba3a729239d204f03}\newline

\par
\noindent
    Ce jeu de données contient les coordonnées des IFRE dans le monde : nom,
adresse, code postal, ville, pays, courriel, site internet, réseaux
sociaux, indicateur téléphonique, numéro de téléphone, numéro de
télécopie.


\vspace{0.5cm}
\needspace{12\baselineskip}
\subsection*{Liste chronologique des ambassadeurs de France à l'étranger depuis 1945
}\index{ambassade}\index{ambassadeur}\index{representation}
  \begin{wrapfigure}{r}{2.5cm}
    \centering
    \qrcode[nolink]{https://data.gouv.fr/dataset/536998cda3a729239d205068}
  \end{wrapfigure}

Licence : \textbf{Licence Ouverte
}\newline
Créé le : 2013-07-08\newline
Modifié le : 2017-06-19\newline
De 1944-12-30 à 2017-06-16\newline
Granularité : au pays\newline
Mise à jour : annuelle\newline
Popularité : 1 réutilisation,  2 suivis\newline
Mots-clé : \emph{ambassade, ambassadeur, representation
}\newline
Permalien : \url{https://data.gouv.fr/dataset/536998cda3a729239d205068}\newline

\par
\noindent
    liste chronologique des ambassadeurs, envoyés extraordinaires, ministres
plénipotentiaires et chargés d'affaires de France à l'étranger depuis
1945


\vspace{0.5cm}
\needspace{12\baselineskip}
\subsection*{Liste des Instituts français et de leurs antennes
}\index{culture}\index{institut!francais}
  \begin{wrapfigure}{r}{2.5cm}
    \centering
    \qrcode[nolink]{https://data.gouv.fr/dataset/5369991ea3a729239d205165}
  \end{wrapfigure}

Licence : \textbf{Licence Ouverte
}\newline
Créé le : 2013-07-08\newline
Modifié le : 2016-03-14\newline
De 2012-04-16 à 2012-04-16\newline
Granularité : au pays\newline
Mise à jour : ponctuelle\newline
Popularité : 1 réutilisation,  1 suivi\newline
Mots-clé : \emph{culture, institut-francais
}\newline
Permalien : \url{https://data.gouv.fr/dataset/5369991ea3a729239d205165}\newline

\par
\noindent
    Ce jeu de données contient les coordonnées des quelque 100 Instituts
français dans le monde qui constituent, avec leurs 128 antennes, le
principal instrument du ministère des Affaires étrangères et européennes
dans la mise en œuvre de la politique culturelle extérieure de la
France.


\vspace{0.5cm}
\needspace{12\baselineskip}
\subsection*{Programme vacances--travail (PVT)
}\index{accord}\index{culture}\index{emploi}\index{expatriation}\index{jeunes}\index{tourisme}\index{travail}\index{vacances!travail}
  \begin{wrapfigure}{r}{2.5cm}
    \centering
    \qrcode[nolink]{https://data.gouv.fr/dataset/55798a4ac751df3ae6e57269}
  \end{wrapfigure}

Licence : \textbf{Licence Ouverte
}\newline
Créé le : 2015-06-11\newline
Modifié le : 2019-02-11\newline
De 2000-01-01 à 2017-12-31\newline
Granularité : au pays\newline
Mise à jour : annuelle\newline
Popularité : 2 réutilisations,  1 suivi\newline
Mots-clé : \emph{accord, culture, emploi, expatriation, jeunes, tourisme, travail, vacances-travail
}\newline
Permalien : \url{https://data.gouv.fr/dataset/55798a4ac751df3ae6e57269}\newline

\par
\noindent
    Jeux de données actualisé pour l'année 2017. Le programme
vacances-travail s'adresse aux jeunes de 18 à 30 ans désireux de
s'expatrier, en général durant une année, à des fins touristique et
culturelle dans l'un des pays partenaires en ayant la possibilité de
travailler sur place, à titre accessoire, pour compléter leurs moyens
financiers. Le cadre de ce programme est précisé, de manière réciproque,
par un accord bilatéral que la France a conclu avec quinze pays ou
territoires à ce jour : Japon, Nouvelle Zélande, Australie, Canada,
Corée du Sud, Russie, Argentine, Hong Kong, Chili, Colombie, Taïwan,
Uruguay, Mexique, Brésil et Pérou (non encore entré en vigueur). Jeux de
données actualisé pour l'année 2017.


\vspace{0.5cm}
\needspace{12\baselineskip}
\subsection*{Publication de l'article 133 du code des marchés publics par le
ministère des Affaires étrangères
}
  \begin{wrapfigure}{r}{2.5cm}
    \centering
    \qrcode[nolink]{https://data.gouv.fr/dataset/53699e72a3a729239d205e9b}
  \end{wrapfigure}

Licence : \textbf{Licence Ouverte
}\newline
Créé le : 2014-02-04\newline
Modifié le : 2016-09-15\newline
De 2014-02-06 à 2015-01-31\newline
Granularité : au pays\newline
Mise à jour : annuelle\newline
Popularité : 1 réutilisation,  0 suivi\newline
Mots-clé : \emph{aucun
}\newline
Permalien : \url{https://data.gouv.fr/dataset/53699e72a3a729239d205e9b}\newline

\par
\noindent
    Liste des marchés conclus en 2013 par le ministère des Affaires
étrangères.


\vspace{0.5cm}
\needspace{3\baselineskip} \rule{4cm}{0.25pt}\newline\textbf{Aussi disponible du même producteur :}\begin{itemize}
\item \href{https://data.gouv.fr/dataset/58dd136188ee382520af1337}{Action extérieure des collectivités territoriales dans les six zones géographiques : Amériques-Caraïbes, Afrique-Océan Indien, Asie, Océanie, Europe et Proche et Moyen-Orient  (mars 2017)}
\item \href{https://data.gouv.fr/dataset/5c73e6a58b4c4118dcaa8867}{Adoptions internationales : âge des enfants adoptés}
\item \href{https://data.gouv.fr/dataset/5c73ebe78b4c412aed925c5e}{Adoptions internationales en fonction des procédures d'adoption}
\item \href{https://data.gouv.fr/dataset/5c659b918b4c417d9571216c}{Adoptions internationales : origine des enfants par zone géographique }
\item \href{https://data.gouv.fr/dataset/5c659a418b4c41732e115ac7}{Adoptions internationales : origines par pays}
\item \href{https://data.gouv.fr/dataset/53698faea3a729239d2037e9}{Boursiers du Gouvernement français}
\item \href{https://data.gouv.fr/dataset/5369917ca3a729239d203c8c}{Concours du ministère des Affaires étrangères et européennes}
\item \href{https://data.gouv.fr/dataset/536991baa3a729239d203d30}{Contributions françaises aux principales organisations internationales en 2010 et 2011}
\item \href{https://data.gouv.fr/dataset/536991c4a3a729239d203d48}{Coordonnées des représentations diplomatiques}
\item \href{https://data.gouv.fr/dataset/5369922ea3a729239d203e5c}{Déclarations du ministre des Affaires étrangères depuis le 25 février 2012}
\item \href{https://data.gouv.fr/dataset/586a34b2c751df2e612b7154}{Déclarations du porte-parole du Ministère des Affaires Etrangères et du Développement International de 1966 à 2016}
\item \href{https://data.gouv.fr/dataset/536992cca3a729239d204002}{Déplacements d'Alain Juppé, ancien ministre d'Etat, ministre des Affaires étrangères et européennes}
\item \href{https://data.gouv.fr/dataset/536992cca3a729239d204003}{Déplacements de Henri de Raincourt, ancien ministre auprès du ministre d’Etat, ministre des Affaires étrangères et européennes, chargé de la Coopération}
\item \href{https://data.gouv.fr/dataset/536992cca3a729239d204004}{Déplacements de Jean Leonetti, ancien ministre auprès du ministre d’Etat, ministre des Affaires étrangères et européennes, chargé des Affaires européennes}
\item \href{https://data.gouv.fr/dataset/536992cda3a729239d204005}{Déplacements du ministre des Affaires étrangères du 18 mai 2012 au 12 février 2015}
\item \href{https://data.gouv.fr/dataset/582dc787c751df5bb4c0bb7e}{Données brutes pour création de statistiques sur le registre des français établis hors de France et sur l'état-civil consulaire de 2013 à 2016}
\item \href{https://data.gouv.fr/dataset/582d6c2bc751df3077c0bb7e}{Données brutes pour création de statistiques sur les visas délivrés par la France de 2006 à 2016}
\item \href{https://data.gouv.fr/dataset/5832fc9ec751df2df1c0bb7e}{Données brutes sur la mobilité d'étudiants étrangers souhaitant étudier dans l’enseignement supérieur français de janvier 2007 à octobre 2015}
\item \href{https://data.gouv.fr/dataset/5acb88e5c751df55a835a4c4}{Election législative partielle  - 5ème circonscription des Français à l’étranger  : résultats du vote (1er tour) }
\item \href{https://data.gouv.fr/dataset/5369938fa3a729239d204216}{Elections législatives 2012 (1er tour) : vote des Français établis hors de France}
\item \href{https://data.gouv.fr/dataset/5369938fa3a729239d204217}{Elections législatives 2012 (2nd tour) : vote des Français établis hors de France}
\item \href{https://data.gouv.fr/dataset/593927b2c751df7c8f595285}{Elections législatives 2017 ( 1er tour ) : vote des Français.es établi.e.s hors de France}
\item \href{https://data.gouv.fr/dataset/594a64e5c751df34801eb8f0}{Elections législatives 2017 ( 2e tour ) :  vote des Français.es établi.e.s hors de France}
\item \href{https://data.gouv.fr/dataset/53699391a3a729239d20421b}{Elections législatives partielles 2013 (1er tour) : vote des Français établis hors de France}
\item \href{https://data.gouv.fr/dataset/53699391a3a729239d20421c}{Elections législatives partielles 2013 (2nd tour) : vote des Français établis hors de France}
\item \href{https://data.gouv.fr/dataset/536993a0a3a729239d20424b}{Elections présidentielles 1995 (2nd tour) : vote des Français établis hors de France}
\item \href{https://data.gouv.fr/dataset/536993a0a3a729239d20424c}{Elections présidentielles 2002 (1er tour) : vote des Français établis hors de France}
\item \href{https://data.gouv.fr/dataset/536993a1a3a729239d20424d}{Elections présidentielles 2002 (2nd tour) : vote des Français établis hors de France}
\item \href{https://data.gouv.fr/dataset/536993a1a3a729239d20424e}{Elections présidentielles 2007 (1er tour) : vote des Français établis hors de France}
\item \href{https://data.gouv.fr/dataset/536993a2a3a729239d20424f}{Elections présidentielles 2007 (2nd tour) : vote des Français établis hors de France}
\item \href{https://data.gouv.fr/dataset/536993a2a3a729239d204250}{Elections présidentielles 2012 (1er tour) : vote des Français établis hors de France}
\item \href{https://data.gouv.fr/dataset/536993a2a3a729239d204251}{Elections présidentielles 2012 (2nd tour) : vote des Français établis hors de France}
\item \href{https://data.gouv.fr/dataset/591195a688ee384917ba49e1}{Elections présidentielles 2017 (2nd tour) : vote des Français.es établi.e.s hors de France}
\item \href{https://data.gouv.fr/dataset/5369940ba3a729239d204352}{Emplois au ministère des Affaires étrangères et européennes en France et à l'étranger}
\item \href{https://data.gouv.fr/dataset/53699414a3a729239d204363}{Employés du MAE à l'étranger par zone géographique - décembre 2010}
\item \href{https://data.gouv.fr/dataset/53699415a3a729239d204364}{Employés du MAE et catégories - décembre 2010}
\item \href{https://data.gouv.fr/dataset/53699445a3a729239d2043df}{Enquête sur l'expatriation des Français en 2010}
\item \href{https://data.gouv.fr/dataset/53699502a3a729239d2045b5}{Etats et capitales du monde}
\item \href{https://data.gouv.fr/dataset/536995b2a3a729239d2047c5}{Fonds de Solidarité Prioritaire adoptés en 2012}
\item \href{https://data.gouv.fr/dataset/536995b2a3a729239d2047c6}{Fonds de Solidarité Prioritaire adoptés en 2013}
\item \href{https://data.gouv.fr/dataset/536995b3a3a729239d2047c8}{Fonds d'urgence humanitaire 2012}
\item \href{https://data.gouv.fr/dataset/5c5d559c8b4c4178380e6da7}{Français de l'étranger : inscriptions au registre des Français établis hors de France }
\item \href{https://data.gouv.fr/dataset/536997f0a3a729239d204e10}{Les ambassadeurs de France depuis 1945}
\item \href{https://data.gouv.fr/dataset/5369980ea3a729239d204e5b}{Les déclarations de politique étrangère de 1966 à février 2015}
\item \href{https://data.gouv.fr/dataset/555f396fc751df1d29c98e10}{Les déplacements à l'étranger de Matthias Fekl, secrétaire d’Etat auprès du ministre des Affaires étrangères et du Développement international chargé du commerce extérieur, de la promotion du tourisme et des Français de l’étranger}
\item \href{https://data.gouv.fr/dataset/53699871a3a729239d204f71}{Les projets de coopération décentralisée : Amérique centrale}
\item \href{https://data.gouv.fr/dataset/53699872a3a729239d204f72}{Les projets de coopération décentralisée : Amérique du Nord}
\item \href{https://data.gouv.fr/dataset/53699872a3a729239d204f73}{Les projets de coopération décentralisée : Amérique du Sud}
\item \href{https://data.gouv.fr/dataset/53699873a3a729239d204f74}{Les projets de coopération décentralisée : Asie Centrale et du Sud}
\item \href{https://data.gouv.fr/dataset/53699873a3a729239d204f75}{Les projets de coopération décentralisée : Asie Extrême-Orient}
\item et 12 autres jeux de données\end{itemize}

\clearpage
\section{Ministère de l'Intérieur}


\begin{center}
  \includegraphics[width=3cm]{images/orga/8e_0925ee5c44429d9e37131258773cb7-100.png}
\end{center}


Placé au cœur de l'État, le ministère de l'Intérieur assure la
permanence et la continuité de l'État. Cette fonction régalienne se
concrétise par le rôle majeur et les services rendus par le réseau des
préfectures et des sous-préfectures aux citoyens sur tout le territoire
national. Au quotidien, le ministère de l'Intérieur est le garant de la
sécurité des Français : sécurités publique, civile, routière\ldots{} Le
ministère de l'Intérieur est, à ce titre, un ministère opérationnel
capable d'agir et de réagir à tout moment pour protéger les populations,
notamment, lors de crises majeures. Il est aussi le garant du libre
exercice et du respect des libertés publiques : libertés de circulation,
de vote, d'association, de culte, d'installation dans des conditions
régulières pour ceux qui viennent de l'étranger. Il veille, enfin, au
respect des libertés locales et des compétences des collectivités
territoriales.

\href{http://www.interieur.gouv.fr}{site du ministère de l'Intérieur}

\href{http://www.prefecturedepolice.interieur.gouv.fr/}{site de la
préfecture de police de Paris}

\href{https://ants.gouv.fr/}{site de l'agence nationale des titres
sécurisés}

\href{https://www.antai.gouv.fr/publiques/accueil}{site de l'agence
nationale de traitement automatisé des infractions}

Portail collectivités
territoriales:\url{https://www.collectivites-locales.gouv.fr/structures-territoriales}


\vspace{0.5cm}

\needspace{12\baselineskip}
\subsection*{15042 - Tableau sur l'évolution du nombre d'ambassades et de consulats
équipés et du nombre de visas biométriques délivrés
}\index{entree!en!france}\index{espace!schengen}\index{etranger}\index{frontiere}\index{immigration}\index{immigre}\index{migrant}\index{politique!des!visas}\index{visa}\index{visiteur}\index{vls}
  \begin{wrapfigure}{r}{2.5cm}
    \centering
    \qrcode[nolink]{https://data.gouv.fr/dataset/55ca187088ee38576ca46ec1}
  \end{wrapfigure}

Licence : \textbf{Licence Ouverte
}\newline
Créé le : 2015-08-11\newline
Modifié le : 2015-12-14\newline
De 2008-01-01 à 2013-12-31\newline
Granularité : au pays\newline
Mise à jour : annuelle\newline
Popularité : 1 réutilisation,  0 suivi\newline
Mots-clé : \emph{entree-en-france, espace-schengen, etranger, frontiere, immigration, immigre, migrant, politique-des-visas, visa, visiteur, vls
}\newline
Permalien : \url{https://data.gouv.fr/dataset/55ca187088ee38576ca46ec1}\newline

\par
\noindent
    La biométrie a pour but de lutter contre la fraude à l'identité grâce à
une identification certaine des personnes auxquelles sont délivrés des
visas, que ce soit lors des contrôles à la frontière, lors des
vérifications d'identité sur le territoire national ou encore dans le
pays d'origine lorsque la délivrance du visa a été assortie d'un
rendez-vous de retour au consulat après expiration de la validité du
visa. La comparaison des empreintes digitales à différents moments et
dans des lieux différents permet d'assurer le suivi de certains
demandeurs ayant attiré l'attention des services intéressés.

En 2013, 5 nouveaux postes sont passés en mode biométrique et les visas
biométriques représentent désormais près des 2/3 des visas délivrés par
la France. A la fin juin 2014, il reste une douzaine de postes qui
délivrent des visas non biométriques.


\vspace{0.5cm}
\needspace{12\baselineskip}
\subsection*{Adhésion des communes à un établissement public de coopération
intercommunale (EPCI) à fiscalité propre en France métropolitaine et Dom
}
  \begin{wrapfigure}{r}{2.5cm}
    \centering
    \qrcode[nolink]{https://data.gouv.fr/dataset/53698e76a3a729239d203485}
  \end{wrapfigure}

Licence : \textbf{Licence Ouverte
}\newline
Créé le : 2013-08-28\newline
Modifié le : 2016-06-10\newline
De 2001-01-01 à 2013-12-31\newline
Mise à jour : annuelle\newline
Popularité : 2 réutilisations,  1 suivi\newline
Mots-clé : \emph{aucun
}\newline
Permalien : \url{https://data.gouv.fr/dataset/53698e76a3a729239d203485}\newline

\par
\noindent
    

\vspace{0.5cm}
\needspace{12\baselineskip}
\subsection*{Chiffres départementaux mensuels relatifs aux crimes et délits
enregistrés par les services de police et de gendarmerie depuis janvier
1996
}\index{crimes}\index{delits}\index{dom!com}\index{etat!4001}\index{france}\index{france!metropolitaine}\index{gendarmerie}\index{outre!mer}\index{police}\index{statistiques}
  \begin{wrapfigure}{r}{2.5cm}
    \centering
    \qrcode[nolink]{https://data.gouv.fr/dataset/5617ad4dc751df6211cdbb49}
  \end{wrapfigure}

Licence : \textbf{Licence Ouverte
}\newline
Créé le : 2015-10-09\newline
Modifié le : 2019-03-11\newline
Mise à jour : mensuelle\newline
Popularité : 3 réutilisations,  14 suivis\newline
Mots-clé : \emph{crimes, delits, dom-com, etat-4001, france, france-metropolitaine, gendarmerie, outre-mer, police, statistiques
}\newline
Permalien : \url{https://data.gouv.fr/dataset/5617ad4dc751df6211cdbb49}\newline

\par
\noindent
    Ces données correspondent aux nombres de crimes et délits enregistrés
mensuellement par les services de police et de gendarmerie. Les tableaux
mis à disposition forment ce que l'on appelait « l'état 4001 ». Ils
contiennent des informations, de caractère administratif, sur l'activité
judiciaire des services de police et de gendarmerie, y compris celles
des DOM-COM, depuis janvier 1996. Ces données sont mises à jour
mensuellement.


\vspace{0.5cm}
\needspace{12\baselineskip}
\subsection*{Circonscriptions législatives : Table de correspondance des communes et
des cantons pour les élections législatives de 2012 et sa mise à jour
pour les élections législatives 2017
}
  \begin{wrapfigure}{r}{2.5cm}
    \centering
    \qrcode[nolink]{https://data.gouv.fr/dataset/5369a160a3a729239d2065bc}
  \end{wrapfigure}

Licence : \textbf{Licence Ouverte
}\newline
Créé le : 2013-07-08\newline
Modifié le : 2017-04-11\newline
De 2012-06-17 à 2017-03-30\newline
Granularité : à la commune\newline
Mise à jour : annuelle\newline
Popularité : 4 réutilisations,  4 suivis\newline
Mots-clé : \emph{aucun
}\newline
Permalien : \url{https://data.gouv.fr/dataset/5369a160a3a729239d2065bc}\newline

\par
\noindent
    Ces fichiers établissent un tableau de concordance entre les communes,
les cantons, les anciennes et nouvelles circonscriptions législatives
pour les élections législatives de 2012, et leur mise à jour pour celles
de 2017. Pour le scrutin de 2017, il tient compte des communes nouvelles
créées depuis 2015.

Seules les dispositions législatives et réglementaires font foi en
matière de découpage des circonscriptions électorales. Pour les
circonscriptions législatives, il convient de se référer aux
dispositions de l'article L. 125 du code électoral, étant précisé que
\textbf{les circonscriptions législatives qui serviront à l'élection des
députés lors du renouvellement général de cette assemblée, en juin 2017,
restent strictement identiques à celles de juin 2012.}


\vspace{0.5cm}
\needspace{12\baselineskip}
\subsection*{Contours des cantons électoraux départementaux 2015
}\index{canton}\index{cantons}\index{contours}\index{decoupage}\index{decoupage!administratif}\index{departement}\index{election}\index{elections}\index{redecoupage}
  \begin{wrapfigure}{r}{2.5cm}
    \centering
    \qrcode[nolink]{https://data.gouv.fr/dataset/54e74d51c751df375a467389}
  \end{wrapfigure}

Licence : \textbf{Licence Ouverte
}\newline
Créé le : 2015-02-20\newline
Modifié le : 2016-03-16\newline
De 2015-01-01 à 2015-12-31\newline
Granularité : au canton\newline
Mise à jour : ponctuelle\newline
Popularité : 8 réutilisations,  5 suivis\newline
Mots-clé : \emph{canton, cantons, contours, decoupage, decoupage-administratif, departement, election, elections, redecoupage
}\newline
Permalien : \url{https://data.gouv.fr/dataset/54e74d51c751df375a467389}\newline

\par
\noindent
    Un redécoupage des cantons français est défini par la loi du 17 mai 2013
et les décrets d'application publiés en février et mars 2014.

En application de cette loi, les circonscriptions que sont les cantons
permettent l'élection des assemblées départementales, rebaptisées
conseils départementaux au scrutin majoritaire, binominal et paritaire.

En effet, ce redécoupage s'accompagne d'un mode de scrutin destiné à
promouvoir la parité : chaque nouveau canton est représenté par deux
conseillers départementaux, un homme et une femme, élus en binôme.

Le ministère de l'intérieur met à disposition sur la plateforme
data.gouv.fr les contours des cantons électoraux pour l'année 2015 dans
un format réutilisable.

Fichier au format shapefile et mapinfo en projection légale locale et
encodage ISO-8859-1.

Copyright : ``Experian / Cartosphere / IGN / Ministère de l'Intérieur''


\vspace{0.5cm}
\needspace{12\baselineskip}
\subsection*{Contours géographiques des nouvelles régions (métropole)
}\index{regions}
  \begin{wrapfigure}{r}{2.5cm}
    \centering
    \qrcode[nolink]{https://data.gouv.fr/dataset/55717121c751df588de5726c}
  \end{wrapfigure}

Licence : \textbf{Licence Ouverte
}\newline
Créé le : 2015-06-05\newline
Modifié le : 2016-03-16\newline
Popularité : 8 réutilisations,  0 suivi\newline
Mots-clé : \emph{regions
}\newline
Permalien : \url{https://data.gouv.fr/dataset/55717121c751df588de5726c}\newline

\par
\noindent
    Contours géographiques des nouvelles régions françaises (Métropole)


\vspace{0.5cm}
\needspace{12\baselineskip}
\subsection*{Contours géographiques des nouvelles régions
(Métropole+Réunion+Guadeloupe)
}\index{guadeloupe}\index{metropole}\index{regions}\index{reunion}
  \begin{wrapfigure}{r}{2.5cm}
    \centering
    \qrcode[nolink]{https://data.gouv.fr/dataset/5576e17ec751df462be5726b}
  \end{wrapfigure}

Licence : \textbf{Licence Ouverte
}\newline
Créé le : 2015-06-09\newline
Modifié le : 2016-03-12\newline
Granularité : à la région\newline
Popularité : 2 réutilisations,  1 suivi\newline
Mots-clé : \emph{guadeloupe, metropole, regions, reunion
}\newline
Permalien : \url{https://data.gouv.fr/dataset/5576e17ec751df462be5726b}\newline

\par
\noindent
    Mise à disposition de l'ensemble des fichiers sous format ZIP


\vspace{0.5cm}
\needspace{12\baselineskip}
\subsection*{Crimes et délits enregistrés par les services de gendarmerie et de
police depuis 2012
}\index{crimes}\index{delits}\index{dom!com}\index{etat!4001}\index{france}\index{france!metropolitaine}\index{gendarmerie}\index{outre!mer}\index{police}\index{statistiques}
  \begin{wrapfigure}{r}{2.5cm}
    \centering
    \qrcode[nolink]{https://data.gouv.fr/dataset/5617b11dc751df083fcdbb48}
  \end{wrapfigure}

Licence : \textbf{Licence Ouverte
}\newline
Créé le : 2015-10-09\newline
Modifié le : 2019-03-11\newline
De 2012-01-01 à 2018-02-09\newline
Mise à jour : annuelle\newline
Popularité : 2 réutilisations,  10 suivis\newline
Mots-clé : \emph{crimes, delits, dom-com, etat-4001, france, france-metropolitaine, gendarmerie, outre-mer, police, statistiques
}\newline
Permalien : \url{https://data.gouv.fr/dataset/5617b11dc751df083fcdbb48}\newline

\par
\noindent
    Ces données constituent une actualisation des volumes Criminalité et
délinquance constatées en France édités par la Direction centrale de la
police judiciaire, publiés jusqu'en 2013 à la Documentation française,
et disponibles sur son site
\href{http://www.ladocumentationfrancaise.fr/rapports-publics/134000490-criminalite-et-delinquance-constatees-en-france-annee-2012}{internet}.
Elles contiennent des informations, de caractère administratif, sur
l'activité judiciaire des services de gendarmerie et de police, y
compris celles des DOM-COM, depuis 2012.


\vspace{0.5cm}
\needspace{12\baselineskip}
\subsection*{Critères de répartition 2018 pour la Dotation Globale de Fonctionnement
(DGF)
}\index{collectivites!territoriales}\index{collectiviteslocales}\index{dgcl}\index{dgf}\index{dotations}
  \begin{wrapfigure}{r}{2.5cm}
    \centering
    \qrcode[nolink]{https://data.gouv.fr/dataset/5b39f139c751df2a68a61d15}
  \end{wrapfigure}

Licence : \textbf{Licence Ouverte
}\newline
Créé le : 2018-07-02\newline
Modifié le : 2018-08-27\newline
De 2018-01-01 à 2018-12-31\newline
Mise à jour : annuelle\newline
Popularité : 1 réutilisation,  3 suivis\newline
Mots-clé : \emph{collectivites-territoriales, collectiviteslocales, dgcl, dgf, dotations
}\newline
Permalien : \url{https://data.gouv.fr/dataset/5b39f139c751df2a68a61d15}\newline

\par
\noindent
    Ces jeux de données comportent l'ensemble des données individuelles
ayant servies au calcul de la dotation globale de fonctionnement (DGF)
des communes, EPCI et départements en 2018.

Retrouvez toutes les données disponibles sur les finances locales et les
dotations
:\url{http://www.dotations-dgcl.interieur.gouv.fr/consultation/dotations_en_ligne.php}


\vspace{0.5cm}
\needspace{12\baselineskip}
\subsection*{Données du Répertoire national des élus
}\index{rne}
  \begin{wrapfigure}{r}{2.5cm}
    \centering
    \qrcode[nolink]{https://data.gouv.fr/dataset/5c34c4d1634f4173183a64f1}
  \end{wrapfigure}

Licence : \textbf{Licence Ouverte version 2.0
}\newline
Créé le : 2019-01-08\newline
Modifié le : 2019-01-09\newline
Mise à jour : trimestrielle\newline
Popularité : 5 réutilisations,  13 suivis\newline
Mots-clé : \emph{rne
}\newline
Permalien : \url{https://data.gouv.fr/dataset/5c34c4d1634f4173183a64f1}\newline

\par
\noindent
    Le Répertoire National des Elus (RNE) a pour finalité le suivi des
titulaires d'un mandat électoral. Il est renseigné et tenu à jour par
les préfectures et par les services du ministère de l'intérieur,
notamment sur la base des éléments fournis par les élus lors de la phase
d'enregistrement des candidatures.

Les données du RNE sont structurées par mandat. Neuf fichiers sont
publiés ici :

1- les conseillers municipaux ; 2- les conseillers communautaires ; 3-
les conseillers départementaux ; 4- les conseillers régionaux ; 5- les
conseillers de l'Assemblée de Corse ; 6- les représentants au Parlement
européen ; 7- les sénateurs ; 8- les députés ; 9- les maires.

Un élu qui dispose de plusieurs mandats figurera dans plusieurs
fichiers.

Les fichiers mis en ligne sont actualisés trimestriellement.

Les données relatives à la profession sont déclaratives.

Les demandes de rectification doivent être adressées directement par
courriel à la préfecture territorialement compétente
(pref-rne-contact@nom-du-departement.gouv.fr). Aucune demande de
rectification adressée sur la plateforme www.data.gouv.fr ne sera prise
en compte. Les rectifications apportées dans le RNE ne seront pas
reportées immédiatement sur la plate-forme www.data.gouv.fr, mais seront
prises en compte pour la publication suivante.


\vspace{0.5cm}
\needspace{12\baselineskip}
\subsection*{Election présidentielle 1995 -- Résultats
}
  \begin{wrapfigure}{r}{2.5cm}
    \centering
    \qrcode[nolink]{https://data.gouv.fr/dataset/5369937aa3a729239d2041da}
  \end{wrapfigure}

Licence : \textbf{Licence Ouverte
}\newline
Créé le : 2013-07-08\newline
Modifié le : 2016-02-06\newline
De 1995-04-23 à 1995-05-07\newline
Popularité : 4 réutilisations,  0 suivi\newline
Mots-clé : \emph{aucun
}\newline
Permalien : \url{https://data.gouv.fr/dataset/5369937aa3a729239d2041da}\newline

\par
\noindent
    Résultats de l'élection présidentielle de 1995, tours 1 et 2, par
régions, départements, circonscriptions législatives, cantons


\vspace{0.5cm}
\needspace{12\baselineskip}
\subsection*{Election présidentielle 1995 -- Résultats
}
  \begin{wrapfigure}{r}{2.5cm}
    \centering
    \qrcode[nolink]{https://data.gouv.fr/dataset/5369937aa3a729239d2041db}
  \end{wrapfigure}

Licence : \textbf{Licence Ouverte
}\newline
Créé le : 2013-07-08\newline
Modifié le : 2015-12-28\newline
De 1995-04-23 à 1995-05-07\newline
Popularité : 6 réutilisations,  1 suivi\newline
Mots-clé : \emph{aucun
}\newline
Permalien : \url{https://data.gouv.fr/dataset/5369937aa3a729239d2041db}\newline

\par
\noindent
    Résultats de l'élection présidentielle de 1995, tours 1 et 2, par
communes


\vspace{0.5cm}
\needspace{12\baselineskip}
\subsection*{Election présidentielle 2002 - Résultats
}
  \begin{wrapfigure}{r}{2.5cm}
    \centering
    \qrcode[nolink]{https://data.gouv.fr/dataset/5369937ba3a729239d2041dd}
  \end{wrapfigure}

Licence : \textbf{Licence Ouverte
}\newline
Créé le : 2013-07-08\newline
Modifié le : 2016-01-24\newline
De 2002-04-21 à 2002-05-05\newline
Popularité : 7 réutilisations,  1 suivi\newline
Mots-clé : \emph{aucun
}\newline
Permalien : \url{https://data.gouv.fr/dataset/5369937ba3a729239d2041dd}\newline

\par
\noindent
    Résultats de l'élection présidentielle 2002, 1er tour, par communes


\vspace{0.5cm}
\needspace{12\baselineskip}
\subsection*{Election présidentielle 2002 - Résultats
}
  \begin{wrapfigure}{r}{2.5cm}
    \centering
    \qrcode[nolink]{https://data.gouv.fr/dataset/5369937ba3a729239d2041de}
  \end{wrapfigure}

Licence : \textbf{Licence Ouverte
}\newline
Créé le : 2013-07-08\newline
Modifié le : 2016-03-07\newline
De 2002-04-21 à 2002-05-05\newline
Popularité : 13 réutilisations,  1 suivi\newline
Mots-clé : \emph{aucun
}\newline
Permalien : \url{https://data.gouv.fr/dataset/5369937ba3a729239d2041de}\newline

\par
\noindent
    Résultats de l'élection présidentielle 2002, 2ème tour, par communes


\vspace{0.5cm}
\needspace{12\baselineskip}
\subsection*{Election présidentielle 2002 - Résultats
}
  \begin{wrapfigure}{r}{2.5cm}
    \centering
    \qrcode[nolink]{https://data.gouv.fr/dataset/5369937ba3a729239d2041dc}
  \end{wrapfigure}

Licence : \textbf{Licence Ouverte
}\newline
Créé le : 2013-07-08\newline
Modifié le : 2016-01-02\newline
De 2002-04-21 à 2002-05-05\newline
Popularité : 1 réutilisation,  0 suivi\newline
Mots-clé : \emph{aucun
}\newline
Permalien : \url{https://data.gouv.fr/dataset/5369937ba3a729239d2041dc}\newline

\par
\noindent
    Résultats de l'élection présidentielle 2002, tours 1 et 2, par régions,
départements, circonscriptions législatives, cantons


\vspace{0.5cm}
\needspace{12\baselineskip}
\subsection*{Election présidentielle 2007 - Résultats
}
  \begin{wrapfigure}{r}{2.5cm}
    \centering
    \qrcode[nolink]{https://data.gouv.fr/dataset/5369937ca3a729239d2041df}
  \end{wrapfigure}

Licence : \textbf{Licence Ouverte
}\newline
Créé le : 2013-07-08\newline
Modifié le : 2015-12-27\newline
De 2007-04-22 à 2007-06-10\newline
Popularité : 2 réutilisations,  0 suivi\newline
Mots-clé : \emph{aucun
}\newline
Permalien : \url{https://data.gouv.fr/dataset/5369937ca3a729239d2041df}\newline

\par
\noindent
    Résultats de l'élection présidentielle 2007, tours 1 et 2, par régions,
départements, circonscriptions législatives, cantons


\vspace{0.5cm}
\needspace{12\baselineskip}
\subsection*{Election présidentielle 2007 - Résultats
}
  \begin{wrapfigure}{r}{2.5cm}
    \centering
    \qrcode[nolink]{https://data.gouv.fr/dataset/5369937ca3a729239d2041e0}
  \end{wrapfigure}

Licence : \textbf{Licence Ouverte
}\newline
Créé le : 2013-07-08\newline
Modifié le : 2016-02-25\newline
De 2007-04-22 à 2007-05-06\newline
Popularité : 7 réutilisations,  3 suivis\newline
Mots-clé : \emph{aucun
}\newline
Permalien : \url{https://data.gouv.fr/dataset/5369937ca3a729239d2041e0}\newline

\par
\noindent
    Résultats de l'élection présidentielle 2007, tours 1 et 2, par communes


\vspace{0.5cm}
\needspace{12\baselineskip}
\subsection*{Election présidentielle 2012 - Résultats
}
  \begin{wrapfigure}{r}{2.5cm}
    \centering
    \qrcode[nolink]{https://data.gouv.fr/dataset/5369937da3a729239d2041e3}
  \end{wrapfigure}

Licence : \textbf{Licence Ouverte
}\newline
Créé le : 2013-07-08\newline
Modifié le : 2016-03-08\newline
De 2012-04-22 à 2012-05-06\newline
Popularité : 5 réutilisations,  2 suivis\newline
Mots-clé : \emph{aucun
}\newline
Permalien : \url{https://data.gouv.fr/dataset/5369937da3a729239d2041e3}\newline

\par
\noindent
    Résultats de l'élection présidentielle 2012, tours 1 et 2, par régions,
départements, circonscriptions législatives, cantons


\vspace{0.5cm}
\needspace{12\baselineskip}
\subsection*{Election présidentielle 2012 - Résultats
}
  \begin{wrapfigure}{r}{2.5cm}
    \centering
    \qrcode[nolink]{https://data.gouv.fr/dataset/5369937da3a729239d2041e6}
  \end{wrapfigure}

Licence : \textbf{Licence Ouverte
}\newline
Créé le : 2013-07-08\newline
Modifié le : 2016-03-14\newline
De 2012-04-22 à 2012-05-06\newline
Popularité : 15 réutilisations,  6 suivis\newline
Mots-clé : \emph{aucun
}\newline
Permalien : \url{https://data.gouv.fr/dataset/5369937da3a729239d2041e6}\newline

\par
\noindent
    Résultats de l'élection présidentielle 2012, tours 1 et 2, par communes


\vspace{0.5cm}
\needspace{12\baselineskip}
\subsection*{Election présidentielle 2012 -- Résultats par bureaux de vote
}\index{bureaux!de!vote}\index{election!presidentielle}\index{elections}\index{resultats}
  \begin{wrapfigure}{r}{2.5cm}
    \centering
    \qrcode[nolink]{https://data.gouv.fr/dataset/5605053fc751df3f0a8dabef}
  \end{wrapfigure}

Licence : \textbf{Licence Ouverte
}\newline
Créé le : 2015-09-25\newline
Modifié le : 2016-03-04\newline
Granularité : à la commune\newline
Mise à jour : ponctuelle\newline
Popularité : 1 réutilisation,  1 suivi\newline
Mots-clé : \emph{bureaux-de-vote, election-presidentielle, elections, resultats
}\newline
Permalien : \url{https://data.gouv.fr/dataset/5605053fc751df3f0a8dabef}\newline

\par
\noindent
    Résultats de l'élection présidentielle 2012, tours 1 et 2, par bureaux
de vote Nota : le découpage communal, la nomenclature et les périmètres
des bureaux de vote enregistrent des évolutions entre les différents
scrutins


\vspace{0.5cm}
\needspace{12\baselineskip}
\subsection*{Election présidentielle des 23 avril et 7 mai 2017 - Résultats
définitifs du 1er tour
}\index{1er!tour}\index{election}\index{election!presidentielle}\index{election!presidentielle!2017}\index{resultats}\index{resultats!definitifs}
  \begin{wrapfigure}{r}{2.5cm}
    \centering
    \qrcode[nolink]{https://data.gouv.fr/dataset/5901a4f8c751df4146504ed7}
  \end{wrapfigure}

Licence : \textbf{Licence Ouverte
}\newline
Créé le : 2017-04-27\newline
Modifié le : 2017-04-27\newline
De 2017-04-23 à 2017-04-27\newline
Granularité : au département\newline
Popularité : 2 réutilisations,  0 suivi\newline
Mots-clé : \emph{1er-tour, election, election-presidentielle, election-presidentielle-2017, resultats, resultats-definitifs
}\newline
Permalien : \url{https://data.gouv.fr/dataset/5901a4f8c751df4146504ed7}\newline

\par
\noindent
    Résultats du 1er tour de l'élection présidentielle 2017, France entière,
métropole, outre-mer, par régions, départements, circonscriptions
législatives et cantons. Ces résultats sont définitifs.


\vspace{0.5cm}
\needspace{12\baselineskip}
\subsection*{Election présidentielle des 23 avril et 7 mai 2017 - Résultats
définitifs du 1er tour par bureaux de vote
}\index{1er!tour}\index{2017}\index{bureaux!de!vote}\index{election}\index{election!presidentielle}\index{presidentielle}\index{resultats}
  \begin{wrapfigure}{r}{2.5cm}
    \centering
    \qrcode[nolink]{https://data.gouv.fr/dataset/5901a718c751df3f49b29ff8}
  \end{wrapfigure}

Licence : \textbf{Licence Ouverte
}\newline
Créé le : 2017-04-27\newline
Modifié le : 2017-04-27\newline
De 2017-04-23 à 2017-04-27\newline
Granularité : à la commune\newline
Popularité : 1 réutilisation,  3 suivis\newline
Mots-clé : \emph{1er-tour, 2017, bureaux-de-vote, election, election-presidentielle, presidentielle, resultats
}\newline
Permalien : \url{https://data.gouv.fr/dataset/5901a718c751df3f49b29ff8}\newline

\par
\noindent
    Résultats du 1er tour de l'élection présidentielle 2017, par bureaux de
vote. Ces résultats sont définitifs.


\vspace{0.5cm}
\needspace{12\baselineskip}
\subsection*{Election présidentielle des 23 avril et 7 mai 2017 - Résultats
définitifs du 1er tour par communes
}\index{1er!tour}\index{communes}\index{election}\index{election!presidentielle}\index{election!presidentielle!2017}\index{resultats}
  \begin{wrapfigure}{r}{2.5cm}
    \centering
    \qrcode[nolink]{https://data.gouv.fr/dataset/5901a61ec751df4146504ed8}
  \end{wrapfigure}

Licence : \textbf{Licence Ouverte
}\newline
Créé le : 2017-04-27\newline
Modifié le : 2017-04-27\newline
De 2017-04-23 à 2017-04-27\newline
Granularité : à la commune\newline
Popularité : 7 réutilisations,  0 suivi\newline
Mots-clé : \emph{1er-tour, communes, election, election-presidentielle, election-presidentielle-2017, resultats
}\newline
Permalien : \url{https://data.gouv.fr/dataset/5901a61ec751df4146504ed8}\newline

\par
\noindent
    Résultats du 1er tour de l'élection présidentielle 2017, par communes.
Ces résultats sont définitifs.


\vspace{0.5cm}
\needspace{12\baselineskip}
\subsection*{Election présidentielle des 23 avril et 7 mai 2017 - Résultats
définitifs du 2nd tour
}\index{2017}\index{2nd!tour}\index{election}\index{election!presidentielle!2017}\index{presidentielle}\index{resultats}
  \begin{wrapfigure}{r}{2.5cm}
    \centering
    \qrcode[nolink]{https://data.gouv.fr/dataset/5914112388ee3877201f2f2d}
  \end{wrapfigure}

Licence : \textbf{Licence Ouverte
}\newline
Créé le : 2017-05-11\newline
Modifié le : 2017-05-11\newline
De 2017-05-07 à 2017-05-11\newline
Granularité : au pays\newline
Popularité : 1 réutilisation,  0 suivi\newline
Mots-clé : \emph{2017, 2nd-tour, election, election-presidentielle-2017, presidentielle, resultats
}\newline
Permalien : \url{https://data.gouv.fr/dataset/5914112388ee3877201f2f2d}\newline

\par
\noindent
    Résultats du 2nd tour de l'élection présidentielle 2017, France entière,
métropole, outre-mer, par régions, départements, circonscriptions
législatives et cantons. Ces résultats sont définitifs.


\vspace{0.5cm}
\needspace{12\baselineskip}
\subsection*{Election présidentielle des 23 avril et 7 mai 2017 - Résultats
définitifs du 2nd tour par bureaux de vote
}\index{2017}\index{2nd!tour}\index{bureaux!de!vote}\index{election}\index{election!presidentielle!2017}\index{presidentielle}\index{resultats}
  \begin{wrapfigure}{r}{2.5cm}
    \centering
    \qrcode[nolink]{https://data.gouv.fr/dataset/5914141088ee386741d8d6f1}
  \end{wrapfigure}

Licence : \textbf{Licence Ouverte
}\newline
Créé le : 2017-05-11\newline
Modifié le : 2017-05-11\newline
De 2017-05-07 à 2017-05-11\newline
Granularité : au pays\newline
Popularité : 1 réutilisation,  1 suivi\newline
Mots-clé : \emph{2017, 2nd-tour, bureaux-de-vote, election, election-presidentielle-2017, presidentielle, resultats
}\newline
Permalien : \url{https://data.gouv.fr/dataset/5914141088ee386741d8d6f1}\newline

\par
\noindent
    Résultats du 2nd tour de l'élection présidentielle 2017, par bureaux de
vote. Ces résultats sont définitifs.


\vspace{0.5cm}
\needspace{12\baselineskip}
\subsection*{Election présidentielle des 23 avril et 7 mai 2017 - Résultats
définitifs du 2nd tour par communes
}\index{2017}\index{2nd!tour}\index{communes}\index{election}\index{election!presidentielle!2017}\index{presidentielle}\index{resultats}
  \begin{wrapfigure}{r}{2.5cm}
    \centering
    \qrcode[nolink]{https://data.gouv.fr/dataset/591411ff88ee386741d8d6f0}
  \end{wrapfigure}

Licence : \textbf{Licence Ouverte
}\newline
Créé le : 2017-05-11\newline
Modifié le : 2017-05-11\newline
De 2017-05-07 à 2017-05-11\newline
Granularité : à la commune\newline
Popularité : 5 réutilisations,  0 suivi\newline
Mots-clé : \emph{2017, 2nd-tour, communes, election, election-presidentielle-2017, presidentielle, resultats
}\newline
Permalien : \url{https://data.gouv.fr/dataset/591411ff88ee386741d8d6f0}\newline

\par
\noindent
    Résultats du 2nd tour de l'élection présidentielle 2017, par communes.
Ces résultats sont définitifs.


\vspace{0.5cm}
\needspace{12\baselineskip}
\subsection*{Election présidentielle des 23 avril et 7 mai 2017 - Résultats du 1er
tour
}\index{departements}\index{election!presidentielle!2017}\index{elections}\index{france!entiere}\index{metropole}\index{outre!mer}\index{presidentielle}\index{regions}\index{resultats}\index{resultats!tour!1}
  \begin{wrapfigure}{r}{2.5cm}
    \centering
    \qrcode[nolink]{https://data.gouv.fr/dataset/58fdaf8e88ee38699ea04329}
  \end{wrapfigure}

Licence : \textbf{Licence Ouverte
}\newline
Créé le : 2017-04-24\newline
Modifié le : 2017-04-24\newline
De 2017-04-23 à 2017-04-24\newline
Popularité : 6 réutilisations,  6 suivis\newline
Mots-clé : \emph{departements, election-presidentielle-2017, elections, france-entiere, metropole, outre-mer, presidentielle, regions, resultats, resultats-tour-1
}\newline
Permalien : \url{https://data.gouv.fr/dataset/58fdaf8e88ee38699ea04329}\newline

\par
\noindent
    Résultats du 1er tour de l'élection présidentielle 2017, France entière,
métropole, outre-mer, par régions, départements, circonscriptions
législatives et cantons. Ces résultats sont ceux de la soirée
électorale. Les résultats définitifs seront proclamés par le Conseil
constitutionnel.


\vspace{0.5cm}
\needspace{12\baselineskip}
\subsection*{Election présidentielle des 23 avril et 7 mai 2017 - Résultats du 1er
tour
}\index{election!presidentielle!2017}\index{elections}\index{presidentielle}\index{resultats}\index{tour!1}
  \begin{wrapfigure}{r}{2.5cm}
    \centering
    \qrcode[nolink]{https://data.gouv.fr/dataset/58fdb08188ee386f219015d8}
  \end{wrapfigure}

Licence : \textbf{Licence Ouverte
}\newline
Créé le : 2017-04-24\newline
Modifié le : 2017-04-24\newline
De 2017-04-23 à 2017-04-24\newline
Granularité : à la commune\newline
Popularité : 2 réutilisations,  3 suivis\newline
Mots-clé : \emph{election-presidentielle-2017, elections, presidentielle, resultats, tour-1
}\newline
Permalien : \url{https://data.gouv.fr/dataset/58fdb08188ee386f219015d8}\newline

\par
\noindent
    Résultats du 1er tour de l'élection présidentielle 2017, par communes.
Ces résultats sont ceux de la soirée électorale. Les résultats
définitifs seront proclamés par le Conseil constitutionnel.


\vspace{0.5cm}
\needspace{12\baselineskip}
\subsection*{Election présidentielle des 23 avril et 7 mai 2017 - Résultats du 2ème
tour
}\index{2eme!tour}\index{communes}\index{election!presidentielle}\index{resultats}
  \begin{wrapfigure}{r}{2.5cm}
    \centering
    \qrcode[nolink]{https://data.gouv.fr/dataset/590fb7d288ee3860e3833281}
  \end{wrapfigure}

Licence : \textbf{Licence Ouverte
}\newline
Créé le : 2017-05-08\newline
Modifié le : 2017-05-08\newline
De 2017-05-07 à 2017-05-08\newline
Granularité : au pays\newline
Popularité : 1 réutilisation,  0 suivi\newline
Mots-clé : \emph{2eme-tour, communes, election-presidentielle, resultats
}\newline
Permalien : \url{https://data.gouv.fr/dataset/590fb7d288ee3860e3833281}\newline

\par
\noindent
    Résultats du 2ème tour de l'élection présidentielle 2017, par communes.
Ces résultats sont ceux de la soirée électorale. Les résultats
définitifs seront proclamés par le Conseil constitutionnel.


\vspace{0.5cm}
\needspace{12\baselineskip}
\subsection*{Election présidentielle des 23 avril et 7 mai 2017 - Résultats du 2ème
tour
}\index{2eme!tour}\index{election!presidentielle}\index{resultats}
  \begin{wrapfigure}{r}{2.5cm}
    \centering
    \qrcode[nolink]{https://data.gouv.fr/dataset/590fb72088ee3860f4bb9c89}
  \end{wrapfigure}

Licence : \textbf{Licence Ouverte
}\newline
Créé le : 2017-05-08\newline
Modifié le : 2017-05-08\newline
De 2017-05-07 à 2017-05-08\newline
Granularité : au pays\newline
Popularité : 5 réutilisations,  0 suivi\newline
Mots-clé : \emph{2eme-tour, election-presidentielle, resultats
}\newline
Permalien : \url{https://data.gouv.fr/dataset/590fb72088ee3860f4bb9c89}\newline

\par
\noindent
    Résultats du 2ème tour de l'élection présidentielle 2017, France
entière, métropole, outre-mer, par régions, départements,
circonscriptions législatives et cantons. Ces résultats sont ceux de la
soirée électorale extraits à 01h30. Les résultats définitifs seront
proclamés par le Conseil constitutionnel.


\vspace{0.5cm}
\needspace{12\baselineskip}
\subsection*{Elections cantonales 1994 -- Résultats
}
  \begin{wrapfigure}{r}{2.5cm}
    \centering
    \qrcode[nolink]{https://data.gouv.fr/dataset/5369937fa3a729239d2041ed}
  \end{wrapfigure}

Licence : \textbf{Licence Ouverte
}\newline
Créé le : 2013-07-08\newline
Modifié le : 2016-01-23\newline
De 1994-03-20 à 1994-03-27\newline
Popularité : 1 réutilisation,  0 suivi\newline
Mots-clé : \emph{aucun
}\newline
Permalien : \url{https://data.gouv.fr/dataset/5369937fa3a729239d2041ed}\newline

\par
\noindent
    Résultats des élections cantonales de 1994, tours 1 et 2, par régions,
départements, circonscriptions législatives, cantons et liste des élus à
l'issue de l'élection


\vspace{0.5cm}
\needspace{12\baselineskip}
\subsection*{Elections cantonales 2008 -- Résultats
}
  \begin{wrapfigure}{r}{2.5cm}
    \centering
    \qrcode[nolink]{https://data.gouv.fr/dataset/53699384a3a729239d2041f8}
  \end{wrapfigure}

Licence : \textbf{Licence Ouverte
}\newline
Créé le : 2013-07-08\newline
Modifié le : 2015-12-13\newline
De 2008-03-09 à 2008-03-16\newline
Popularité : 1 réutilisation,  0 suivi\newline
Mots-clé : \emph{aucun
}\newline
Permalien : \url{https://data.gouv.fr/dataset/53699384a3a729239d2041f8}\newline

\par
\noindent
    Résultats des élections cantonales de 2008, tours 1 et 2, par communes


\vspace{0.5cm}
\needspace{12\baselineskip}
\subsection*{Elections cantonales 2011 -- Résultats
}
  \begin{wrapfigure}{r}{2.5cm}
    \centering
    \qrcode[nolink]{https://data.gouv.fr/dataset/53699385a3a729239d2041fb}
  \end{wrapfigure}

Licence : \textbf{Licence Ouverte
}\newline
Créé le : 2013-07-08\newline
Modifié le : 2016-02-08\newline
De 2011-03-20 à 2011-03-27\newline
Popularité : 1 réutilisation,  0 suivi\newline
Mots-clé : \emph{aucun
}\newline
Permalien : \url{https://data.gouv.fr/dataset/53699385a3a729239d2041fb}\newline

\par
\noindent
    Résultats des élections cantonales de 2011, tours 1 et 2, par régions,
départements, circonscriptions législatives, cantons et liste des élus à
l'issue de l'élection


\vspace{0.5cm}
\needspace{12\baselineskip}
\subsection*{Elections départementales 2015 - Candidatures 1er tour
}\index{elections}\index{elections!departementales!2015}
  \begin{wrapfigure}{r}{2.5cm}
    \centering
    \qrcode[nolink]{https://data.gouv.fr/dataset/54e361cac751df7447467389}
  \end{wrapfigure}

Licence : \textbf{Licence Ouverte
}\newline
Créé le : 2015-02-17\newline
Modifié le : 2016-02-01\newline
Granularité : au canton\newline
Mise à jour : ponctuelle\newline
Popularité : 4 réutilisations,  1 suivi\newline
Mots-clé : \emph{elections, elections-departementales-2015
}\newline
Permalien : \url{https://data.gouv.fr/dataset/54e361cac751df7447467389}\newline

\par
\noindent
    Fichier des candidatures pour le 1er tour des élections départementales
2015 (Fichier consolidé 9 mars 2015)


\vspace{0.5cm}
\needspace{12\baselineskip}
\subsection*{Elections départementales 2015 - Candidatures 2ème tour
}\index{2eme!tour}\index{candidatures}\index{elections}\index{elections!departementales}
  \begin{wrapfigure}{r}{2.5cm}
    \centering
    \qrcode[nolink]{https://data.gouv.fr/dataset/55128a74c751df305b882844}
  \end{wrapfigure}

Licence : \textbf{Licence Ouverte
}\newline
Créé le : 2015-03-25\newline
Modifié le : 2016-02-01\newline
Granularité : au canton\newline
Popularité : 1 réutilisation,  0 suivi\newline
Mots-clé : \emph{2eme-tour, candidatures, elections, elections-departementales
}\newline
Permalien : \url{https://data.gouv.fr/dataset/55128a74c751df305b882844}\newline

\par
\noindent
    Fichier des candidatures pour le second tour des élections
départementales 2015.


\vspace{0.5cm}
\needspace{12\baselineskip}
\subsection*{Elections départementales 2015 - Résultats par bureaux de vote
}\index{bureaux!de!vote}\index{elections}\index{elections!departementales}\index{elections!departementales!2015}\index{resultats}
  \begin{wrapfigure}{r}{2.5cm}
    \centering
    \qrcode[nolink]{https://data.gouv.fr/dataset/5605091fc751df3f0a8dabf1}
  \end{wrapfigure}

Licence : \textbf{Licence Ouverte
}\newline
Créé le : 2015-09-25\newline
Modifié le : 2016-03-10\newline
Granularité : à la commune\newline
Mise à jour : ponctuelle\newline
Popularité : 1 réutilisation,  0 suivi\newline
Mots-clé : \emph{bureaux-de-vote, elections, elections-departementales, elections-departementales-2015, resultats
}\newline
Permalien : \url{https://data.gouv.fr/dataset/5605091fc751df3f0a8dabf1}\newline

\par
\noindent
    Résultats des élections départementales 2015, tours 1 et 2, par bureaux
de vote

Nota : le découpage communal, la nomenclature et les périmètres des
bureaux de vote enregistrent des évolutions entre les différents
scrutins


\vspace{0.5cm}
\needspace{12\baselineskip}
\subsection*{Elections départementales 2015 - Résultats tour 1
}\index{elections!departementales!2015}
  \begin{wrapfigure}{r}{2.5cm}
    \centering
    \qrcode[nolink]{https://data.gouv.fr/dataset/551036aec751df0dad882844}
  \end{wrapfigure}

Licence : \textbf{Licence Ouverte
}\newline
Créé le : 2015-03-23\newline
Modifié le : 2016-03-03\newline
Granularité : à la commune\newline
Popularité : 6 réutilisations,  1 suivi\newline
Mots-clé : \emph{elections-departementales-2015
}\newline
Permalien : \url{https://data.gouv.fr/dataset/551036aec751df0dad882844}\newline

\par
\noindent
    Résultats du 1er tour des élections départementales 2015 par communes.


\vspace{0.5cm}
\needspace{12\baselineskip}
\subsection*{Elections départementales 2015 - Résultats tour 1
}\index{elections!departementales!2015}
  \begin{wrapfigure}{r}{2.5cm}
    \centering
    \qrcode[nolink]{https://data.gouv.fr/dataset/55103071c751df04f9882845}
  \end{wrapfigure}

Licence : \textbf{Licence Ouverte
}\newline
Créé le : 2015-03-23\newline
Modifié le : 2016-02-09\newline
Granularité : au canton\newline
Popularité : 1 réutilisation,  0 suivi\newline
Mots-clé : \emph{elections-departementales-2015
}\newline
Permalien : \url{https://data.gouv.fr/dataset/55103071c751df04f9882845}\newline

\par
\noindent
    Résultats du 1er tour des élections départementales 2015, France
entière, métropole, outre-mer, par régions, départements,
circonscriptions législatives, cantons et liste des élus.


\vspace{0.5cm}
\needspace{12\baselineskip}
\subsection*{Elections départementales 2015 - Résultats tour 2
}\index{elections!departementales}\index{elus}\index{resultats}
  \begin{wrapfigure}{r}{2.5cm}
    \centering
    \qrcode[nolink]{https://data.gouv.fr/dataset/55194c1fc751df5017057c91}
  \end{wrapfigure}

Licence : \textbf{Licence Ouverte
}\newline
Créé le : 2015-03-30\newline
Modifié le : 2016-03-12\newline
Granularité : au canton\newline
Popularité : 2 réutilisations,  1 suivi\newline
Mots-clé : \emph{elections-departementales, elus, resultats
}\newline
Permalien : \url{https://data.gouv.fr/dataset/55194c1fc751df5017057c91}\newline

\par
\noindent
    Résultats du 2ème tour des élections départementales 2015, France
entière, métropole, outre-mer, par régions, départements,
circonscriptions législatives, cantons, liste des élus et présidents des
conseils départementaux.


\vspace{0.5cm}
\needspace{12\baselineskip}
\subsection*{Elections départementales 2015 - Résultats tour 2
}\index{communes}\index{elections!departementales}\index{elections!departementales!2015}\index{resultats}
  \begin{wrapfigure}{r}{2.5cm}
    \centering
    \qrcode[nolink]{https://data.gouv.fr/dataset/55194cdcc751df5372057c91}
  \end{wrapfigure}

Licence : \textbf{Licence Ouverte
}\newline
Créé le : 2015-03-30\newline
Modifié le : 2016-02-01\newline
Granularité : à la commune\newline
Popularité : 3 réutilisations,  1 suivi\newline
Mots-clé : \emph{communes, elections-departementales, elections-departementales-2015, resultats
}\newline
Permalien : \url{https://data.gouv.fr/dataset/55194cdcc751df5372057c91}\newline

\par
\noindent
    Résultats du 2ème tour des élections départementales 2015 par communes.


\vspace{0.5cm}
\needspace{12\baselineskip}
\subsection*{Elections européennes 2009 - Résultats
}
  \begin{wrapfigure}{r}{2.5cm}
    \centering
    \qrcode[nolink]{https://data.gouv.fr/dataset/53699389a3a729239d204204}
  \end{wrapfigure}

Licence : \textbf{Licence Ouverte
}\newline
Créé le : 2013-07-08\newline
Modifié le : 2015-12-12\newline
De 2012-11-23 à 2012-11-23\newline
Popularité : 4 réutilisations,  1 suivi\newline
Mots-clé : \emph{aucun
}\newline
Permalien : \url{https://data.gouv.fr/dataset/53699389a3a729239d204204}\newline

\par
\noindent
    Résultats des élections européennes de 2009 par circonscriptions
européennes, régions, départements, circonscriptions législatives et
cantons


\vspace{0.5cm}
\needspace{12\baselineskip}
\subsection*{Elections européennes 2014 - Résultats
}\index{elect}\index{elections!europeennes!2014}\index{resultats!electoraux}
  \begin{wrapfigure}{r}{2.5cm}
    \centering
    \qrcode[nolink]{https://data.gouv.fr/dataset/53893403a3a7291ffe8c5032}
  \end{wrapfigure}

Licence : \textbf{Licence Ouverte
}\newline
Créé le : 2014-05-30\newline
Modifié le : 2016-01-03\newline
De 2014-01-01 à 2014-12-31\newline
Popularité : 6 réutilisations,  1 suivi\newline
Mots-clé : \emph{elect, elections-europeennes-2014, resultats-electoraux
}\newline
Permalien : \url{https://data.gouv.fr/dataset/53893403a3a7291ffe8c5032}\newline

\par
\noindent
    Résultats des élections européennes 2014, France entière, par
circonscriptions européennes, par régions, par départements, par
circonscriptions législatives, par cantons et liste des élus, proclamés
par la commission nationale de recensement des votes.


\vspace{0.5cm}
\needspace{12\baselineskip}
\subsection*{Elections européennes 2014 - Résultats par communes
}\index{election}\index{elections!europeennes!2014}\index{resultats!electoraux}
  \begin{wrapfigure}{r}{2.5cm}
    \centering
    \qrcode[nolink]{https://data.gouv.fr/dataset/53893405a3a7291ffe8c5033}
  \end{wrapfigure}

Licence : \textbf{Licence Ouverte
}\newline
Créé le : 2014-05-30\newline
Modifié le : 2016-03-05\newline
De 2014-01-01 à 2014-12-31\newline
Granularité : à la commune\newline
Popularité : 3 réutilisations,  1 suivi\newline
Mots-clé : \emph{election, elections-europeennes-2014, resultats-electoraux
}\newline
Permalien : \url{https://data.gouv.fr/dataset/53893405a3a7291ffe8c5033}\newline

\par
\noindent
    Résultats des élections européennes 2014 par communes, proclamés par la
commission nationale de recensement des votes


\vspace{0.5cm}
\needspace{12\baselineskip}
\subsection*{Elections législatives 1997 -- Résultats
}
  \begin{wrapfigure}{r}{2.5cm}
    \centering
    \qrcode[nolink]{https://data.gouv.fr/dataset/5369938ba3a729239d20420c}
  \end{wrapfigure}

Licence : \textbf{Licence Ouverte
}\newline
Créé le : 2013-07-08\newline
Modifié le : 2016-01-12\newline
De 1997-05-25 à 1997-06-01\newline
Popularité : 2 réutilisations,  0 suivi\newline
Mots-clé : \emph{aucun
}\newline
Permalien : \url{https://data.gouv.fr/dataset/5369938ba3a729239d20420c}\newline

\par
\noindent
    Résultats des élections législatives de 1997, tours 1 et 2, par régions,
départements, circonscriptions législatives, cantons et liste des élus à
l'issue de l'élection


\vspace{0.5cm}
\needspace{12\baselineskip}
\subsection*{Elections législatives 1997 -- Résultats
}
  \begin{wrapfigure}{r}{2.5cm}
    \centering
    \qrcode[nolink]{https://data.gouv.fr/dataset/5369938ba3a729239d20420d}
  \end{wrapfigure}

Licence : \textbf{Licence Ouverte
}\newline
Créé le : 2013-07-08\newline
Modifié le : 2015-12-13\newline
De 1997-05-25 à 1997-06-01\newline
Popularité : 1 réutilisation,  0 suivi\newline
Mots-clé : \emph{aucun
}\newline
Permalien : \url{https://data.gouv.fr/dataset/5369938ba3a729239d20420d}\newline

\par
\noindent
    Résultats des élections législatives de 1997, tours 1 et 2, par communes


\vspace{0.5cm}
\needspace{12\baselineskip}
\subsection*{Elections législatives 2002 -- Résultats
}
  \begin{wrapfigure}{r}{2.5cm}
    \centering
    \qrcode[nolink]{https://data.gouv.fr/dataset/5369938da3a729239d204210}
  \end{wrapfigure}

Licence : \textbf{Licence Ouverte
}\newline
Créé le : 2013-07-08\newline
Modifié le : 2016-01-25\newline
De 2002-06-09 à 2002-06-16\newline
Popularité : 10 réutilisations,  0 suivi\newline
Mots-clé : \emph{aucun
}\newline
Permalien : \url{https://data.gouv.fr/dataset/5369938da3a729239d204210}\newline

\par
\noindent
    Résultats des élections législatives de 2002, tour 2, par communes


\vspace{0.5cm}
\needspace{12\baselineskip}
\subsection*{Elections législatives 2002 -- Résultats
}
  \begin{wrapfigure}{r}{2.5cm}
    \centering
    \qrcode[nolink]{https://data.gouv.fr/dataset/5369938ca3a729239d20420f}
  \end{wrapfigure}

Licence : \textbf{Licence Ouverte
}\newline
Créé le : 2013-07-08\newline
Modifié le : 2015-12-13\newline
De 2002-06-09 à 2002-06-16\newline
Popularité : 2 réutilisations,  0 suivi\newline
Mots-clé : \emph{aucun
}\newline
Permalien : \url{https://data.gouv.fr/dataset/5369938ca3a729239d20420f}\newline

\par
\noindent
    Résultats des élections législatives de 2002, tour 1, par communes


\vspace{0.5cm}
\needspace{12\baselineskip}
\subsection*{Elections législatives 2007 -- Résultats
}
  \begin{wrapfigure}{r}{2.5cm}
    \centering
    \qrcode[nolink]{https://data.gouv.fr/dataset/5369938ea3a729239d204214}
  \end{wrapfigure}

Licence : \textbf{Licence Ouverte
}\newline
Créé le : 2013-07-08\newline
Modifié le : 2015-12-13\newline
De 2007-06-10 à 2007-06-17\newline
Popularité : 2 réutilisations,  0 suivi\newline
Mots-clé : \emph{aucun
}\newline
Permalien : \url{https://data.gouv.fr/dataset/5369938ea3a729239d204214}\newline

\par
\noindent
    Résultats des élections législatives de 2007, tour 2, par communes


\vspace{0.5cm}
\needspace{12\baselineskip}
\subsection*{Elections législatives 2007 -- Résultats
}
  \begin{wrapfigure}{r}{2.5cm}
    \centering
    \qrcode[nolink]{https://data.gouv.fr/dataset/5369938ea3a729239d204213}
  \end{wrapfigure}

Licence : \textbf{Licence Ouverte
}\newline
Créé le : 2013-07-08\newline
Modifié le : 2016-01-24\newline
De 2007-06-10 à 2007-06-17\newline
Popularité : 4 réutilisations,  0 suivi\newline
Mots-clé : \emph{aucun
}\newline
Permalien : \url{https://data.gouv.fr/dataset/5369938ea3a729239d204213}\newline

\par
\noindent
    Résultats des élections législatives de 2007, tour 1, par communes


\vspace{0.5cm}
\needspace{12\baselineskip}
\subsection*{Elections législatives 2012 -- Résultats
}
  \begin{wrapfigure}{r}{2.5cm}
    \centering
    \qrcode[nolink]{https://data.gouv.fr/dataset/53699390a3a729239d204219}
  \end{wrapfigure}

Licence : \textbf{Licence Ouverte
}\newline
Créé le : 2013-07-08\newline
Modifié le : 2016-03-14\newline
De 2012-06-10 à 2012-06-17\newline
Popularité : 2 réutilisations,  0 suivi\newline
Mots-clé : \emph{aucun
}\newline
Permalien : \url{https://data.gouv.fr/dataset/53699390a3a729239d204219}\newline

\par
\noindent
    Résultats des élections législatives de 2012, tours 1 et 2, par régions,
départements, circonscriptions législatives, cantons et liste des élus à
l'issue de l'élection


\vspace{0.5cm}
\needspace{12\baselineskip}
\subsection*{Elections législatives 2012 -- Résultats
}
  \begin{wrapfigure}{r}{2.5cm}
    \centering
    \qrcode[nolink]{https://data.gouv.fr/dataset/53699391a3a729239d20421a}
  \end{wrapfigure}

Licence : \textbf{Licence Ouverte
}\newline
Créé le : 2013-07-08\newline
Modifié le : 2016-03-09\newline
De 2012-06-10 à 2012-06-17\newline
Popularité : 1 réutilisation,  0 suivi\newline
Mots-clé : \emph{aucun
}\newline
Permalien : \url{https://data.gouv.fr/dataset/53699391a3a729239d20421a}\newline

\par
\noindent
    Résultats des élections législatives de 2012, tours 1 et 2, par communes


\vspace{0.5cm}
\needspace{12\baselineskip}
\subsection*{Elections législatives des 11 et 18 juin 2017 - Liste des candidats du
1er tour
}\index{11!et!17!juin!2017}\index{1er!tour}\index{2017}\index{candidats}\index{candidatures}\index{elections}\index{elections!legislatives}\index{legislatives}
  \begin{wrapfigure}{r}{2.5cm}
    \centering
    \qrcode[nolink]{https://data.gouv.fr/dataset/59243930c751df0bc7ed1852}
  \end{wrapfigure}

Licence : \textbf{Licence Ouverte
}\newline
Créé le : 2017-05-23\newline
Modifié le : 2017-06-02\newline
De 2017-05-23 à 2017-05-24\newline
Granularité : au pays\newline
Popularité : 1 réutilisation,  0 suivi\newline
Mots-clé : \emph{11-et-17-juin-2017, 1er-tour, 2017, candidats, candidatures, elections, elections-legislatives, legislatives
}\newline
Permalien : \url{https://data.gouv.fr/dataset/59243930c751df0bc7ed1852}\newline

\par
\noindent
    Liste définitive des candidats du 1er tour des élections législatives
des 11 et 18 juin 2017


\vspace{0.5cm}
\needspace{12\baselineskip}
\subsection*{Elections législatives des 11 et 18 juin 2017 - Liste des candidats du
2ème tour
}\index{2eme!tour}\index{candidats}\index{candidatures}\index{elections}\index{elections!legislatives}\index{elections!legislatives!2017}\index{legislatives}
  \begin{wrapfigure}{r}{2.5cm}
    \centering
    \qrcode[nolink]{https://data.gouv.fr/dataset/5942391e88ee386dbe1db03a}
  \end{wrapfigure}

Licence : \textbf{Licence Ouverte
}\newline
Créé le : 2017-06-15\newline
Modifié le : 2017-06-15\newline
De 2017-06-13 à 2017-06-15\newline
Granularité : au pays\newline
Popularité : 1 réutilisation,  0 suivi\newline
Mots-clé : \emph{2eme-tour, candidats, candidatures, elections, elections-legislatives, elections-legislatives-2017, legislatives
}\newline
Permalien : \url{https://data.gouv.fr/dataset/5942391e88ee386dbe1db03a}\newline

\par
\noindent
    Liste des candidats du 2ème tour des élections législatives des 11 et 18
juin 2017


\vspace{0.5cm}
\needspace{12\baselineskip}
\subsection*{Elections législatives des 11 et 18 juin 2017 - Résultats du 1er tour
}\index{1er!tour}\index{elections}\index{elections!legislatives}\index{elections!legislatives!2017}\index{legislatives}\index{resultats}
  \begin{wrapfigure}{r}{2.5cm}
    \centering
    \qrcode[nolink]{https://data.gouv.fr/dataset/593e3fd588ee38328ca16cc0}
  \end{wrapfigure}

Licence : \textbf{Licence Ouverte
}\newline
Créé le : 2017-06-12\newline
Modifié le : 2017-06-13\newline
De 2017-06-11 à 2017-06-13\newline
Granularité : au pays\newline
Popularité : 5 réutilisations,  3 suivis\newline
Mots-clé : \emph{1er-tour, elections, elections-legislatives, elections-legislatives-2017, legislatives, resultats
}\newline
Permalien : \url{https://data.gouv.fr/dataset/593e3fd588ee38328ca16cc0}\newline

\par
\noindent
    Résultats définitifs du 1er tour des élections législatives 2017, France
entière, métropole, outre-mer, par régions, départements,
circonscriptions législatives et cantons.


\vspace{0.5cm}
\needspace{12\baselineskip}
\subsection*{Elections législatives des 11 et 18 juin 2017 - Résultats du 2nd tour
}\index{2nd!tour}\index{elections}\index{elections!legislatives!2017}\index{legislatives}\index{resultats}
  \begin{wrapfigure}{r}{2.5cm}
    \centering
    \qrcode[nolink]{https://data.gouv.fr/dataset/59477748c751df22962982cc}
  \end{wrapfigure}

Licence : \textbf{Licence Ouverte
}\newline
Créé le : 2017-06-19\newline
Modifié le : 2017-06-20\newline
De 2017-06-18 à 2017-06-19\newline
Granularité : au pays\newline
Popularité : 2 réutilisations,  1 suivi\newline
Mots-clé : \emph{2nd-tour, elections, elections-legislatives-2017, legislatives, resultats
}\newline
Permalien : \url{https://data.gouv.fr/dataset/59477748c751df22962982cc}\newline

\par
\noindent
    Résultats définitifs du 2nd tour des élections législatives 2017, France
entière, métropole, outre-mer, par régions, départements,
circonscriptions législatives et cantons.


\vspace{0.5cm}
\needspace{12\baselineskip}
\subsection*{Elections législatives des 11 et 18 juin 2017 - Résultats du 2nd tour
par bureaux de vote
}\index{2nd!tour}\index{bureaux!de!vote}\index{elections}\index{elections!legislatives}\index{legislatives}\index{resultats}
  \begin{wrapfigure}{r}{2.5cm}
    \centering
    \qrcode[nolink]{https://data.gouv.fr/dataset/5948d371c751df471327bc4a}
  \end{wrapfigure}

Licence : \textbf{Licence Ouverte
}\newline
Créé le : 2017-06-20\newline
Modifié le : 2017-06-20\newline
De 2017-06-18 à 2017-06-20\newline
Popularité : 1 réutilisation,  1 suivi\newline
Mots-clé : \emph{2nd-tour, bureaux-de-vote, elections, elections-legislatives, legislatives, resultats
}\newline
Permalien : \url{https://data.gouv.fr/dataset/5948d371c751df471327bc4a}\newline

\par
\noindent
    Résultats définitifs du 2nd tour des élections législatives 2017, par
bureaux de vote


\vspace{0.5cm}
\needspace{12\baselineskip}
\subsection*{Elections législatives des 11 et 18 juin 2017 - Résultats du 2nd tour
par communes
}\index{2nd!tour}\index{elections}\index{elections!legislatives!2017}\index{legislatives}\index{resultats}
  \begin{wrapfigure}{r}{2.5cm}
    \centering
    \qrcode[nolink]{https://data.gouv.fr/dataset/5947786dc751df22962982cd}
  \end{wrapfigure}

Licence : \textbf{Licence Ouverte
}\newline
Créé le : 2017-06-19\newline
Modifié le : 2017-06-20\newline
De 2017-06-18 à 2017-06-19\newline
Granularité : à la commune\newline
Popularité : 3 réutilisations,  0 suivi\newline
Mots-clé : \emph{2nd-tour, elections, elections-legislatives-2017, legislatives, resultats
}\newline
Permalien : \url{https://data.gouv.fr/dataset/5947786dc751df22962982cd}\newline

\par
\noindent
    Résultats définitifs du 2nd tour des élections législatives 2017, par
communes.


\vspace{0.5cm}
\needspace{12\baselineskip}
\subsection*{Elections législatives des 11 et 18 juin 2017 - Résultats par bureaux de
vote du 1er tour
}\index{2017}\index{bureaux!de!vote}\index{elections}\index{elections!legislatives}\index{elections!legislatives!2017}\index{resultats}
  \begin{wrapfigure}{r}{2.5cm}
    \centering
    \qrcode[nolink]{https://data.gouv.fr/dataset/593f9c7488ee386d6e48e8ab}
  \end{wrapfigure}

Licence : \textbf{Licence Ouverte
}\newline
Créé le : 2017-06-13\newline
Modifié le : 2017-06-13\newline
De 2017-06-11 à 2017-06-13\newline
Popularité : 1 réutilisation,  2 suivis\newline
Mots-clé : \emph{2017, bureaux-de-vote, elections, elections-legislatives, elections-legislatives-2017, resultats
}\newline
Permalien : \url{https://data.gouv.fr/dataset/593f9c7488ee386d6e48e8ab}\newline

\par
\noindent
    Résultats définitifs du 1er tour des élections législatives 2017, par
bureaux de vote.


\vspace{0.5cm}
\needspace{12\baselineskip}
\subsection*{Elections législatives des 11 et 18 juin 2017 - Résultats par communes
du 1er tour
}\index{communes}\index{elections}\index{elections!legislatives}\index{elections!legislatives!2017}\index{resultats}\index{tour!1}
  \begin{wrapfigure}{r}{2.5cm}
    \centering
    \qrcode[nolink]{https://data.gouv.fr/dataset/593e40c688ee3832624a49dc}
  \end{wrapfigure}

Licence : \textbf{Licence Ouverte
}\newline
Créé le : 2017-06-12\newline
Modifié le : 2017-06-13\newline
De 2017-06-11 à 2017-06-12\newline
Granularité : à la commune\newline
Popularité : 3 réutilisations,  1 suivi\newline
Mots-clé : \emph{communes, elections, elections-legislatives, elections-legislatives-2017, resultats, tour-1
}\newline
Permalien : \url{https://data.gouv.fr/dataset/593e40c688ee3832624a49dc}\newline

\par
\noindent
    Résultats définitifs du 1er tour des élections législatives 2017, par
communes.


\vspace{0.5cm}
\needspace{12\baselineskip}
\subsection*{Elections municipales 2008 -- Résultats
}
  \begin{wrapfigure}{r}{2.5cm}
    \centering
    \qrcode[nolink]{https://data.gouv.fr/dataset/53699393a3a729239d204224}
  \end{wrapfigure}

Licence : \textbf{Licence Ouverte
}\newline
Créé le : 2013-07-08\newline
Modifié le : 2016-03-12\newline
De 2008-03-09 à 2008-03-16\newline
Popularité : 6 réutilisations,  5 suivis\newline
Mots-clé : \emph{aucun
}\newline
Permalien : \url{https://data.gouv.fr/dataset/53699393a3a729239d204224}\newline

\par
\noindent
    Résultats des élections municipales de 2008, tours 1 et 2, pour les
communes de plus de 3 500 habitants (scrutin de liste et déclaration de
candidature obligatoire)


\vspace{0.5cm}
\needspace{12\baselineskip}
\subsection*{Elections municipales 2014 - Candidats dans les communes de moins de 1
000 habitants
}\index{candidats}\index{candidatures}\index{elections}\index{listes}\index{municipales}
  \begin{wrapfigure}{r}{2.5cm}
    \centering
    \qrcode[nolink]{https://data.gouv.fr/dataset/53699396a3a729239d20422c}
  \end{wrapfigure}

Licence : \textbf{Licence Ouverte
}\newline
Créé le : 2014-03-11\newline
Modifié le : 2016-03-08\newline
De 2014-01-01 à 2014-12-31\newline
Granularité : à la commune\newline
Popularité : 1 réutilisation,  0 suivi\newline
Mots-clé : \emph{candidats, candidatures, elections, listes, municipales
}\newline
Permalien : \url{https://data.gouv.fr/dataset/53699396a3a729239d20422c}\newline

\par
\noindent
    Elections municipales 2014 - Liste des candidats dans les communes de
moins de 1 000 habitants Ces listes sont publiées sous réserve : . des
recours de candidats formés devant le tribunal administratif . des refus
de récépissés définitifs non encore délivrés . des erreurs de saisies
éventuelles.


\vspace{0.5cm}
\needspace{12\baselineskip}
\subsection*{Elections municipales 2014 -- Candidatures - Listes détaillées dans les
communes de 1 000 habitants et plus
}\index{candidats}\index{candidatures}\index{elections}\index{listes}\index{municipales}
  \begin{wrapfigure}{r}{2.5cm}
    \centering
    \qrcode[nolink]{https://data.gouv.fr/dataset/53699396a3a729239d20422d}
  \end{wrapfigure}

Licence : \textbf{Licence Ouverte
}\newline
Créé le : 2014-03-11\newline
Modifié le : 2016-02-21\newline
De 2014-01-01 à 2014-12-31\newline
Granularité : à la commune\newline
Popularité : 1 réutilisation,  1 suivi\newline
Mots-clé : \emph{candidats, candidatures, elections, listes, municipales
}\newline
Permalien : \url{https://data.gouv.fr/dataset/53699396a3a729239d20422d}\newline

\par
\noindent
    Elections municipales 2014 -- Candidatures - Liste détaillées dans les
communes de 1 000 habitants et plus.

Ces listes sont publiées sous réserve : . des recours de candidats
formés devant le tribunal administratif . des refus de récépissés
définitifs non encore délivrés . des erreurs de saisies éventuelles.


\vspace{0.5cm}
\needspace{12\baselineskip}
\subsection*{Elections municipales 2014 - Elus au 1er tour (communes de 1 000
habitants et plus)
}\index{communes}\index{elections}\index{elus}\index{municipales}\index{tour!1}
  \begin{wrapfigure}{r}{2.5cm}
    \centering
    \qrcode[nolink]{https://data.gouv.fr/dataset/53699396a3a729239d20422f}
  \end{wrapfigure}

Licence : \textbf{Licence Ouverte
}\newline
Créé le : 2014-03-25\newline
Modifié le : 2016-02-20\newline
De 2014-01-01 à 2014-12-31\newline
Granularité : à la commune\newline
Popularité : 1 réutilisation,  1 suivi\newline
Mots-clé : \emph{communes, elections, elus, municipales, tour-1
}\newline
Permalien : \url{https://data.gouv.fr/dataset/53699396a3a729239d20422f}\newline

\par
\noindent
    Liste des élus au premier tour des élections municipales 2014, pour les
communes de 1 000 habitants et plus (sous réserve de rectifications par
le juge de l'élection)


\vspace{0.5cm}
\needspace{12\baselineskip}
\subsection*{Elections municipales 2014 - Elus au 1er tour (communes de moins de 1
000 habitants)
}\index{communes}\index{elections}\index{elus}\index{municipales}\index{tour!1}
  \begin{wrapfigure}{r}{2.5cm}
    \centering
    \qrcode[nolink]{https://data.gouv.fr/dataset/53699397a3a729239d204231}
  \end{wrapfigure}

Licence : \textbf{Licence Ouverte
}\newline
Créé le : 2014-03-25\newline
Modifié le : 2016-01-28\newline
De 2014-01-01 à 2014-12-31\newline
Granularité : à la commune\newline
Popularité : 1 réutilisation,  2 suivis\newline
Mots-clé : \emph{communes, elections, elus, municipales, tour-1
}\newline
Permalien : \url{https://data.gouv.fr/dataset/53699397a3a729239d204231}\newline

\par
\noindent
    Liste des élus au premier tour des élections municipales 2014, pour les
communes de moins de 1 000 habitants (sous réserve de rectifications par
le juge de l'élection)


\vspace{0.5cm}
\needspace{12\baselineskip}
\subsection*{Elections municipales 2014 - Résultats 1er tour
}\index{communes}\index{elections}\index{municipales}\index{resultats}\index{tour!1}
  \begin{wrapfigure}{r}{2.5cm}
    \centering
    \qrcode[nolink]{https://data.gouv.fr/dataset/53699399a3a729239d204239}
  \end{wrapfigure}

Licence : \textbf{Licence Ouverte
}\newline
Créé le : 2014-03-25\newline
Modifié le : 2016-03-07\newline
De 2014-01-01 à 2014-12-31\newline
Granularité : à la commune\newline
Popularité : 2 réutilisations,  2 suivis\newline
Mots-clé : \emph{communes, elections, municipales, resultats, tour-1
}\newline
Permalien : \url{https://data.gouv.fr/dataset/53699399a3a729239d204239}\newline

\par
\noindent
    Résultats des élections municipales 2014, tour 1, pour les communes de
moins de 1 000 habitants (sous réserve de rectifications par le juge de
l'élection)


\vspace{0.5cm}
\needspace{12\baselineskip}
\subsection*{Elections Municipales 2014 - Résultats 1er tour
}\index{communes}\index{elections}\index{municipales}\index{resultats}\index{tour!1}
  \begin{wrapfigure}{r}{2.5cm}
    \centering
    \qrcode[nolink]{https://data.gouv.fr/dataset/5369939aa3a729239d20423c}
  \end{wrapfigure}

Licence : \textbf{Licence Ouverte
}\newline
Créé le : 2014-03-25\newline
Modifié le : 2016-03-13\newline
De 2014-01-01 à 2014-12-31\newline
Granularité : à la commune\newline
Popularité : 3 réutilisations,  3 suivis\newline
Mots-clé : \emph{communes, elections, municipales, resultats, tour-1
}\newline
Permalien : \url{https://data.gouv.fr/dataset/5369939aa3a729239d20423c}\newline

\par
\noindent
    Résultats des élections municipales 2014, tour 1, pour les communes de 1
000 habitants et plus (sous réserve de rectifications par le juge de
l'élection)


\vspace{0.5cm}
\needspace{12\baselineskip}
\subsection*{Elections régionales 1998
}
  \begin{wrapfigure}{r}{2.5cm}
    \centering
    \qrcode[nolink]{https://data.gouv.fr/dataset/536993a3a3a729239d204253}
  \end{wrapfigure}

Licence : \textbf{Licence Ouverte
}\newline
Créé le : 2013-07-08\newline
Modifié le : 2016-02-23\newline
De 1998-03-15 à 1998-03-15\newline
Popularité : 1 réutilisation,  0 suivi\newline
Mots-clé : \emph{aucun
}\newline
Permalien : \url{https://data.gouv.fr/dataset/536993a3a3a729239d204253}\newline

\par
\noindent
    Résultats des élections régionales 1998 par régions et départements


\vspace{0.5cm}
\needspace{12\baselineskip}
\subsection*{Elections régionales 2004 -- Résultats
}
  \begin{wrapfigure}{r}{2.5cm}
    \centering
    \qrcode[nolink]{https://data.gouv.fr/dataset/536993a6a3a729239d204259}
  \end{wrapfigure}

Licence : \textbf{Licence Ouverte
}\newline
Créé le : 2013-07-08\newline
Modifié le : 2016-02-10\newline
De 2004-03-21 à 2004-03-28\newline
Popularité : 2 réutilisations,  0 suivi\newline
Mots-clé : \emph{aucun
}\newline
Permalien : \url{https://data.gouv.fr/dataset/536993a6a3a729239d204259}\newline

\par
\noindent
    Résultats des élections régionales de 2004, tours 1 et 2, par communes


\vspace{0.5cm}
\needspace{12\baselineskip}
\subsection*{Elections régionales 2004 -- Résultats
}
  \begin{wrapfigure}{r}{2.5cm}
    \centering
    \qrcode[nolink]{https://data.gouv.fr/dataset/536993a5a3a729239d204258}
  \end{wrapfigure}

Licence : \textbf{Licence Ouverte
}\newline
Créé le : 2013-07-08\newline
Modifié le : 2016-02-10\newline
De 2004-03-21 à 2004-03-28\newline
Popularité : 1 réutilisation,  0 suivi\newline
Mots-clé : \emph{aucun
}\newline
Permalien : \url{https://data.gouv.fr/dataset/536993a5a3a729239d204258}\newline

\par
\noindent
    Résultats des élections régionales de 2004, tours 1 et 2, par régions,
départements, circonscriptions législatives, cantons et liste des élus


\vspace{0.5cm}
\needspace{12\baselineskip}
\subsection*{Elections régionales 2010 -- Résultats
}
  \begin{wrapfigure}{r}{2.5cm}
    \centering
    \qrcode[nolink]{https://data.gouv.fr/dataset/536993a9a3a729239d204261}
  \end{wrapfigure}

Licence : \textbf{Licence Ouverte
}\newline
Créé le : 2013-07-08\newline
Modifié le : 2016-03-10\newline
De 2010-03-14 à 2010-03-21\newline
Popularité : 4 réutilisations,  1 suivi\newline
Mots-clé : \emph{aucun
}\newline
Permalien : \url{https://data.gouv.fr/dataset/536993a9a3a729239d204261}\newline

\par
\noindent
    Résultats des élections régionales de 2010, tours 1 et 2, par communes


\vspace{0.5cm}
\needspace{12\baselineskip}
\subsection*{Elections régionales 2010 -- Résultats
}
  \begin{wrapfigure}{r}{2.5cm}
    \centering
    \qrcode[nolink]{https://data.gouv.fr/dataset/536993a8a3a729239d204260}
  \end{wrapfigure}

Licence : \textbf{Licence Ouverte
}\newline
Créé le : 2013-07-08\newline
Modifié le : 2016-01-07\newline
De 2010-03-14 à 2010-03-21\newline
Popularité : 1 réutilisation,  0 suivi\newline
Mots-clé : \emph{aucun
}\newline
Permalien : \url{https://data.gouv.fr/dataset/536993a8a3a729239d204260}\newline

\par
\noindent
    Résultats des élections régionales de 2010, tours 1 et 2, par régions,
départements, circonscriptions législatives, cantons et liste des élus


\vspace{0.5cm}
\needspace{12\baselineskip}
\subsection*{Elections régionales 2010 -- Résultats par bureaux de vote
}\index{bureaux!de!vote}\index{elections}\index{elections!regionales}\index{elections!regionales!2010}\index{resultats}
  \begin{wrapfigure}{r}{2.5cm}
    \centering
    \qrcode[nolink]{https://data.gouv.fr/dataset/56051467c751df578e8dabf3}
  \end{wrapfigure}

Licence : \textbf{Licence Ouverte
}\newline
Créé le : 2015-09-25\newline
Modifié le : 2016-03-16\newline
Granularité : à la commune\newline
Mise à jour : ponctuelle\newline
Popularité : 1 réutilisation,  0 suivi\newline
Mots-clé : \emph{bureaux-de-vote, elections, elections-regionales, elections-regionales-2010, resultats
}\newline
Permalien : \url{https://data.gouv.fr/dataset/56051467c751df578e8dabf3}\newline

\par
\noindent
    Résultats des élections régionales 2010, tours 1 et 2, par bureaux de
vote

Nota : le découpage communal, la nomenclature et les périmètres des
bureaux de vote enregistrent des évolutions entre les différents
scrutins


\vspace{0.5cm}
\needspace{12\baselineskip}
\subsection*{Elections régionales 2015 et des assemblées de Corse, de Guyane et de
Martinique -- Résultats par bureaux de vote -- Tour 2
}\index{bureaux!de!vote}\index{elections}\index{elections!regionales!2015}\index{resultats}\index{tour!2}
  \begin{wrapfigure}{r}{2.5cm}
    \centering
    \qrcode[nolink]{https://data.gouv.fr/dataset/56728d35c751df240dc664bd}
  \end{wrapfigure}

Licence : \textbf{Licence Ouverte
}\newline
Créé le : 2015-12-17\newline
Modifié le : 2016-03-10\newline
De 2015-12-16 à 2015-12-17\newline
Granularité : à la région\newline
Mise à jour : ponctuelle\newline
Popularité : 1 réutilisation,  1 suivi\newline
Mots-clé : \emph{bureaux-de-vote, elections, elections-regionales-2015, resultats, tour-2
}\newline
Permalien : \url{https://data.gouv.fr/dataset/56728d35c751df240dc664bd}\newline

\par
\noindent
    Résultats du 2ème tour des élections régionales, de l'Assemblée de Corse
et des assemblées de Guyane et de Martinique du 6 décembre 2015, par
bureaux de vote. Nota : le découpage communal, la nomenclature et les
périmètres des bureaux de vote enregistrent des évolutions entre les
différents scrutins


\vspace{0.5cm}
\needspace{12\baselineskip}
\subsection*{Liste des collectivités territoriales utilisant la verbalisation
électronique en France
}\index{collectivite!territoriale}\index{commune}\index{proces!verbal!electronique}\index{pve}\index{verbalisation!electronique}
  \begin{wrapfigure}{r}{2.5cm}
    \centering
    \qrcode[nolink]{https://data.gouv.fr/dataset/53a37437a3a7297d730b2f51}
  \end{wrapfigure}

Licence : \textbf{Licence Ouverte
}\newline
Créé le : 2014-06-16\newline
Modifié le : 2015-11-27\newline
Granularité : à la commune\newline
Popularité : 1 réutilisation,  3 suivis\newline
Mots-clé : \emph{collectivite-territoriale, commune, proces-verbal-electronique, pve, verbalisation-electronique
}\newline
Permalien : \url{https://data.gouv.fr/dataset/53a37437a3a7297d730b2f51}\newline

\par
\noindent
    Le fichier recense les services verbalisateurs des collectivités
territoriales françaises utilisant la verbalisation électronique.

Le fichier contient notamment : - Le nom de la collectivité territoriale
et son département de rattachement ; - Le nom du service verbalisateur ;
- La date de démarrage de la verbalisation électronique ; - Le matériel
de verbalisation électronique utilisé ; - Le logiciel utilisé ; - Le
prestataire choisi par la collectivité territoriale.


\vspace{0.5cm}
\needspace{12\baselineskip}
\subsection*{Liste des maires au 17 juin 2014
}\index{elus}\index{elus!municipaux}\index{liste}\index{maires}
  \begin{wrapfigure}{r}{2.5cm}
    \centering
    \qrcode[nolink]{https://data.gouv.fr/dataset/53a3747ca3a7297d730b2f53}
  \end{wrapfigure}

Licence : \textbf{Licence Ouverte
}\newline
Créé le : 2014-06-17\newline
Modifié le : 2016-03-03\newline
De 2014-06-17 à 2014-06-17\newline
Granularité : à la commune\newline
Mise à jour : ponctuelle\newline
Popularité : 3 réutilisations,  5 suivis\newline
Mots-clé : \emph{elus, elus-municipaux, liste, maires
}\newline
Permalien : \url{https://data.gouv.fr/dataset/53a3747ca3a7297d730b2f53}\newline

\par
\noindent
    Liste des maires au 17 juin 2014 (mises à jour effectuées par les
préfectures)


\vspace{0.5cm}
\needspace{12\baselineskip}
\subsection*{Liste des services de police accueillant du public avec géolocalisation
}\index{accueil}\index{accueil!de!jour}\index{accueil!du!public}\index{adresse}\index{coordonnees!geographique}\index{fichier!service}\index{geocodage}\index{point!geographique}\index{police}\index{police!nationale}\index{service}\index{service!de!police}\index{stsisi}
  \begin{wrapfigure}{r}{2.5cm}
    \centering
    \qrcode[nolink]{https://data.gouv.fr/dataset/53ba5222a3a729219b7beade}
  \end{wrapfigure}

Licence : \textbf{Licence Ouverte
}\newline
Créé le : 2014-07-03\newline
Modifié le : 2019-03-17\newline
Granularité : au point d'intérêt\newline
Mise à jour : quotienne\newline
Popularité : 6 réutilisations,  7 suivis\newline
Mots-clé : \emph{accueil, accueil-de-jour, accueil-du-public, adresse, coordonnees-geographique, fichier-service, geocodage, point-geographique, police, police-nationale, service, service-de-police, stsisi
}\newline
Permalien : \url{https://data.gouv.fr/dataset/53ba5222a3a729219b7beade}\newline

\par
\noindent
    Fichier d'export sur l'ensemble des commissariats de police. Chaque
service est identifié par : Le libellé du service, le numéro de
téléphone, l'adresse géographique et les coordonnées
géographiques(projection WGS 84).


\vspace{0.5cm}
\needspace{12\baselineskip}
\subsection*{Liste des unités de gendarmerie accueillant du public, comprenant leur
géolocalisation et leurs horaires d'ouverture
}\index{accueil}\index{accueil!de!jour}\index{accueil!du!public}\index{adresse}\index{coordonnees!geographiques}\index{fichier!unite}\index{gendarmerie}\index{gendarmerie!nationale}\index{geocodage}\index{obligations!de!service!public}\index{point!geographique}\index{stsisi}\index{unite}
  \begin{wrapfigure}{r}{2.5cm}
    \centering
    \qrcode[nolink]{https://data.gouv.fr/dataset/5369993fa3a729239d2051cd}
  \end{wrapfigure}

Licence : \textbf{Licence Ouverte
}\newline
Créé le : 2013-11-22\newline
Modifié le : 2019-03-17\newline
Granularité : au point d'intérêt\newline
Mise à jour : quotienne\newline
Popularité : 7 réutilisations,  14 suivis\newline
Mots-clé : \emph{accueil, accueil-de-jour, accueil-du-public, adresse, coordonnees-geographiques, fichier-unite, gendarmerie, gendarmerie-nationale, geocodage, obligations-de-service-public, point-geographique, stsisi, unite
}\newline
Permalien : \url{https://data.gouv.fr/dataset/5369993fa3a729239d2051cd}\newline

\par
\noindent
    \textbf{\emph{Couverture géographique}}

Ce jeu de données couvre certaines unités de gendarmerie (brigade,
centre d'informations et de recrutement, unités de sécurité routière)
permanentes ou temporaires, localisées en France métropolitaine et dans
les départements et collectivités d'outre-mer qui ont vocation à
accueillir du public.

\textbf{\emph{Origine des données}}

Le jeu de données provient de la Gendarmerie Nationale, service public
certifié. Les données sont agrégées à partir de plusieurs bases de
données et sont publiées sous licence ouverte.

Les horaires d'accueil journaliers des unités ne sont pas exclusifs
d'une disponibilité 24H/24, 7J/7 en cas d'urgence. Les Centres
Opérationnels de la Gendarmerie (COG) assurent également en permanence
un accueil téléphonique dédié aux urgences au travers du numéro de
téléphone 17 ainsi que la mise en relation avec les militaires des
unités territoriales.

\textbf{\emph{Format et contenu des fichiers}}

Deux fichiers CSV mis à jour chaque 24h sont disponibles. Pour chaque
unité, les attributs suivants sont ajoutées : identifiant public unité,
nom de l'unité, adresse géographique, téléphone, département, code INSEE
de la commune à 5 caractères, voie, code postal, commune,
geocodage\_epsg, geocodage x, geocodage y, geocodage x GPS, geocodage y
GPS, horaires d'ouverture.

Dans le 1er fichier (export-gn.csv) les horaires d'accueil sont
présentés dans un format lisible (text) sur une seule colonne. Dans le
2ème fichier (export-gn2.csv) les horaires d'accueil sont présentés sous
un format technique sur 48 colonnes. Deux colonnes pour chaque plage
horaire, 3 plages horaires maximum par jour. Les six dernières colonnes
correspondent aux horaires d'ouverture des jours fériés.


\vspace{0.5cm}
\needspace{12\baselineskip}
\subsection*{Liste et localisation des SGAMI
}\index{contours}\index{departements}\index{dsic}\index{gendarmerie}\index{police}\index{sgami}\index{stsisi}
  \begin{wrapfigure}{r}{2.5cm}
    \centering
    \qrcode[nolink]{https://data.gouv.fr/dataset/56a6b3fbc751df5795ade713}
  \end{wrapfigure}

Licence : \textbf{Licence Ouverte
}\newline
Créé le : 2016-01-26\newline
Modifié le : 2016-03-15\newline
Granularité : au pays\newline
Popularité : 1 réutilisation,  0 suivi\newline
Mots-clé : \emph{contours, departements, dsic, gendarmerie, police, sgami, stsisi
}\newline
Permalien : \url{https://data.gouv.fr/dataset/56a6b3fbc751df5795ade713}\newline

\par
\noindent
    Liste et localisation des Secrétariats Généraux pour l'Administration du
Ministère de l'Intérieur (SGAMI) en métropole.

Géométries de base : IGN Géofla (MultiPolygon).

4 fichiers :

\begin{itemize}

\item
  Un fichier SHP : ressorts des SGAMI (fusion de départements)
\item
  Un fichier SHP associant département et SGAMI, système de coordonnées
  WGS84 (EPSG:4326)
\item
  Un fichier SHP des implantations des SGAMI, géométrie : points,
  système de coordonnées WGS84 (EPSG:4326)
\item
  Un fichier de projet QGIS (.QGS) pour la visualisation.
\end{itemize}

Plus d'informations sur légifrance
:\url{http://legifrance.gouv.fr/affichTexte.do?cidTexte=JORFTEXT000028690922\&dateTexte=\&categorieLien=id}


\vspace{0.5cm}
\needspace{12\baselineskip}
\subsection*{Permis à points - année 2009
}
  \begin{wrapfigure}{r}{2.5cm}
    \centering
    \qrcode[nolink]{https://data.gouv.fr/dataset/53699c1da3a729239d2058ca}
  \end{wrapfigure}

Licence : \textbf{Licence Ouverte
}\newline
Créé le : 2013-07-08\newline
Modifié le : 2016-01-20\newline
De 2009-01-01 à 2009-12-31\newline
Mise à jour : annuelle\newline
Popularité : 1 réutilisation,  0 suivi\newline
Mots-clé : \emph{aucun
}\newline
Permalien : \url{https://data.gouv.fr/dataset/53699c1da3a729239d2058ca}\newline

\par
\noindent
    Données nationales (synthèse générale) et départementales sur le permis
à points (Nombre et évolution de chacune des infractions pour l'année
2009, nombre des infractions par famille d'infractions, nombre
d'infractions entraînant un retrait de points, nombre de points retirés,
nombre de permis invalidés pour solde de points nul, permis probatoires
invalidés pour solde de points nul, nombre des permis au capital de
points reconstitué)


\vspace{0.5cm}
\needspace{12\baselineskip}
\subsection*{Permis à points - année 2010
}
  \begin{wrapfigure}{r}{2.5cm}
    \centering
    \qrcode[nolink]{https://data.gouv.fr/dataset/53699c1da3a729239d2058cb}
  \end{wrapfigure}

Licence : \textbf{Licence Ouverte
}\newline
Créé le : 2013-07-08\newline
Modifié le : 2015-09-30\newline
De 2010-01-01 à 2010-12-31\newline
Mise à jour : annuelle\newline
Popularité : 1 réutilisation,  0 suivi\newline
Mots-clé : \emph{aucun
}\newline
Permalien : \url{https://data.gouv.fr/dataset/53699c1da3a729239d2058cb}\newline

\par
\noindent
    Données nationales (synthèse générale) et départementales sur le permis
à points (Nombre et évolution de chacune des infractions pour l'année
2010, nombre des infractions par famille d'infractions, nombre
d'infractions entraînant un retrait de points, nombre de points retirés,
nombre de permis invalidés pour solde de points nul, permis probatoires
invalidés pour solde de points nul, nombre des permis au capital de
points reconstitué)


\vspace{0.5cm}
\needspace{12\baselineskip}
\subsection*{Permis à points en France et DOM année 2010
}
  \begin{wrapfigure}{r}{2.5cm}
    \centering
    \qrcode[nolink]{https://data.gouv.fr/dataset/53699c1ea3a729239d2058cc}
  \end{wrapfigure}

Licence : \textbf{Licence Ouverte
}\newline
Créé le : 2013-07-08\newline
Modifié le : 2016-02-06\newline
De 2010-01-01 à 2010-12-31\newline
Mise à jour : annuelle\newline
Popularité : 1 réutilisation,  0 suivi\newline
Mots-clé : \emph{aucun
}\newline
Permalien : \url{https://data.gouv.fr/dataset/53699c1ea3a729239d2058cc}\newline

\par
\noindent
    Données nationales (synthèse générale) et départementales sur le permis
à points (Nombre et évolution de chacune des infractions pour l'année
2010, nombre des infractions par famille d'infractions, nombre
d'infractions entraînant un retrait de points, nombre de points retirés,
nombre de permis invalidés pour solde de points nul, permis probatoires
invalidés pour solde de points nul, nombre des permis au capital de
points reconstitué)


\vspace{0.5cm}
\needspace{12\baselineskip}
\subsection*{Places de stationnement réservées aux véhicules utilisés par les
personnes titulaires d'une carte européenne de stationnement
}\index{carte}\index{europeenne}\index{gic}\index{gig}\index{gig!gic}\index{handicap}\index{handicapes}\index{stationnement}
  \begin{wrapfigure}{r}{2.5cm}
    \centering
    \qrcode[nolink]{https://data.gouv.fr/dataset/53699c93a3a729239d2059ea}
  \end{wrapfigure}

Licence : \textbf{Licence Ouverte
}\newline
Créé le : 2013-07-08\newline
Modifié le : 2016-03-13\newline
De 2015-09-28 à 2016-09-28\newline
Mise à jour : annuelle\newline
Popularité : 1 réutilisation,  0 suivi\newline
Mots-clé : \emph{carte, europeenne, gic, gig, gig-gic, handicap, handicapes, stationnement
}\newline
Permalien : \url{https://data.gouv.fr/dataset/53699c93a3a729239d2059ea}\newline

\par
\noindent
    Cartographie des places de stationnement réservées aux personnes
titulaires d'une carte européenne de stationnement.


\vspace{0.5cm}
\needspace{12\baselineskip}
\subsection*{Points d'accueil police (coordonnées) - Paris
}\index{paris}\index{police}\index{prefecture}\index{prefecture!de!police}\index{prefecture!de!police!de!paris}
  \begin{wrapfigure}{r}{2.5cm}
    \centering
    \qrcode[nolink]{https://data.gouv.fr/dataset/53699cf4a3a729239d205ae9}
  \end{wrapfigure}

Licence : \textbf{Licence Ouverte
}\newline
Créé le : 2013-07-08\newline
Modifié le : 2018-11-16\newline
De 2017-01-11 à 2018-01-01\newline
Mise à jour : annuelle\newline
Popularité : 1 réutilisation,  0 suivi\newline
Mots-clé : \emph{paris, police, prefecture, prefecture-de-police, prefecture-de-police-de-paris
}\newline
Permalien : \url{https://data.gouv.fr/dataset/53699cf4a3a729239d205ae9}\newline

\par
\noindent
    Liste des points d'accueil police pour Paris répertoriant par
arrondissement les services, adresses, téléphones, horaires, conditions
d'accessibilité.


\vspace{0.5cm}
\needspace{12\baselineskip}
\subsection*{Points d'accueil police (coordonnées) - Petite couronne
}
  \begin{wrapfigure}{r}{2.5cm}
    \centering
    \qrcode[nolink]{https://data.gouv.fr/dataset/53699cf5a3a729239d205aea}
  \end{wrapfigure}

Licence : \textbf{Licence Ouverte
}\newline
Créé le : 2013-07-08\newline
Modifié le : 2017-10-30\newline
De 2011-01-01 à 2011-01-01\newline
Mise à jour : annuelle\newline
Popularité : 1 réutilisation,  0 suivi\newline
Mots-clé : \emph{aucun
}\newline
Permalien : \url{https://data.gouv.fr/dataset/53699cf5a3a729239d205aea}\newline

\par
\noindent
    Liste des points d'accueil police pour les départements de petite
couronne (92, 93, 94) répertoriant par district les services, adresses,
téléphones, horaires


\vspace{0.5cm}
\needspace{12\baselineskip}
\subsection*{Points d'accueil police (coordonnées) - Province
}
  \begin{wrapfigure}{r}{2.5cm}
    \centering
    \qrcode[nolink]{https://data.gouv.fr/dataset/53699cf5a3a729239d205aeb}
  \end{wrapfigure}

Licence : \textbf{Licence Ouverte
}\newline
Créé le : 2013-07-08\newline
Modifié le : 2015-12-28\newline
De 2011-01-01 à 2011-01-01\newline
Mise à jour : annuelle\newline
Popularité : 1 réutilisation,  0 suivi\newline
Mots-clé : \emph{aucun
}\newline
Permalien : \url{https://data.gouv.fr/dataset/53699cf5a3a729239d205aeb}\newline

\par
\noindent
    Liste des points d'accueil police pour la province répertoriant par
département les services, adresses, téléphones, horaires, conditions
d'accessibilité.


\vspace{0.5cm}
\needspace{12\baselineskip}
\subsection*{Police municipale : Effectifs par commune
}\index{2012}\index{communes}\index{dlpaj}\index{polices!municipales}
  \begin{wrapfigure}{r}{2.5cm}
    \centering
    \qrcode[nolink]{https://data.gouv.fr/dataset/5369986ba3a729239d204f55}
  \end{wrapfigure}

Licence : \textbf{Licence Ouverte
}\newline
Créé le : 2014-01-16\newline
Modifié le : 2017-08-17\newline
Granularité : à la commune\newline
Mise à jour : annuelle\newline
Popularité : 7 réutilisations,  13 suivis\newline
Mots-clé : \emph{2012, communes, dlpaj, polices-municipales
}\newline
Permalien : \url{https://data.gouv.fr/dataset/5369986ba3a729239d204f55}\newline

\par
\noindent
    Les fichiers recensent pour les années 2012 à 2016 les communes de votre
département ayant un service de police municipale et contient :

\begin{itemize}

\item
  le nombre de policiers municipaux,
\item
  le nombre d'agents de surveillance de la voie publique,
\item
  le nombre de gardes-champêtres,
\item
  le nombre de centres de supervision urbaine (CSU).
\end{itemize}

Pour les années 2013,2014,2015 et 2016 il est précisé le nombre de
brigades canines par commune.

Les polices municipales sont régies par le code général des
collectivités territoriales, le code de procédure pénale et le livre V
du code de la sécurité intérieure, qui ont défini leur organisation et
leur fonctionnement. Placés sous l'autorité des maires, les agents de
police municipale disposent de compétences de police administrative et
de certaines compétences de police judiciaire définies par la loi,
qu'ils exercent sous le contrôle du procureur de la République.

\textbf{Découvrez en bas de cette page les contributions de la
communauté data.gouv.fr en lien avec ce jeu de données }


\vspace{0.5cm}
\needspace{12\baselineskip}
\subsection*{Répertoire National des Associations
}\index{1901}\index{asso}\index{association}\index{associations}\index{loi!1901}\index{rna}\index{waldec}
  \begin{wrapfigure}{r}{2.5cm}
    \centering
    \qrcode[nolink]{https://data.gouv.fr/dataset/58e53811c751df03df38f42d}
  \end{wrapfigure}

Licence : \textbf{Licence Ouverte
}\newline
Créé le : 2017-04-05\newline
Modifié le : 2019-03-09\newline
De 1901-01-01 à 2017-04-05\newline
Granularité : au point d'intérêt\newline
Mise à jour : mensuelle\newline
Popularité : 6 réutilisations,  22 suivis\newline
Mots-clé : \emph{1901, asso, association, associations, loi-1901, rna, waldec
}\newline
Permalien : \url{https://data.gouv.fr/dataset/58e53811c751df03df38f42d}\newline

\par
\noindent
    Le Répertoire National des Associations (RNA) contient l'ensemble des
associations relevant de la loi 1901, à savoir toutes les associations
de France, dont le siège est déclaré en métropole ou dans les
départements d'outre-mer, sauf dans les départements de la Moselle, du
Bas-Rhin et du Haut-Rhin, qui relèvent du régime du Concordat. Le RNA
contient également les associations reconnues d'utilité publique (dites
``ARUP'').

Les associations relevant de la loi 1901 sont déclarées en préfecture ou
en sous-préfecture (au greffe des associations) : la création et les
changements de statuts, tels que la modification du nom, du siège, des
dirigeants, etc. doivent être déclarés et sont mis à jour dans la base
du RNA. La mise à jour est effective une fois les données validées par
le greffe ou suite à la publication au Journal officiel des associations
et des fondations d'entreprise (JOAFE) d'une création, d'une dissolution
(obligatoire) ou d'un changement de situation (publication au JOAFE non
obligatoires).

La base du Répertoire National des Associations (RNA) est alimentée par
la validation par les greffes des associations et la publication au
Journal officiel des associations et des fondations d'entreprise des
créations et modifications des associations relevant de la loi 1901.

\textbf{Diffusion des données du Répertoire National des Associations
(RNA)}

Conformément aux dispositions de la loi pour une République numérique,
les données du Répertoire National des Associations (RNA), produites par
la Direction des libertés publiques et des affaires juridiques (DLPAJ)
du Ministère de l'intérieur, sont aujourd'hui accessibles ci-dessous.

Cette base, qui comprend toutes les associations relevant de la loi 1901
intègre désormais le service public de la donnée. Vous pouvez maintenant
télécharger l'intégralité de la base, ainsi que la documentation
associée. Les mises à jour quotidiennes seront également téléchargeables
prochainement.

Le contenu des données à télécharger est scindé en deux extractions :

\emph{RNA\_waldec} : liste des associations disposant d'un n\degree{}
RNA. Toutes les associations créées ou ayant déclaré un changement de
situation depuis 2009 disposent d'un n\degree{} RNA.

\emph{RNA\_import} : liste des associations créées depuis 1901 et qui
n'ont pas effectué de déclaration de changement de situation depuis
2009.

\emph{RNA\_Liste\_donnees\_diffusees} : Ce fichier décrit les données
exposées par le RNA pour les fichiers typés ``import'' et ``waldec''.

\textbf{Contenu des extractions :}

\begin{itemize}

\item
  le cas échéant, le n\degree{} RNA
\item
  le nom de l'association et son sigle
\item
  l'objet de l'association et son objet social
\item
  l'adresse du siège
\item
  le cas échéant, l'adresse de gestion
\item
  le cas échéant, le site internet de l'association
\end{itemize}

\textbf{Fréquence d'extraction}

La fréquence est pour le moment mensuelle.


\vspace{0.5cm}
\needspace{12\baselineskip}
\subsection*{Statistiques admissions au sejour par motif
}
  \begin{wrapfigure}{r}{2.5cm}
    \centering
    \qrcode[nolink]{https://data.gouv.fr/dataset/5369a04fa3a729239d206331}
  \end{wrapfigure}

Licence : \textbf{Licence Ouverte
}\newline
Créé le : 2013-07-08\newline
Modifié le : 2015-03-29\newline
De 2009-01-01 à 2009-12-31\newline
Popularité : 1 réutilisation,  1 suivi\newline
Mots-clé : \emph{aucun
}\newline
Permalien : \url{https://data.gouv.fr/dataset/5369a04fa3a729239d206331}\newline

\par
\noindent
    Admission au séjour Titre de séjour (DSED)


\vspace{0.5cm}
\needspace{12\baselineskip}
\subsection*{Subventions allouées au titre de la ``réserve parlementaire''
}\index{communes}\index{subventions!aux!communes}
  \begin{wrapfigure}{r}{2.5cm}
    \centering
    \qrcode[nolink]{https://data.gouv.fr/dataset/5369a10fa3a729239d2064ee}
  \end{wrapfigure}

Licence : \textbf{Licence Ouverte
}\newline
Créé le : 2013-08-16\newline
Modifié le : 2016-02-27\newline
De 2011-01-01 à 2012-12-31\newline
Granularité : à la commune\newline
Mise à jour : annuelle\newline
Popularité : 4 réutilisations,  4 suivis\newline
Mots-clé : \emph{communes, subventions-aux-communes
}\newline
Permalien : \url{https://data.gouv.fr/dataset/5369a10fa3a729239d2064ee}\newline

\par
\noindent
    Le ministère de l'Intérieur assure l'exécution de l'action 01 du
programme budgétaire 122, « aides exceptionnelles aux collectivités
territoriales ». Ce programme comprend les subventions d'investissement
des collectivités territoriales accordées au titre de la « réserve
parlementaire ».

Le ministère de l'Intérieur communique chaque année à la Cour des
comptes les montant alloués par département au titre de la réserve
parlementaire et publie au sein de cette rubrique les tableaux
récapitulatifs d'attribution de ces subventions depuis l'année 2011


\vspace{0.5cm}
\needspace{12\baselineskip}
\subsection*{Subventions attribuées au titre de la réserve ministérielle
}\index{ministerielle}\index{reserve}\index{subvention}\index{subventions}
  \begin{wrapfigure}{r}{2.5cm}
    \centering
    \qrcode[nolink]{https://data.gouv.fr/dataset/5369a10fa3a729239d2064f3}
  \end{wrapfigure}

Licence : \textbf{Licence Ouverte
}\newline
Créé le : 2013-08-16\newline
Modifié le : 2018-04-10\newline
De 2012-01-01 à 2013-12-31\newline
Granularité : à la commune\newline
Mise à jour : annuelle\newline
Popularité : 2 réutilisations,  1 suivi\newline
Mots-clé : \emph{ministerielle, reserve, subvention, subventions
}\newline
Permalien : \url{https://data.gouv.fr/dataset/5369a10fa3a729239d2064f3}\newline

\par
\noindent
    Le ministère de l'Intérieur assure l'exécution de l'action 01 du
programme budgétaire 122, « aides exceptionnelles aux collectivités
territoriales ». Ce programme comprend les subventions d'investissement
des collectivités territoriales accordées au titre de la « réserve
parlementaire » et de la « réserve ministérielle ».


\vspace{0.5cm}
\needspace{12\baselineskip}
\subsection*{Vidéoprotection - Implantation des caméras - KML + ods
}\index{dispositif!de!securite}\index{police}\index{prefecture}\index{protection}\index{securite!publique}\index{surveillance}\index{video}\index{videoprotection}\index{videoprotectioncamerasprotection}\index{videosurveillance}
  \begin{wrapfigure}{r}{2.5cm}
    \centering
    \qrcode[nolink]{https://data.gouv.fr/dataset/5369a36ba3a729239d206a70}
  \end{wrapfigure}

Licence : \textbf{Licence Ouverte
}\newline
Créé le : 2013-07-08\newline
Modifié le : 2018-11-16\newline
De 2011-12-21 à 2012-08-31\newline
Mise à jour : annuelle\newline
Popularité : 3 réutilisations,  6 suivis\newline
Mots-clé : \emph{dispositif-de-securite, police, prefecture, protection, securite-publique, surveillance, video, videoprotection, videoprotectioncamerasprotection, videosurveillance
}\newline
Permalien : \url{https://data.gouv.fr/dataset/5369a36ba3a729239d206a70}\newline

\par
\noindent
    Implantation des caméras prévues par le plan de vidéoprotection pour
Paris (été 2012).200 caméras sont opérationnelles pour la mise en
service de la première tranche du projet le 21 décembre 2011. 1105
caméras seront opérationnelles pour la livraison globale du système PVPP
prévue à l'été 2012


\vspace{0.5cm}
\needspace{12\baselineskip}
\subsection*{Zones de défense et de sécurité 2016
}\index{defense}\index{prefecture}\index{securite}\index{stsisi}\index{zds}\index{zonage!reglementaire}
  \begin{wrapfigure}{r}{2.5cm}
    \centering
    \qrcode[nolink]{https://data.gouv.fr/dataset/57ed25b388ee384cf05ff490}
  \end{wrapfigure}

Licence : \textbf{Open Data Commons Open Database License (ODbL)
}\newline
Créé le : 2016-09-29\newline
Modifié le : 2017-01-11\newline
Granularité : au pays\newline
Mise à jour : annuelle\newline
Popularité : 1 réutilisation,  1 suivi\newline
Mots-clé : \emph{defense, prefecture, securite, stsisi, zds, zonage-reglementaire
}\newline
Permalien : \url{https://data.gouv.fr/dataset/57ed25b388ee384cf05ff490}\newline

\par
\noindent
    Découpages des zones de défense et de sécurité modifiées en 2016 suite à
la réforme territoriale des régions (loi NOTRe).

Le système de coordonnées est WGS84 (EPSG:4326)

Chaque zone géographique possède des attributs : - id - Nom - siège -
ressort (régions) - population - superficie (en km\textsuperscript{2})

Source :

\begin{itemize}

\item
  Texte définissant les zones de défense (métropole)
  :\url{https://www.legifrance.gouv.fr/affichCodeArticle.do?idArticle=LEGIARTI000006574206\&cidTexte=LEGITEXT000006071307}-
  Zones de défense en outre-mer (ministère de la santé)
  :\url{http://social-sante.gouv.fr/IMG/pdf/Zones_de_defense_et_de_securite_en_Outre-mer.pdf}
\end{itemize}


\vspace{0.5cm}
\needspace{3\baselineskip} \rule{4cm}{0.25pt}\newline\textbf{Aussi disponible du même producteur :}\begin{itemize}
\item \href{https://data.gouv.fr/dataset/53698e09a3a729239d203370}{01 - Tableau synthétique de l’évolution de la demande et de la délivrance pour les principales catégories de visas}
\item \href{https://data.gouv.fr/dataset/55c8a8f688ee3812d6a46ec1}{ 15001 - Stock de titres et autorisations provisoires de séjour en cours de validité}
\item \href{https://data.gouv.fr/dataset/55c8ab3688ee386070a46ec2}{15002 - Stock de titres et autorisations provisoires de séjour en cours de validité par durée de titre (pays tiers)}
\item \href{https://data.gouv.fr/dataset/55c8ae8488ee3814a9a46ec1}{15003 - Total des titres par année et lieu d’enregistrement}
\item \href{https://data.gouv.fr/dataset/55c8b08488ee382617a46ec1}{15004 - Stock par type de document et motifs (pays tiers), métropole seulement}
\item \href{https://data.gouv.fr/dataset/55c8b25488ee382617a46ec2}{15005 - Détail des titres valides}
\item \href{https://data.gouv.fr/dataset/55c8b4fc88ee380f5ea46ec1}{15006 - Stock de titres et autorisations provisoires de séjour en cours de validité par nationalité (pays tiers)}
\item \href{https://data.gouv.fr/dataset/55c8b79288ee3812d6a46ec2}{15007 - Stock de titres et autorisations provisoires de séjour en cours de validité par département}
\item \href{https://data.gouv.fr/dataset/55c8b96c88ee3814a9a46ec2}{15008 - Présence étrangère dans les pays de l’OCDE}
\item \href{https://data.gouv.fr/dataset/55c8bad488ee387c7da46ec1}{15009 - Admission au séjour des ressortissants de pays tiers}
\item \href{https://data.gouv.fr/dataset/55c8bc1088ee38397ba46ec1}{15010 - Flux d’immigration en provenance des pays tiers, OCDE (en milliers)}
\item \href{https://data.gouv.fr/dataset/55c8c21788ee384067a46ec1}{15011 - Regroupement des titres par motif juridique}
\item \href{https://data.gouv.fr/dataset/55c8c30a88ee384067a46ec2}{15012 - Admission au séjour des ressortissants de pays tiers}
\item \href{https://data.gouv.fr/dataset/55c8c41988ee384067a46ec3}{15013 - Nombre de titres délivrés aux ressortissants des pays tiers}
\item \href{https://data.gouv.fr/dataset/55c8c6e988ee384067a46ec4}{15014 - Admission au séjour des ressortissants des nouveaux États membres soumis à dispositions transitoires}
\item \href{https://data.gouv.fr/dataset/55c8c7df88ee38397ba46ec2}{15015 - Admission au séjour - les dix premières nationalités par motif}
\item \href{https://data.gouv.fr/dataset/55c9b4ad88ee383a10a46ec1}{15016 - Aperçu sur les migrants économiques}
\item \href{https://data.gouv.fr/dataset/55c9b5d588ee383153a46ec3}{15017 - Détail des titres accordés au motif - famille de Français - (pays tiers)}
\item \href{https://data.gouv.fr/dataset/55c9b6e288ee387e2ca46ec1}{15018 - Signataires du CAI : Aperçu sur les migrants familiaux}
\item \href{https://data.gouv.fr/dataset/55c9b80688ee3825f1a46ec1}{15019 - Permis délivrés pour la première fois dans l’Union européenne à 27 pays}
\item \href{https://data.gouv.fr/dataset/55c9b91b88ee38226ea46ec1}{15020 - Population, population née à l’étranger, proportions (millions, \%)}
\item \href{https://data.gouv.fr/dataset/55c9b9ed88ee383153a46ec4}{15021 - Population, population étrangère (millions, \%)}
\item \href{https://data.gouv.fr/dataset/55c9bae188ee383153a46ec5}{15022 - Acquisition de la nationalité du pays de résidence}
\item \href{https://data.gouv.fr/dataset/55c9bbbf88ee387e2ca46ec2}{15023 - Flux d'immigration 2012}
\item \href{https://data.gouv.fr/dataset/55c9bd0988ee3825f1a46ec3}{15024 - Flux d'émigrations 2012}
\item \href{https://data.gouv.fr/dataset/55ca015f88ee382815a46ec1}{15030 - Tableau synthétique de l'évolution de la demande et de la délivrance pour les principales catégories de visas}
\item \href{https://data.gouv.fr/dataset/55ca02e188ee38362aa46ec1}{15031 - L'évolution de la délivrance des visas de courts séjours}
\item \href{https://data.gouv.fr/dataset/55ca046088ee38114fa46ec1}{15032 - Visas délivrés aux étudiants}
\item \href{https://data.gouv.fr/dataset/55ca062388ee382815a46ec2}{15033 - Visas délivrés aux conjoints de Français}
\item \href{https://data.gouv.fr/dataset/55ca093088ee38362aa46ec2}{15034 - Visas délivrés au titre du regroupement familial}
\item \href{https://data.gouv.fr/dataset/55ca0fc088ee3845cfa46ec2}{15035 - Les visas pour les familles des réfugiés}
\item \href{https://data.gouv.fr/dataset/55ca10c788ee38114da46ec2}{15036 - Visas délivrés pour l'établissement de mineurs en France}
\item \href{https://data.gouv.fr/dataset/55ca11cd88ee38362aa46ec3}{15037 - Visas délivrés pour l'établissement professionnel}
\item \href{https://data.gouv.fr/dataset/55ca146488ee38362aa46ec4}{15038 - Visas pour les départements, les collectivités et les territoires d'outre-mer}
\item \href{https://data.gouv.fr/dataset/55ca153088ee384ce3a46ec1}{15039 - Les visas délivrés sur passeport diplomatique ou de service}
\item \href{https://data.gouv.fr/dataset/55ca162e88ee3845cfa46ec3}{15040 - Répartition par zone géographique des visas délivrés}
\item \href{https://data.gouv.fr/dataset/55ca170f88ee384ce3a46ec2}{15041 -	Visas délivrés pour les 15 principaux pays}
\item \href{https://data.gouv.fr/dataset/55ca195d88ee38576ca46ec2}{15043 - Évolution du nombre des recours enregistrés et examinés par la CRRV}
\item \href{https://data.gouv.fr/dataset/55ca1a4c88ee3845cfa46ec4}{15044 - Les différents recours formés devant la juridiction administrative}
\item \href{https://data.gouv.fr/dataset/55ca1ae988ee385941a46ec1}{15045 - Décisions rendues par la juridiction administrative}
\item \href{https://data.gouv.fr/dataset/55d19e0288ee3820c6a46ec1}{15050 - Délivrance de premiers titres de séjour}
\item \href{https://data.gouv.fr/dataset/55d1a0d988ee3820c6a46ec2}{15051 - Carte de séjour temporaire délivrés aux primo-arrivants (métropole)}
\item \href{https://data.gouv.fr/dataset/55d1a2e988ee3820c6a46ec3}{15052 - Nombre de visas de longs séjours valant titre de séjour}
\item \href{https://data.gouv.fr/dataset/55d1a3ae88ee3820c6a46ec4}{15053 - Cartes de retraité}
\item \href{https://data.gouv.fr/dataset/55d1a6c888ee3812bfa46ec1}{15054 - Certificats de résidence pour Algériens}
\item \href{https://data.gouv.fr/dataset/55d1aad488ee383afca46ec1}{15055 - Premiers titres de séjour communautaires et titres - Espace économique européen}
\item \href{https://data.gouv.fr/dataset/55d1af3c88ee383afca46ec2}{15056 - Cartes de résident délivrées pour la première fois (en premier titre de séjour ou après un titre de séjour}
\item \href{https://data.gouv.fr/dataset/55d1b13688ee383afca46ec3}{15057 - Cartes de résident délivrées pour la première fois (après un autre titre de séjour ou pas) aux conjoints de Français ou parents d'enfants français}
\item \href{https://data.gouv.fr/dataset/55d1b4a988ee384740a46ec1}{15058 - Volumes d'admission exceptionnelle au séjour des ressortissants étrangers}
\item \href{https://data.gouv.fr/dataset/55d34b0ac751df3cdb1f92b2}{15060 - Non-admissions et réadmissions simplifiées}
\item et 406 autres jeux de données\end{itemize}

\clearpage
\section{Ministère des Armées}


\begin{center}
  \includegraphics[width=3cm]{images/orga/2f_257154e5f946b1aa3990a605688666-100.png}
\end{center}


La Constitution de la Ve République régit l'organisation des pouvoirs en
matière de défense entre le Président de la République (Chef des
Armées), le Premier ministre et le Ministre des armées.

Le Ministre des armées prépare et met en œuvre la politique de défense
dont il assume, avec le Premier ministre, la responsabilité devant le
Parlement. Un ministre délégué auprès du ministre de la défense est
chargé des anciens combattants.


\vspace{0.5cm}

\needspace{12\baselineskip}
\subsection*{Evolution des effectifs totaux de la défense (hors Gendarmerie)
}
  \begin{wrapfigure}{r}{2.5cm}
    \centering
    \qrcode[nolink]{https://data.gouv.fr/dataset/53699536a3a729239d204668}
  \end{wrapfigure}

Licence : \textbf{Licence Ouverte
}\newline
Créé le : 2013-07-08\newline
Modifié le : 2015-09-03\newline
De 2001-01-01 à 2010-12-31\newline
Mise à jour : annuelle\newline
Popularité : 1 réutilisation,  0 suivi\newline
Mots-clé : \emph{aucun
}\newline
Permalien : \url{https://data.gouv.fr/dataset/53699536a3a729239d204668}\newline

\par
\noindent
    Série de 2001 à 2010


\vspace{0.5cm}
\needspace{12\baselineskip}
\subsection*{Evolution du personnel civil de la défense par statut
}
  \begin{wrapfigure}{r}{2.5cm}
    \centering
    \qrcode[nolink]{https://data.gouv.fr/dataset/5369953ba3a729239d204677}
  \end{wrapfigure}

Licence : \textbf{Licence Ouverte
}\newline
Créé le : 2013-07-08\newline
Modifié le : 2016-02-01\newline
De 2007-01-01 à 2010-12-31\newline
Granularité : au pays\newline
Mise à jour : annuelle\newline
Popularité : 1 réutilisation,  0 suivi\newline
Mots-clé : \emph{aucun
}\newline
Permalien : \url{https://data.gouv.fr/dataset/5369953ba3a729239d204677}\newline

\par
\noindent
    Evolution du personnel civil de la défense par statut


\vspace{0.5cm}
\needspace{12\baselineskip}
\subsection*{Part des dépenses de fonctionnement (hors pensions) dans le PIB en norme
Otan
}
  \begin{wrapfigure}{r}{2.5cm}
    \centering
    \qrcode[nolink]{https://data.gouv.fr/dataset/53699b91a3a729239d205783}
  \end{wrapfigure}

Licence : \textbf{Licence Ouverte
}\newline
Créé le : 2013-07-08\newline
Modifié le : 2016-02-01\newline
Mise à jour : annuelle\newline
Popularité : 1 réutilisation,  0 suivi\newline
Mots-clé : \emph{aucun
}\newline
Permalien : \url{https://data.gouv.fr/dataset/53699b91a3a729239d205783}\newline

\par
\noindent
    Part des dépenses de fonctionnement dans le PIB en norme Otan (France,
Allemagne, Royaume-Uni, Espagne, Italie et Etats-Unis)Série de 2001 à
2010


\vspace{0.5cm}
\needspace{12\baselineskip}
\subsection*{Surcoûts des opérations extérieures
}
  \begin{wrapfigure}{r}{2.5cm}
    \centering
    \qrcode[nolink]{https://data.gouv.fr/dataset/5369a12da3a729239d20653c}
  \end{wrapfigure}

Licence : \textbf{Licence Ouverte
}\newline
Créé le : 2013-07-08\newline
Modifié le : 2016-02-01\newline
De 2003-12-31 à 2012-12-31\newline
Mise à jour : annuelle\newline
Popularité : 1 réutilisation,  0 suivi\newline
Mots-clé : \emph{aucun
}\newline
Permalien : \url{https://data.gouv.fr/dataset/5369a12da3a729239d20653c}\newline

\par
\noindent
    

\vspace{0.5cm}
\needspace{3\baselineskip} \rule{4cm}{0.25pt}\newline\textbf{Aussi disponible du même producteur :}\begin{itemize}
\item \href{https://data.gouv.fr/dataset/55b499db88ee3846ba3ca288}{1-les congés de reconversion 2014}
\item \href{https://data.gouv.fr/dataset/53d37b7ca3a729042701368f}{Accompagnement des conjoints du personnel de la défense et de la gendarmerie 2013}
\item \href{https://data.gouv.fr/dataset/53698e52a3a729239d203427}{Accompagnement des conjoints du personnels de la Défense et de la gendarmerie}
\item \href{https://data.gouv.fr/dataset/53698e56a3a729239d203432}{Achats du ministère de la défense et des anciens combattants}
\item \href{https://data.gouv.fr/dataset/53698e7ca3a729239d20349e}{Adresses géographiques et téléphoniques des centres d'information et de recrutement des forces armées (CIRFA)}
\item \href{https://data.gouv.fr/dataset/57f63dee88ee384ee85ff490}{Annuaire statistique de la défense - édition 2016}
\item \href{https://data.gouv.fr/dataset/53698f1ca3a729239d203659}{Assymétrie de l'effort de défense entre l'Union européenne et les Etats-Unis}
\item \href{https://data.gouv.fr/dataset/53698f5ba3a729239d203717}{Bases de défense}
\item \href{https://data.gouv.fr/dataset/53698f8fa3a729239d20379e}{Bibliothèques de la défense ouvertes au public}
\item \href{https://data.gouv.fr/dataset/53698f92a3a729239d2037a5}{Bilan activité (JDC)}
\item \href{https://data.gouv.fr/dataset/53698f96a3a729239d2037ad}{Bilan des contrats de redynamisation et des plans locaux de redynamisation du ministère de la défense}
\item \href{https://data.gouv.fr/dataset/53698f9ca3a729239d2037bb}{Bilan reconversion}
\item \href{https://data.gouv.fr/dataset/53698f9ea3a729239d2037bf}{Bilan secourisme (JDC)}
\item \href{https://data.gouv.fr/dataset/53698fc1a3a729239d203819}{Budget de la défense - Dépenses en capital par habitant en norme Otan}
\item \href{https://data.gouv.fr/dataset/53698fc1a3a729239d20381a}{Budget de la défense ventilé selon les rubrique de comptabilité nationale}
\item \href{https://data.gouv.fr/dataset/53698fc4a3a729239d203821}{Budget du ministère de la défense par destination et nature}
\item \href{https://data.gouv.fr/dataset/53698ff1a3a729239d203896}{CADETS DE LA DEFENSE}
\item \href{https://data.gouv.fr/dataset/53699056a3a729239d20399d}{centres d'archives du ministère de la défense et des anciens combattants}
\item \href{https://data.gouv.fr/dataset/5369905ca3a729239d2039ad}{Centres d'Information et de Recrutement des Forces Armées (CIRFA)}
\item \href{https://data.gouv.fr/dataset/5369905da3a729239d2039b0}{Centres du Service National (JDC)}
\item \href{https://data.gouv.fr/dataset/536990c1a3a729239d203aaa}{CLASSES DEFENSE ET SECURITE GLOBALES}
\item \href{https://data.gouv.fr/dataset/53699123a3a729239d203ba0}{Comparaison de l'emploi dans la branche BA2000 et de l'emploi manufacturier}
\item \href{https://data.gouv.fr/dataset/53699124a3a729239d203ba1}{Comparaison des principaux soldes extérieurs}
\item \href{https://data.gouv.fr/dataset/53699124a3a729239d203ba2}{Comparaison internationale des dépenses de R\&{}D en 2008 et 2009}
\item \href{https://data.gouv.fr/dataset/53699124a3a729239d203ba3}{Comparaison internationale des effectifs de défense\_effectifs civils au sens de l'Otan}
\item \href{https://data.gouv.fr/dataset/53699125a3a729239d203ba4}{Comparaison internationale des effectifs de défense\_Effectifs militaires en norme Otan}
\item \href{https://data.gouv.fr/dataset/53699125a3a729239d203ba6}{Comparaison internationale des effectifs de défense\_Evolution des effectifs totaux au sens de l'Otan}
\item \href{https://data.gouv.fr/dataset/56698a21c751df0c4bc664bd}{COMPARAISONS INTERNATIONALES  LA DEFENSE DANS LES GRANDS PAYS INDUSTRIALISES (2006 - 2014)}
\item \href{https://data.gouv.fr/dataset/53699126a3a729239d203ba8}{Comparaisons internationales - PIB nominal}
\item \href{https://data.gouv.fr/dataset/5369914ba3a729239d203c0c}{Compte satellite de la défense - Tableau des entrées-sorties}
\item \href{https://data.gouv.fr/dataset/5369914ba3a729239d203c0d}{Compte satellite de la Défense - Tableau des entrées-Sorties (TES) 2009}
\item \href{https://data.gouv.fr/dataset/536991e7a3a729239d203da8}{coût de la reconversion}
\item \href{https://data.gouv.fr/dataset/53d37fa0a3a7290427013692}{Coût de la reconversion 2013}
\item \href{https://data.gouv.fr/dataset/53699237a3a729239d203e7c}{DEFENSE 2EME CHANCE}
\item \href{https://data.gouv.fr/dataset/53699263a3a729239d203ef5}{Démographie du personnel militaire de la défense}
\item \href{https://data.gouv.fr/dataset/53699281a3a729239d203f44}{Dépenses de défense par habitant (hors pensions) en norme Otan}
\item \href{https://data.gouv.fr/dataset/53699283a3a729239d203f48}{Dépenses de fonctionnement en volume et en norme Otan (hors pensions)}
\item \href{https://data.gouv.fr/dataset/53699283a3a729239d203f49}{Dépenses de fonctionnement par habitant (hors pensions) en norme Otan}
\item \href{https://data.gouv.fr/dataset/53699283a3a729239d203f4a}{Dépenses de fonctionnement rapportées aux effectifs totaux de la défense}
\item \href{https://data.gouv.fr/dataset/53699286a3a729239d203f52}{Dépenses d'équipement des trois armées et des services communs}
\item \href{https://data.gouv.fr/dataset/5369928da3a729239d203f63}{Dépenses des administrations publiques par fonction de dépense}
\item \href{https://data.gouv.fr/dataset/5369928ea3a729239d203f64}{Dépenses des administrations publiques par fonction de dépense}
\item \href{https://data.gouv.fr/dataset/536992a1a3a729239d203f96}{Dépenses en capital ( budgets de la défense) en volume et en norme Otan}
\item \href{https://data.gouv.fr/dataset/536992a2a3a729239d203f97}{Dépenses en capital rapportées aux effectifs totaux de la défense}
\item \href{https://data.gouv.fr/dataset/536992a6a3a729239d203fa3}{Dépenses et recettes des budgets exécutés de l'Etat et de la défense}
\item \href{https://data.gouv.fr/dataset/536992a6a3a729239d203fa4}{Dépenses hors équipement et dépenses d'équipement des trois armées et des services communs}
\item \href{https://data.gouv.fr/dataset/536992a6a3a729239d203fa5}{Dépenses hors équipement et équipement}
\item \href{https://data.gouv.fr/dataset/536992c4a3a729239d203fee}{Dépenses publiques de Défense et sécurité par habitant en 2008 dans les principaux pays européens}
\item \href{https://data.gouv.fr/dataset/536992d0a3a729239d20400f}{Destination des exportations françaises d'armement (commandes) par région géographique}
\item \href{https://data.gouv.fr/dataset/536992d1a3a729239d204010}{Destination des exportations françaises d'armement (livraisons) par région géographique}
\item et 138 autres jeux de données\end{itemize}

\clearpage
\section{Ministère des Solidarités et de la Santé}


\begin{center}
  \includegraphics[width=3cm]{images/orga/3c_2a35d973814c2681b6941e11810ff3-100.png}
\end{center}


Le ministère des solidarités et de la santé prépare et met en œuvre la
politique du Gouvernement dans les domaines des affaires sociales, de la
solidarité et de la cohésion sociale, de la santé publique et de
l'organisation du système de soins. Sous réserve des compétences du
ministre de l'économie et des finances, il prépare et met en œuvre la
politique du Gouvernement dans le domaine de la protection sociale.


\vspace{0.5cm}

\needspace{12\baselineskip}
\subsection*{Allocataires du RSA
}\index{allocataires}\index{rsa}
  \begin{wrapfigure}{r}{2.5cm}
    \centering
    \qrcode[nolink]{https://data.gouv.fr/dataset/53698ed1a3a729239d20358c}
  \end{wrapfigure}

Licence : \textbf{Licence Ouverte
}\newline
Créé le : 2013-07-08\newline
Modifié le : 2016-02-02\newline
De 2009-12-31 à 2010-12-31\newline
Granularité : au département\newline
Mise à jour : annuelle\newline
Popularité : 2 réutilisations,  1 suivi\newline
Mots-clé : \emph{allocataires, rsa
}\newline
Permalien : \url{https://data.gouv.fr/dataset/53698ed1a3a729239d20358c}\newline

\par
\noindent
    Nombre d'allocataires et de personnes couvertes par le RSA avec
distinction du RSA socle et du RSA activité


\vspace{0.5cm}
\needspace{12\baselineskip}
\subsection*{Base de données publique des médicaments (base officielle)
}\index{base}\index{donnee}\index{medicament}\index{medicaments}\index{publique}\index{sante}\index{substances}
  \begin{wrapfigure}{r}{2.5cm}
    \centering
    \qrcode[nolink]{https://data.gouv.fr/dataset/53698f50a3a729239d2036f5}
  \end{wrapfigure}

Licence : \textbf{Licence Ouverte
}\newline
Créé le : 2013-12-12\newline
Modifié le : 2016-03-16\newline
Granularité : au pays\newline
Popularité : 12 réutilisations,  19 suivis\newline
Mots-clé : \emph{base, donnee, medicament, medicaments, publique, sante, substances
}\newline
Permalien : \url{https://data.gouv.fr/dataset/53698f50a3a729239d2036f5}\newline

\par
\noindent
    La base de données publique des médicaments permet au grand public et
aux professionnels de santé d'accéder à des données et documents de
référence sur les médicaments commercialisés ou ayant été commercialisés
durant les deux dernières années en France.

Cette base de données administratives et scientifiques sur les
traitements et le bon usage des produits de santé est mise en œuvre par
l'Agence nationale de sécurité du médicament et des produits de santé
(ANSM), en liaison avec la Haute Autorité de santé (HAS) et l'Union
nationale des caisses d'assurance maladie (UNCAM), sous l'égide du
ministère des Affaires sociales et de la santé.


\vspace{0.5cm}
\needspace{12\baselineskip}
\subsection*{Bénéficiaires du minimum vieillesse
}\index{allocation!de!solidarite!aux!per}\index{aspa}\index{minimum!vieillesse}\index{personnes!agees}\index{retraite}
  \begin{wrapfigure}{r}{2.5cm}
    \centering
    \qrcode[nolink]{https://data.gouv.fr/dataset/5369963ca3a729239d20493a}
  \end{wrapfigure}

Licence : \textbf{Licence Ouverte
}\newline
Créé le : 2013-12-16\newline
Modifié le : 2016-03-04\newline
De 1960-01-01 à 2011-12-31\newline
Popularité : 1 réutilisation,  0 suivi\newline
Mots-clé : \emph{allocation-de-solidarite-aux-per, aspa, minimum-vieillesse, personnes-agees, retraite
}\newline
Permalien : \url{https://data.gouv.fr/dataset/5369963ca3a729239d20493a}\newline

\par
\noindent
    Évolution depuis 1960 du nombre de bénéficiaires du minimum vieillesse.
Évolution depuis 1970 du minimum vieillesse, personne seule et couple,
des pensions de retraite du régime général et de l'indice des prix.
Évolution depuis 2001 par régime des effectifs de bénéficiaires de
l'allocation spéciale vieillesse (ASV) et de l'allocation de solidarité
aux personnes âgées (ASPA). Sources: DREES, Fonds de solidarité
vieillesse, CNAV, CNAMTS, Caisse des dépôts et consignations, INSEE.


\vspace{0.5cm}
\needspace{12\baselineskip}
\subsection*{Comptes nationaux de la santé - 2013
}\index{cns}\index{cns!2013}\index{comptes!nationaux!de!la!sante}\index{depenses}\index{depenses!de!sante}\index{sante}
  \begin{wrapfigure}{r}{2.5cm}
    \centering
    \qrcode[nolink]{https://data.gouv.fr/dataset/5446725ec751df07d25d0572}
  \end{wrapfigure}

Licence : \textbf{Licence Ouverte
}\newline
Créé le : 2014-10-21\newline
Modifié le : 2015-11-01\newline
De 2013-01-01 à 2013-01-01\newline
Granularité : au pays\newline
Mise à jour : annuelle\newline
Popularité : 1 réutilisation,  1 suivi\newline
Mots-clé : \emph{cns, cns-2013, comptes-nationaux-de-la-sante, depenses, depenses-de-sante, sante
}\newline
Permalien : \url{https://data.gouv.fr/dataset/5446725ec751df07d25d0572}\newline

\par
\noindent
    \emph{Source : Drees,}

\emph{Auteurs : Catherine ZAIDMAN, Romain ROUSSEL, Marie-Anne LE GARREC,
Marion BOUVET, Julie SOLARD, Myriam MIKOU}

Le montant des dépenses courantes de santé s'élève à 247,7 milliards
d'euros en 2013, soit 11,7\% du produit intérieur brut (PIB) en base
2010. La consommation de soins et de biens médicaux (CSBM), qui en
représente les trois quarts, atteint pour sa part 186,7 milliards
d'euros. Elle s'établit à 8,8\% du PIB en 2013, contre 8,7\% en 2012.


\vspace{0.5cm}
\needspace{12\baselineskip}
\subsection*{Données sur l'indicateur d'accessibilité potentielle localisée (APL)
}\index{accessibilite!potentielle!locali}\index{apl}\index{indicateur}\index{profession!de!la!sante}\index{professionel!de!sante}
  \begin{wrapfigure}{r}{2.5cm}
    \centering
    \qrcode[nolink]{https://data.gouv.fr/dataset/545a3dd8c751df3b749b6045}
  \end{wrapfigure}

Licence : \textbf{Licence Ouverte
}\newline
Créé le : 2014-11-05\newline
Modifié le : 2016-02-29\newline
De 2010-01-01 à 2010-01-01\newline
Granularité : au pays\newline
Mise à jour : ponctuelle\newline
Popularité : 2 réutilisations,  0 suivi\newline
Mots-clé : \emph{accessibilite-potentielle-locali, apl, indicateur, profession-de-la-sante, professionel-de-sante
}\newline
Permalien : \url{https://data.gouv.fr/dataset/545a3dd8c751df3b749b6045}\newline

\par
\noindent
    \emph{Source : DREES}

L'indicateur d'accessibilité potentielle localisée (APL) a été développé
par la DREES et l'IRDES pour mesurer l'adéquation spatiale entre l'offre
et la demande de soins de premier recours à un échelon géographique fin.
Il vise à améliorer les indicateurs usuels d'accessibilité aux soins.


\vspace{0.5cm}
\needspace{12\baselineskip}
\subsection*{FINESS Extraction des autorisations d'activités de soins
}\index{activite!de!soins}\index{finess}\index{hopitaux}\index{sante}
  \begin{wrapfigure}{r}{2.5cm}
    \centering
    \qrcode[nolink]{https://data.gouv.fr/dataset/5369956aa3a729239d2046f7}
  \end{wrapfigure}

Licence : \textbf{Licence Ouverte
}\newline
Créé le : 2013-07-08\newline
Modifié le : 2019-03-07\newline
De 2015-03-18 à 2031-03-07\newline
Granularité : à la commune\newline
Mise à jour : bi-mesnsuelle\newline
Popularité : 1 réutilisation,  5 suivis\newline
Mots-clé : \emph{activite-de-soins, finess, hopitaux, sante
}\newline
Permalien : \url{https://data.gouv.fr/dataset/5369956aa3a729239d2046f7}\newline

\par
\noindent
    Liste des autorisations d'activités de soins


\vspace{0.5cm}
\needspace{12\baselineskip}
\subsection*{FINESS Extraction des entités juridiques
}\index{activites!de!soins}\index{autorisations}\index{etablissement}\index{etablissement!hospitalier}\index{etablissements!sanitaires}\index{finess}
  \begin{wrapfigure}{r}{2.5cm}
    \centering
    \qrcode[nolink]{https://data.gouv.fr/dataset/5369956ba3a729239d2046fc}
  \end{wrapfigure}

Licence : \textbf{Licence Ouverte
}\newline
Créé le : 2013-07-08\newline
Modifié le : 2019-03-07\newline
De 2015-03-18 à 2026-02-27\newline
Granularité : à la commune\newline
Mise à jour : bi-mesnsuelle\newline
Popularité : 4 réutilisations,  17 suivis\newline
Mots-clé : \emph{activites-de-soins, autorisations, etablissement, etablissement-hospitalier, etablissements-sanitaires, finess
}\newline
Permalien : \url{https://data.gouv.fr/dataset/5369956ba3a729239d2046fc}\newline

\par
\noindent
    Liste des entités juridiques concernant le domaine sanitaire et social


\vspace{0.5cm}
\needspace{12\baselineskip}
\subsection*{FINESS Extraction des équipements sociaux et médico-sociaux
}\index{activite!medicale}\index{equipement}\index{sante}\index{sociaux}
  \begin{wrapfigure}{r}{2.5cm}
    \centering
    \qrcode[nolink]{https://data.gouv.fr/dataset/5369956ca3a729239d204701}
  \end{wrapfigure}

Licence : \textbf{Licence Ouverte
}\newline
Créé le : 2013-07-08\newline
Modifié le : 2019-03-07\newline
De 2015-03-18 à 2017-04-30\newline
Granularité : à la commune\newline
Mise à jour : bi-mesnsuelle\newline
Popularité : 1 réutilisation,  5 suivis\newline
Mots-clé : \emph{activite-medicale, equipement, sante, sociaux
}\newline
Permalien : \url{https://data.gouv.fr/dataset/5369956ca3a729239d204701}\newline

\par
\noindent
    Liste des équipements sociaux et médico-sociaux.

Le 30/12/2016 : Modification du format de fichier : Ajout des rubriques
indsupinst, indsupaut, datemajaut et datemajinst pour permettre le
filtre des installations et des autorisations supprimées. Le 10/01/2017
: Suite à l'instruction DGCS/2016/300 relative à la prise en compte dans
la gestion du FINESS des modifications apportées par la loi
n\degree{}2015-1776 du 28 décembre 2015, les capacités sur la catégorie
202 ``résidences autonomie'' s'expriment en nombre de places et plus en
nombre de logements. Le 28/09/2018 : Intitulés de la rubrique sourceinfo
: 9 Indéterminée V Visite de conformité D Document de tarification E
Enquête statistique A Autre I Interrogation de l'ET C Convention R
Arrêté S Inspection L Labellisation


\vspace{0.5cm}
\needspace{12\baselineskip}
\subsection*{FINESS Extraction du Fichier des établissements
}\index{activites!de!soins}\index{autorisations}\index{etablissement}\index{etablissement!hospitalier}\index{etablissements!sanitaires}\index{finess}
  \begin{wrapfigure}{r}{2.5cm}
    \centering
    \qrcode[nolink]{https://data.gouv.fr/dataset/53699569a3a729239d2046eb}
  \end{wrapfigure}

Licence : \textbf{Licence Ouverte
}\newline
Créé le : 2013-07-08\newline
Modifié le : 2019-03-07\newline
De 2015-03-18 à 2018-11-15\newline
Granularité : à la commune\newline
Mise à jour : bi-mesnsuelle\newline
Popularité : 14 réutilisations,  62 suivis\newline
Mots-clé : \emph{activites-de-soins, autorisations, etablissement, etablissement-hospitalier, etablissements-sanitaires, finess
}\newline
Permalien : \url{https://data.gouv.fr/dataset/53699569a3a729239d2046eb}\newline

\par
\noindent
    Liste des établissements du domaine sanitaire et social.

Informations sur la Géo-localisation : Le système d'information source
contenant les coordonnées géographiques permettant de géo-localiser les
établissements répertoriés dans FINESS est le produit BD-ADRESSE en
version 2.1 de l'IGN (Institut Géographique National).

\textbf{ZONE SYSTÈME GEODESIQUE ELLIPSOÏDE ASSOCIEE PROJECTION }
//France métropolitaine : RGF93 IAG GRS 1980 Coniques conformes 9 zones
- Lambert 93 //Guadeloupe, Martinique: WGS84 IAG GRS 1980 UTM Nord
fuseau 20 //Guyane : RGFG95 IAG GRS 1980 UTM Nord fuseau 22 //Réunion :
RGR92 IAG GRS 1980 UTM Sud fuseau 40 //Mayotte : RGM04 IAG GRS 1980 UTM
Sud fuseau 38 FAQ : Pourquoi les colonnes n'apparaissent pas sur le
fichier proposé ? Pour permettre l'utilisation d'automates, la structure
du fichier s'affiche en format XSD. Pour retrouver les colonnes d'un
format type tableur, il faut vous référer à la description du fichier
disponible en format PDF.


\vspace{0.5cm}
\needspace{12\baselineskip}
\subsection*{La démographie des infirmiers à l'horizon 2030 : un exercice de
projections aux niveaux national et régional
}\index{hopital}\index{infirmier}\index{infirmiers}\index{personnel!infirmier}
  \begin{wrapfigure}{r}{2.5cm}
    \centering
    \qrcode[nolink]{https://data.gouv.fr/dataset/5369979ca3a729239d204d35}
  \end{wrapfigure}

Licence : \textbf{Licence Ouverte
}\newline
Créé le : 2013-07-08\newline
Modifié le : 2015-09-11\newline
De 2006-01-01 à 2030-12-31\newline
Granularité : au pays\newline
Mise à jour : ponctuelle\newline
Popularité : 1 réutilisation,  3 suivis\newline
Mots-clé : \emph{hopital, infirmier, infirmiers, personnel-infirmier
}\newline
Permalien : \url{https://data.gouv.fr/dataset/5369979ca3a729239d204d35}\newline

\par
\noindent
    Des scénarios ont permis de tester des hypothèses d'évolution du quota
national et de report des cessations d'activité, afin de mesurer
l'impact de la réforme statutaire des infirmiers salariés de
l'hôpitalpublic ou de la réforme des retraites sur l'évolution des
effectifs d'infirmiers.Différents scénarios ont été simulés : - le
scénario tendanciel s'appuie sur l'hypothèse de comportements constants,
à savoir que les comportements des infirmiers observés dans un
passérécent et les mesures de régulation des pouvoirs publics
resteraient inchangés sur l'ensemble de lapériode de projection ;- les
autres scénarios, appelés variantes, ne diffèrentdu scénario tendanciel
que par une hypothèse -- correspondant à l'évolution d'un comportement
ou àla mise en place d'une mesure - ce qui permet d'en mesurer l'impact
propre sur l'évolution des effectifs.


\vspace{0.5cm}
\needspace{12\baselineskip}
\subsection*{La démographie des médecins (RPPS)
}\index{demographie}\index{demographie!des!medecins}\index{demographie!medicale}\index{medecin}\index{medecin!generaliste}\index{medecin!specialiste}\index{medecins!liberaux}\index{nombre!de!medecin}\index{rpps}
  \begin{wrapfigure}{r}{2.5cm}
    \centering
    \qrcode[nolink]{https://data.gouv.fr/dataset/537893d3a3a7295dd332d9e0}
  \end{wrapfigure}

Licence : \textbf{Licence Ouverte
}\newline
Créé le : 2014-05-12\newline
Modifié le : 2016-03-14\newline
De 2013-01-01 à 2014-12-31\newline
Granularité : au département\newline
Mise à jour : ponctuelle\newline
Popularité : 10 réutilisations,  5 suivis\newline
Mots-clé : \emph{demographie, demographie-des-medecins, demographie-medicale, medecin, medecin-generaliste, medecin-specialiste, medecins-liberaux, nombre-de-medecin, rpps
}\newline
Permalien : \url{https://data.gouv.fr/dataset/537893d3a3a7295dd332d9e0}\newline

\par
\noindent
    \emph{Source : DREES}

Au 1er janvier 2014, la base statistique du RPPS recense 219 834
médecins en activité, dont 214 594 en France métropolitaine. Par rapport
au 1er janvier 2013, le nombre de médecins a augmenté de 0,7 \% (218 296
médecins au 1er janvier 2013 en France et 213 227 en France
métropolitaine). On enregistre sur la même période une légère hausse de
la densité de l'ensemble des médecins (+0,2 \%).


\vspace{0.5cm}
\needspace{12\baselineskip}
\subsection*{L'allocation personnalisée d'autonomie (APA) Résultats nationaux
trimestriels
}
  \begin{wrapfigure}{r}{2.5cm}
    \centering
    \qrcode[nolink]{https://data.gouv.fr/dataset/536997a8a3a729239d204d55}
  \end{wrapfigure}

Licence : \textbf{Licence Ouverte
}\newline
Créé le : 2013-07-08\newline
Modifié le : 2016-01-19\newline
De 2001-12-31 à 2011-03-31\newline
Popularité : 1 réutilisation,  0 suivi\newline
Mots-clé : \emph{aucun
}\newline
Permalien : \url{https://data.gouv.fr/dataset/536997a8a3a729239d204d55}\newline

\par
\noindent
    Données agrégées transmises par les conseils généraux. Une enquête
trimestrielle donne une estimation des résultats nationaux quatre fois
par an ; une enquête annuelle, permet de disposer de données définitives
sur tous les départements au 31 décembre de chaque année.


\vspace{0.5cm}
\needspace{12\baselineskip}
\subsection*{La Statistique annuelle des établissements (SAE)
}\index{sae}\index{statistique!annuelle}\index{statistique!annuelle!des!etablis}
  \begin{wrapfigure}{r}{2.5cm}
    \centering
    \qrcode[nolink]{https://data.gouv.fr/dataset/54730e00c751df4f2ec2acbe}
  \end{wrapfigure}

Licence : \textbf{Licence Ouverte
}\newline
Créé le : 2014-11-24\newline
Modifié le : 2016-03-16\newline
De 2000-01-01 à 2013-01-01\newline
Granularité : au pays\newline
Mise à jour : annuelle\newline
Popularité : 2 réutilisations,  0 suivi\newline
Mots-clé : \emph{sae, statistique-annuelle, statistique-annuelle-des-etablis
}\newline
Permalien : \url{https://data.gouv.fr/dataset/54730e00c751df4f2ec2acbe}\newline

\par
\noindent
    \emph{Source : DREES}

La Statistique annuelle des établissements de santé (SAE) est l'une des
principales sources de référence du Ministère sur les établissements de
santé, complémentaire du PMSI, puisqu'elle renseigne sur les capacités,
les équipements et les personnels.


\vspace{0.5cm}
\needspace{12\baselineskip}
\subsection*{Le baromètre d'opinion de la Drees
}\index{assurance!maladie}\index{barometre}\index{cohesion!sociale}\index{dependance!des!personnes!agees}\index{drees}\index{enquete}\index{famille}\index{handicap}\index{inegalites}\index{montant!minimum!pour!vivre}\index{opinion}\index{pauvrete}\index{perceptions}\index{politique}\index{protection!sociale}\index{quotas}\index{retraite}\index{sante}\index{statistique}\index{stereotypes!femme!homme}\index{subjectif}
  \begin{wrapfigure}{r}{2.5cm}
    \centering
    \qrcode[nolink]{https://data.gouv.fr/dataset/55716693c751df441fe5726a}
  \end{wrapfigure}

Licence : \textbf{Licence Ouverte
}\newline
Créé le : 2015-06-05\newline
Modifié le : 2016-02-20\newline
De 2000-01-01 à 2015-06-05\newline
Granularité : à la région\newline
Mise à jour : annuelle\newline
Popularité : 3 réutilisations,  0 suivi\newline
Mots-clé : \emph{assurance-maladie, barometre, cohesion-sociale, dependance-des-personnes-agees, drees, enquete, famille, handicap, inegalites, montant-minimum-pour-vivre, opinion, pauvrete, perceptions, politique, protection-sociale, quotas, retraite, sante, statistique, stereotypes-femme-homme, subjectif
}\newline
Permalien : \url{https://data.gouv.fr/dataset/55716693c751df441fe5726a}\newline

\par
\noindent
    \textbf{Un suivi sur longue période de la perception de la protection
sociale}

Le Baromètre d'opinion de la DREES est une enquête de suivi de l'opinion
en France métropolitaine sur la santé, la protection sociale (assurance
maladie, retraite, famille, handicap-dépendance, pauvreté-exclusion),
les inégalités et la cohésion sociale. Commandée par la DREES tous les
ans depuis 2000 (sauf en 2003), elle est réalisée par l'institut BVA
depuis 2004, après l'avoir été par l'IFOP de 2000 à 2002.

\textbf{Une enquête par quotas sur 3000 enquêtés}

L'enquête est effectuée en face à face en octobre-novembre auprès d'un
échantillon d'environ 3 000 personnes représentatives de la population
habitant en France métropolitaine et âgées de plus de 18 ans.
L'échantillon est construit selon la méthode des quotas (par sexe, âge,
profession de la personne de référence, après stratification par région
et catégorie d'agglomération). En 2014, le questionnaire a été refondu,
avec une réduction de l'échantillon de 4 000 à 3 000 personnes, et
l'intégration de questions portant sur la cohésion sociale avec la
collaboration de la DGCS. Depuis 2013, le Baromètre comprend également
un module méthodologique permettant de tester l'effet de la formulation
des questions sur les réponses.

\textbf{Des comparaisons des opinions dans la durée et entre catégories,
à mettre en perspective selon le contexte et la formulation des
questions}

Les réponses à une enquête d'opinion sont particulièrement sensibles à
la formulation des questions et à leur place dans le questionnaire. Ces
enquêtes permettent néanmoins des comparaisons entre catégories (selon
le revenu, l'âge, etc.) et dans la durée. Elles peuvent notamment capter
l'évolution des réponses, au fil des années, lorsque la formulation des
questions et l'organisation du questionnaire restent les mêmes. De
telles variations donnent une information sur la manière dont les
opinions évoluent dans le temps, selon la conjoncture, au fil des
actions politiques mises en œuvre et du débat médiatique. Toutefois, les
plus petites variations (de l'ordre d'un ou deux points de pourcentage)
ne reflètent, elles, que des imperfections de mesure.


\vspace{0.5cm}
\needspace{12\baselineskip}
\subsection*{Les indicateurs relatifs aux infections nosocomiales dans les
établissements de santé - 2011
}
  \begin{wrapfigure}{r}{2.5cm}
    \centering
    \qrcode[nolink]{https://data.gouv.fr/dataset/5369984aa3a729239d204efe}
  \end{wrapfigure}

Licence : \textbf{Licence Ouverte
}\newline
Créé le : 2013-10-13\newline
Modifié le : 2016-03-15\newline
De 2011-01-01 à 2011-12-31\newline
Granularité : au pays\newline
Mise à jour : annuelle\newline
Popularité : 1 réutilisation,  2 suivis\newline
Mots-clé : \emph{aucun
}\newline
Permalien : \url{https://data.gouv.fr/dataset/5369984aa3a729239d204efe}\newline

\par
\noindent
    Tous les établissements de santé sont concernés par la lutte contre les
infections nosocomiales. Les sept indicateurs qui constituent le tableau
de bord des infections nosocomiales en 2011 sont les suivants :
1\degree{} Un indicateur général sur la lutte contre les infections
nosocomiales : l'indicateur composite des activités de lutte contre les
infections nosocomiales (ICALIN.2). L'ICALIN.2 objective l'organisation
de la lutte contre les infections nosocomiales dans l'établissement, les
moyens mobilisés et les actions mises en œuvre ; 2\degree{} Un
indicateur spécifique sur l'hygiène des mains : l'indicateur de
consommation de solutions hydro-alcooliques pour l'hygiène des mains :
ICSHA.2. L'indicateur ICSHA.2 est un marqueur indirect de la mise en
œuvre effective de l'hygiène des mains, une mesure-clé de prévention de
nombreuses infections nosocomiales ; 3\degree{} Un indicateur spécifique
sur le risque infectieux opératoire : l'indicateur composite de lutte
contre les infections du site opératoire (ICA-LISO). L'indicateur
ICA-LISO rend visible l'engagement de l'établissement dans une démarche
d'évaluation et d'amélioration des pratiques et de maîtrise du risque
infectieux en chirurgie. Trois indicateurs complémentaires pour mieux
lutter contre les bactéries multi-résistantes : 4\degree{} L'indice
composite de bon usage des antibiotiques (ICATB). L'ICATB reflète le
niveau d'engagement de l'établissement de santé, dans une stratégie
d'optimisation de l'efficacité des traitements antibiotiques. 5\degree{}
L'indicateur composite de maîtrise de la diffusion des bactéries
multi-résistantes (ICA-BMR). L'ICA-BMR rend visible le niveau
d'engagement de l'établissement de santé, dans une démarche visant à
maitriser la diffusion des bactéries multi-résistantes dans leur
ensemble. 6\degree{} L'indice triennal du SARM permet de refléter
l'écologie microbienne du Staphylococcus aureus (staphylocoques dorés)
résistant à la méticilline (SARM) de l'établissement et sa capacité à la
maîtriser par des mesures de prévention de la transmission de patient à
patient et par une politique de maîtrise des prescriptions
d'antibiotiques. Cette bactérie multi-résistante aux antibiotiques est
fréquemment en cause dans les infections nococomiales. 7\degree{} Une
agrégation des indicateurs pour une vision globale : le score agrégé. La
prévention des infections nosocomiales est un sujet complexe qui
nécessite la mise en œuvre de nombreuses stratégies complémentaires.
Seule l'interprétation de plusieurs indicateurs permet d'avoir une
vision plus complète du niveau d'engagement des établissements de santé
dans ce domaine. C'est pourquoi le score agrégé activités 2011 a été
construit à partir des 5 indicateurs composites (ICALIN.2, ICSHA.2,
ICA-LISO, ICATB et ICA-BMR) différents selon le type et l'activité des
établissements. Par exemple, ICA-LISO ne concerne que les établissements
de santé ayant une activité de chirurgie, ou d'obstétrique. Pour chaque
catégorie, les établissements concernés ont été répartis en 5 classes de
performance de A à E : - la classe A est composée des établissements
ayant les scores les plus élevés. Ce sont les structures les plus en
avance selon l'indicateur ; - la classe E réunit les établissements
ayant les scores les moins élevés. Ce sont les structures les plus en
retard selon l'indicateur ; - les classes B, C et D correspondent à des
établissements en situation intermédiaire.


\vspace{0.5cm}
\needspace{12\baselineskip}
\subsection*{Les interruptions volontaires de grossesse en 2008 et 2009
}
  \begin{wrapfigure}{r}{2.5cm}
    \centering
    \qrcode[nolink]{https://data.gouv.fr/dataset/5369984ca3a729239d204f07}
  \end{wrapfigure}

Licence : \textbf{Licence Ouverte
}\newline
Créé le : 2013-07-08\newline
Modifié le : 2016-03-15\newline
De 2008-01-01 à 2009-12-31\newline
Popularité : 3 réutilisations,  1 suivi\newline
Mots-clé : \emph{aucun
}\newline
Permalien : \url{https://data.gouv.fr/dataset/5369984ca3a729239d204f07}\newline

\par
\noindent
    Un peu plus de 222 000 interruptions volontaires de grossesse (IVG) ont
été réalisées en France en 2008 ainsi qu'en 2009. Avec 15 IVG pour 1 000
femmes, la France se situe dans la moyenne européenne.Plusieurs sources
peuvent actuellement être utilisées pour le suivi annuel des IVG : les
bulletins d'interruption de grossesse (BIG), dont le remplissage est
prévu par la loi, la Statistique annuelle des établissements de santé
(SAE) et les données issues du Programme médicalisédes systèmes
d'informations (PMSI).L'enquête de 2007 de la DREES auprès de 11 500
femmes ayant eu recours à une IVG permet de recueillir des informations
sur les établissements et les professionnels pratiquant des IVG et
d'interroger les femmes sur leur prise en charge, leur trajectoire et
leur contraception afin de mieux comprendre le recours à l'IVG.Le
baromètre santé de l'Institut national de la prévention et d'éducation
pour la santé (INPES), réalisé tous les cinq ans, permet d'obtenir des
données sur les IVG et la contraception.


\vspace{0.5cm}
\needspace{12\baselineskip}
\subsection*{L'état de santé de la population en France - Édition 2015
}\index{etat!de!sante}\index{inegalite!de!sante}\index{inegalite!sociale}\index{population}\index{sante}
  \begin{wrapfigure}{r}{2.5cm}
    \centering
    \qrcode[nolink]{https://data.gouv.fr/dataset/54e76a87c751df6220467389}
  \end{wrapfigure}

Licence : \textbf{Licence Ouverte
}\newline
Créé le : 2015-02-20\newline
Modifié le : 2016-03-16\newline
De 2014-01-01 à 2014-01-01\newline
Granularité : au pays\newline
Mise à jour : annuelle\newline
Popularité : 1 réutilisation,  0 suivi\newline
Mots-clé : \emph{etat-de-sante, inegalite-de-sante, inegalite-sociale, population, sante
}\newline
Permalien : \url{https://data.gouv.fr/dataset/54e76a87c751df6220467389}\newline

\par
\noindent
    \emph{Source : DREES}

À travers plus de 200 indicateurs, ce panorama détaillé de la santé
conjugue approches par population, par déterminants et par pathologies,
illustrant l'état de santé globalement favorable des Français. Il met
également en lumière les principaux problèmes de santé auxquels sont
confrontées les politiques publiques, dont les inégalités sociales de
santé.


\vspace{0.5cm}
\needspace{12\baselineskip}
\subsection*{Profils et trajectoires des personnes ayant des idées suicidaires
}\index{idees!suicidaires}\index{maladie!chronique}\index{pathologies}\index{personnes!agees}\index{sante}\index{suicide}
  \begin{wrapfigure}{r}{2.5cm}
    \centering
    \qrcode[nolink]{https://data.gouv.fr/dataset/53d056bca3a72970fff91f1e}
  \end{wrapfigure}

Licence : \textbf{Licence Ouverte
}\newline
Créé le : 2014-07-23\newline
Modifié le : 2016-03-02\newline
De 2010-01-01 à 2014-12-31\newline
Granularité : au pays\newline
Mise à jour : ponctuelle\newline
Popularité : 4 réutilisations,  1 suivi\newline
Mots-clé : \emph{idees-suicidaires, maladie-chronique, pathologies, personnes-agees, sante, suicide
}\newline
Permalien : \url{https://data.gouv.fr/dataset/53d056bca3a72970fff91f1e}\newline

\par
\noindent
    \emph{Source : DREES, Études et résultats N\degree{}886}

\emph{Auteurs : Nicolas de Riccardis, avec les conseils de Muriel Moisy
et Marie-Claude Mouquet}

En 2010, 5 \% des personnes âgées de 40 à 59 ans déclarent avoir eu des
idées suicidaires dans les deux semaines précédant l'étude. Leurs
indicateurs de santé sont plus dégradés que ceux des autres personnes.
Elles souffrent plus souvent d'une maladie chronique et déclarent plus
de pathologies.


\vspace{0.5cm}
\needspace{12\baselineskip}
\subsection*{Quel risque de décès un an après une fracture du col du fémur ?
}\index{deces}\index{fracture}
  \begin{wrapfigure}{r}{2.5cm}
    \centering
    \qrcode[nolink]{https://data.gouv.fr/dataset/56d424e688ee382a8ffa79cf}
  \end{wrapfigure}

Licence : \textbf{Licence Ouverte
}\newline
Créé le : 2016-02-29\newline
Modifié le : 2016-03-01\newline
Popularité : 1 réutilisation,  0 suivi\newline
Mots-clé : \emph{deces, fracture
}\newline
Permalien : \url{https://data.gouv.fr/dataset/56d424e688ee382a8ffa79cf}\newline

\par
\noindent
    Source : Études et Résultats n\degree{} 948, janvier 2016

Auteur : Philippe Oberlin, Marie-Claude Mouquet

En 2008-2009, près de 95 000 patients de plus de 54 ans, assurés au
régime général de l'Assurance maladie, dont trois quarts de femmes, ont
été hospitalisés pour une fracture du col du fémur.

Une femme sur cinq et un homme sur trois sont morts dans l'année qui a
suivi. Le décès est corrélé avec l'âge pour les deux sexes, mais la
surmortalité par rapport à la population du même âge est plus élevée
chez les hommes que chez les femmes.

Les types de fractures, les catégories d'établissements et les types de
traitements influent peu sur la mortalité, sauf lorsque les patients ne
peuvent pas être opérés. En revanche, l'état de santé du patient au
moment de la fracture est déterminant.

En analyse multivariée, le risque de décès à un an augmente dès qu'il
existe une pathologie chronique significative et croît jusqu'à 4,6 fois
chez les patients les plus graves.


\vspace{0.5cm}
\needspace{12\baselineskip}
\subsection*{Statistique annuelle des établissements de santé (SAE)
}\index{activite}\index{equipement}\index{etablissement!de!sante}\index{hopital}\index{indicateur}\index{personnel}\index{ressource}\index{sante}\index{soins}
  \begin{wrapfigure}{r}{2.5cm}
    \centering
    \qrcode[nolink]{https://data.gouv.fr/dataset/5369a047a3a729239d206320}
  \end{wrapfigure}

Licence : \textbf{Licence Ouverte
}\newline
Créé le : 2013-07-08\newline
Modifié le : 2017-11-20\newline
De 2007-01-01 à 2011-12-31\newline
Granularité : au point d'intérêt\newline
Mise à jour : annuelle\newline
Popularité : 2 réutilisations,  3 suivis\newline
Mots-clé : \emph{activite, equipement, etablissement-de-sante, hopital, indicateur, personnel, ressource, sante, soins
}\newline
Permalien : \url{https://data.gouv.fr/dataset/5369a047a3a729239d206320}\newline

\par
\noindent
    Indicateurs sur l'activité et les ressources en équipements et en
personnels des établissements de santé entrant dans le champs de
l'enquête annuelle des établissements de santé (SAE)


\vspace{0.5cm}
\needspace{12\baselineskip}
\subsection*{Urgences : la moitié des patients restent moins de deux heures, hormis
ceux maintenus en observation
}\index{nourrissons}\index{patients}\index{personnes!agees}\index{taux!de!recours}\index{urgences}
  \begin{wrapfigure}{r}{2.5cm}
    \centering
    \qrcode[nolink]{https://data.gouv.fr/dataset/53dfdf9aa3a729110ca8d375}
  \end{wrapfigure}

Licence : \textbf{Licence Ouverte
}\newline
Créé le : 2014-07-30\newline
Modifié le : 2016-02-15\newline
De 2013-01-01 à 2013-12-31\newline
Granularité : au pays\newline
Mise à jour : ponctuelle\newline
Popularité : 7 réutilisations,  2 suivis\newline
Mots-clé : \emph{nourrissons, patients, personnes-agees, taux-de-recours, urgences
}\newline
Permalien : \url{https://data.gouv.fr/dataset/53dfdf9aa3a729110ca8d375}\newline

\par
\noindent
    \emph{Source : DREES, Études et résultats n\degree{} 889}

\emph{Auteurs : Bénédicte Boisguérin et Hélène Valdelièvre}

Huit patients sur dix restent moins de quatre heures aux urgences,
d'après l'enquête réalisée dans les services d'urgences le 11 juin 2013.
Les nourrissons et les personnes âgées enregistrent les taux de recours
les plus élevés.


\vspace{0.5cm}
\needspace{3\baselineskip} \rule{4cm}{0.25pt}\newline\textbf{Aussi disponible du même producteur :}\begin{itemize}
\item \href{https://data.gouv.fr/dataset/544514f7c751df59bfef488d}{1,2 million de travailleurs sociaux en 2011}
\item \href{https://data.gouv.fr/dataset/555f140dc751df7d44c98e0f}{24 heures chrono dans la vie d’un jeune : les modes de vie des 15-24 ans}
\item \href{https://data.gouv.fr/dataset/548aebffc751df57134120e7}{3,8 millions de prestations d’aide sociale attribuées par les départements en 2013}
\item \href{https://data.gouv.fr/dataset/5649ff58c751df25f3aad370}{4,2 millions de prestations d'aide sociale atttribuées par les départements en 2014}
\item \href{https://data.gouv.fr/dataset/58d0fd49c751df08d83b0dc4}{55 ans de diversification des financements de la protection sociale}
\item \href{https://data.gouv.fr/dataset/548ae9e1c751df53904120eb}{693 000 résidents en établissements d’hébergement pour personnes âgées en 2011}
\item \href{https://data.gouv.fr/dataset/5630ec4488ee385064531576}{7 860 étudiants en médecine affectés à l'issue des épreuves classantes nationales en 2014}
\item \href{https://data.gouv.fr/dataset/555eedccc751df3757c98e10}{97 000 jeunes en grande précarité bénéficient du fonds d’aide aux jeunes en 2013}
\item \href{https://data.gouv.fr/dataset/54451092c751df3eaeef488f}{Accessibilité et accès aux établissements d’hébergement pour personnes âgées dépendantes en 2011}
\item \href{https://data.gouv.fr/dataset/555f0818c751df57adc98e10}{Aide sociale à l’hébergement et allocation personnalisée d’autonomie en 2011 : profil des bénéficiaires en établissement}
\item \href{https://data.gouv.fr/dataset/53698ed0a3a729239d20358a}{Allocataires des minima sociaux}
\item \href{https://data.gouv.fr/dataset/53698ed0a3a729239d203589}{Allocataires des minima sociaux}
\item \href{https://data.gouv.fr/dataset/53698ecfa3a729239d203588}{Allocataires des minima sociaux}
\item \href{https://data.gouv.fr/dataset/53698ed1a3a729239d20358b}{Allocataires du RSA}
\item \href{https://data.gouv.fr/dataset/56a633fb88ee3808ec822418}{Amendement Creton : 6 000 jeunes adultes dans des établissements pour enfants handicapés}
\item \href{https://data.gouv.fr/dataset/55e06e27c751df0fe31f92b1}{Combien dépensent les familles pour la garde de leurs enfants de moins de 3 ans ?}
\item \href{https://data.gouv.fr/dataset/53d37e62a3a7290427013691}{Comment les organismes complémentaires fixent leurs tarifs}
\item \href{https://data.gouv.fr/dataset/58d0fec5c751df0b34f822e7}{Complémentaire santé : un organisme d’assurances sur quatre gère exclusivement des contrats individuels en 2015}
\item \href{https://data.gouv.fr/dataset/5649df47c751df5e92aad370}{Compte provisoire des prestations de protection sociale en 2014 : première estimation}
\item \href{https://data.gouv.fr/dataset/53699166a3a729239d203c55}{Comptes nationaux de la santé 2010 : consommation de soins et de biens médicaux 2005-2010}
\item \href{https://data.gouv.fr/dataset/53699167a3a729239d203c58}{Comptes nationaux de la santé 2010 : consommation de soins et de biens médicaux en indices de prix 2005-2010}
\item \href{https://data.gouv.fr/dataset/53699167a3a729239d203c56}{Comptes nationaux de la santé 2010 : consommation de soins et de biens médicaux en indices de valeur 2005-2010}
\item \href{https://data.gouv.fr/dataset/53699167a3a729239d203c57}{Comptes nationaux de la santé 2010 : consommation de soins et de biens médicaux en indices de volume 2005-2010}
\item \href{https://data.gouv.fr/dataset/53699168a3a729239d203c59}{Comptes nationaux de la santé 2010 : consommation de soins et de biens médicaux en valeur 2005-2010}
\item \href{https://data.gouv.fr/dataset/53699168a3a729239d203c5a}{Comptes nationaux de la santé 2010 : consommation de soins et de biens médicaux en volume (base 100 année 2005) 2005-2010}
\item \href{https://data.gouv.fr/dataset/53699169a3a729239d203c5b}{Comptes nationaux de la santé 2010 : dépense courante santé 2005-2010}
\item \href{https://data.gouv.fr/dataset/53699169a3a729239d203c5c}{Comptes nationaux de la santé 2010 : dépense courante santé en valeur 2005-2010}
\item \href{https://data.gouv.fr/dataset/5369916aa3a729239d203c5d}{Comptes nationaux de la santé 2010 : dépenses de santé par type de financeur 2005}
\item \href{https://data.gouv.fr/dataset/5369916aa3a729239d203c5e}{Comptes nationaux de la santé 2010 : dépenses de santé par type de financeur 2006}
\item \href{https://data.gouv.fr/dataset/5369916aa3a729239d203c5f}{Comptes nationaux de la santé 2010 : dépenses de santé par type de financeur 2007}
\item \href{https://data.gouv.fr/dataset/5369916ba3a729239d203c60}{Comptes nationaux de la santé 2010 : dépenses de santé par type de financeur 2008}
\item \href{https://data.gouv.fr/dataset/5369916ba3a729239d203c61}{Comptes nationaux de la santé 2010 : dépenses de santé par type de financeur 2009}
\item \href{https://data.gouv.fr/dataset/5369916ba3a729239d203c62}{Comptes nationaux de la santé 2010 : dépenses de santé par type de financeur 2010}
\item \href{https://data.gouv.fr/dataset/5369917aa3a729239d203c87}{Comptes simplifiés des régimes de base de sécurité sociale}
\item \href{https://data.gouv.fr/dataset/56e7d67fc751df35c69abcab}{Dépenses d’aide sociale départementale : une hausse de 9 \% depuis 2010}
\item \href{https://data.gouv.fr/dataset/56122b7688ee387d41a80558}{Diplômés de formations sociales en 2010 :  une insertion professionnelle qui résiste à la crise }
\item \href{https://data.gouv.fr/dataset/53788f04a3a7295dd332d9d6}{Données concernant l’allocation personnalisée d’autonomie (APA)}
\item \href{https://data.gouv.fr/dataset/53788f05a3a7295dd332d9d7}{Données concernant la prestation de compensation du handicap (PCH) et de l’allocation compensatrice tierce personne (ACTP)}
\item \href{https://data.gouv.fr/dataset/53788f05a3a7295dd332d9d8}{Données concernant la protection maternelle et infantile (PMI)}
\item \href{https://data.gouv.fr/dataset/57fce28ac751df0e4079df72}{Données de rapportage de la saison balnéaire}
\item \href{https://data.gouv.fr/dataset/53788f0da3a7295dd332d9d9}{Données disponibles sur la morbidité hospitalière}
\item \href{https://data.gouv.fr/dataset/53d38145a3a7290427013694}{Échec et retard scolaire des enfants hébergés par l’aide sociale à l’enfance}
\item \href{https://data.gouv.fr/dataset/55f285bec751df532c1f92c9}{Emplois et salaires dans le secteur hospitalier en 2012}
\item \href{https://data.gouv.fr/dataset/539a62fca3a7293bc272837c}{Enquête auprès des établissements d’hébergement pour personnes âgées (EHPA)}
\item \href{https://data.gouv.fr/dataset/53d3830ea3a7290427013697}{Établissements de santé : le personnel soignant de plus en plus âgé}
\item \href{https://data.gouv.fr/dataset/53d3831da3a7290427013698}{État de santé et renoncement aux soins des bénéficiaires du RSA}
\item \href{https://data.gouv.fr/dataset/5369958ba3a729239d20475d}{Financement et dépenses de la sécurité sociale}
\item \href{https://data.gouv.fr/dataset/5369956aa3a729239d2046f9}{FINESS Extraction des disciplines d'enseignement}
\item \href{https://data.gouv.fr/dataset/5369956ba3a729239d2046ff}{FINESS Extraction des équipements matériels lourds}
\item \href{https://data.gouv.fr/dataset/574bfb82c751df12078cc4b3}{FINESS Extraction des principales nomenclatures}
\item et 117 autres jeux de données\end{itemize}

\clearpage
\section{Ministère des sports}


\begin{center}
  \includegraphics[width=3cm]{images/orga/31_e45bdf8598454a96745a642f209bfe-100.png}
\end{center}


Le ministère de la ville, de la jeunesse et des sports prépare et met en
œuvre la politique du Gouvernement relative à la politique de la ville,
la politique en faveur de la jeunesse, la politique en matière
d'activités physiques et sportives et la politique en matière de vie
associative et d'éducation populaire.


\vspace{0.5cm}

\needspace{12\baselineskip}
\subsection*{Liste des médailles obtenues par des sportifs français aux Jeux
Olympiques et paralympiques de 2012 et de 2014
}\index{jeux!olympiques}\index{medailles}\index{sport}
  \begin{wrapfigure}{r}{2.5cm}
    \centering
    \qrcode[nolink]{https://data.gouv.fr/dataset/57d587c5c751df66e397bae5}
  \end{wrapfigure}

Licence : \textbf{Licence Ouverte
}\newline
Créé le : 2016-09-11\newline
Modifié le : 2016-09-15\newline
Granularité : au pays\newline
Mise à jour : ponctuelle\newline
Popularité : 5 réutilisations,  0 suivi\newline
Mots-clé : \emph{jeux-olympiques, medailles, sport
}\newline
Permalien : \url{https://data.gouv.fr/dataset/57d587c5c751df66e397bae5}\newline

\par
\noindent
    \textbf{Généralités }Maintenir le rang de la France parmi les grandes
nations sportives est un axe important de la politique sportive de
l'Etat et qui passe notamment par l'obtention de bons résultats lors des
rencontres sportives internationales, et notamment par l'obtention de
médailles lors des Jeux Olympiques (JO) et des Jeux Paralympiques (JP)
d'été et d'hiver.

Dans cette optique, la direction des sports dispose des données
relatives aux médailles obtenues lors des JO par des athlètes français.
Ce sont ces données qui sont restituées dans le jeu de données «~liste
des médailles obtenus par des sportifs français aux jeux olympiques et
paralympiques de 2012 et de 2014~», pour les JO et JP d'été 2012 et pour
les JO et JP d'hiver 2014.

\textbf{La présentation par médaille }L'objet de ce jeu de données étant
d'identifier toutes les médailles obtenues par les sportifs français
lors des JO, il est présenté par médaille obtenue. Ainsi, un sportif
ayant obtenu plusieurs médailles dans différentes épreuves sera répété
autant de fois qu'il aura obtenue de médaille. De même lorsqu'une équipe
a obtenu un titre, tous les membres de cet équipe sont listés dans le
tableau puisqu'ils ont tous obtenus une médaille d'or, quand bien même
un seul titre aura été décerné).


\vspace{0.5cm}
\needspace{12\baselineskip}
\subsection*{Recensement des équipements sportifs, espaces et sites de pratiques
}\index{activite!sportive}\index{equipement!sportif}\index{recensement}\index{sport}
  \begin{wrapfigure}{r}{2.5cm}
    \centering
    \qrcode[nolink]{https://data.gouv.fr/dataset/53699ebba3a729239d205f4f}
  \end{wrapfigure}

Licence : \textbf{Licence Ouverte
}\newline
Créé le : 2013-07-08\newline
Modifié le : 2018-01-12\newline
Granularité : au point d'intérêt\newline
Mise à jour : trimestrielle\newline
Popularité : 41 réutilisations,  37 suivis\newline
Mots-clé : \emph{activite-sportive, equipement-sportif, recensement, sport
}\newline
Permalien : \url{https://data.gouv.fr/dataset/53699ebba3a729239d205f4f}\newline

\par
\noindent
    Le recen­se­ment natio­nal de l'inté­gra­lité des équipements spor­tifs,
espa­ces et sites de pra­ti­ques cons­ti­tue l'une des actions
prio­ri­tai­res conduite par le ministère chargé des sports. La
démar­che enga­gée a pour objec­tif de per­met­tre une bonne
connais­sance partagée des équipements et sites existants et d'aider à
une meilleure perception des inégali­tés ter­ri­to­ria­les dans leur
répar­ti­tion. C'est un élément préalable à toute démarche prospective
d'aménagement du territoire.


\vspace{0.5cm}
\needspace{3\baselineskip} \rule{4cm}{0.25pt}\newline\textbf{Aussi disponible du même producteur :}\begin{itemize}
\item \href{https://data.gouv.fr/dataset/57d588ddc751df68ed97bae5}{Athlètes inscrits sur la liste des sportifs de haut niveau en 2015}
\item \href{https://data.gouv.fr/dataset/53698f4da3a729239d2036eb}{Base de données "Corps \&{} diplômes" du ministère des sports, de la jeunesse, de l’éducation populaire et de la vie associative}
\item \href{https://data.gouv.fr/dataset/57d5947fc751df7e0c97bae5}{Base des Conseillers Techniques Sportifs }
\item \href{https://data.gouv.fr/dataset/57d59600c751df032a97bae5}{Base des diplômés des métiers du sport 2015}
\item \href{https://data.gouv.fr/dataset/57d4239bc751df60a697bae5}{Emploi d’avenir et CUI - CAE  dans les domaines du sport et de l’animation}
\item \href{https://data.gouv.fr/dataset/57d66dc788ee383e857fc1d6}{Facteurs d'émissions d'équipements de sport}
\item \href{https://data.gouv.fr/dataset/57f669ffc751df746179df72}{Opération balle jaune de la fédération française de tennis - Liste des structures bénéficiaires au 1er septembre 2016}
\item \href{https://data.gouv.fr/dataset/5c7e67138b4c4151a709037d}{Référentiel d’activités physiques et sportives}
\item \href{https://data.gouv.fr/dataset/57d6708388ee383e857fc1d7}{Référentiel d'indicateurs pour auto-évaluer  l'éco-responsabilité d'événements sportifs}
\item \href{https://data.gouv.fr/dataset/57d574d8c751df448697bae5}{Relations géographiques distantes des licences 2013 des fédérations sportives à leur club}
\item \href{https://data.gouv.fr/dataset/57d579a4c751df4d2a97bae5}{Subventions allouées en 2015 par le Centre National pour le Développement du Sport (CNDS)}
\item \href{https://data.gouv.fr/dataset/57d580c4c751df5b9397bae5}{Subventions « Equipements sportifs » allouées en 2015 dans le cadre de la Dotation d’Equipement des Territoires Ruraux  - DETR}
\item \href{https://data.gouv.fr/dataset/57d59543c751df013097bae5}{« Optimouv » : logiciel d'optimisation des déplacements dans l’organisation des rencontres sportives}
\end{itemize}

\clearpage
\section{Ministère du travail}


\begin{center}
  \includegraphics[width=3cm]{images/orga/dd_2812860f2947dca173902e1f877d4f-100.png}
\end{center}


La ministre du Travail prépare et met en œuvre la politique du
gouvernement dans les domaines du travail, de l'emploi, de la formation
professionnelle, du dialogue social et de la prévention des accidents du
travail et des maladies professionnelles. Les missions de la ministre

\begin{itemize}

\item
  Elle prépare et met en œuvre les règles relatives aux conditions de
  travail, à la négociation collective et aux droits des salariés.
\item
  Conjointement avec la ministre des Affaires sociales, de la Santé et
  des Droits des femmes, elle prépare et met en œuvre les règles
  relatives aux régimes et à la gestion des organismes de sécurité
  sociale en matière d'accidents du travail et de maladies
  professionnelles.
\item
  Elle est compétente pour la défense et la promotion de l'emploi, y
  compris la politique de retour à l'emploi, ainsi que pour la formation
  professionnelle des jeunes et des adultes.
\item
  Elle participe à l'action du gouvernement en matière de lutte contre
  la fraude.
\end{itemize}

\textbf{L'organisation du ministère}

\emph{Les structures sur lesquelles la ministre a autorité}

\begin{itemize}

\item
  La direction générale du travail (DGT)
\item
  La direction de l'animation, de la recherche, des études et des
  statistiques (DARES)
\item
  La délégation générale à l'emploi et à la formation professionnelle
  (DGEFP) conjointement avec le ministre de la Ville, de la Jeunesse et
  des Sports
\item
  La direction de la recherche, des études, de l'évaluation et des
  statistiques (DREES) conjointement avec la ministre des Affaires
  sociales, de la Santé et des Droits des femmes
\item
  L'inspection générale des affaires sociales (IGAS) conjointement avec
  la ministre des Affaires sociales, de la Santé et des Droits des
  femmes
\item
  Le secrétariat général des ministères chargés des affaires sociales
  (SGMAS) conjointement avec la ministre des Affaires sociales, de la
  Santé et des Droits des femmes et le ministre de la Ville, de la
  Jeunesse et des Sports
\end{itemize}

\emph{Les structures dont la ministre peut disposer}

\begin{itemize}

\item
  La direction de la sécurité sociale (DSS) pour ses attributions en
  matière d'accidents du travail et de maladies professionnelles.
\item
  La direction générale de l'Institut national de la statistique et des
  études économiques (Insee) et, en tant que de besoin, de la délégation
  nationale à la lutte contre la fraude (DNLF), de la direction générale
  de l'enseignement scolaire (DGESCO), du délégué à l'information et à
  l'orientation, du secrétariat général des ministères économique et
  financier et de la direction des affaires juridiques des ministères
  économique et financier. Pour les questions liées aux mutations
  économiques, il peut disposer du commissariat général à l'égalité des
  territoires (CGET). La ministre du Travail peut également faire appel
  à la direction générale des étrangers en France. {[}Voir le décret
  n\degree{} 2014-406 du 16 avril 2014 modifié relatif aux attributions
  du ministre du travail, de l'emploi, de la formation professionnelle
  et du dialogue social{]}
\end{itemize}

\emph{Un réseau territorial et des opérateurs à compétence nationale qui
relayent l'action de la ministre}

\begin{itemize}

\item
  Les directions régionales des entreprises, de la concurrence, de la
  consommation, du travail et de l'emploi (DIRECCTE).
\item
  Des établissements publics à compétence nationale (l'agence nationale
  pour l'amélioration des conditions de travail, pôle emploi,
  ,\ldots{}).
\end{itemize}


\vspace{0.5cm}

\needspace{12\baselineskip}
\subsection*{A quelle distance de chez soi se fait-on hospitaliser ?
}
  \begin{wrapfigure}{r}{2.5cm}
    \centering
    \qrcode[nolink]{https://data.gouv.fr/dataset/53698effa3a729239d20360b}
  \end{wrapfigure}

Licence : \textbf{Licence Ouverte
}\newline
Créé le : 2013-07-08\newline
Modifié le : 2016-02-04\newline
De 2008-01-01 à 2008-12-31\newline
Popularité : 2 réutilisations,  0 suivi\newline
Mots-clé : \emph{aucun
}\newline
Permalien : \url{https://data.gouv.fr/dataset/53698effa3a729239d20360b}\newline

\par
\noindent
    Dans le Programme de médicalisation des systèmes d'information,
l'ensemble des séjours hospitaliers en médecine, chirurgie et
obstétrique) figure le code géographique de résidence du patient, qui
correspond le plus souvent au code postal. A l'aide d'une méthode
d'imputation aléatoire, un code communal de résidence est attribué à
chaque séjour. Le code communal de l'établissement est, lui,obtenu à
partir de l'Enquête statistique annuelle des établissements de santé
(SAE).La distance entre commune de résidence et commune
d'hospitalisation est ensuite calculée grâce au logiciel Odomatrix
développé par l'INRA. Ces distances sont calculées à vol d'oiseau, en
kilomètres-route, et en temps de trajet. C'est cette dernièreméthode qui
est utilisée dans l'étude, en faisant la moyenne entre temps en heures
creuses et temps en heures de pointe. Les patients hospitalisés dans
leur commune de résidence se voient affecter un temps de trajet
nul.L'étude se limite aux séjours des métropolitains (excepté les
corses) en France métropolitaine, hors Corse.


\vspace{0.5cm}
\needspace{12\baselineskip}
\subsection*{Base communale des zones d'emploi
}
  \begin{wrapfigure}{r}{2.5cm}
    \centering
    \qrcode[nolink]{https://data.gouv.fr/dataset/53698f4ba3a729239d2036dc}
  \end{wrapfigure}

Licence : \textbf{Licence Ouverte
}\newline
Créé le : 2013-07-08\newline
Modifié le : 2016-03-15\newline
De 2010-01-01 à 2010-12-31\newline
Popularité : 1 réutilisation,  3 suivis\newline
Mots-clé : \emph{aucun
}\newline
Permalien : \url{https://data.gouv.fr/dataset/53698f4ba3a729239d2036dc}\newline

\par
\noindent
    Fichiers des nouvelles zones d'emploi (espace géographique à l'intérieur
duquel la plupart des actifs résident et travaillent, et dans lesquels
les établissements peuvent trouver l'essentiel de la main d'œuvre
nécessaire pour occuper les emplois offerts). Le découpage se fonde sur
les flux de déplacement domicile-travail des actifs observés lors du
recensement de 2006.


\vspace{0.5cm}
\needspace{12\baselineskip}
\subsection*{Les écarts de salaires entre les hommes et les femmes
}\index{atteinte!aux!droits!de!l!homme!e}\index{egalite}\index{femme}\index{salaire}\index{travail}
  \begin{wrapfigure}{r}{2.5cm}
    \centering
    \qrcode[nolink]{https://data.gouv.fr/dataset/5369981ba3a729239d204e80}
  \end{wrapfigure}

Licence : \textbf{Licence Ouverte
}\newline
Créé le : 2013-10-29\newline
Modifié le : 2015-12-15\newline
De 2009-01-01 à 2009-12-31\newline
Granularité : au pays\newline
Mise à jour : annuelle\newline
Popularité : 2 réutilisations,  1 suivi\newline
Mots-clé : \emph{atteinte-aux-droits-de-l-homme-e, egalite, femme, salaire, travail
}\newline
Permalien : \url{https://data.gouv.fr/dataset/5369981ba3a729239d204e80}\newline

\par
\noindent
    

\vspace{0.5cm}
\needspace{3\baselineskip} \rule{4cm}{0.25pt}\newline\textbf{Aussi disponible du même producteur :}\begin{itemize}
\item \href{https://data.gouv.fr/dataset/53698e6aa3a729239d203465}{Activité féminine et composition familiale}
\item \href{https://data.gouv.fr/dataset/53698f1da3a729239d20365d}{Atlas des zones d'emploi}
\item \href{https://data.gouv.fr/dataset/53699042a3a729239d20396c}{Catalogue des enquêtes réalisées par la Dares}
\item \href{https://data.gouv.fr/dataset/53699042a3a729239d20396b}{Catalogue des enquêtes réalisées par la DARES - 2012}
\item \href{https://data.gouv.fr/dataset/5369925da3a729239d203ee1}{Demandeurs d'emploi par zones d'emploi en données brutes - août 2013}
\item \href{https://data.gouv.fr/dataset/5369925da3a729239d203ee2}{Demandeurs d'emploi par zones d'emploi en données brutes - juin 2013}
\item \href{https://data.gouv.fr/dataset/53699298a3a729239d203f7f}{Dépenses de santé remboursées par l’assurance maladie par région (soins de ville, établissements de santé publics et privés, établissements médico-sociaux) - juillet 2012}
\item \href{https://data.gouv.fr/dataset/536992eda3a729239d204059}{Disparités sur le marché du travail entre les femmes et les hommes}
\item \href{https://data.gouv.fr/dataset/536992f5a3a729239d20406b}{Distances et temps d’accès aux soins en France métropolitaine}
\item \href{https://data.gouv.fr/dataset/53699325a3a729239d2040ee}{Données de cadrage sur les allocations du minimum vieillesse}
\item \href{https://data.gouv.fr/dataset/53699345a3a729239d204152}{Durée collective hebdomadaire moyenne du travail de l'ensemble des salariés à temps complet}
\item \href{https://data.gouv.fr/dataset/53699346a3a729239d204153}{Durée collective hebdomadaire moyenne du travail des salariés à temps complet - mars 2013}
\item \href{https://data.gouv.fr/dataset/53699368a3a729239d2041a9}{Effectif de l'emploi salarié en France}
\item \href{https://data.gouv.fr/dataset/5369940aa3a729239d204350}{Emploi salarié}
\item \href{https://data.gouv.fr/dataset/536996cba3a729239d204b21}{Indices de salaire de base par secteur d’activité et catégorie socioprofessionnelle - décembre 2012}
\item \href{https://data.gouv.fr/dataset/536996cca3a729239d204b23}{Indices de salaire de base par secteur d’activité et catégorie socioprofessionnelle - mars 2013}
\item \href{https://data.gouv.fr/dataset/536996cda3a729239d204b26}{Indices de salaires de base par secteur d’activité et catégorie socioprofessionnelle}
\item \href{https://data.gouv.fr/dataset/536997a7a3a729239d204d50}{La liquidation des droits à la retraite}
\item \href{https://data.gouv.fr/dataset/536997d7a3a729239d204dcd}{Le montant des pensions et son évolution}
\item \href{https://data.gouv.fr/dataset/536997d8a3a729239d204dd0}{L'emploi salarié au premier trimestre 2013}
\item \href{https://data.gouv.fr/dataset/536997eea3a729239d204e08}{Les affectations des étudiants en médecine à l’issue des épreuves classantes nationales en 2010}
\item \href{https://data.gouv.fr/dataset/536997f0a3a729239d204e0e}{Les allocataires de l'assurance vieillesse}
\item \href{https://data.gouv.fr/dataset/536997f4a3a729239d204e1a}{Les bénéficiaires de l’allocation compensatrice pour tierce personne et de la prestation de compensation du handicap : deux populations bien différentes}
\item \href{https://data.gouv.fr/dataset/536997f6a3a729239d204e1e}{Les bénéficiaires du minimum vieillesse}
\item \href{https://data.gouv.fr/dataset/53699828a3a729239d204ea1}{Les effectifs des retraités}
\item \href{https://data.gouv.fr/dataset/53699843a3a729239d204eed}{Les événements indésirables graves dans les établissements de santé : fréquence, évitabilité et acceptabilité}
\item \href{https://data.gouv.fr/dataset/53699844a3a729239d204ef1}{Les flux de main-d'oeuvre}
\item \href{https://data.gouv.fr/dataset/53699858a3a729239d204f26}{Les maternités en 2010 - Premiers résultats de l’enquête nationale périnatale}
\item \href{https://data.gouv.fr/dataset/5369985aa3a729239d204f2b}{Les motivations du départ à la retraite}
\item \href{https://data.gouv.fr/dataset/5369985ba3a729239d204f2d}{Les mouvements de main-d'oeuvre}
\item \href{https://data.gouv.fr/dataset/5369985ba3a729239d204f2c}{Les mouvements de main-d'oeuvre}
\item \href{https://data.gouv.fr/dataset/5369986ca3a729239d204f5d}{Les portraits statistiques des métiers de 1982 à 2009}
\item \href{https://data.gouv.fr/dataset/5369986ca3a729239d204f5e}{Les portraits statistiques des métiers de 1982 à 2011}
\item \href{https://data.gouv.fr/dataset/53699881a3a729239d204f9d}{Les statistiques d'emploi intérimaire}
\item \href{https://data.gouv.fr/dataset/53699881a3a729239d204f9c}{Les statistiques d'emploi intérimaire}
\item \href{https://data.gouv.fr/dataset/53699882a3a729239d204f9e}{Les statistiques d'emploi intérimaire - mars 2013}
\item \href{https://data.gouv.fr/dataset/536998caa3a729239d205059}{L’implication de l’entourage et des professionnels auprès des personnes âgées à domicile}
\item \href{https://data.gouv.fr/dataset/585145edc751df272cc0bb7e}{Liste Publique des Organismes de Formation (L.6351-7-1 du Code du Travail)}
\item \href{https://data.gouv.fr/dataset/5aec1eb0c751df792b114bd6}{Résultats des élections professionnelles - période 2009-2012}
\item \href{https://data.gouv.fr/dataset/5b06878fc751df7d56f2501b}{Résultats des élections professionnelles - période 2013-2016}
\item \href{https://data.gouv.fr/dataset/53699f3da3a729239d2060ac}{Résultats détaillés des enquêtes sur les Conditions de travail}
\item \href{https://data.gouv.fr/dataset/53699fd2a3a729239d206204}{Séries mensuelles nationales en données corrigées des variations saisonnières et des jours ouvrables sur les offres et demandes d'emploi}
\item \href{https://data.gouv.fr/dataset/53699fd8a3a729239d206209}{Séries mensuelles nationales en données corrigées des variations saisonnières et des jours ouvrables sur les offres et demandes d'emploi - août 2013}
\item \href{https://data.gouv.fr/dataset/53699fd2a3a729239d206206}{Séries mensuelles nationales en données corrigées des variations saisonnières et des jours ouvrables sur les offres et demandes d'emploi - février 2013}
\item \href{https://data.gouv.fr/dataset/53699fd2a3a729239d206205}{Séries mensuelles nationales en données corrigées des variations saisonnières et des jours ouvrables sur les offres et demandes d'emploi - janvier 2013}
\item \href{https://data.gouv.fr/dataset/53699fd3a3a729239d206208}{Séries mensuelles nationales en données corrigées des variations saisonnières et des jours ouvrables sur les offres et demandes d'emploi - juin 2013}
\item \href{https://data.gouv.fr/dataset/53699fd3a3a729239d206207}{Séries mensuelles nationales en données corrigées des variations saisonnières et des jours ouvrables sur les offres et demandes d'emploi - mai 2013}
\item \href{https://data.gouv.fr/dataset/53699fd8a3a729239d20620a}{Séries par zone d'emploi sur les offres et demandes d'emploi - février 2013}
\item \href{https://data.gouv.fr/dataset/53699fd9a3a729239d20620b}{Séries par zone d'emploi sur les offres et demandes d'emploi - janvier 2013}
\item \href{https://data.gouv.fr/dataset/53699fd9a3a729239d20620c}{Séries par zone d'emploi sur les offres et demandes d'emploi - mai 2013}
\item et 15 autres jeux de données\end{itemize}

\clearpage
\section{Musée du Louvre}


\begin{center}
  \includegraphics[width=3cm]{images/orga/02_1435bfbd43421ebd9c0114801e1572-100.jpg}
\end{center}


Le musée du Louvre présente des œuvres de l'art occidental du Moyen Âge
à 1848, des civilisations antiques qui l'ont précédé et influencé et des
arts d'Islam. Les collections sont réparties en huit départements qui
ont leur histoire propre, liée aux conservateurs, aux collectionneurs et
aux donateurs.

Plus d'informations :

\begin{itemize}

\item
  Site du \href{http://www.louvre.fr/}{musée du Louvre}
\item
  Site des \href{http://www.louvre.fr/departements}{collections}
\end{itemize}


\vspace{0.5cm}

\needspace{12\baselineskip}
\subsection*{Programmation événementielle et culturelle du Musée du Louvre
}\index{agenda}\index{culture}\index{evenement}\index{spectacles}
  \begin{wrapfigure}{r}{2.5cm}
    \centering
    \qrcode[nolink]{https://data.gouv.fr/dataset/53699e30a3a729239d205e02}
  \end{wrapfigure}

Licence : \textbf{Licence Ouverte
}\newline
Créé le : 2013-10-20\newline
Modifié le : 2016-08-04\newline
Mise à jour : trimestrielle\newline
Popularité : 1 réutilisation,  0 suivi\newline
Mots-clé : \emph{agenda, culture, evenement, spectacles
}\newline
Permalien : \url{https://data.gouv.fr/dataset/53699e30a3a729239d205e02}\newline

\par
\noindent
    Liste des manifestations (expositions, ateliers, visites
guidées\ldots{}) programmées au musée du Louvre pour la période. Fichier
Unique de Programmation du musée du Louvre.


\vspace{0.5cm}

\clearpage
\section{Observatoire national de la délinquance et des réponses pénales (ondrp)}


\begin{center}
  \includegraphics[width=3cm]{images/orga/87_a13072359e4c9d8bf70a1aaf771bfd-100.jpg}
\end{center}


L'Observatoire national de la délinquance et des réponses pénales
(ONDRP) est un département de l'Institut national des hautes études de
la sécurité et de la Justice. Il est doté d'un conseil d'orientation
chargé d'assurer l'indépendance de ses travaux. Il a comme activité
principale la production et la diffusion de statistiques sur la
criminalité et la délinquance. L'ONDRP inscrit ses travaux dans le cadre
de la statistique publique et du code des bonnes pratiques de la
statistique européenne.

L'ONDRP a notamment pour mission de recueillir les données statistiques
relatives à la délinquance auprès de tous les départements ministériels
et organismes publics ou privés ayant à connaître directement ou
indirectement de faits ou de situations d'atteinte aux personnes ou aux
biens. A ce titre, il analyse et diffuse les données sur les crimes et
délits enregistrés par les services de police et les unités de la
gendarmerie nationales. Avec l'INSEE, il conçoit et exploite l'enquête
nationale de victimation « Cadre de vie et sécurité ». Il a également la
responsabilité de la production d'études sur l'évolution des phénomènes
criminels à travers une approche multi-sources et, depuis le 1er janvier
2010, en y intégrant les données sur la réponse pénale produites par le
ministère de la Justice. Il organise la communication des résultats de
ses études à travers des publications régulières. Le département est
dirigé par Christophe SOULLEZ.


\vspace{0.5cm}

\needspace{12\baselineskip}
\subsection*{Faits constatés mensuels par départements
}\index{crime}\index{delinquance}\index{delit}\index{faits!constates}
  \begin{wrapfigure}{r}{2.5cm}
    \centering
    \qrcode[nolink]{https://data.gouv.fr/dataset/53699576a3a729239d20471b}
  \end{wrapfigure}

Licence : \textbf{Licence Ouverte
}\newline
Créé le : 2013-08-16\newline
Modifié le : 2016-03-15\newline
De 2012-01-01 à 2012-12-31\newline
Granularité : au département\newline
Mise à jour : annuelle\newline
Popularité : 2 réutilisations,  1 suivi\newline
Mots-clé : \emph{crime, delinquance, delit, faits-constates
}\newline
Permalien : \url{https://data.gouv.fr/dataset/53699576a3a729239d20471b}\newline

\par
\noindent
    Faits constatés mensuels par département en 2012


\vspace{0.5cm}
\needspace{12\baselineskip}
\subsection*{Faits constatés Zone Gendarmerie
}\index{crimes}\index{delinquance}\index{delits}\index{faits!constates}\index{gendarmerie}
  \begin{wrapfigure}{r}{2.5cm}
    \centering
    \qrcode[nolink]{https://data.gouv.fr/dataset/53699577a3a729239d204720}
  \end{wrapfigure}

Licence : \textbf{Licence Ouverte
}\newline
Créé le : 2013-08-16\newline
Modifié le : 2016-02-28\newline
De 2012-01-01 à 2012-01-31\newline
Granularité : au département\newline
Mise à jour : mensuelle\newline
Popularité : 1 réutilisation,  2 suivis\newline
Mots-clé : \emph{crimes, delinquance, delits, faits-constates, gendarmerie
}\newline
Permalien : \url{https://data.gouv.fr/dataset/53699577a3a729239d204720}\newline

\par
\noindent
    Faits Constatés en 2012 zone gendarmerie


\vspace{0.5cm}
\needspace{12\baselineskip}
\subsection*{Faits constatés Zone Police
}\index{criminalite}\index{faits!constates}\index{infractions}\index{police}
  \begin{wrapfigure}{r}{2.5cm}
    \centering
    \qrcode[nolink]{https://data.gouv.fr/dataset/53699578a3a729239d204722}
  \end{wrapfigure}

Licence : \textbf{Licence Ouverte
}\newline
Créé le : 2013-08-16\newline
Modifié le : 2016-02-25\newline
De 2012-01-01 à 2012-01-31\newline
Granularité : au département\newline
Mise à jour : mensuelle\newline
Popularité : 2 réutilisations,  3 suivis\newline
Mots-clé : \emph{criminalite, faits-constates, infractions, police
}\newline
Permalien : \url{https://data.gouv.fr/dataset/53699578a3a729239d204722}\newline

\par
\noindent
    Faits constatés en 2012 zone police


\vspace{0.5cm}
\needspace{12\baselineskip}
\subsection*{Les crimes et délits enregistrés par la gendarmerie nationale
}
  \begin{wrapfigure}{r}{2.5cm}
    \centering
    \qrcode[nolink]{https://data.gouv.fr/dataset/5387f348a3a7291cb36754a3}
  \end{wrapfigure}

Licence : \textbf{Licence Ouverte
}\newline
Créé le : 2014-05-27\newline
Modifié le : 2016-02-16\newline
De 2013-01-01 à 2013-12-31\newline
Granularité : au département\newline
Mise à jour : semestrielle\newline
Popularité : 2 réutilisations,  1 suivi\newline
Mots-clé : \emph{aucun
}\newline
Permalien : \url{https://data.gouv.fr/dataset/5387f348a3a7291cb36754a3}\newline

\par
\noindent
    Les données des faits constatés sont fournies sous forme de tables
comprenant en lignes les 107 index d'infractions de l'état statistique
4001 et en colonnes les départements et collectivités d'outre-mer. Il
s'agit des faits enregistrés par l'ensemble des unités de la gendarmerie
nationale localisés au lieu de leur enregistrement.


\vspace{0.5cm}
\needspace{12\baselineskip}
\subsection*{Les crimes et délits enregistrés par la police nationale
}
  \begin{wrapfigure}{r}{2.5cm}
    \centering
    \qrcode[nolink]{https://data.gouv.fr/dataset/5387f349a3a7291cb36754a5}
  \end{wrapfigure}

Licence : \textbf{Licence Ouverte
}\newline
Créé le : 2014-05-27\newline
Modifié le : 2016-03-14\newline
De 2013-01-01 à 2013-12-21\newline
Granularité : au département\newline
Mise à jour : semestrielle\newline
Popularité : 2 réutilisations,  1 suivi\newline
Mots-clé : \emph{aucun
}\newline
Permalien : \url{https://data.gouv.fr/dataset/5387f349a3a7291cb36754a5}\newline

\par
\noindent
    Les données des faits constatés sont fournies sous forme de tables
comprenant en lignes les 107 index d'infractions de l'état statistique
4001 et en colonnes les départements et collectivités d'outre-mer. Il
s'agit des faits enregistrés par l'ensemble des services de la police
nationale localisés au lieu de leur enregistrement.


\vspace{0.5cm}
\needspace{3\baselineskip} \rule{4cm}{0.25pt}\newline\textbf{Aussi disponible du même producteur :}\begin{itemize}
\item \href{https://data.gouv.fr/dataset/5387eafca3a7291cb3675498}{Crimes et délits constatés par la police nationale en 2013}
\item \href{https://data.gouv.fr/dataset/53699202a3a729239d203dea}{Crimes et délits enregistrés par les services de police et les unités de gendarmerie}
\item \href{https://data.gouv.fr/dataset/53699575a3a729239d20471a}{Faits constatés annuels par index 4001 pour les services centraux}
\item \href{https://data.gouv.fr/dataset/53699576a3a729239d20471d}{Faits constatés par département, par index et par mois}
\item \href{https://data.gouv.fr/dataset/5387f347a3a7291cb36754a1}{Les crimes et délits enregistrés par la gendarmerie nationale}
\item \href{https://data.gouv.fr/dataset/5387f34aa3a7291cb36754a8}{Les crimes et délits enregistrés par la police nationale}
\end{itemize}

\clearpage
\section{Occitanie Pyrénées en Intelligence Géomatique}


\begin{center}
  \includegraphics[width=3cm]{images/orga/58_92f5bb2b7949bd96d2328000fe2a42-100.png}
\end{center}


Dans le champ d'activité qui est le sien et dans le respect de
l'autonomie scientifique et administrative de ses membres, OPenIG a des
missions définies selon les finalités principales suivantes :

\begin{itemize}

\item
  Diffuser et promouvoir l'information géographique numérique,
\item
  faciliter le montage et le portage de projets par ses membres,
\item
  acquérir et mettre à disposition des bases de données géographiques.
\end{itemize}

L'association OPenIG (ancienne SIG LR) porte depuis 20 ans des projets
structurants en lien avec l'information géographique numérique au
service de nos territoires. Elle regroupe différentes collectivités
locales (communes, EPCI, conseils généraux, conseil régional), les
Services de l'État, des organismes publics (Safer, EID, EIPFM, etc.)
ainsi que des associations ou des entreprises.


\vspace{0.5cm}

\needspace{12\baselineskip}
\subsection*{Les points d'apport volontaires enterrés de Montpellier Méditerranée
Métropole
}\index{colonne}\index{colonnes}\index{colonnes!enterrees}\index{dechet}\index{dechets}\index{donnees!ouvertes}\index{dpgd}\index{ordures!menageres}\index{papier}\index{passerelle!inspire}\index{pav}\index{point!dapport!volontaire}\index{secteur!des!services}\index{structure}\index{verre}
  \begin{wrapfigure}{r}{2.5cm}
    \centering
    \qrcode[nolink]{https://data.gouv.fr/dataset/559ac538c751df6ddb390bd3}
  \end{wrapfigure}

Licence : \textbf{Licence Ouverte version 2.0
}\newline
Créé le : 2015-07-06\newline
Modifié le : 2019-02-08\newline
Popularité : 1 réutilisation,  0 suivi\newline
Mots-clé : \emph{colonne, colonnes, colonnes-enterrees, dechet, dechets, donnees-ouvertes, dpgd, ordures-menageres, papier, passerelle-inspire, pav, point-dapport-volontaire, secteur-des-services, structure, verre
}\newline
Permalien : \url{https://data.gouv.fr/dataset/559ac538c751df6ddb390bd3}\newline

\par
\noindent
    Couche localisant les points d'apport volontaires enterrés de
Montpellier Méditerranée Métropole (colonnes verre/papier/ordures
ménagères/tri sélectif)

\textbf{Origine}

Montpellier Méditerranée Métropole

\textbf{Organisations partenaires}

Montpellier Méditerranée Métropole

➞
\href{https://geo.data.gouv.fr/fr/datasets/3e35ea9b7c19c73c575bf54703b4d5d8fadd1ee6}{Consulter
cette fiche sur geo.data.gouv.fr}


\vspace{0.5cm}
\needspace{12\baselineskip}
\subsection*{Stations météorologiques dans l'Hérault
}\index{donnees!ouvertes}\index{geoscientific!information}\index{meteorologie}\index{passerelle!inspire}\index{surveillance!meteorologique}\index{temperature}\index{temperature!de!lair}\index{usage!des!sols}
  \begin{wrapfigure}{r}{2.5cm}
    \centering
    \qrcode[nolink]{https://data.gouv.fr/dataset/561270abc751df24ac756894}
  \end{wrapfigure}

Licence : \textbf{Licence Ouverte version 2.0
}\newline
Créé le : 2015-10-05\newline
Modifié le : 2019-02-08\newline
Popularité : 1 réutilisation,  0 suivi\newline
Mots-clé : \emph{donnees-ouvertes, geoscientific-information, meteorologie, passerelle-inspire, surveillance-meteorologique, temperature, temperature-de-lair, usage-des-sols
}\newline
Permalien : \url{https://data.gouv.fr/dataset/561270abc751df24ac756894}\newline

\par
\noindent
    Stations météorologiques dans l'Hérault fournies par l'Association
Climatologique de l'Hérault (ACH 34). Ces stations sont la source de
données climatiques dans le département.

\textbf{Origine}

Données libres issues des stations météo de l'Hérault.

\textbf{Organisations partenaires}

Association Climatologique de l'Hérault

\textbf{Liens annexes}

\begin{itemize}

\item
  \href{http://www.ach34.fr}{Association Climatologique de l'Hérault}
\end{itemize}

➞
\href{https://geo.data.gouv.fr/fr/datasets/351f20f1380c1067d89b6b5f49adf25c9dd8d8fb}{Consulter
cette fiche sur geo.data.gouv.fr}


\vspace{0.5cm}
\needspace{3\baselineskip} \rule{4cm}{0.25pt}\newline\textbf{Aussi disponible du même producteur :}\begin{itemize}
\item \href{https://data.gouv.fr/dataset/588a538a88ee381e139b81a4}{Arrêts de bus Communauté d'Agglomération le Grand Narbonne}
\item \href{https://data.gouv.fr/dataset/5735c8a488ee38237cd1b934}{[CD34] Aléa chute de blocs dans l'Hérault}
\item \href{https://data.gouv.fr/dataset/5735c8a4c751df709f8cc4b3}{[CD34] Aléa effondrement dans l'Hérault}
\item \href{https://data.gouv.fr/dataset/5735e4eec751df256b8cc4b3}{[CD34] Aléa glissement de terrain dans l'Hérault}
\item \href{https://data.gouv.fr/dataset/576d4b16c751df66d0537a11}{[CD34] Aménagement cyclables en site propre du Conseil Départemental de l’Hérault}
\item \href{https://data.gouv.fr/dataset/576d4ea9c751df193f537a11}{[CD34] Boucles cyclotouristiques du Conseil Départemental de l’Hérault}
\item \href{https://data.gouv.fr/dataset/57ab8f4e88ee386299b627a9}{[CD34] Cantons de l'Hérault}
\item \href{https://data.gouv.fr/dataset/57ab8c4c88ee382fa8b627a9}{[CD34] Collèges publics de l'Hérault}
\item \href{https://data.gouv.fr/dataset/57ab8c4ec751df416b97bae5}{[CD34] Domaines départementaux en espaces naturels}
\item \href{https://data.gouv.fr/dataset/576d415b88ee3843deab6512}{[CD34] Installations de transit, de traitement et de stockage des déchets non dangereux}
\item \href{https://data.gouv.fr/dataset/57ab8c52c751df3c2197bae5}{[CD34] Point d'eau DFCI de l'Hérault}
\item \href{https://data.gouv.fr/dataset/57ab8c55c751df44f997bae5}{[CD34] Réseau routier départemental de l'Hérault}
\item \href{https://data.gouv.fr/dataset/57ab8c5988ee38565db627aa}{[CD34] Sites du département de l'Hérault}
\item \href{https://data.gouv.fr/dataset/57ab8c5cc751df416b97bae6}{[CD34] Stations d'épuration et points de rejet des effluents traités de l'Hérault}
\item \href{https://data.gouv.fr/dataset/57ab8c5e88ee382fa8b627aa}{[CD34] Tour de guet DFCI de l'Hérault}
\item \href{https://data.gouv.fr/dataset/5735e0ddc751df1ec58cc4b3}{[CD34] Zones de fracturation dans l'Hérault}
\item \href{https://data.gouv.fr/dataset/5a0aeb17c751df2c545d86c7}{[DDTM34] Documents d'urbanisme 34063\_caux}
\item \href{https://data.gouv.fr/dataset/5612749bc751df1d77756894}{Défibrillateurs}
\item \href{https://data.gouv.fr/dataset/5a0aeb16c751df2fdda1d849}{Indice de Qualité des Sols}
\item \href{https://data.gouv.fr/dataset/571f81bec751df2249fcca0e}{Occupation du sol : 2001, 2009 et 2012 sur le territoire de la CC Grand Pic Saint-Loup}
\item \href{https://data.gouv.fr/dataset/5c5a8eff634f416988995ec7}{Occupation du sol : 2003, 2012 et 2015 sur le territoire du Grand Narbonne Communauté d'Agglomération et du Parc naturel régional de la Narbonnaise en Méditerranée}
\item \href{https://data.gouv.fr/dataset/571f781f88ee380797a19f12}{Occupation du sol : 2003 et 2012 sur le territoire du Grand Narbonne Communauté d'Agglomération et du Parc naturel régional de la Narbonnaise en Méditerranée}
\item \href{https://data.gouv.fr/dataset/573c374b88ee3870d8d1b934}{Orthophotographie Express 2015 du Gard}
\item \href{https://data.gouv.fr/dataset/559aebe388ee381360764f5c}{PLU Zonages - Montpellier Méditerranée Métropole}
\item \href{https://data.gouv.fr/dataset/559ac36ac751df767c390bd4}{Stations MODULAUTO - Montpellier Méditerranée Métropole}
\end{itemize}

\clearpage
\section{Office National des Forêts}


\begin{center}
  \includegraphics[width=3cm]{images/orga/94_70e48244df43e78c8fe3813b753967-100.jpg}
\end{center}


Autant par culture que par nécessité, les forestiers ont été les
pionniers du développement durable (l'ordonnance royale de Brunoy en
1346 demande déjà aux forestiers de gérer la forêt de telle sorte que
les « bois se puissent perpétuellement soustenir en bon état » !).
Aujourd'hui, appuyée sur ses savoir-faire en matière de gestion
multifonctionnelle et durable des forêts et des espaces naturels,
l'action de l'Office national des forêts, gestionnaire des forêts
publiques, s'inscrit naturellement dans la modernité. Au service de la
société, l'Office prépare avec ses partenaires la forêt et les espaces
naturels de demain et agit pour qu'ils participent activement à la
résolution des grands enjeux du développement durable : lutte contre les
changements climatiques, développement des énergies renouvelables,
conservation de la biodiversité, qualité de l'eau, prévention contre les
risques naturels\ldots{} tout en assurant au meilleur niveau la fonction
essentielle de production de bois.


\vspace{0.5cm}

\needspace{12\baselineskip}
\subsection*{Forêts publiques
}\index{foret}\index{onf}
  \begin{wrapfigure}{r}{2.5cm}
    \centering
    \qrcode[nolink]{https://data.gouv.fr/dataset/536995b6a3a729239d2047d0}
  \end{wrapfigure}

Licence : \textbf{Licence Ouverte
}\newline
Créé le : 2013-07-08\newline
Modifié le : 2019-01-15\newline
Mise à jour : annuelle\newline
Popularité : 2 réutilisations,  3 suivis\newline
Mots-clé : \emph{foret, onf
}\newline
Permalien : \url{https://data.gouv.fr/dataset/536995b6a3a729239d2047d0}\newline

\par
\noindent
    Contours des forêts publiques relevant du régime forestier : terrains
domaniaux et communaux, y compris les terrains qui ne sont pas en nature
de forêt. Ces contours ne doivent pas être considérés comme un
référentiel mais uniquement à titre informatif.


\vspace{0.5cm}
\needspace{12\baselineskip}
\subsection*{Réserves biologiques de métropole
}\index{onf}\index{reserve!biologique}
  \begin{wrapfigure}{r}{2.5cm}
    \centering
    \qrcode[nolink]{https://data.gouv.fr/dataset/53699f17a3a729239d20604c}
  \end{wrapfigure}

Licence : \textbf{Licence Ouverte
}\newline
Créé le : 2013-07-08\newline
Modifié le : 2016-06-09\newline
Mise à jour : annuelle\newline
Popularité : 1 réutilisation,  2 suivis\newline
Mots-clé : \emph{onf, reserve-biologique
}\newline
Permalien : \url{https://data.gouv.fr/dataset/53699f17a3a729239d20604c}\newline

\par
\noindent
    Contours géographiques des réserves biologiques (RB) intégrales,
dirigées oumixtes gérées par l'ONF. Ce sont des espaces forestiers rares
ou fragiles, dans lesforêts domaniales et dans les forêts non domaniales
relevant du régime forestier(principalement les forêts communales). Les
réserves biologiques sont protégées pararrêté du ministre chargé de la
forêt et leur gestion particulière est orientée vers lasauvegarde de la
faune, de la flore ou de toute autre ressource naturelle


\vspace{0.5cm}

\clearpage
\section{OpenDataFrance}


\begin{center}
  \includegraphics[width=3cm]{images/orga/90_0f5fc12ba34021ba5c1728bdc62241-100.png}
\end{center}


L'association OpenDataFrance fédère et accompagne les collectivités
locales dans leur projet d'ouverture des données. Elle est composée de
collectivités territoriales et de membres partenaires (Etalab, Fing,
Libertic, \ldots{})


\vspace{0.5cm}

\needspace{12\baselineskip}
\subsection*{Données de l'Observatoire open data des territoires - Millésime Mars
2018
}\index{namr}\index{odater}\index{opendatafrance}
  \begin{wrapfigure}{r}{2.5cm}
    \centering
    \qrcode[nolink]{https://data.gouv.fr/dataset/5ad9eba988ee38286a988685}
  \end{wrapfigure}

Licence : \textbf{Licence Ouverte
}\newline
Créé le : 2018-04-20\newline
Modifié le : 2018-10-09\newline
Granularité : à la commune\newline
Popularité : 2 réutilisations,  1 suivi\newline
Mots-clé : \emph{namr, odater, opendatafrance
}\newline
Permalien : \url{https://data.gouv.fr/dataset/5ad9eba988ee38286a988685}\newline

\par
\noindent
    Ce jeu de données recense les plateformes et les organisations qui
participent au développement de l'open data dans les territoires. Il ne
concerne, pour le moment, que les collectivités locales françaises
(restreintes aux communes, EPCI à fiscalité propre, départements et
régions) qui publient des données ouvertes (a minima un jeu de données).
Il a été co-produit avec \href{http://namr.com/}{namR}.

Il contient les ressources suivantes :

\begin{itemize}

\item
  Modèle de données qui décrit la structure des 2 tables
\item
  Données brutes de la table des plateformes (PTF)
\item
  Données brutes de la table organisations (ORGA)
\item
  Carte des collectivités open data (voir la
  \href{https://umap.openstreetmap.fr/fr/map/carte-odt-prefiguration_206077}{carte
  uMap})
\end{itemize}


\vspace{0.5cm}
\needspace{3\baselineskip} \rule{4cm}{0.25pt}\newline\textbf{Aussi disponible du même producteur :}\begin{itemize}
\item \href{https://data.gouv.fr/dataset/5bbcb304634f41769f799b9a}{Données de l'Observatoire open data des territoires - Millésime Octobre 2018}
\item \href{https://data.gouv.fr/dataset/5aa0060ac751df126665fc65}{Liste des collectivités territoriales ouvertes}
\end{itemize}

\clearpage
\section{Pôle emploi}


\begin{center}
  \includegraphics[width=3cm]{images/orga/79_02694fd7de42b9901a0d220857e9fc-100.png}
\end{center}


Pôle emploi est un acteur majeur du marché de l'emploi où il s'investit
pour faciliter le retour à l'emploi des demandeurs d'emploi et offrir
aux entreprises des réponses adaptées à leurs besoins de recrutement.

Les 54 000 collaborateurs de Pôle emploi œuvrent au quotidien pour être
le trait d'union entre les demandeurs d'emploi et les entreprises. Pour
cela, ils peuvent s'appuyer sur une offre de services simplifiée, issue
du projet stratégique 2015-2020, et sur un réseau de partenaires qui
s'investissent sur les territoires au plus près des besoins.


\vspace{0.5cm}

\needspace{12\baselineskip}
\subsection*{Offres d'emploi diffusées à Pôle emploi
}\index{assurance!chomage}\index{employeur}\index{offre!d!emploi}
  \begin{wrapfigure}{r}{2.5cm}
    \centering
    \qrcode[nolink]{https://data.gouv.fr/dataset/561fc15688ee3836b1628efc}
  \end{wrapfigure}

Licence : \textbf{Licence Ouverte
}\newline
Créé le : 2015-10-15\newline
Modifié le : 2017-11-29\newline
Granularité : au pays\newline
Mise à jour : ponctuelle\newline
Popularité : 1 réutilisation,  1 suivi\newline
Mots-clé : \emph{assurance-chomage, employeur, offre-d-emploi
}\newline
Permalien : \url{https://data.gouv.fr/dataset/561fc15688ee3836b1628efc}\newline

\par
\noindent
    Le nombre d'offres publiées ici correspond à l'ensemble des offres
accessibles par les demandeurs d'emploi sur le site pole-emploi.fr

Les offres d'emploi diffusées par Pôle emploi proviennent ainsi de deux
sources : - les offres déposées directement à Pôle emploi par les
employeurs ; - les offres transmises à Pôle emploi par des sites
partenaires pour rediffusion.

Les données publiées sur les offres d'emploi diffusées par Pôle emploi
sont des données brutes et sont mises à jour trimestriellement.


\vspace{0.5cm}
\needspace{12\baselineskip}
\subsection*{Répertoire Opérationnel des Métiers et des Emplois (ROME)
}\index{competences}\index{demande!d!emploi}\index{fiche!metier}\index{offre!d!emploi}\index{pole!emploi}\index{rome}\index{skills}
  \begin{wrapfigure}{r}{2.5cm}
    \centering
    \qrcode[nolink]{https://data.gouv.fr/dataset/58da857388ee384902e505f5}
  \end{wrapfigure}

Licence : \textbf{Licence Ouverte
}\newline
Créé le : 2017-03-28\newline
Modifié le : 2018-03-20\newline
Mise à jour : trimestrielle\newline
Popularité : 2 réutilisations,  8 suivis\newline
Mots-clé : \emph{competences, demande-d-emploi, fiche-metier, offre-d-emploi, pole-emploi, rome, skills
}\newline
Permalien : \url{https://data.gouv.fr/dataset/58da857388ee384902e505f5}\newline

\par
\noindent
    Dans un contexte marqué par de fortes mutations de l'environnement
économique et social, le ROME (Répertoire Opérationnel des Métiers et
des Emplois) est un outil au service de la mobilité professionnelle et
du rapprochement entre offres et candidats.

Le ROME a été construit par les équipes de Pôle emploi avec la
contribution d'un large réseau de partenaires (entreprises, branches et
syndicats professionnels, AFPA\ldots{}), en s'appuyant sur une démarche
pragmatique : inventaire des dénominations d'emplois/métiers les plus
courantes, analyse des activités et compétences, regroupement des
emplois selon un principe d'équivalence ou de proximité.

En décembre 2016, les référentiels de compétences du ROME évoluent afin
d'améliorer la transversalité lors du rapprochement entre l'offre et la
demande. Cette évolution consiste à : • réorganiser les compétences en
savoir-faire et savoirs • reformuler les libellés en les simplifiant et
les décontextualisant.


\vspace{0.5cm}
\needspace{3\baselineskip} \rule{4cm}{0.25pt}\newline\textbf{Aussi disponible du même producteur :}\begin{itemize}
\item \href{https://data.gouv.fr/dataset/561fa8bbc751df54a1cdbb48}{Allocataires de l'assurance chômage}
\item \href{https://data.gouv.fr/dataset/561fc0a888ee38330f628efc}{Annuaires de la demande d'emploi 2013, 2014, 2015 et 2016}
\item \href{https://data.gouv.fr/dataset/59a5379bc751df5e208d72db}{API Catalogue des services Emploi Store }
\item \href{https://data.gouv.fr/dataset/5ac7761bc751df636cad973f}{Calcul de l’allocation d’aide au retour à l’emploi (ARE) }
\item \href{https://data.gouv.fr/dataset/561fc00488ee381fd2628ef9}{Demandeurs d'emploi bénéficiaires du RSA}
\item \href{https://data.gouv.fr/dataset/561fbf0d88ee3836b1628efb}{Demandeurs d’emploi inscrits à Pôle emploi}
\item \href{https://data.gouv.fr/dataset/561fc26a88ee3836b1628efd}{Emploi intérimaire}
\item \href{https://data.gouv.fr/dataset/561fa564c751df4f2acdbb48}{Enquête Besoins en Main d'Oeuvre (BMO)}
\item \href{https://data.gouv.fr/dataset/58595bab88ee387949c65bb3}{Entrées en formation des demandeurs d’emploi }
\item \href{https://data.gouv.fr/dataset/593156ff88ee38351b503df3}{Informations sur le marché du travail - API Infotravail}
\item \href{https://data.gouv.fr/dataset/57235a51c751df3898fcca0d}{Montant d'allocation chômage et salaires de référence des allocataires de l'Assurance chômage}
\item \href{https://data.gouv.fr/dataset/5931593b88ee38351ad532c6}{Offres d'emploi anonymisées - API Infotravail}
\item \href{https://data.gouv.fr/dataset/572355fcc751df3016fcca0d}{Part des demandeurs d'emploi indemnisables}
\item \href{https://data.gouv.fr/dataset/5931587388ee385fac21149e}{Référentiel des agences Pôle emploi - API Infotravail }
\item \href{https://data.gouv.fr/dataset/59315a2c88ee385fac21149f}{Répertoire Opérationnel des Métiers et des Emplois - API Infotravail}
\item \href{https://data.gouv.fr/dataset/593158d188ee385faa18c99b}{Résultats des enquêtes Besoins en Main-d'Œuvre - API infotravail}
\item \href{https://data.gouv.fr/dataset/59315d6988ee385fab9c51ef}{Sélection d'entreprises à fort potentiel d'embauche - API La Bonne Boite}
\item \href{https://data.gouv.fr/dataset/58595d4888ee387bd4c65bb3}{Sortants de formation et retour à l'emploi}
\item \href{https://data.gouv.fr/dataset/593159a188ee385fade6010e}{Statistiques sur le marché du travail - API infotravail}
\item \href{https://data.gouv.fr/dataset/5948de7dc751df5abbf5adb3}{Taux de retour à l'emploi par formation  - API Retour à l'emploi suite formation}
\end{itemize}

\clearpage
\section{Premier ministre}


\begin{center}
  \includegraphics[width=3cm]{images/orga/08_98902e29244685862bcdd3198cef7b-100.png}
\end{center}


Aux côtés de son cabinet, l'administration du Premier ministre comprend
de nombreux services qui l'assistent et prennent part à l'élaboration de
la politique du Gouvernement. Sont rattachés au Premier ministre les
organismes chargés de missions de coordination interministérielle qui ne
peuvent être attribuées à un seul ministère.


\vspace{0.5cm}

\needspace{12\baselineskip}
\subsection*{Acteurs de la vie publique - vie-publique.fr
}\index{acteurs}\index{acteurs!administratifs}\index{acteurs!institutionnels}\index{dila}\index{vie!publique}
  \begin{wrapfigure}{r}{2.5cm}
    \centering
    \qrcode[nolink]{https://data.gouv.fr/dataset/53698e5ca3a729239d203440}
  \end{wrapfigure}

Licence : \textbf{Licence Ouverte
}\newline
Créé le : 2013-07-08\newline
Modifié le : 2016-02-25\newline
Granularité : au pays\newline
Popularité : 1 réutilisation,  0 suivi\newline
Mots-clé : \emph{acteurs, acteurs-administratifs, acteurs-institutionnels, dila, vie-publique
}\newline
Permalien : \url{https://data.gouv.fr/dataset/53698e5ca3a729239d203440}\newline

\par
\noindent
    Répertoire des principaux acteurs de la vie publique dans la sphère
administrative, institutionnelle, politique ou sociale. La présentation
des acteurs administratifs et institutionnels concerne principalement
les acteurs dont il est question sur le site vie-publique.fr. Seuls sont
mentionnés les partis français ayant des élus à l'Assemblée nationale,
au Sénat ou au Parlement européen. Les organismes sociaux répertoriés
sont ceux qui représentent les principales activités économiques et
sociales au Conseil économique et social.


\vspace{0.5cm}
\needspace{12\baselineskip}
\subsection*{Apprentissage et enseignement professionnel jusqu'en 2004 - collection
politiques publiques - vie-publique.fr
}\index{apprentissage}\index{apprentissage!professionnel}\index{dila}\index{diplomes}\index{enseignement!professionnel}\index{formation!en!alternance}\index{vie!publiquefr}
  \begin{wrapfigure}{r}{2.5cm}
    \centering
    \qrcode[nolink]{https://data.gouv.fr/dataset/536c3cdfa3a72933d8d1b391}
  \end{wrapfigure}

Licence : \textbf{Licence Ouverte
}\newline
Créé le : 2013-07-08\newline
Modifié le : 2016-02-25\newline
Granularité : au pays\newline
Mise à jour : ponctuelle\newline
Popularité : 2 réutilisations,  0 suivi\newline
Mots-clé : \emph{apprentissage, apprentissage-professionnel, dila, diplomes, enseignement-professionnel, formation-en-alternance, vie-publiquefr
}\newline
Permalien : \url{https://data.gouv.fr/dataset/536c3cdfa3a72933d8d1b391}\newline

\par
\noindent
    La relance des formations en alternance : Chronologie ; Rénovation des
diplômes : acteurs et objectifs ; La filière scolaire de l'enseignement
professionnel ; L'apprentissage ; Chiffres clés ; Bibliographie ;
Glossaire


\vspace{0.5cm}
\needspace{12\baselineskip}
\subsection*{Base de données des obligations d'information pesant sur les entreprises
}
  \begin{wrapfigure}{r}{2.5cm}
    \centering
    \qrcode[nolink]{https://data.gouv.fr/dataset/53698f4ea3a729239d2036ee}
  \end{wrapfigure}

Licence : \textbf{Licence Ouverte
}\newline
Créé le : 2013-07-08\newline
Modifié le : 2016-03-02\newline
De 2007-09-01 à 2012-07-31\newline
Mise à jour : annuelle\newline
Popularité : 3 réutilisations,  1 suivi\newline
Mots-clé : \emph{aucun
}\newline
Permalien : \url{https://data.gouv.fr/dataset/53698f4ea3a729239d2036ee}\newline

\par
\noindent
    La base de données des obligations d'information pesant sur les
entreprises concerne l'ensemble des informations que les entreprises
doivent communiquer à une autorité publique ou à des tiers. Ce
recensement des obligations qui s'imposent aux entreprises du fait de la
réglementation provient d'une analyse des textes législatifs et
réglementaires (décrets) depuis 2007. Ce capital de données publiques
compte à ce jour plus de 12000 obligations et restitue pour chacune
d'entre elles des éléments d'identité (libellé, références
juridiques,\ldots{}), de statut (date de création par ex.) et des
informations générales qualitatives (type, thème, phase du cycle de vie
de l'entreprise, code NAF pour une part d'entre elles).


\vspace{0.5cm}
\needspace{12\baselineskip}
\subsection*{Brèves d'actualité de vie-publique.fr
}\index{actualites}\index{dila}\index{document!public}\index{entreprises}\index{europe}\index{politique!publique}\index{rapport!public}\index{union!europeenne}\index{vie!publiquefr}
  \begin{wrapfigure}{r}{2.5cm}
    \centering
    \qrcode[nolink]{https://data.gouv.fr/dataset/53698fb6a3a729239d2037fc}
  \end{wrapfigure}

Licence : \textbf{Licence Ouverte
}\newline
Créé le : 2013-07-08\newline
Modifié le : 2017-08-10\newline
Granularité : au pays\newline
Mise à jour : quotienne\newline
Popularité : 1 réutilisation,  0 suivi\newline
Mots-clé : \emph{actualites, dila, document-public, entreprises, europe, politique-publique, rapport-public, union-europeenne, vie-publiquefr
}\newline
Permalien : \url{https://data.gouv.fr/dataset/53698fb6a3a729239d2037fc}\newline

\par
\noindent
    Les brèves d'actualité de vie-publique.fr assurent un traitement réactif
de l'actualité de la vie publique. L'objectif est de présenter,
expliquer, élargir un sujet abordé dans un document public récent
(rapport, étude ou enquête) faisant un bilan et proposant des
recommandations. Parmi les brèves hebdomadaires, l'une d'entre elles
traite de l'actualité de l'Union européenne.


\vspace{0.5cm}
\needspace{12\baselineskip}
\subsection*{CASS
}\index{cass}\index{cassation}\index{cour!de!cassation}\index{cours!administratives}\index{dila}\index{jurisprudence}
  \begin{wrapfigure}{r}{2.5cm}
    \centering
    \qrcode[nolink]{https://data.gouv.fr/dataset/55f134e188ee385f7fa46ec1}
  \end{wrapfigure}

Licence : \textbf{Licence Ouverte
}\newline
Créé le : 2015-09-10\newline
Modifié le : 2018-06-21\newline
Granularité : au département\newline
Mise à jour : quotienne\newline
Popularité : 6 réutilisations,  3 suivis\newline
Mots-clé : \emph{cass, cassation, cour-de-cassation, cours-administratives, dila, jurisprudence
}\newline
Permalien : \url{https://data.gouv.fr/dataset/55f134e188ee385f7fa46ec1}\newline

\par
\noindent
    Les grands arrêts de la jurisprudence judiciaire ; les arrêts de la Cour
de cassation :

\begin{itemize}
\item
  publiés au Bulletin des chambres civiles depuis 1960,
\item
  publiés au Bulletin de la chambre criminelle depuis 1963.
\end{itemize}

Texte intégral des arrêts complété par des titrages et des sommaires
rédigés par les magistrats de la Cour de Cassation.

\textbf{La mise à disposition par la DILA de jeux de données pouvant
contenir des données personnelles n'affranchit pas le réutilisateur du
respect de la loi Informatique et Libertés, conformément au document}
\href{https://echanges.dila.gouv.fr/OPENDATA/AVERTISSEMENT-Donnees_a_caractere_personnel.pdf}{AVERTISSEMENT}

\href{ftp://echanges.dila.gouv.fr/CASS/}{Pour accéder au répertoire des
données CASS sous le protocole ftp cliquer ici}

\textbf{Référentiel de DTD : DTD LEGIFRANCE}

Les bases LEGI, KALI, JORF, CAPP, CASS, INCA, JADE, CNIL et CONSTIT ont
des DTD génériques en commun et des DTD spécifiques. La modification
d'une DTD générique peut donc impacter différentes bases. Pour
simplifier la gestion des mises à jour des DTD de ces bases et faciliter
la prise en compte des impacts à chaque changement d'une des DTD, La
DILA met à disposition un référentiel unique des DTD de toutes ces
bases. Ce référentiel, nommé DTD LEGIFRANCE contient un dossier
DTD\_Legifrance. Ce dossier comporte :

\begin{itemize}
\item
  L'ensemble des DTD génériques et spécifiques des bases juridiques
  (hors CIRCULAIRES) applicables à la date indiquée dans le nom du
  dossier ;
\item
  Un document technique générique pour l'ensemble des bases ;
\item
  Un tableau LEGIFRANCE\_20170711\_dtd\_map.xlsx qui présente une vue
  des DTD nécessaires par base et une vue impact qui indique par DTD,
  les impacts sur les différentes bases.
\end{itemize}

A chaque changement d'une DTD, un nouveau dossier est publié dans le
répertoire en remplacement de l'ancien ; il porte la date de mise à jour
du référentiel. Une indication des DTD modifiées sera également fournie.

\href{ftp://echanges.dila.gouv.fr/DTD_LEGIFRANCE/}{Pour accéder au
répertoire DTD sous le protocole ftp cliquer ici}

Vous avez la possibilité d'être informé des éventuelles modifications à
postériori des arrêts déjà diffusés, en demandant votre inscription à la
liste de diffusion via ce
\href{http://rip.journal-officiel.gouv.fr/index.php/pages/Formulaire-de-contact}{Formulaire
de contact.}

Vous pouvez nous écrire ou vous abonner à une alerte par mail adressé à
: \textbf{donnees-dila@dila.gouv.fr}


\vspace{0.5cm}
\needspace{12\baselineskip}
\subsection*{Chronique de la politique d'immigration en 2006 - collection politiques
publiques
}\index{dila}\index{immigration}\index{integration}\index{politique!d!immigration}\index{vie!publiquefr}
  \begin{wrapfigure}{r}{2.5cm}
    \centering
    \qrcode[nolink]{https://data.gouv.fr/dataset/536990aea3a729239d203a7b}
  \end{wrapfigure}

Licence : \textbf{Licence Ouverte
}\newline
Créé le : 2013-07-08\newline
Modifié le : 2016-02-25\newline
Mise à jour : ponctuelle\newline
Popularité : 1 réutilisation,  0 suivi\newline
Mots-clé : \emph{dila, immigration, integration, politique-d-immigration, vie-publiquefr
}\newline
Permalien : \url{https://data.gouv.fr/dataset/536990aea3a729239d203a7b}\newline

\par
\noindent
    Rupture, transition ou continuité ? ; La maîtrise des flux migratoires :
la preuve par les chiffres ? ; Un arsenal législatif et réglementaire
renforcé, au service de la maîtrise des flux migratoires ; Limiter
l'immigration subie\ldots{} promouvoir l'immigration choisie ; Flux
migratoires et codéveloppement ; Bibliographie


\vspace{0.5cm}
\needspace{12\baselineskip}
\subsection*{Chronologies de la vie publique - vie-publique.fr
}\index{administration}\index{agriculture}\index{amenagement!du!territoire}\index{budget}\index{budget!de!l!etat}\index{chronologie}\index{citoyennete}\index{collectivite!territoriale}\index{cooperation}\index{culture}\index{decentralisation}\index{defense}\index{developpement!durable}\index{dila}\index{droits!fondamentaux}\index{economie}\index{elections!europeennes}\index{elections!regionales}\index{emploi}\index{energie}\index{enseignement}\index{ethique}\index{famille}\index{femmes}\index{finances!locales}\index{fonction!publique}\index{formation!professionnelle}\index{handicapes}\index{immigration}\index{industrie}\index{institutions}\index{jeunesse}\index{justice}\index{logement}\index{media}\index{medias}\index{nouvelles!technologies}\index{politiques!communes}\index{politiques!europeennes}\index{protection!sociale}\index{recherche}\index{relations!exterieures}\index{retraites}\index{sante}\index{securite}\index{societe}\index{sports!et!loisirs}\index{tra}\index{transport}\index{travail}\index{union!europeenne}\index{vie!publique}\index{vie!publiquefr}\index{ville}
  \begin{wrapfigure}{r}{2.5cm}
    \centering
    \qrcode[nolink]{https://data.gouv.fr/dataset/536990afa3a729239d203a7e}
  \end{wrapfigure}

Licence : \textbf{Licence Ouverte
}\newline
Créé le : 2014-02-21\newline
Modifié le : 2018-06-21\newline
Granularité : au pays\newline
Mise à jour : hebdomadaire\newline
Popularité : 1 réutilisation,  3 suivis\newline
Mots-clé : \emph{administration, agriculture, amenagement-du-territoire, budget, budget-de-l-etat, chronologie, citoyennete, collectivite-territoriale, cooperation, culture, decentralisation, defense, developpement-durable, dila, droits-fondamentaux, economie, elections-europeennes, elections-regionales, emploi, energie, enseignement, ethique, famille, femmes, finances-locales, fonction-publique, formation-professionnelle, handicapes, immigration, industrie, institutions, jeunesse, justice, logement, media, medias, nouvelles-technologies, politiques-communes, politiques-europeennes, protection-sociale, recherche, relations-exterieures, retraites, sante, securite, societe, sports-et-loisirs, tra, transport, travail, union-europeenne, vie-publique, vie-publiquefr, ville
}\newline
Permalien : \url{https://data.gouv.fr/dataset/536990afa3a729239d203a7e}\newline

\par
\noindent
    Les dates, événements clés les faits marquants qui jalonnent la vie
publique en France. Les chronologies, regroupées par année depuis l'an
2000, sont présentées mois par mois.


\vspace{0.5cm}
\needspace{12\baselineskip}
\subsection*{CIRCULAIRES : Instructions et circulaires des ministères
}\index{circulaires}\index{dila}\index{instruction}\index{ministere}\index{ministre}
  \begin{wrapfigure}{r}{2.5cm}
    \centering
    \qrcode[nolink]{https://data.gouv.fr/dataset/53ba489ca3a729219b7beacf}
  \end{wrapfigure}

Licence : \textbf{Licence Ouverte
}\newline
Créé le : 2014-06-25\newline
Modifié le : 2017-09-04\newline
Mise à jour : ponctuelle\newline
Popularité : 4 réutilisations,  2 suivis\newline
Mots-clé : \emph{circulaires, dila, instruction, ministere, ministre
}\newline
Permalien : \url{https://data.gouv.fr/dataset/53ba489ca3a729219b7beacf}\newline

\par
\noindent
    Il s'agit des instructions et circulaires applicables, adressées par les
ministres aux services et établissements de l'Etat (décret n\degree{}
2008-1281 du 8 décembre 2008 relatif aux conditions de publication des
instructions et circulaires).

\textbf{L'attention du « Réutilisateur » est appelée sur le cadre
juridique particulier dans lequel s'inscrit la réutilisation des données
juridiques :}

\href{http://www.legifrance.gouv.fr/affichTexte.do;jsessionid=6CF268E473632E1932F55A719251387B.tpdjo12v_2\&dateTexte=?cidTexte=JORFTEXT000000413818\&categorieLien=cid}{Décret
n\degree{} 2002-1064 du 7 août 2002 relatif au service public de la
diffusion du droit par l'internet}

\href{ftp://echanges.dila.gouv.fr/CIRCULAIRES/}{Pour accéder au
répertoire des données CIRCULAIRES sous le protocole ftp cliquer ici}

Vous pouvez nous écrire ou vous abonner à une alerte par mail adressé à
: \textbf{donnees-dila@dila.gouv.fr}


\vspace{0.5cm}
\needspace{12\baselineskip}
\subsection*{Composition des gouvernements de la Vème République (1959 - 2014)
}
  \begin{wrapfigure}{r}{2.5cm}
    \centering
    \qrcode[nolink]{https://data.gouv.fr/dataset/5369912ca3a729239d203bb9}
  \end{wrapfigure}

Licence : \textbf{Licence Ouverte
}\newline
Créé le : 2013-07-08\newline
Modifié le : 2016-03-16\newline
De 1959-01-08 à 2014-12-31\newline
Mise à jour : ponctuelle\newline
Popularité : 2 réutilisations,  5 suivis\newline
Mots-clé : \emph{aucun
}\newline
Permalien : \url{https://data.gouv.fr/dataset/5369912ca3a729239d203bb9}\newline

\par
\noindent
    Tableaux présentant la composition des gouvernements de la cinquième
République. Liste des Premiers ministres, des ministres et des
secrétaires d'Etat. Dates de nomination et de fin de fonction.


\vspace{0.5cm}
\needspace{12\baselineskip}
\subsection*{COMPTES ASSOCIATIONS
}\index{associations}\index{dila}\index{donnees!comptables}\index{donnees!economiques}\index{fondations}\index{fonds!de!dotation}\index{joafe}
  \begin{wrapfigure}{r}{2.5cm}
    \centering
    \qrcode[nolink]{https://data.gouv.fr/dataset/53ca2e62a3a7294a1ddd784b}
  \end{wrapfigure}

Licence : \textbf{Licence Ouverte
}\newline
Créé le : 2014-07-18\newline
Modifié le : 2018-06-21\newline
Granularité : à la commune\newline
Mise à jour : annuelle\newline
Popularité : 2 réutilisations,  8 suivis\newline
Mots-clé : \emph{associations, dila, donnees-comptables, donnees-economiques, fondations, fonds-de-dotation, joafe
}\newline
Permalien : \url{https://data.gouv.fr/dataset/53ca2e62a3a7294a1ddd784b}\newline

\par
\noindent
    Les comptes annuels des associations, fondations et fonds de dotation.
Vous trouverez les comptes des années 1998, 2003 à 2015 ainsi que le
flux du dernier mois (année 2016).

\href{ftp://echanges.dila.gouv.fr/ASSOCIATIONS/}{Pour accéder au
répertoire des données COMPTES ASSOCIATIONS sous le protocole ftp
cliquer ici}

\textbf{La mise à disposition par la DILA de jeux de données pouvant
contenir des données personnelles n'affranchit pas le réutilisateur du
respect de la loi Informatique et Libertés, conformément au document}
\href{https://echanges.dila.gouv.fr/OPENDATA/AVERTISSEMENT-Donnees_a_caractere_personnel.pdf}{AVERTISSEMENT}.

\textbf{Le réutilisateur s'oblige à prendre en compte les demandes de
mise à jour de données publiées ponctuellement par la DILA dans le forum
de discussion du jeu de données sur data.gouv.fr}.

Vous pouvez nous écrire ou vous abonner à une alerte par mail adressé à
: \textbf{donnees-dila@dila.gouv.fr}


\vspace{0.5cm}
\needspace{12\baselineskip}
\subsection*{DebatesCore, norme de description des débats publics - vie-publique.fr
}\index{debat}\index{debats!publics}\index{dila}\index{dublin!core}\index{norme}\index{rdf}\index{vie!publiquefr}\index{web!semantique}\index{xml}
  \begin{wrapfigure}{r}{2.5cm}
    \centering
    \qrcode[nolink]{https://data.gouv.fr/dataset/5465f7fec751df09436c8d07}
  \end{wrapfigure}

Licence : \textbf{Licence Ouverte
}\newline
Créé le : 2014-11-14\newline
Modifié le : 2016-02-26\newline
Mise à jour : ponctuelle\newline
Popularité : 1 réutilisation,  4 suivis\newline
Mots-clé : \emph{debat, debats-publics, dila, dublin-core, norme, rdf, vie-publiquefr, web-semantique, xml
}\newline
Permalien : \url{https://data.gouv.fr/dataset/5465f7fec751df09436c8d07}\newline

\par
\noindent
    1ère norme internationale de description en web sémantique des débats
publics. Norme fondée sur le modèle DCAP du Dublin Core. Norme éditée
par la Direction de l'information légale et administrative et la
Commission nationale du débat public. La norme ``DebatesCore'' vise à
établir un standard de description des débats publics. A l'initiative de
la Commission nationale du débat public et de vie-publique.fr, les
différents partenaires ont défini un cadre et six objectifs au projet
``DebatesCore'' : Est considéré comme débat public : ``Tout dispositif
en ligne (à minima d'information) permettant au public et aux parties
prenantes de contribuer/participer à l'élaboration d'une décision
publique ou d'une politique publique'' Les objectifs assignés sont les
suivants : 1. assurer la mission de recensement exhaustif de tous les
débats en France (recensement automatisé ) 2. développer un meilleur
partage et une mutualisation de l'information entre acteurs du débat
public 3. améliorer la qualité d'information du citoyen pour une
meilleure participation à la concertation 4. rationaliser les moyens et
les coûts pour chaque partenaire dans ses développements et faciliter le
développement des outils de la concertation privés et publics 5. montrer
l'intérêt du web sémantique sur une communauté spécifique : le débat
public 6. Rentrer concrètement dans l'open data des données afférentes
aux débats publics


\vspace{0.5cm}
\needspace{12\baselineskip}
\subsection*{Déclaration en ligne de changement de coordonnées : statistiques
2005-2013
}\index{administration!electronique}\index{adresse}\index{agirc!arrco}\index{assurance!maladie}\index{caf}\index{caisse!nationale!des!industries}\index{caisse!nationale!militaire!de!se}\index{changement!d!adresse!en!ligne}\index{cnav}\index{coordonnees}\index{demenagement}\index{dgfip}\index{dila}\index{direct!energie}\index{edf}\index{gdf}\index{impots}\index{la!poste}\index{msa}\index{pole!emploi}\index{service!national}\index{systeme!d!immatriculation!des!ve}
  \begin{wrapfigure}{r}{2.5cm}
    \centering
    \qrcode[nolink]{https://data.gouv.fr/dataset/54292a8e88ee380327a5915b}
  \end{wrapfigure}

Licence : \textbf{Licence Ouverte
}\newline
Créé le : 2014-09-29\newline
Modifié le : 2015-11-23\newline
De 2005-05-01 à 2013-12-31\newline
Mise à jour : annuelle\newline
Popularité : 1 réutilisation,  1 suivi\newline
Mots-clé : \emph{administration-electronique, adresse, agirc-arrco, assurance-maladie, caf, caisse-nationale-des-industries, caisse-nationale-militaire-de-se, changement-d-adresse-en-ligne, cnav, coordonnees, demenagement, dgfip, dila, direct-energie, edf, gdf, impots, la-poste, msa, pole-emploi, service-national, systeme-d-immatriculation-des-ve
}\newline
Permalien : \url{https://data.gouv.fr/dataset/54292a8e88ee380327a5915b}\newline

\par
\noindent
    ce fichier contient les statistiques des déclarations de changement de
coordonnées effectuées par les usagers sur le téléservice
changement-adresse.gouv.fr (de 2005 à début 2010) puis sur
mon.service-public.fr (à partir de mars 2010) et transmises aux
organismes publics ou privés connectés au téléservice et sélectionnés
par l'usager.


\vspace{0.5cm}
\needspace{12\baselineskip}
\subsection*{Dossiers d'actualité de vie-publique.fr
}\index{actualite!de!la!vie!publique}\index{actualites}\index{budget}\index{dila}\index{elections}\index{vie!publiquefr}
  \begin{wrapfigure}{r}{2.5cm}
    \centering
    \qrcode[nolink]{https://data.gouv.fr/dataset/5369933ba3a729239d204137}
  \end{wrapfigure}

Licence : \textbf{Licence Ouverte
}\newline
Créé le : 2013-07-08\newline
Modifié le : 2016-03-15\newline
Granularité : au pays\newline
Mise à jour : bi-mesnsuelle\newline
Popularité : 1 réutilisation,  0 suivi\newline
Mots-clé : \emph{actualite-de-la-vie-publique, actualites, budget, dila, elections, vie-publiquefr
}\newline
Permalien : \url{https://data.gouv.fr/dataset/5369933ba3a729239d204137}\newline

\par
\noindent
    Les dossiers d'actualité de vie-publique.fr donnent des repères et offre
une analyse complète d'un sujet d'actualité faisant débat (projets
gouvernementaux, dispositifs en vigueur, évolutions de la société ou des
institutions\ldots{}). Ils mettent en perspective le thème en question
et valorisent les ressources publiques en ligne, y compris celles du
portail vie-publique.fr.


\vspace{0.5cm}
\needspace{12\baselineskip}
\subsection*{Gestion de la dette et crise financière - collection politiques
publiques - vie-publique.fr
}\index{dette!publique}\index{dettes}\index{dila}\index{endettement}\index{gestion!de!la!dette}\index{vie!publiquefr}
  \begin{wrapfigure}{r}{2.5cm}
    \centering
    \qrcode[nolink]{https://data.gouv.fr/dataset/536995f9a3a729239d20488e}
  \end{wrapfigure}

Licence : \textbf{Licence Ouverte
}\newline
Créé le : 2013-07-08\newline
Modifié le : 2016-03-13\newline
Granularité : au pays\newline
Mise à jour : ponctuelle\newline
Popularité : 1 réutilisation,  1 suivi\newline
Mots-clé : \emph{dette-publique, dettes, dila, endettement, gestion-de-la-dette, vie-publiquefr
}\newline
Permalien : \url{https://data.gouv.fr/dataset/536995f9a3a729239d20488e}\newline

\par
\noindent
    La France dans le contexte européen de la crise de la dette souveraine ;
Chronologie ; La dette publique en débat ; Les objectifs et les
instruments de gestion de la dette


\vspace{0.5cm}
\needspace{12\baselineskip}
\subsection*{INCA
}\index{cour!de!cassation}\index{cours!administratives}\index{dila}\index{inca}\index{jurisprudence}
  \begin{wrapfigure}{r}{2.5cm}
    \centering
    \qrcode[nolink]{https://data.gouv.fr/dataset/55f1373b88ee385f7fa46ec2}
  \end{wrapfigure}

Licence : \textbf{Licence Ouverte
}\newline
Créé le : 2015-09-10\newline
Modifié le : 2018-06-21\newline
Granularité : au département\newline
Mise à jour : quotienne\newline
Popularité : 4 réutilisations,  2 suivis\newline
Mots-clé : \emph{cour-de-cassation, cours-administratives, dila, inca, jurisprudence
}\newline
Permalien : \url{https://data.gouv.fr/dataset/55f1373b88ee385f7fa46ec2}\newline

\par
\noindent
    Les arrêts inédits (non publiés au Bulletin) diffusés par le fonds de
concours de la Cour de cassation depuis 1989.

\textbf{La mise à disposition par la DILA de jeux de données pouvant
contenir des données personnelles n'affranchit pas le réutilisateur du
respect de la loi Informatique et Libertés, conformément au document}
\href{https://echanges.dila.gouv.fr/OPENDATA/AVERTISSEMENT-Donnees_a_caractere_personnel.pdf}{AVERTISSEMENT}

\href{ftp://echanges.dila.gouv.fr/INCA/}{Pour accéder au répertoire des
données INCA sous le protocole ftp cliquer ici}

\textbf{Référentiel de DTD : DTD LEGIFRANCE}

Les bases LEGI, KALI, JORF, CAPP, CASS, INCA, JADE, CNIL et CONSTIT ont
des DTD génériques en commun et des DTD spécifiques. La modification
d'une DTD générique peut donc impacter différentes bases. Pour
simplifier la gestion des mises à jour des DTD de ces bases et faciliter
la prise en compte des impacts à chaque changement d'une des DTD, La
DILA met à disposition un référentiel unique des DTD de toutes ces
bases. Ce référentiel, nommé DTD LEGIFRANCE contient un dossier
DTD\_Legifrance. Ce dossier comporte :

\begin{itemize}
\item
  L'ensemble des DTD génériques et spécifiques des bases juridiques
  (hors CIRCULAIRES) applicables à la date indiquée dans le nom du
  dossier ;
\item
  Un document technique générique pour l'ensemble des bases ;
\item
  Un tableau LEGIFRANCE\_20170711\_dtd\_map.xlsx qui présente une vue
  des DTD nécessaires par base et une vue impact qui indique par DTD,
  les impacts sur les différentes bases.
\end{itemize}

A chaque changement d'une DTD, un nouveau dossier est publié dans le
répertoire en remplacement de l'ancien ; il porte la date de mise à jour
du référentiel. Une indication des DTD modifiées sera également fournie.

\href{ftp://echanges.dila.gouv.fr/DTD_LEGIFRANCE/}{Pour accéder au
répertoire DTD sous le protocole ftp cliquer ici}

Vous avez la possibilité d'être informé des éventuelles modifications à
postériori des arrêts déjà diffusés, en demandant votre inscription à la
liste de diffusion via ce
\href{http://rip.journal-officiel.gouv.fr/index.php/pages/Formulaire-de-contact}{Formulaire
de contact.}

Vous pouvez nous écrire ou vous abonner à une alerte par mail adressé à
: \textbf{donnees-dila@dila.gouv.fr}


\vspace{0.5cm}
\needspace{12\baselineskip}
\subsection*{Jeunes et justice (1945-2005) - collection politiques publiques -
vie-publique.fr
}\index{delinquance}\index{delinquance!juvenile}\index{delinquants!precoces}\index{dila}\index{enfance}\index{enfants}\index{europe}\index{jeunes}\index{justice}\index{ordonnances}\index{vie!publiquefr}
  \begin{wrapfigure}{r}{2.5cm}
    \centering
    \qrcode[nolink]{https://data.gouv.fr/dataset/5369975ea3a729239d204c94}
  \end{wrapfigure}

Licence : \textbf{Licence Ouverte
}\newline
Créé le : 2013-07-08\newline
Modifié le : 2016-02-26\newline
Granularité : au pays\newline
Mise à jour : ponctuelle\newline
Popularité : 1 réutilisation,  0 suivi\newline
Mots-clé : \emph{delinquance, delinquance-juvenile, delinquants-precoces, dila, enfance, enfants, europe, jeunes, justice, ordonnances, vie-publiquefr
}\newline
Permalien : \url{https://data.gouv.fr/dataset/5369975ea3a729239d204c94}\newline

\par
\noindent
    Permanences et évolutions de l'ordonnance de 1945 sur l'enfance
délinquante : Les jeunes dans la société ; La justice des mineurs ; Les
réponses à la délinquance des mineurs à partir de 2002 ; Jeunes et
Justice en Europe; Chiffres-clés ; Textes de référence ; Bibliographie ;
Sites de référence


\vspace{0.5cm}
\needspace{12\baselineskip}
\subsection*{Juges et justice de proximité (1980-2006) - collection politiques
publiques - vie-publique.fr
}\index{dila}\index{juge}\index{justice}\index{justice!de!proximite}\index{vie!publiquefr}
  \begin{wrapfigure}{r}{2.5cm}
    \centering
    \qrcode[nolink]{https://data.gouv.fr/dataset/53699791a3a729239d204d18}
  \end{wrapfigure}

Licence : \textbf{Licence Ouverte
}\newline
Créé le : 2013-07-08\newline
Modifié le : 2016-02-26\newline
Granularité : au pays\newline
Mise à jour : ponctuelle\newline
Popularité : 1 réutilisation,  0 suivi\newline
Mots-clé : \emph{dila, juge, justice, justice-de-proximite, vie-publiquefr
}\newline
Permalien : \url{https://data.gouv.fr/dataset/53699791a3a729239d204d18}\newline

\par
\noindent
    Tableau de la justice de proximité en 2006 : Chronologie ; Les juges de
proximité ; Accès au droit et réseau judiciaire de proximité ; Les modes
alternatifs de règlement des conflits ; Glossaire ; Sélection de sites ;
Bibliographie


\vspace{0.5cm}
\needspace{12\baselineskip}
\subsection*{KALI : Conventions collectives nationales
}\index{accord!collectif}\index{accords!collectifs}\index{bulletin!officiel}\index{convention!collective}\index{dila}\index{droit!du!travail}\index{salaires}\index{syndicats}
  \begin{wrapfigure}{r}{2.5cm}
    \centering
    \qrcode[nolink]{https://data.gouv.fr/dataset/53ba5033a3a729219b7bead9}
  \end{wrapfigure}

Licence : \textbf{Licence Ouverte
}\newline
Créé le : 2014-06-24\newline
Modifié le : 2017-09-04\newline
Granularité : à la région\newline
Mise à jour : quotienne\newline
Popularité : 9 réutilisations,  7 suivis\newline
Mots-clé : \emph{accord-collectif, accords-collectifs, bulletin-officiel, convention-collective, dila, droit-du-travail, salaires, syndicats
}\newline
Permalien : \url{https://data.gouv.fr/dataset/53ba5033a3a729219b7bead9}\newline

\par
\noindent
    Toutes les conventions collectives et textes associés. La base donne
également accès à certaines conventions collectives nationales non
étendues ainsi qu'à des conventions collectives régionales,
départementales étendues ou non. Les textes associés comprennent les
accords se rattachant à une convention collective, les salaires et les
arrêtés d'extension. Les données sont mises à jour à partir du Bulletin
officiel « Conventions collectives » édité sous la responsabilité du
Ministère du travail, de la solidarité et de la fonction publique et
diffusé par la DILA.

\href{ftp://echanges.dila.gouv.fr/KALI/}{Pour accéder au répertoire des
données KALI sous le protocole ftp cliquer ici}

\textbf{Référentiel de DTD : DTD LEGIFRANCE}

Les bases LEGI, KALI, JORF, CAPP, CASS, INCA, JADE, CNIL et CONSTIT ont
des DTD génériques en commun et des DTD spécifiques. La modification
d'une DTD générique peut donc impacter différentes bases. Pour
simplifier la gestion des mises à jour des DTD de ces bases et faciliter
la prise en compte des impacts à chaque changement d'une des DTD, La
DILA met à disposition un référentiel unique des DTD de toutes ces
bases. Ce référentiel, nommé DTD LEGIFRANCE contient un dossier
DTD\_Legifrance. Ce dossier comporte :

-L'ensemble des DTD génériques et spécifiques des bases juridiques (hors
CIRCULAIRES) applicables à la date indiquée dans le nom du dossier ;

-Un document technique générique pour l'ensemble des bases ;

-Un tableau LEGIFRANCE\_20170711\_dtd\_map.xlsx qui présente une vue des
DTD nécessaires par base et une vue impact qui indique par DTD, les
impacts sur les différentes bases.

A chaque changement d'une DTD, un nouveau dossier est publié dans le
répertoire en remplacement de l'ancien ; il porte la date de mise à jour
du référentiel. Une indication des DTD modifiées sera également fournie.

\href{ftp://echanges.dila.gouv.fr/DTD_LEGIFRANCE/}{Pour accéder au
répertoire DTD LEGIFRANCE sous le protocole ftp cliquer ici}

Vous pouvez nous écrire ou vous abonner à une alerte par mail adressé à
: \textbf{donnees-dila@dila.gouv.fr}


\vspace{0.5cm}
\needspace{12\baselineskip}
\subsection*{La condition étudiante (1960 - 2008) - collection politiques publiques -
vie-publique.fr
}\index{dila}\index{enseignement!superieur}\index{etudes}\index{etudiants}\index{systeme!universitaire}\index{universites}\index{vie!publiquefr}
  \begin{wrapfigure}{r}{2.5cm}
    \centering
    \qrcode[nolink]{https://data.gouv.fr/dataset/53699798a3a729239d204d2c}
  \end{wrapfigure}

Licence : \textbf{Licence Ouverte
}\newline
Créé le : 2013-07-08\newline
Modifié le : 2016-02-26\newline
Granularité : au pays\newline
Mise à jour : ponctuelle\newline
Popularité : 1 réutilisation,  0 suivi\newline
Mots-clé : \emph{dila, enseignement-superieur, etudes, etudiants, systeme-universitaire, universites, vie-publiquefr
}\newline
Permalien : \url{https://data.gouv.fr/dataset/53699798a3a729239d204d2c}\newline

\par
\noindent
    Une nouvelle composante de la politique de l'enseignement supérieur :
Chronologie ; Accès aux études supérieures et égalité des chances ; Les
conditions de vie des étudiants ; Les étudiants, acteurs du système
universitaire ; Sélection de sites ; Glossaire


\vspace{0.5cm}
\needspace{12\baselineskip}
\subsection*{La décentralisation - collection politiques publiques - vie-publique.fr
}\index{action!publique}\index{collectivites!locales}\index{collectivites!territoriales}\index{decentralisation}\index{dila}
  \begin{wrapfigure}{r}{2.5cm}
    \centering
    \qrcode[nolink]{https://data.gouv.fr/dataset/5369979aa3a729239d204d30}
  \end{wrapfigure}

Licence : \textbf{Licence Ouverte
}\newline
Créé le : 2013-07-08\newline
Modifié le : 2016-02-26\newline
Granularité : à la région\newline
Mise à jour : ponctuelle\newline
Popularité : 1 réutilisation,  0 suivi\newline
Mots-clé : \emph{action-publique, collectivites-locales, collectivites-territoriales, decentralisation, dila
}\newline
Permalien : \url{https://data.gouv.fr/dataset/5369979aa3a729239d204d30}\newline

\par
\noindent
    Si elle concerne au premier chef les collectivités locales, la politique
de décentralisation peut aussi s'analyser comme une politique publique
impulsée et conduite par le sommet de l'Etat : ``en matière de
décentralisation, le pouvoir politique gouverne de façon centralisée''
(Jean-Claude Thoenig, 1992). Les deux grands moments de la
décentralisation se sont concrétisés par l'adoption de lois élaborées
par le gouvernement : ce sont les lois Defferre en 1982-1983, puis la
réforme constitutionnelle en 2003. Ces lois marquent la volonté
politique d'opérer une redistribution des pouvoirs entre l'Etat et les
collectivités locales avec comme objectifs une meilleure efficacité de
l'action publique et le développement d'une démocratie de proximité.


\vspace{0.5cm}
\needspace{12\baselineskip}
\subsection*{La formation professionnelle continue (1971-2009) - collection
politiques publiques - vie-publique.fr
}\index{dila}\index{formation}\index{formation!et!apprentissage}\index{formation!professionnelle}\index{formation!professionnelle!contin}\index{vie!publiquefr}
  \begin{wrapfigure}{r}{2.5cm}
    \centering
    \qrcode[nolink]{https://data.gouv.fr/dataset/536997a1a3a729239d204d42}
  \end{wrapfigure}

Licence : \textbf{Licence Ouverte
}\newline
Créé le : 2013-07-08\newline
Modifié le : 2016-02-26\newline
Granularité : au pays\newline
Mise à jour : ponctuelle\newline
Popularité : 1 réutilisation,  0 suivi\newline
Mots-clé : \emph{dila, formation, formation-et-apprentissage, formation-professionnelle, formation-professionnelle-contin, vie-publiquefr
}\newline
Permalien : \url{https://data.gouv.fr/dataset/536997a1a3a729239d204d42}\newline

\par
\noindent
    La consécration de l'individualisation et de la personnalisation des
parcours de formation : Chronologie ; Acteurs et gouvernance ; Publics
et financements ; De l'objectif de promotion sociale à
l'individualisation de la formation ; Glossaire


\vspace{0.5cm}
\needspace{12\baselineskip}
\subsection*{La gestion de la dette : objectifs et moyens (1985-2007) - collection
politiques publiques - vie-publique.fr
}\index{depenses}\index{depenses!publiques}\index{dette!exterieure}\index{dette!exterieure!publique}\index{dette!publique}\index{dila}\index{vie!publiquefr}
  \begin{wrapfigure}{r}{2.5cm}
    \centering
    \qrcode[nolink]{https://data.gouv.fr/dataset/536997a2a3a729239d204d43}
  \end{wrapfigure}

Licence : \textbf{Licence Ouverte
}\newline
Créé le : 2013-07-08\newline
Modifié le : 2016-02-26\newline
Granularité : au pays\newline
Mise à jour : ponctuelle\newline
Popularité : 1 réutilisation,  0 suivi\newline
Mots-clé : \emph{depenses, depenses-publiques, dette-exterieure, dette-exterieure-publique, dette-publique, dila, vie-publiquefr
}\newline
Permalien : \url{https://data.gouv.fr/dataset/536997a2a3a729239d204d43}\newline

\par
\noindent
    La stabilisation progressive des dépenses de l'État à l'épreuve ;
Chronologie ; La dette publique en débat ; Les objectifs ; Les moyens ;
Éléments de comparaison ; Glossaire ; Bibliographie ; Sites de référence


\vspace{0.5cm}
\needspace{12\baselineskip}
\subsection*{La justice de proximité (1945-2002) - collection politiques publiques -
vie-publique.fr
}\index{acces!a!la!justice}\index{acces!aux!droits}\index{carte!judiciaire}\index{dila}\index{justice}\index{justice!administrative}\index{justice!de!proximite}\index{vie!publiquefr}
  \begin{wrapfigure}{r}{2.5cm}
    \centering
    \qrcode[nolink]{https://data.gouv.fr/dataset/536997a5a3a729239d204d4c}
  \end{wrapfigure}

Licence : \textbf{Licence Ouverte
}\newline
Créé le : 2013-07-08\newline
Modifié le : 2016-02-26\newline
Granularité : au pays\newline
Mise à jour : ponctuelle\newline
Popularité : 1 réutilisation,  0 suivi\newline
Mots-clé : \emph{acces-a-la-justice, acces-aux-droits, carte-judiciaire, dila, justice, justice-administrative, justice-de-proximite, vie-publiquefr
}\newline
Permalien : \url{https://data.gouv.fr/dataset/536997a5a3a729239d204d4c}\newline

\par
\noindent
    Accès au droit, accès à la justice : Chronologie ; L'accessibilité
économique : l'aide juridictionnelle ; Justice et territoire : la carte
judiciaire ; Les réseaux de proximité ; Les modes alternatifs de
règlement des conflits ; La justice administrative : réforme du référé;
Glossaire ; Bibliographie ; Sites Internet


\vspace{0.5cm}
\needspace{12\baselineskip}
\subsection*{La justice pénale (1990-2005) - collection politiques publiques -
vie-publique.fr
}\index{dila}\index{juridiction}\index{justice}\index{justice!penale}\index{reforme!penale}\index{vie!publiquefr}
  \begin{wrapfigure}{r}{2.5cm}
    \centering
    \qrcode[nolink]{https://data.gouv.fr/dataset/536997a6a3a729239d204d4d}
  \end{wrapfigure}

Licence : \textbf{Licence Ouverte
}\newline
Créé le : 2013-07-08\newline
Modifié le : 2016-02-26\newline
Granularité : au pays\newline
Mise à jour : ponctuelle\newline
Popularité : 1 réutilisation,  0 suivi\newline
Mots-clé : \emph{dila, juridiction, justice, justice-penale, reforme-penale, vie-publiquefr
}\newline
Permalien : \url{https://data.gouv.fr/dataset/536997a6a3a729239d204d4d}\newline

\par
\noindent
    Quinze ans de réformes pénales : Chronologie ; Acteurs, juridictions et
interdit pénal ; La procédure pénale ; Chiffres clés ; Glossaire ;
Bibliographie ; Sélection de sites


\vspace{0.5cm}
\needspace{12\baselineskip}
\subsection*{La politique de coopération pour le développement (1958 - 2007) -
collection politiques publiques - vie-publique.fr
}\index{aide!au!developpement}\index{dila}\index{diplomatie}\index{politique!de!cooperation}\index{vie!publiquefr}
  \begin{wrapfigure}{r}{2.5cm}
    \centering
    \qrcode[nolink]{https://data.gouv.fr/dataset/536997aba3a729239d204d5c}
  \end{wrapfigure}

Licence : \textbf{Licence Ouverte
}\newline
Créé le : 2013-07-08\newline
Modifié le : 2016-02-26\newline
Mise à jour : ponctuelle\newline
Popularité : 1 réutilisation,  1 suivi\newline
Mots-clé : \emph{aide-au-developpement, dila, diplomatie, politique-de-cooperation, vie-publiquefr
}\newline
Permalien : \url{https://data.gouv.fr/dataset/536997aba3a729239d204d5c}\newline

\par
\noindent
    Une politique en évolution depuis la réforme de 1998 : Chronologie ; Le
cadre de la politique de coopération ; L'aide publique au développement
; Solidarité à l'égard des PVD : des spécificités françaises.
Bibliographie ; Glossaire ; Sélection de sites


\vspace{0.5cm}
\needspace{12\baselineskip}
\subsection*{La politique de défense 1945-2014 - collection politiques publiques -
vie-publique.fr
}\index{defense}\index{dila}\index{otan}\index{politique!de!defense}\index{securite}\index{strategie}\index{vie!publiquefr}
  \begin{wrapfigure}{r}{2.5cm}
    \centering
    \qrcode[nolink]{https://data.gouv.fr/dataset/536997aba3a729239d204d5e}
  \end{wrapfigure}

Licence : \textbf{Licence Ouverte
}\newline
Créé le : 2013-07-08\newline
Modifié le : 2016-02-26\newline
Granularité : au pays\newline
Mise à jour : ponctuelle\newline
Popularité : 2 réutilisations,  1 suivi\newline
Mots-clé : \emph{defense, dila, otan, politique-de-defense, securite, strategie, vie-publiquefr
}\newline
Permalien : \url{https://data.gouv.fr/dataset/536997aba3a729239d204d5e}\newline

\par
\noindent
    Mise en oeuvre du Livre blanc de 1994 et réintégration dans l'OTAN ; Les
livres blancs de 2008 et 2013 ; Chronologie (1945-2013) ; L'organisation
de la défense nationale ; Les choix stratégiques ; L'effort de défense ;
Les engagements extérieurs ; La réintégration de la France dans l'OTAN ;
Sites de référence ; Glossaire ; Bibliographie


\vspace{0.5cm}
\needspace{12\baselineskip}
\subsection*{La politique de l'audiovisuel - collection politique publique -
vie-publique.fr
}\index{audiovisuel}\index{dila}\index{politique!de!l!audiovisuel}\index{radio}\index{television}\index{vie!publiquefr}
  \begin{wrapfigure}{r}{2.5cm}
    \centering
    \qrcode[nolink]{https://data.gouv.fr/dataset/536997aca3a729239d204d60}
  \end{wrapfigure}

Licence : \textbf{Licence Ouverte
}\newline
Créé le : 2013-07-08\newline
Modifié le : 2016-02-26\newline
Granularité : au pays\newline
Mise à jour : ponctuelle\newline
Popularité : 1 réutilisation,  0 suivi\newline
Mots-clé : \emph{audiovisuel, dila, politique-de-l-audiovisuel, radio, television, vie-publiquefr
}\newline
Permalien : \url{https://data.gouv.fr/dataset/536997aca3a729239d204d60}\newline

\par
\noindent
    Du monopole d'Etat à la régulation ; Chronologie (1850-2010) ; Le
secteur public ; La réforme de l'audiovisuel public de 2009 ;
Réglementation et régulation ; Le financement ; L'audiovisuel extérieur
en 2004 ; La politique audiovisuelle extérieure en 2010 ; La politique
européenne ; Bibliographie ; Sélection de sites


\vspace{0.5cm}
\needspace{12\baselineskip}
\subsection*{La politique de la ville (1970 - 2005) - collection politiques publiques
- vie-publique.fr
}\index{developpement!social}\index{dila}\index{politique!de!la!ville}\index{politique!urbaine}\index{renovation!urbaine}\index{urbanisation}\index{urbanisme}\index{vie!publiquefr}\index{ville}\index{villes}
  \begin{wrapfigure}{r}{2.5cm}
    \centering
    \qrcode[nolink]{https://data.gouv.fr/dataset/536997ada3a729239d204d61}
  \end{wrapfigure}

Licence : \textbf{Licence Ouverte
}\newline
Créé le : 2013-07-08\newline
Modifié le : 2016-02-26\newline
Granularité : au pays\newline
Mise à jour : ponctuelle\newline
Popularité : 1 réutilisation,  0 suivi\newline
Mots-clé : \emph{developpement-social, dila, politique-de-la-ville, politique-urbaine, renovation-urbaine, urbanisation, urbanisme, vie-publiquefr, ville, villes
}\newline
Permalien : \url{https://data.gouv.fr/dataset/536997ada3a729239d204d61}\newline

\par
\noindent
    Chronologie ; Cadre de la politique de la ville ; Rénovation et
renouvellement urbains ; Sécurité et prévention ; Développement social ;
Emploi et revitalisation économique ; Glossaire ; Bibliographie


\vspace{0.5cm}
\needspace{12\baselineskip}
\subsection*{La politique de l'eau (1964-2004) - collection politiques publiques -
vie-publique.fr
}\index{assainissement}\index{dila}\index{eau}\index{politique}\index{politique!de!l!eau}\index{politique!de!l!environnement}\index{services!de!l!eau}\index{vie!publiquefr}
  \begin{wrapfigure}{r}{2.5cm}
    \centering
    \qrcode[nolink]{https://data.gouv.fr/dataset/536997ada3a729239d204d62}
  \end{wrapfigure}

Licence : \textbf{Licence Ouverte
}\newline
Créé le : 2013-07-08\newline
Modifié le : 2017-08-10\newline
Granularité : au pays\newline
Mise à jour : ponctuelle\newline
Popularité : 1 réutilisation,  1 suivi\newline
Mots-clé : \emph{assainissement, dila, eau, politique, politique-de-l-eau, politique-de-l-environnement, services-de-l-eau, vie-publiquefr
}\newline
Permalien : \url{https://data.gouv.fr/dataset/536997ada3a729239d204d62}\newline

\par
\noindent
    40 ans d'une gestion décentralisée : Chronologie ; Les acteurs ; Les
services d'eau et d'assainissement ; L'eau, une ressource à préserver ;
En Europe ; La position de la France au plan international ; Glossaire ;
Liste de sites ; Bibliographie


\vspace{0.5cm}
\needspace{12\baselineskip}
\subsection*{La politique de l'énergie (2003 - 2005) - collection politiques
publiques - vie-publique.fr
}\index{consommation!d!energie}\index{dila}\index{energie}\index{politique!de!l!energie}\index{vie!publiquefr}
  \begin{wrapfigure}{r}{2.5cm}
    \centering
    \qrcode[nolink]{https://data.gouv.fr/dataset/536997aea3a729239d204d63}
  \end{wrapfigure}

Licence : \textbf{Licence Ouverte
}\newline
Créé le : 2013-07-08\newline
Modifié le : 2016-02-26\newline
Granularité : au pays\newline
Mise à jour : ponctuelle\newline
Popularité : 1 réutilisation,  0 suivi\newline
Mots-clé : \emph{consommation-d-energie, dila, energie, politique-de-l-energie, vie-publiquefr
}\newline
Permalien : \url{https://data.gouv.fr/dataset/536997aea3a729239d204d63}\newline

\par
\noindent
    Un débat national en 2003 et l'adoption d'une loi de programme en 2005 ;
Chronologie ; L'État et la politique de l'énergie ; La maîtrise de la
production et de la consommation d'énergie ; La politique de l'énergie :
évaluations et positions ; Chiffres clés ; Bibliographie ; Textes
officiels ; Glossaire ; Sélection de sites


\vspace{0.5cm}
\needspace{12\baselineskip}
\subsection*{La politique d'immigration - collection politiques publiques -
vie-publique.fr
}\index{dila}\index{etrangers}\index{flux!migratoires}\index{immigration}\index{immigration!familiale}\index{immigres}\index{politique!d!immigration}\index{vie!publiquefr}
  \begin{wrapfigure}{r}{2.5cm}
    \centering
    \qrcode[nolink]{https://data.gouv.fr/dataset/536997aea3a729239d204d65}
  \end{wrapfigure}

Licence : \textbf{Licence Ouverte
}\newline
Créé le : 2013-07-08\newline
Modifié le : 2016-02-26\newline
Granularité : au pays\newline
Mise à jour : ponctuelle\newline
Popularité : 1 réutilisation,  0 suivi\newline
Mots-clé : \emph{dila, etrangers, flux-migratoires, immigration, immigration-familiale, immigres, politique-d-immigration, vie-publiquefr
}\newline
Permalien : \url{https://data.gouv.fr/dataset/536997aea3a729239d204d65}\newline

\par
\noindent
    La maîtrise des flux migratoires : Chronologie (1974 - 2015) ;
Législation et réglementation : évolution du statut des étrangers ; Le
contexte européen ; L'acquisition de la nationalité française ; Le droit
de vote ; Les immigrés (étrangers ou français) dans la cité ; Chiffres
clés et sources statistiques en matière d'immigration ; Sélection de
sites ; Bibliographie ; Glossaire


\vspace{0.5cm}
\needspace{12\baselineskip}
\subsection*{La politique du handicap - Collection politiques publiques -
vie-publique.fr
}\index{dila}\index{handicap}\index{handicape!mental}\index{handicapes}\index{handicaps}\index{protection!juridique}\index{vie!publiquefr}
  \begin{wrapfigure}{r}{2.5cm}
    \centering
    \qrcode[nolink]{https://data.gouv.fr/dataset/536997afa3a729239d204d66}
  \end{wrapfigure}

Licence : \textbf{Licence Ouverte
}\newline
Créé le : 2013-07-08\newline
Modifié le : 2016-02-26\newline
Granularité : au pays\newline
Mise à jour : ponctuelle\newline
Popularité : 1 réutilisation,  0 suivi\newline
Mots-clé : \emph{dila, handicap, handicape-mental, handicapes, handicaps, protection-juridique, vie-publiquefr
}\newline
Permalien : \url{https://data.gouv.fr/dataset/536997afa3a729239d204d66}\newline

\par
\noindent
    La politique du handicap : Chronologie; La refondation de la politique
du handicap depuis 2005; Protection juridique des personnes handicapées;
Les acteurs; Les handicapés dans la cité (jusqu'à la loi de 2005); Les
personnes handicapées et le monde du travail; \ldots{} A l'école; La
politique à l'égard des personnes handicapées en Europe; Ethique et
handicap; Chiffres-clés; Glossaire; Bibliographie; Sélection de sites


\vspace{0.5cm}
\needspace{12\baselineskip}
\subsection*{La politique du livre (1979 - 2010) - collection politiques publiques -
vie-publique.fr
}\index{dila}\index{editeur}\index{edition}\index{livre}\index{livres}\index{politique!du!livre}\index{vie!publiquefr}
  \begin{wrapfigure}{r}{2.5cm}
    \centering
    \qrcode[nolink]{https://data.gouv.fr/dataset/536997afa3a729239d204d67}
  \end{wrapfigure}

Licence : \textbf{Licence Ouverte
}\newline
Créé le : 2013-07-08\newline
Modifié le : 2016-02-26\newline
Granularité : au pays\newline
Mise à jour : ponctuelle\newline
Popularité : 1 réutilisation,  0 suivi\newline
Mots-clé : \emph{dila, editeur, edition, livre, livres, politique-du-livre, vie-publiquefr
}\newline
Permalien : \url{https://data.gouv.fr/dataset/536997afa3a729239d204d67}\newline

\par
\noindent
    La politique du livre : enjeux et mutations ; Chronologie ; Les acteurs
publics de la politique du livre ; Le soutien à l'économie du livre ;
L'adaptation du livre aux enjeux numériques ; Sites de référence


\vspace{0.5cm}
\needspace{12\baselineskip}
\subsection*{La politique du médicament - Collection politiques publiques -
vie-publique.fr
}\index{depense!de!sante}\index{depenses!de!sante}\index{dila}\index{medicament}\index{medicaments}\index{sante}\index{sante!publique}\index{securite!sanitaire}\index{vie!publiquefr}
  \begin{wrapfigure}{r}{2.5cm}
    \centering
    \qrcode[nolink]{https://data.gouv.fr/dataset/536997b0a3a729239d204d68}
  \end{wrapfigure}

Licence : \textbf{Licence Ouverte
}\newline
Créé le : 2013-07-08\newline
Modifié le : 2016-02-26\newline
Granularité : au pays\newline
Mise à jour : ponctuelle\newline
Popularité : 1 réutilisation,  1 suivi\newline
Mots-clé : \emph{depense-de-sante, depenses-de-sante, dila, medicament, medicaments, sante, sante-publique, securite-sanitaire, vie-publiquefr
}\newline
Permalien : \url{https://data.gouv.fr/dataset/536997b0a3a729239d204d68}\newline

\par
\noindent
    La politique du médicament a toujours été soucieuse de concilier
diverses exigences touchant à la sécurité sanitaire, à la maitrise des
dépenses de santé mais aussi aux exigences du développement industriel
et de l'innovation ; cette politique, de par sa spécificité, présente un
triple aspect, sanitaire, économique et industriel. La politique du
médicament : Chronologie; Prix et taux de remboursement du médicament :
quelle régulation ? Médicament et maîtrise des dépenses de santé;
Médicament et sécurité sanitaire.


\vspace{0.5cm}
\needspace{12\baselineskip}
\subsection*{La politique du nucléaire civil - Collection politiques publiques -
vie-publique.fr
}\index{dila}\index{nucleaire}\index{nucleaire!civil}\index{surete!nucleaire}\index{vie!publiquefr}
  \begin{wrapfigure}{r}{2.5cm}
    \centering
    \qrcode[nolink]{https://data.gouv.fr/dataset/536997b0a3a729239d204d6a}
  \end{wrapfigure}

Licence : \textbf{Licence Ouverte
}\newline
Créé le : 2013-07-08\newline
Modifié le : 2016-02-26\newline
Granularité : au pays\newline
Mise à jour : ponctuelle\newline
Popularité : 1 réutilisation,  0 suivi\newline
Mots-clé : \emph{dila, nucleaire, nucleaire-civil, surete-nucleaire, vie-publiquefr
}\newline
Permalien : \url{https://data.gouv.fr/dataset/536997b0a3a729239d204d6a}\newline

\par
\noindent
    Le programme nucléaire civil français : de la phase des réalisations à
l'ère des doutes : Chronologie (1952-2013); Contexte historique;
Panorama institutionnel; La sûreté nucléaire; Le futur bouquet
énergétique; Glossaire.


\vspace{0.5cm}
\needspace{12\baselineskip}
\subsection*{La politique du patrimoine - collection politiques publiques -
vie-publique.fr
}\index{dila}\index{monument!historique}\index{patrimoine}\index{patrimoine!culturel}\index{protection!du!patrimoine}\index{vie!publiquefr}
  \begin{wrapfigure}{r}{2.5cm}
    \centering
    \qrcode[nolink]{https://data.gouv.fr/dataset/536997b1a3a729239d204d6b}
  \end{wrapfigure}

Licence : \textbf{Licence Ouverte
}\newline
Créé le : 2013-07-08\newline
Modifié le : 2016-02-26\newline
Granularité : au pays\newline
Mise à jour : ponctuelle\newline
Popularité : 1 réutilisation,  0 suivi\newline
Mots-clé : \emph{dila, monument-historique, patrimoine, patrimoine-culturel, protection-du-patrimoine, vie-publiquefr
}\newline
Permalien : \url{https://data.gouv.fr/dataset/536997b1a3a729239d204d6b}\newline

\par
\noindent
    1913-2013 : cent ans de protection ; Chronologie ; Protection du
patrimoine ; Gestion du patrimoine ; Valorisation du patrimoine ; Action
internationale ; La gestion des monuments historiques; Bibliographie ;
Sélection de sites internet ; Glossaire


\vspace{0.5cm}
\needspace{12\baselineskip}
\subsection*{La politique du sport (1984-2008) - Collection politiques publiques -
vie-publique.fr
}\index{association}\index{dila}\index{federations!sportives}\index{mouvement!sportif}\index{sport}\index{sports!et!loisirs}\index{vie!publiquefr}
  \begin{wrapfigure}{r}{2.5cm}
    \centering
    \qrcode[nolink]{https://data.gouv.fr/dataset/536997b1a3a729239d204d6c}
  \end{wrapfigure}

Licence : \textbf{Licence Ouverte
}\newline
Créé le : 2013-07-08\newline
Modifié le : 2016-02-26\newline
Granularité : au pays\newline
Mise à jour : ponctuelle\newline
Popularité : 1 réutilisation,  0 suivi\newline
Mots-clé : \emph{association, dila, federations-sportives, mouvement-sportif, sport, sports-et-loisirs, vie-publiquefr
}\newline
Permalien : \url{https://data.gouv.fr/dataset/536997b1a3a729239d204d6c}\newline

\par
\noindent
    Le « modèle français » de l'organisation du sport repose sur la
coopération entre l'État, qui assure des fonctions régaliennes, et le
Mouvement sportif, structuré en fédérations et associations sportives
qui assurent de véritables missions de service public. L'organisation de
la pratique sportive en France repose essentiellement sur le secteur
associatif. Les deux principales lois de 1975 et de 1984, désormais
intégrées dans le code du sport, soulignent que les activités physiques
et sportives sont « d'intérêt général » et constituent un « droit ».
Elles confirment le rôle essentiel reconnu aux fédérations sportives
agréées. La politique du sport (1984-2008) : Chronologie ; La
gouvernance du sport en France ; Le sport pour tous : une mission de
service public ; Le sport de haut niveau et le sport professionnel ;
Sites de référence.


\vspace{0.5cm}
\needspace{12\baselineskip}
\subsection*{La politique hospitalière - collection politiques publiques -
vie-publique.fr
}\index{depense!de!sante}\index{dila}\index{hopital}\index{systeme!de!sante}\index{vie!publiquefr}
  \begin{wrapfigure}{r}{2.5cm}
    \centering
    \qrcode[nolink]{https://data.gouv.fr/dataset/536997b1a3a729239d204d6d}
  \end{wrapfigure}

Licence : \textbf{Licence Ouverte
}\newline
Créé le : 2013-07-08\newline
Modifié le : 2016-02-26\newline
Granularité : au pays\newline
Mise à jour : ponctuelle\newline
Popularité : 1 réutilisation,  0 suivi\newline
Mots-clé : \emph{depense-de-sante, dila, hopital, systeme-de-sante, vie-publiquefr
}\newline
Permalien : \url{https://data.gouv.fr/dataset/536997b1a3a729239d204d6d}\newline

\par
\noindent
    Concilier qualité des soins et maîtrise des dépenses : Chronologie
(1958-2010) ; Le système hospitalier en 2005 ; Les instruments ;
L'hôpital face à de nouveaux défis ; La nouvelle gouvernance
hospitalière ; Hôpital et territoire ; Glossaire ; Bibliographie ;
Sélection de sites


\vspace{0.5cm}
\needspace{12\baselineskip}
\subsection*{La politique industrielle (1945-2006) - collection politiques publiques
- vie-publique.fr
}\index{dila}\index{etat}\index{industrie}\index{industries}\index{politique!industrielle}\index{vie!publiquefr}
  \begin{wrapfigure}{r}{2.5cm}
    \centering
    \qrcode[nolink]{https://data.gouv.fr/dataset/536997b2a3a729239d204d6e}
  \end{wrapfigure}

Licence : \textbf{Licence Ouverte
}\newline
Créé le : 2013-07-08\newline
Modifié le : 2016-02-26\newline
Granularité : au pays\newline
Mise à jour : ponctuelle\newline
Popularité : 1 réutilisation,  0 suivi\newline
Mots-clé : \emph{dila, etat, industrie, industries, politique-industrielle, vie-publiquefr
}\newline
Permalien : \url{https://data.gouv.fr/dataset/536997b2a3a729239d204d6e}\newline

\par
\noindent
    De ``l'Etat entrepreneur'' à ``l'Etat facilitateur'' Chronologie ; Le
cadre de la politique industrielle ; Les orientations de la politique
industrielle ; Les piliers de la politique industrielle; Sélection de
sites ; Bibliographie ; Chiffres clés ; Glossaire


\vspace{0.5cm}
\needspace{12\baselineskip}
\subsection*{La politique pénitentiaire (1945-2005) - collection politiques publiques
- vie-publique.fr
}\index{dila}\index{justice}\index{justice!penale}\index{politique!penitentiaire}\index{prison}\index{reinsertion}\index{vie!publiquefr}
  \begin{wrapfigure}{r}{2.5cm}
    \centering
    \qrcode[nolink]{https://data.gouv.fr/dataset/536997b2a3a729239d204d6f}
  \end{wrapfigure}

Licence : \textbf{Licence Ouverte
}\newline
Créé le : 2013-07-08\newline
Modifié le : 2017-08-10\newline
Granularité : au pays\newline
Mise à jour : ponctuelle\newline
Popularité : 1 réutilisation,  0 suivi\newline
Mots-clé : \emph{dila, justice, justice-penale, politique-penitentiaire, prison, reinsertion, vie-publiquefr
}\newline
Permalien : \url{https://data.gouv.fr/dataset/536997b2a3a729239d204d6f}\newline

\par
\noindent
    Exécution des sentences pénales et principe de réinsertion : Chronologie
; Les acteurs ; La mission de garde ; La mission de réinsertion;
Chiffres clés ; Glossaire ; Bibliographie ; Sélection de sites


\vspace{0.5cm}
\needspace{12\baselineskip}
\subsection*{La politique portuaire (1992 - 2008) - collection politique publique -
vie-publique.fr
}\index{dila}\index{politique!portuaire}\index{ports}\index{ports!autonomes}\index{vie!publiquefr}
  \begin{wrapfigure}{r}{2.5cm}
    \centering
    \qrcode[nolink]{https://data.gouv.fr/dataset/536997b3a3a729239d204d70}
  \end{wrapfigure}

Licence : \textbf{Licence Ouverte
}\newline
Créé le : 2013-07-08\newline
Modifié le : 2016-02-26\newline
Granularité : au pays\newline
Mise à jour : ponctuelle\newline
Popularité : 1 réutilisation,  0 suivi\newline
Mots-clé : \emph{dila, politique-portuaire, ports, ports-autonomes, vie-publiquefr
}\newline
Permalien : \url{https://data.gouv.fr/dataset/536997b3a3a729239d204d70}\newline

\par
\noindent
    Chronologie ; Le diagnostic sur la situation des ports autonomes ; La
réforme portuaire du 4 juillet 2008 ; Glossaire ; Sites de référence


\vspace{0.5cm}
\needspace{12\baselineskip}
\subsection*{La politique spatiale - Collection politiques publiques -
vie-publique.fr
}\index{activites!spatiales}\index{astronautique}\index{dila}\index{espace}\index{espace!militaire}\index{politique!spatiale}\index{vie!publiquefr}
  \begin{wrapfigure}{r}{2.5cm}
    \centering
    \qrcode[nolink]{https://data.gouv.fr/dataset/536997b3a3a729239d204d71}
  \end{wrapfigure}

Licence : \textbf{Licence Ouverte
}\newline
Créé le : 2013-07-08\newline
Modifié le : 2016-02-26\newline
Granularité : au pays\newline
Mise à jour : ponctuelle\newline
Popularité : 1 réutilisation,  0 suivi\newline
Mots-clé : \emph{activites-spatiales, astronautique, dila, espace, espace-militaire, politique-spatiale, vie-publiquefr
}\newline
Permalien : \url{https://data.gouv.fr/dataset/536997b3a3a729239d204d71}\newline

\par
\noindent
    La politique spatiale, de l'astronautique française à l'Europe spatiale
: Chronologie; Essor de la politique spatiale; Cadre et acteurs de la
politique spatiale; L'espace militaire; Les activités spatiales face à
la concurrence.


\vspace{0.5cm}
\needspace{12\baselineskip}
\subsection*{La présomption d'innocence - collection politiques publiques -
vie-publique.fr
}\index{dila}\index{justice}\index{presomption!d!innocence}\index{vie!publiquefr}
  \begin{wrapfigure}{r}{2.5cm}
    \centering
    \qrcode[nolink]{https://data.gouv.fr/dataset/536997b8a3a729239d204d7c}
  \end{wrapfigure}

Licence : \textbf{Licence Ouverte
}\newline
Créé le : 2013-07-08\newline
Modifié le : 2016-02-26\newline
Granularité : au pays\newline
Mise à jour : ponctuelle\newline
Popularité : 1 réutilisation,  0 suivi\newline
Mots-clé : \emph{dila, justice, presomption-d-innocence, vie-publiquefr
}\newline
Permalien : \url{https://data.gouv.fr/dataset/536997b8a3a729239d204d7c}\newline

\par
\noindent
    La loi du 15 juin 2000 sur le renforcement de la présomption d'innocence
: Chronologie ; L'essentiel de la loi ; L'application de la loi ; Le
contexte international ; Exemples étrangers ; Glossaire ; Bibliographie


\vspace{0.5cm}
\needspace{12\baselineskip}
\subsection*{La prévention des risques technologiques et industriels majeurs
(1976-2001) - collection politique publique - vie-publique.fr
}\index{dila}\index{prevention!des!risques}\index{radioprotection}\index{risque!industriel}\index{risque!majeur}\index{risque!technologique}\index{securite!nucleaire}\index{vie!publiquefr}
  \begin{wrapfigure}{r}{2.5cm}
    \centering
    \qrcode[nolink]{https://data.gouv.fr/dataset/536997b8a3a729239d204d7d}
  \end{wrapfigure}

Licence : \textbf{Licence Ouverte
}\newline
Créé le : 2013-07-08\newline
Modifié le : 2016-02-26\newline
Granularité : au pays\newline
Mise à jour : ponctuelle\newline
Popularité : 1 réutilisation,  0 suivi\newline
Mots-clé : \emph{dila, prevention-des-risques, radioprotection, risque-industriel, risque-majeur, risque-technologique, securite-nucleaire, vie-publiquefr
}\newline
Permalien : \url{https://data.gouv.fr/dataset/536997b8a3a729239d204d7d}\newline

\par
\noindent
    De Seveso à AZF (Toulouse) : Chronologie ; Les risques : typologie ; Les
acteurs publics de la prévention ; Les moyens de la prévention ; Le
contrôle de la sûreté nucléaire et de la radioprotection ; La
coopération internationale ; Bibliographie ; Sites ; Glossaire


\vspace{0.5cm}
\needspace{12\baselineskip}
\subsection*{La régulation des relations de travail (1950-2014) - collection
politiques publiques - vie-publique.fr
}\index{dialogue!social}\index{dila}\index{pouvoirs!publics}\index{relations!professionnelles}\index{securite!publique}\index{travail}\index{vie!publiquefr}
  \begin{wrapfigure}{r}{2.5cm}
    \centering
    \qrcode[nolink]{https://data.gouv.fr/dataset/536997bba3a729239d204d84}
  \end{wrapfigure}

Licence : \textbf{Licence Ouverte
}\newline
Créé le : 2013-07-08\newline
Modifié le : 2016-02-26\newline
Granularité : au pays\newline
Mise à jour : ponctuelle\newline
Popularité : 1 réutilisation,  0 suivi\newline
Mots-clé : \emph{dialogue-social, dila, pouvoirs-publics, relations-professionnelles, securite-publique, travail, vie-publiquefr
}\newline
Permalien : \url{https://data.gouv.fr/dataset/536997bba3a729239d204d84}\newline

\par
\noindent
    Quel rôle pour les pouvoirs publics ? Chronologie ; L'Etat régulateur
des relations du travail ; L'Etat, acteur du développement du dialogue
social ; L'Etat garant de la protection de la santé et sécurité au
travail; Glossaire ; Bibliographie ; Sites de référence


\vspace{0.5cm}
\needspace{12\baselineskip}
\subsection*{La régulation des services publics (2001 - 2007) - collection politiques
publiques - vie-publique.fr
}\index{dila}\index{services!d!utilite!publique!et!s}\index{services!postaux}\index{services!publics}\index{transport!de!voyageurs}\index{vie!publiquefr}
  \begin{wrapfigure}{r}{2.5cm}
    \centering
    \qrcode[nolink]{https://data.gouv.fr/dataset/536997bba3a729239d204d85}
  \end{wrapfigure}

Licence : \textbf{Licence Ouverte
}\newline
Créé le : 2013-07-08\newline
Modifié le : 2016-02-26\newline
Granularité : au pays\newline
Mise à jour : ponctuelle\newline
Popularité : 1 réutilisation,  0 suivi\newline
Mots-clé : \emph{dila, services-d-utilite-publique-et-s, services-postaux, services-publics, transport-de-voyageurs, vie-publiquefr
}\newline
Permalien : \url{https://data.gouv.fr/dataset/536997bba3a729239d204d85}\newline

\par
\noindent
    Nouveaux enjeux et acteurs de la régulation des services publics de 2001
à 2007 ; Chronologie ; Ouverture à la concurrence et régulation :
nouveaux enjeux ; Régulations sectorielles depuis 2001 ; Bibliographie ;
Glossaire ; Sites de référence


\vspace{0.5cm}
\needspace{12\baselineskip}
\subsection*{La sécurité alimentaire (1990-2003) - collection politiques publiques -
vie-publique.fr
}\index{dila}\index{ogm}\index{securite!alimentaire}\index{vie!publiquefr}
  \begin{wrapfigure}{r}{2.5cm}
    \centering
    \qrcode[nolink]{https://data.gouv.fr/dataset/536997c5a3a729239d204d9f}
  \end{wrapfigure}

Licence : \textbf{Licence Ouverte
}\newline
Créé le : 2013-07-08\newline
Modifié le : 2016-02-26\newline
Granularité : au pays\newline
Mise à jour : ponctuelle\newline
Popularité : 1 réutilisation,  0 suivi\newline
Mots-clé : \emph{dila, ogm, securite-alimentaire, vie-publiquefr
}\newline
Permalien : \url{https://data.gouv.fr/dataset/536997c5a3a729239d204d9f}\newline

\par
\noindent
    Crise de l'ESB, OGM et sécurité des aliments ; Le dispositif français de
sécurité alimentaire ; La dimension européenne de la sécurité
alimentaire ; La dimension internationale de la sécurité alimentaire ;
ESB : la crise de la vache folle ; Les OGM ; Glossaire ; Bibliographie


\vspace{0.5cm}
\needspace{12\baselineskip}
\subsection*{La sécurité intérieure (1995-2009) - Collection politiques publiques -
vie-publique.fr
}\index{dila}\index{insecurite}\index{police}\index{securite}\index{securite!interieure}\index{securite!publique}\index{vie!publiquefr}
  \begin{wrapfigure}{r}{2.5cm}
    \centering
    \qrcode[nolink]{https://data.gouv.fr/dataset/536997c6a3a729239d204da0}
  \end{wrapfigure}

Licence : \textbf{Licence Ouverte
}\newline
Créé le : 2013-07-08\newline
Modifié le : 2016-03-10\newline
Granularité : au pays\newline
Mise à jour : ponctuelle\newline
Popularité : 2 réutilisations,  0 suivi\newline
Mots-clé : \emph{dila, insecurite, police, securite, securite-interieure, securite-publique, vie-publiquefr
}\newline
Permalien : \url{https://data.gouv.fr/dataset/536997c6a3a729239d204da0}\newline

\par
\noindent
    La sécurité intérieure (1995-2009), Évolution des concepts et des
pratiques : Chronologie; Les Français et l'insécurité; Les acteurs des
politiques de sécurité intérieure; La mise en oeuvre des politiques de
sécurité intérieure; Exemples étrangers; Chiffres-clés; Bibliographie;
Sélection de sites.


\vspace{0.5cm}
\needspace{12\baselineskip}
\subsection*{La sécurité sanitaire (1990-2008) - collection politiques publiques -
vie-publique.fr
}\index{aliments}\index{crise!sanitaire}\index{dila}\index{risque!sanitaire}\index{securite!sanitaire}\index{veille!sanitaire}\index{vie!publiquefr}
  \begin{wrapfigure}{r}{2.5cm}
    \centering
    \qrcode[nolink]{https://data.gouv.fr/dataset/536997c7a3a729239d204da2}
  \end{wrapfigure}

Licence : \textbf{Licence Ouverte
}\newline
Créé le : 2013-07-08\newline
Modifié le : 2016-02-26\newline
Granularité : au pays\newline
Mise à jour : ponctuelle\newline
Popularité : 1 réutilisation,  0 suivi\newline
Mots-clé : \emph{aliments, crise-sanitaire, dila, risque-sanitaire, securite-sanitaire, veille-sanitaire, vie-publiquefr
}\newline
Permalien : \url{https://data.gouv.fr/dataset/536997c7a3a729239d204da2}\newline

\par
\noindent
    Construction d'un dispositif de veille et de gestion des risques
sanitaires : Chronologie ; Veille et évaluation des risques sanitaires ;
Gestion des risques et des crises sanitaires ; Le cas particulier de la
sécurité sanitaire des aliments ; Glossaire ; Sélection de sites


\vspace{0.5cm}
\needspace{12\baselineskip}
\subsection*{L'enseignement du premier degré - collection politiques publiques -
vie-publique.fr
}\index{dila}\index{ecole}\index{ecole!ecoleprimaire}\index{enseignement!primaire}\index{politique!publique}\index{vie!publiquefr}
  \begin{wrapfigure}{r}{2.5cm}
    \centering
    \qrcode[nolink]{https://data.gouv.fr/dataset/536c4661a3a72933d8d1b3a0}
  \end{wrapfigure}

Licence : \textbf{Licence Ouverte
}\newline
Créé le : 2013-07-08\newline
Modifié le : 2016-02-26\newline
Granularité : au pays\newline
Mise à jour : ponctuelle\newline
Popularité : 1 réutilisation,  0 suivi\newline
Mots-clé : \emph{dila, ecole, ecole-ecoleprimaire, enseignement-primaire, politique-publique, vie-publiquefr
}\newline
Permalien : \url{https://data.gouv.fr/dataset/536c4661a3a72933d8d1b3a0}\newline

\par
\noindent
    De loi d'orientation en réforme : Chronologie (1789-2014) ; Les
enseignants ; La réforme de l'école primaire de 2008 ; Organisation du
premier degré (1989-2005) ; Les priorités pédagogiques (1989-2005) ;
Glossaire ; Bibliographie ; Sélection de sites


\vspace{0.5cm}
\needspace{12\baselineskip}
\subsection*{L'enseignement du second degré : le collège - Collection politiques
publiques - vie-publique.fr
}\index{college}\index{dila}\index{education!prioritaire}\index{enseignement}\index{enseignement!public!et!prive}\index{vie!publiquefr}
  \begin{wrapfigure}{r}{2.5cm}
    \centering
    \qrcode[nolink]{https://data.gouv.fr/dataset/536997dba3a729239d204dd6}
  \end{wrapfigure}

Licence : \textbf{Licence Ouverte
}\newline
Créé le : 2013-09-07\newline
Modifié le : 2016-02-26\newline
Granularité : au pays\newline
Mise à jour : ponctuelle\newline
Popularité : 1 réutilisation,  0 suivi\newline
Mots-clé : \emph{college, dila, education-prioritaire, enseignement, enseignement-public-et-prive, vie-publiquefr
}\newline
Permalien : \url{https://data.gouv.fr/dataset/536997dba3a729239d204dd6}\newline

\par
\noindent
    Ce dossier revient sur les quelques 40 années d'existence du collège
unique : Chronologie; Le collège unique; L'organisation du collège;
L'éducation prioritaire.


\vspace{0.5cm}
\needspace{12\baselineskip}
\subsection*{L'enseignement supérieur (1968-2010) - collection politiques publiques -
vie-publique.fr
}\index{dila}\index{enseignement!superieur}\index{etudiants}\index{universite}\index{universites}\index{vie!publiquefr}
  \begin{wrapfigure}{r}{2.5cm}
    \centering
    \qrcode[nolink]{https://data.gouv.fr/dataset/536997dba3a729239d204dd8}
  \end{wrapfigure}

Licence : \textbf{Licence Ouverte
}\newline
Créé le : 2013-07-08\newline
Modifié le : 2016-02-26\newline
Granularité : au pays\newline
Popularité : 1 réutilisation,  0 suivi\newline
Mots-clé : \emph{dila, enseignement-superieur, etudiants, universite, universites, vie-publiquefr
}\newline
Permalien : \url{https://data.gouv.fr/dataset/536997dba3a729239d204dd8}\newline

\par
\noindent
    L'université : vers quelle autonomie ? : Chronologie ; Quelle
gouvernance pour les universités ?; Nouvelles compétences, nouvelles
ressources


\vspace{0.5cm}
\needspace{12\baselineskip}
\subsection*{Les droits des femmes - Collection politiques publiques -
vie-publique.fr
}\index{dila}\index{droits!de!la!femme}\index{egalite!homme!femme}\index{parite}\index{vie!publiquefr}
  \begin{wrapfigure}{r}{2.5cm}
    \centering
    \qrcode[nolink]{https://data.gouv.fr/dataset/5369981ba3a729239d204e7f}
  \end{wrapfigure}

Licence : \textbf{Licence Ouverte
}\newline
Créé le : 2013-07-08\newline
Modifié le : 2016-02-26\newline
Granularité : au pays\newline
Mise à jour : ponctuelle\newline
Popularité : 1 réutilisation,  0 suivi\newline
Mots-clé : \emph{dila, droits-de-la-femme, egalite-homme-femme, parite, vie-publiquefr
}\newline
Permalien : \url{https://data.gouv.fr/dataset/5369981ba3a729239d204e7f}\newline

\par
\noindent
    La politique de l'égalité entre les femmes et les hommes couvre
plusieurs domaines, traités dans ce dossier : les droits dans la sphère
privée, la lutte contre les violences faites aux femmes, l'égalité
professionnelle, la parité en politique. Les droits des femmes,
Elaboration d'une politique de l'égalité : Chronologie; Egalité et
droits dans la sphère privée; Les violences faites aux femmes; L'égalité
professionnelle; La parité politique.


\vspace{0.5cm}
\needspace{12\baselineskip}
\subsection*{Les politiques d'insertion (1980-2009) - Collection politiques publiques
- vie-publique.fr
}\index{departement}\index{departements}\index{dila}\index{exclusion}\index{insertion}\index{insertion!sociale}\index{politique!d!insertion}\index{rsa}\index{vie!publiquefr}
  \begin{wrapfigure}{r}{2.5cm}
    \centering
    \qrcode[nolink]{https://data.gouv.fr/dataset/5369986ba3a729239d204f5c}
  \end{wrapfigure}

Licence : \textbf{Licence Ouverte
}\newline
Créé le : 2013-07-08\newline
Modifié le : 2016-02-26\newline
Granularité : au pays\newline
Mise à jour : ponctuelle\newline
Popularité : 1 réutilisation,  0 suivi\newline
Mots-clé : \emph{departement, departements, dila, exclusion, insertion, insertion-sociale, politique-d-insertion, rsa, vie-publiquefr
}\newline
Permalien : \url{https://data.gouv.fr/dataset/5369986ba3a729239d204f5c}\newline

\par
\noindent
    Les politiques d'insertion (1980-2009), De l'assistance à la lutte
contre l'exclusion : Chronologie; Lutter contre la pauvreté; Favoriser
l'accès à l'emploi; Le RSA et la réforme des politiques d'insertion.


\vspace{0.5cm}
\needspace{12\baselineskip}
\subsection*{L'Etat et les cultes - collection politiques publiques - vie-publique.fr
}\index{culte}\index{dila}\index{eglise}\index{etat}\index{laicite}\index{religion}\index{vie!publiquefr}
  \begin{wrapfigure}{r}{2.5cm}
    \centering
    \qrcode[nolink]{https://data.gouv.fr/dataset/5369988ba3a729239d204fb6}
  \end{wrapfigure}

Licence : \textbf{Licence Ouverte
}\newline
Créé le : 2013-07-08\newline
Modifié le : 2016-03-01\newline
Granularité : au pays\newline
Mise à jour : ponctuelle\newline
Popularité : 1 réutilisation,  0 suivi\newline
Mots-clé : \emph{culte, dila, eglise, etat, laicite, religion, vie-publiquefr
}\newline
Permalien : \url{https://data.gouv.fr/dataset/5369988ba3a729239d204fb6}\newline

\par
\noindent
    Exposé des principaux éléments de laïcité telle qu'elle est issue de la
loi de 1905 concernant la séparation des églises et de l'Etat : le
régime de séparation établi entre l'Etat et les cultes mais aussi la
reconnaissance de la liberté de conscience et de la liberté religieuse,
enfin les relations maintenues entre l'administration et les cultes. La
loi de 1905 ne s'applique pas sur l'ensemble du territoire français et
les régimes dérogatoires sont également présentés.


\vspace{0.5cm}
\needspace{12\baselineskip}
\subsection*{L'Etat et l'internet - Collection politiques publiques - vie-publique.fr
}\index{dila}\index{droits!d!auteur}\index{etat}\index{internet}\index{regulation}\index{vie!publiquefr}
  \begin{wrapfigure}{r}{2.5cm}
    \centering
    \qrcode[nolink]{https://data.gouv.fr/dataset/5369988ba3a729239d204fb7}
  \end{wrapfigure}

Licence : \textbf{Licence Ouverte
}\newline
Créé le : 2013-07-08\newline
Modifié le : 2016-02-26\newline
Granularité : au pays\newline
Mise à jour : ponctuelle\newline
Popularité : 1 réutilisation,  1 suivi\newline
Mots-clé : \emph{dila, droits-d-auteur, etat, internet, regulation, vie-publiquefr
}\newline
Permalien : \url{https://data.gouv.fr/dataset/5369988ba3a729239d204fb7}\newline

\par
\noindent
    L'Etat et l'internet entre développement et régulation. Parallèlement à
une politique volontariste visant à favoriser l'Internet, les pouvoirs
publics mettent en place des politiques de régulation afin, d'une part,
de protéger les droits d'auteur, et d'autre part, de garantir les
libertés individuelles et le respect des données privées.


\vspace{0.5cm}
\needspace{12\baselineskip}
\subsection*{Le temps de travail (1980-2008) - collection politiques publiques -
vie-publique.fr
}\index{dila}\index{politique!de!l!emploi}\index{temps!de!travail}\index{vie!publiquefr}
  \begin{wrapfigure}{r}{2.5cm}
    \centering
    \qrcode[nolink]{https://data.gouv.fr/dataset/5369988ba3a729239d204fb9}
  \end{wrapfigure}

Licence : \textbf{Licence Ouverte
}\newline
Créé le : 2013-07-08\newline
Modifié le : 2016-03-16\newline
Granularité : au pays\newline
Mise à jour : ponctuelle\newline
Popularité : 1 réutilisation,  0 suivi\newline
Mots-clé : \emph{dila, politique-de-l-emploi, temps-de-travail, vie-publiquefr
}\newline
Permalien : \url{https://data.gouv.fr/dataset/5369988ba3a729239d204fb9}\newline

\par
\noindent
    Une intervention publique à la fois confirmée et contestée. Chronologie
; La régulation du temps de travail ; Du partage du travail à
l'augmentation du temps de travail ; Au final, un temps de travail plus
individualisé ; Glossaire


\vspace{0.5cm}
\needspace{12\baselineskip}
\subsection*{L'hébergement d'urgence (1980-2008) - Collection politiques publiques -
vie-publique.fr
}\index{aide!au!logement}\index{dila}\index{hebergement!d!urgence}\index{sdf}\index{vie!publiquefr}
  \begin{wrapfigure}{r}{2.5cm}
    \centering
    \qrcode[nolink]{https://data.gouv.fr/dataset/536998b0a3a729239d205014}
  \end{wrapfigure}

Licence : \textbf{Licence Ouverte
}\newline
Créé le : 2013-07-08\newline
Modifié le : 2016-02-26\newline
Granularité : au pays\newline
Mise à jour : ponctuelle\newline
Popularité : 1 réutilisation,  0 suivi\newline
Mots-clé : \emph{aide-au-logement, dila, hebergement-d-urgence, sdf, vie-publiquefr
}\newline
Permalien : \url{https://data.gouv.fr/dataset/536998b0a3a729239d205014}\newline

\par
\noindent
    L'hébergement d'urgence (1980-2008), une politique publique pour
l'accueil des personnes sans domicile : Chronologie; Un dispositif
croissant pour des publics diversifiés; Un dispositif sous la
responsabilité de l'Etat; L'institution d'un droit à l'hébergement;
Sites de référence.


\vspace{0.5cm}
\needspace{12\baselineskip}
\subsection*{Liste des applications et des versions mobiles des sites Internet
publics
}\index{application}\index{application!mobile}\index{dila}\index{site!internet}\index{site!web}
  \begin{wrapfigure}{r}{2.5cm}
    \centering
    \qrcode[nolink]{https://data.gouv.fr/dataset/542d44b788ee38615a8561ab}
  \end{wrapfigure}

Licence : \textbf{Licence Ouverte
}\newline
Créé le : 2014-10-02\newline
Modifié le : 2016-03-11\newline
Granularité : au département\newline
Mise à jour : ponctuelle\newline
Popularité : 1 réutilisation,  1 suivi\newline
Mots-clé : \emph{application, application-mobile, dila, site-internet, site-web
}\newline
Permalien : \url{https://data.gouv.fr/dataset/542d44b788ee38615a8561ab}\newline

\par
\noindent
    La DILA a recensé les sites en responsive design, les sites mobiles et
les applications mobiles de l'Etat, des régions et des départements.


\vspace{0.5cm}
\needspace{12\baselineskip}
\subsection*{Panorama des lois de vie-publique.fr
}\index{assemblee!nationale}\index{dila}\index{gouvernement}\index{loi}\index{loi!de!finances}\index{vie!publiquefr}
  \begin{wrapfigure}{r}{2.5cm}
    \centering
    \qrcode[nolink]{https://data.gouv.fr/dataset/53699b6ea3a729239d20572c}
  \end{wrapfigure}

Licence : \textbf{Licence Ouverte
}\newline
Créé le : 2013-07-08\newline
Modifié le : 2016-02-26\newline
Granularité : au pays\newline
Popularité : 1 réutilisation,  0 suivi\newline
Mots-clé : \emph{assemblee-nationale, dila, gouvernement, loi, loi-de-finances, vie-publiquefr
}\newline
Permalien : \url{https://data.gouv.fr/dataset/53699b6ea3a729239d20572c}\newline

\par
\noindent
    L'objectif du panorama des lois est de suivre l'activité parlementaire
au jour le jour et d'offrir un descriptif synthétique des textes
législatifs débattus pendant l'actuelle législature. Sont sélectionnés
les textes susceptibles de faire l'objet d'une politique publique ou de
modifier une politique en cours. Les différentes étapes suivies par les
textes dans le processus législatif sont indiquées, de l'élaboration du
projet et jusqu'à l'évaluation de la mise en application de la loi. Des
liens sont proposés vers des documents complémentaires pertinents
(rapports, présentation des projets en conseil des ministres, documents
législatifs, Journal officiel).


\vspace{0.5cm}
\needspace{12\baselineskip}
\subsection*{Parc naturel régional - PNR
}
  \begin{wrapfigure}{r}{2.5cm}
    \centering
    \qrcode[nolink]{https://data.gouv.fr/dataset/53699b78a3a729239d205743}
  \end{wrapfigure}

Licence : \textbf{Licence Ouverte
}\newline
Créé le : 2013-07-08\newline
Modifié le : 2016-02-13\newline
De 2011-01-01 à 2011-12-31\newline
Mise à jour : annuelle\newline
Popularité : 1 réutilisation,  0 suivi\newline
Mots-clé : \emph{aucun
}\newline
Permalien : \url{https://data.gouv.fr/dataset/53699b78a3a729239d205743}\newline

\par
\noindent
    Les parcs naturels régionaux (PNR) concourent à la politique de
protection de l'environnement, d'aménagement du territoire, de
développement économique et social et d'éducation et de formation du
public. Ils constituent un cadre privilégié des actions menées par les
collectivités publiques en faveur de la préservation des paysages et du
patrimoine naturel et culturel.La charte du parc détermine pour le
territoire du parc les orientations de protection, de mise en valeur et
de développement et les mesures permettant de les mettre en oeuvre. Elle
comporte un plan élaboré à partir d'un inventaire du patrimoine
indiquant les différentes zones du parc et leur vocation, accompagné
d'un document déterminant les orientations et les principes fondamentaux
de protection des structures paysagères sur le territoire du parc.


\vspace{0.5cm}
\needspace{12\baselineskip}
\subsection*{Protocole du Gouvernement
}\index{composition!du!gouvernement}\index{dila}\index{gouvernement}\index{gouvernements}\index{liste!des!ministres}\index{ministere}\index{ministeres}\index{ministre}\index{ministres}\index{protocole}
  \begin{wrapfigure}{r}{2.5cm}
    \centering
    \qrcode[nolink]{https://data.gouv.fr/dataset/598d7459c751df5a5e67466d}
  \end{wrapfigure}

Licence : \textbf{Licence Ouverte
}\newline
Créé le : 2017-08-11\newline
Modifié le : 2019-02-11\newline
Mise à jour : ponctuelle\newline
Popularité : 1 réutilisation,  0 suivi\newline
Mots-clé : \emph{composition-du-gouvernement, dila, gouvernement, gouvernements, liste-des-ministres, ministere, ministeres, ministre, ministres, protocole
}\newline
Permalien : \url{https://data.gouv.fr/dataset/598d7459c751df5a5e67466d}\newline

\par
\noindent
    La DILA met à disposition, exclusivement en open data, le
\textbf{Protocole du gouvernement français}.

Le fichier comprend la structure des différents gouvernements (hors
secrétaires d'Etat) depuis le 2 avril 2014.

L'ordre d'affichage des ministères dans le fichier correspond à l'ordre
protocolaire.

Pour chaque gouvernement, les informations fournies sont :

\begin{itemize}

\item
  La date de création du gouvernement
\item
  Le prénom et nom du Président de la République
\item
  Le prénom et nom du Premier ministre
\end{itemize}

Le fichier est mis à jour à chaque remaniement ministériel.\\
Il s'agit d'une relivraison globale.


\vspace{0.5cm}
\needspace{12\baselineskip}
\subsection*{Référentiel de l'organisation administrative de l'Etat
}\index{administration}\index{administration!generale}\index{administration!publique}\index{annuaire}\index{annuaire!de!services}\index{autorites!publiques}\index{base}\index{coordonnees}\index{coordonnees!geographiques}\index{dila}\index{nationale}\index{organisation!administrative}\index{organisme!public}\index{referentiel}\index{service!public}\index{services!d!utilite!publique}\index{services!de!l!etat}
  \begin{wrapfigure}{r}{2.5cm}
    \centering
    \qrcode[nolink]{https://data.gouv.fr/dataset/57343feb88ee3823b0d1b934}
  \end{wrapfigure}

Licence : \textbf{Licence Ouverte
}\newline
Créé le : 2016-05-12\newline
Modifié le : 2019-02-07\newline
De 2016-05-12 à 2026-05-12\newline
Mise à jour : hebdomadaire\newline
Popularité : 4 réutilisations,  14 suivis\newline
Mots-clé : \emph{administration, administration-generale, administration-publique, annuaire, annuaire-de-services, autorites-publiques, base, coordonnees, coordonnees-geographiques, dila, nationale, organisation-administrative, organisme-public, referentiel, service-public, services-d-utilite-publique, services-de-l-etat
}\newline
Permalien : \url{https://data.gouv.fr/dataset/57343feb88ee3823b0d1b934}\newline

\par
\noindent
    \href{https://www.legifrance.gouv.fr/eli/decret/2017/3/14/PRMJ1636987D/jo/texte}{Le
décret du 14 mars 2017} a institué le \textbf{Service Public de la
Donnée}.

Celui-ci met à la disposition du public \textbf{9 jeux de données de
référence} parmi lesquels la base nationale de l'organisation
administrative de l'Etat, produite et diffusée par la Dila sur son site
\href{https://lannuaire.service-public.fr/}{service-public.fr}.

Le Référentiel de l'organisation administrative de l'Etat, nouvelle
appellation de la base, comprend toutes les institutions régies par la
Constitution de la Ve République et les administrations qui en
dépendent, soit environ 6000 organismes. Le périmètre couvre \textbf{les
services centraux de l'Etat}, jusqu'au niveau des bureaux.

Le référentiel comprend les missions, l'organisation hiérarchique des
services et leurs coordonnées complètes.

\href{https://echanges.dila.gouv.fr/OPENDATA/RefOrgaAdminEtat/FluxHistorique/}{Accéder
à l'historique des versions}

Vous pouvez nous écrire ou vous abonner à une alerte par mail adressé à
: \textbf{donnees-dila@dila.gouv.fr}


\vspace{0.5cm}
\needspace{12\baselineskip}
\subsection*{Registre de prévention des conflits d intérêts
}\index{attributions}\index{gouvernement}\index{liste!des!ministres}\index{ministre}\index{ministres}\index{secretaire!d!etat}\index{secretaires!detat}
  \begin{wrapfigure}{r}{2.5cm}
    \centering
    \qrcode[nolink]{https://data.gouv.fr/dataset/5a3a630088ee38031ddf0a1e}
  \end{wrapfigure}

Licence : \textbf{Licence Ouverte
}\newline
Créé le : 2017-12-20\newline
Modifié le : 2019-01-18\newline
De 2018-01-01 à 2019-01-08\newline
Mise à jour : ponctuelle\newline
Popularité : 1 réutilisation,  1 suivi\newline
Mots-clé : \emph{attributions, gouvernement, liste-des-ministres, ministre, ministres, secretaire-d-etat, secretaires-detat
}\newline
Permalien : \url{https://data.gouv.fr/dataset/5a3a630088ee38031ddf0a1e}\newline

\par
\noindent
    Registre de prévention des conflits d'intérêts.

Le
\href{https://www.legifrance.gouv.fr/affichTexteArticle.do?cidTexte=JORFTEXT000000856038\&idArticle=LEGIARTI000006530688\&dateTexte=19590123\&categorieLien=cid}{décret
n\degree{} 59-178 du 22 janvier 1959 relatif aux attributions des
ministres} prévoit, dans sa rédaction résultant du décret n\degree{}
2014-34 du 16 janvier 2014 relatif à la prévention des conflits
d'intérêts dans l'exercice des fonctions ministérielles, que les
ministres ou les membres du Gouvernement placés auprès d'un ministre qui
ont estimé se trouver en situation de conflit d'intérêts en informent
par écrit le Premier ministre. Des décrets déterminent en conséquence
les attributions que le Premier ministre ou le ministre auprès duquel
sont placés les membres du Gouvernement concernés, exerce à leur place.
Un dispositif comparable est prévu lorsque le Premier ministre estime se
trouver lui-même en situation de conflit d'intérêts pour l'exercice de
certains de ses pouvoirs.

Le
\href{https://www.legifrance.gouv.fr/affichTexte.do?cidTexte=JORFTEXT000036334126}{décret
n\degree{} 2017-1792 du 28 décembre 2017 relatif au registre recensant
les cas dans lesquels un membre du Gouvernement estime ne pas devoir
exercer ses attributions en raison d'une situation de conflit
d'intérêts}, pris pour l'application de
l'\href{https://www.legifrance.gouv.fr/affichTexteArticle.do?cidTexte=JORFTEXT000035567974\&idArticle=JORFARTI000035567990\&categorieLien=cid}{article
6 de la loi n\degree{}2017-1339 du 15 septembre 2017} pour la confiance
dans la vie politique, précise les modalités de tenue d'un registre
électronique accessible au public sur le site internet « gouvernement.fr
» (dit «
\href{https://www.gouvernement.fr/registre-de-prevention-des-conflits-d-interets}{registre
de prévention des conflits d'intérêt} ») qui recensera, à compter du 1er
janvier 2018 :

1\degree{} Les délégations du Premier ministre prises sur le fondement
de l'article 2 du décret du 22 janvier 1959 ;

2\degree{} Les décrets pris pour les ministres sur le fondement de
l'article 2-1 du même décret ;

3\degree{} Les décrets pris pour les ministres placés auprès d'un
ministre et pour les secrétaires d'Etat sur le fondement de l'article
2-2 du même décret ;

4\degree{} Les cas dans lesquels un membre du Gouvernement a estimé ne
pas pouvoir participer à une délibération en Conseil des ministres en
raison d'une situation de conflit d'intérêts relative à la question
débattue.

Le fichier XML est interopérable avec le
\href{https://www.data.gouv.fr/fr/datasets/protocole-du-gouvernement/}{jeu
de données ``Protocole du gouvernement''}


\vspace{0.5cm}
\needspace{12\baselineskip}
\subsection*{Répertoire des débats et consultations publics - vie-publique.fr
}\index{collectivite!locale}\index{collectivite!territoriale}\index{consultations}\index{debat}\index{debats!publics}\index{dila}\index{etablissements!publics}\index{etat}\index{forum}\index{vie!publiquefr}
  \begin{wrapfigure}{r}{2.5cm}
    \centering
    \qrcode[nolink]{https://data.gouv.fr/dataset/53699f06a3a729239d20601c}
  \end{wrapfigure}

Licence : \textbf{Licence Ouverte
}\newline
Créé le : 2013-07-08\newline
Modifié le : 2016-02-26\newline
Granularité : à la région\newline
Popularité : 1 réutilisation,  0 suivi\newline
Mots-clé : \emph{collectivite-locale, collectivite-territoriale, consultations, debat, debats-publics, dila, etablissements-publics, etat, forum, vie-publiquefr
}\newline
Permalien : \url{https://data.gouv.fr/dataset/53699f06a3a729239d20601c}\newline

\par
\noindent
    Répertoire des principaux débats, consultations et forums publics
répartis sur le territoire, avec accès aux synthèses finales
lorsqu'elles existent. Sont recensés les débats publics en ligne ou les
débats publics mettant à disposition une documentation en ligne. Depuis
le 1er janvier 2012, sont en outre référencées, les consultations
ouvertes sur l'Internet par l'Etat, ses établissements publics ou les
collectivités territoriales préalablement à l'adoption d'un texte
normatif.


\vspace{0.5cm}
\needspace{12\baselineskip}
\subsection*{SARDE
}\index{dila}\index{sarde}\index{textes!legislatifs}
  \begin{wrapfigure}{r}{2.5cm}
    \centering
    \qrcode[nolink]{https://data.gouv.fr/dataset/574453aec751df2faf8cc4b3}
  \end{wrapfigure}

Licence : \textbf{Licence Ouverte
}\newline
Créé le : 2016-05-24\newline
Modifié le : 2017-09-04\newline
Mise à jour : quotienne\newline
Popularité : 1 réutilisation,  2 suivis\newline
Mots-clé : \emph{dila, sarde, textes-legislatifs
}\newline
Permalien : \url{https://data.gouv.fr/dataset/574453aec751df2faf8cc4b3}\newline

\par
\noindent
    SARDE (Système d'Aide à la Recherche Documentaire Elaborée) est un
référentiel conçu pour alimenter un mode de recherche thématique sur la
majeure partie des textes législatifs et réglementaires en vigueur.

Sont référencés les textes publiés dans l'édition ``Lois et décrets'' du
Journal officiel ainsi que dans les Bulletins officiels diffusés par la
DILA.

Ce référentiel comporte environ 16000 descripteurs organisés sur 2
niveaux hiérarchiques.

\href{ftp://echanges.dila.gouv.fr/SARDE/}{Pour accéder au répertoire des
données SARDE sous le protocole ftp cliquer ici}

Vous pouvez nous écrire ou vous abonner à une alerte par mail adressé à
: \textbf{donnees-dila@dila.gouv.fr}


\vspace{0.5cm}
\needspace{12\baselineskip}
\subsection*{Service-public.fr - Actualités Particuliers
}\index{actualite!de!la!vie!publique}\index{actualites}\index{dila}\index{service!public}
  \begin{wrapfigure}{r}{2.5cm}
    \centering
    \qrcode[nolink]{https://data.gouv.fr/dataset/53699fe3a3a729239d206225}
  \end{wrapfigure}

Licence : \textbf{Licence Ouverte
}\newline
Créé le : 2013-07-08\newline
Modifié le : 2015-12-27\newline
De 2012-01-01 à 2023-02-20\newline
Granularité : au pays\newline
Mise à jour : quotienne\newline
Popularité : 2 réutilisations,  2 suivis\newline
Mots-clé : \emph{actualite-de-la-vie-publique, actualites, dila, service-public
}\newline
Permalien : \url{https://data.gouv.fr/dataset/53699fe3a3a729239d206225}\newline

\par
\noindent
    10 derniers articles et brèves d'actualité publiés sur service-public.fr
au format ATOM.


\vspace{0.5cm}
\needspace{12\baselineskip}
\subsection*{Service-public.fr - Annuaire de l'administration - Base de données
locales
}\index{administration!locale}\index{administration!publique}\index{annuaire}\index{annuaire!de!services}\index{communes}\index{coordonnees}\index{coordonnees!geographiques}\index{dila}\index{geolocalisation}\index{guichets}\index{mairie}\index{organisme!public}\index{services!d!utilite!publique!et!s}
  \begin{wrapfigure}{r}{2.5cm}
    \centering
    \qrcode[nolink]{https://data.gouv.fr/dataset/53699fe4a3a729239d206227}
  \end{wrapfigure}

Licence : \textbf{Licence Ouverte
}\newline
Créé le : 2013-07-08\newline
Modifié le : 2016-03-16\newline
De 2012-03-06 à 2020-03-06\newline
Granularité : à la commune\newline
Mise à jour : quotienne\newline
Popularité : 12 réutilisations,  44 suivis\newline
Mots-clé : \emph{administration-locale, administration-publique, annuaire, annuaire-de-services, communes, coordonnees, coordonnees-geographiques, dila, geolocalisation, guichets, mairie, organisme-public, services-d-utilite-publique-et-s
}\newline
Permalien : \url{https://data.gouv.fr/dataset/53699fe4a3a729239d206227}\newline

\par
\noindent
    La Base de données locales référence plus de 60 000 guichets publics
locaux (mairies, organismes sociaux, services de l'état, etc.). Elle
fournit leurs coordonnées (adresses, téléphones, site internet, horaires
d'ouverture, coordonnées de géolocalisation). En complément, sont
indexés plus de 36 000 fichiers des communes, précisant la compétence
géographique des guichets. La liste des types d'organismes référencés
dans la base est consultable dans le fichier PDF joint ci-dessous. Le
schéma et sa documentation sont disponibles à l'adresse
:\url{https://www.service-public.fr/partenaires/comarquage/documentation}


\vspace{0.5cm}
\needspace{12\baselineskip}
\subsection*{Service-public.fr- Guide « vos droits et démarches »
Professionnels-Entreprises
}\index{administration!electronique}\index{agenda}\index{cessation!d!activite}\index{cessations}\index{cession}\index{commerce}\index{creation}\index{dila}\index{finances}\index{fiscalite}\index{gestion}\index{professionnel}\index{ressources!humaines}\index{secteurs}\index{ventes}
  \begin{wrapfigure}{r}{2.5cm}
    \centering
    \qrcode[nolink]{https://data.gouv.fr/dataset/53699fe5a3a729239d206233}
  \end{wrapfigure}

Licence : \textbf{Licence Ouverte
}\newline
Créé le : 2013-07-08\newline
Modifié le : 2018-09-18\newline
De 2013-01-01 à 2020-01-31\newline
Granularité : à la commune\newline
Mise à jour : quotienne\newline
Popularité : 3 réutilisations,  7 suivis\newline
Mots-clé : \emph{administration-electronique, agenda, cessation-d-activite, cessations, cession, commerce, creation, dila, finances, fiscalite, gestion, professionnel, ressources-humaines, secteurs, ventes
}\newline
Permalien : \url{https://data.gouv.fr/dataset/53699fe5a3a729239d206233}\newline

\par
\noindent
    Vos droits et démarches pour les professionnels contient : 650 fiches et
questions réponses, plusieurs milliers de ressources et de liens vers
les sites publics (formulaires, démarches en ligne, textes de référence,
sites web publics, etc.) pour exercer ses droits et accomplir ses
démarches. Pour plus d'information, consulter la
\href{https://www.service-public.fr/partenaires/comarquage/}{rubrique
comarquage} du site Service-public.


\vspace{0.5cm}
\needspace{12\baselineskip}
\subsection*{Simulateur CHOIX DE FORFAIT DE PUBLICATION BOAMP
}\index{acheteur}\index{annonces}\index{boamp}\index{boamp!fr}\index{calcul}\index{dila}\index{forfait}\index{forfait!de!publication}\index{marche!public}\index{marches!publics}\index{publication}\index{simulateur}\index{simulateur!boamp}\index{simulateurs}\index{simulateurs!dila}
  \begin{wrapfigure}{r}{2.5cm}
    \centering
    \qrcode[nolink]{https://data.gouv.fr/dataset/5886230b88ee387e3f9b81a5}
  \end{wrapfigure}

Licence : \textbf{Licence Ouverte
}\newline
Créé le : 2017-01-23\newline
Modifié le : 2017-09-04\newline
Mise à jour : ponctuelle\newline
Popularité : 1 réutilisation,  0 suivi\newline
Mots-clé : \emph{acheteur, annonces, boamp, boamp-fr, calcul, dila, forfait, forfait-de-publication, marche-public, marches-publics, publication, simulateur, simulateur-boamp, simulateurs, simulateurs-dila
}\newline
Permalien : \url{https://data.gouv.fr/dataset/5886230b88ee387e3f9b81a5}\newline

\par
\noindent
    La DILA diffuse, sur le site boamp.fr, les avis d'appel public à la
concurrence et les résultats de marchés de l'État, l'armée, les
collectivités territoriales et leurs établissements publics. On y trouve
également des contrats de partenariat public-privé et des avis de
concession. Une diffusion est obligatoire au BOAMP pour tous les marchés
dont le montant est supérieur aux seuils européens. En dessous de ces
seuils, une publication reste obligatoire au BOAMP ou dans un journal
habilité à recevoir des annonces légales. Pour les marchés à procédure
adaptée (MAPA) inférieurs à 90 000 \euro{} HT, l'acheteur public a le
choix des supports qu'il utilisera pour assurer la publicité de ses
marchés.

\textbf{Pour faciliter le travail des acheteurs publics}, la Dila a mis
au point 2 simulateurs publiés sur son site boamp.fr :

\begin{itemize}

\item
  Simulateur d'aide au choix des formulaires de publication au BOAMP
\item
  Simulateur d'aide au choix du forfait de publication au BOAMP
\end{itemize}

\textbf{Pour les entreprises}, la Dila a conçu et publié sur ses sites
service-public.fr et boamp.fr un simulateur de calcul des intérêts
moratoires des marchés publics.

\textbf{Le simulateur d'aide au choix du forfait de publication} permet
aux acheteurs publics clients du BOAMP de déterminer le forfait le mieux
adapté à leurs besoins de publication en vue de bénéficier de tarifs
préférentiels.

Il leur permet d'optimiser leur budget de publication en faisant une
estimation des unités de publication nécessaires aux différents types
d'avis à publier dans l'année (sur la base de l'année précédente par
exemple).

Les données du simulateur sont mises à jour en fonction des évolutions
de la règlementation.

Les simulateurs développés par la DILA utilisent le \textbf{moteur de
simulation G6K} dont les sources sont accessibles via le lien
:\url{https://github.com/eureka2/G6K} Outre les sources, sont à
disposition via ce lien :

\begin{itemize}

\item
  La définition (étapes, règles,.), au format XSD, valable pour tous les
  simulateurs développés avec le moteur G6K ;
\item
  La procédure de mise à disposition et d'installation du moteur.
\end{itemize}

\textbf{Les données du simulateur mises à disposition par la DILA} sont
constituées d'un fichier XML pour la définition du simulateur.


\vspace{0.5cm}
\needspace{12\baselineskip}
\subsection*{Simulateur CHOIX DE FORMULAIRE DE PUBLICATION BOAMP
}\index{boamp}\index{boamp!fr}\index{choix!de!formulaire}\index{dila}\index{formulaire!de!publication}\index{simulateur}\index{simulateur!boamp}\index{simulateurs}\index{simulateurs!dila}\index{site!boamp}
  \begin{wrapfigure}{r}{2.5cm}
    \centering
    \qrcode[nolink]{https://data.gouv.fr/dataset/59a982a188ee3821ebb0ee9f}
  \end{wrapfigure}

Licence : \textbf{Licence Ouverte
}\newline
Créé le : 2017-09-01\newline
Modifié le : 2017-09-04\newline
Granularité : au pays\newline
Mise à jour : ponctuelle\newline
Popularité : 1 réutilisation,  0 suivi\newline
Mots-clé : \emph{boamp, boamp-fr, choix-de-formulaire, dila, formulaire-de-publication, simulateur, simulateur-boamp, simulateurs, simulateurs-dila, site-boamp
}\newline
Permalien : \url{https://data.gouv.fr/dataset/59a982a188ee3821ebb0ee9f}\newline

\par
\noindent
    La DILA diffuse, sur le site boamp.fr, les avis d'appel public à la
concurrence et les résultats de marchés de l'État, l'armée, les
collectivités territoriales et leurs établissements publics. On y trouve
également des contrats de partenariat public-privé et des avis de
concession. Une diffusion est obligatoire au BOAMP pour tous les marchés
dont le montant est supérieur aux seuils européens. En dessous de ces
seuils, une publication reste obligatoire au BOAMP ou dans un journal
habilité à recevoir des annonces légales. Pour les marchés à procédure
adaptée (MAPA) inférieurs à 90 000 \euro{} HT, l'acheteur public a le
choix des supports qu'il utilisera pour assurer la publicité de ses
marchés.

\textbf{Pour faciliter le travail des acheteurs publics,} la Dila a mis
au point 2 simulateurs publiés sur son site boamp.fr :

\begin{itemize}
\item
  Simulateur d'aide au choix des formulaires de publication au BOAMP
\item
  Simulateur d'aide au choix du forfait de publication au BOAMP
\item
\end{itemize}

\textbf{Pour les entreprises}, la Dila a conçu et publié sur ses sites
service-public.fr et boamp.fr un simulateur de calcul des intérêts
moratoires des marchés publics.

\textbf{Le simulateur d'aide au choix du formulaire} permet à l'acheteur
public de déterminer le formulaire à utiliser pour publier son avis
parmi la quarantaine de formulaires proposés par le BOAMP.

En répondant à 4 questions, il est orienté vers le formulaire
correspondant à ses attentes.

Le simulateur prend en compte les modifications apportées par les
directives européennes et les nouveaux décrets relatifs aux marchés
publics, aux marchés défense ou sécurité et aux concessions entrés en
vigueur le 1er avril 2016.

Les données du simulateur sont mises à jour en fonction des évolutions
de la règlementation.

Les simulateurs développés par la DILA utilisent le \textbf{moteur de
simulation G6K} dont les sources sont accessibles via le lien
:\url{https://github.com/eureka2/G6K.} Outre les sources, sont à
disposition via ce lien :

\begin{itemize}

\item
  La définition (étapes, règles,.), au format XSD, valable pour tous les
  simulateurs développés avec le moteur G6K ;
\item
  La procédure de mise à disposition et d'installation du moteur.
\end{itemize}

\textbf{Les données du simulateur mises à disposition par la DILA} sont
constituées d'un fichier XML pour la définition du simulateur.


\vspace{0.5cm}
\needspace{12\baselineskip}
\subsection*{Simulateur de CALCUL DE LA GRATIFICATION MINIMALE D'UN STAGIAIRE
}\index{calculer}\index{dila}\index{evaluer}\index{gratification!minimale}\index{montant!minimum}\index{paie!stagiaire}\index{remuneration!minimale}\index{remuneration!stagiaire}\index{salaire!minimum}\index{salaire!stagiaire}\index{service!public}\index{service!public!fr}\index{simulateur}\index{simulateurs}\index{simulateurs!dila}\index{simulateurs!service!public}\index{stage}\index{stagiaire}
  \begin{wrapfigure}{r}{2.5cm}
    \centering
    \qrcode[nolink]{https://data.gouv.fr/dataset/59a96c63c751df6908045f76}
  \end{wrapfigure}

Licence : \textbf{Licence Ouverte
}\newline
Créé le : 2017-09-01\newline
Modifié le : 2017-09-04\newline
Granularité : au pays\newline
Mise à jour : ponctuelle\newline
Popularité : 1 réutilisation,  0 suivi\newline
Mots-clé : \emph{calculer, dila, evaluer, gratification-minimale, montant-minimum, paie-stagiaire, remuneration-minimale, remuneration-stagiaire, salaire-minimum, salaire-stagiaire, service-public, service-public-fr, simulateur, simulateurs, simulateurs-dila, simulateurs-service-public, stage, stagiaire
}\newline
Permalien : \url{https://data.gouv.fr/dataset/59a96c63c751df6908045f76}\newline

\par
\noindent
    Le site officiel de l'administration française service-public.fr
référence une cinquantaine de simulateurs disponibles pour répondre à un
large éventail de questions administratives qui se posent aux
particuliers et aux professionnels. Certains de ces simulateurs ont été
développés par la DILA, dont le simulateur de calcul de la gratification
minimum d'un stagiaire.

Ce simulateur a été conçu en 2015, à la demande du ministère de
l'Education nationale, de l'Enseignement supérieur et de la Recherche.
Il permet aux employeurs et aux étudiants de calculer, pour tout stage
supérieur à 2 mois, le montant de la gratification minimale, rendue
obligatoire, par la loi n\degree{}2014-788 du 10 juillet 2014 ; cette
loi vise au développement, à l'encadrement des stages et à
l'amélioration du statut des stagiaires.

À partir de la date de signature de la convention de stage et du nombre
d'heures de présence effective du stagiaire dans l'organisme d'accueil,
ce simulateur permet de calculer :

\begin{itemize}

\item
  le montant de la gratification minimale due pour chaque mois du stage
  (gratification mensuelle),
\item
  le montant total de la gratification due pour toute la durée du stage
  (gratification totale),
\item
  le montant mensuel à verser en cas de lissage de la gratification sur
  la totalité de la durée du stage (gratification mensuelle lissée).
\end{itemize}

Les données du simulateur sont mises à jour en fonction des évolutions
de la règlementation.

Les simulateurs développés par la DILA utilisent le \textbf{moteur de
simulation G6K} dont les sources sont accessibles via le lien
:\url{https://github.com/eureka2/G6K.}

Outre les sources, sont à disposition via ce lien :

\begin{itemize}

\item
  La définition (étapes, règles,.), au format XSD, valable pour tous les
  simulateurs développés avec le moteur G6K ;
\item
  La procédure de mise à disposition et d'installation du moteur.
\end{itemize}

\textbf{Les données du simulateur mises à disposition par la DILA sont
constituées} :

\begin{itemize}

\item
  d'un fichier XML pour la définition du simulateur ;
\item
  d'un schéma des données au format JSON ;
\item
  des données au format JSON.
\end{itemize}


\vspace{0.5cm}
\needspace{12\baselineskip}
\subsection*{Simulateur de CALCUL DES DROITS DE SUCCESSION
}\index{calculer}\index{cout}\index{dila}\index{droits!de!succession}\index{frais!de!succession}\index{montant}\index{montant!des!droits}\index{service!public}\index{service!public!fr}\index{simulateur}\index{simulateurs}\index{simulateurs!dila}\index{simulateurs!service!public}
  \begin{wrapfigure}{r}{2.5cm}
    \centering
    \qrcode[nolink]{https://data.gouv.fr/dataset/592ee34588ee38048d74eeb4}
  \end{wrapfigure}

Licence : \textbf{Licence Ouverte
}\newline
Créé le : 2017-05-31\newline
Modifié le : 2017-09-04\newline
Mise à jour : ponctuelle\newline
Popularité : 1 réutilisation,  0 suivi\newline
Mots-clé : \emph{calculer, cout, dila, droits-de-succession, frais-de-succession, montant, montant-des-droits, service-public, service-public-fr, simulateur, simulateurs, simulateurs-dila, simulateurs-service-public
}\newline
Permalien : \url{https://data.gouv.fr/dataset/592ee34588ee38048d74eeb4}\newline

\par
\noindent
    Le site officiel de l'administration française, service-public.fr,
référence une cinquantaine de simulateurs disponibles pour répondre à un
large éventail de questions administratives qui se posent aux
particuliers et aux professionnels. Certains de ces simulateurs ont été
développés par la DILA, dont le simulateur de calcul des droits de
succession.

La DILA a conçu et mis en ligne ce simulateur sur service-public.fr en
mai 2017.

Le simulateur permet d'effectuer une estimation indicative des frais de
succession dont un usager peut être personnellement redevable suite au
décès d'un proche. Ce module de calcul permet en 4 étapes simples de
disposer d'une estimation des frais de succession (hors frais de
notaire).

Pour réaliser une simulation, l'usager doit connaître la valeur des
biens qui composent la succession et le montant de la part lui revenant,
ainsi que le montant des dettes laissées par le défunt.

Les données du simulateur sont mises à jour en fonction des évolutions
de la règlementation.

Les simulateurs développés par la DILA utilisent le \textbf{moteur de
simulation G6K} dont les sources sont accessibles via le lien
:\url{https://github.com/eureka2/G6K.} Outre les sources, sont à
disposition via ce lien :

\begin{itemize}

\item
  La définition (étapes, règles,.), au format XSD, valable pour tous les
  simulateurs développés avec le moteur G6K;
\item
  La procédure de mise à disposition et d'installation du moteur.
\end{itemize}

\textbf{Les données du simulateur mises à disposition par la DILA sont
constituées} :

\begin{itemize}

\item
  d'un fichier XML pour la définition du simulateur ;
\item
  d'un schéma des données au format JSON;
\item
  des données au format JSON.
\end{itemize}


\vspace{0.5cm}
\needspace{12\baselineskip}
\subsection*{Simulateur de CALCUL DES INTERETS MORATOIRES DES MARCHES PUBLICS
}\index{calcul!des!interets}\index{cout}\index{delais!de!paiement}\index{entreprises}\index{interets!moratoires}\index{marches}\index{marches!conclus}\index{marches!publics}\index{penalites!de!retard}\index{service!public}\index{simulateur}
  \begin{wrapfigure}{r}{2.5cm}
    \centering
    \qrcode[nolink]{https://data.gouv.fr/dataset/57fe3bc988ee3846385ff490}
  \end{wrapfigure}

Licence : \textbf{Licence Ouverte
}\newline
Créé le : 2016-10-12\newline
Modifié le : 2017-09-04\newline
Granularité : au pays\newline
Mise à jour : ponctuelle\newline
Popularité : 1 réutilisation,  0 suivi\newline
Mots-clé : \emph{calcul-des-interets, cout, delais-de-paiement, entreprises, interets-moratoires, marches, marches-conclus, marches-publics, penalites-de-retard, service-public, simulateur
}\newline
Permalien : \url{https://data.gouv.fr/dataset/57fe3bc988ee3846385ff490}\newline

\par
\noindent
    Le site officiel de l'administration française service-public.fr
référence une cinquantaine de simulateurs disponibles pour répondre à un
large éventail de questions administratives qui se posent aux
particuliers et aux professionnels. Certains ont été développés par la
DILA, dont le simulateur de calcul des intérêts moratoires ; ce
simulateur est également diffusé sur le site boamp.fr dédié aux marchés
publics.

Lors de l'exécution d'un marché public, si l'administration ne respecte
pas les délais réglementaires pour payer son fournisseur et son
sous-traitant le cas échéant, des pénalités financières sont appliquées.
Ce simulateur d'intérêts moratoires permet de calculer les pénalités de
retard de paiement dans le cadre d'un contrat régi par la réglementation
des marchés publics.

Les données du simulateur sont mises à jour en fonction des évolutions
de la règlementation, au minimum deux fois par an, en janvier et
juillet.

Les simulateurs développés par la DILA utilisent le \textbf{moteur de
simulation G6K} dont les sources sont accessibles via le lien
:\url{https://github.com/eureka2/G6K.} Outre les sources, sont à
disposition via ce lien :

\begin{itemize}

\item
  La définition (étapes, règles,.), au format XSD, valable pour tous les
  simulateurs développés avec le moteur G6K ;
\item
  La procédure de mise à disposition et d'installation du moteur.
\end{itemize}

\textbf{Les données du simulateur mises à disposition par la DILA sont
constituées :}

\begin{itemize}

\item
  d'un fichier XML pour la définition du simulateur ;
\item
  d'un schéma des données au format JSON ;
\item
  des données au format JSON.
\end{itemize}


\vspace{0.5cm}
\needspace{12\baselineskip}
\subsection*{Simulateur pour CONNAITRE LA ZONE DE SA COMMUNE : Abis, A, B1, B2 ou C
}\index{commune}\index{dila}\index{logement}\index{simulateur}\index{simulateurs!dila}\index{zonage!immobilier}\index{zone}\index{zone!geographique}
  \begin{wrapfigure}{r}{2.5cm}
    \centering
    \qrcode[nolink]{https://data.gouv.fr/dataset/5bc5fe4f8b4c413c0d4fff87}
  \end{wrapfigure}

Licence : \textbf{Licence Ouverte
}\newline
Créé le : 2018-10-16\newline
Modifié le : 2018-10-19\newline
Granularité : au département\newline
Mise à jour : ponctuelle\newline
Popularité : 1 réutilisation,  0 suivi\newline
Mots-clé : \emph{commune, dila, logement, simulateur, simulateurs-dila, zonage-immobilier, zone, zone-geographique
}\newline
Permalien : \url{https://data.gouv.fr/dataset/5bc5fe4f8b4c413c0d4fff87}\newline

\par
\noindent
    Le site officiel de l'administration française
\href{https://www.service-public.fr}{service-public.fr} référence une
soixantaine de simulateurs disponibles pour répondre à un large éventail
de questions administratives qui se posent aux particuliers et aux
professionnels. Certains de ces simulateurs ont été développés par la
Direction de l'information légale et administrative (DILA), dont le
\href{https://www.service-public.fr/particuliers/vosdroits/R46110}{simulateur
pour Connaître la zone de sa commune : Abis, A, B1, B2 ou C}.

Ce simulateur permet de trouver à quelle zone géographique appartient
une commune (zonage immobilier A, Abis, B1, B2 ou C) dont dépend le
logement concerné.

La zone détermine :

• le revenu maximum pour avoir droit à un logement social ; • le revenu
maximum pour avoir droit au prêt à taux zéro ou prêt d'accession sociale
; • pour un bailleur, le droit à une réduction d'impôt ; • pour le
bailleur d'un logement conventionné avec l'Anah ; • le droit à une
déduction fiscale sur les revenus fonciers ; • le revenu maximum du
candidat locataire ; • le loyer initial maximum.

Les données du simulateur sont mises à jour en fonction des évolutions
de la règlementation.

Les simulateurs développés par la DILA utilisent le moteur de simulation
G6K dont les sources sont accessibles via le lien
:\url{https://github.com/eureka2/G6K} Outre les sources, sont à
disposition via ce lien :

\begin{itemize}

\item
  La définition (étapes, règles,.), au format XSD, valable pour tous les
  simulateurs développés avec le moteur G6K ;
\item
  La procédure de mise à disposition et d'installation du moteur.
\end{itemize}

Les données du simulateur mises à disposition par la DILA sont
constituées :

\begin{itemize}

\item
  d'une présentation du simulateur (PDF) ;
\item
  des règles de calcul du simulateur (XML) ;
\item
  d'une feuille de style (CSS) ;
\item
  des données de référence (JSON) ;
\item
  d'un schéma des données de référence (Schéma JSON).
\end{itemize}


\vspace{0.5cm}
\needspace{12\baselineskip}
\subsection*{Territoires 2040, n\degree{}4 - Des systèmes spatiaux en prospective
}
  \begin{wrapfigure}{r}{2.5cm}
    \centering
    \qrcode[nolink]{https://data.gouv.fr/dataset/5369a222a3a729239d20677e}
  \end{wrapfigure}

Licence : \textbf{Licence Ouverte
}\newline
Créé le : 2013-07-08\newline
Modifié le : 2015-03-23\newline
Popularité : 1 réutilisation,  0 suivi\newline
Mots-clé : \emph{aucun
}\newline
Permalien : \url{https://data.gouv.fr/dataset/5369a222a3a729239d20677e}\newline

\par
\noindent
    Revue ayant pour vocation de diffuser les travaux produits dans le cadre
du programme de prospective de la Datar ``Territoires 2040, aménager le
changement''


\vspace{0.5cm}
\needspace{12\baselineskip}
\subsection*{Thésaurus Information Publique - vie-publique.fr
}\index{documentation}\index{rdf}\index{referentiel}\index{skos}\index{thesaurus}\index{vie!publiquefr}\index{web!semantique}
  \begin{wrapfigure}{r}{2.5cm}
    \centering
    \qrcode[nolink]{https://data.gouv.fr/dataset/5600134d88ee381b24ccb97b}
  \end{wrapfigure}

Licence : \textbf{Licence Ouverte
}\newline
Créé le : 2015-09-21\newline
Modifié le : 2016-01-14\newline
De 2014-12-01 à 2025-12-31\newline
Mise à jour : quotienne\newline
Popularité : 3 réutilisations,  1 suivi\newline
Mots-clé : \emph{documentation, rdf, referentiel, skos, thesaurus, vie-publiquefr, web-semantique
}\newline
Permalien : \url{https://data.gouv.fr/dataset/5600134d88ee381b24ccb97b}\newline

\par
\noindent
    Le référentiel d'indexation ``Thésaurus Information Publique'' géré par
la Dila, offre un arbre sémantique permettant de couvrir les différents
domaines de l'information publique et des politiques publiques. Il est
accessible au format sémantique SKOS. Tous les textes, discours,
interviews, communiqués, conférences de presse\ldots{}, recensés dans la
Collection des discours publics, font l'objet d'une indexation : des
mots-clés qui rendent compte du contenu sont sélectionnés par les
documentalistes du Département de l'Information publique dans une liste
commune de mots-clés autorisés. Cette liste est normalisée et structurée
: les mots-clés sont classés par domaine thématique et hiérarchisés du
plus générique au plus précis. Ce Thesaurus régulièrement actualisé
comprend environ 6000 termes répartis en 26 domaines thématiques. Il
comprend en outre des listes annexes dans lesquelles on trouve des noms
propres ou des mots outils. Les mots outils sont destinés à être
utilisés avec des mots-clés ou des noms propres. Les domaines
thématiques sont très structurés et descendent jusqu'à cinq niveaux de
termes spécifiques. Les relations utilisées sont les suivantes : TS :
terme spécifique EP : employé pour (relation de synonymie). Ce thésaurus
est édité et maintenu par la Direction de l'information légale et
administrative - vie-publique.fr.


\vspace{0.5cm}
\needspace{12\baselineskip}
\subsection*{Zone de revitalisation rurale - ZRR
}
  \begin{wrapfigure}{r}{2.5cm}
    \centering
    \qrcode[nolink]{https://data.gouv.fr/dataset/5369a397a3a729239d206ad1}
  \end{wrapfigure}

Licence : \textbf{Licence Ouverte
}\newline
Créé le : 2013-07-08\newline
Modifié le : 2016-03-12\newline
Mise à jour : ponctuelle\newline
Popularité : 1 réutilisation,  1 suivi\newline
Mots-clé : \emph{aucun
}\newline
Permalien : \url{https://data.gouv.fr/dataset/5369a397a3a729239d206ad1}\newline

\par
\noindent
    Les zones de revitalisation rurale (ZRR) visent à aider le développement
des territoires ruraux principalement à travers des mesures fiscales et
sociales.Des mesures spécifiques en faveur du développement économique
s'y appliquent. L'objectif est de concentrer les mesures d'aide de
l'état au bénéfice des entreprises créatrices d'emplois dans les zones
rurales les moins peuplées et les plus touchées par le déclin
démographique et économique.Elles ont été créées par la loi
d'Orientation pour l'Aménagement et le Développement du Territoire
(LOADT) du 4 février 1995. Le CIADT du 3 septembre 2003 a défini de
nouvelles orientations pour adapter cet outil aux besoins actuels. Les
dispositions correspondantes sont inscrites dans la loi relative au
développement des territoires ruraux du 23 février 2005 et dans le
décret n\degree{} 2005-1435 du 21 novembre 2005.La liste constatant le
classement des communes en ZRR est établie et révisée chaque année par
arrêté du Premier ministre en fonction des créations, suppressions et
modifications de périmètres des EPCI à fiscalité propre constatées au 31
décembre de l'année précédente.


\vspace{0.5cm}
\needspace{3\baselineskip} \rule{4cm}{0.25pt}\newline\textbf{Aussi disponible du même producteur :}\begin{itemize}
\item \href{https://data.gouv.fr/dataset/5a0b0be188ee3871db07ce4e}{ACCO : Accords d’entreprise}
\item \href{https://data.gouv.fr/dataset/53698e5aa3a729239d20343b}{Actes du séminaire Prospective info "l'aménagement du territoire à l'international"}
\item \href{https://data.gouv.fr/dataset/53698e5aa3a729239d20343c}{Actes du séminaire Prospective Info "La prospective territoriale en France : état des lieux et perspective"}
\item \href{https://data.gouv.fr/dataset/53698e5aa3a729239d20343d}{Actes du séminaire Prospective info "L'économie des services, moteur de développement durable pour les territoires"}
\item \href{https://data.gouv.fr/dataset/53698e5ba3a729239d20343e}{Actes du séminaire Prospective Info "Territoires 2040, Des systèmes spatiaux à l'heure du changement"}
\item \href{https://data.gouv.fr/dataset/53698e64a3a729239d203456}{Activité de la Miviludes en 2010 : affectation des courriers par pôles spécialisés}
\item \href{https://data.gouv.fr/dataset/53698e64a3a729239d203457}{Activité de la Miviludes : nombre d'appels reçus en 2010}
\item \href{https://data.gouv.fr/dataset/53698e65a3a729239d203458}{Activité de la Miviludes : nombre et types de courriers reçus en 2010}
\item \href{https://data.gouv.fr/dataset/53698e65a3a729239d203459}{Activité de la Miviludes : profil des émetteurs des courriers reçus en 2010}
\item \href{https://data.gouv.fr/dataset/587ccbcd88ee382c669b81a4}{API BOAMP ( bêta)}
\item \href{https://data.gouv.fr/dataset/53698f43a3a729239d2036c4}{Baromètre de la qualité de service}
\item \href{https://data.gouv.fr/dataset/53698f5da3a729239d20371d}{Bassin parisien : l'offre d'enseignement supérieur et de recherche face aux besoins de l'économie et de l'emploi}
\item \href{https://data.gouv.fr/dataset/5a5e2039c751df7306b4dd6f}{Bibliothèque des Rapports Publics diffusée par la Documentation Française}
\item \href{https://data.gouv.fr/dataset/5369909da3a729239d203a4e}{Chocs démographiques et technologiques : quels impacts sur le développement des territoires ?}
\item \href{https://data.gouv.fr/dataset/536990afa3a729239d203a7c}{Chronique de la politique d’immigration en 2007 - collection politiques publiques}
\item \href{https://data.gouv.fr/dataset/536990afa3a729239d203a7d}{Chronique de la politique judiciaire en 2007 - collection politiques publiques}
\item \href{https://data.gouv.fr/dataset/53699109a3a729239d203b60}{Commission pour l'indemnisation des victimes de spoliations (CIVS) - Chiffres clés}
\item \href{https://data.gouv.fr/dataset/53699109a3a729239d203b61}{Commission pour l'indemnisation des victimes de spoliations (CIVS) - Statistiques mensuelles 2013}
\item \href{https://data.gouv.fr/dataset/5369910aa3a729239d203b62}{Commission pour l'indemnisation des victimes de spoliations (CIVS) - Statistiques mensuelles 2014}
\item \href{https://data.gouv.fr/dataset/54f97f63c751df1ffe882844}{Commission pour l'indemnisation des victimes de spoliations (CIVS) - Statistiques mensuelles 2015}
\item \href{https://data.gouv.fr/dataset/56bdf179c751df36b49f9dea}{Commission pour l'indemnisation des victimes de spoliations (CIVS) - Statistiques pour l'année 2016}
\item \href{https://data.gouv.fr/dataset/58b02846c751df5c5d8fb41d}{Commission pour l'indemnisation des victimes de spoliations (CIVS) - Statistiques pour l'année 2017}
\item \href{https://data.gouv.fr/dataset/5a86e2f288ee3822688839e6}{Commission pour l'indemnisation des victimes de spoliations (CIVS) - Statistiques pour l'année 2018}
\item \href{https://data.gouv.fr/dataset/53699be9a3a729239d205856}{Composition communale des  pays}
\item \href{https://data.gouv.fr/dataset/5369912da3a729239d203bbe}{Composition des gouvernements provisoires suivant la Libération et des gouvernements de la 4ème République (1944-1959)}
\item \href{https://data.gouv.fr/dataset/5369917aa3a729239d203c86}{Comptes rendus des Débats de l'Assemblée nationale}
\item \href{https://data.gouv.fr/dataset/5369917aa3a729239d203c88}{Conception et analyse d'indicateurs stratégiques de l'innovation dans les territoires}
\item \href{https://data.gouv.fr/dataset/53699211a3a729239d203e13}{Datar mode d'emploi}
\item \href{https://data.gouv.fr/dataset/536992cea3a729239d204007}{Déploiement des réseaux très haut débit sur l'ensemble du territoire national}
\item \href{https://data.gouv.fr/dataset/53d6e509a3a72954d9dd62f2}{Développement agricole - enquête auprès des chefs d'exploitation}
\item \href{https://data.gouv.fr/dataset/5bc8705b8b4c414453fe08da}{DOLE : les dossiers législatifs}
\item \href{https://data.gouv.fr/dataset/56698addc751df0c34c664bc}{Enquête sur l'engagement citoyen international des jeunes}
\item \href{https://data.gouv.fr/dataset/53699500a3a729239d2045b2}{Etat des lieux de huit coopérations dans sept métropoles. Évaluation de l'appel à coopération métropolitaine}
\item \href{https://data.gouv.fr/dataset/53699503a3a729239d2045b9}{Etude 2008 Evénement de vie : Associations}
\item \href{https://data.gouv.fr/dataset/53699504a3a729239d2045ba}{Etude 2008 Evénement de vie : Collectivités}
\item \href{https://data.gouv.fr/dataset/53699504a3a729239d2045bb}{Etude 2008 Evénement de vie : Entreprises}
\item \href{https://data.gouv.fr/dataset/53699506a3a729239d2045bc}{Etude 2008 Evénement de vie : Particuliers}
\item \href{https://data.gouv.fr/dataset/53699506a3a729239d2045bd}{Etude 2010 Evénement de vie : Entreprises}
\item \href{https://data.gouv.fr/dataset/53699507a3a729239d2045be}{Etude 2010 Evénement de vie : Particuliers}
\item \href{https://data.gouv.fr/dataset/53699507a3a729239d2045c0}{Etude 2012 Evénement de vie : Particuliers}
\item \href{https://data.gouv.fr/dataset/53ba4d68a3a729219b7bead6}{Etude auprès de jeunes raccrocheurs et de leurs parents}
\item \href{https://data.gouv.fr/dataset/53699508a3a729239d2045c1}{Etude bibliographique du marché des téléactivités pour les entreprises et les particuliers}
\item \href{https://data.gouv.fr/dataset/53699515a3a729239d204614}{Evaluation à mi-parcours des CPER 2007-2013 / Volet enseignement supérieur et recherche}
\item \href{https://data.gouv.fr/dataset/53699527a3a729239d204642}{Evaluation des contrats d'agglomération (synthèse)}
\item \href{https://data.gouv.fr/dataset/53699527a3a729239d204643}{Evaluation des démarches contractuelles de Pays}
\item \href{https://data.gouv.fr/dataset/53699528a3a729239d204644}{Evaluation des systèmes productifs locaux}
\item \href{https://data.gouv.fr/dataset/53699528a3a729239d204645}{Evaluation nationale du volet ferroviaire et transport en commun en site propre (TCSP) des CPER 2007-2013}
\item \href{https://data.gouv.fr/dataset/536995c7a3a729239d2047fc}{Formes de mutualisation des services publics en milieu rural}
\item \href{https://data.gouv.fr/dataset/53699602a3a729239d2048a6}{Grappes d’entreprises}
\item \href{https://data.gouv.fr/dataset/5369970fa3a729239d204bca}{Ingénierie du développement territorial : dynamisme et enjeux économiques d'un secteur d'activité}
\item et 71 autres jeux de données\end{itemize}

\clearpage
\section{Régie autonome des transports parisiens (RATP)}


\begin{center}
  \includegraphics[width=3cm]{images/orga/91_5b9234998a4a288198a725129b4b0b-100.jpg}
\end{center}


Le Groupe RATP est le cinquième acteur mondial du transport public.
Associé depuis toujours à Paris et à son métro centenaire, le plus dense
au monde, le groupe RATP est fort d'une expérience unique dans la
conception, la gestion des projets, l'exploitation et la maintenance de
tous les modes de transports publics urbains et interurbains :
ferroviaire, métro, tramway et bus. Il assure quotidiennement la
mobilité de 12 millions de personnes en France et dans le monde. Avec
ses 14 lignes de métro, ses deux lignes de RER (lignes A et B), ses 6
lignes de tramway (T1, T2, T3a, T3b, T5, T7), ses 350 lignes de bus et
ses services de navette en direction de deux aéroports de la région
parisienne, le réseau multimodal exploité par la RATP en Ile-de-France
est l'un des plus importants au monde.


\vspace{0.5cm}

\needspace{12\baselineskip}
\subsection*{Trafic annuel entrant par station (2013)
}\index{voyageurs}
  \begin{wrapfigure}{r}{2.5cm}
    \centering
    \qrcode[nolink]{https://data.gouv.fr/dataset/5369a24ba3a729239d2067e8}
  \end{wrapfigure}

Licence : \textbf{Licence Ouverte
}\newline
Créé le : 2013-10-28\newline
Modifié le : 2016-03-09\newline
Granularité : au point d'intérêt\newline
Mise à jour : annuelle\newline
Popularité : 5 réutilisations,  0 suivi\newline
Mots-clé : \emph{voyageurs
}\newline
Permalien : \url{https://data.gouv.fr/dataset/5369a24ba3a729239d2067e8}\newline

\par
\noindent
    Ce jeu de donnée détaille le trafic des entrants directs sur le réseau
ferré RATP.


\vspace{0.5cm}
\needspace{12\baselineskip}
\subsection*{Trafic annuel entrant par station (2013)
}\index{voyageurs}
  \begin{wrapfigure}{r}{2.5cm}
    \centering
    \qrcode[nolink]{https://data.gouv.fr/dataset/538947c1a3a7291ffe8c5037}
  \end{wrapfigure}

Licence : \textbf{Licence Ouverte
}\newline
Créé le : 2014-05-30\newline
Modifié le : 2015-12-28\newline
Mise à jour : annuelle\newline
Popularité : 1 réutilisation,  1 suivi\newline
Mots-clé : \emph{voyageurs
}\newline
Permalien : \url{https://data.gouv.fr/dataset/538947c1a3a7291ffe8c5037}\newline

\par
\noindent
    Ce jeu de donnée détaille le trafic des entrants directs sur le réseau
ferré RATP.


\vspace{0.5cm}
\needspace{3\baselineskip} \rule{4cm}{0.25pt}\newline\textbf{Aussi disponible du même producteur :}\begin{itemize}
\item \href{https://data.gouv.fr/dataset/53698e47a3a729239d20340c}{Accessibilité des arrêts de bus RATP}
\item \href{https://data.gouv.fr/dataset/536991c4a3a729239d203d49}{Coordonnées des stations sur le plan schématique RATP}
\item \href{https://data.gouv.fr/dataset/536996cea3a729239d204b2a}{Indices des lignes de bus du réseau RATP}
\item \href{https://data.gouv.fr/dataset/53699782a3a729239d204cef}{Jeux de données - Régie autonome des transports parisiens (RATP)}
\item \href{https://data.gouv.fr/dataset/53699cbea3a729239d205a51}{Plan RATP schématique Île-de-France}
\item \href{https://data.gouv.fr/dataset/563d99bea3a729734ab81571}{Qualité de l'air mesurée dans la station Auber}
\item \href{https://data.gouv.fr/dataset/53699e77a3a729239d205ea6}{Qualité de l'air mesurée dans nos stations (T1 2012)}
\item \href{https://data.gouv.fr/dataset/563d99beb595084ae95aba7c}{Sanisettes Ville de Paris - Données géographiques}
\item \href{https://data.gouv.fr/dataset/5959371ca3a7291dd09c8278}{Trafic annuel entrant par station du réseau ferré 2015}
\end{itemize}

\clearpage
\section{Région Bourgogne-Franche-Comté}


\begin{center}
  \includegraphics[width=3cm]{images/orga/6d_a3354add0e4241b5d77737bb6c11cb-100.jpg}
\end{center}


Avec sa frontière commune avec la Suisse, l'Ile-de-France, le Centre-Val
de Loire, la région Alsace -Champagne-Ardenne-Lorraine et
l'Auvergne-Rhône-Alpes, la région Bourgogne-Franche-Comté, avec ses
2,821 millions d'habitants et ses 14 gares TGV, bénéficie d'une place
privilégiée au milieu de l'Europe.

\begin{verbatim}
Population en 2013 : 2,821 millions
Superficie : 47 784 km\textsuperscript{2}
Population active 2013 : 1 287 926 personnes
8 départements : Côte-d’Or, Doubs, Haute-Saône, Jura, Nièvre, Saône-et-Loire, Territoire de Belfort, Yonne
Communes : 3 831, dont 25 communes de plus de 10 000 habitants
Densité : 59 habitants au km\textsuperscript{2}
\end{verbatim}

\begin{itemize}
\item
  14 gares desservies par le TGV
\item
  451 kilomètres de Ligne à Grande Vitesse
\item
  868 kilomètres d'autoroutes
\item
  1 951 kilomètres de ligne TER
\item
  230 kilomètres de frontière avec la Suisse

  Avec la loi MAPTAM, La région va coordonner l'action de toutes les
  collectivités dans plusieurs domaines : aménagement durable,
  biodiversité, climat, qualité de l'air et de l'énergie, enseignement
  supérieur et recherche, développement économique, innovation et
  intermodalité des transports.

  Avec la loi NOTRe, La région voit ses pouvoirs renforcés en matière
  économique, de transport et d'aménagement du territoire. La région
  devient seule compétente sur les aides aux entreprises. Elle hérite de
  la gestion des transports interurbains (cars hors agglomération) et
  des transports scolaires, jusque-là du ressort des départements. Côté
  aménagement du territoire, les orientations du schéma qu'elle devra
  produire devront être respectées par les collectivités, dans leurs
  documents d'urbanisme, comme le plan local d'urbanisme, le plan local
  de l'habitat, etc.
\end{itemize}


\vspace{0.5cm}

\needspace{12\baselineskip}
\subsection*{Marchés publics (expérimentation OpenDataLocale) de la région
Bourgogne-Franche-Comté
}\index{commande!publique}\index{marche!public}\index{marches!attribues}\index{marches!publics}
  \begin{wrapfigure}{r}{2.5cm}
    \centering
    \qrcode[nolink]{https://data.gouv.fr/dataset/5a0effeac751df5832ded865}
  \end{wrapfigure}

Licence : \textbf{Licence Ouverte
}\newline
Créé le : 2017-11-17\newline
Modifié le : 2019-02-22\newline
De 2016-01-01 à 2018-04-30\newline
Granularité : à la région\newline
Mise à jour : ponctuelle\newline
Popularité : 1 réutilisation,  1 suivi\newline
Mots-clé : \emph{commande-publique, marche-public, marches-attribues, marches-publics
}\newline
Permalien : \url{https://data.gouv.fr/dataset/5a0effeac751df5832ded865}\newline

\par
\noindent
    Vous êtes sur le point de consulter les données des contrats relatifs
aux marchés passés sur le profil acheteur e-bourgogne-franche-comté par
la Région Bourgogne-Franche-Comté depuis le début 2016.

\textbf{La mise à disposition de ces données s'opère dans le cadre d'une
expérimentation ayant pour objectif d'anticiper l'échéance règlementaire
du 1er octobre 2018} relative aux données essentielles des marchés
publics. Il doit être noté à ce titre qu'en attendant la mise à jour de
la plateforme, actuellement la saisie des champs « contrat » n'est pas
obligatoire et dépend des pratiques internes de chaque organisme.

Ainsi, \textbf{il manque encore certaines informations} (à titre
d'exemple : le lieu d'exécution) pour être totalement conforme à
l'article 107 du décret relatif aux marchés publics. De même,
\textbf{tous les contrats des expérimentateurs pour la période
concernées n'y sont pas encore référencés.} Ces informations ne sauront
donc pas être interprétées comme une représentation fidèle de la
commande publique, au niveau local comme au niveau régional. En cas
d'interrogation, vous pourrez vous rapprocher de l'organisme producteur
des données pour obtenir des explications complémentaires.

Cependant, cette expérimentation vise précisément à identifier et à
anticiper les besoins en termes d'évolution des pratiques de saisie et
permet également de communiquer avant l'ouverture des données relatives
aux marchés publics. En attendant l'arrivée des données essentielles
(Octobre 2018), cette extraction sera mise à jour mensuellement pour
intégrer d'éventuelles corrections et de nouveaux contrats conclus.


\vspace{0.5cm}
\needspace{3\baselineskip} \rule{4cm}{0.25pt}\newline\textbf{Aussi disponible du même producteur :}\begin{itemize}
\item \href{https://data.gouv.fr/dataset/59bbbea988ee387f8e130a43}{Bâtiments de la région Bourgogne-Franche-Comté}
\item \href{https://data.gouv.fr/dataset/59bbbea9c751df4fbf7a1a0f}{Bourg centre 2016}
\item \href{https://data.gouv.fr/dataset/5afee2bfc751df5fa97be309}{Centres de Formation d'Apprentis (CFA) en Bourgogne-Franche-Comté}
\item \href{https://data.gouv.fr/dataset/5b47538b88ee384734866b22}{Gares de Bourgogne-Franche-Comté}
\item \href{https://data.gouv.fr/dataset/5b47538ac751df42667b278b}{Itinéraires touristiques de randonnée d'intérêt régional}
\item \href{https://data.gouv.fr/dataset/5afee2c6c751df5fb1475450}{Itinéraires touristiques équestres d'intérêt régional}
\item \href{https://data.gouv.fr/dataset/5b47538a88ee38472d660f0d}{Itinéraires touristiques nordiques d'intérêt régional}
\item \href{https://data.gouv.fr/dataset/55812b6fc751df598a1d4c01}{les Schémas de Cohérence Territorial (SCOT) en Bourgogne en 2013}
\item \href{https://data.gouv.fr/dataset/55812b7388ee38294df4b5a1}{les zones d'emploi en Bourgogne}
\item \href{https://data.gouv.fr/dataset/58ef7b3388ee386a03cd22d5}{Lycées publics de Bourgogne-Franche-Comté}
\item \href{https://data.gouv.fr/dataset/59bbbea8c751df516ddd96bb}{Massifs en Bourgogne Franche Comté}
\item \href{https://data.gouv.fr/dataset/55812b6dc751df32f51d4bfd}{Pays en Bourgogne en 2012}
\item \href{https://data.gouv.fr/dataset/5b47538ac751df4253841f3b}{Réseau ferré régional en Bourgogne-Franche-Comté}
\item \href{https://data.gouv.fr/dataset/59bbbea888ee3802a17984c9}{Scènes nationales labellisées en 2015}
\item \href{https://data.gouv.fr/dataset/59bbbea888ee3879e1598f69}{Territoires de contractualisation en région Bourgogne-Franche-Comté au 01/03/2018}
\item \href{https://data.gouv.fr/dataset/5afee2bb88ee382cad16e8ac}{Véloroutes voies vertes d'intérêt régional en Bourgogne-Franche-Comté}
\item \href{https://data.gouv.fr/dataset/55812baf88ee38240ef4b5a7}{véloroutes voies vertes en Bourgogne}
\item \href{https://data.gouv.fr/dataset/5b47538a88ee38466a23b642}{Voies navigables de Bourgogne-Franche-Comté}
\end{itemize}

\clearpage
\section{Région Bretagne}


\begin{center}
  \includegraphics[width=3cm]{images/orga/7a_5e61561c444af1b8bb0c76547f0043-100.jpg}
\end{center}


Données géographiques opendata de la Région Bretagne


\vspace{0.5cm}

\needspace{12\baselineskip}
\subsection*{Base Sirene des entreprises et de leurs établissements en Bretagne
}\index{bretagne}\index{donnees!ouvertes}\index{economie}\index{economy}\index{entreprise}\index{passerelle!inspire}
  \begin{wrapfigure}{r}{2.5cm}
    \centering
    \qrcode[nolink]{https://data.gouv.fr/dataset/5971fbe588ee38522976e8f0}
  \end{wrapfigure}

Licence : \textbf{Licence Ouverte version 2.0
}\newline
Créé le : 2017-07-21\newline
Modifié le : 2019-02-08\newline
Popularité : 1 réutilisation,  0 suivi\newline
Mots-clé : \emph{bretagne, donnees-ouvertes, economie, economy, entreprise, passerelle-inspire
}\newline
Permalien : \url{https://data.gouv.fr/dataset/5971fbe588ee38522976e8f0}\newline

\par
\noindent
    Vous trouverez plus d'informations sur le lien suivant
:\url{https://www.data.gouv.fr/fr/datasets/base-sirene-des-entreprises-et-de-leurs-etablissements-siren-siret/}
\textbf{Origine}

Téléchargement de la base SIRENE géocodée par Christian QUEST. Le
fichier stock a été géocodé à l'aide de la BAN et de BANO (voir scripts
sur\href{https://github.com/cquest/geocodage-sirene}{https://github.com/cquest/geocodage-sirene).}.)Il
est séparé en un fichier par département et pour chaque arrondissement
parisien. Les champs ajoutés sont: longitude (en degrés décimaux, WGS84)
latitude (en dégrés décimaux, WGS84) geo\_score : indice de similarité
fournit par le moteur de géocodage geo\_type : ``housenumber'' =
n\degree{} trouvé, ``interpolation'' = n\degree{} interpolé, ``street''
= voie trouvée, ``locality'' = lieu-dit (ou position de la mairie) pour
les adresses indiquées ``MAIRIE'' ou ``HOTEL DE VILLE'',
``municipality'' = position de la commune car l'adresse n'a pas été
trouvée. geo\_adresse : libellé de l'adresse trouvée geo\_id : id dans
le référentiel BAN, ou BANO (si commence par ``BANO\_'') geo\_ligne :
ligne d'adresse géocodée (G = géographique, N = normalisée, D =
déclarée) Merci de signaler les problèmes liés au géocodage
sur\url{https://github.com/cquest/geocodage-sirene/issues}

\textbf{Organisations partenaires}

Région Bretagne, Institut national de la statistique et des études
économiques

➞
\href{https://geo.data.gouv.fr/fr/datasets/d9a12656af646e4d2f1b2e7568c98aba932adf60}{Consulter
cette fiche sur geo.data.gouv.fr}


\vspace{0.5cm}
\needspace{12\baselineskip}
\subsection*{Fab Labs \& espaces de dissémination des usages numériques en Bretagne
}\index{bretagne}\index{cantines!numeriques}\index{donnees!ouvertes}\index{espaces!de!coworking}\index{fablab}\index{french!tech}\index{hackerspaces}\index{loire!atlantique}\index{makerspaces}\index{passerelle!inspire}\index{tiers!lieux}\index{utilities!communication}
  \begin{wrapfigure}{r}{2.5cm}
    \centering
    \qrcode[nolink]{https://data.gouv.fr/dataset/575962aa88ee3872a6640390}
  \end{wrapfigure}

Licence : \textbf{Licence Ouverte version 2.0
}\newline
Créé le : 2016-06-09\newline
Modifié le : 2019-02-08\newline
Popularité : 1 réutilisation,  0 suivi\newline
Mots-clé : \emph{bretagne, cantines-numeriques, donnees-ouvertes, espaces-de-coworking, fablab, french-tech, hackerspaces, loire-atlantique, makerspaces, passerelle-inspire, tiers-lieux, utilities-communication
}\newline
Permalien : \url{https://data.gouv.fr/dataset/575962aa88ee3872a6640390}\newline

\par
\noindent
    Cette carte recense les laboratoires de fabrication et lieux de
dissémination des usages numériques en Bretagne et Loire Atlantique :
Fab Labs, Hackerspaces, Makerspaces, Tiers-Lieux, Cantines numériques ou
espaces de coworking, French Tech, Associations de développement des
usages, Pôles de formations ou de recherche, etc.

\textbf{Origine}

Localisation via OpenStreetMap. La donnée est récupérée depuis github
:\url{https://github.com/grouan/bzh_fablab/}puis publiée en flux normés
OGC grâce au GeoServer de la région Bretagne.

\textbf{Organisations partenaires}

Région Bretagne, Guillaume Rouan

\textbf{Liens annexes}

\begin{itemize}

\item
  \href{http://guillaume-rouan.net/blog/2015/10/10/carte-des-fablab-de-bretagne/}{Source}
\item
  \href{http://kartenn.region-bretagne.fr/sviewer/?layers=rb:fablab}{Visualiseur
  simple}
\end{itemize}

➞
\href{https://geo.data.gouv.fr/fr/datasets/206d5e788a9c5854625af80f88c38f5f0a362bb8}{Consulter
cette fiche sur geo.data.gouv.fr}


\vspace{0.5cm}
\needspace{12\baselineskip}
\subsection*{Inventaire du patrimoine architectural breton
}\index{architecture}\index{batiments}\index{bibliographie}\index{bretagne}\index{datation}\index{designation}\index{donnees!ouvertes}\index{illustration}\index{inventaire}\index{localisation}\index{passerelle!inspire}\index{patrimoine}\index{patrimoine!culturel}\index{planning!cadastre}
  \begin{wrapfigure}{r}{2.5cm}
    \centering
    \qrcode[nolink]{https://data.gouv.fr/dataset/5846bac188ee3841cbc65bb3}
  \end{wrapfigure}

Licence : \textbf{Licence Ouverte version 2.0
}\newline
Créé le : 2016-12-06\newline
Modifié le : 2019-02-27\newline
Popularité : 1 réutilisation,  0 suivi\newline
Mots-clé : \emph{architecture, batiments, bibliographie, bretagne, datation, designation, donnees-ouvertes, illustration, inventaire, localisation, passerelle-inspire, patrimoine, patrimoine-culturel, planning-cadastre
}\newline
Permalien : \url{https://data.gouv.fr/dataset/5846bac188ee3841cbc65bb3}\newline

\par
\noindent
    Le corpus, restitué sous forme d'entités points et centroïdes de
polygones, réunit les édifices et objets mobiliers étudiés par le
Service de l'Inventaire du Patrimoine depuis 1997. Il contient également
quelques dossiers numérisés issus d'études antérieures. Les dossiers ont
été créés selon la méthodologie de l'Inventaire général.

\textbf{Origine}

Données historiquement acquises sur Scan 25 ou sur des feuilles
cadastrales géoréférencées par nos soins. Désormais, l'acquisition à
partir du cadastre numérique fourni par la Direction Générale des Impôts
est la règle.

\textbf{Organisations partenaires}

Région Bretagne

\textbf{Liens annexes}

\begin{itemize}

\item
  \href{http://kartenn.region-bretagne.fr/patrimoine}{Visualiseur donnée
  patrimoine}
\end{itemize}

➞
\href{https://geo.data.gouv.fr/fr/datasets/73a90c3ca69dec14ee722e59d65e39f4e61b1656}{Consulter
cette fiche sur geo.data.gouv.fr}


\vspace{0.5cm}
\needspace{12\baselineskip}
\subsection*{Inventaire du patrimoine breton : dossiers `papier'
}\index{architecture}\index{batiments}\index{bibliographie}\index{bretagne}\index{datation}\index{designation}\index{donnees!ouvertes}\index{illustration}\index{inventaire}\index{localisation}\index{objets!mobiliers}\index{passerelle!inspire}\index{patrimoine}\index{patrimoine!culturel}\index{planning!cadastre}
  \begin{wrapfigure}{r}{2.5cm}
    \centering
    \qrcode[nolink]{https://data.gouv.fr/dataset/57fb91b1c751df297b79df72}
  \end{wrapfigure}

Licence : \textbf{Licence Ouverte version 2.0
}\newline
Créé le : 2016-10-10\newline
Modifié le : 2019-02-08\newline
Popularité : 1 réutilisation,  0 suivi\newline
Mots-clé : \emph{architecture, batiments, bibliographie, bretagne, datation, designation, donnees-ouvertes, illustration, inventaire, localisation, objets-mobiliers, passerelle-inspire, patrimoine, patrimoine-culturel, planning-cadastre
}\newline
Permalien : \url{https://data.gouv.fr/dataset/57fb91b1c751df297b79df72}\newline

\par
\noindent
    Le corpus, restitué sous forme de détourages des communes, réunit les
enquêtes réalisées par le Service de l'Inventaire du patrimoine avant
2000. Ces dossiers ont été créés selon la méthodologie de l'Inventaire
général (http://www.inventaire.culture.gouv.fr/).

\textbf{Origine}

Données historiquement acquises sur Scan 25 ou sur des feuilles
cadastrales géoréférencées par nos soins. Désormais, l'acquisition à
partir du cadastre numérique fourni par la Direction Générale des Impôts
est la règle. La donnée est symbolisée sous forme d'un point qui est le
centroïde de l'entité si celle-ci est un polygone ou une polyligne.

\textbf{Organisations partenaires}

Région Bretagne

\textbf{Liens annexes}

\begin{itemize}

\item
  \href{http://kartenn.region-bretagne.fr/sviewer/?layers=rb:inventaire_papiers}{Visualiseur
  simple}
\end{itemize}

➞
\href{https://geo.data.gouv.fr/fr/datasets/04d3b2ecfbdfa8035c244d4094d21c8a35a4eacf}{Consulter
cette fiche sur geo.data.gouv.fr}


\vspace{0.5cm}
\needspace{12\baselineskip}
\subsection*{Inventaire du patrimoine mobilier breton
}\index{batiments}\index{bibliographie}\index{bretagne}\index{datation}\index{designation}\index{donnees!ouvertes}\index{illustration}\index{inventaire}\index{localisation}\index{objets!mobiliers}\index{passerelle!inspire}\index{patrimoine}\index{patrimoine!culturel}\index{planning!cadastre}
  \begin{wrapfigure}{r}{2.5cm}
    \centering
    \qrcode[nolink]{https://data.gouv.fr/dataset/57fb924c88ee38493d5ff490}
  \end{wrapfigure}

Licence : \textbf{Licence Ouverte version 2.0
}\newline
Créé le : 2016-10-10\newline
Modifié le : 2019-02-22\newline
Popularité : 1 réutilisation,  0 suivi\newline
Mots-clé : \emph{batiments, bibliographie, bretagne, datation, designation, donnees-ouvertes, illustration, inventaire, localisation, objets-mobiliers, passerelle-inspire, patrimoine, patrimoine-culturel, planning-cadastre
}\newline
Permalien : \url{https://data.gouv.fr/dataset/57fb924c88ee38493d5ff490}\newline

\par
\noindent
    Le corpus, restitué sous forme d'entités points et centroïdes de
polygones, réunit les édifices et objets mobiliers étudiés par le
Service de l'Inventaire du Patrimoine depuis 1997. Il contient également
quelques dossiers numérisés issus d'études antérieures. Les dossiers ont
été créés selon la méthodologie de l'Inventaire général.

\textbf{Origine}

Données historiquement acquises sur Scan 25 ou sur des feuilles
cadastrales géoréférencées par nos soins. Désormais, l'acquisition à
partir du cadastre numérique fourni par la Direction Générale des Impôts
est la règle.

\textbf{Organisations partenaires}

Région Bretagne

\textbf{Liens annexes}

\begin{itemize}

\item
  \href{http://kartenn.region-bretagne.fr/patrimoine}{Visualiseur donnée
  patrimoine}
\end{itemize}

➞
\href{https://geo.data.gouv.fr/fr/datasets/5e927f4768dbc7b7efdffe18f09e3418b23faa8f}{Consulter
cette fiche sur geo.data.gouv.fr}


\vspace{0.5cm}
\needspace{12\baselineskip}
\subsection*{Ports appartenant à la région Bretagne
}\index{bateau}\index{bretagne}\index{donnees!ouvertes}\index{mer}\index{passerelle!inspire}\index{port}\index{reseaux!de!transport}\index{transport}\index{transportation}
  \begin{wrapfigure}{r}{2.5cm}
    \centering
    \qrcode[nolink]{https://data.gouv.fr/dataset/559b9272c751df277b390bd3}
  \end{wrapfigure}

Licence : \textbf{Licence Ouverte version 2.0
}\newline
Créé le : 2015-07-07\newline
Modifié le : 2019-02-08\newline
Popularité : 1 réutilisation,  0 suivi\newline
Mots-clé : \emph{bateau, bretagne, donnees-ouvertes, mer, passerelle-inspire, port, reseaux-de-transport, transport, transportation
}\newline
Permalien : \url{https://data.gouv.fr/dataset/559b9272c751df277b390bd3}\newline

\par
\noindent
    Cette couche contient le positionnement des ports appartenant à la
région suite au transfert de certains ports départementaux en 2017.

\textbf{Origine}

Les ports ont été localisés grâce à la base des ports de Bretagne datant
de 2016 dont les points ont été positionnés grâce à deux référentiels :
- les contours de la BDCARTO (IGN) pour que les points restent à
l'intérieur du polygone de chaque commune ; - le SCAN25 (IGN) pour les
positionner de la façon la plus précise possible.

\textbf{Organisations partenaires}

Région Bretagne

\textbf{Liens annexes}

\begin{itemize}

\item
  \href{http://kartenn.region-bretagne.fr/sviewer/?layers=rb:port}{Visualiseur
  simple}
\end{itemize}

➞
\href{https://geo.data.gouv.fr/fr/datasets/12678f24655de4a5472ef7ae76a507cbdd01ad8a}{Consulter
cette fiche sur geo.data.gouv.fr}


\vspace{0.5cm}
\needspace{12\baselineskip}
\subsection*{Recensement du patrimoine culturel breton
}\index{batiments}\index{bretagne}\index{donnees!ouvertes}\index{inventaire}\index{passerelle!inspire}\index{patrimoine}\index{recensement}\index{structure}
  \begin{wrapfigure}{r}{2.5cm}
    \centering
    \qrcode[nolink]{https://data.gouv.fr/dataset/559b927288ee384ee4764f5b}
  \end{wrapfigure}

Licence : \textbf{Licence Ouverte version 2.0
}\newline
Créé le : 2015-07-07\newline
Modifié le : 2019-02-18\newline
Popularité : 1 réutilisation,  0 suivi\newline
Mots-clé : \emph{batiments, bretagne, donnees-ouvertes, inventaire, passerelle-inspire, patrimoine, recensement, structure
}\newline
Permalien : \url{https://data.gouv.fr/dataset/559b927288ee384ee4764f5b}\newline

\par
\noindent
    Identification sur le territoire régional de tous les éléments bâtis
antérieurs au milieu du 20e siècle (désignation, localisation, datation,
état sanitaire).

\textbf{Origine}

La position géographique des éléments inventoriés est traitée sous la
forme d'un point à partir de fonds tels que la BD Ortho (photographie
aérienne ortho rectifiée), ainsi que les couches du cadastre moderne
(parcellaire et bâti). Les polygones saisis ont été transformé en point
en prenant le centroïde de celui-ci.

\textbf{Organisations partenaires}

Région Bretagne

\textbf{Liens annexes}

\begin{itemize}

\item
  \href{http://kartenn.region-bretagne.fr/sviewer/?layers=rb:recensement_patrimoine}{Visualiseur
  simple}
\end{itemize}

➞
\href{https://geo.data.gouv.fr/fr/datasets/0a970beb9551226a39ad0f19152394587dbec78d}{Consulter
cette fiche sur geo.data.gouv.fr}


\vspace{0.5cm}
\needspace{3\baselineskip} \rule{4cm}{0.25pt}\newline\textbf{Aussi disponible du même producteur :}\begin{itemize}
\item \href{https://data.gouv.fr/dataset/559b927d88ee385c1e764f63}{Acteurs Bretons de la Coopération Internationale et de la Solidarité}
\item \href{https://data.gouv.fr/dataset/559b927e88ee3877ce764f5e}{Actions menées par les Acteurs Bretons de la Coopération Internationale et de la Solidarité}
\item \href{https://data.gouv.fr/dataset/5649b627c751df37d5aad370}{Aires de retournement présentes sur canaux gérés par la région Bretagne}
\item \href{https://data.gouv.fr/dataset/5971fbe5c751df3651824f72}{Annexes hydrauliques et zones de frayères des voies navigables morbihannaises}
\item \href{https://data.gouv.fr/dataset/5846bab2c751df6b40c0bb7e}{Appels à projets "maisons éclusières"}
\item \href{https://data.gouv.fr/dataset/5835cb2b88ee383975c65bb3}{Arrêts ferroviaires régionaux de la région Bretagne}
\item \href{https://data.gouv.fr/dataset/5b6839ce634f41446ca3eb09}{Arrêts logiques routiers départementaux de la région Bretagne}
\item \href{https://data.gouv.fr/dataset/5acb394588ee381ebe544e20}{Arrêts physiques routiers départementaux de la région Bretagne}
\item \href{https://data.gouv.fr/dataset/5836a8fd88ee3843e5c65bb3}{Arrêts routiers régionaux de la région Bretagne}
\item \href{https://data.gouv.fr/dataset/5b4f2f52c751df4ea419aef3}{Base de données multimodale transports publics en Bretagne - MobiBreizh}
\item \href{https://data.gouv.fr/dataset/559b927bc751df301b390bda}{Bâtiments du domaine public fluvial propriété de la région Bretagne}
\item \href{https://data.gouv.fr/dataset/559b927e88ee380c27764f5f}{Bornes kilométriques présentes sur les voies navigables gérées par la Région Bretagne}
\item \href{https://data.gouv.fr/dataset/559b927c88ee385c1e764f62}{Bornes multiservices présentes sur les voies navigables appartenant à la région Bretagne}
\item \href{https://data.gouv.fr/dataset/5649b62a88ee38769fe72046}{Bouées présentes sur les voies navigables appartenant à la région Bretagne}
\item \href{https://data.gouv.fr/dataset/559b9278c751df13ef390bda}{Cales de mise à l'eau présentes sur les voies navigables gérées par la Région Bretagne}
\item \href{https://data.gouv.fr/dataset/559b9276c751df301b390bd9}{Centres de Formation aux Apprentis en Bretagne : formations}
\item \href{https://data.gouv.fr/dataset/559b927b88ee385c1e764f61}{Centres de Formation aux Apprentis en Bretagne : sites}
\item \href{https://data.gouv.fr/dataset/559b927888ee385c1e764f5f}{Centres des impôts fonciers par commune en Bretagne}
\item \href{https://data.gouv.fr/dataset/592c1d3c88ee3804e81345ba}{Circonscriptions législatives en Bretagne}
\item \href{https://data.gouv.fr/dataset/5bb72b80634f4174efea311c}{Communes du patrimoine rural de Bretagne}
\item \href{https://data.gouv.fr/dataset/5971fbe588ee385258276456}{Communes nouvelles en Bretagne : suivi des évolutions depuis le 01/01/2016}
\item \href{https://data.gouv.fr/dataset/5971fbe388ee38522976e8ef}{Compteurs Cyclistes/piétons présents sur les canaux gérés par la Région Bretagne}
\item \href{https://data.gouv.fr/dataset/5971fbe4c751df380134c03b}{Connaissance de l’appareil productif en Bretagne - Caractéristiques des établissements}
\item \href{https://data.gouv.fr/dataset/59fc6ab1c751df63968ec72e}{Courbes de niveaux en Bretagne - pas 10m}
\item \href{https://data.gouv.fr/dataset/5bb72b82634f4175110803be}{Criées de Bretagne}
\item \href{https://data.gouv.fr/dataset/5603ef1288ee3814cb5b59ef}{Croix et calvaires de Bretagne}
\item \href{https://data.gouv.fr/dataset/5649b62d88ee3876a5e72047}{Délimitation des biefs des voies navigables appartenant à la région Bretagne}
\item \href{https://data.gouv.fr/dataset/5649b62cc751df3104aad373}{Délimitation des zones d'intervention des centres d'exploitation de la direction déléguée des voies navigables de la région Bretagne}
\item \href{https://data.gouv.fr/dataset/5b6839cd8b4c414db4a45c72}{Demandeur d'emploi de catégorie ABC et cat A. en Bretagne}
\item \href{https://data.gouv.fr/dataset/559b927688ee385c1e764f5c}{Démarche de Gestion Intégré de la Zone Côtière (GIZC) en Bretagne - OBSOLETE}
\item \href{https://data.gouv.fr/dataset/559b927888ee385c1e764f5e}{Destinations touristiques de Bretagne}
\item \href{https://data.gouv.fr/dataset/559b927a88ee3877ce764f5d}{Domaines des ports et aéroports appartenant à la région Bretagne}
\item \href{https://data.gouv.fr/dataset/559b927ec751df301b390bdb}{Écluses sur les voies navigables bretonnes}
\item \href{https://data.gouv.fr/dataset/57fb917188ee3847425ff490}{Enquêtes thématiques du patrimoine breton}
\item \href{https://data.gouv.fr/dataset/586d110688ee3805fa3f4e5f}{EPCI de Bretagne au 1er janvier 2016}
\item \href{https://data.gouv.fr/dataset/559b9276c751df13ef390bd9}{Espaces de concessions des ports et aéroports appartenant à la région Bretagne}
\item \href{https://data.gouv.fr/dataset/559b9272c751df13ef390bd7}{Etangs d'alimentation des voies navigables appartenant à la région Bretagne}
\item \href{https://data.gouv.fr/dataset/58b557f388ee3854467f0426}{Etat de publication des données orthophotos de Bretagne}
\item \href{https://data.gouv.fr/dataset/5b1fcfbbc751df3ad3c97e91}{Index du développement durable breton - Dimension économique}
\item \href{https://data.gouv.fr/dataset/5b1fcfbc88ee3870502aa0dd}{Index du développement durable breton - Dimension sociale}
\item \href{https://data.gouv.fr/dataset/57fb919288ee3847535ff490}{Infrastructures des ports appartenant à la région Bretagne}
\item \href{https://data.gouv.fr/dataset/5846babb88ee3841c9c65bb3}{Inventaire du patrimoine breton : couche simplifiée}
\item \href{https://data.gouv.fr/dataset/5b1fcfbc88ee387205b056d8}{Les structures d’accompagnement au développement économique et innovation en Bretagne}
\item \href{https://data.gouv.fr/dataset/5b1fcfbdc751df3ad2084329}{Liaisons maritimes gérées par la région Bretagne}
\item \href{https://data.gouv.fr/dataset/5acb3945c751df55fdd531c9}{Lignes routières départementales gérées par la région Bretagne}
\item \href{https://data.gouv.fr/dataset/5835cb2cc751df45ebc0bb7e}{Lignes routières régionales de la région Bretagne}
\item \href{https://data.gouv.fr/dataset/5971fbe3c751df383c0f0422}{Limites des communes Open Street Map avec les EPCI associées en Bretagne}
\item \href{https://data.gouv.fr/dataset/559b927ac751df277b390bd5}{Lycées en Bretagne}
\item \href{https://data.gouv.fr/dataset/559b9276c751df1cab390bd7}{Mobilier présent sur les canaux appartenant à la région Bretagne}
\item \href{https://data.gouv.fr/dataset/5acb3945c751df55bcbc70d6}{Observations faunistiques sur et aux abords des voies navigables bretonnes gérés par la Région Bretagne}
\item et 54 autres jeux de données\end{itemize}

\clearpage
\section{Région Île-de-France}


\begin{center}
  \includegraphics[width=3cm]{images/orga/2015-03-05_a6cd667d87434dce8ed85084db889359_LOGO-IDF-SIGNATURE-CAMPAGNE-100.jpg}
\end{center}


\url{http://data.iledefrance.fr}


\vspace{0.5cm}

\needspace{12\baselineskip}
\subsection*{Bancs des parcs et jardins de la ville de Versailles
}\index{bancs}\index{cadre!de!vie!environnement}\index{environnement}\index{geolocalisation}\index{jardins}\index{parcs}
  \begin{wrapfigure}{r}{2.5cm}
    \centering
    \qrcode[nolink]{https://data.gouv.fr/dataset/59591d02a3a7291dcf9c8158}
  \end{wrapfigure}

Licence : \textbf{Licence Ouverte version 2.0
}\newline
Créé le : 2017-07-02\newline
Modifié le : 2019-03-17\newline
Popularité : 1 réutilisation,  0 suivi\newline
Mots-clé : \emph{bancs, cadre-de-vie-environnement, environnement, geolocalisation, jardins, parcs
}\newline
Permalien : \url{https://data.gouv.fr/dataset/59591d02a3a7291dcf9c8158}\newline

\par
\noindent
    Ce jeu de données contient la position des bancs situés dans les parcs
et jardins de Versailles ou sur le domaine public dans les zones
arborées.


\vspace{0.5cm}
\needspace{12\baselineskip}
\subsection*{Bancs des parcs et jardins de la ville de Versailles
}\index{bancs}\index{environnement}\index{geolocalisation}\index{jardins}\index{parcs}
  \begin{wrapfigure}{r}{2.5cm}
    \centering
    \qrcode[nolink]{https://data.gouv.fr/dataset/53698f42a3a729239d2036bd}
  \end{wrapfigure}

Licence : \textbf{Licence Ouverte
}\newline
Créé le : 2014-04-22\newline
Modifié le : 2015-07-16\newline
Popularité : 1 réutilisation,  0 suivi\newline
Mots-clé : \emph{bancs, environnement, geolocalisation, jardins, parcs
}\newline
Permalien : \url{https://data.gouv.fr/dataset/53698f42a3a729239d2036bd}\newline

\par
\noindent
    Ce jeu de données contient la position des bancs situés dans les parcs
et jardins de Versailles ou sur le domaine public dans les zones
arborées.


\vspace{0.5cm}
\needspace{12\baselineskip}
\subsection*{Budget primitif (BP) - 2012
}\index{budget}\index{budget!regional}\index{finances}
  \begin{wrapfigure}{r}{2.5cm}
    \centering
    \qrcode[nolink]{https://data.gouv.fr/dataset/53698fd4a3a729239d20384b}
  \end{wrapfigure}

Licence : \textbf{Licence Ouverte
}\newline
Créé le : 2014-04-22\newline
Modifié le : 2015-09-11\newline
Popularité : 1 réutilisation,  0 suivi\newline
Mots-clé : \emph{budget, budget-regional, finances
}\newline
Permalien : \url{https://data.gouv.fr/dataset/53698fd4a3a729239d20384b}\newline

\par
\noindent
    Budget prévisionnel annuel de la région Île-de-France du niveau le plus
global au niveau le plus fin.

Le jeu comprend : le code secteur, le libellé secteur, le chapitre, la
sous-fonction, le code fonctionnel, le programme, le type de contrat, le
code action, le libellé action et le budget primitif en euros.

\textbf{Pour plus d'informations :}

-
\href{http://www.collectivites-locales.gouv.fr/budget-des-collectivites-0}{Budget
des collectivités \textbar{} Collectivités locales}\\
- \href{http://www.iledefrance.fr/competence/budget-regional}{Budget
régional \textbar{} Région Île-de-France}


\vspace{0.5cm}
\needspace{12\baselineskip}
\subsection*{Budget primitif (BP) - 2013
}\index{budget}\index{budget!regional}\index{finances}
  \begin{wrapfigure}{r}{2.5cm}
    \centering
    \qrcode[nolink]{https://data.gouv.fr/dataset/53698fd6a3a729239d20384e}
  \end{wrapfigure}

Licence : \textbf{Licence Ouverte
}\newline
Créé le : 2014-04-22\newline
Modifié le : 2015-08-10\newline
Popularité : 1 réutilisation,  0 suivi\newline
Mots-clé : \emph{budget, budget-regional, finances
}\newline
Permalien : \url{https://data.gouv.fr/dataset/53698fd6a3a729239d20384e}\newline

\par
\noindent
    Budget prévisionnel annuel de la région Île-de-France du niveau le plus
global au niveau le plus fin.

Le jeu comprend : le code secteur, le libellé secteur, le chapitre, la
sous-fonction, le code fonctionnel, le programme, le type de contrat, le
code action, le libellé action et le budget primitif en euros.

\textbf{Pour plus d'informations :}

\begin{itemize}

\item
  \href{http://www.collectivites-locales.gouv.fr/budget-des-collectivites-0}{Budget
  des collectivités \textbar{} Collectivités locales}\\
\item
  \href{http://www.iledefrance.fr/competence/budget-regional}{Budget
  régional \textbar{} Région Île-de-France}
\end{itemize}


\vspace{0.5cm}
\needspace{12\baselineskip}
\subsection*{Budget primitif (BP) - 2014
}\index{budget}\index{budget!regional}\index{finances}
  \begin{wrapfigure}{r}{2.5cm}
    \centering
    \qrcode[nolink]{https://data.gouv.fr/dataset/53698fd6a3a729239d20384f}
  \end{wrapfigure}

Licence : \textbf{Licence Ouverte
}\newline
Créé le : 2014-04-22\newline
Modifié le : 2016-01-02\newline
Popularité : 1 réutilisation,  0 suivi\newline
Mots-clé : \emph{budget, budget-regional, finances
}\newline
Permalien : \url{https://data.gouv.fr/dataset/53698fd6a3a729239d20384f}\newline

\par
\noindent
    Budget prévisionnel annuel de la région Île-de-France du niveau le plus
global au niveau le plus fin.

Le jeu comprend : le code secteur, le libellé secteur, le chapitre, la
sous-fonction, le code fonctionnel, le programme, le type de contrat, le
code action, le libellé action et le budget primitif en euros.

\textbf{Pour plus d'informations :~}

-~\href{http://www.collectivites-locales.gouv.fr/budget-des-collectivites-0}{Budget
des collectivités \textbar{} Collectivités locales}\\
-~\href{http://www.iledefrance.fr/competence/budget-regional}{Budget
régional \textbar{} Région Île-de-France}


\vspace{0.5cm}
\needspace{12\baselineskip}
\subsection*{Budget réalisé - 2012
}\index{budget}\index{budget!regional}\index{finances}
  \begin{wrapfigure}{r}{2.5cm}
    \centering
    \qrcode[nolink]{https://data.gouv.fr/dataset/53698fdaa3a729239d20385a}
  \end{wrapfigure}

Licence : \textbf{Licence Ouverte
}\newline
Créé le : 2014-04-22\newline
Modifié le : 2016-02-18\newline
Popularité : 2 réutilisations,  0 suivi\newline
Mots-clé : \emph{budget, budget-regional, finances
}\newline
Permalien : \url{https://data.gouv.fr/dataset/53698fdaa3a729239d20385a}\newline

\par
\noindent
    Budget réalisé annuel de la région Île-de-France du niveau le plus
global au niveau le plus fin.

Le jeu comprend : le code secteur, le libellé secteur, le chapitre, la
sous-fonction, le code fonctionnel, le programme, le type de contrat, le
code action, le libellé action et le total affecté en euros.

\textbf{Pour plus d'informations :}

\begin{itemize}

\item
  \href{http://www.collectivites-locales.gouv.fr/budget-des-collectivites-0}{Budget
  des collectivités \textbar{} Collectivités locales}\\
\item
  \href{http://www.iledefrance.fr/competence/budget-regional}{Budget
  régional \textbar{} Région Île-de-France}
\end{itemize}


\vspace{0.5cm}
\needspace{12\baselineskip}
\subsection*{Cartes postales (Hauts-de-Seine)
}\index{archives}\index{communes}\index{patrimoine}\index{photos}\index{vie!culturelle}
  \begin{wrapfigure}{r}{2.5cm}
    \centering
    \qrcode[nolink]{https://data.gouv.fr/dataset/59591a6ea3a7291dcf9c810b}
  \end{wrapfigure}

Licence : \textbf{Licence Ouverte version 2.0
}\newline
Créé le : 2017-07-02\newline
Modifié le : 2019-03-17\newline
Popularité : 1 réutilisation,  0 suivi\newline
Mots-clé : \emph{archives, communes, patrimoine, photos, vie-culturelle
}\newline
Permalien : \url{https://data.gouv.fr/dataset/59591a6ea3a7291dcf9c810b}\newline

\par
\noindent
    Cartes postales d'époque (1900-1944) consacrées au territoire des
Hauts-de-Seine conservées aux Archives départementales.

\begin{center}\rule{0.5\linewidth}{\linethickness}\end{center}

Les Archives départementales des Hauts-de-Seine conservent une
importante collection de cartes postales constituée essentiellement à
partir d'acquisitions.

Ce fonds de près de 10 500 cartes postales, illustrant les paysages
urbains et champêtres des 36 communes des Hauts-de-Seine ainsi que les
différents modes de locomotion à travers le temps, se compose de 3
sous-séries :

\begin{itemize}

\item
  9 Fi, Cartes postales anciennes (1900-1944),
\item
  10 Fi, Cartes postales postérieures à la Seconde Guerre mondiale (1945
  à nos jours)
\item
  12 Fi 2, Collection Lapie (1950-1955).
\end{itemize}

Ces cartes postales, classées par communes et par thèmes, évoquent le
passé de nos 36 communes : scènes de rues, marchés animés, promenades
d'élégantes dans les bois ou les parcs. Le thème des transports y est
largement représenté avec le chemin de fer, le tramway, l'automobile ou
encore l'aviation.

Les Archives départementales proposent dans ce jeu de données la
sous-série de cartes postales anciennes (9 Fi) consacrée à la période
1900 à 1944, sous-série tombée dans le domaine public. Cette collection
mise en ligne est amenée à s'enrichir régulièrement au fur et à mesure
de l'avancée des opérations de numérisation de ce fonds iconographique.

\begin{center}\rule{0.5\linewidth}{\linethickness}\end{center}


\vspace{0.5cm}
\needspace{12\baselineskip}
\subsection*{Cartographie des emplacements des commissariats à Paris et petite
couronne
}\index{geolocalisation}\index{police}
  \begin{wrapfigure}{r}{2.5cm}
    \centering
    \qrcode[nolink]{https://data.gouv.fr/dataset/53699039a3a729239d203955}
  \end{wrapfigure}

Licence : \textbf{Licence Ouverte
}\newline
Créé le : 2014-04-22\newline
Modifié le : 2016-01-17\newline
Popularité : 1 réutilisation,  0 suivi\newline
Mots-clé : \emph{geolocalisation, police
}\newline
Permalien : \url{https://data.gouv.fr/dataset/53699039a3a729239d203955}\newline

\par
\noindent
    Points d'accueil police pour Paris et la petite couronne


\vspace{0.5cm}
\needspace{12\baselineskip}
\subsection*{Cartographie des établissements ``Tourisme \& Handicap''
}\index{accessibilite}\index{equipement!culturel}\index{geolocalisation}\index{loisirs}\index{tourisme}\index{vivre!avec!un!handicap}
  \begin{wrapfigure}{r}{2.5cm}
    \centering
    \qrcode[nolink]{https://data.gouv.fr/dataset/5369903aa3a729239d203957}
  \end{wrapfigure}

Licence : \textbf{Licence Ouverte
}\newline
Créé le : 2013-12-10\newline
Modifié le : 2016-01-16\newline
Popularité : 1 réutilisation,  0 suivi\newline
Mots-clé : \emph{accessibilite, equipement-culturel, geolocalisation, loisirs, tourisme, vivre-avec-un-handicap
}\newline
Permalien : \url{https://data.gouv.fr/dataset/5369903aa3a729239d203957}\newline

\par
\noindent
    Liste et géolocalisation des établissements ou prestataires labellisés
``Tourisme \& Handicap'' compilée par la sous-direction du Tourisme sur
la base de fichiers fournis par l'association Tourisme et Handicaps,
gestionnaire du label.

La marque « Tourisme \& Handicap » vise à apporter aux personnes en
situation de handicap une information fiable et objective sur le niveau
d'accessibilité des sites et des hébergements touristiques : lieux de
visite, musées, offices de tourisme, hôtels, restaurants\ldots{}

Les conditions générales d'attribution du label et son fonctionnement
sont décrits sur le
site\url{http://www.dgcis.gouv.fr/marques-nationales-tourisme/} Les
données sont filtrées sur l'Île-de-France et géolocalisées (WGS84).

(Source : Ministère de l'économie, des finances et d'industrie ;
24-06-2013)


\vspace{0.5cm}
\needspace{12\baselineskip}
\subsection*{Correspondances Code INSEE - Code Postal
}\index{communes}\index{geolocalisation}
  \begin{wrapfigure}{r}{2.5cm}
    \centering
    \qrcode[nolink]{https://data.gouv.fr/dataset/536991c5a3a729239d203d51}
  \end{wrapfigure}

Licence : \textbf{Licence Ouverte
}\newline
Créé le : 2014-04-22\newline
Modifié le : 2016-03-10\newline
Popularité : 2 réutilisations,  1 suivi\newline
Mots-clé : \emph{communes, geolocalisation
}\newline
Permalien : \url{https://data.gouv.fr/dataset/536991c5a3a729239d203d51}\newline

\par
\noindent
    Correspondance entre les codes postaux et codes INSEE des communes
franciliennes.

Ce jeu de données intègre par ailleurs des données de référence sur les
communes:

\begin{itemize}

\item
  Population
\item
  Superficie
\item
  Altitude moyenne
\item
  Forme géographique
\end{itemize}

Ce jeu de données a été constitué à partir des données de la base
GEOFLA® mise à disposition par l'IGN (codes INSEE) ainsi qu'à partir de
données Wikipédia (codes postaux).


\vspace{0.5cm}
\needspace{12\baselineskip}
\subsection*{Élections municipales 2014 - Les candidats du 2e tour (communes de 1000
hab. et plus)
}\index{commune}\index{communes}\index{election}\index{geolocalisation}
  \begin{wrapfigure}{r}{2.5cm}
    \centering
    \qrcode[nolink]{https://data.gouv.fr/dataset/53699397a3a729239d204234}
  \end{wrapfigure}

Licence : \textbf{Licence Ouverte
}\newline
Créé le : 2014-04-22\newline
Modifié le : 2016-02-21\newline
Popularité : 1 réutilisation,  0 suivi\newline
Mots-clé : \emph{commune, communes, election, geolocalisation
}\newline
Permalien : \url{https://data.gouv.fr/dataset/53699397a3a729239d204234}\newline

\par
\noindent
    Élections municipales 2014, 2ème tour : liste des candidats dans les
communes franciliennes ~de moins de 1000 habitants. Ces listes sont
publiées sous réserve d'erreurs de saisies éventuelles.


\vspace{0.5cm}
\needspace{12\baselineskip}
\subsection*{Ensembles monumentaux d'Île-de-France
}\index{cadre!de!vie}\index{environnement}\index{geolocalisation}\index{patrimoine}
  \begin{wrapfigure}{r}{2.5cm}
    \centering
    \qrcode[nolink]{https://data.gouv.fr/dataset/53699456a3a729239d204408}
  \end{wrapfigure}

Licence : \textbf{Licence Ouverte
}\newline
Créé le : 2014-04-22\newline
Modifié le : 2016-01-29\newline
Popularité : 1 réutilisation,  1 suivi\newline
Mots-clé : \emph{cadre-de-vie, environnement, geolocalisation, patrimoine
}\newline
Permalien : \url{https://data.gouv.fr/dataset/53699456a3a729239d204408}\newline

\par
\noindent
    Ce jeu de données inventorie les monuments historiques protégés en
Île-de-France.~Remarque : ce jeu de données complète celui des monuments
:
``\href{http://data.iledefrance.fr/explore/dataset/monuments-inscrits-ou-classes-dile-de-france/?tab=metas}{Monuments
inscrits ou classés d'Île-de-France}''.

Ces 2 jeux de données représentent la même chose. Ils sont
Indissociables.

Version : 1


\vspace{0.5cm}
\needspace{12\baselineskip}
\subsection*{Gares du réseau ferré d'Île-de-France
}\index{gare}\index{geolocalisation}\index{transports!en!commun}\index{voyageurs}
  \begin{wrapfigure}{r}{2.5cm}
    \centering
    \qrcode[nolink]{https://data.gouv.fr/dataset/536995eca3a729239d204861}
  \end{wrapfigure}

Licence : \textbf{Licence Ouverte
}\newline
Créé le : 2014-04-22\newline
Modifié le : 2016-02-22\newline
Popularité : 2 réutilisations,  1 suivi\newline
Mots-clé : \emph{gare, geolocalisation, transports-en-commun, voyageurs
}\newline
Permalien : \url{https://data.gouv.fr/dataset/536995eca3a729239d204861}\newline

\par
\noindent
    Ce jeu de données représente les gares (SNCF et RER) d'Île-de-France en
2007.

La première mise à jour date de 1992. Il y a eu une réactualisation en
2002 et une autre en 2006.

En 2007, la couche a été remodelée, deux noms de colonnes ont été
changés et une colonne concernant la circulation des RER et des
Transiliens a été rajoutée.

Version : 1


\vspace{0.5cm}
\needspace{12\baselineskip}
\subsection*{Indicateurs de résultat des lycées d'enseignement général et
technologique (2012)
}\index{lycees}
  \begin{wrapfigure}{r}{2.5cm}
    \centering
    \qrcode[nolink]{https://data.gouv.fr/dataset/5369968fa3a729239d204a82}
  \end{wrapfigure}

Licence : \textbf{Licence Ouverte
}\newline
Créé le : 2013-12-10\newline
Modifié le : 2015-12-04\newline
Popularité : 1 réutilisation,  0 suivi\newline
Mots-clé : \emph{lycees
}\newline
Permalien : \url{https://data.gouv.fr/dataset/5369968fa3a729239d204a82}\newline

\par
\noindent
    Données statistiques : les indicateurs de résultat des lycées permettent
d'évaluer l'action propre de chaque lycée. Ils sont établis à partir des
résultats des élèves au baccalauréat et de leur parcours scolaire dans
l'établissement. Les lycées d'enseignement général et technologique,
publics et privés sous contrat, sont concernés. Il ne s'agit aucunement
d'un classement mais d'un regard croisé sur les trois indicateurs et les
« valeurs ajoutées » correspondantes.


\vspace{0.5cm}
\needspace{12\baselineskip}
\subsection*{Infogreffe - Chiffres Clés 2016
}\index{budget}\index{effectifs}\index{entreprises}\index{vie!economique!innovation}
  \begin{wrapfigure}{r}{2.5cm}
    \centering
    \qrcode[nolink]{https://data.gouv.fr/dataset/59591a7ea3a7291dd09c80ea}
  \end{wrapfigure}

Licence : \textbf{Licence Ouverte version 2.0
}\newline
Créé le : 2017-07-02\newline
Modifié le : 2019-03-17\newline
Popularité : 2 réutilisations,  0 suivi\newline
Mots-clé : \emph{budget, effectifs, entreprises, vie-economique-innovation
}\newline
Permalien : \url{https://data.gouv.fr/dataset/59591a7ea3a7291dd09c80ea}\newline

\par
\noindent
    Les chiffres clés des sociétés commerciales ayant déposé leurs comptes
annuels pour l' exercices 2016,~enrichis des années 2015 et 2014.


\vspace{0.5cm}
\needspace{12\baselineskip}
\subsection*{Jeux de données - Île-de-France Open Data
}\index{liste!de!jeux!de!donnees}
  \begin{wrapfigure}{r}{2.5cm}
    \centering
    \qrcode[nolink]{https://data.gouv.fr/dataset/53699776a3a729239d204cd1}
  \end{wrapfigure}

Licence : \textbf{Licence Ouverte
}\newline
Créé le : 2013-12-10\newline
Modifié le : 2016-01-19\newline
Popularité : 1 réutilisation,  0 suivi\newline
Mots-clé : \emph{liste-de-jeux-de-donnees
}\newline
Permalien : \url{https://data.gouv.fr/dataset/53699776a3a729239d204cd1}\newline

\par
\noindent
    Les jeux de données fournis par Île-de-France Open Data pour
data.gouv.fr.


\vspace{0.5cm}
\needspace{12\baselineskip}
\subsection*{La carte des hôtels classés en Île-de-France
}\index{geolocalisation}\index{tourisme}
  \begin{wrapfigure}{r}{2.5cm}
    \centering
    \qrcode[nolink]{https://data.gouv.fr/dataset/53699796a3a729239d204d25}
  \end{wrapfigure}

Licence : \textbf{Licence Ouverte
}\newline
Créé le : 2013-12-10\newline
Modifié le : 2016-03-12\newline
Popularité : 2 réutilisations,  2 suivis\newline
Mots-clé : \emph{geolocalisation, tourisme
}\newline
Permalien : \url{https://data.gouv.fr/dataset/53699796a3a729239d204d25}\newline

\par
\noindent
    Liste et géolocalisation des hôtels franciliens classés de 1 à 5 étoiles
selon l'article 4 de l'arrêté du 23 décembre 2009 fixant les normes et
la procédure de classement des hôtels de tourisme

\begin{itemize}

\item
  1 étoile correspond à l'hôtellerie économique\\
  - Surface minimum d'une chambre double doit être de 9
  m\textsuperscript{2}, hors sanitaires. Ceux-ci peuvent être privés ou
  communs.
\item
  2 et 3 étoiles correspondent au milieu de gamme\\
  - Personnel qui parle au moins une langue officielle européenne en
  plus du français.

  \begin{itemize}
  
  \item
    Accueil garanti au moins dix heures par jour.
  \item
    Surface minimale de la chambre double est de 9 m\textsuperscript{2}
    hors sanitaires pour les 2 étoiles et de 13,5 m\textsuperscript{2},
    sanitaires inclus, pour les 3 étoiles.
  \end{itemize}
\item
  4 et 5 étoiles indiquent une hôtellerie haut de gamme et très haut de
  gamme\\
  - Chambres spacieuses, au moins 16 m\textsuperscript{2}, sanitaires
  inclus, en 4 étoiles, et 24 m\textsuperscript{2} en 5 étoiles.

  \begin{itemize}
  
  \item
    Dans les hôtels de plus de 30 chambres, l'accueil est assuré 24 h
    sur 24.
  \item
    Deux langues étrangères, dont l'anglais, sont requises dans un 5
    étoiles.
  \item
    Service en chambre.
  \item
    Accompagnement jusqu'à la chambre.
  \item
    Possibilité de dîner à l'hôtel.
  \item
    D'autres avantages caractérisent le 5 étoiles, comme un service de
    voiturier, une conciergerie ainsi que des équipements spécifiques
    dans les chambres tels qu'un coffre-fort et l'accès à Internet.
  \item
    Climatisation obligatoire.
  \end{itemize}
\end{itemize}

(Source Atout France - Agence de développement touristique de la France
; 24/06/13)


\vspace{0.5cm}
\needspace{12\baselineskip}
\subsection*{Les conservatoires et écoles de musique en Île-de-France
}\index{arts}\index{conservatoires}\index{ecoles}\index{geolocalisation}\index{musique}
  \begin{wrapfigure}{r}{2.5cm}
    \centering
    \qrcode[nolink]{https://data.gouv.fr/dataset/5369980ca3a729239d204e58}
  \end{wrapfigure}

Licence : \textbf{Licence Ouverte
}\newline
Créé le : 2014-04-22\newline
Modifié le : 2016-03-12\newline
Popularité : 1 réutilisation,  1 suivi\newline
Mots-clé : \emph{arts, conservatoires, ecoles, geolocalisation, musique
}\newline
Permalien : \url{https://data.gouv.fr/dataset/5369980ca3a729239d204e58}\newline

\par
\noindent
    Cette liste est composée des conservatoires classés, des écoles de
musique municipales ou privées, avec leurs adresses.


\vspace{0.5cm}
\needspace{12\baselineskip}
\subsection*{Les lieux de diffusion du spectacle vivant en Île-de-France
}\index{arts}\index{geolocalisation}\index{musique}\index{spectacle!vivant}
  \begin{wrapfigure}{r}{2.5cm}
    \centering
    \qrcode[nolink]{https://data.gouv.fr/dataset/53699853a3a729239d204f19}
  \end{wrapfigure}

Licence : \textbf{Open Data Commons Open Database License (ODbL)
}\newline
Créé le : 2014-04-22\newline
Modifié le : 2016-01-24\newline
Popularité : 1 réutilisation,  1 suivi\newline
Mots-clé : \emph{arts, geolocalisation, musique, spectacle-vivant
}\newline
Permalien : \url{https://data.gouv.fr/dataset/53699853a3a729239d204f19}\newline

\par
\noindent
    Ce jeu de données référence les lieux de diffusion régulière ou
occasionnelle du spectacle vivant en Île-de-France.

\textbf{Structure des données}\\
nom de l'établissement ; adresse ; e-mail ; téléphone ; fax ; site
internet ; géolocalisation (WGS84).


\vspace{0.5cm}
\needspace{12\baselineskip}
\subsection*{Les salles de cinéma en Île-de-France
}\index{arts}\index{cinema}\index{geolocalisation}
  \begin{wrapfigure}{r}{2.5cm}
    \centering
    \qrcode[nolink]{https://data.gouv.fr/dataset/5369987aa3a729239d204f89}
  \end{wrapfigure}

Licence : \textbf{Licence Ouverte
}\newline
Créé le : 2013-12-10\newline
Modifié le : 2016-03-12\newline
Popularité : 2 réutilisations,  1 suivi\newline
Mots-clé : \emph{arts, cinema, geolocalisation
}\newline
Permalien : \url{https://data.gouv.fr/dataset/5369987aa3a729239d204f89}\newline

\par
\noindent
    Liste des cinémas franciliens et leur adresse géocodée
(WGS84)\\[2\baselineskip](Source Centre national du cinéma et de l'image
animée)\\


\vspace{0.5cm}
\needspace{12\baselineskip}
\subsection*{Les services aux particuliers par commune ou arrondissement (base
permanente des équipements)
}\index{artisanat}\index{communes}\index{economie}\index{gendarmerie}\index{geolocalisation}\index{police}\index{services!publics}\index{tribunaux}
  \begin{wrapfigure}{r}{2.5cm}
    \centering
    \qrcode[nolink]{https://data.gouv.fr/dataset/5369987da3a729239d204f92}
  \end{wrapfigure}

Licence : \textbf{Licence Ouverte
}\newline
Créé le : 2014-04-22\newline
Modifié le : 2016-03-11\newline
Popularité : 2 réutilisations,  1 suivi\newline
Mots-clé : \emph{artisanat, communes, economie, gendarmerie, geolocalisation, police, services-publics, tribunaux
}\newline
Permalien : \url{https://data.gouv.fr/dataset/5369987da3a729239d204f92}\newline

\par
\noindent
    La base permanente des équipements (BPE), produite par l'INSEE, est
destinée à fournir le niveau d'équipement et de services rendus par un
territoire à la population.\\
Ce jeu de données recouvre les domaines des services aux particuliers.

\textbf{Liste des variables}

\begin{itemize}

\item
  \emph{Services publics}

  \begin{itemize}
  
  \item
    Police ;
  \item
    Trésorerie ;
  \item
    Agence de proximité Pôle emploi ;
  \item
    Relais Pôle emploi ;
  \item
    Permanence Pôle emploi ;
  \item
    Agence de services spécialisés ;
  \item
    Agence thématique ;
  \item
    Gendarmerie ;
  \item
    Cour d'appel ;
  \item
    Tribunal de grande instance ;
  \item
    Tribunal d'instance ;
  \item
    Conseil de prud'hommes ;
  \end{itemize}
\item
  \emph{Services généraux}

  \begin{itemize}
  
  \item
    Tribunal de commerce ;
  \item
    Banque, Caisse d'Épargne ;
  \item
    Pompes funèbres ;
  \item
    Bureau de poste ;
  \item
    Relais poste commerçant ;
  \item
    Agence postale communale ;
  \end{itemize}
\item
  \emph{Services automobiles}

  \begin{itemize}
  
  \item
    Réparation auto et de matériel agricole ;
  \item
    Contrôle technique automobile ;
  \item
    Location auto-utilitaires légers ;
  \item
    École de conduite ;
  \end{itemize}
\item
  Artisanat du bâtiment

  \begin{itemize}
  
  \item
    Maçon ;
  \item
    Plâtrier peintre ;
  \item
    Menuisier, charpentier, serrurier ;
  \item
    Plombier, couvreur, chauffagiste ;
  \item
    Électricien ;
  \item
    Entreprise générale du bâtiment ;
  \end{itemize}
\item
  \emph{Autres services à la population}

  \begin{itemize}
  
  \item
    Coiffure ;
  \item
    Vétérinaire ;
  \item
    Agence de travail temporaire ;
  \item
    Restaurant ;
  \item
    Agence immobilière ;
  \item
    Blanchisserie-Teinturerie ;
  \item
    Soins de beauté.
  \end{itemize}
\end{itemize}


\vspace{0.5cm}
\needspace{12\baselineskip}
\subsection*{Lignes de transport en commun existantes (Grand Paris)
}\index{amenagement!du!territoire}\index{deplacements!transports}\index{geolocalisation}\index{metro}\index{transports!en!commun}
  \begin{wrapfigure}{r}{2.5cm}
    \centering
    \qrcode[nolink]{https://data.gouv.fr/dataset/59591ea1a3a7291dd09c8162}
  \end{wrapfigure}

Licence : \textbf{Open Data Commons Open Database License (ODbL)
}\newline
Créé le : 2017-07-02\newline
Modifié le : 2019-03-17\newline
Popularité : 1 réutilisation,  0 suivi\newline
Mots-clé : \emph{amenagement-du-territoire, deplacements-transports, geolocalisation, metro, transports-en-commun
}\newline
Permalien : \url{https://data.gouv.fr/dataset/59591ea1a3a7291dd09c8162}\newline

\par
\noindent
    Lignes de transport en commun existantes (Grand Paris)


\vspace{0.5cm}
\needspace{12\baselineskip}
\subsection*{Liste des musées franciliens
}\index{arts}\index{geolocalisation}
  \begin{wrapfigure}{r}{2.5cm}
    \centering
    \qrcode[nolink]{https://data.gouv.fr/dataset/53699936a3a729239d2051b5}
  \end{wrapfigure}

Licence : \textbf{Licence Ouverte
}\newline
Créé le : 2013-12-10\newline
Modifié le : 2016-03-02\newline
Popularité : 1 réutilisation,  0 suivi\newline
Mots-clé : \emph{arts, geolocalisation
}\newline
Permalien : \url{https://data.gouv.fr/dataset/53699936a3a729239d2051b5}\newline

\par
\noindent
    Liste des musées d'Île-de-France : nom, adresse, site internet,
fermeture annuelle, périodes d'ouverture (jour et horaires), nocturnes,
géolocalisation (WGS84)\\[2\baselineskip](Source Ministère de la Culture
et de la Communication \textgreater{} Département de la politique des
publics)\\


\vspace{0.5cm}
\needspace{12\baselineskip}
\subsection*{Liste des organismes publics culturels géolocalisés
}\index{arts}\index{geolocalisation}
  \begin{wrapfigure}{r}{2.5cm}
    \centering
    \qrcode[nolink]{https://data.gouv.fr/dataset/5369993aa3a729239d2051c1}
  \end{wrapfigure}

Licence : \textbf{Licence Ouverte
}\newline
Créé le : 2013-12-10\newline
Modifié le : 2016-01-23\newline
Popularité : 1 réutilisation,  0 suivi\newline
Mots-clé : \emph{arts, geolocalisation
}\newline
Permalien : \url{https://data.gouv.fr/dataset/5369993aa3a729239d2051c1}\newline

\par
\noindent
    Liste des organismes publics culturels géolocalisés.

Nom de l'organisme, adresse, code postal et ville, URL, latitude et
longitude


\vspace{0.5cm}
\needspace{12\baselineskip}
\subsection*{Localisation des sites de fouille archéologiques de l'Inrap
}\index{archeologie}\index{geolocalisation}\index{patrimoine}
  \begin{wrapfigure}{r}{2.5cm}
    \centering
    \qrcode[nolink]{https://data.gouv.fr/dataset/5369996fa3a729239d205254}
  \end{wrapfigure}

Licence : \textbf{Licence Ouverte
}\newline
Créé le : 2014-04-22\newline
Modifié le : 2016-02-02\newline
Popularité : 2 réutilisations,  0 suivi\newline
Mots-clé : \emph{archeologie, geolocalisation, patrimoine
}\newline
Permalien : \url{https://data.gouv.fr/dataset/5369996fa3a729239d205254}\newline

\par
\noindent
    Ce jeu de données présente les localisations géographiques d'une
sélection d'opérations de fouilles archéologiques préventives menées sur
le territoire francilien par l'Inrap (Institut national de recherches
archéologiques préventives).\\
Structure des données :~

\begin{itemize}

\item
  géolocalisation (WGS84),~
\item
  département,
\item
  commune,~
\item
  nom du site,
\item
  dates de début et fin,~
\item
  période(s),
\item
  thème(s)
\end{itemize}


\vspace{0.5cm}
\needspace{12\baselineskip}
\subsection*{Mobilités professionnelles en 2014 : déplacements domicile - lieu de
travail
}\index{communes}\index{deplacements}\index{deplacements!transports}\index{emploi}\index{rediffusion}
  \begin{wrapfigure}{r}{2.5cm}
    \centering
    \qrcode[nolink]{https://data.gouv.fr/dataset/5a067daeb595082e62938863}
  \end{wrapfigure}

Licence : \textbf{Licence Ouverte version 2.0
}\newline
Créé le : 2017-11-11\newline
Modifié le : 2019-03-17\newline
Popularité : 1 réutilisation,  0 suivi\newline
Mots-clé : \emph{communes, deplacements, deplacements-transports, emploi, rediffusion
}\newline
Permalien : \url{https://data.gouv.fr/dataset/5a067daeb595082e62938863}\newline

\par
\noindent
    \textbf{Flux de mobilité - déplacements domicile-travail}

Chaque ligne fournit le flux de personnes se déplaçant entre une commune
ou arrondissement de résidence et une commune ou arrondissement du lieu
de travail.~

Chaque commune ou arrondissement est représenté par son code
géographique, le libellé correspondant et son département. La
consultation de ce tableau permet d'avoir rapidement une idée sur
l'importance des flux de déplacements domicile-travail entre deux
communes.~

Du fait de l'étalement de la collecte, les flux entrants dans un
territoire et les flux sortants peuvent ne pas être comptabilisés la
même année. Ainsi, par exemple, pour une commune (ou un arrondissement)
de moins de 10.000 habitants recensée l'année N, les flux sortants sont
relatifs à l'année d'enquête, soit l'année N, alors que les flux
entrants datent de l'année d'enquête des communes (ou arrondissement)
d'origine des « navetteurs ». Cela ne remet pas en cause la fiabilité de
la mesure des déplacements domicile-travail. Toutefois, compte tenu
notamment du sondage, les flux faibles (moins de 200) devront être
considérés comme des ordres de grandeur.


\vspace{0.5cm}
\needspace{12\baselineskip}
\subsection*{Mode d'occupation du sol (MOS) en 11 postes en 2012
}\index{amenagement}\index{geolocalisation}\index{mos}\index{urbanisme}
  \begin{wrapfigure}{r}{2.5cm}
    \centering
    \qrcode[nolink]{https://data.gouv.fr/dataset/536999ffa3a729239d2053a1}
  \end{wrapfigure}

Licence : \textbf{Open Data Commons Open Database License (ODbL)
}\newline
Créé le : 2014-04-22\newline
Modifié le : 2016-03-15\newline
Popularité : 1 réutilisation,  0 suivi\newline
Mots-clé : \emph{amenagement, geolocalisation, mos, urbanisme
}\newline
Permalien : \url{https://data.gouv.fr/dataset/536999ffa3a729239d2053a1}\newline

\par
\noindent
    Mode d'occupation du sol en Ile de France en 2012 en 11 postes de
légende au pas de 25m.

Le Mos (Modes d'Occupation du Sol) est l'atlas cartographique
informatisé de l'occupation du sol de l'Île-de-France. Actualisé
régulièrement depuis sa première édition de 1982, le Mos permet de
suivre et d'analyser en détail l'évolution de l'occupation du sol sur
tout le territoire Régional. L'IAU îdF vient de terminer la mise à jour
2012 du Mos. Le premier inventaire complet de l'occupation du sol en
Île-de-France (Modes d'Occupation des Sols : MOS) date de 1982. Depuis
cette date, le Mos a été mis à jour huit fois (1987, 1990, 1994, 1999,
2003, 2008 et 2012). Grâce à ces mises à jour très régulières et à sa
précision à la fois thématique (la nomenclature de base comporte 81
postes de légende) et géométrique (précision du 1/5000) le Mos permet de
visualiser et d'analyser en détail les évolutions de l'occupation du sol
Régional : extension de l'urbanisation, mutation des tissus urbains,
transformation des espaces ruraux\ldots{} Chaque mise à jour du Mos est
établie à partir d'une couverture photographique aérienne complète de
l'Île-de-France et de diverses sources d'information complémentaires
(fichiers administratifs, informations adressées par les communes,
etc.). Comme la précédente, la mise à jour 2012 a été réalisée
directement à l'écran, à partir d'une orthophotographie numérique
Régionale en couleur de résolution 12.5 cm acquise auprès d'InterAtlas.
Cette technique efficace permet un travail plus précis et plus fiable
que la méthode traditionnelle (photo-interprétation sur calques, à
partir de clichés papier), tant pour la photo-interprétation visuelle
que pour la saisie des modifications géométriques. Une base de
connaissance sur le MOS, véritable outil de communication entre les
photo-interprètes et l'IAU îdF a été mise en ligne sur Internet pour
cette sixième édition :
\url{http://www.IAU}{]}(http://www.IAU{]}(http://www.IAU))îdF.fr/basemos

\begin{longtable}[]{@{}l@{}}
\toprule
\textbf{Description du processus :}grille au pas de 25 m constituée à
partir de la couche vecteur d'evolumos 1982 : nomenclature en 130 postes
- achevé en 1985 - RGP 1987 : première mise à jour 1990 : migration vers
ArcInfo - RGP 1994 : passage de 130 à 110 postes de légende 1999 : mise
à jour sur orthophoto - recalage - vérification des équipements -
passage de 110 à 83 postes de légende - RGP 2003 : mise à jour sur
orthophoto précédée de corrections géométriques et thématiques sur les
MOS de 1982 à 1999 2008 : mise à jour sur orthophoto - passage de 83 à
81 postes de légende 2012 : mise à jour sur orthophoto - modification du
classement des 81 postes de légende\tabularnewline
\textbf{Logique :}Grille (au format vecteur) au pas de 25
m\tabularnewline
\textbf{Résolution de l'échelle :}5000\tabularnewline
\textbf{Résolution de la distance :}2\tabularnewline
\textbf{Ouest :}1.440529\tabularnewline
\textbf{Est :}3.565709\tabularnewline
\textbf{Nord :}49.248454\tabularnewline
\textbf{Sud :}48.110677\tabularnewline
\bottomrule
\end{longtable}

\begin{longtable}[]{@{}l@{}}
\toprule
\textbf{Code11 :}\tabularnewline
\emph{1 = Forêts}\tabularnewline
\emph{2 = Milieux semi-naturels}\tabularnewline
\emph{3 = Espaces agricoles}\tabularnewline
\emph{4 = Eau}\tabularnewline
\emph{5 = Espaces ouverts artificialisés}\tabularnewline
\emph{6 = Habitat individuel}\tabularnewline
\emph{7 = Habitat collectif}\tabularnewline
\emph{8 = Activités}\tabularnewline
\emph{9 = Equipements}\tabularnewline
\emph{10 = Transports}\tabularnewline
\emph{11 = Carrières, décharges et chantiers}\tabularnewline
\bottomrule
\end{longtable}


\vspace{0.5cm}
\needspace{12\baselineskip}
\subsection*{Parcs Naturels Régionaux (PNR)
}\index{parc!naturel!regional}
  \begin{wrapfigure}{r}{2.5cm}
    \centering
    \qrcode[nolink]{https://data.gouv.fr/dataset/53699b7fa3a729239d205753}
  \end{wrapfigure}

Licence : \textbf{Open Data Commons Open Database License (ODbL)
}\newline
Créé le : 2013-12-10\newline
Modifié le : 2016-03-05\newline
Popularité : 2 réutilisations,  0 suivi\newline
Mots-clé : \emph{parc-naturel-regional
}\newline
Permalien : \url{https://data.gouv.fr/dataset/53699b7fa3a729239d205753}\newline

\par
\noindent
    Les Parcs naturels régionaux sont créés pour protéger et mettre en
valeur de grands espaces ruraux habités. Peut être classé ``Parc naturel
régional'' un territoire à dominante rurale dont les paysages, les
milieux naturels et le patrimoine culturel sont de grande qualité, mais
dont l'équilibre est fragile. Un Parc naturel régional s'organise autour
d'un projet concerté de développement durable, fondé sur la protection
et la valorisation de son patrimoine naturel et culturel.


\vspace{0.5cm}
\needspace{12\baselineskip}
\subsection*{Paris - liste des équipements de proximité (écoles, piscines, jardins,
\ldots{})
}\index{ecole}\index{equipements!sportifs}\index{espaces!verts}\index{geolocalisation}\index{loisirs}\index{petite!enfance}\index{services!publics}
  \begin{wrapfigure}{r}{2.5cm}
    \centering
    \qrcode[nolink]{https://data.gouv.fr/dataset/53699b81a3a729239d20575a}
  \end{wrapfigure}

Licence : \textbf{Open Data Commons Open Database License (ODbL)
}\newline
Créé le : 2013-12-10\newline
Modifié le : 2016-03-15\newline
Popularité : 1 réutilisation,  0 suivi\newline
Mots-clé : \emph{ecole, equipements-sportifs, espaces-verts, geolocalisation, loisirs, petite-enfance, services-publics
}\newline
Permalien : \url{https://data.gouv.fr/dataset/53699b81a3a729239d20575a}\newline

\par
\noindent
    Liste des équipements de proximité gérés par la Mairie de Paris :
écoles, crèches, médiathèques, centres d'animation, équipements
sportifs, espaces verts, \ldots{}


\vspace{0.5cm}
\needspace{12\baselineskip}
\subsection*{Points d'intérêt de la ville d'Issy-les-Moulineaux
}\index{college}\index{commerce}\index{ecoles}\index{entreprises}\index{equipement!culturel}\index{equipements!sportifs}\index{geolocalisation}\index{loisirs}\index{lycees}\index{services!publics}\index{transports!en!commun}\index{velo}
  \begin{wrapfigure}{r}{2.5cm}
    \centering
    \qrcode[nolink]{https://data.gouv.fr/dataset/53699cfba3a729239d205af8}
  \end{wrapfigure}

Licence : \textbf{Licence Ouverte
}\newline
Créé le : 2014-04-22\newline
Modifié le : 2015-09-13\newline
Popularité : 2 réutilisations,  0 suivi\newline
Mots-clé : \emph{college, commerce, ecoles, entreprises, equipement-culturel, equipements-sportifs, geolocalisation, loisirs, lycees, services-publics, transports-en-commun, velo
}\newline
Permalien : \url{https://data.gouv.fr/dataset/53699cfba3a729239d205af8}\newline

\par
\noindent
    Les points d'intérêt géolocalisés à Issy-les-Moulineaux:

\begin{itemize}

\item
  Administrations, bureaux de poste, cimetières, secours
\item
  Education (écoles maternelles et élémentaires, collèges,
  lycées\ldots{})
\item
  Cadre de vie (aires de jeux, arbres remarquables, conteneurs à verre,
  jardins, parcs, squares\ldots{})
\item
  Hôtels et chambres d'hôtes
\item
  Petite enfance (crèches, haltes-garderies\ldots{})
\item
  Santé (hôpitaux, cliniques, défibrillateurs, PMI, pharmacies\ldots{})
\item
  Seniors (hébergement, services\ldots{})
\item
  Les équipements sportifs
\item
  Transports (Autolib', location automobile, métro, parkings 2 roues,
  parkings voitures en sous-sol, RER C, Tramway T2, Vélib')
\item
  Culture et loisirs (cinéma, médiathèques, musique, tourisme, scultures
  de métal\ldots{})
\item
  Tous commerces
\item
  Entreprises TIC de + de 50 salariés
\end{itemize}


\vspace{0.5cm}
\needspace{12\baselineskip}
\subsection*{Positions géographiques des stations du réseau ferré RATP
}\index{transports}\index{transports!en!commun}
  \begin{wrapfigure}{r}{2.5cm}
    \centering
    \qrcode[nolink]{https://data.gouv.fr/dataset/53699d65a3a729239d205c0d}
  \end{wrapfigure}

Licence : \textbf{Licence Ouverte
}\newline
Créé le : 2013-12-10\newline
Modifié le : 2016-10-07\newline
Popularité : 1 réutilisation,  0 suivi\newline
Mots-clé : \emph{transports, transports-en-commun
}\newline
Permalien : \url{https://data.gouv.fr/dataset/53699d65a3a729239d205c0d}\newline

\par
\noindent
    Liste des points d'arrêts générée à partir de l'archive GTFS de la RATP
(Bus, Tramway, Métro et RER).


\vspace{0.5cm}
\needspace{12\baselineskip}
\subsection*{Projets d'aménagement en Île-de-France
}\index{environnement}\index{geolocalisation}\index{territoire}
  \begin{wrapfigure}{r}{2.5cm}
    \centering
    \qrcode[nolink]{https://data.gouv.fr/dataset/53699e64a3a729239d205e7f}
  \end{wrapfigure}

Licence : \textbf{Open Data Commons Open Database License (ODbL)
}\newline
Créé le : 2014-04-22\newline
Modifié le : 2016-02-10\newline
Popularité : 1 réutilisation,  0 suivi\newline
Mots-clé : \emph{environnement, geolocalisation, territoire
}\newline
Permalien : \url{https://data.gouv.fr/dataset/53699e64a3a729239d205e7f}\newline

\par
\noindent
    Recensement des principaux secteurs de projets d'aménagement en
Île-de-France. Situation en Février 2014.

Une Base plus complète (données sur le phasage, la programmation) est
disponible à l'IAU et fait l'objet d'une convention.

Version : 1


\vspace{0.5cm}
\needspace{12\baselineskip}
\subsection*{Recensement des équipements sportifs à Paris
}\index{equipements!sportifs}\index{geolocalisation}\index{loisirs}\index{sport}
  \begin{wrapfigure}{r}{2.5cm}
    \centering
    \qrcode[nolink]{https://data.gouv.fr/dataset/53699eb7a3a729239d205f47}
  \end{wrapfigure}

Licence : \textbf{Licence Ouverte
}\newline
Créé le : 2013-12-10\newline
Modifié le : 2016-02-25\newline
Popularité : 1 réutilisation,  0 suivi\newline
Mots-clé : \emph{equipements-sportifs, geolocalisation, loisirs, sport
}\newline
Permalien : \url{https://data.gouv.fr/dataset/53699eb7a3a729239d205f47}\newline

\par
\noindent
    Recensement des équipements sportifs réalisé par le Ministère des Sports
avec notamment :

\begin{itemize}

\item
  type d'équipement (salle multisports, court de tennis, dojo, \ldots{})
  ;
\item
  nature de l'équipement (intérieur, découvert, \ldots{}) ;
\item
  nature du sol (synthétique, parquet, béton, \ldots{}) ;
\item
  activités (danse, gymnastique, football, badminton, \ldots{}) ;
\item
  adresse et coordonnées GPS .
\end{itemize}

(Dernière mise à jour par le Ministère : 20/02/2013


\vspace{0.5cm}
\needspace{12\baselineskip}
\subsection*{Unités de gendarmerie accueillant du public en IDF
}\index{geolocalisation}\index{securite}
  \begin{wrapfigure}{r}{2.5cm}
    \centering
    \qrcode[nolink]{https://data.gouv.fr/dataset/5369a31aa3a729239d2069ac}
  \end{wrapfigure}

Licence : \textbf{Licence Ouverte
}\newline
Créé le : 2013-12-10\newline
Modifié le : 2015-11-09\newline
Popularité : 2 réutilisations,  0 suivi\newline
Mots-clé : \emph{geolocalisation, securite
}\newline
Permalien : \url{https://data.gouv.fr/dataset/5369a31aa3a729239d2069ac}\newline

\par
\noindent
    Liste de l'ensemble des unités de gendarmerie accueillant du public.

Chaque unité a : - Code unité - Nom de l'unité - La catégorie d'unité -
La localité de l'unité - L'email de l'unité - Le numéro de téléphone de
l'unité - Le numéro de faxe de l'unité - Le code postal~ - la
géolocalisation (WGS 84)


\vspace{0.5cm}
\needspace{3\baselineskip} \rule{4cm}{0.25pt}\newline\textbf{Aussi disponible du même producteur :}\begin{itemize}
\item \href{https://data.gouv.fr/dataset/53698e47a3a729239d20340b}{Accessibilité des arrêts de bus RATP}
\item \href{https://data.gouv.fr/dataset/59591960a3a7291dd09c80c7}{Accessibilité des arrêts de bus RATP}
\item \href{https://data.gouv.fr/dataset/53698e48a3a729239d20340d}{Accessibilité des équipements de la Ville de Paris}
\item \href{https://data.gouv.fr/dataset/5c5727019ce2e7132fcc53f4}{Accessibilité des équipements de la Ville de Paris}
\item \href{https://data.gouv.fr/dataset/59591eeba3a7291dd09c816b}{Accessibilité des gares et stations de Métro et RER RATP}
\item \href{https://data.gouv.fr/dataset/59591f04a3a7291dd09c816e}{Accessibilité des lignes du réseau de surface RATP}
\item \href{https://data.gouv.fr/dataset/59591ceba3a7291dd09c8133}{Accidentologie à Paris (2012-2013)}
\item \href{https://data.gouv.fr/dataset/53698e52a3a729239d203428}{Accompagnement des lycéens pour les Clubs Théâtre}
\item \href{https://data.gouv.fr/dataset/59591cd4a3a7291dd09c8130}{Accompagnement des lycéens pour les Clubs Théâtre}
\item \href{https://data.gouv.fr/dataset/53698e5ea3a729239d203447}{Action culturelle "Musiques au Lycée" 2011-2013}
\item \href{https://data.gouv.fr/dataset/59591816a3a7291dcf9c80c3}{Action culturelle "Musiques au Lycée" 2011-2014}
\item \href{https://data.gouv.fr/dataset/5959152ca3a7291dd09c803d}{Actions culturelles de l'Orchestre national d'Île-de-France - 2012}
\item \href{https://data.gouv.fr/dataset/59591cdba3a7291dd09c8131}{Actions éducatives dans les lycées en 2012-2013}
\item \href{https://data.gouv.fr/dataset/53698e60a3a729239d20344b}{Actions éducatives dans les lycées en 2012-2013}
\item \href{https://data.gouv.fr/dataset/59591a2da3a7291dd09c80e0}{Actions éducatives dans les lycées en 2014-2015}
\item \href{https://data.gouv.fr/dataset/595918d0a3a7291dd09c80b7}{Activités des îles de loisirs}
\item \href{https://data.gouv.fr/dataset/59591d0ba3a7291dd09c8135}{Activités des médecins généralistes et MEP en 2010}
\item \href{https://data.gouv.fr/dataset/53698e6ca3a729239d20346b}{Activités des médecins généralistes et MEP en 2010}
\item \href{https://data.gouv.fr/dataset/53698e6ca3a729239d20346c}{Activités des médecins spécialistes et autres PSL en 2010}
\item \href{https://data.gouv.fr/dataset/53698e73a3a729239d20347c}{Activités par âge quinquennal au recensement (exhaustif)}
\item \href{https://data.gouv.fr/dataset/59591680a3a7291dcf9c8093}{Activités par âge quinquennal au recensement (exhaustif)}
\item \href{https://data.gouv.fr/dataset/595918c2a3a7291dd09c80b5}{Adhérents à la démarche "Mangeons Local en Île-de-France"}
\item \href{https://data.gouv.fr/dataset/59591b0da3a7291dd09c80fd}{Adresse et géolocalisation des établissements d'enseignement du premier et second degrés}
\item \href{https://data.gouv.fr/dataset/53698e7ca3a729239d20349d}{Adresses et données GPS des bibliothèques municipales}
\item \href{https://data.gouv.fr/dataset/59591881a3a7291dd09c80ad}{Aérodromes (CDGT 2013)}
\item \href{https://data.gouv.fr/dataset/53698e90a3a729239d2034d2}{Agriculture bio 2008-2011 - productions végétales - surfaces par département}
\item \href{https://data.gouv.fr/dataset/59591ae7a3a7291dd09c80f8}{Aide à la demi-pension des formations post-bac 2012-2013}
\item \href{https://data.gouv.fr/dataset/53698eada3a729239d20351e}{Aide à la demi-pension des formations post-bac 2012-2013}
\item \href{https://data.gouv.fr/dataset/53698eaea3a729239d20351f}{Aide à la demi-pension des lycéens 2012-2013}
\item \href{https://data.gouv.fr/dataset/53698eaea3a729239d203520}{Aide à la diffusion des éditeurs indépendants}
\item \href{https://data.gouv.fr/dataset/59591dc9a3a7291dcf9c8171}{Aide à la diffusion des éditeurs indépendants}
\item \href{https://data.gouv.fr/dataset/53698eb0a3a729239d203526}{Aide à la résidence de musiques actuelles et au développement d’artistes 2011-2013}
\item \href{https://data.gouv.fr/dataset/59591d38a3a7291dcf9c815f}{Aide à la résidence de musiques actuelles et au développement d’artistes 2011-2014}
\item \href{https://data.gouv.fr/dataset/53698eb1a3a729239d203527}{Aide à la résidence territoriale pour les arts de la rue et les arts du cirque 2013}
\item \href{https://data.gouv.fr/dataset/59591602a3a7291dd09c805c}{Aide à l'équipement des formations post-bac 2012-2013}
\item \href{https://data.gouv.fr/dataset/53698eb2a3a729239d203529}{Aide à l'équipement des formations post-bac 2012-2013}
\item \href{https://data.gouv.fr/dataset/59591af5a3a7291dcf9c811d}{Aide à l'équipement des lycéens 2012-2013}
\item \href{https://data.gouv.fr/dataset/53698eb2a3a729239d20352a}{Aide à l'équipement des lycéens 2012-2013}
\item \href{https://data.gouv.fr/dataset/53698eb3a3a729239d20352d}{Aide à projet arts de la rue et arts du cirque 2013}
\item \href{https://data.gouv.fr/dataset/59591d3aa3a7291dd09c813b}{Aide à projet arts de la rue et arts du cirque 2013}
\item \href{https://data.gouv.fr/dataset/53698eb4a3a729239d20352e}{Aide à projet « Musiques actuelles et amplifiées » 2011-2014}
\item \href{https://data.gouv.fr/dataset/59591817a3a7291dd09c809e}{Aide à projet « Musiques actuelles et amplifiées » 2011-2014}
\item \href{https://data.gouv.fr/dataset/53698eb4a3a729239d20352f}{Aide au développement de la permanence artistique et culturelle 2007-2013}
\item \href{https://data.gouv.fr/dataset/59591ccca3a7291dd09c812f}{Aide au développement de la permanence artistique et culturelle 2007-2014}
\item \href{https://data.gouv.fr/dataset/53698eb5a3a729239d203532}{Aide aux départs en vacances : les organismes relais (2012)}
\item \href{https://data.gouv.fr/dataset/53698eb6a3a729239d203533}{Aide aux départs en vacances : où sont partis les bénéficiaires en 2012 ?}
\item \href{https://data.gouv.fr/dataset/59591725a3a7291dd09c8087}{Aide aux départs en vacances : où sont partis les bénéficiaires en 2012 ?}
\item \href{https://data.gouv.fr/dataset/53698eb6a3a729239d203534}{Aide aux festivals de musiques actuelles et amplifiées 2011-2013}
\item \href{https://data.gouv.fr/dataset/59591ccba3a7291dcf9c8152}{Aide aux festivals de musiques actuelles et amplifiées 2011-2014}
\item \href{https://data.gouv.fr/dataset/53698eb6a3a729239d203535}{Aide aux fonds pour les disquaires indépendants 2011-2013}
\item et 879 autres jeux de données\end{itemize}

\clearpage
\section{Rennes Métropole}


\begin{center}
  \includegraphics[width=3cm]{images/orga/37_f3155ec3f64e76b612b8794cda868f-100.png}
\end{center}


La communauté d'agglomération de Rennes regroupe 43 communes et 416 000
habitants dont plus de 60 000 étudiants.


\vspace{0.5cm}

\needspace{12\baselineskip}
\subsection*{Statistiques de fréquentation des expositions du Musée de Bretagne et
des Champs Libres
}\index{culture}\index{statistiques}
  \begin{wrapfigure}{r}{2.5cm}
    \centering
    \qrcode[nolink]{https://data.gouv.fr/dataset/5369a073a3a729239d206384}
  \end{wrapfigure}

Licence : \textbf{Open Data Commons Open Database License (ODbL)
}\newline
Créé le : 2013-11-13\newline
Modifié le : 2015-08-26\newline
De 2006-01-01 à 2013-09-01\newline
Mise à jour : annuelle\newline
Popularité : 1 réutilisation,  0 suivi\newline
Mots-clé : \emph{culture, statistiques
}\newline
Permalien : \url{https://data.gouv.fr/dataset/5369a073a3a729239d206384}\newline

\par
\noindent
    Chiffres de fréquentations des expositions du Musée de Bretagne et des
Champs Libres depuis l'ouverture en 2006.


\vspace{0.5cm}
\needspace{3\baselineskip} \rule{4cm}{0.25pt}\newline\textbf{Aussi disponible du même producteur :}\begin{itemize}
\item \href{https://data.gouv.fr/dataset/53698e75a3a729239d203483}{Actualités du STAR}
\item \href{https://data.gouv.fr/dataset/5a0aebd088ee383ee1c3fd44}{Adresses du référentiel voies et adresses de Rennes Métropole}
\item \href{https://data.gouv.fr/dataset/53698e85a3a729239d2034b4}{Agenda culturel des Champs Libres}
\item \href{https://data.gouv.fr/dataset/5a0aebc388ee38398fcc707a}{Aires de jeux des espaces verts de la Ville de Rennes}
\item \href{https://data.gouv.fr/dataset/5a0aebb7c751df2c545d86cb}{Aménagements vélo et zones de circulation apaisée sur Rennes Métropole}
\item \href{https://data.gouv.fr/dataset/5a0aec02c751df300c85a261}{Arbres d'alignement en accompagnement de voirie sur la ville de Rennes}
\item \href{https://data.gouv.fr/dataset/5a0aec02c751df2c545d86d2}{Arbres d'ornement des espaces verts de la Ville de Rennes}
\item \href{https://data.gouv.fr/dataset/5a0aec09c751df339cabb231}{Bornes de recharge dédiées aux véhicules électriques sur le territoire de Rennes Métropole}
\item \href{https://data.gouv.fr/dataset/5a0aec0ec751df300c85a263}{Cantons d'Ille-et-Vilaine (version Rennes Métropole)}
\item \href{https://data.gouv.fr/dataset/5a0aebc688ee3835dfd324ca}{Centres de vote sur Rennes Métropole}
\item \href{https://data.gouv.fr/dataset/5a0aebdf88ee3835dfd324cc}{Comités de secteur de Rennes Métropole}
\item \href{https://data.gouv.fr/dataset/5a0aec0688ee383ee1c3fd48}{Composteurs collectifs sur Rennes Métropole}
\item \href{https://data.gouv.fr/dataset/5a0aebb7c751df2fdda1d84d}{Constructions bâties sur Rennes Métropole}
\item \href{https://data.gouv.fr/dataset/5a0aec04c751df339cabb230}{Déchèteries et plateformes végétaux sur Rennes Métropole}
\item \href{https://data.gouv.fr/dataset/536992f1a3a729239d204062}{Disponibilité des ascenceurs et escalators du métro}
\item \href{https://data.gouv.fr/dataset/536992f1a3a729239d204063}{Disponibilité des pars relais du réseau STAR}
\item \href{https://data.gouv.fr/dataset/536992f2a3a729239d204064}{Disponibilité des VéloStar}
\item \href{https://data.gouv.fr/dataset/5a0aebdb88ee3835dfd324cb}{Domanialité des voies du Référentiel Voies et Adresses de Rennes Métropole}
\item \href{https://data.gouv.fr/dataset/5a0aec0bc751df300c85a262}{Données géographiques du réseau STAR : arrêts logiques}
\item \href{https://data.gouv.fr/dataset/5a0aebd088ee38398fcc707c}{Données géographiques du réseau STAR : arrêts physiques}
\item \href{https://data.gouv.fr/dataset/5a0aec0788ee38398fcc707e}{Données géographiques du réseau STAR : itinéraires des lignes}
\item \href{https://data.gouv.fr/dataset/5a0aebe988ee3840a6bbc179}{Emprise de Rennes Métropole}
\item \href{https://data.gouv.fr/dataset/53699584a3a729239d20474b}{Fichiers Horaires du réseau STAR}
\item \href{https://data.gouv.fr/dataset/5a0aec0ec751df2c545d86d3}{Ilots Regroupés pour l'Information Statistique (Iris) version Rennes Métropole}
\item \href{https://data.gouv.fr/dataset/53699709a3a729239d204bba}{Information Trafic du réseau STAR}
\item \href{https://data.gouv.fr/dataset/5a0aebdb88ee383ee1c3fd45}{Inventaire 2014 des commerces du Pays de Rennes : cellules commerciales}
\item \href{https://data.gouv.fr/dataset/5a0aec0c88ee3835dfd324ce}{Les œuvres d'art sur le domaine public de la ville de Rennes}
\item \href{https://data.gouv.fr/dataset/55eee73f88ee3866daa46ec1}{Limites communales référentielles de Rennes Métropole (polygones)}
\item \href{https://data.gouv.fr/dataset/5a0aebeb88ee383ee1c3fd46}{Limites communales référentielles de Rennes Métropole (polylignes)}
\item \href{https://data.gouv.fr/dataset/5a0aec13c751df300c85a264}{Métro du réseau STAR : localisation des stations}
\item \href{https://data.gouv.fr/dataset/5a0aebe9c751df2c545d86ce}{Murs d'expression libre (Street Art et graffitis) sur le territoire de la Ville de Rennes}
\item \href{https://data.gouv.fr/dataset/5a0aec1cc751df339cabb233}{Parcs vélos sécurisés gérés par Rennes Métropole}
\item \href{https://data.gouv.fr/dataset/5a0aebe7c751df300c85a260}{Périmètres des 12 quartiers de la Ville de Rennes}
\item \href{https://data.gouv.fr/dataset/5a0aebf4c751df2c545d86cf}{Périmètres des 45 sous-quartiers de la Ville de Rennes}
\item \href{https://data.gouv.fr/dataset/5a0aec17c751df339cabb232}{Périmètres des bureaux de vote sur Rennes Métropole}
\item \href{https://data.gouv.fr/dataset/5a0aebf688ee38398fcc707d}{Périmètres des centres de vote sur Rennes Métropole}
\item \href{https://data.gouv.fr/dataset/5a0aebe188ee3835dfd324cd}{Points culminants issus du modèle Numérique de Terrain de 2014 sur le territoire de Rennes Métropole}
\item \href{https://data.gouv.fr/dataset/5a0aec1288ee38398fcc7080}{Points d'apport volontaire des déchets ménagers sur Rennes Métropole}
\item \href{https://data.gouv.fr/dataset/5a0aec1188ee383ee1c3fd4a}{Points de présentation des bacs roulants sur Rennes Métropole}
\item \href{https://data.gouv.fr/dataset/5a0aebfdc751df2c545d86d1}{Points de regroupement des bacs roulants sur Rennes Métropole}
\item \href{https://data.gouv.fr/dataset/5a0aec1388ee383d09b1ab28}{Référentiel bâtiment sur Rennes Métropole}
\item \href{https://data.gouv.fr/dataset/5a0aec1188ee383d09b1ab27}{Repères de nivellement du canevas géodésique de Rennes Métropole}
\item \href{https://data.gouv.fr/dataset/5a0aebdfc751df2c545d86cd}{Stations de réparation et gonflage pour vélo sur Rennes Métropole}
\item \href{https://data.gouv.fr/dataset/5a0aec1cc751df300c85a265}{Stations du canevas géodésique de Rennes Métropole}
\item \href{https://data.gouv.fr/dataset/5369a070a3a729239d20637e}{Statistiques de fréquentation de la bibliothèque des Champs Libres par heure 2012}
\item \href{https://data.gouv.fr/dataset/5369a071a3a729239d206380}{Statistiques de fréquentation de la bibliothèque des Champs Libres par jour - 2010}
\item \href{https://data.gouv.fr/dataset/5369a072a3a729239d206381}{Statistiques de fréquentation de la bibliothèque des Champs Libres par jour - 2011}
\item \href{https://data.gouv.fr/dataset/5369a072a3a729239d206382}{Statistiques de fréquentation de la bibliothèque des Champs Libres par jour 2012}
\item \href{https://data.gouv.fr/dataset/5369a073a3a729239d206383}{Statistiques de fréquentation de la bibliothèque des Champs Libres par jour - 2013}
\item \href{https://data.gouv.fr/dataset/5369a08ba3a729239d2063b9}{Statistiques de prêts de la bibliothèque des Champs Libres en 2006}
\item et 10 autres jeux de données\end{itemize}

\clearpage
\section{Rennes Métropole en accès libre}


\begin{center}
  \includegraphics[width=3cm]{images/orga/66_cb171cad4040edb21549eefeed27bc-100.png}
\end{center}


Le programme open data de Rennes Métropole qui publie les données de la
Ville de Rennes, de Rennes Métropole et de ses
partenaires.\url{http://www.data.rennesmetropole.fr}


\vspace{0.5cm}

\needspace{12\baselineskip}
\subsection*{Menus des cantines des écoles - Ville de Rennes
}\index{education}\index{etablissements!scolaires}
  \begin{wrapfigure}{r}{2.5cm}
    \centering
    \qrcode[nolink]{https://data.gouv.fr/dataset/580a5784a3a7292dcfa9d200}
  \end{wrapfigure}

Licence : \textbf{Open Data Commons Open Database License (ODbL)
}\newline
Créé le : 2016-10-21\newline
Modifié le : 2019-03-17\newline
Popularité : 1 réutilisation,  0 suivi\newline
Mots-clé : \emph{education, etablissements-scolaires
}\newline
Permalien : \url{https://data.gouv.fr/dataset/580a5784a3a7292dcfa9d200}\newline

\par
\noindent
    Ce jeu de données comprend le contenu des menus des cantines des écoles
maternelles et primaires publiques rennaises, divisées par secteur,
ainsi que les centres de loisirs pour l'été.~

\textbf{ATTENTION:~Pour savoir à quel secteur correspond chaque école,
vous pouvez consulter le fichier suivant:}
~\href{https://rennes-metropole.opendatasoft.com/explore/dataset/ecoles-rennes/}{https://rennes-metropole.opendatasoft.com/explore/\ldots{}}{]}(https://rennes-metropole.opendatasoft.com/explore/\ldots{}{]}(https://rennes-metropole.opendatasoft.com/explore/dataset/ecoles-rennes/))
\href{https://rennes-metropole.opendatasoft.com/explore/dataset/ecoles-rennes/}{}AB
=~Agriculture biologique

PLC =~Plat à base de légumes et céréales

FLS =~Fruits et légumes de saison

VBF =~Viande bovine française

VPF =~Viande de porc française

BBC =~Bleu blanc coeur

EBR = Eau du bassin Rennais~

SVP = Sans viande ni poisson


\vspace{0.5cm}
\needspace{12\baselineskip}
\subsection*{Topologie des points d'arrêt de bus du réseau STAR
}\index{bus}\index{mobilite}\index{transport}
  \begin{wrapfigure}{r}{2.5cm}
    \centering
    \qrcode[nolink]{https://data.gouv.fr/dataset/580a5784a3a7292dcea9d209}
  \end{wrapfigure}

Licence : \textbf{Open Data Commons Open Database License (ODbL)
}\newline
Créé le : 2016-10-21\newline
Modifié le : 2019-03-17\newline
Popularité : 1 réutilisation,  0 suivi\newline
Mots-clé : \emph{bus, mobilite, transport
}\newline
Permalien : \url{https://data.gouv.fr/dataset/580a5784a3a7292dcea9d209}\newline

\par
\noindent
    \textbf{Liste des points d'arrêt de bus~du réseau STAR, comprenant
notamment leur nom, leur commune~de rattachement et la géolocalisation.}


\vspace{0.5cm}
\needspace{3\baselineskip} \rule{4cm}{0.25pt}\newline\textbf{Aussi disponible du même producteur :}\begin{itemize}
\item \href{https://data.gouv.fr/dataset/586db27aa3a7290df6f4bea9}{Accidents corporels de la circulation 2015 - Rennes Métropole}
\item \href{https://data.gouv.fr/dataset/580a5746a3a7292dcfa9d1d6}{Actes d'état civil à Rennes}
\item \href{https://data.gouv.fr/dataset/5a275e43a3a7293578c3f7c1}{Adresses du référentiel voies et adresses au format BAL}
\item \href{https://data.gouv.fr/dataset/580a577ba3a7292dcfa9d1fa}{Adresses du référentiel voies et adresses de Rennes Métropole}
\item \href{https://data.gouv.fr/dataset/580a5771a3a7292dcfa9d1f5}{Agences commerciales du réseau STAR}
\item \href{https://data.gouv.fr/dataset/5b208a36b595082912f1ace8}{Aire de camping car - Cesson-Sévigné}
\item \href{https://data.gouv.fr/dataset/5b208a3cb595082912f1ace9}{Aires de jeux - Cesson-Sévigné}
\item \href{https://data.gouv.fr/dataset/580a5735a3a7292dcea9d1d3}{Alertes trafic en temps réel sur les lignes du réseau STAR}
\item \href{https://data.gouv.fr/dataset/580a5786a3a7292dcea9d20b}{Aménagements vélo et zones de circulation apaisée sur Rennes Métropole}
\item \href{https://data.gouv.fr/dataset/586db2c8a3a7290df5f4be9e}{Ancienneté d'emménagement par age}
\item \href{https://data.gouv.fr/dataset/589e7ecba3a72974c1f1125b}{Arbres d'ornement des espaces verts de la Ville de Rennes}
\item \href{https://data.gouv.fr/dataset/580a574fa3a7292dcea9d1e8}{Arbres du jardin du Thabor}
\item \href{https://data.gouv.fr/dataset/580a5737a3a7292dcfa9d1cb}{Artistes et concerts aux Transmusicales}
\item \href{https://data.gouv.fr/dataset/59b0b6d0b59508797d111c83}{Artistes et concerts aux Transmusicales}
\item \href{https://data.gouv.fr/dataset/5b2c6827a3a7292d14b11d6f}{Base élus de Rennes Ville et Métropole 2018}
\item \href{https://data.gouv.fr/dataset/580a5741a3a7292dcea9d1dd}{Base élus Rennes Ville et Métropole 2017}
\item \href{https://data.gouv.fr/dataset/586db267a3a7290df5f4be91}{Base organismes et équipements "Vivre à Rennes"}
\item \href{https://data.gouv.fr/dataset/5a5d6c3db5950807eeecba32}{Base SIRENE - Cesson-Sévigné}
\item \href{https://data.gouv.fr/dataset/5a5ebd65b5950876ccecba32}{Base SIRENE - Noyal-Châtillon-sur-Seiche}
\item \href{https://data.gouv.fr/dataset/5a73d543b5950827d4a2256b}{Base SIRENE - Saint-Jacques de la Lande}
\item \href{https://data.gouv.fr/dataset/5c89c8cd9ce2e72817e27c1e}{Base Sirene v3 ß - Rennes Métropole}
\item \href{https://data.gouv.fr/dataset/580a5736a3a7292dcfa9d1ca}{Bornes de recharge dédiées aux véhicules électriques sur le territoire de Rennes Métropole}
\item \href{https://data.gouv.fr/dataset/580a572ba3a7292dcfa9d1c2}{BP 2012 - Ville de Rennes - Budget Principal}
\item \href{https://data.gouv.fr/dataset/580a5766a3a7292dcea9d1f9}{BP 2012 - Ville de Rennes - Budgets Annexes}
\item \href{https://data.gouv.fr/dataset/580a572ea3a7292dcea9d1cd}{BP 2012 - Ville de Rennes - Subventions d'équipement aux associations}
\item \href{https://data.gouv.fr/dataset/580a572ea3a7292dcfa9d1c4}{BP 2012 - Ville de Rennes - Subventions exceptionnelles aux associations}
\item \href{https://data.gouv.fr/dataset/580a5766a3a7292dcfa9d1ee}{BP 2012 - Ville de Rennes - Subventions ordinaires aux associations}
\item \href{https://data.gouv.fr/dataset/580a5795a3a7292dcfa9d20b}{BP 2013 - BUDGET PRINCIPAL}
\item \href{https://data.gouv.fr/dataset/580a5763a3a7292dcea9d1f7}{BP 2013 - Ville de Rennes - Budget Annexes}
\item \href{https://data.gouv.fr/dataset/580a5729a3a7292dcfa9d1c0}{BP 2013 - Ville de Rennes - Subventions d'équipement aux associations}
\item \href{https://data.gouv.fr/dataset/580a578ba3a7292dcea9d20e}{BP 2013 - Ville de Rennes - Subventions exceptionnelles aux associations}
\item \href{https://data.gouv.fr/dataset/580a5763a3a7292dcfa9d1ec}{BP 2013 - Ville de Rennes - Subventions ordinaires aux associations}
\item \href{https://data.gouv.fr/dataset/580a575fa3a7292dcea9d1f4}{BP 2014 - Ville de Rennes - Budget Annexes}
\item \href{https://data.gouv.fr/dataset/580a5726a3a7292dcea9d1c6}{BP 2014 - Ville de Rennes - Budget Principal par article}
\item \href{https://data.gouv.fr/dataset/580a5725a3a7292dcfa9d1bc}{BP 2014 - Ville de Rennes - Budget Principal par sous fonction}
\item \href{https://data.gouv.fr/dataset/580a5727a3a7292dcea9d1c7}{BP 2014 - Ville de Rennes - Subventions d'équipement aux associations}
\item \href{https://data.gouv.fr/dataset/580a5726a3a7292dcfa9d1bd}{BP 2014 - Ville de Rennes - Subventions exceptionnelles aux associations}
\item \href{https://data.gouv.fr/dataset/580a575ea3a7292dcfa9d1e9}{BP 2014 - Ville de Rennes - Subventions ordinaires aux associations}
\item \href{https://data.gouv.fr/dataset/580a5791a3a7292dcea9d212}{BP 2015 - Ville de Rennes - Budget Principal par article}
\item \href{https://data.gouv.fr/dataset/580a5722a3a7292dcfa9d1ba}{BP 2015 - Ville de Rennes - Budget Principal par sous fonctions}
\item \href{https://data.gouv.fr/dataset/580a5752a3a7292dcea9d1ea}{BP 2015 - Ville de Rennes - Budgets Annexes}
\item \href{https://data.gouv.fr/dataset/580a575aa3a7292dcea9d1f0}{BP 2015 - Ville de Rennes - Subventions d'équipement aux associations}
\item \href{https://data.gouv.fr/dataset/580a5759a3a7292dcfa9d1e5}{BP 2015 - Ville de Rennes - Subventions exceptionnelles aux associations}
\item \href{https://data.gouv.fr/dataset/580a5723a3a7292dcea9d1c4}{BP 2015 - Ville de Rennes - Subventions ordinaires aux associations}
\item \href{https://data.gouv.fr/dataset/580a5791a3a7292dcfa9d209}{BP 2016 - Rennes Métropole}
\item \href{https://data.gouv.fr/dataset/580a575ba3a7292dcea9d1f1}{BP 2016 - Ville de Rennes - Budget Principal}
\item \href{https://data.gouv.fr/dataset/580a575ba3a7292dcfa9d1e6}{BP 2016 - Ville De Rennes - Budgets Annexes}
\item \href{https://data.gouv.fr/dataset/580a5749a3a7292dcfa9d1d9}{BP 2016 - Ville de Rennes - Subventions aux associations}
\item \href{https://data.gouv.fr/dataset/586f0854a3a7291134880609}{BP 2017 - Rennes Métropole}
\item \href{https://data.gouv.fr/dataset/58be2261a3a7293affefbd25}{BP 2017 - Ville de Rennes - Budget Principal}
\item et 216 autres jeux de données\end{itemize}

\clearpage
\section{Réseau ferré de France}


\begin{center}
  \includegraphics[width=3cm]{images/orga/fe_8e4e1e78984792b515835ec00b8241-100.png}
\end{center}


Réseau Ferré de France est propriétaire et gestionnaire du réseau ferré
national. A ce titre, il définit les objectifs applicables en matière de
gestion du trafic, de fonctionnement, d'entretien et de développement du
réseau. Son activité commerciale principale consiste à vendre des
sillons, c'est-à-dire des créneaux horaires permettant de faire circuler
les trains d'un point du réseau à un autre.


\vspace{0.5cm}

\needspace{12\baselineskip}
\subsection*{Gares ferroviaires de tous types, exploitées ou non
}\index{transport}
  \begin{wrapfigure}{r}{2.5cm}
    \centering
    \qrcode[nolink]{https://data.gouv.fr/dataset/536995eda3a729239d204864}
  \end{wrapfigure}

Licence : \textbf{Licence Ouverte
}\newline
Créé le : 2013-07-08\newline
Modifié le : 2017-07-02\newline
De 2011-01-01 à 2011-12-31\newline
Granularité : au point d'intérêt\newline
Mise à jour : mensuelle\newline
Popularité : 5 réutilisations,  14 suivis\newline
Mots-clé : \emph{transport
}\newline
Permalien : \url{https://data.gouv.fr/dataset/536995eda3a729239d204864}\newline

\par
\noindent
    Gares ferroviaires exploitées dans une ou plusieurs des situations
suivantes :- transport des voyageurs,- transport du fret,- opérations
d'infrastructureou non exploitées.Le fichier contient :- le code de la
ligne sur laquelle la gare est située,- le nom de la gare,- la nature de
la gare : desserte voyageur et/ou desserte fret et/ou infrastructure ou
non exploitée- la position WGS84 de la gare, projetée sur la ligne, en
degrés décimaux.Une gare peut apparaître plusieurs fois dans le fichier,
une fois par ligne ferroviaire sur laquelle elle est située.


\vspace{0.5cm}
\needspace{12\baselineskip}
\subsection*{Passages à niveau
}
  \begin{wrapfigure}{r}{2.5cm}
    \centering
    \qrcode[nolink]{https://data.gouv.fr/dataset/53699bd5a3a729239d205825}
  \end{wrapfigure}

Licence : \textbf{Licence Ouverte
}\newline
Créé le : 2013-07-08\newline
Modifié le : 2016-03-10\newline
Mise à jour : mensuelle\newline
Popularité : 2 réutilisations,  2 suivis\newline
Mots-clé : \emph{aucun
}\newline
Permalien : \url{https://data.gouv.fr/dataset/53699bd5a3a729239d205825}\newline

\par
\noindent
    Passages à niveau permettant le franchissement piéton ou routier d'une
ligne ferroviaire à son niveau.Le fichier contient :- le code de la
ligne sur laquelle est situé le passage à niveau- le type de passage à
niveau- le statut ``prioritaire'' du passage à niveau (circulation
importante de trains et de véhicules)- la position WGS84 du passage à
niveau en degrés décimaux


\vspace{0.5cm}

\clearpage
\section{Réseau Sitra}


Sitra est à la fois un réseau et une plateforme qui regroupe l'ensemble
des acteurs touristiques de 8 départements de la région Rhône-Alpes et 1
département de Provence Alpes Cote d'Azur. L'ambition collective du
projet est de mettre à disposition de tous les membres et utilisateurs
du réseau des outils et des services pour permettre à chacun de
développer ses propres stratégies numériques, individuellement ou
collectivement. La plateforme Sitra contient environ 240 000 données
(POI) qui décrivent les destinations du territoire, et qui sont remises
à jour quotidiennement par l'ensemble des utilisateurs. Une bonne partie
de ces données est accessible, sous Open Licence, par API au format
JSON. Pour consulter les données se rendre sur la plateforme Sitra :
www.sitra-tourisme.com


\vspace{0.5cm}

\needspace{12\baselineskip}
\subsection*{Patrimoine culturel de la région Rhône-Alpes
}\index{architecture}\index{musee}\index{parcs!et!jardins}\index{patrimoine}\index{patrimoine!architectural}\index{patrimoine!culturel}\index{patrimoine!industriel}\index{patrimoine!religieux}\index{tourisme}\index{villages!de!france}
  \begin{wrapfigure}{r}{2.5cm}
    \centering
    \qrcode[nolink]{https://data.gouv.fr/dataset/53699bdaa3a729239d205831}
  \end{wrapfigure}

Licence : \textbf{Licence Ouverte
}\newline
Créé le : 2013-11-04\newline
Modifié le : 2016-03-03\newline
Granularité : au point d'intérêt\newline
Popularité : 1 réutilisation,  0 suivi\newline
Mots-clé : \emph{architecture, musee, parcs-et-jardins, patrimoine, patrimoine-architectural, patrimoine-culturel, patrimoine-industriel, patrimoine-religieux, tourisme, villages-de-france
}\newline
Permalien : \url{https://data.gouv.fr/dataset/53699bdaa3a729239d205831}\newline

\par
\noindent
    L'ensemble des lieux de visites sur le thème du patrimoine et de la
culture : sites et monuments historiques, parcs et jardins, musées et
écomusées, lieux de mémoire, patrimoine industriel, militaire,
religieux, ouvrages d'art, centres d'interprétation, villes d'art et
d'histoire, plus beaux villages de France \ldots{}


\vspace{0.5cm}

\clearpage
\section{SAINT-MALO AGGLOMERATION}


\begin{center}
  \includegraphics[width=3cm]{images/orga/2014-12-18_c80f1cb599d84a7ab84489440c7cf3ad_logoSma-100.jpg}
\end{center}


\textbf{Créée le 1er janvier 2001}, Saint-Malo Agglomération, Communauté
d'agglomération du Pays de Saint-Malo, regroupe 18 communes entre bords
de Rance et Baie du Mont-Saint-Michel : Cancale, Châteauneuf
d'Ille-et-Vilaine, Hirel, La Fresnais, La Gouesnière, La
Ville-es-Nonais, Le Tronchet, Lillemer, Miniac-Morvan, Plerguer,
Saint-Benoît-des-Ondes, Saint-Coulomb, Saint-Guinoux,
Saint-Jouan-des-Guérets, Saint-Malo, Saint-Méloir-des-Ondes,
Saint-Père-Marc-en-Poulet et Saint-Suliac.

\textbf{Au cœur de la Côte d'Emeraude}, à bientôt 2h15 de Paris grâce au
nouveau réseau ferroviaire en cours d'aménagement, et avec des liaisons
aériennes et maritimes vers la Grande Bretagne et les îles
anglo-normandes, Saint-Malo Agglomération compte déjà près de 80 000
habitants sur environ 245 km\textsuperscript{2}.

\textbf{Parmi ses principaux champs de compétences, Saint-Malo
Agglomération gère notamment :} - Le développement économique ; -
L'enseignement supérieur et la recherche ; - Les transports publics
(réseau de transports KSMA) ; - L'équilibre social de l'habitat ; - La
collecte, le traitement des déchets ; - La protection et la mise en
valeur de l'environnement et du cadre de vie \ldots{}

Le 18 décembre 2014, le Conseil Communautaire a adopté le principe
d'engager la commune dans une démarche d'ouverture de ses données
publiques et a adopté la Licence Ouverte pour la réutilisation de ses
données.


\vspace{0.5cm}

\needspace{12\baselineskip}
\subsection*{Emplacement des Arrêts de Bus du Réseau Transport Saint-Malo
Agglomération
}\index{arret}\index{bus}\index{reseau!de!transport}\index{transport}
  \begin{wrapfigure}{r}{2.5cm}
    \centering
    \qrcode[nolink]{https://data.gouv.fr/dataset/54942524c751df6da904805a}
  \end{wrapfigure}

Licence : \textbf{Licence Ouverte
}\newline
Créé le : 2014-12-19\newline
Modifié le : 2016-08-25\newline
Granularité : à l'EPCI\newline
Mise à jour : annuelle\newline
Popularité : 1 réutilisation,  2 suivis\newline
Mots-clé : \emph{arret, bus, reseau-de-transport, transport
}\newline
Permalien : \url{https://data.gouv.fr/dataset/54942524c751df6da904805a}\newline

\par
\noindent
    Localisation géographique des arrêts de bus du réseau de transport
urbain et péri-urbain sur le territoire de l'agglomération du Pays de
Saint-Malo.

La projection utilisée est RGF93/CC48.


\vspace{0.5cm}
\needspace{12\baselineskip}
\subsection*{Liste des conseillers Communautaires de Saint-Malo Agglomération
}\index{conseiller}\index{elu}
  \begin{wrapfigure}{r}{2.5cm}
    \centering
    \qrcode[nolink]{https://data.gouv.fr/dataset/5494242bc751df6d8e04805a}
  \end{wrapfigure}

Licence : \textbf{Licence Ouverte
}\newline
Créé le : 2014-12-19\newline
Modifié le : 2016-08-25\newline
Granularité : à l'EPCI\newline
Mise à jour : ponctuelle\newline
Popularité : 1 réutilisation,  1 suivi\newline
Mots-clé : \emph{conseiller, elu
}\newline
Permalien : \url{https://data.gouv.fr/dataset/5494242bc751df6d8e04805a}\newline

\par
\noindent
    Liste des conseillers Communautaires, avec leurs fonctions
communautaires et communales, délégations et communes d'appartenance,
pour la période 2014-2020.


\vspace{0.5cm}
\needspace{12\baselineskip}
\subsection*{Points d'Apport Volontaire
}\index{dechets}
  \begin{wrapfigure}{r}{2.5cm}
    \centering
    \qrcode[nolink]{https://data.gouv.fr/dataset/54940946c751df415b04805c}
  \end{wrapfigure}

Licence : \textbf{Licence Ouverte
}\newline
Créé le : 2014-12-19\newline
Modifié le : 2016-08-25\newline
Granularité : à l'EPCI\newline
Mise à jour : ponctuelle\newline
Popularité : 1 réutilisation,  1 suivi\newline
Mots-clé : \emph{dechets
}\newline
Permalien : \url{https://data.gouv.fr/dataset/54940946c751df415b04805c}\newline

\par
\noindent
    Localisation géographique des points d'apport volontaire, avec
identification de la commune et du type de point (colonne papier ou
verre, en surface ou enterrée,\ldots{})

La projection utilisée est RGF93/CC48


\vspace{0.5cm}
\needspace{3\baselineskip} \rule{4cm}{0.25pt}\newline\textbf{Aussi disponible du même producteur :}\begin{itemize}
\item \href{https://data.gouv.fr/dataset/57bea1c6c751df054897bae5}{Liste des marchés publics}
\item \href{https://data.gouv.fr/dataset/57be98f3c751df750997bae5}{Population légale de Saint-Malo Agglomération}
\end{itemize}

\clearpage
\section{Seine-Saint-Denis - Le Département}


\begin{center}
  \includegraphics[width=3cm]{images/orga/2014-11-03_1368537f4075452eb44393ea68b6bfdc_SeineSaintDenis-Logo-100.png}
\end{center}


La Seine-Saint-Denis est un territoire d'une superficie de 236
km\textsuperscript{2} situé au Nord-Est de Paris. Département de la
petite couronne, il compte 1,529 million d'habitants.

Territoire populaire, il se caractérise par un nombre important de
jeunes et par son dynamisme économique : développement d'un tissu
économique dense et diversifié, d'un réseau de transport publics
abondants, montée en puissance de pôles de compétitivité de dimension
mondiale, campus universitaires et d'enseignement supérieur de haut
niveau\ldots{}

Le Département de la Seine-Saint-Denis est une collectivité
territoriale. Dotée en 2014 d'un budget de 2 milliards d'euros, il
compte 8000 agents. Ce service public met en œuvre de nombreuses
politiques dans les champs de la solidarité (enfance et famille,
insertion et santé, autonomie), de l'éducation, de la culture, de la
citoyenneté, de l'aménagement et du développement durable.

\textbf{En savoir plus}

\begin{enumerate}
\def\labelenumi{\arabic{enumi}.}
\item
  \href{http://data.seine-saint-denis.fr/}{Le site Open Data de la
  Seine-Saint-Denis}
\item
  \href{http://www.seine-saint-denis.fr/-L-essentiel-du-Departement-.html}{Le
  site du département}
\item
  \href{http://geoportail93.fr/}{Géoportail93} : Portail cartographique
  de la Seine-Saint-Denis permettant de croiser l'information
  géographique produite et acquise par le Département.
\item
  \href{http://cooperation-territoriale.seine-saint-denis.fr/}{Le site
  de la coopération territoriale en Seine-Saint-Denis}
\item
  \href{http://fr.wikipedia.org/wiki/Seine-Saint-Denis}{Wikipédia sur la
  Seine-Saint-Denis}
\end{enumerate}


\vspace{0.5cm}

\needspace{12\baselineskip}
\subsection*{DSI - Sites distants
}\index{distants}\index{donnees!ouvertes}\index{dsi}\index{economie}\index{equipement}\index{equipement!economie}\index{seine!saint!denis}\index{sites}
  \begin{wrapfigure}{r}{2.5cm}
    \centering
    \qrcode[nolink]{https://data.gouv.fr/dataset/57066700c751df6914dafbbd}
  \end{wrapfigure}

Licence : \textbf{Licence Ouverte
}\newline
Créé le : 2016-04-07\newline
Modifié le : 2016-04-07\newline
Popularité : 1 réutilisation,  0 suivi\newline
Mots-clé : \emph{distants, donnees-ouvertes, dsi, economie, equipement, equipement-economie, seine-saint-denis, sites
}\newline
Permalien : \url{https://data.gouv.fr/dataset/57066700c751df6914dafbbd}\newline

\par
\noindent
    Sites distants dont l'informatique est géré par la DSI.

\subsection{Attributs :}\label{attributs}

\begin{itemize}

\item
  \textbf{geocodage} `int2`
\item
  \textbf{ref\_site} `varchar`
\item
  \textbf{direction} `varchar`
\item
  \textbf{nom\_site} `varchar`
\item
  \textbf{type\_court} `varchar`
\item
  \textbf{type\_site} `bpchar`
\item
  \textbf{adresse\_sig} `bpchar`
\item
  \textbf{adresse} `bpchar`
\item
  \textbf{ville} `bpchar`
\item
  \textbf{code\_poste} `bpchar`
\item
  \textbf{tel\_1} `bpchar`
\item
  \textbf{tel\_2} `bpchar`
\item
  \textbf{num\_isdn} `bpchar`
\item
  \textbf{etat\_prod} `bpchar`
\item
  \textbf{ref\_prod} `bpchar`
\item
  \textbf{etat} `bpchar`
\item
  \textbf{num\_rtc} `bpchar`
\item
  \textbf{debit\_max} `bpchar`
\item
  \textbf{plage\_ip} `bpchar`
\item
  \textbf{geom} `geometry`
\item
  \textbf{code\_insee} `bpchar`
\end{itemize}

\textbf{Organisations partenaires}

Département de la Seine-Saint-Denis

\textbf{Liens annexes}

\begin{itemize}

\item
  \href{http://geoportail93.fr/?LAYERS=781}{Voir sur Géoportail93}
\end{itemize}


\vspace{0.5cm}
\needspace{3\baselineskip} \rule{4cm}{0.25pt}\newline\textbf{Aussi disponible du même producteur :}\begin{itemize}
\item \href{https://data.gouv.fr/dataset/5a0aeb6888ee3835dfd324c7}{Actions contre l'isolement des personnes âgées}
\item \href{https://data.gouv.fr/dataset/594bcd0b88ee386228dfc067}{Actions d'insertion PDI}
\item \href{https://data.gouv.fr/dataset/5a0aeb68c751df2fe4b66943}{Actions d'insertion PDI (thématique professionnelle, mettre en œuvre les projets, accéder au monde du travail)}
\item \href{https://data.gouv.fr/dataset/5a0aeb6988ee383ee1c3fd3d}{Actions d'insertion PDI (thématique socio-professionnelle, améliorer les conditions de réussite et préparer les projets)}
\item \href{https://data.gouv.fr/dataset/5bc86c658b4c41355f699063}{Agences CAF}
\item \href{https://data.gouv.fr/dataset/5bc86c62634f412b4ce5ade6}{Agences CPAM}
\item \href{https://data.gouv.fr/dataset/5bc7641f634f4173dbd4a0b1}{Agences de la DGFIP}
\item \href{https://data.gouv.fr/dataset/5627a06488ee38430212613c}{Agents départementaux payés au 31 décembre}
\item \href{https://data.gouv.fr/dataset/594bcd1c88ee3862296b5b04}{Autres protections}
\item \href{https://data.gouv.fr/dataset/594bcd0cc751df718db2b04b}{Belvédères de la Seine-Saint-Denis.}
\item \href{https://data.gouv.fr/dataset/5627a904c751df21d9bd3534}{Bénéficiaires de l'aide départementale à la demi-pension en Seine-Saint Denis}
\item \href{https://data.gouv.fr/dataset/551a962dc751df401b4b1077}{Bénéficiaires des principales prestations départementales}
\item \href{https://data.gouv.fr/dataset/59a0314988ee380d020ea91d}{Cantons électoraux}
\item \href{https://data.gouv.fr/dataset/59a0314bc751df7ae56088b5}{Cantons électoraux (avant 2015)}
\item \href{https://data.gouv.fr/dataset/551901e1c751df61b1480f88}{Capacité d'accueil chez les assistants maternels par commune}
\item \href{https://data.gouv.fr/dataset/5518fdf4c751df5938480f89}{Capacité installée dans les crèches départementales par commune}
\item \href{https://data.gouv.fr/dataset/594bcd0588ee38623690e2ac}{Centres de planification familiale}
\item \href{https://data.gouv.fr/dataset/57066776c751df6914dafbc1}{Centres sociaux de Seine-Saint-Denis}
\item \href{https://data.gouv.fr/dataset/594bcd0288ee3862296b5afe}{Chemins des Parcs Départementaux}
\item \href{https://data.gouv.fr/dataset/594bcd0f88ee3870ccb4d522}{Chemins et itinéraires de randonnées (PDIPR)}
\item \href{https://data.gouv.fr/dataset/5706677488ee3809f0f061cc}{Cluster ville durable}
\item \href{https://data.gouv.fr/dataset/594bcd0288ee386228dfc064}{Collèges}
\item \href{https://data.gouv.fr/dataset/594bcd1588ee38623690e2ae}{Commissariat}
\item \href{https://data.gouv.fr/dataset/594bcd0ac751df719b826685}{Commissions Locales d'Information et de Coordination}
\item \href{https://data.gouv.fr/dataset/594bcd1388ee3870ccb4d524}{Commissions Locales d'Insertion (CLI)}
\item \href{https://data.gouv.fr/dataset/594bcd0788ee3870ccb4d51f}{Conférences Territoriales d'Insertion (CTI)}
\item \href{https://data.gouv.fr/dataset/5706677388ee381631f061d7}{Contrat de Développement Territorial (CDT)}
\item \href{https://data.gouv.fr/dataset/57066770c751df7027dafbbd}{Coordinateurs de centres sociaux de Seine-Saint-Denis}
\item \href{https://data.gouv.fr/dataset/594bcd0788ee38623da9d14a}{Couloirs de bus}
\item \href{https://data.gouv.fr/dataset/551aa8f4c751df5e324b1078}{Couloirs de bus	}
\item \href{https://data.gouv.fr/dataset/594bcd0588ee386228dfc065}{Cours d'eau 1887}
\item \href{https://data.gouv.fr/dataset/5706676fc751df1f3fdafbc7}{Cours d'eau 1887}
\item \href{https://data.gouv.fr/dataset/594bcd0888ee386228dfc066}{Cours d'eau 1936}
\item \href{https://data.gouv.fr/dataset/5706676ec751df1f3fdafbc6}{Cours d'eau 1936}
\item \href{https://data.gouv.fr/dataset/594bcd10c751df718db2b04c}{Crèches départementales}
\item \href{https://data.gouv.fr/dataset/594bcd0188ee3870ccb4d51e}{Cuisines Départementales et Offices dans les collèges}
\item \href{https://data.gouv.fr/dataset/5706676ec751df73d9dafbc3}{DAD - Projets d’aménagement en Seine-Saint-Denis}
\item \href{https://data.gouv.fr/dataset/5706676788ee381631f061d5}{DAD - Zone Urbaine Sensible (ZUS) - avant le 1er janvier 2015}
\item \href{https://data.gouv.fr/dataset/5706676588ee381f24f061da}{Data centers en Seine-Saint-Denis}
\item \href{https://data.gouv.fr/dataset/5706675188ee381631f061d4}{DCOM - DSOE - Cantons antérieurs à 2015}
\item \href{https://data.gouv.fr/dataset/5706674ec751df1f3fdafbc4}{DCOM - Elus du Conseil Départemental 93}
\item \href{https://data.gouv.fr/dataset/5706674d88ee3809f6f061d2}{DCPSL - Autres protections}
\item \href{https://data.gouv.fr/dataset/5706674ac751df6914dafbc0}{DCPSL - Hydrographie 1820 (plans d'eau)}
\item \href{https://data.gouv.fr/dataset/5706674a88ee3809f0f061cb}{DCPSL - Hydrographie 1820 (rus)}
\item \href{https://data.gouv.fr/dataset/5706674888ee381631f061d3}{DCPSL - Lieux culturels}
\item \href{https://data.gouv.fr/dataset/5706674788ee381631f061d2}{DCPSL - Ludothèques}
\item \href{https://data.gouv.fr/dataset/5706674688ee381631f061d1}{DCPSL - Monuments protégés (loi de 1913)}
\item \href{https://data.gouv.fr/dataset/5706674488ee381631f061d0}{DCPSL - Patrimoine architectural}
\item \href{https://data.gouv.fr/dataset/57066743c751df7027dafbbc}{DCPSL - Sites archéologiques}
\item \href{https://data.gouv.fr/dataset/5706674288ee381f24f061d9}{DCPSL - Voies romaines}
\item et 162 autres jeux de données\end{itemize}

\clearpage
\section{Sénat}


\begin{center}
  \includegraphics[width=3cm]{images/orga/e2_f97988f21b4216baa4e79334eafe6c-100.png}
\end{center}


Chambre haute du Parlement français

\href{http://data.senat.fr/}{data.senat.fr}, la plateforme des données
ouvertes du Sénat

Contact :
\href{mailto:opendata-tech@senat.fr}{\nolinkurl{opendata-tech@senat.fr}}


\vspace{0.5cm}

\needspace{12\baselineskip}
\subsection*{Amendements déposés au Sénat
}\index{amendements}\index{projet!de!loi}\index{proposition!de!loi}\index{senat}\index{senateurs}
  \begin{wrapfigure}{r}{2.5cm}
    \centering
    \qrcode[nolink]{https://data.gouv.fr/dataset/53a8b7f8a3a72905b7ce595d}
  \end{wrapfigure}

Licence : \textbf{Licence Ouverte
}\newline
Créé le : 2014-06-20\newline
Modifié le : 2017-03-10\newline
Granularité : au pays\newline
Mise à jour : quotienne\newline
Popularité : 1 réutilisation,  0 suivi\newline
Mots-clé : \emph{amendements, projet-de-loi, proposition-de-loi, senat, senateurs
}\newline
Permalien : \url{https://data.gouv.fr/dataset/53a8b7f8a3a72905b7ce595d}\newline

\par
\noindent
    La base « Amendements (Ameli) » vous permet d'accéder à l'ensemble des
amendements déposés au Sénat en commission (depuis octobre 2010) et en
séance publique (depuis octobre 2001). Pour chaque amendement sont
indiqués son ou ses auteur(s), son contenu, son objet, s'il a été adopté
ou non, etc.

Le jeu complet des amendements déposés sur chaque texte, en commission
ou en séance, est disponible au format CSV, par lecture et par ordre de
dépôt, dans le dossier législatif du texte considéré.

Pour plus d'informations, voir :
\url{http://data.senat.fr/ameli/}{]}(http://data.senat.fr/ameli/{]}(http://data.senat.fr/ameli/))


\vspace{0.5cm}
\needspace{12\baselineskip}
\subsection*{Dotation d'action parlementaire (Sénat)
}\index{parlement}\index{reserve!parlementaire}\index{senat}
  \begin{wrapfigure}{r}{2.5cm}
    \centering
    \qrcode[nolink]{https://data.gouv.fr/dataset/55674661c751df75c6e5726a}
  \end{wrapfigure}

Licence : \textbf{Licence Ouverte
}\newline
Créé le : 2015-05-28\newline
Modifié le : 2018-08-24\newline
De 2014-01-01 à 2014-12-31\newline
Granularité : au département\newline
Mise à jour : annuelle\newline
Popularité : 1 réutilisation,  3 suivis\newline
Mots-clé : \emph{parlement, reserve-parlementaire, senat
}\newline
Permalien : \url{https://data.gouv.fr/dataset/55674661c751df75c6e5726a}\newline

\par
\noindent
    La base ``dotation d'action parlementaire'' vous permet d'accéder à
l'ensemble des subventions proposées par les sénateurs au titre de la
dotation d'action parlementaire ``réserve parlementaire''), à compter de
l'exercice 2014. Vous trouverez, pour chaque subvention, les
informations concernant le bénéficiaire, le montant, le programme
budgétaire d'imputation, etc.


\vspace{0.5cm}
\needspace{3\baselineskip} \rule{4cm}{0.25pt}\newline\textbf{Aussi disponible du même producteur :}\begin{itemize}
\item \href{https://data.gouv.fr/dataset/53a8bb31a3a72905b7ce595f}{Comptes rendus du Sénat}
\item \href{https://data.gouv.fr/dataset/58c2c63f88ee387e365cfcf1}{Les Sénateurs}
\item \href{https://data.gouv.fr/dataset/53a8cab5a3a72905b7ce5971}{Questions du Sénat}
\item \href{https://data.gouv.fr/dataset/53ae96eaa3a729709f56d51d}{Travaux législatifs (Sénat)}
\end{itemize}

\clearpage
\section{Service départemental d'incendie et de secours de l'Essonne}


\begin{center}
  \includegraphics[width=3cm]{images/orga/55_0bd979aa3643dca1654d4a8d82e7dc-100.png}
\end{center}


Partenaire du Conseil départemental, le Service départemental d'incendie
et de secours de l'Essonne est un établissement public autonome qui
participe à la politique de sécurité civile engagé par le département.

Les missions du Sdis sont la lutte contre l'incendie, le secours
d'urgence aux personnes ainsi que la protection des biens et de
l'environnement. Ces engagements sont complétés en amont par
l'évaluation et la prévention des risques.

Le Sdis 91, a mené une réflexion aboutissant à la rédaction d'une charte
des valeurs qui repose sur cinq piliers fondamentaux :

\begin{itemize}

\item
  perpétuer les valeurs traditionnelles
\item
  l'esprit d'équipe, ciment d'une institution
\item
  un savoir-faire au service de tous
\item
  professionnalisme pour un service de qualité
\item
  promouvoir les échanges et l'esprit d'ouverture
\end{itemize}

S'adossant au Schéma départemental d'analyse et de couverture des
risques (Sdacr) et à la convention pluriannuelle avec le Conseil
départemental, le projet d'établissement du Sdis 91 formalise les
objectifs que doivent s'approprier les agents pour garantir aux
Essonniens des secours efficaces et équitables dans un cadre
organisationnel efficient.


\vspace{0.5cm}

\needspace{12\baselineskip}
\subsection*{Interventions des pompiers
}\index{accident}\index{essonne}\index{incendie}\index{interventions}\index{operation}\index{pompier}\index{sdis}\index{secours!a!personne}
  \begin{wrapfigure}{r}{2.5cm}
    \centering
    \qrcode[nolink]{https://data.gouv.fr/dataset/5bd9c7c8634f412c5f28cc48}
  \end{wrapfigure}

Licence : \textbf{Licence Ouverte
}\newline
Créé le : 2018-10-31\newline
Modifié le : 2019-03-15\newline
De 2010-01-01 à 2017-12-31\newline
Granularité : à la commune\newline
Mise à jour : hebdomadaire\newline
Popularité : 1 réutilisation,  1 suivi\newline
Mots-clé : \emph{accident, essonne, incendie, interventions, operation, pompier, sdis, secours-a-personne
}\newline
Permalien : \url{https://data.gouv.fr/dataset/5bd9c7c8634f412c5f28cc48}\newline

\par
\noindent
    Ce que contient ce jeu de données Ce jeu de données contient les
\textbf{données hebdomadaires sur les interventions des sapeurs-pompiers
de l'Essonne}. Il renseigne le nombre d'interventions des
sapeurs-pompiers selon 5 catégories, pour chaque commune (définie par
leurs codes INSEE) : - SUAP : Secours d'urgence à personne - INCN :
Incendie naturel - INCU : Incendie urbain - ACCI : Accident de la route
- AUTR : Autre

Ce jeu de données comporte deux fichiers. Le premier contient
l'historique de 2010 à 2017, et le second contient l'historique depuis
2018, mis à jour chaque semaine, le mardi matin (habituellement).

Dictionnaire de données - \textbf{ope\_code\_insee} : le code INSEE de
la commune dans laquelle se sont déroulées les interventions\\
- \textbf{nb\_ope} : nombre d'interventions réalisées -
\textbf{ope\_annee} : année pendant laquelle se sont déroulées les
interventions\\
- \textbf{ope\_semaine} : semaine pendant laquelle se sont déroulées les
interventions. Les semaines commencent le lundi matin et se terminent le
dimanche soir minuit (ISO 8601) - \textbf{ope\_categorie} : catégorie
des interventions (voir plus haut) - \textbf{ope\_code\_postal} : le
code postal de la commune dans laquelle se sont déroulées les
interventions. Celui-ci est donnée à tire informatif, car il ne permet
pas l'unicité des communes (il n'existe pas de bijection entre les codes
postaux et les communes). Pour le formuler autrement, un code postal
peut être partage entre plusieurs communes, ce qui pose des problèmes de
cartographie. - \textbf{ope\_nom\_commune} : nom de la commune dans
laquelle se sont déroulées les interventions (A titre informatif, car
seul le code INSEE fait foi)

Ce jeu de données est produit par la Direction du SDIS91.


\vspace{0.5cm}
\needspace{3\baselineskip} \rule{4cm}{0.25pt}\newline\textbf{Aussi disponible du même producteur :}\begin{itemize}
\item \href{https://data.gouv.fr/dataset/5682d1b888ee38356baf0bf4}{Découpage géographique des Etablissements Publics de Coopération Intercommunale en Essonne   au 1er janvier 2016}
\end{itemize}

\clearpage
\section{Service Départemental d'Incendie et de Secours du Bas-Rhin}


\begin{center}
  \includegraphics[width=3cm]{images/orga/c1_9599a7afa84e019ea304bf2bfb0a8a-100.jpg}
\end{center}


Le service départemental d'incendie et de secours du Bas-Rhin (SDIS 67)
est un établissement public autonome, ayant pour mission la défense
incendie, le secours à personnes et la protection des biens et de
l'environnement.

Dans le cadre de ses missions, le SDIS 67 dispose d'une compétence
exclusive à savoir la prévention, la protection et la lutte contre les
incendies.

Il concourt également avec d'autres services (équipements, services
médicaux d'urgence, etc.) à : - la protection et la lutte contre les
autres accidents, sinistres et catastrophes, - l'évaluation et la
prévention des risques technologiques ou naturels, - la préparation des
mesures de sauvegarde et l'organisation des moyens de secours, - aux
secours d'urgence aux personnes victimes d'accidents, de sinistres ou de
catastrophes ainsi qu'à leur évacuation.


\vspace{0.5cm}

\needspace{12\baselineskip}
\subsection*{DONNEE THEMATIQUE : Localisation des Défibrillateurs Automatiques
Externes (DAE) - Bas-Rhin
}\index{defibrillateur}\index{donnees!ouvertes}\index{geoportail}\index{passerelle!inspire}\index{secours}\index{urgence}
  \begin{wrapfigure}{r}{2.5cm}
    \centering
    \qrcode[nolink]{https://data.gouv.fr/dataset/59256d13c751df5259330037}
  \end{wrapfigure}

Licence : \textbf{Licence Ouverte version 2.0
}\newline
Créé le : 2017-05-24\newline
Modifié le : 2019-02-08\newline
Popularité : 1 réutilisation,  0 suivi\newline
Mots-clé : \emph{defibrillateur, donnees-ouvertes, geoportail, passerelle-inspire, secours, urgence
}\newline
Permalien : \url{https://data.gouv.fr/dataset/59256d13c751df5259330037}\newline

\par
\noindent
    Localisation des Défibrillateurs Automatiques Externes (DAE) du Bas-Rhin
accessibles au public

\textbf{Organisations partenaires}

Service Départemental d'Incendie et de Secours du Bas-Rhin (SDIS 67)

➞
\href{https://geo.data.gouv.fr/fr/datasets/68e850877c1327e239b862c9b8714ad256ad2214}{Consulter
cette fiche sur geo.data.gouv.fr}


\vspace{0.5cm}
\needspace{3\baselineskip} \rule{4cm}{0.25pt}\newline\textbf{Aussi disponible du même producteur :}\begin{itemize}
\item \href{https://data.gouv.fr/dataset/591d4e97c751df2f42a74b85}{DONNEE THEMATIQUE : Localisation des Centres d'Interventions et de Secours - Bas-Rhin}
\item \href{https://data.gouv.fr/dataset/587dfa8c88ee385cd89b81a4}{Organisation territoriale du SDIS 67 au 01/01/2019}
\end{itemize}

\clearpage
\section{Shom}


\begin{center}
  \includegraphics[width=3cm]{images/orga/85_a692053e8f401fbfd7bde9d8604cb3-100.png}
\end{center}


Le Shom est l'opérateur public pour l'information géographique maritime
et littorale de référence.

Établissement public de l'État à caractère administratif (EPA) depuis le
11 mai 2007 sous tutelle du ministère de la défense, le Service
hydrographique et océanographique de la marine (Shom) a pour mission de
connaître et de décrire l'environnement physique marin dans ses
relations avec l'atmosphère, avec les fonds marins et les zones
littorales, d'en prévoir l'évolution et d'assurer la diffusion des
informations correspondantes. L'exercice de cette mission se traduit par
trois activités primordiales, opérationnelles, orientées par leurs
finalités directes.

Il s'agit:

du soutien de la défense, caractérisé par l'expertise apportée par le
SHOM dans les domaines hydro-océanographiques à la direction générale de
l'armement et par ses capacités de soutien opérationnel des forces ;

de l'hydrographie nationale , pour satisfaire les besoins de la
navigation de surface, dans les eaux sous juridiction française et dans
les zones placées sous la responsabilité cartographique de la France ;

du soutien des politiques publiques de la mer et du littoral, par lequel
le SHOM valorise ses données patrimoniales et son expertise en les
mettant à la disposition des pouvoirs publics, et plus généralement de
tous les acteurs de la mer et du littoral.


\vspace{0.5cm}

\needspace{12\baselineskip}
\subsection*{Carte ancienne de la minute ``Ouessant, Le Four, Les Pierres Noires''
}\index{bathymetrie}\index{carte!ancienne}\index{iroise}\index{minute}\index{molene}\index{nature!de!fond}\index{topographie}\index{toponymie}
  \begin{wrapfigure}{r}{2.5cm}
    \centering
    \qrcode[nolink]{https://data.gouv.fr/dataset/53699025a3a729239d20391c}
  \end{wrapfigure}

Licence : \textbf{Licence Ouverte
}\newline
Créé le : 2014-03-12\newline
Modifié le : 2015-12-11\newline
Popularité : 1 réutilisation,  1 suivi\newline
Mots-clé : \emph{bathymetrie, carte-ancienne, iroise, minute, molene, nature-de-fond, topographie, toponymie
}\newline
Permalien : \url{https://data.gouv.fr/dataset/53699025a3a729239d20391c}\newline

\par
\noindent
    Carte numérique ``non géoréférencée'' issue du scannage de la minute
``Ouessant, Le Four, Les Pierres Noires'' de Beautemps-Beaupré
(Charles-François) en 1816, couvrant Balanec, Molène, Quéménes, Triélen,
La Helle, Béniguet (échelle non précisée). Latitude sud :
48\degree{}20``N Latitude nord : 48\degree{}25''N Longitude ouest :
005\degree{}05``W Longitude est : 004\degree{}50''W


\vspace{0.5cm}
\needspace{12\baselineskip}
\subsection*{Cartes Marines Anciennes (ARCHIVES)
}\index{archives}\index{cartes}\index{cartes!marines!anciennes}\index{maritimes}
  \begin{wrapfigure}{r}{2.5cm}
    \centering
    \qrcode[nolink]{https://data.gouv.fr/dataset/584eae3888ee382d86c65bb3}
  \end{wrapfigure}

Licence : \textbf{Licence Ouverte
}\newline
Créé le : 2016-12-12\newline
Modifié le : 2016-12-14\newline
De 1770-01-01 à 2016-12-12\newline
Mise à jour : ponctuelle\newline
Popularité : 1 réutilisation,  0 suivi\newline
Mots-clé : \emph{archives, cartes, cartes-marines-anciennes, maritimes
}\newline
Permalien : \url{https://data.gouv.fr/dataset/584eae3888ee382d86c65bb3}\newline

\par
\noindent
    Le SHOM met à disposition sous forme numérique ses archives de cartes
marines anciennes publiées depuis 1822 et n'étant plus utilisées pour la
navigation. Il s'agit donc de plusieurs milliers de documents,
majoritairement en noir et blanc, puis en couleurs à partir des années
1970.

Remarque : Aucune carte marine de navigation actuellement en vigueur
n'est proposée ici. De plus, ces documents étant obsolètes, ils ne
doivent pas être utilisés pour la navigation maritime.

Plusieurs éditions disponibles :

Pour un même numéro de carte, plusieurs éditions peuvent être proposées
:\\
- La publication ou première édition ; - Une ou des édition(s)
intermédiaire(s) ; - L'édition lors de sa suppression.

\begin{verbatim}
Formats :
\end{verbatim}

Les documents sont diffusés sous forme d'image au format JPEG2000. Le
logiciel utilisé par le SHOM est Kakadu

Seul quelques logiciels permettent de lire des fichiers JPEG2000. Parmi
ceux communément utilisés et gratuits IfranView, Gimp, safari\ldots{},
le SHOM utilise et préconise le logiciel gratuit de visualisation
associé à Kakadu Software
(\href{http://kakadusoftware.com/}{www.kakadusoftware.com}) kdu\_show.
Celui-ci est accessible aux liens suivants : - pour Windows 32 bits :
\href{http://kakadusoftware.com/wp-content/uploads/2014/06/KDU78_Demo_Apps_for_Win32_160226.msi_.zip}{ici}
- pour Mac :
\href{http://kakadusoftware.com/wp-content/uploads/2014/06/KDU78_Demo_Apps_for_OSX1011_160322.dmg_.zip}{ici}

Les fichiers au format JPEG2000 sont très facilement exploitables avec
le logiciel gratuit IGNMap
(\url{http://ignmap.ign.fr/}).{]}(http://ignmap.ign.fr/{]}(http://ignmap.ign.fr/)).)
L'image est également accompagnée par - Un fichier de métadonnées au
format xml contenant les informations sur le document; - Une image
miniature au format png; - Un fichier gml fournissant l'emprise
géographique générale de la carte.

Ces cartes sont - consultables
sur\url{http://data.shom.fr}et\url{http://diffusion.shom.fr}-
téléchargeables via la table ci jointe et
sur\url{http://data.shom.fr}et\url{http://diffusion.shom.fr.}


\vspace{0.5cm}
\needspace{12\baselineskip}
\subsection*{Epaves et obstructions sur la zone Iroise
}\index{epaves}\index{obstacles}\index{obstructions}
  \begin{wrapfigure}{r}{2.5cm}
    \centering
    \qrcode[nolink]{https://data.gouv.fr/dataset/536994afa3a729239d2044e7}
  \end{wrapfigure}

Licence : \textbf{Licence Ouverte
}\newline
Créé le : 2013-12-10\newline
Modifié le : 2018-05-24\newline
Popularité : 1 réutilisation,  1 suivi\newline
Mots-clé : \emph{epaves, obstacles, obstructions
}\newline
Permalien : \url{https://data.gouv.fr/dataset/536994afa3a729239d2044e7}\newline

\par
\noindent
    La base de données du SHOM contient environ 5400 épaves et 2600
obstructions réparties sur les côtes de France métropolitaine et
d'outre-mer. Outre la position géographique, toutes les caractéristiques
connues de l'épave ou de l'obstruction sont renseignées : son état, son
nom, sa nationalité, la date et les circonstances du naufrage\ldots{} La
base de données des épaves et des obstructions a pour vocation
principale de contribuer à la sécurité nautique, mais une épave est un
centre d'intérêt à divers égards : richesse de la vie sous-marine, étude
des habitats marins, archéologie, histoire des naufrages, danger de
croche pour le chalutage ou le dragage\ldots{} Jeu de données test en
Open Data sur la mer d'Iroise


\vspace{0.5cm}
\needspace{12\baselineskip}
\subsection*{Jeu de données bathymétriques du sondeur multi-faisceaux EM 710 - Zone
Iroise
}\index{bathymetrie}\index{epave}\index{ouessant}
  \begin{wrapfigure}{r}{2.5cm}
    \centering
    \qrcode[nolink]{https://data.gouv.fr/dataset/5369975ca3a729239d204c8c}
  \end{wrapfigure}

Licence : \textbf{Licence Ouverte
}\newline
Créé le : 2014-03-05\newline
Modifié le : 2015-07-06\newline
Popularité : 1 réutilisation,  1 suivi\newline
Mots-clé : \emph{bathymetrie, epave, ouessant
}\newline
Permalien : \url{https://data.gouv.fr/dataset/5369975ca3a729239d204c8c}\newline

\par
\noindent
    Le SHOM met a disposition les jeux de données bathymétriques suivants
acquis au sondeur multi-faisceaux EM 710 :

-Epave du Drummond Castle investiguée avec le BH2 La Pérouse par 55 m de
fond

-Fosse d'Ouessant et épave du sous-marin U-821 investiguées avec le BH2
Borda par 100 m de fond


\vspace{0.5cm}
\needspace{3\baselineskip} \rule{4cm}{0.25pt}\newline\textbf{Aussi disponible du même producteur :}\begin{itemize}
\item \href{https://data.gouv.fr/dataset/53698fe7a3a729239d20387b}{Câbles sous-marins sur la zone Iroise}
\item \href{https://data.gouv.fr/dataset/536991dba3a729239d203d89}{Courants de marée sur la zone Iroise}
\item \href{https://data.gouv.fr/dataset/53699210a3a729239d203e0c}{Dalles bathymétriques sur la zone Iroise}
\item \href{https://data.gouv.fr/dataset/53699332a3a729239d204118}{Données numériques vectorielles des cartes marines sur la zone Iroise}
\item \href{https://data.gouv.fr/dataset/53fd1f11a3a729390ba568a5}{Les types de marée dans le monde}
\item \href{https://data.gouv.fr/dataset/5c2f1f14634f415527798ab7}{Limite des affaires maritimes}
\item \href{https://data.gouv.fr/dataset/5c2f1f158b4c411b8b372484}{Limite de salure des eaux}
\item \href{https://data.gouv.fr/dataset/58511715c751df4f55c0bb7e}{Minutes de levés hydrographiques (ARCHIVES)}
\item \href{https://data.gouv.fr/dataset/54ca0e13c751df31e646738a}{MNT littoral Litto3D® - Éparses 2012}
\item \href{https://data.gouv.fr/dataset/54bfc88cc751df05f15fa5a2}{MNT littoral Litto3D® Finistère 2014}
\item \href{https://data.gouv.fr/dataset/54ca0c57c751df31e6467389}{MNT littoral Litto3D® - Guadeloupe 2013}
\item \href{https://data.gouv.fr/dataset/54bfd23dc751df240f5fa5a2}{MNT littoral Litto3D® - Languedoc-Roussillon 2009}
\item \href{https://data.gouv.fr/dataset/54ca0d13c751df2b0f467389}{MNT littoral Litto3D® - Martinique 2012}
\item \href{https://data.gouv.fr/dataset/54c9f689c751df0c4c467389}{MNT littoral Litto3D® - Mayotte 2012}
\item \href{https://data.gouv.fr/dataset/54f5d272c751df33fc882844}{MNT littoral Litto3D® - PACA 2015}
\item \href{https://data.gouv.fr/dataset/54ca0afec751df2e7a46738a}{MNT littoral Litto3D® - Réunion 2012}
\item \href{https://data.gouv.fr/dataset/53d97ab6a3a729019f37e2b6}{Morpho-sédimentologie et courants de marée 3D de la zone éolien posé de Saint-Nazaire}
\item \href{https://data.gouv.fr/dataset/53699a1ea3a729239d2053ee}{Morpho-sédimentologie et courants de marée 3D du passage du Fromveur}
\item \href{https://data.gouv.fr/dataset/53699a4da3a729239d205464}{Natures de fond au 50 000 sur la zone Iroise}
\item \href{https://data.gouv.fr/dataset/54ca15bfc751df3c2d467389}{Partie maritime MNT littoral Litto3D® - Languedoc-Roussillon 2011}
\item \href{https://data.gouv.fr/dataset/555b0b87c751df3f2f190c78}{Partie maritime MNT littoral Litto3D® - PACA 2014}
\item \href{https://data.gouv.fr/dataset/54ca16eac751df4155467389}{Partie maritime MNT littoral Litto3D® - Parc Naturel Marin d'Iroise 2012}
\item \href{https://data.gouv.fr/dataset/53699e9ca3a729239d205f05}{RasterMarine sur la zone Iroise}
\item \href{https://data.gouv.fr/dataset/5925720b88ee385b69385a47}{Références Altimétriques Maritimes}
\item \href{https://data.gouv.fr/dataset/53fd2816a3a729390ba568a8}{ Statistiques des niveaux marins extrêmes des côtes de France - Edition 2012}
\item \href{https://data.gouv.fr/dataset/53698f60a3a729239d203724}{Surfaces BATHYELLI Zéro Hydrographique / ellipsoide (CD/GRS80)}
\item \href{https://data.gouv.fr/dataset/5369a239a3a729239d2067bb}{Toponymes marins et côtiers sur la zone Iroise}
\item \href{https://data.gouv.fr/dataset/5847d779c751df7358c0bb7e}{Trait de côte Histolitt}
\item \href{https://data.gouv.fr/dataset/547d8d97c751df48e1090fca}{Zones de marée}
\end{itemize}

\clearpage
\section{SNCF}


\begin{center}
  \includegraphics[width=3cm]{images/orga/f3_0b8ad932f74086a6ab3f291ee9243f-100.png}
\end{center}


SNCF est l'un des premiers groupes mondiaux de mobilité et de
logistique, avec une présence dans 120 pays, 33,8 milliards d'euros de
chiffre d'affaires dont près de 25 \% à l'international et 250 000
collaborateurs en 2012. Groupe public à vocation de service public, fort
de son socle ferroviaire français, SNCF élargit l'offre des services de
transport afin de proposer une mobilité fluide et de porte à porte à ses
clients, voyageurs, chargeurs ou Autorités Organisatrices.


\vspace{0.5cm}

\needspace{12\baselineskip}
\subsection*{Cartographie OpenStreetMap des gares SNCF Transilien en Ile-de-France
}\index{equipement}\index{gare!de!voyageurs}\index{gares!et!infrastructures}\index{ile!de!france}\index{openstreetmap}
  \begin{wrapfigure}{r}{2.5cm}
    \centering
    \qrcode[nolink]{https://data.gouv.fr/dataset/5369903ca3a729239d20395d}
  \end{wrapfigure}

Licence : \textbf{Open Data Commons Open Database License (ODbL)
}\newline
Créé le : 2014-03-15\newline
Modifié le : 2019-03-17\newline
Granularité : au point d'intérêt\newline
Popularité : 4 réutilisations,  0 suivi\newline
Mots-clé : \emph{equipement, gare-de-voyageurs, gares-et-infrastructures, ile-de-france, openstreetmap
}\newline
Permalien : \url{https://data.gouv.fr/dataset/5369903ca3a729239d20395d}\newline

\par
\noindent
    Cartographie~\href{https://openstreetmap.fr/}{OpenStreetMap} des gares
Transilien en Ile-de-France.

Les données sont collectées par la communauté OpenStreetMap mais
également la~\href{http://www.sprint-je.com/}{Junior-Entreprise Sprint}
de Telecom SudParis et Telecom École de Management et le personnel de la
SNCF lui-même.\\
Ces données sont mises à jour~chaque nuit à 2 heures du matin.~

Les données collectées sont relatives aux :

\begin{itemize}

\item
  Équipements : automates de vente, bancs, écrans d'information, balises
  sonores \ldots{}
\item
  Cheminements : quais, portillons, ascenseurs, escaliers, bandes
  podo-tactiles \ldots{}
\item
  Services en gare : guichets d'information, photomatons, distributeurs,
  boîtes aux lettres \ldots{}
\item
  Points d'intérêt liés à la mobilité : parkings, arrêts de bus, bornes
  de taxi, passages piétons \ldots{}
\end{itemize}

La structure des données respecte le format du référentiel définit en
collaboration avec la communauté OpenStreetMap à la page
\url{http://wiki.openstreetmap.org/wiki/FR:Railway_stations\%3E} Vous
pouvez également exporter au format geoJSON, GPX ou KML
depuis~\url{http://overpass-turbo.eu/s/axF\%3E.} Mise à jour 13/12/3015
: ajout de la gare de Rosa Parks ouverte le jour même.

Mise à jour 30/03/2016 : nous procédons à la
\href{http://wiki.openstreetmap.org/wiki/WikiProject_France/Transilien}{création
de nombreuses données} issues de sources diverses et continueront dans
les semaines à venir.

Mise à jour 06/04/2016 : ajout des données des grandes gares parisiennes
: gare de Lyon, gare du Nord, gare de l'Est, Montparnasse, Paris
Austerlitz, Magenta, Haussmann.

Mise à jour 31/05/2016 : intégration récente de nombreuses données
issues des plans d'architectes et des plans de pôle affiché en gare.
\href{http://wiki.openstreetmap.org/wiki/WikiProject_France/Transilien}{Plus
de 70 gares sont concernées}.


\vspace{0.5cm}
\needspace{3\baselineskip} \rule{4cm}{0.25pt}\newline\textbf{Aussi disponible du même producteur :}\begin{itemize}
\item \href{https://data.gouv.fr/dataset/59593664a3a7291dd09c8269}{Abonnement national TER \&{} Intercités pour Elèves, Etudiants, Apprentis}
\item \href{https://data.gouv.fr/dataset/540f9af6a3a72928898af4a8}{Accidents passagers depuis 2008}
\item \href{https://data.gouv.fr/dataset/540f9b52a3a72928898af4aa}{Agents SNCF ayant leur repos le dimanche}
\item \href{https://data.gouv.fr/dataset/540f9b52a3a72928898af4a9}{Agents SNCF de nationalité étrangère}
\item \href{https://data.gouv.fr/dataset/540f9b53a3a72928898af4ab}{Agents SNCF en situation de handicap}
\item \href{https://data.gouv.fr/dataset/538348a0a3a72906c7ec5c44}{API micro-services Intercités}
\item \href{https://data.gouv.fr/dataset/538348a1a3a72906c7ec5c45}{​API micro-services TER}
\item \href{https://data.gouv.fr/dataset/538348a3a3a72906c7ec5c47}{API Prochains départs des gares Transilien}
\item \href{https://data.gouv.fr/dataset/5c5738629ce2e74f11cc53f4}{API temps réel Transilien}
\item \href{https://data.gouv.fr/dataset/59593616a3a7291dd09c8235}{Archives SNCF}
\item \href{https://data.gouv.fr/dataset/5b2201dfa3a7290138e6cc20}{AUTRES CHARGES EXTERNES 2017 CCO}
\item \href{https://data.gouv.fr/dataset/5b2201e1b595083acbf1acf5}{AUTRES CHARGES EXTERNES 2017 CSO}
\item \href{https://data.gouv.fr/dataset/59593629a3a7291dd09c8248}{Autres charges externes SNCF Réseau 2016 CCO}
\item \href{https://data.gouv.fr/dataset/59593629a3a7291dcf9c8287}{Autres charges externes SNCF Réseau 2016 CSO}
\item \href{https://data.gouv.fr/dataset/5b2201beb595083acbf1aced}{AVANTAGES AU PERSONNEL 2017 CCO}
\item \href{https://data.gouv.fr/dataset/59593665a3a7291dd09c826a}{Avantages au personnel SNCF Réseau 2016 CCO}
\item \href{https://data.gouv.fr/dataset/5959362da3a7291dcf9c828a}{Barème national de l'abonnement de travail TER et Intercités}
\item \href{https://data.gouv.fr/dataset/59593625a3a7291dcf9c8283}{Barème national kilométrique de prix TER}
\item \href{https://data.gouv.fr/dataset/59593624a3a7291dcf9c8281}{Baromètre notes d'opinion SNCF}
\item \href{https://data.gouv.fr/dataset/5c5739889ce2e75298cc53f4}{Baromètre satisfaction client en gare}
\item \href{https://data.gouv.fr/dataset/59593625a3a7291dcf9c8282}{Besoin en Fonds de Roulement exploitation SNCF Réseau 2016 CCO}
\item \href{https://data.gouv.fr/dataset/5b22018eb595083ac5f1aced}{BFR EXPLOITATION 2017 CCO}
\item \href{https://data.gouv.fr/dataset/5b2201e5b595083acbf1acf7}{BILAN ACTIF 2017 CCO}
\item \href{https://data.gouv.fr/dataset/5b2201dcb595083acbf1acf2}{BILAN ACTIF 2017 CSO}
\item \href{https://data.gouv.fr/dataset/5959366fa3a7291dd09c8271}{Bilan actif SNCF Réseau 2016 CCO}
\item \href{https://data.gouv.fr/dataset/59593666a3a7291dcf9c82ab}{Bilan actif SNCF Réseau 2016 CSO}
\item \href{https://data.gouv.fr/dataset/5b22018fa3a7297ffee6cc1f}{BILAN PASSIF 2017 CCO}
\item \href{https://data.gouv.fr/dataset/5b22018ea3a7297ffee6cc1e}{BILAN PASSIF 2017 CSO}
\item \href{https://data.gouv.fr/dataset/59593633a3a7291dd09c8252}{Bilan passif SNCF Réseau 2016 CCO}
\item \href{https://data.gouv.fr/dataset/59593634a3a7291dcf9c8291}{Bilan passif SNCF Réseau 2016 CSO}
\item \href{https://data.gouv.fr/dataset/5b2201dfb595083acbf1acf4}{CAPITAUX PROPRES RECYCLABLES 2017 CCO}
\item \href{https://data.gouv.fr/dataset/59593628a3a7291dcf9c8286}{Capitaux propres recyclables SNCF Réseau 2016 CCO}
\item \href{https://data.gouv.fr/dataset/5b2201b6b595083acbf1acea}{CAPITAUX PROPRES TABLEAUX 2017 CSO}
\item \href{https://data.gouv.fr/dataset/59593672a3a7291dd09c8273}{Capitaux propres tableaux SNCF Réseau 2016 CSO}
\item \href{https://data.gouv.fr/dataset/5959361ea3a7291dcf9c827b}{Caractéristique des voies et déclivité}
\item \href{https://data.gouv.fr/dataset/563dd03cb5950814b0588712}{Caractéristiques des abris vélo Véligo}
\item \href{https://data.gouv.fr/dataset/5959365fa3a7291dd09c8265}{Carte de restauration des bars TGV}
\item \href{https://data.gouv.fr/dataset/563dd039b5950814b058870f}{Cartographie OpenStreetMap des gares SNCF en région Lorraine}
\item \href{https://data.gouv.fr/dataset/59593637a3a7291dcf9c8293}{Cartographie OpenStreetMap des gares SNCF en région PACA}
\item \href{https://data.gouv.fr/dataset/59593660a3a7291dcf9c82a6}{Chantiers de transport combinés}
\item \href{https://data.gouv.fr/dataset/5b2201b1a3a7297ffee6cc28}{CHARGES DE PERSONNEL 2017 CCO}
\item \href{https://data.gouv.fr/dataset/5b2201c0b595083acbf1acef}{CHARGES DE PERSONNEL 2017 CSO}
\item \href{https://data.gouv.fr/dataset/59593661a3a7291dd09c8267}{Charges de personnel SNCF Réseau 2016 CCO}
\item \href{https://data.gouv.fr/dataset/59593662a3a7291dd09c8268}{Charges de personnel SNCF Réseau 2016 CSO}
\item \href{https://data.gouv.fr/dataset/59593659a3a7291dcf9c82a0}{Classification d'armement des voies}
\item \href{https://data.gouv.fr/dataset/53834b39a3a72906c7ec5c4b}{Codes couleur des lignes Transilien}
\item \href{https://data.gouv.fr/dataset/540f9ed6a3a72928898af4ac}{Collisions sur les passages à niveau depuis 2004}
\item \href{https://data.gouv.fr/dataset/563dd03ca3a729642bec3cc1}{Comptage des voyageurs montants dans les trains Transilien}
\item \href{https://data.gouv.fr/dataset/5b2201b2a3a7297ffee6cc29}{COMPTE DE RESULTAT 2017 CSO}
\item \href{https://data.gouv.fr/dataset/5959362ba3a7291dd09c824a}{Compte de résultat SNCF Réseau 2016 CSO}
\item et 191 autres jeux de données\end{itemize}

\clearpage
\section{Société du Grand Paris}


\begin{center}
  \includegraphics[width=3cm]{images/orga/38_676967f8a549ed9bf9ac65006ea64a-100.png}
\end{center}


Créée par la loi du 3 juin 2010 relative au Grand Paris,
\href{http://www.societedugrandparis.fr/}{la Société du Grand Paris} est
un établissement public de l'État à caractère industriel et commercial.
Elle a pour mission principale de concevoir et de réaliser les projets
d'infrastructures qui composent le réseau de transport public du Grand
Paris. Par ailleurs, la Société du Grand Paris assiste le préfet de la
région Île-de-France pour la préparation et la mise en cohérence des
contrats de développement territorial passés entre l'État, les communes
et les intercommunalités. Elle peut également conduire des opérations
d'aménagement et de construction.
\href{http://www.societedugrandparis.fr/focus/ambition-numerique/la-demarche-open-data}{La
politique open data}de la Société du Grand Paris est mise en place dans
le cadre de
\href{http://www.societedugrandparis.fr/focus/ambition-numerique/inventons-metro-digital-du-monde}{son
projet numérique}.


\vspace{0.5cm}

\needspace{12\baselineskip}
\subsection*{Fuseau de la zone d'intervention potentielle de la ligne 15 sud, 15
ouest et ligne 16
}\index{futur}\index{gpe}\index{grand!paris}\index{grand!paris!express}\index{metro}\index{sgp}\index{societe!du!grand!paris}\index{transport}\index{transport!public}
  \begin{wrapfigure}{r}{2.5cm}
    \centering
    \qrcode[nolink]{https://data.gouv.fr/dataset/54eee954c751df073339d677}
  \end{wrapfigure}

Licence : \textbf{Licence Ouverte
}\newline
Créé le : 2015-02-26\newline
Modifié le : 2017-04-07\newline
Mise à jour : ponctuelle\newline
Popularité : 1 réutilisation,  0 suivi\newline
Mots-clé : \emph{futur, gpe, grand-paris, grand-paris-express, metro, sgp, societe-du-grand-paris, transport, transport-public
}\newline
Permalien : \url{https://data.gouv.fr/dataset/54eee954c751df073339d677}\newline

\par
\noindent
    Cette couche SIG représente la Zone d'intervention potentielle (ZIP)
présentée dans la déclaration d'utilité publique de la ligne 15 sud, 15
ouest et de la ligne 16.

Description

\begin{verbatim}
Format : Fichier de forme Esri (Shp)
Source : Société du Grand Paris
Date   : 24 décembre 2014
Objet  : Polygone, Nombre d'objets : 1
Projection : RGF93-CC49 
\end{verbatim}

\textbf{Avertissement}

Les données sont indicatives au moment de leur publication. Elles sont
susceptibles d'être mises à jour tout au long de l'avancement de la
conception des projets d'infrastructure du Réseau de Transport Public du
Grand Paris. La cartographie ne peut pas à elle seule rendre compte de
la situation précise. Il appartient à l'organisme ayant droit de
s'assurer de l'adéquation des données à ses propres besoins. La
consultation des plans annexés aux décrets de DUP et aux arrêtés
préfectoraux (arrêtés de cessibilité notamment) demeure indispensable
pour une parfaite et exacte information. En effet, la cartographie
fournie ne peut être opposée à ces documents de référence et seuls les
plans qui sont annexés aux déclarations d'utilité publique et aux
arrêtés de cessibilité sont juridiquement contraignants et opposables.
Les données diffusées sont la propriété de la Société du Grand Paris.


\vspace{0.5cm}
\needspace{12\baselineskip}
\subsection*{Grand Paris Express et lieux culturels
}\index{atlas}\index{atlas!regional}\index{cinema}\index{culture}\index{culture!societe!service!tourisme}\index{grand!paris}\index{grand!paris!express}\index{metropole}\index{musee}\index{musee!de!france}\index{patrimoine}\index{patrimoine!architectural!et!urba}\index{patrimoine!culturel}\index{patrimoine!naturel}\index{salle!d!exposition}\index{salle!de!spectacle}\index{spectacle!vivant}\index{theatre}
  \begin{wrapfigure}{r}{2.5cm}
    \centering
    \qrcode[nolink]{https://data.gouv.fr/dataset/5627359288ee38145412613c}
  \end{wrapfigure}

Licence : \textbf{Open Data Commons Open Database License (ODbL)
}\newline
Créé le : 2015-10-21\newline
Modifié le : 2016-03-14\newline
De 2015-06-01 à 2015-06-25\newline
Granularité : à la région\newline
Mise à jour : ponctuelle\newline
Popularité : 2 réutilisations,  0 suivi\newline
Mots-clé : \emph{atlas, atlas-regional, cinema, culture, culture-societe-service-tourisme, grand-paris, grand-paris-express, metropole, musee, musee-de-france, patrimoine, patrimoine-architectural-et-urba, patrimoine-culturel, patrimoine-naturel, salle-d-exposition, salle-de-spectacle, spectacle-vivant, theatre
}\newline
Permalien : \url{https://data.gouv.fr/dataset/5627359288ee38145412613c}\newline

\par
\noindent
    Ce jeu de données est issu de l'atlas des lieux culturels du Grand Paris
réalisé par L'Atelier parisien d'urbanisme (Apur), la Société du Grand
Paris (SGP) et la Direction Régionale des Affaires Culturelles
d'Ile-de-France (DRAC) dans le cadre de l'étude
\href{http://www.apur.org/sites/default/files/documents/grand_paris_express_lieux_culturels.pdf}{Grand
Paris Express et lieux culturels}.

Cette base de données a vocation à être partagée et enrichie dans un
esprit collaboratif.


\vspace{0.5cm}
\needspace{12\baselineskip}
\subsection*{Point de localisation des gares du Grand Paris Express
}\index{futur}\index{gpe}\index{grand!paris!express}\index{metro}\index{sgp}\index{societe!du!grand!paris}\index{transport}\index{transport!public}
  \begin{wrapfigure}{r}{2.5cm}
    \centering
    \qrcode[nolink]{https://data.gouv.fr/dataset/54eee9a5c751df08ed39d677}
  \end{wrapfigure}

Licence : \textbf{Licence Ouverte
}\newline
Créé le : 2015-02-26\newline
Modifié le : 2016-12-19\newline
Mise à jour : ponctuelle\newline
Popularité : 1 réutilisation,  1 suivi\newline
Mots-clé : \emph{futur, gpe, grand-paris-express, metro, sgp, societe-du-grand-paris, transport, transport-public
}\newline
Permalien : \url{https://data.gouv.fr/dataset/54eee9a5c751df08ed39d677}\newline

\par
\noindent
    Cette couche SIG représente les points de localisation de l'ensemble des
68 gares du projet du Grand Paris Express.

\textbf{Description}

\begin{verbatim}
Format : Fichier de forme Esri (Shp)
Source: Société du Grand Paris
Date  : juin 2016
Objet  : Points 
Nombre d'objets : 25
Projection : RGF93-CC49
\end{verbatim}

\textbf{Avertissement}

Les données sont indicatives au moment de leur publication. Elles sont
susceptibles d'être mises à jour tout au long de l'avancement de la
conception des projets d'infrastructure du Réseau de Transport Public du
Grand Paris. La cartographie ne peut pas à elle seule rendre compte de
la situation précise. Il appartient à l'organisme ayant droit de
s'assurer de l'adéquation des données à ses propres besoins. La
consultation des plans annexés aux décrets de DUP et aux arrêtés
préfectoraux (arrêtés de cessibilité notamment) demeure indispensable
pour une parfaite et exacte information. En effet, la cartographie
fournie ne peut être opposée à ces documents de référence et seuls les
plans qui sont annexés aux déclarations d'utilité publique et aux
arrêtés de cessibilité sont juridiquement contraignants et opposables.
Les données diffusées sont la propriété de la Société du Grand Paris.


\vspace{0.5cm}
\needspace{12\baselineskip}
\subsection*{Temps de parcours intergares prévisionnels
}\index{grand!paris}\index{grand!paris!express}\index{metro}\index{transport}\index{transport!public}
  \begin{wrapfigure}{r}{2.5cm}
    \centering
    \qrcode[nolink]{https://data.gouv.fr/dataset/54e61b83c751df775b467389}
  \end{wrapfigure}

Licence : \textbf{Licence Ouverte
}\newline
Créé le : 2015-02-19\newline
Modifié le : 2016-03-15\newline
Mise à jour : ponctuelle\newline
Popularité : 1 réutilisation,  1 suivi\newline
Mots-clé : \emph{grand-paris, grand-paris-express, metro, transport, transport-public
}\newline
Permalien : \url{https://data.gouv.fr/dataset/54e61b83c751df775b467389}\newline

\par
\noindent
    Les données présentées dans ce jeu de données sont les temps de parcours
prévisionnels entre deux gares sur chacun des tronçons de ligne
constitutifs du réseau de transport public du Grand Paris. Les temps de
parcours présentés intègrent le temps d'arrêt des trains à chaque gare.

\textbf{--- Avertissement ---}

Les données sont indicatives au moment de leur publication. Elles sont
susceptibles d'être mises à jour tout au long de l'avancement de la
conception des projets d'infrastructure du Réseau de Transport Public du
Grand Paris. La cartographie ne peut pas à elle seule rendre compte de
la situation précise. Il appartient à l'organisme ayant droit de
s'assurer de l'adéquation des données à ses propres besoins. La
consultation des plans annexés aux décrets de DUP et aux arrêtés
préfectoraux (arrêtés de cessibilité notamment) demeure indispensable
pour une parfaite et exacte information. En effet, la cartographie
fournie ne peut être opposée à ces documents de référence et seuls les
plans qui sont annexés aux déclarations d'utilité publique et aux
arrêtés de cessibilité sont juridiquement contraignants et opposables.
Les données diffusées sont la propriété de la Société du Grand Paris.


\vspace{0.5cm}
\needspace{3\baselineskip} \rule{4cm}{0.25pt}\newline\textbf{Aussi disponible du même producteur :}\begin{itemize}
\item \href{https://data.gouv.fr/dataset/5548c77dc751df0245a7b26c}{Centralité et linéaire commercial dans le secteur des gares}
\item \href{https://data.gouv.fr/dataset/54e355f5c751df64ee467389}{Fréquences prévisionnelles des trains aux heures de pointe du matin}
\item \href{https://data.gouv.fr/dataset/578cfb6e88ee3843047d97c2}{Inventaire de la faune et de la flore le long du fuseau du Grand Paris Express}
\item \href{https://data.gouv.fr/dataset/54e61cd1c751df75ad467389}{Noms et codes des gares}
\item \href{https://data.gouv.fr/dataset/58e7536188ee3850f271e773}{Notice relative aux données numériques recueillies dans le cadre de l'Observatoire des quartiers de gares du Grand Paris}
\item \href{https://data.gouv.fr/dataset/5824a256c751df0613c0bb7e}{Sondages géotechniques ligne 15 sud du Grand Paris Express}
\item \href{https://data.gouv.fr/dataset/54eef002c751df0fde39d677}{Temps de correspondance prévisionnels dans les gares}
\end{itemize}

\clearpage
\section{Syndicat des Transports d'Ile-de-France}


\begin{center}
  \includegraphics[width=3cm]{images/orga/11_36458e54d84094a7fcca2053afb162-100.png}
\end{center}


Le \textbf{Syndicat des Transports d'Ile-de-France} est l'Autorité
Organisatrice de la mobilité durable en Ile-de-France. Il imagine,
organise et finance les transports publics pour tous les Franciliens.
Chaque jour, ce sont plus de \textbf{10 millions de voyageurs} qui
empruntent le réseau de transport francilien, opéré par les 75
entreprises OPTILE, la RATP et la SNCF.

\href{http://opendata.stif.info/}{Le portail Open data du Syndicat des
Transports d'Ile-de-France}.


\vspace{0.5cm}

\needspace{12\baselineskip}
\subsection*{Stations et espaces AutoLib de la métropole parisienne
}\index{autolib}\index{infrastructure!intermodalite}\index{intermodalite}\index{voiture}
  \begin{wrapfigure}{r}{2.5cm}
    \centering
    \qrcode[nolink]{https://data.gouv.fr/dataset/56f26f0eb595087c67b9b300}
  \end{wrapfigure}

Licence : \textbf{Open Data Commons Open Database License (ODbL)
}\newline
Créé le : 2016-03-23\newline
Modifié le : 2018-08-07\newline
Popularité : 1 réutilisation,  0 suivi\newline
Mots-clé : \emph{autolib, infrastructure-intermodalite, intermodalite, voiture
}\newline
Permalien : \url{https://data.gouv.fr/dataset/56f26f0eb595087c67b9b300}\newline

\par
\noindent
    Ce jeu de données fournit~pour l'ensemble des stations Autolib'
d'Ile-de-France, leur positionnement géographique et leur capacité
d'accueil de véhicules.~

Définition:

-~ID:~Ce numero ID est composé d'un code commune suivi d'un code numéro
station par commune.

-~Identifiant Autolib':~Cette identifiant est compsé du nom de la
commune d'implantation de la station, du nom de la rue et du numéro de
rue.

-~Rue:~Numéro de rue et rue de la station

-~Code postal:~Code postal de la commune d'implantation de la station

-~Ville :Commune d'implantation de la station

-~Emplacement:~Espace d'emplacement de la station (Voirie ou Parking)

-~Autolib':~Nombre de places destinées aux véhicules Autolib'

-~Tiers:~Nombre de places destinées aux véhicules Tiers

-~Abri:~Précision sur le fait qu'il y ai ou non un espace d'abonnement
sur le lieu de la station (1 = oui, 0 = non).


\vspace{0.5cm}
\needspace{3\baselineskip} \rule{4cm}{0.25pt}\newline\textbf{Aussi disponible du même producteur :}\begin{itemize}
\item \href{https://data.gouv.fr/dataset/5c5911e29ce2e70433cc53f4}{Aménagements vélo en Île-de-France}
\item \href{https://data.gouv.fr/dataset/56f26f0fa3a729648fbc2069}{Arrêts par lignes de transport en commun en Ile-de-France}
\item \href{https://data.gouv.fr/dataset/5c56fe0b9ce2e72c1acc53f4}{Cartes des travaux 2019}
\item \href{https://data.gouv.fr/dataset/56f26f0ca3a7296490bc205f}{Description et tarif des titres de transport en Ile-de-France}
\item \href{https://data.gouv.fr/dataset/56f26f0ea3a729648fbc2065}{Gares et stations du réseau ferré d'Île-de-France (donnée généralisée)}
\item \href{https://data.gouv.fr/dataset/56f26f0cb595087c66b9b2f8}{Gares et stations du réseau ferré d'Île-de-France (par ligne)}
\item \href{https://data.gouv.fr/dataset/59592872a3a7291dcf9c8247}{Gares et stations du réseau ferré schématique d'Île-de-France (grand format)}
\item \href{https://data.gouv.fr/dataset/5959286ea3a7291dd09c81dc}{Gares et stations du réseau ferré schématique d'Île-de-France (petit format)}
\item \href{https://data.gouv.fr/dataset/56f26f0db595087c66b9b2f9}{Gares, stations et pôles d'échanges multimodaux en projet en Ile-de-France}
\item \href{https://data.gouv.fr/dataset/5b0f77fba3a7294d8d124e74}{Historique des données de validation (2015-2017)}
\item \href{https://data.gouv.fr/dataset/56f26f0da3a7296490bc2061}{Lieux d'Arrêt d'Île-de-France (LDA)}
\item \href{https://data.gouv.fr/dataset/56f26f0ea3a729648fbc2066}{Lignes de transport en commun déclarées accessibles (UFR) en Ile-de-France}
\item \href{https://data.gouv.fr/dataset/56f26f0fa3a729648fbc2068}{Lignes de transport en projet en Ile-de-France}
\item \href{https://data.gouv.fr/dataset/56f26f0fb595087c67b9b302}{Liste des transporteurs exploitant des lignes de transport en commun en Ile-de-France}
\item \href{https://data.gouv.fr/dataset/5ac72070b59508040fc04a3c}{Nuage de points de calage}
\item \href{https://data.gouv.fr/dataset/56f26f0ea3a729648fbc2067}{Parcs Relais en Île-de-France}
\item \href{https://data.gouv.fr/dataset/5959286ea3a7291dcf9c8245}{Périmètre des données temps réel disponibles  sur la plateforme d'échanges Île-de-France Mobilités}
\item \href{https://data.gouv.fr/dataset/59d323e5a3a7291993b1eb02}{Plan Métro - Petit format}
\item \href{https://data.gouv.fr/dataset/56f26f0eb595087c67b9b301}{Plan régional des transports en commun en Ile-de-France - Grand format}
\item \href{https://data.gouv.fr/dataset/56f26f0db595087c66b9b2fa}{Plan régional des transports en commun en Ile-de-France - Moyen format}
\item \href{https://data.gouv.fr/dataset/56f26f0da3a7296490bc2060}{Plan régional des transports en commun en Ile-de-France - Petit format}
\item \href{https://data.gouv.fr/dataset/5acc661ab59508482dc04a3d}{Pôles d'échanges en projet en Île-de-France}
\item \href{https://data.gouv.fr/dataset/56f26f0ea3a7296490bc2064}{Référentiel des arrêts de transport en commun en Ile-de-France}
\item \href{https://data.gouv.fr/dataset/56f26f0fa3a7296490bc2065}{Référentiel des lignes de transport en commun d'Ile-de-France}
\item \href{https://data.gouv.fr/dataset/5c5911e906e3e7246e232bba}{Stationnement vélo en Île-de-France}
\item \href{https://data.gouv.fr/dataset/56f26f0cb595087c66b9b2f7}{Stations bus en Île-de-France}
\item \href{https://data.gouv.fr/dataset/56f26f0eb595087c66b9b2fc}{Stations de vélo en libre service (VélO2 et Cristolib) - Disponibilités en temps réel}
\item \href{https://data.gouv.fr/dataset/5c318dd406e3e771f7552c59}{Subventions Ile-de-France Mobilités}
\item \href{https://data.gouv.fr/dataset/5b7e6e219ce2e752222daa23}{Tracés des lignes régulières de bus en Ile-de-France}
\item \href{https://data.gouv.fr/dataset/56f26f0db595087c67b9b2fc}{Tracés du réseau de transport ferré d'Ile-de-France}
\item \href{https://data.gouv.fr/dataset/5acc6615b59508482dc04a3c}{Tracés du réseau de transport ferré d'Ile-de-France (schématique)}
\item \href{https://data.gouv.fr/dataset/59592871a3a7291dcf9c8246}{Tracés schématiques du réseau de transport ferré d'Ile-de-France (grand format)}
\item \href{https://data.gouv.fr/dataset/59592872a3a7291dd09c81dd}{Tracés schématiques du réseau de transport ferré d'Ile-de-France (petit format)}
\item \href{https://data.gouv.fr/dataset/56f26f0db595087c67b9b2fe}{Validations sur le réseau de surface : Nombre de validations par jour (1er semestre 2018)}
\item \href{https://data.gouv.fr/dataset/56f26f0db595087c67b9b2fd}{Validations sur le réseau de surface : Nombre de validations par jour (2e semestre 2018)}
\item \href{https://data.gouv.fr/dataset/56f26f0eb595087c67b9b2ff}{Validations sur le réseau de surface : Profils horaires par jour type (1er semestre 2018)}
\item \href{https://data.gouv.fr/dataset/56f26f0da3a729648fbc2064}{Validations sur le réseau de surface : Profils horaires par jour type (2e semestre 2018)}
\item \href{https://data.gouv.fr/dataset/56f26f0bb595087c67b9b2fb}{Validations sur le réseau ferré : Nombre de validations par jour (1er semestre 2018)}
\item \href{https://data.gouv.fr/dataset/56f26f0eb595087c66b9b2fd}{Validations sur le réseau ferré : Nombre de validations par jour (2e semestre 2018)}
\item \href{https://data.gouv.fr/dataset/56f26f0fb595087c66b9b2fe}{Validations sur le réseau ferré : Profils horaires par jour type (2e semestre 2018)}
\item \href{https://data.gouv.fr/dataset/56f26f0ea3a7296490bc2063}{Zones de Lieu d'Île-de-France (ZDL)}
\item \href{https://data.gouv.fr/dataset/56f26f0da3a729648fbc2063}{Zones d'Embarquement d'Île-de-France (ZDE)}
\end{itemize}

\clearpage
\section{Système d'Information sur l'Eau}


\begin{center}
  \includegraphics[width=3cm]{images/orga/fe_ff49ef93b74da8ae9c1d211ba3dbef-100.png}
\end{center}


Le système d'information sur l'eau (SIE) vise au recueil, à la
conservation et à la diffusion des données et des indicateurs sur l'eau,
les milieux aquatiques, leurs usages et les services publics de
distribution d'eau et d'assainissement, conformément à l'article L.
213-2 du code de l'environnement. C'est un système d'information
partenarial, coordonnant différents acteurs, selon une organisation
définie par le Schéma national des données sur l'eau, approuvé par
l'Arrêté du 26 juillet 2010.


\vspace{0.5cm}

\needspace{12\baselineskip}
\subsection*{Stations de traitement des eaux usées - France entière
}\index{donnees!ouvertes}\index{environment}\index{france}\index{guadeloupe}\index{guyane}\index{martinique}\index{mayotte}\index{metropole}\index{ouvrage}\index{passerelle!inspire}\index{politique!de!lenvironnement}\index{reunion}\index{station!depuration}
  \begin{wrapfigure}{r}{2.5cm}
    \centering
    \qrcode[nolink]{https://data.gouv.fr/dataset/5369a03fa3a729239d20630a}
  \end{wrapfigure}

Licence : \textbf{Licence Ouverte version 2.0
}\newline
Créé le : 2013-09-14\newline
Modifié le : 2019-03-08\newline
Popularité : 2 réutilisations,  1 suivi\newline
Mots-clé : \emph{donnees-ouvertes, environment, france, guadeloupe, guyane, martinique, mayotte, metropole, ouvrage, passerelle-inspire, politique-de-lenvironnement, reunion, station-depuration
}\newline
Permalien : \url{https://data.gouv.fr/dataset/5369a03fa3a729239d20630a}\newline

\par
\noindent
    Le système de traitement d'eaux usées est un ouvrage de dépollution des
eaux usées par des procédés divers : biologiques,
physico-chimiques\ldots{} localisés sur un espace géographique continu
et homogène. Elle est urbaine ou industrielle en fonction de la nature
du maître d'ouvrage. Plus concrètement, quelles que soient les
configurations, une station d'épuration est tout l'espace géographique
``délimité par la clôture'' contenant un ensemble de constructions de
génie civil dotée``s d'appareillages et appartenant à un seul maître
d'ouvrage. Le système de traitement d'eaux usées comprend la station
d'épuration et le déversoir en tête de station (ouvrage du système de
traitement qui permet de dériver tout ou partie des effluents qui
arrivent à la station). Le constructeur global de la station d'épuration
est la désignation sociale de la principale société de BTP qui a
construit la station d'épuration. Quand plusieurs sociétés sont
intervenues dans la construction de la station d'épuration, c'est celle
qui a réalisé la part la plus importante des travaux qui sera retenue.
Quand une station d'épuration a fait l'objet de plusieurs programme de
travaux, c'est le dernier constructeur principal qui est pris en compte.
Les information sur les systèmes de traitement d'eaux usées relèvent de
la responsabilité des Agences de l'eau.

\textbf{Origine}

Mise à jour sur la base de la base de données BDERU du ministère chargé
du développement durable.

\textbf{Organisations partenaires}

SANDRE, Ministère chargé de l'environnement - Direction de l'Eau et de
la Biodiversité

\textbf{Liens annexes}

\begin{itemize}

\item
  \href{http://assainissement.developpement-durable.gouv.fr/}{Consulter
  le portail d'information sur l'assainissement communal}
\item
  \href{http://www.sandre.eaufrance.fr/?urn=urn:sandre:dictionnaire:ODP::entite:SysTraitementEauxUsees:ressource:latest:::html}{Consulter
  la documentation sur le site Sandre}
\end{itemize}

➞
\href{https://geo.data.gouv.fr/fr/datasets/fe75ec25ccdc46a2d54d9e5d1a9b05eb204e1ff7}{Consulter
cette fiche sur geo.data.gouv.fr}


\vspace{0.5cm}
\needspace{12\baselineskip}
\subsection*{Zones de production conchylicole - Métropole
}\index{conchyliculture}\index{directive!2000!60!ec}\index{donnees!ouvertes}\index{france}\index{inland!waters}\index{metropole}\index{passerelle!inspire}\index{politique!de!lenvironnement}\index{zonage}\index{zone!protegee}
  \begin{wrapfigure}{r}{2.5cm}
    \centering
    \qrcode[nolink]{https://data.gouv.fr/dataset/5369a3a0a3a729239d206ae3}
  \end{wrapfigure}

Licence : \textbf{Licence Ouverte version 2.0
}\newline
Créé le : 2013-09-14\newline
Modifié le : 2019-03-05\newline
Popularité : 2 réutilisations,  1 suivi\newline
Mots-clé : \emph{conchyliculture, directive-2000-60-ec, donnees-ouvertes, france, inland-waters, metropole, passerelle-inspire, politique-de-lenvironnement, zonage, zone-protegee
}\newline
Permalien : \url{https://data.gouv.fr/dataset/5369a3a0a3a729239d206ae3}\newline

\par
\noindent
    Les zones de production conchylicole sont identifiées au titre du paquet
européen hygiène (CE/854/2004) et de l'arrêté du 21 mai 1999 relatif au
classement de salubrité et à la surveillance des zones de production et
des zones de reparcage des coquillages vivants. L'ensemble des zones de
production de coquillages (zones d'élevage et de pêche professionnelle)
fait ainsi l'objet d'un classement sanitaire, défini par arrêté
préfectoral. Celui-ci est établi sur la base d'analyses des coquillages
présents : analyses microbiologiques utilisant Escherichia coli (E.
coli) comme indicateur de contamination (en nombre d'E. coli pour 100 g
de chair et de liquide intervalvaire - CLI) et dosage de la
contamination en métaux lourds (plomb, cadmium et mercure), exprimé en
mg/kg de chair humide. Le classement et le suivi des zones de production
de coquillages distingue 3 groupes de coquillages au regard de leur
physiologie : - groupe 1 : les gastéropodes (bulots etc.), les
échinodermes (oursins) et les tuniciers (violets) ; plus généralement
des coquillages sauvages de gisements naturels, - groupe 2 : les
bivalves fouisseurs, c'est-à-dire les mollusques bivalves filtreurs dont
l'habitat est constitué par les sédiments (palourdes, coques\ldots{}) ;
plus généralement des coquillages sauvages de gisements naturels, -
groupe 3 : les bivalves non fouisseurs, c'est-à-dire les autres
mollusques bivalves filtreurs (huîtres, moules\ldots{}) ; plus
généralement des coquillages d'élevage.

Ce concept est bien distinct de celui de `Zone de qualité des eaux
conchylicoles'. Le contour de la Zone de production conchylicole ne
correspond pas au cadastre conchylicole (= cadastre des établissements
de culture marine).

\textbf{Origine}

Mise à jour sur la base des informations transmises par les services de
l'Etat

\textbf{Organisations partenaires}

Office Internationale de l'Eau (OIEAU), Directions Départementales des
Territoires et de la Mer (DDTM), Service d'administration national des
données et référentiels sur l'eau (Sandre), Agence française pour la
biodiversité (AFB) - Pôle de Vincennes (ex Office national de l'eau et
des milieux aquatiques)

\textbf{Liens annexes}

\begin{itemize}

\item
  \href{http://www.zones-conchylicoles.eaufrance.fr/}{Atlas des zones
  conchylicoles}
\item
  \href{http://www.sandre.eaufrance.fr/?urn=urn:sandre:dictionnaire:ZON::entite:ZoneProdConchy:ressource:latest:::html}{Consulter
  la documentation sur le site Sandre}
\item
  \href{http://www.sandre.eaufrance.fr/?urn=urn:sandre:dictionnaire:ZON::::ressource:latest:::xml}{Consulter
  le dictionnaire ``Zonages techniques et réglementaires du domaine de
  l'eau (ZON)'' sur le site Sandre}
\end{itemize}

➞
\href{https://geo.data.gouv.fr/fr/datasets/6d7e241278efea04675c1a3da449cc4f125a2f71}{Consulter
cette fiche sur geo.data.gouv.fr}


\vspace{0.5cm}
\needspace{3\baselineskip} \rule{4cm}{0.25pt}\newline\textbf{Aussi disponible du même producteur :}\begin{itemize}
\item \href{https://data.gouv.fr/dataset/5c0e301d8b4c4169b87962a2}{Aires d'Alimentation de Captage - France entière}
\item \href{https://data.gouv.fr/dataset/5c0e301e634f413525ba4dad}{Aires d'Alimentation de Captage - Métropole}
\item \href{https://data.gouv.fr/dataset/55277bbe88ee385c18756379}{Bassins DCE}
\item \href{https://data.gouv.fr/dataset/5a0aeb8588ee383ee1c3fd43}{bathymétrie plans d'eau}
\item \href{https://data.gouv.fr/dataset/552cdd8788ee38437785ce60}{BD Carthage Guyane / Cours d'eau - 2011}
\item \href{https://data.gouv.fr/dataset/597b451e88ee383d3c10c6ad}{BD Carthage Guyane / Noeuds hydrographiques - 2011}
\item \href{https://data.gouv.fr/dataset/597b451bc751df0709bff263}{BD Carthage Guyane / Points d'eau isolés - 2011}
\item \href{https://data.gouv.fr/dataset/552cdd8dc751df565937dddb}{BD Carthage Guyane / Régions hydrographiques - 2011}
\item \href{https://data.gouv.fr/dataset/552cdda3c751df565c37dddd}{BD Carthage Guyane / Secteurs hydrographiques - 2011}
\item \href{https://data.gouv.fr/dataset/552cdd90c751df565c37dddb}{BD Carthage Guyane / Sous-secteurs hydrographiques - 2011}
\item \href{https://data.gouv.fr/dataset/597b453d88ee383d4357929a}{BD Carthage Guyane / Trait de côte - 2011}
\item \href{https://data.gouv.fr/dataset/597b454088ee383d2110a724}{BD Carthage Guyane / Tronçons hydrographiques élémentaires - 2011}
\item \href{https://data.gouv.fr/dataset/552cdda4c751df514137dddb}{BD Carthage Guyane / Zones hydrographiques - 2011}
\item \href{https://data.gouv.fr/dataset/597b450388ee383927b14c48}{BDLISA Métropole / Entités hydrogéologiques}
\item \href{https://data.gouv.fr/dataset/584f2156c751df3874c0bb7e}{Chaînage des zones hydrographiques - Métropole}
\item \href{https://data.gouv.fr/dataset/552cdd9488ee38488f85ce61}{Circonscriptions administratives de bassin}
\item \href{https://data.gouv.fr/dataset/597b450588ee383927b14c49}{Communes administratives au 1er janvier 2017}
\item \href{https://data.gouv.fr/dataset/5c0e3019634f4133677dc459}{Concentrations en nitrates d'origine agricole - 2013/2014 - Eaux superficielles - France entière}
\item \href{https://data.gouv.fr/dataset/5c0e30198b4c416baa87d244}{Concentrations en nitrates d'origine agricole - 2014/2015 - Eaux souterraines - France entière}
\item \href{https://data.gouv.fr/dataset/5c0e301e8b4c4169ea9eec35}{Concentrations en nitrates d'origine agricole - 2014/2015 - Eaux superficielles - France entière}
\item \href{https://data.gouv.fr/dataset/5c0e301c8b4c416bb21ddddb}{Concentrations en nitrates d'origine agricole - 2015/2016 - Eaux souterraines - France entière}
\item \href{https://data.gouv.fr/dataset/5c0e30198b4c416bb21dddda}{Concentrations en nitrates d'origine agricole - 2015/2016 - Eaux superficielles - France entière}
\item \href{https://data.gouv.fr/dataset/584f215b88ee380452c65bb5}{Contextes piscicoles - Métropole}
\item \href{https://data.gouv.fr/dataset/597b4509c751df06dd254d45}{Contrats de Milieu - Guadeloupe}
\item \href{https://data.gouv.fr/dataset/597b450ac751df0709bff25e}{Contrats de Milieu - Martinique}
\item \href{https://data.gouv.fr/dataset/597b4506c751df0709bff25d}{Contrats de Milieu - Métropole}
\item \href{https://data.gouv.fr/dataset/552cdd9dc751df565c37dddc}{Contrats de milieu Métropôle}
\item \href{https://data.gouv.fr/dataset/5522773988ee38026e45f4fd}{Cours d'eau - Guadeloupe 2009 - BD Carthage}
\item \href{https://data.gouv.fr/dataset/5c0e301c634f4133465f3fb1}{Cours d'eau - Guyane 2015 - BD Carthage}
\item \href{https://data.gouv.fr/dataset/597b450888ee383927b14c4a}{Cours d'eau - Martinique 2004 - BD Carthage}
\item \href{https://data.gouv.fr/dataset/597b450a88ee383b56468b08}{Départements administratifs au 1er janvier 2017}
\item \href{https://data.gouv.fr/dataset/599ede7f88ee3858e42271f8}{Données issues des campagnes exceptionnelles (CAMPEX 2011 - 2013), substances eaux de surface continentales - fichiers bruts producteur}
\item \href{https://data.gouv.fr/dataset/599ede7988ee38573abf9f13}{Données issues des campagnes exceptionnelles (CAMPEX 2011 - 2013), substances eaux littorales - fichier brut producteur}
\item \href{https://data.gouv.fr/dataset/599ede8088ee3856fdb2fa81}{Données issues des campagnes exceptionnelles (CAMPEX 2011 - 2013), substances eaux souterraines - fichiers bruts producteur}
\item \href{https://data.gouv.fr/dataset/5c0e302f634f4133465f3fb8}{Eléments hydrographiques de surface - Guyane 2015 - BD Carthage}
\item \href{https://data.gouv.fr/dataset/5c0e3021634f413525ba4dae}{Entités Hydrogéologiques - Guadeloupe}
\item \href{https://data.gouv.fr/dataset/5c0e30238b4c4169ea9eec36}{Entités Hydrogéologiques - Guyane}
\item \href{https://data.gouv.fr/dataset/5c0e301d8b4c416bb21ddddc}{Entités Hydrogéologiques - Martinique}
\item \href{https://data.gouv.fr/dataset/5c0e302b8b4c416bbb756743}{Entités Hydrogéologiques - Mayotte}
\item \href{https://data.gouv.fr/dataset/5c0e30218b4c416bb21ddddd}{Entités Hydrogéologiques - Métropole}
\item \href{https://data.gouv.fr/dataset/5c0e3033634f413525ba4db2}{Entités Hydrogéologiques - Réunion}
\item \href{https://data.gouv.fr/dataset/597b4512c751df06d6233af6}{Entités hydrographiques de surface - Guadeloupe 2009 - BD Carthage}
\item \href{https://data.gouv.fr/dataset/597b452cc751df07157f54ee}{Entités hydrographiques de surface - Mayotte 2013 - BD Carthage}
\item \href{https://data.gouv.fr/dataset/597b450888ee383d2110a71c}{Entités hydrographiques de surface - Métropole 2014 - BD Carthage}
\item \href{https://data.gouv.fr/dataset/597b450788ee383d3c10c6a8}{Entités hydrographiques de surface - Réunion 2012 - BD Carthage}
\item \href{https://data.gouv.fr/dataset/552cdda688ee384a3985ce60}{Hydroecorégions de niveau 2 (HER-2)}
\item \href{https://data.gouv.fr/dataset/5c0e301a634f413525ba4dac}{Hydrogéologie - SAQ}
\item \href{https://data.gouv.fr/dataset/584f215888ee380450c65bb4}{Hydrométrie - HYD}
\item \href{https://data.gouv.fr/dataset/597b453b88ee383d3c10c6b6}{Laisse des eaux - Guadeloupe 2009 - BD Carthage}
\item \href{https://data.gouv.fr/dataset/597b4502c751df06d6233af5}{Laisse des eaux - Mayotte 2013 - BD Carthage}
\item et 221 autres jeux de données\end{itemize}

\clearpage
\section{Toulouse métrople}


\begin{center}
  \includegraphics[width=3cm]{images/orga/c0_2606ddc0b241378babea024e259e5b-100.jpg}
\end{center}
\needspace{12\baselineskip}
\subsection*{Agenda des manifestations culturelles - Toulouse
}\index{agenda}\index{culture}\index{evenements}\index{expositions}\index{festivals}\index{manifestations}\index{spectacles}
  \begin{wrapfigure}{r}{2.5cm}
    \centering
    \qrcode[nolink]{https://data.gouv.fr/dataset/56b0c2ffb595086d5569cb9c}
  \end{wrapfigure}

Licence : \textbf{Open Data Commons Open Database License (ODbL)
}\newline
Créé le : 2016-02-02\newline
Modifié le : 2019-03-17\newline
Popularité : 1 réutilisation,  0 suivi\newline
Mots-clé : \emph{agenda, culture, evenements, expositions, festivals, manifestations, spectacles
}\newline
Permalien : \url{https://data.gouv.fr/dataset/56b0c2ffb595086d5569cb9c}\newline

\par
\noindent
    Agenda des manifestations culturelles, des événements, spectacles,
festivals, expositions, animations, etc. de la ville de Toulouse et des
villes alentours. Les mises à jours sont quotidiennes.


\vspace{0.5cm}
\needspace{12\baselineskip}
\subsection*{API Temps Réel TISSEO - Real Time TISSEO API
}\index{bus}\index{horaires}\index{metro}\index{tisseo}\index{tramway}\index{transport}
  \begin{wrapfigure}{r}{2.5cm}
    \centering
    \qrcode[nolink]{https://data.gouv.fr/dataset/56b0c2caa3a7294d39b88a63}
  \end{wrapfigure}

Licence : \textbf{Open Data Commons Open Database License (ODbL)
}\newline
Créé le : 2016-02-02\newline
Modifié le : 2016-02-02\newline
Popularité : 1 réutilisation,  1 suivi\newline
Mots-clé : \emph{bus, horaires, metro, tisseo, tramway, transport
}\newline
Permalien : \url{https://data.gouv.fr/dataset/56b0c2caa3a7294d39b88a63}\newline

\par
\noindent
    \textbf{Current API version : 1.2}

English documentation is now available (see in ``Tableau'' section)

-\/-

L'API Tisséo permet d'interroger les données dynamiquement.

Elle~offre les services suivants :

\begin{itemize}

\item
  stops\_schedules : horaires à un ou plusieurs arrêts en temps réel
\item
  journeys : calcul d'itinéraire
\item
  places : recherche de lieux
\item
  lines : liste~des lignes (et de leurs perturbations)
\item
  rolling\_stocks : liste les modes de transport disponibles sur le
  réseau
\item
  stop\_areas : liste des~zones d'arrêts
\item
  stop\_points : liste des poteaux d'arrêt
\item
  messages : liste des messages d'information réseau
\item
  services\_density : connaitre la densité de services de transport
  autour d'un lieu à une heure donnée
\item
  networks : liste des réseaux de transport disponibles
\end{itemize}

La documentation contient les informations techniques nécessaires à
l'utilisation de l'API ainsi que les conditions d'accès au service
(limitations, clefs d'accès, identification).


\vspace{0.5cm}
\needspace{12\baselineskip}
\subsection*{Balade Nature Culture - De la Garonne à la Ramée
}\index{balade}\index{chemin}\index{cyclable}\index{garonne}\index{itineraire}\index{mode!doux}\index{nature}\index{parcours}\index{ramee}\index{randonnee}\index{riviere}\index{velo}
  \begin{wrapfigure}{r}{2.5cm}
    \centering
    \qrcode[nolink]{https://data.gouv.fr/dataset/54b4e466c751df07a4ae2616}
  \end{wrapfigure}

Licence : \textbf{Open Data Commons Open Database License (ODbL)
}\newline
Créé le : 2015-01-13\newline
Modifié le : 2016-03-16\newline
De 2015-01-01 à 2016-01-01\newline
Granularité : à l'EPCI\newline
Mise à jour : annuelle\newline
Popularité : 1 réutilisation,  0 suivi\newline
Mots-clé : \emph{balade, chemin, cyclable, garonne, itineraire, mode-doux, nature, parcours, ramee, randonnee, riviere, velo
}\newline
Permalien : \url{https://data.gouv.fr/dataset/54b4e466c751df07a4ae2616}\newline

\par
\noindent
    Parcours Balade Nature Culture dans Toulouse Métropole sur le secteur
``De la garonne à la Ramée''. Dépliant téléchargeable sur le site de
Toulouse
Métropole:\url{http://www.toulouse-metropole.fr/documents/10180/532925/carte_laramee/dc811dce-02bc-49b4-9968-685eb79e1739}


\vspace{0.5cm}
\needspace{12\baselineskip}
\subsection*{Balade Nature Culture - Fil de l'Hers
}\index{balade}\index{chemin}\index{cyclable}\index{hers}\index{itineraire}\index{mode!doux}\index{nature}\index{parcours}\index{randonnee}\index{riviere}\index{velo}
  \begin{wrapfigure}{r}{2.5cm}
    \centering
    \qrcode[nolink]{https://data.gouv.fr/dataset/54b4e31dc751df0605ae2616}
  \end{wrapfigure}

Licence : \textbf{Open Data Commons Open Database License (ODbL)
}\newline
Créé le : 2015-01-13\newline
Modifié le : 2015-07-17\newline
De 2015-01-01 à 2016-01-01\newline
Granularité : à l'EPCI\newline
Mise à jour : annuelle\newline
Popularité : 1 réutilisation,  0 suivi\newline
Mots-clé : \emph{balade, chemin, cyclable, hers, itineraire, mode-doux, nature, parcours, randonnee, riviere, velo
}\newline
Permalien : \url{https://data.gouv.fr/dataset/54b4e31dc751df0605ae2616}\newline

\par
\noindent
    Parcours Balade Nature Culture dans Toulouse Métropole sur le secteur
``Au Fil de l'Hers''. Dépliant téléchargeable sur le site de Toulouse
Métropole:\url{http://www.toulouse-metropole.fr/documents/10180/532925/carte_hers/06c78ecc-a7f1-4bb5-9f85-67699078a1d4}


\vspace{0.5cm}
\needspace{12\baselineskip}
\subsection*{Balade Nature Culture - Garonne Aval
}\index{balade}\index{chemin}\index{coteaux}\index{culture}\index{cyclable}\index{garonne}\index{itineraire}\index{mode!doux}\index{nature}\index{parcours}\index{randonnee}\index{riviere}\index{velo}
  \begin{wrapfigure}{r}{2.5cm}
    \centering
    \qrcode[nolink]{https://data.gouv.fr/dataset/54b3ef1ec751df1088ae2617}
  \end{wrapfigure}

Licence : \textbf{Open Data Commons Open Database License (ODbL)
}\newline
Créé le : 2015-01-12\newline
Modifié le : 2015-11-20\newline
De 2015-01-01 à 2016-01-01\newline
Granularité : à l'EPCI\newline
Mise à jour : annuelle\newline
Popularité : 1 réutilisation,  0 suivi\newline
Mots-clé : \emph{balade, chemin, coteaux, culture, cyclable, garonne, itineraire, mode-doux, nature, parcours, randonnee, riviere, velo
}\newline
Permalien : \url{https://data.gouv.fr/dataset/54b3ef1ec751df1088ae2617}\newline

\par
\noindent
    Parcours Balade Nature Culture dans Toulouse Métropole sur de l'aval de
la Garonne jusqu'a ses côteaux``. Dépliant téléchargeable sur le site de
Toulouse
Métropole\url{http://www.toulouse-metropole.fr/documents/10180/532925/carte_garonne/6711e3f0-e6e4-4813-86c1-ccba47c9d858}


\vspace{0.5cm}
\needspace{12\baselineskip}
\subsection*{Base Mérimée
}\index{architecture}\index{archives}\index{classe}\index{historique}\index{inscrit}\index{inventaire}\index{merimee}\index{monument}\index{patrimoine}\index{site}\index{urbanisme}
  \begin{wrapfigure}{r}{2.5cm}
    \centering
    \qrcode[nolink]{https://data.gouv.fr/dataset/54b4e584c751df56c0ae2624}
  \end{wrapfigure}

Licence : \textbf{Open Data Commons Open Database License (ODbL)
}\newline
Créé le : 2015-01-13\newline
Modifié le : 2016-02-18\newline
Granularité : à la commune\newline
Mise à jour : ponctuelle\newline
Popularité : 1 réutilisation,  0 suivi\newline
Mots-clé : \emph{architecture, archives, classe, historique, inscrit, inventaire, merimee, monument, patrimoine, site, urbanisme
}\newline
Permalien : \url{https://data.gouv.fr/dataset/54b4e584c751df56c0ae2624}\newline

\par
\noindent
    Fiches ``Mérimée'' du patrimoine architectural toulousain les plus à
jour


\vspace{0.5cm}
\needspace{12\baselineskip}
\subsection*{Emplacements taxis
}\index{deplacement}\index{mobilib}\index{mobilite}\index{stationnement}\index{taxis}\index{transport}\index{voiture}
  \begin{wrapfigure}{r}{2.5cm}
    \centering
    \qrcode[nolink]{https://data.gouv.fr/dataset/536993cfa3a729239d2042c0}
  \end{wrapfigure}

Licence : \textbf{Open Data Commons Open Database License (ODbL)
}\newline
Créé le : 2013-11-18\newline
Modifié le : 2016-03-16\newline
Mise à jour : hebdomadaire\newline
Popularité : 1 réutilisation,  0 suivi\newline
Mots-clé : \emph{deplacement, mobilib, mobilite, stationnement, taxis, transport, voiture
}\newline
Permalien : \url{https://data.gouv.fr/dataset/536993cfa3a729239d2042c0}\newline

\par
\noindent
    Localisation des emplacements taxis


\vspace{0.5cm}
\needspace{12\baselineskip}
\subsection*{Fontaines à boire
}\index{boire}\index{borne}\index{canicule}\index{chaleur}\index{eau}\index{fontaine}\index{fraicheur}\index{robinet}
  \begin{wrapfigure}{r}{2.5cm}
    \centering
    \qrcode[nolink]{https://data.gouv.fr/dataset/54b3edccc751df108fae2615}
  \end{wrapfigure}

Licence : \textbf{Open Data Commons Open Database License (ODbL)
}\newline
Créé le : 2015-01-12\newline
Modifié le : 2015-11-08\newline
De 2015-01-01 à 2016-01-01\newline
Granularité : à l'EPCI\newline
Mise à jour : annuelle\newline
Popularité : 2 réutilisations,  0 suivi\newline
Mots-clé : \emph{boire, borne, canicule, chaleur, eau, fontaine, fraicheur, robinet
}\newline
Permalien : \url{https://data.gouv.fr/dataset/54b3edccc751df108fae2615}\newline

\par
\noindent
    Localisation des points de distribution public d'eau potable (fontaines
à boire) sur le territoire de Toulouse Métropole. La gestion de ces
données est multi-acteurs et une campagne de mise à jour/ fiabilisation
est en cours. Ce jeu de données est proposé à la réutilisation, sous
réserve d'arrêt temporaire de service pour certaines bornes. N'hésitez
pas à nous faire part d'éventuelles anomalies relevées sur le terrain.


\vspace{0.5cm}
\needspace{12\baselineskip}
\subsection*{Localisation des défibrillateurs - Toulouse
}\index{accident}\index{cardiaque}\index{defibrillateur}\index{externe}\index{hygiene}\index{secours}\index{securite}\index{services}
  \begin{wrapfigure}{r}{2.5cm}
    \centering
    \qrcode[nolink]{https://data.gouv.fr/dataset/56b0c2cda3a7294d39b88a65}
  \end{wrapfigure}

Licence : \textbf{Open Data Commons Open Database License (ODbL)
}\newline
Créé le : 2016-02-02\newline
Modifié le : 2019-03-17\newline
Popularité : 1 réutilisation,  0 suivi\newline
Mots-clé : \emph{accident, cardiaque, defibrillateur, externe, hygiene, secours, securite, services
}\newline
Permalien : \url{https://data.gouv.fr/dataset/56b0c2cda3a7294d39b88a65}\newline

\par
\noindent
    Localisation des défibrillateurs automatisés externe (DAE) situés sur le
domaine public, dans les stations de métros ou dans des établissements
recevant du public sur la commune de Toulouse.


\vspace{0.5cm}
\needspace{12\baselineskip}
\subsection*{Menus des écoles maternelles et élémentaires - Ville de Toulouse --
Décembre
}\index{cantine}\index{elementaire}\index{enfance}\index{maternelle}\index{repas}
  \begin{wrapfigure}{r}{2.5cm}
    \centering
    \qrcode[nolink]{https://data.gouv.fr/dataset/56b0c2c8a3a7294d39b88a61}
  \end{wrapfigure}

Licence : \textbf{Open Data Commons Open Database License (ODbL)
}\newline
Créé le : 2016-02-02\newline
Modifié le : 2016-02-02\newline
Popularité : 2 réutilisations,  0 suivi\newline
Mots-clé : \emph{cantine, elementaire, enfance, maternelle, repas
}\newline
Permalien : \url{https://data.gouv.fr/dataset/56b0c2c8a3a7294d39b88a61}\newline

\par
\noindent
    Menus des écoles maternelles et élémentaires de Toulouse.

Si vous souhaitez consulter le menu de votre enfant, veuillez

1- Sélectionner en cliquant dans l'espace de recherche de gauche

- ECOLE : ELEM pour elementaire ou MAT pour maternelle

- SEMAINE

- JOUR

2- Puis visualiser le résultat dans l'onglet ``Tableau''

\begin{center}\rule{0.5\linewidth}{\linethickness}\end{center}

Veuillez trouver ci joint les allergènes :

\begin{enumerate}
\def\labelenumi{\arabic{enumi}.}

\item
  Gluten
\item
  Lait de vache (ou produits dérivés, beurre, crème)
\item
  Lait de brebis ou chèvre
\item
  Œuf
\item
  Moutarde
\item
  Poisson
\item
  Arachide
\item
  Fruit à coque
\item
  Crustacés
\item
  Mollusques
\item
  Céleri
\item
  Soja
\item
  Anhydride sulfureux et sulfite
\item
  Sésame
\end{enumerate}


\vspace{0.5cm}
\needspace{12\baselineskip}
\subsection*{Menus des écoles maternelles et élémentaires - Ville de Toulouse --
Février
}\index{cantine}\index{elementaire}\index{enfance}\index{maternelle}\index{repas}
  \begin{wrapfigure}{r}{2.5cm}
    \centering
    \qrcode[nolink]{https://data.gouv.fr/dataset/56b0c2d6a3a7294d39b88a6c}
  \end{wrapfigure}

Licence : \textbf{Open Data Commons Open Database License (ODbL)
}\newline
Créé le : 2016-02-02\newline
Modifié le : 2016-02-02\newline
Popularité : 2 réutilisations,  0 suivi\newline
Mots-clé : \emph{cantine, elementaire, enfance, maternelle, repas
}\newline
Permalien : \url{https://data.gouv.fr/dataset/56b0c2d6a3a7294d39b88a6c}\newline

\par
\noindent
    Menus des écoles maternelles et élémentaires de Toulouse.

Si vous souhaitez consulter le menu de votre enfant, veuillez

1- Sélectionner en cliquant dans l'espace de recherche de gauche

- ECOLE : ELEM pour elementaire ou MAT pour maternelle

- SEMAINE

- JOUR

2- Puis visualiser le résultat dans l'onglet ``Tableau''

\begin{center}\rule{0.5\linewidth}{\linethickness}\end{center}

Veuillez trouver ci joint les allergènes (Attention certains allergènes
ne s'affichent pas, nous sommes en cours de correction):

\begin{enumerate}
\def\labelenumi{\arabic{enumi}.}

\item
  Gluten
\item
  Lait de vache (ou produits dérivés, beurre, crème)
\item
  Lait de brebis ou chèvre
\item
  Œuf
\item
  Moutarde
\item
  Poisson
\item
  Arachide
\item
  Fruit à coque
\item
  Crustacés
\item
  Mollusques
\item
  Céleri
\item
  Soja
\item
  Anhydride sulfureux et sulfite
\item
  Sésame
\end{enumerate}


\vspace{0.5cm}
\needspace{12\baselineskip}
\subsection*{Menus des écoles maternelles et élémentaires - Ville de Toulouse --
Février (Accueils de loisirs)
}\index{cantine}\index{elementaire}\index{enfance}\index{maternelle}\index{repas}
  \begin{wrapfigure}{r}{2.5cm}
    \centering
    \qrcode[nolink]{https://data.gouv.fr/dataset/56b0c2eca3a7294d38b88a7b}
  \end{wrapfigure}

Licence : \textbf{Open Data Commons Open Database License (ODbL)
}\newline
Créé le : 2016-02-02\newline
Modifié le : 2016-02-02\newline
Popularité : 2 réutilisations,  0 suivi\newline
Mots-clé : \emph{cantine, elementaire, enfance, maternelle, repas
}\newline
Permalien : \url{https://data.gouv.fr/dataset/56b0c2eca3a7294d38b88a7b}\newline

\par
\noindent
    Menus des écoles maternelles et élémentaires de Toulouse.

Si vous souhaitez consulter le menu de votre enfant, veuillez

1- Sélectionner en cliquant dans l'espace de recherche de gauche

- ECOLE : ELEM pour elementaire ou MAT pour maternelle

- SEMAINE

- JOUR

2- Puis visualiser le résultat dans l'onglet ``Tableau''

\begin{center}\rule{0.5\linewidth}{\linethickness}\end{center}

Veuillez trouver ci joint les allergènes (Attention certains allergènes
ne s'affichent pas, nous sommes en cours de correction):

\begin{enumerate}
\def\labelenumi{\arabic{enumi}.}

\item
  Gluten
\item
  Lait de vache (ou produits dérivés, beurre, crème)
\item
  Lait de brebis ou chèvre
\item
  Œuf
\item
  Moutarde
\item
  Poisson
\item
  Arachide
\item
  Fruit à coque
\item
  Crustacés
\item
  Mollusques
\item
  Céleri
\item
  Soja
\item
  Anhydride sulfureux et sulfite
\item
  Sésame
\end{enumerate}


\vspace{0.5cm}
\needspace{12\baselineskip}
\subsection*{Menus des écoles maternelles et élémentaires - Ville de Toulouse --
Janvier
}\index{cantine}\index{elementaire}\index{enfance}\index{maternelle}\index{repas}
  \begin{wrapfigure}{r}{2.5cm}
    \centering
    \qrcode[nolink]{https://data.gouv.fr/dataset/56b0c2e9b595086d5569cb8c}
  \end{wrapfigure}

Licence : \textbf{Open Data Commons Open Database License (ODbL)
}\newline
Créé le : 2016-02-02\newline
Modifié le : 2016-02-02\newline
Popularité : 2 réutilisations,  0 suivi\newline
Mots-clé : \emph{cantine, elementaire, enfance, maternelle, repas
}\newline
Permalien : \url{https://data.gouv.fr/dataset/56b0c2e9b595086d5569cb8c}\newline

\par
\noindent
    Menus des écoles maternelles et élémentaires de Toulouse.

Si vous souhaitez consulter le menu de votre enfant, veuillez

1- Sélectionner en cliquant dans l'espace de recherche de gauche

- ECOLE : ELEM pour elementaire ou MAT pour maternelle

- SEMAINE

- JOUR

2- Puis visualiser le résultat dans l'onglet ``Tableau''

\begin{center}\rule{0.5\linewidth}{\linethickness}\end{center}

Veuillez trouver ci joint les allergènes :

\begin{enumerate}
\def\labelenumi{\arabic{enumi}.}

\item
  Gluten
\item
  Lait de vache (ou produits dérivés, beurre, crème)
\item
  Lait de brebis ou chèvre
\item
  Œuf
\item
  Moutarde
\item
  Poisson
\item
  Arachide
\item
  Fruit à coque
\item
  Crustacés
\item
  Mollusques
\item
  Céleri
\item
  Soja
\item
  Anhydride sulfureux et sulfite
\item
  Sésame
\end{enumerate}


\vspace{0.5cm}
\needspace{12\baselineskip}
\subsection*{Réseau cyclable et vert
}\index{circulation}\index{cyclable}\index{deplacement}\index{itineraire}\index{mobilite}\index{mode!doux}\index{piste}\index{transport}\index{velo}\index{vert}
  \begin{wrapfigure}{r}{2.5cm}
    \centering
    \qrcode[nolink]{https://data.gouv.fr/dataset/53699f0ca3a729239d20602f}
  \end{wrapfigure}

Licence : \textbf{Open Data Commons Open Database License (ODbL)
}\newline
Créé le : 2013-11-18\newline
Modifié le : 2016-02-29\newline
Mise à jour : hebdomadaire\newline
Popularité : 1 réutilisation,  0 suivi\newline
Mots-clé : \emph{circulation, cyclable, deplacement, itineraire, mobilite, mode-doux, piste, transport, velo, vert
}\newline
Permalien : \url{https://data.gouv.fr/dataset/53699f0ca3a729239d20602f}\newline

\par
\noindent
    Réseau cyclable et vert du Grand Toulouse


\vspace{0.5cm}
\needspace{12\baselineskip}
\subsection*{Sanisette
}\index{sanisettes}\index{toilette}
  \begin{wrapfigure}{r}{2.5cm}
    \centering
    \qrcode[nolink]{https://data.gouv.fr/dataset/53699fada3a729239d2061ae}
  \end{wrapfigure}

Licence : \textbf{Open Data Commons Open Database License (ODbL)
}\newline
Créé le : 2013-11-18\newline
Modifié le : 2015-11-24\newline
Mise à jour : hebdomadaire\newline
Popularité : 1 réutilisation,  1 suivi\newline
Mots-clé : \emph{sanisettes, toilette
}\newline
Permalien : \url{https://data.gouv.fr/dataset/53699fada3a729239d2061ae}\newline

\par
\noindent
    Localisation des Sanisettes


\vspace{0.5cm}
\needspace{12\baselineskip}
\subsection*{Tisséo : offre de transport - GTFS
}\index{bus}\index{metro}\index{tisseo}\index{tramway}\index{transport}\index{transports}
  \begin{wrapfigure}{r}{2.5cm}
    \centering
    \qrcode[nolink]{https://data.gouv.fr/dataset/56b0c2fba3a7294d39b88a86}
  \end{wrapfigure}

Licence : \textbf{Open Data Commons Open Database License (ODbL)
}\newline
Créé le : 2016-02-02\newline
Modifié le : 2016-02-02\newline
Popularité : 1 réutilisation,  2 suivis\newline
Mots-clé : \emph{bus, metro, tisseo, tramway, transport, transports
}\newline
Permalien : \url{https://data.gouv.fr/dataset/56b0c2fba3a7294d39b88a86}\newline

\par
\noindent
    Données descriptives de l'offre de transport (lignes, arrêts, horaires)
de Tisséo. Ces données sont disponibles sur ce portail en téléchargement
libre sous licence Odbl.

Particularités des lignes :

\begin{itemize}

\item
  Certaines lignes (TAD 105, 201, 202, 204, 205) sous soumises à
  réservation, les arrêts ne sont desservis qu'en cas de réservation
\item
  Certaines lignes (106, 118, 119, 120) correspondant à des TAD Zonaux
  ne sont pas décrites, le format ne le permettant pas.
\end{itemize}

Les données sont publiées au format GTFS. Pour plus d'informations sur
ce format :
\href{https://developers.google.com/transit/gtfs/reference}{la
référence}

Pour toute question, vous pouvez adresser vos mails à opendata@tisseo.fr


\vspace{0.5cm}
\needspace{12\baselineskip}
\subsection*{VéloToulouse
}\index{borne}\index{circulation}\index{cyclable}\index{deplacement}\index{libre}\index{mobilite}\index{mode!doux}\index{piste}\index{service}\index{station}\index{transport}\index{velo}
  \begin{wrapfigure}{r}{2.5cm}
    \centering
    \qrcode[nolink]{https://data.gouv.fr/dataset/5369a350a3a729239d206a32}
  \end{wrapfigure}

Licence : \textbf{Open Data Commons Open Database License (ODbL)
}\newline
Créé le : 2013-11-18\newline
Modifié le : 2015-11-10\newline
Mise à jour : hebdomadaire\newline
Popularité : 1 réutilisation,  0 suivi\newline
Mots-clé : \emph{borne, circulation, cyclable, deplacement, libre, mobilite, mode-doux, piste, service, station, transport, velo
}\newline
Permalien : \url{https://data.gouv.fr/dataset/5369a350a3a729239d206a32}\newline

\par
\noindent
    Localisation des stations de Vélo Toulouse


\vspace{0.5cm}
\needspace{3\baselineskip} \rule{4cm}{0.25pt}\newline\textbf{Aussi disponible du même producteur :}\begin{itemize}
\item \href{https://data.gouv.fr/dataset/5b3b2212a3a72978c6b11d6f}{20 ans de prêts des collections du musée des Augustins}
\item \href{https://data.gouv.fr/dataset/5aa3aa1db595080d160d3df0}{Accessibilité des établissements recevant du public (granularite fine) - Toulouse}
\item \href{https://data.gouv.fr/dataset/5bfe183606e3e7262704ccef}{Accessibilité des établissements recevant du public - Toulouse}
\item \href{https://data.gouv.fr/dataset/5b596cadb59508340fd496c9}{Affluences Toulouse Olympique 13 - Championship 2018}
\item \href{https://data.gouv.fr/dataset/56b0c2c1a3a7294d38b88a5a}{Agenda des événements - Flourens}
\item \href{https://data.gouv.fr/dataset/53698ecaa3a729239d203573}{Aires de livraison}
\item \href{https://data.gouv.fr/dataset/53698ecca3a729239d203580}{Aires piétonnes}
\item \href{https://data.gouv.fr/dataset/5c56f6e49ce2e715e8cc53f5}{Album Plan de ville - 2015 - Ville de Toulouse}
\item \href{https://data.gouv.fr/dataset/53698ecda3a729239d203582}{Album plan de ville toulouse}
\item \href{https://data.gouv.fr/dataset/53698ed4a3a729239d203594}{Altimétrie des voies}
\item \href{https://data.gouv.fr/dataset/56b0c2d3b595086d5569cb7c}{Arbres d'alignement - Toulouse}
\item \href{https://data.gouv.fr/dataset/5aab5c59a3a7294fa12c89f1}{Associations ayant leur siège à Toulouse- Les données du Journal officiel des associations et fondations d'entreprise}
\item \href{https://data.gouv.fr/dataset/56b0c2dca3a7294d39b88a71}{Budget Primitif 2013 - Toulouse Métropole}
\item \href{https://data.gouv.fr/dataset/53698fd0a3a729239d203841}{Budget Primitif 2013 Toulouse Métropole}
\item \href{https://data.gouv.fr/dataset/53698fd8a3a729239d203855}{Budget Primitif du Grand Toulouse}
\item \href{https://data.gouv.fr/dataset/56b0c2bbb595086d5569cb66}{Bureaux de Poste : points de contact du réseau postal}
\item \href{https://data.gouv.fr/dataset/53699025a3a729239d20391b}{Carrefours Feux Sonores}
\item \href{https://data.gouv.fr/dataset/5369906ca3a729239d2039d2}{Chantiers en cours}
\item \href{https://data.gouv.fr/dataset/5369906ca3a729239d2039d3}{Chantiers en cours traces}
\item \href{https://data.gouv.fr/dataset/56b0c2c4a3a7294d38b88a5d}{Circonscriptions électorales Haute Garonne}
\item \href{https://data.gouv.fr/dataset/56b0c2eda3a7294d38b88a7c}{Collections du musée des Augustins – ville de Toulouse}
\item \href{https://data.gouv.fr/dataset/53699121a3a729239d203b9c}{Communes}
\item \href{https://data.gouv.fr/dataset/5369913ea3a729239d203bed}{Compte administratif 2012 Toulouse métropole}
\item \href{https://data.gouv.fr/dataset/5c09f63c9ce2e7740abea585}{Cuisine centrale de Toulouse : Typologie des achats 2018}
\item \href{https://data.gouv.fr/dataset/5c1ded2e06e3e72e6b04ccef}{Dataviz - Election présidentielle 2017 - second tour - Résultats - Ville de Toulouse}
\item \href{https://data.gouv.fr/dataset/56b0c2c4a3a7294d39b88a5e}{Déchets ménagers et assimilés collectés}
\item \href{https://data.gouv.fr/dataset/56b0c2bca3a7294d38b88a56}{Ecoles}
\item \href{https://data.gouv.fr/dataset/56b0c2b9a3a7294d39b88a54}{Elections législatives 2012 - 1er tour - Ville de Toulouse (Résultats)}
\item \href{https://data.gouv.fr/dataset/56b0c2c7b595086d5569cb71}{Etat annuel des Collections de la bibliothèque de Toulous}
\item \href{https://data.gouv.fr/dataset/5a0e61cbb5950840efec248b}{Expositions du Musée Saint-Raymond}
\item \href{https://data.gouv.fr/dataset/56b0c2ccb595086d5569cb76}{Gymnases - Toulouse}
\item \href{https://data.gouv.fr/dataset/53699634a3a729239d204926}{Horodateurs}
\item \href{https://data.gouv.fr/dataset/56b0c2cba3a7294d38b88a63}{Installations sportives}
\item \href{https://data.gouv.fr/dataset/5a0e614fb5950840f0ec24bb}{Inventaire principal - Musée Saint-Raymond}
\item \href{https://data.gouv.fr/dataset/53699799a3a729239d204d2d}{Lacs des bases de loisirs}
\item \href{https://data.gouv.fr/dataset/5982acf1b595086a5047410d}{Liste des dépôts du magazine TIM}
\item \href{https://data.gouv.fr/dataset/5c4fd6449ce2e71d2b275b6c}{Liste des élus au Conseil Municipal - Mairie de Toulouse}
\item \href{https://data.gouv.fr/dataset/5c53cab39ce2e75d11275b6c}{Liste des élus au Conseil Municipal - Mairie de Toulouse}
\item \href{https://data.gouv.fr/dataset/536999a7a3a729239d2052d1}{Mairie}
\item \href{https://data.gouv.fr/dataset/56b0c2e5a3a7294d38b88a77}{Mairies}
\item \href{https://data.gouv.fr/dataset/56b0c2dab595086d5569cb81}{Maison de la justice et du droit}
\item \href{https://data.gouv.fr/dataset/56b0c2f6b595086d5669cb94}{Marchés couverts et de plein-vent - Toulouse}
\item \href{https://data.gouv.fr/dataset/536999b7a3a729239d2052f7}{Marchés de plein vent}
\item \href{https://data.gouv.fr/dataset/536999bca3a729239d205303}{Marchés publics 2012 Toulouse Métropole}
\item \href{https://data.gouv.fr/dataset/59a63a37b595086587ad726f}{Menus cantine - Ville de Toulouse – Aout}
\item \href{https://data.gouv.fr/dataset/59593224a3a7291dcf9c825e}{Menus cantine - Ville de Toulouse – Juin}
\item \href{https://data.gouv.fr/dataset/595931cea3a7291dd09c8213}{Menus cantine - Ville de Toulouse – Mai}
\item \href{https://data.gouv.fr/dataset/59a63a36b595086586ad724e}{Menus cantine - Ville de Toulouse – Septembre}
\item \href{https://data.gouv.fr/dataset/53699a50a3a729239d20546c}{N de rue}
\item \href{https://data.gouv.fr/dataset/53699b57a3a729239d2056f4}{Orthophotoplan 2007}
\item et 21 autres jeux de données\end{itemize}

\clearpage
\section{Ville de Boulogne-Billancourt}


\begin{center}
  \includegraphics[width=3cm]{images/orga/2015-02-09_b5bcf47b74044a88b2bc684dd2c40631_BB_03_noir_2009_rogne-100.jpg}
\end{center}


La Ville de Boulogne-Billancourt compte 118 000 habitants, 56 000
personnes actives, 81 000 emplois et accueille 13 510 établissements
privés dont 2 700 artisans et commerçants sur son territoire. Elle
rejoint le mouvement des collectivités qui libèrent des données.
Réutilisez ces données, par exemple en créant des représentations
graphiques avec les outils conseillés par
\href{https://wiki.data.gouv.fr/wiki/Outillage_pour_les_datavisualisations}{data.gouv.fr}
ou des applications mobiles grâce au projet européen
\href{http://www.citadelonthemove.eu/}{Citadel on the move}.


\vspace{0.5cm}

\needspace{12\baselineskip}
\subsection*{Enfance
}\index{accueil}\index{accueil!enfance!handicapee}\index{creche}\index{enfance}\index{enfance!et!familles}\index{garde}\index{garderie}\index{gardes}
  \begin{wrapfigure}{r}{2.5cm}
    \centering
    \qrcode[nolink]{https://data.gouv.fr/dataset/55094a3bc751df601a882845}
  \end{wrapfigure}

Licence : \textbf{Licence Ouverte
}\newline
Créé le : 2015-03-18\newline
Modifié le : 2015-05-18\newline
Granularité : à la commune\newline
Mise à jour : ponctuelle\newline
Popularité : 1 réutilisation,  0 suivi\newline
Mots-clé : \emph{accueil, accueil-enfance-handicapee, creche, enfance, enfance-et-familles, garde, garderie, gardes
}\newline
Permalien : \url{https://data.gouv.fr/dataset/55094a3bc751df601a882845}\newline

\par
\noindent
    Liste des lieux accueillant des enfants


\vspace{0.5cm}
\needspace{3\baselineskip} \rule{4cm}{0.25pt}\newline\textbf{Aussi disponible du même producteur :}\begin{itemize}
\item \href{https://data.gouv.fr/dataset/55094510c751df5792882844}{Administration générale}
\item \href{https://data.gouv.fr/dataset/550988bfc751df3c55882845}{Bureaux de vote}
\item \href{https://data.gouv.fr/dataset/54e3696fc751df7ce2467389}{Bureaux de vote}
\item \href{https://data.gouv.fr/dataset/550986abc751df3a78882844}{Cantons}
\item \href{https://data.gouv.fr/dataset/55094b29c751df602d882844}{Centres de loisirs - accueil }
\item \href{https://data.gouv.fr/dataset/5509880cc751df3c2d882844}{Circonscriptions}
\item \href{https://data.gouv.fr/dataset/54eaf495c751df0cb4467389}{Clubs Seniors}
\item \href{https://data.gouv.fr/dataset/550ac8fbc751df133b882844}{Culture-Animation}
\item \href{https://data.gouv.fr/dataset/54eaf91ac751df1899467389}{Enseignement}
\item \href{https://data.gouv.fr/dataset/55094bdcc751df6386882844}{Equipements de type Social}
\item \href{https://data.gouv.fr/dataset/5683f182c751df0ad4c664bc}{ERP - Catégorie O}
\item \href{https://data.gouv.fr/dataset/5683f2e1c751df79b0c664bd}{ERP - Catégorie R}
\item \href{https://data.gouv.fr/dataset/5683f46ec751df1df6c664bc}{ERP - Catégorie U}
\item \href{https://data.gouv.fr/dataset/5683f53ac751df1df6c664bd}{ERP - Catégorie V}
\item \href{https://data.gouv.fr/dataset/5683f5e6c751df1df6c664be}{ERP - Catégorie X}
\item \href{https://data.gouv.fr/dataset/5683f703c751df79b0c664be}{ERP - Etablissements "Rafraîchis"}
\item \href{https://data.gouv.fr/dataset/5683f869c751df0ad4c664bd}{ERP par catégorie}
\item \href{https://data.gouv.fr/dataset/54ec3c9cc751df70e2467389}{Hotspots wifi}
\item \href{https://data.gouv.fr/dataset/54e365c9c751df796746738a}{Lieux de culte}
\item \href{https://data.gouv.fr/dataset/55094d2fc751df6385882844}{Locaux associatifs}
\item \href{https://data.gouv.fr/dataset/550945c4c751df5af6882844}{Marchés}
\item \href{https://data.gouv.fr/dataset/550990b8c751df4835882844}{Quartiers}
\item \href{https://data.gouv.fr/dataset/55099134c751df485d882844}{Rues}
\item \href{https://data.gouv.fr/dataset/54e36716c751df7e8546738a}{Secours et Urgences}
\item \href{https://data.gouv.fr/dataset/5512f56fc751df6e9bcea128}{Secours et Urgences}
\item \href{https://data.gouv.fr/dataset/550989d4c751df3a78882846}{Sectorisation élémentaires}
\item \href{https://data.gouv.fr/dataset/55098955c751df3c55882846}{Sectorisation maternelles}
\item \href{https://data.gouv.fr/dataset/55098ab3c751df3fa1882845}{Services}
\item \href{https://data.gouv.fr/dataset/55098b4cc751df3fc7882844}{Services de santé}
\item \href{https://data.gouv.fr/dataset/54e367aec751df7cf0467389}{Sport}
\end{itemize}

\clearpage
\section{Ville de Lanester}


\begin{center}
  \includegraphics[width=3cm]{images/orga/7f_88a74c4d514560a5ad14b8407a0148-100.jpg}
\end{center}


Lanester, ville de 23000 habitants dans le Morbihan au coeur de
l'agglomération lorientaise. Une ville à taille humaine à proximité d'un
cadre naturel de caractère.


\vspace{0.5cm}

\needspace{12\baselineskip}
\subsection*{Nombre de décès
}\index{etat!civil}\index{mort}\index{registre!detat!civil}
  \begin{wrapfigure}{r}{2.5cm}
    \centering
    \qrcode[nolink]{https://data.gouv.fr/dataset/5a46435c88ee3877de69bb8e}
  \end{wrapfigure}

Licence : \textbf{Licence Ouverte
}\newline
Créé le : 2017-12-29\newline
Modifié le : 2017-12-29\newline
De 1999-01-01 à 2017-12-29\newline
Mise à jour : annuelle\newline
Popularité : 1 réutilisation,  0 suivi\newline
Mots-clé : \emph{etat-civil, mort, registre-detat-civil
}\newline
Permalien : \url{https://data.gouv.fr/dataset/5a46435c88ee3877de69bb8e}\newline

\par
\noindent
    Ce fichier recense les décès survenus sur le territoire de la commune de
Lanester. Ces chiffres sont répartis par lieu du décès, par sexe et
tranche d'âge de la défunte ou du défunt. Les données proviennent du
traitement de l'extraction du logiciel métier d'état civil. Leur
actualisation a lieu au cours du 1er trimestre de chaque année. Il
n'existe pas de format convenu pour ce jeu de données, à l'heure
actuelle.


\vspace{0.5cm}
\needspace{3\baselineskip} \rule{4cm}{0.25pt}\newline\textbf{Aussi disponible du même producteur :}\begin{itemize}
\item \href{https://data.gouv.fr/dataset/5a3bb760c751df27e375554a}{Liste des prénoms}
\end{itemize}

\clearpage
\section{Ville de Limoges}


\begin{center}
  \includegraphics[width=3cm]{images/orga/28_620638b5094bb683ab13442f553167-100.jpg}
\end{center}


Ville de Limoges - service géomatique en charge de la diffusion de
l'information géographique. Commune centre de 138 000 habitants d'une
communauté d'agglomération de 210 000 habitants.

contact :

\href{http://www.ville-limoges.fr}{site Web}

\texttt{mailto:geomatique@ville-limoges.fr}


\vspace{0.5cm}

\needspace{12\baselineskip}
\subsection*{Référentiel adresse de la Ville de Limoges
}\index{adresse}\index{base!adresse!locale}\index{filaire}\index{limoges}\index{rues}\index{voie}
  \begin{wrapfigure}{r}{2.5cm}
    \centering
    \qrcode[nolink]{https://data.gouv.fr/dataset/54b52ca6c751df7556ae2614}
  \end{wrapfigure}

Licence : \textbf{Open Data Commons Open Database License (ODbL)
}\newline
Créé le : 2015-01-13\newline
Modifié le : 2018-11-30\newline
De 2016-08-04 à 2016-11-04\newline
Granularité : à la commune\newline
Mise à jour : trimestrielle\newline
Popularité : 1 réutilisation,  0 suivi\newline
Mots-clé : \emph{adresse, base-adresse-locale, filaire, limoges, rues, voie
}\newline
Permalien : \url{https://data.gouv.fr/dataset/54b52ca6c751df7556ae2614}\newline

\par
\noindent
    Référentiel adresse sur la commune de Limoges constitué des points
d'adresses, des axes de voies et de leurs libellés. Les points sont
connus en coordonnées géographiques (système de projection
lambert93/CC46).


\vspace{0.5cm}
\needspace{3\baselineskip} \rule{4cm}{0.25pt}\newline\textbf{Aussi disponible du même producteur :}\begin{itemize}
\item \href{https://data.gouv.fr/dataset/5b6c61b18b4c417b09ccaaf6}{PLU de Limoges}
\item \href{https://data.gouv.fr/dataset/54b4df1dc751df026fae2614}{Zonage du PLU de la commune de Limoges}
\end{itemize}

\clearpage
\section{Ville de Meudon}


\begin{center}
  \includegraphics[width=3cm]{images/orga/2015-02-09_c8e02b8db0004b17b0aa311404798324_logo_meudon_couleur_HD-100.jpg}
\end{center}


La Ville de Meudon libère des données pour stimuler la création de
nouveaux services. Réutilisez ces données, par exemple en créant des
représentations graphiques avec les outils conseillés par data.gouv.fr

\href{http://www.meudon.fr/}{Meudon.fr}
\href{https://www.facebook.com/villedemeudon}{Facebook}
\href{https://twitter.com/VilledeMeudon}{Twitter}


\vspace{0.5cm}

\needspace{12\baselineskip}
\subsection*{Défibrillateurs
}\index{defibrillateurs}\index{localisation}\index{meudon}\index{meudon!la!foret}\index{sante}\index{urgence}
  \begin{wrapfigure}{r}{2.5cm}
    \centering
    \qrcode[nolink]{https://data.gouv.fr/dataset/562a404ac751df09c2bd3534}
  \end{wrapfigure}

Licence : \textbf{Licence Ouverte
}\newline
Créé le : 2015-10-23\newline
Modifié le : 2016-01-26\newline
De 2015-10-23 à 2016-12-31\newline
Granularité : à la commune\newline
Mise à jour : annuelle\newline
Popularité : 1 réutilisation,  0 suivi\newline
Mots-clé : \emph{defibrillateurs, localisation, meudon, meudon-la-foret, sante, urgence
}\newline
Permalien : \url{https://data.gouv.fr/dataset/562a404ac751df09c2bd3534}\newline

\par
\noindent
    Ce jeu de données regroupe la localisation des défibrillateurs sur la
Ville de Meudon.


\vspace{0.5cm}
\needspace{3\baselineskip} \rule{4cm}{0.25pt}\newline\textbf{Aussi disponible du même producteur :}\begin{itemize}
\item \href{https://data.gouv.fr/dataset/54d89707c751df15af467389}{Administrations}
\item \href{https://data.gouv.fr/dataset/563cc77d88ee3851fa531576}{Cimetières de Meudon}
\item \href{https://data.gouv.fr/dataset/5a130b51c751df34a4640b25}{Colleges de la Ville de Meudon pour l'année 2017-2018}
\item \href{https://data.gouv.fr/dataset/582edae7c751df5642c0bb7e}{Compte administratif 2012 - balance générale - mandats émis}
\item \href{https://data.gouv.fr/dataset/582edbb8c751df56a1c0bb7e}{Compte administratif 2012 - balance générale - titres émis}
\item \href{https://data.gouv.fr/dataset/5857fcdfc751df0c57c0bb7e}{Compte administratif 2012 - Détail des opérations pour compte de tiers.}
\item \href{https://data.gouv.fr/dataset/585aa610c751df75a02b7157}{Compte administratif 2012 - Engagements hors bilan - Situation des autorisations d'engagement et crédits de paiement.}
\item \href{https://data.gouv.fr/dataset/585aa4aec751df75a42b7155}{Compte administratif 2012 - Engagements hors bilan - Situation des autorisations de programme et crédits de paiement.}
\item \href{https://data.gouv.fr/dataset/5846e088c751df32cbc0bb7e}{Compte administratif 2012 - équilibre des opérations financières - détail des dépenses.}
\item \href{https://data.gouv.fr/dataset/5846e3bcc751df390cc0bb7e}{Compte administratif 2012 - équilibre des opérations financières - détail des recettes.}
\item \href{https://data.gouv.fr/dataset/5857f773c751df09a1c0bb7e}{Compte administratif 2012 - Etat de répartition de la TEOM - section de fonctionnement - dépenses.}
\item \href{https://data.gouv.fr/dataset/5857f831c751df09edc0bb7e}{Compte administratif 2012 - Etat de répartition de la TEOM - section d'investissement - dépenses.}
\item \href{https://data.gouv.fr/dataset/58498cbf88ee38711fc65bb3}{Compte administratif 2012 - Etat de ventilation des dépenses et recettes des services assujettis à la TVA - Section de fonctionnement.}
\item \href{https://data.gouv.fr/dataset/58498da388ee387338c65bb3}{Compte administratif 2012 - Etat de ventilation des dépenses et recettes des services assujettis à la TVA - Section d'investisement.}
\item \href{https://data.gouv.fr/dataset/58384824c751df7f43c0bb7e}{Compte administratif 2012 - opération d'équipement - centre culturel de Meudon-la-Forêt}
\item \href{https://data.gouv.fr/dataset/585aab6ec751df75a32b7155}{Compte administratif 2012 - Présentation agrégée du budget principal et des budgets annexes.}
\item \href{https://data.gouv.fr/dataset/58386d23c751df1290c0bb7e}{Compte administratif 2012 - présentation croisée par fonction\_détail fonctionnement\_actions économiques}
\item \href{https://data.gouv.fr/dataset/58386ca6c751df1250c0bb7e}{Compte administratif 2012 - présentation croisée par fonction\_détail fonctionnement\_aménagement et services urbains et environnement}
\item \href{https://data.gouv.fr/dataset/583869e1c751df10d2c0bb7e}{Compte administratif 2012 - présentation croisée par fonction\_détail fonctionnement\_culture}
\item \href{https://data.gouv.fr/dataset/58386950c751df108ac0bb7e}{Compte administratif 2012 - présentation croisée par fonction\_détail fonctionnement\_enseignement et formation}
\item \href{https://data.gouv.fr/dataset/58386b78c751df119dc0bb7e}{Compte administratif 2012 - présentation croisée par fonction\_détail fonctionnement\_famille}
\item \href{https://data.gouv.fr/dataset/58386b06c751df115dc0bb7e}{Compte administratif 2012 - présentation croisée par fonction\_détail fonctionnement\_interventions sociales et de santé}
\item \href{https://data.gouv.fr/dataset/58386bf2c751df11fac0bb7e}{Compte administratif 2012 - présentation croisée par fonction\_détail fonctionnement\_logement}
\item \href{https://data.gouv.fr/dataset/58386897c751df100cc0bb7e}{Compte administratif 2012 - présentation croisée par fonction\_détail fonctionnement\_sécurité et salubrité publiques}
\item \href{https://data.gouv.fr/dataset/583867c3c751df0f9cc0bb7e}{Compte administratif 2012 - présentation croisée par fonction\_détail fonctionnement\_services généraux des admin. publiques locales}
\item \href{https://data.gouv.fr/dataset/58386a5cc751df1107c0bb7e}{Compte administratif 2012 - présentation croisée par fonction\_détail fonctionnement\_sport et jeunesse}
\item \href{https://data.gouv.fr/dataset/5846d7d6c751df2305c0bb7e}{Compte administratif 2012 - Présentation croisée par fonction - détail investissement - Action économique.}
\item \href{https://data.gouv.fr/dataset/5846d6f0c751df20f6c0bb7e}{Compte administratif 2012 - Présentation croisée par fonction - détail investissement - Aménagement et services urbains et environnement}
\item \href{https://data.gouv.fr/dataset/584055c188ee385512c65bb3}{Compte administratif 2012 - Présentation croisée par fonction - détail investissement - Culture.}
\item \href{https://data.gouv.fr/dataset/584054cd88ee38548fc65bb3}{Compte administratif 2012 - Présentation croisée par fonction - détail investissement - Enseignement et formation.}
\item \href{https://data.gouv.fr/dataset/5846d0f0c751df1605c0bb7e}{Compte administratif 2012 - Présentation croisée par fonction - détail investissement - Famille}
\item \href{https://data.gouv.fr/dataset/5840576788ee385908c65bb3}{Compte administratif 2012 - Présentation croisée par fonction - détail investissement - Interventions sociales et santé.}
\item \href{https://data.gouv.fr/dataset/5846d589c751df1ea8c0bb7e}{Compte administratif 2012 - Présentation croisée par fonction - détail investissement - Logement.}
\item \href{https://data.gouv.fr/dataset/583ef9a0c751df7595c0bb7e}{Compte administratif 2012 - Présentation croisée par fonction - détail investissement - Sécurité et salubrité publiques}
\item \href{https://data.gouv.fr/dataset/583ef664c751df70aac0bb7e}{Compte administratif 2012 - Présentation croisée par fonction - détail investissement - Services généraux des administrations publiques locales}
\item \href{https://data.gouv.fr/dataset/5840568d88ee385706c65bb3}{Compte administratif 2012 - Présentation croisée par fonction - détail investissement - Sport et jeunesse.}
\item \href{https://data.gouv.fr/dataset/58386294c751df0cfdc0bb7e}{Compte administratif 2012 - présentation croisée par fonction\_vue d'ensemble}
\item \href{https://data.gouv.fr/dataset/583864c3c751df0e20c0bb7e}{Compte administratif 2012 - présentation croisée par fonction\_vue d'ensemble 2}
\item \href{https://data.gouv.fr/dataset/58230764c751df2dffc0bb7e}{Compte administratif 2012 - présentation générale de la Ville de Meudon}
\item \href{https://data.gouv.fr/dataset/5823081dc751df2feac0bb7e}{Compte administratif 2012 - présentation générale de la Ville de Meudon}
\item \href{https://data.gouv.fr/dataset/58234f7ac751df32d5c0bb7e}{Compte administratif 2012 - section fonctionnement}
\item \href{https://data.gouv.fr/dataset/58235431c751df3b80c0bb7e}{Compte administratif 2012 - section investissement}
\item \href{https://data.gouv.fr/dataset/582eddacc751df5abfc0bb7e}{Compte administratif 2012 - vote du budget-section de fonctionnement-détail des dépenses}
\item \href{https://data.gouv.fr/dataset/582edf4ec751df5d0cc0bb7e}{Compte administratif 2012 - vote du budget-section de fonctionnement-détail des dépenses 2}
\item \href{https://data.gouv.fr/dataset/582f0dfdc751df334cc0bb7e}{Compte administratif 2012 - vote du budget-section de fonctionnement-détail des recettes}
\item \href{https://data.gouv.fr/dataset/582f0f64c751df3724c0bb7e}{Compte administratif 2012 - vote du budget-section de fonctionnement-détail des recettes 2}
\item \href{https://data.gouv.fr/dataset/582f1b75c751df4cdec0bb7e}{Compte administratif 2012 - vote du budget-section investissement-détail des dépenses}
\item \href{https://data.gouv.fr/dataset/582f1e58c751df5167c0bb7e}{Compte administratif 2012 - vote du budget-section investissement-détail des dépenses 2}
\item \href{https://data.gouv.fr/dataset/582f250988ee38608cc65bbb}{Compte administratif 2012 - vote du budget-section investissement-détail des recettes}
\item \href{https://data.gouv.fr/dataset/582f2b8288ee385991c65bb3}{Compte administratif 2012 - vote du budget-section investissement-détail des recettes 2}
\item et 128 autres jeux de données\end{itemize}

\clearpage
\section{Ville de Pirae}


\begin{center}
  \includegraphics[width=3cm]{images/orga/1c_d3129aab4b45e9a1e34bd0e88b057d-100.jpg}
\end{center}


La ville de Pirae est située au nord de l'île de Tahiti en Polynésie
française.

La commune de Pirae a été créée le 18 janvier 1965. Et son tout premier
conseil municipal a eu lieu le 5 mai de la même année.

D'une superficie de 3 542 hectares (dont 1/5ème seulement soit 700 ha
est urbanisable), le territoire de la commune de Pirae s'étire entre
Papeete et Arue, de la baie de Taaone jusqu'au sommet du Mont Aorai à 2
066 m d'altitude. Il est drainé par trois rivières : Fautaua, Hamuta et
Nahoata.

Sa population compte 14 209 habitants.


\vspace{0.5cm}

\needspace{12\baselineskip}
\subsection*{Liste annuelle des prénoms des nouveaux-nés déclarés à l'état-civil de
Pirae
}\index{etat!civil}\index{naissance}\index{pirae}\index{polynesie}\index{prenom}\index{tahiti}
  \begin{wrapfigure}{r}{2.5cm}
    \centering
    \qrcode[nolink]{https://data.gouv.fr/dataset/5a90572888ee384a37a8bf83}
  \end{wrapfigure}

Licence : \textbf{Licence Ouverte
}\newline
Créé le : 2018-02-23\newline
Modifié le : 2019-01-03\newline
Mise à jour : annuelle\newline
Popularité : 1 réutilisation,  0 suivi\newline
Mots-clé : \emph{etat-civil, naissance, pirae, polynesie, prenom, tahiti
}\newline
Permalien : \url{https://data.gouv.fr/dataset/5a90572888ee384a37a8bf83}\newline

\par
\noindent
    Ce jeu de données fait partie du Socle Commun des Données Locales
(SCDL).Le jeu de données PRENOM contient les données essentielles
suivantes (SCDL Prénoms des nouveaux-nés V 1.1.2) :


\vspace{0.5cm}
\needspace{3\baselineskip} \rule{4cm}{0.25pt}\newline\textbf{Aussi disponible du même producteur :}\begin{itemize}
\item \href{https://data.gouv.fr/dataset/5a944adc88ee3849d5288fb7}{Catalogue des jeux de données publiés en OpenData}
\item \href{https://data.gouv.fr/dataset/586c5864c751df5a392b7154}{Délibérations du conseil municipal - année 2013}
\item \href{https://data.gouv.fr/dataset/5859c9a788ee3824b3609f67}{Délibérations du conseil municipal - année 2014}
\item \href{https://data.gouv.fr/dataset/580f1be5c751df1ec2c562c5}{Délibérations du conseil municipal - année 2015}
\item \href{https://data.gouv.fr/dataset/580d7292c751df6d02c562c5}{Délibérations du conseil municipal - année 2016}
\item \href{https://data.gouv.fr/dataset/586c59dfc751df5ad12b7154}{Délibérations du conseil municipal - année 2017}
\item \href{https://data.gouv.fr/dataset/5a53c218c751df79c9458764}{Délibérations du conseil municipal - année 2018}
\item \href{https://data.gouv.fr/dataset/5c1d8af08b4c412e46e0ad97}{Délibérations du conseil municipal - année 2019}
\item \href{https://data.gouv.fr/dataset/5a8f387988ee381bb2a62a46}{Liste des délibérations adoptées par le conseil municipal de Pirae}
\end{itemize}

\clearpage
\section{Ville de Rennes}


\begin{center}
  \includegraphics[width=3cm]{images/orga/2f_e8474cb3f84c26a7315573588d3ccd-100.png}
\end{center}


La Ville de Rennes est la capitale historique et administrative de la
Bretagne. Elle compte plus de 210 000 habitants. Première collectivité à
ouvrir ses données en 2010, Rennes et sa métropole sont engagées dans
l'association Open Data France et revendique sa réputation de territoire
d'expérimentation sur le numérique et l'open data.


\vspace{0.5cm}

\needspace{12\baselineskip}
\subsection*{Statistiques de fréquentation des expositions du Musée des Beaux-Arts de
Rennes
}\index{culture}\index{statistiques}
  \begin{wrapfigure}{r}{2.5cm}
    \centering
    \qrcode[nolink]{https://data.gouv.fr/dataset/5369a073a3a729239d206385}
  \end{wrapfigure}

Licence : \textbf{Open Data Commons Open Database License (ODbL)
}\newline
Créé le : 2013-11-13\newline
Modifié le : 2015-12-30\newline
Popularité : 1 réutilisation,  0 suivi\newline
Mots-clé : \emph{culture, statistiques
}\newline
Permalien : \url{https://data.gouv.fr/dataset/5369a073a3a729239d206385}\newline

\par
\noindent
    Statistiques de fréquentation individuelle, par groupe et globale des
expositions du Musée des Beaux-Arts de Rennes depuis 1997.


\vspace{0.5cm}
\needspace{3\baselineskip} \rule{4cm}{0.25pt}\newline\textbf{Aussi disponible du même producteur :}\begin{itemize}
\item \href{https://data.gouv.fr/dataset/53698e75a3a729239d203482}{Actualités de la Ville de Rennes}
\item \href{https://data.gouv.fr/dataset/53698f51a3a729239d2036fa}{Base de données Vivre à Rennes}
\item \href{https://data.gouv.fr/dataset/53698fb0a3a729239d2037ef}{BP 2010 - BUDGET PRINCIPAL}
\item \href{https://data.gouv.fr/dataset/53698fb1a3a729239d2037f0}{BP 2010 - BUDGETS ANNEXES}
\item \href{https://data.gouv.fr/dataset/53698fb1a3a729239d2037f1}{BP 2010 - SUBVENTIONS AUX ASSOCIATIONS}
\item \href{https://data.gouv.fr/dataset/53698fb2a3a729239d2037f2}{BP 2011 - BUDGET PRINCIPAL}
\item \href{https://data.gouv.fr/dataset/53698fb2a3a729239d2037f3}{BP 2011 - BUDGETS ANNEXES}
\item \href{https://data.gouv.fr/dataset/53698fb2a3a729239d2037f4}{BP 2011 - SUBVENTIONS AUX ASSOCIATIONS}
\item \href{https://data.gouv.fr/dataset/53698fb3a3a729239d2037f5}{BP 2012 - BUDGET PRINCIPAL}
\item \href{https://data.gouv.fr/dataset/53698fb3a3a729239d2037f6}{BP 2012 - BUDGETS ANNEXES}
\item \href{https://data.gouv.fr/dataset/53698fb4a3a729239d2037f7}{BP 2012 - SUBVENTIONS AUX ASSOCIATIONS}
\item \href{https://data.gouv.fr/dataset/53698fb4a3a729239d2037f9}{BP 2013 - BUDGET PRINCIPAL}
\item \href{https://data.gouv.fr/dataset/53698fb5a3a729239d2037fa}{BP 2013 - SUBVENTIONS AUX ASSOCIATIONS}
\item \href{https://data.gouv.fr/dataset/53698fe1a3a729239d20386c}{CA 2008 - BUDGET PRINCIPAL}
\item \href{https://data.gouv.fr/dataset/53698fe1a3a729239d20386d}{CA 2008 - BUDGETS ANNEXES}
\item \href{https://data.gouv.fr/dataset/53698fe2a3a729239d20386e}{CA 2008 - SUBVENTIONS AUX ASSOCIATIONS}
\item \href{https://data.gouv.fr/dataset/53698fe2a3a729239d20386f}{CA 2009 - BUDGET PRINCIPAL}
\item \href{https://data.gouv.fr/dataset/53698fe2a3a729239d203870}{CA 2009 - BUDGETS ANNEXES}
\item \href{https://data.gouv.fr/dataset/53698fe3a3a729239d203871}{CA 2009 - SUBVENTIONS AUX ASSOCIATIONS}
\item \href{https://data.gouv.fr/dataset/53698fe3a3a729239d203872}{CA 2010 - BUDGET PRINCIPAL}
\item \href{https://data.gouv.fr/dataset/53698fe4a3a729239d203873}{CA 2010 - BUDGETS ANNEXES}
\item \href{https://data.gouv.fr/dataset/53698fe4a3a729239d203874}{CA 2010 - SUBVENTIONS AUX ASSOCIATIONS}
\item \href{https://data.gouv.fr/dataset/53698fe4a3a729239d203875}{CA 2011 - Budget Principal}
\item \href{https://data.gouv.fr/dataset/53698fe5a3a729239d203876}{CA 2011 - Budgets annexes}
\item \href{https://data.gouv.fr/dataset/53698fe5a3a729239d203877}{CA 2011 - Subventions aux associations}
\item \href{https://data.gouv.fr/dataset/53698fe6a3a729239d203878}{CA 2012 - Budget Principal}
\item \href{https://data.gouv.fr/dataset/53698fe6a3a729239d203879}{CA 2012 - Budgets annexes}
\item \href{https://data.gouv.fr/dataset/53698fe6a3a729239d20387a}{CA 2012 - Subventions aux associations}
\item \href{https://data.gouv.fr/dataset/5369930ba3a729239d2040a4}{Documents prêtés dans les bibliothèques municipales de Rennes le 1er octobre 2013}
\item \href{https://data.gouv.fr/dataset/53699d9da3a729239d205c97}{Prénoms des enfants nés à Rennes de 2007 à 2011}
\item \href{https://data.gouv.fr/dataset/53699da7a3a729239d205caf}{Prêts d'oeuvres du musée des Beaux-Arts de Rennes}
\item \href{https://data.gouv.fr/dataset/53699f36a3a729239d20609c}{Résultats des élections cantonales à Rennes de 2004 à 2011}
\item \href{https://data.gouv.fr/dataset/53699f37a3a729239d20609f}{Résultats des élections européennes à Rennes de 2004 à 2011}
\item \href{https://data.gouv.fr/dataset/53699f39a3a729239d2060a2}{Résultats des élections législatives à Rennes de 2007 à 2012}
\item \href{https://data.gouv.fr/dataset/53699f3ca3a729239d2060a9}{Résultats des élections présidentielles à Rennes depuis 2007}
\item \href{https://data.gouv.fr/dataset/53699f3ca3a729239d2060aa}{Résultats des élections régionales à Rennes de 2004 à 2011}
\item \href{https://data.gouv.fr/dataset/53699f3fa3a729239d2060b2}{Résultats du référendum à Rennes de 2004 à 2011}
\end{itemize}

\clearpage
\section{Ville de Saint-Malo}


\begin{center}
  \includegraphics[width=3cm]{images/orga/71_d80c263317463ca8050811b5b22be6-100.png}
\end{center}


Située à 75 km au nord de Rennes en bordure de l'estuaire de la Rance,
l'actuelle commune de Saint-Malo résulte de la fusion de 1967 de
l'ancienne commune de Saint-Malo (la vieille ville intra-muros) avec
celles de Paramé et Saint-Servan. Aujourd'hui, Saint-Malo, peuplée de 47
000 habitants, est la deuxième ville d'Ille-et-Vilaine après Rennes.

Mondialement connue et première destination touristique bretonne, la
\href{http://www.saint-malo.fr/}{ville de Saint-Malo} se caractérise
aussi par son économie fortement tournée vers la mer, grâce à un port de
commerce et de pêche de dimension régionale, un port de plaisance et un
port de voyageurs vers les iles anglo-normandes et l'Angleterre.

Grâce au réseau ferroviaire à grande vitesse, \textbf{Saint-Malo sera en
2017 la ville littorale la plus proche de Paris en temps de
déplacement}.

Le conseil municipal, par délibération du 3 juillet 2014, a adopté le
principe d'engager la commune dans une démarche d'ouverture de ses
données publiques et a adopté la Licence Ouverte pour la réutilisation
de ses données.


\vspace{0.5cm}

\needspace{12\baselineskip}
\subsection*{Défibrillateurs
}\index{sante}\index{secours}\index{securite!civile}\index{urgence}
  \begin{wrapfigure}{r}{2.5cm}
    \centering
    \qrcode[nolink]{https://data.gouv.fr/dataset/54295e4188ee380326a5915d}
  \end{wrapfigure}

Licence : \textbf{Licence Ouverte
}\newline
Créé le : 2014-09-29\newline
Modifié le : 2015-10-28\newline
Granularité : à la commune\newline
Mise à jour : ponctuelle\newline
Popularité : 1 réutilisation,  1 suivi\newline
Mots-clé : \emph{sante, secours, securite-civile, urgence
}\newline
Permalien : \url{https://data.gouv.fr/dataset/54295e4188ee380326a5915d}\newline

\par
\noindent
    Localisation des défibrillateurs automatiques externes (DAE) gérés par
la commune.


\vspace{0.5cm}
\needspace{12\baselineskip}
\subsection*{Menus de la Cuisine Centrale
}\index{cantine}\index{centres!de!loisirs}\index{cuisine}\index{repas}\index{restauration}\index{scolaire}
  \begin{wrapfigure}{r}{2.5cm}
    \centering
    \qrcode[nolink]{https://data.gouv.fr/dataset/55882e79c751df7991a453b9}
  \end{wrapfigure}

Licence : \textbf{Licence Ouverte
}\newline
Créé le : 2015-06-22\newline
Modifié le : 2016-08-25\newline
Granularité : à la commune\newline
Mise à jour : ponctuelle\newline
Popularité : 1 réutilisation,  1 suivi\newline
Mots-clé : \emph{cantine, centres-de-loisirs, cuisine, repas, restauration, scolaire
}\newline
Permalien : \url{https://data.gouv.fr/dataset/55882e79c751df7991a453b9}\newline

\par
\noindent
    Les menus proposés dans les cantines des écoles et des centres de
loisirs de Saint-Malo sont élaborés par la Cuisine Centrale.

Les menus sont conçus plusieurs semaines à l'avance : une Commission
Restauration, réunissant des élus, des représentants de la Direction de
l'Education, des agents de services, des parents d'élèves, des
directeurs d'écoles et des représentants de la DDEN, a lieu tous les 2
mois pour valider les menus proposés pour la période à venir.

Les fichiers fournis ci-dessous correspondent aux menus validés qui
seront réellement proposés dans les cantines, sous réserve de rares et
éventuelles modifications de dernière minute.

Pour tous les aliments mis en oeuvre, des informations qualitatives
peuvent être précisées :

\begin{itemize}

\item
  Arôme naturel
\item
  Bio
\item
  Local : \emph{en provenance de l'Ille-et-Vilaine, en majorité autour
  de Saint-Malo. L'indication Local pour le poisson signifie que
  l'approvisionnement se fait auprès de la Compagnie des Pêches de
  Saint-Malo}
\item
  Label Rouge
\item
  CCP : \emph{Certification de Conformité Produit}, qui signale le
  respect d'un cahier des charges de critères qualitatifs, contrôlé par
  un organisme certificateur tiers
\item
  Boeuf VF
\item
  Boeuf race à viande VF
\item
  MSC : \emph{Marine Stewardship Council, label de pêche durable}
\item
  Bleu Blanc Coeur : \emph{Démarche d'alimentation animale basée sur
  l'herbe et les graines de lin oléagineux, de manière à produire des
  aliments plus riches en oméga-3}
\item
  AOC-AOP
\item
  Présence de porc
\item
  Elaboré par le Chef et son équipe : \emph{plat entièrement élaboré à
  la cuisine centrale}
\item
  Dessert du Chef : \emph{dessert entièrement élaboré à la cuisine
  centrale}
\item
  Nouveauté
\end{itemize}

Source : SODEXO, titulaire du marché de prestation de services de
restauration scolaire et hospitalière


\vspace{0.5cm}
\needspace{12\baselineskip}
\subsection*{Nombre annuel de mariages à Saint-Malo
}\index{etat!civil}\index{mariage}
  \begin{wrapfigure}{r}{2.5cm}
    \centering
    \qrcode[nolink]{https://data.gouv.fr/dataset/5407f3f6a3a7292ef9c20a5f}
  \end{wrapfigure}

Licence : \textbf{Licence Ouverte
}\newline
Créé le : 2014-09-03\newline
Modifié le : 2016-02-21\newline
De 1993-01-01 à 2014-12-31\newline
Granularité : à la commune\newline
Mise à jour : annuelle\newline
Popularité : 1 réutilisation,  0 suivi\newline
Mots-clé : \emph{etat-civil, mariage
}\newline
Permalien : \url{https://data.gouv.fr/dataset/5407f3f6a3a7292ef9c20a5f}\newline

\par
\noindent
    Nombre de mariages célébrés chaque année en mairie de Saint-Malo. Source
: service de l'état civil de la Mairie.


\vspace{0.5cm}
\needspace{12\baselineskip}
\subsection*{ROUTE DU RHUM 2014 - Implantation des bateaux dans le port de Saint-Malo
}
  \begin{wrapfigure}{r}{2.5cm}
    \centering
    \qrcode[nolink]{https://data.gouv.fr/dataset/5433b5f088ee38108e64e607}
  \end{wrapfigure}

Licence : \textbf{Licence Ouverte
}\newline
Créé le : 2014-10-07\newline
Modifié le : 2016-01-20\newline
De 2014-10-24 à 2014-11-02\newline
Granularité : à la commune\newline
Mise à jour : ponctuelle\newline
Popularité : 4 réutilisations,  0 suivi\newline
Mots-clé : \emph{aucun
}\newline
Permalien : \url{https://data.gouv.fr/dataset/5433b5f088ee38108e64e607}\newline

\par
\noindent
    Implantation géographique des bateaux concurrents, sur les pontons et
les quais des bassins intérieurs du port de Saint-Malo

Les fichiers sont fournis en date du 7 octobre 2014. En fonction des
inscriptions définitives et des résultats des qualifications,
l'implantation peut varier jusqu'à la date du départ, le 2 novembre
2014.

Des mises à jour pourront être publiées d'ici là : {[}EDIT{]} 17 octobre
2014 - Ajout d'un 91ème bateau


\vspace{0.5cm}
\needspace{12\baselineskip}
\subsection*{Statistiques démographiques à Saint-Malo
}\index{deces}\index{demographie}\index{etat!civil}\index{naissances}\index{population}\index{recensement}
  \begin{wrapfigure}{r}{2.5cm}
    \centering
    \qrcode[nolink]{https://data.gouv.fr/dataset/54296a4788ee380326a59161}
  \end{wrapfigure}

Licence : \textbf{Licence Ouverte
}\newline
Créé le : 2014-09-29\newline
Modifié le : 2015-08-28\newline
De 1993-01-01 à 2014-12-31\newline
Granularité : à la commune\newline
Mise à jour : annuelle\newline
Popularité : 1 réutilisation,  0 suivi\newline
Mots-clé : \emph{deces, demographie, etat-civil, naissances, population, recensement
}\newline
Permalien : \url{https://data.gouv.fr/dataset/54296a4788ee380326a59161}\newline

\par
\noindent
    Chiffres annuels des naissances et décès.

Le fichier contient les données suivantes :

\begin{itemize}

\item
  \emph{Naissances INSEE} : naissances \textbf{domiciliées }à
  Saint-Malo, comptabilisées à partir des bulletins de l'état civil et
  des transcriptions des jugements déclaratifs de naissance établis par
  les tribunaux.
\item
  \emph{Naissances Etat civil} : naissances intervenues en tous lieux
  sur le territoire de la commune, \textbf{quel que soit le domicile des
  parents} (hors transcriptions)
\item
  \emph{Décès INSEE} : décès \textbf{domiciliés }à Saint-Malo,
  comptabilisés à partir des bulletins de l'état civil et des
  transcriptions des jugements déclaratifs de décès établis par les
  tribunaux.
\item
  \emph{Décès Etat civil} : décès intervenus en tous lieux sur le
  territoire de la commune, \textbf{quel que soit le domicile de la
  personne} (hors transcriptions et actes d'enfants sans vie)
\end{itemize}


\vspace{0.5cm}
\needspace{12\baselineskip}
\subsection*{Statistiques logements
}\index{logement}\index{residence}
  \begin{wrapfigure}{r}{2.5cm}
    \centering
    \qrcode[nolink]{https://data.gouv.fr/dataset/5587de2bc751df728ea453bd}
  \end{wrapfigure}

Licence : \textbf{Licence Ouverte
}\newline
Créé le : 2015-06-22\newline
Modifié le : 2016-03-07\newline
De 1968-01-01 à 2012-12-31\newline
Granularité : à la commune\newline
Mise à jour : annuelle\newline
Popularité : 1 réutilisation,  0 suivi\newline
Mots-clé : \emph{logement, residence
}\newline
Permalien : \url{https://data.gouv.fr/dataset/5587de2bc751df728ea453bd}\newline

\par
\noindent
    Chiffres de l'INSEE concernant la commune de Saint-Malo, détaillant :

\begin{itemize}

\item
  Année (\emph{de 1968 à 1999, comptages dans le cadre des recensements
  ``quinquennaux''. A partir de 2006, méthode de recensement en
  continu})
\item
  Nombre de logements
\item
  dont Résidences principales
\item
  dont Résidences secondaires (\emph{logements utilisés pour les
  week-ends, loisirs ou vacances. Les locations meublées pour des
  séjours touristiques sont également classés en résidences
  secondaires})
\item
  dont Logements vacants (\emph{logements proposés à la vente ou à la
  location, en attente de règlement de succession, logements de fonction
  inoccupés})
\end{itemize}


\vspace{0.5cm}
\needspace{3\baselineskip} \rule{4cm}{0.25pt}\newline\textbf{Aussi disponible du même producteur :}\begin{itemize}
\item \href{https://data.gouv.fr/dataset/54881d6dc751df32f8a3fc15}{Campings}
\item \href{https://data.gouv.fr/dataset/54f71c9ec751df1d1e882844}{Cantons de Saint-Malo}
\item \href{https://data.gouv.fr/dataset/54885606c751df0c3f3f322c}{Centrales photovoltaïques communales}
\item \href{https://data.gouv.fr/dataset/54885216c751df038b3f322c}{Certificats d'économies d'énergie}
\item \href{https://data.gouv.fr/dataset/548febecc751df2ef96accbb}{Cimetières malouins}
\item \href{https://data.gouv.fr/dataset/54f70a69c751df7e92882844}{Collecteurs de piles et batteries usagées}
\item \href{https://data.gouv.fr/dataset/553f4e8fc751df20b8e96a3b}{Effectifs scolaires}
\item \href{https://data.gouv.fr/dataset/53bb1b17a3a72947192f0bc6}{Emplacement des bureaux de vote}
\item \href{https://data.gouv.fr/dataset/5489d06bc751df32434120e7}{Emplacement des stations service à Saint-Malo}
\item \href{https://data.gouv.fr/dataset/540917e1a3a72959e0628f93}{Fréquentation de l'Office du Tourisme}
\item \href{https://data.gouv.fr/dataset/54d0980cc751df4efc467389}{Fréquentation du site saint-malo.fr}
\item \href{https://data.gouv.fr/dataset/5409181fa3a72959e0628f94}{Grandes Marées à Saint-Malo}
\item \href{https://data.gouv.fr/dataset/55882244c751df5f2ca453b9}{Hébergements touristiques}
\item \href{https://data.gouv.fr/dataset/55a766b2c751df158c330fd5}{Index des rues}
\item \href{https://data.gouv.fr/dataset/540af7e5a3a7297a99d24efd}{Indice IQA de la qualité de l'air}
\item \href{https://data.gouv.fr/dataset/53bfa522a3a7293e19309ef8}{Liste des conseillers municipaux}
\item \href{https://data.gouv.fr/dataset/57bea758c751df0fb197bae5}{Liste des marchés publics}
\item \href{https://data.gouv.fr/dataset/57beb6b1c751df2b6697bae5}{Médiathèque de Saint-Malo : les auteurs les plus empruntés}
\item \href{https://data.gouv.fr/dataset/57beb7a3c751df2d8297bae5}{Médiathèque de Saint-Malo : les BD les plus empruntés}
\item \href{https://data.gouv.fr/dataset/57beb23fc751df233897bae5}{Médiathèque de Saint-Malo : les documents les plus empruntés}
\item \href{https://data.gouv.fr/dataset/57beaf9dc751df1ed197bae5}{Médiathèque de Saint-Malo : statistiques}
\item \href{https://data.gouv.fr/dataset/540afc18a3a7297a99d24efe}{Mesures de la qualité de l'air}
\item \href{https://data.gouv.fr/dataset/5429648588ee380327a59164}{Monuments historiques classés à Saint-Malo}
\item \href{https://data.gouv.fr/dataset/5488088cc751df17b3a3fc15}{Nuitées dans les campings municipaux}
\item \href{https://data.gouv.fr/dataset/54d09b06c751df5458467389}{Origine géographique des visiteurs du site saint-malo.fr}
\item \href{https://data.gouv.fr/dataset/54886ce8c751df27a53f322c}{Parcours historique Saint-Malo Intra-Muros}
\item \href{https://data.gouv.fr/dataset/57bea96ac751df13cc97bae5}{Parking relais Paul Féval - Navettes de bus}
\item \href{https://data.gouv.fr/dataset/553e499fc751df122ee96a3b}{Périmètres des bureaux de vote}
\item \href{https://data.gouv.fr/dataset/55c36f1888ee383e51a46ec2}{Plages sans tabac}
\item \href{https://data.gouv.fr/dataset/55a75ffac751df765f330fd5}{Points d'Apport Volontaire Textile}
\item \href{https://data.gouv.fr/dataset/5429631488ee380329a59160}{Population légale de Saint-Malo}
\item \href{https://data.gouv.fr/dataset/53bcff48a3a729713651c36b}{Postes de secours}
\item \href{https://data.gouv.fr/dataset/5407f862a3a7292ef9c20a61}{Prénoms donnés aux enfants nés à Saint-Malo}
\item \href{https://data.gouv.fr/dataset/54885f53c751df18303f322d}{Production photovoltaïque sur les bâtiments communaux}
\item \href{https://data.gouv.fr/dataset/53bb26a9a3a72947192f0bc8}{Qualité des eaux de baignade sur le littoral de Saint-Malo}
\item \href{https://data.gouv.fr/dataset/54f710fac751df06ac882844}{Quartiers de Saint-Malo}
\item \href{https://data.gouv.fr/dataset/53bfabf6a3a7293e19309efa}{Résultats d'analyse des eaux de baignade}
\item \href{https://data.gouv.fr/dataset/553e52bbc751df2029e96a3b}{Résultats électoraux}
\item \href{https://data.gouv.fr/dataset/5582f0e6c751df0c14a453b9}{Statistiques annuelles de crémations}
\item \href{https://data.gouv.fr/dataset/549038cdc751df227c6accbb}{Suivez l'hermine !}
\item \href{https://data.gouv.fr/dataset/55561c46c751df7ecf190c78}{Tracés des balades Chlorophylle}
\item \href{https://data.gouv.fr/dataset/53bd05b0a3a729713651c36d}{Zones de mouillages}
\end{itemize}

\clearpage
\section{VILLE DE VINCENNES}


\begin{center}
  \includegraphics[width=3cm]{images/orga/56_005cf409b14f338bfc36d4df6ffc21-100.jpg}
\end{center}


Bénéficiant d'un cadre privilégié, Vincennes offre une qualité de vie
indéniable : des commerces dynamiques, des activités associatives
foisonnantes, un réseau de transports dense, des équipements publics de
qualité, une sécurité avérée et un patrimoine exceptionnel. Quelques
chiffres

\begin{verbatim}
Département : Val-de-Marne (94)
Territoire : Paris Est Marne & Bois
Arrondissement : Nogent-sur-Marne
Cantons : Vincennes Est et Vincennes Ouest
Population : 49 908 habitants (population légale au 1er janvier 2018), les Vincennoises et Vincennois
Superficie : 1,91 km\textsuperscript{2}
Densité : 26 130 habitants par km\textsuperscript{2}
Altitude : 52 m
\end{verbatim}


\vspace{0.5cm}

\needspace{12\baselineskip}
\subsection*{Localisation des places de stationnement réservées aux personnes
titulaires d'une carte de stationnement pour handicapés à Vincennes
}\index{handicap}\index{mobilite}\index{stationnement}\index{stationnement!en!voirie}\index{vincennes}
  \begin{wrapfigure}{r}{2.5cm}
    \centering
    \qrcode[nolink]{https://data.gouv.fr/dataset/5c1cfab7634f415ae823e505}
  \end{wrapfigure}

Licence : \textbf{Open Data Commons Open Database License (ODbL)
}\newline
Créé le : 2018-12-21\newline
Modifié le : 2018-12-21\newline
De 1992-01-01 à 2018-12-31\newline
Granularité : à la commune\newline
Mise à jour : annuelle\newline
Popularité : 1 réutilisation,  0 suivi\newline
Mots-clé : \emph{handicap, mobilite, stationnement, stationnement-en-voirie, vincennes
}\newline
Permalien : \url{https://data.gouv.fr/dataset/5c1cfab7634f415ae823e505}\newline

\par
\noindent
    Ce jeu de données présente la localisation des places de stationnement
réservées aux personnes titulaires d'une carte de stationnement pour
handicapés (carte européenne de stationnement ou carte mobilité
inclusion) dans la ville de Vincennes. Les champs proposés sont les
suivants : COMMUNE\_NOM : Nom officiel de la commune / Format Texte
PROJECTION : Projection des coordonnées / Format Texte, code
alphanumérique X : Longitude / Format Texte, Code numérique avec
ponctuation. A noter : la longitude est appréciée par rapport à la
projection RGF93 / Lambert-93 Y : Latitude / Format Texte, Code
numérique avec ponctuation. A noter : la latitude est appréciée par
rapport à la projection RGF93 / Lambert-93 ADRESSE : Adresse de
localisation de la place de stationnement réservée / Format Texte\\
ANNEE\_CREATION : Année de création de la place / Format Numérique\\
RESPECT\_NORMES : Respect des normes sur le stationnement réservé aux
personnes handicapées titulaires d'une carte de stationnement / Format
Texte (oui/non). A noter : les bases applicables sont les décrets
n\degree{}2006-1657 et 2006-1658, et les arrêtés du 15 janvier 2007 et
du 20 avril 2017.


\vspace{0.5cm}
\needspace{12\baselineskip}
\subsection*{Nombre de déclarations de décès effectuées ou transcrites à Vincennes
depuis 1992
}\index{deces}\index{vincennes}
  \begin{wrapfigure}{r}{2.5cm}
    \centering
    \qrcode[nolink]{https://data.gouv.fr/dataset/5c125ff3634f41445bbc27a3}
  \end{wrapfigure}

Licence : \textbf{Open Data Commons Open Database License (ODbL)
}\newline
Créé le : 2018-12-13\newline
Modifié le : 2018-12-21\newline
De 1992-01-01 à 2017-12-31\newline
Granularité : à la commune\newline
Mise à jour : annuelle\newline
Popularité : 1 réutilisation,  0 suivi\newline
Mots-clé : \emph{deces, vincennes
}\newline
Permalien : \url{https://data.gouv.fr/dataset/5c125ff3634f41445bbc27a3}\newline

\par
\noindent
    Ce jeu de données présente le nombre annuel de déclarations de décès :
-Effectuées à Vincennes, lorsque la personne est décédée à Vincennes ;
-Transcrites à Vincennes, lorsque la personne est vincennoise et est
décédée dans une autre commune. Les champs proposés sont les suivants :
COMMUNE\_NOM : Nom officiel de la commune / Format Texte ANNEE : Année
du décès sur 4 chiffres / Format Numérique\\
NOMBRE\_DECES : Nombre de déclarations de décès effectuées ou
transcrites à Vincennes / Format Numérique. A noter : les déclarations
de décès sont effectuées à Vincennes, lorsque la personne est décédée à
Vincennes, et transcrites à Vincennes, lorsque la personne est
vincennoise et est décédée dans une autre commune


\vspace{0.5cm}
\needspace{3\baselineskip} \rule{4cm}{0.25pt}\newline\textbf{Aussi disponible du même producteur :}\begin{itemize}
\item \href{https://data.gouv.fr/dataset/5c125dd1634f4141cbf10251}{Nombre de mariages célébrés à la mairie de Vincennes depuis 1992}
\item \href{https://data.gouv.fr/dataset/5c125f08634f4141ba10d759}{Nombre de PACS enregistrés à la mairie de Vincennes}
\item \href{https://data.gouv.fr/dataset/5c1259eb634f413942df57d8}{Prénoms des nouveaux nés donnés à l’état civil à Vincennes en 2015}
\item \href{https://data.gouv.fr/dataset/5c125c1b634f413b99dc89d0}{Prénoms des nouveaux nés donnés à l’état civil à Vincennes en 2016}
\item \href{https://data.gouv.fr/dataset/5c125c9b634f413de32266bb}{Prénoms des nouveaux nés donnés à l’état civil à Vincennes en 2017}
\item \href{https://data.gouv.fr/dataset/5c1d0675634f41014b04cfd7}{Subventions de plus de 23 000 \euro{} accordées en 2018 par la ville de Vincennes}
\end{itemize}

\clearpage
\section{Ville d'Issy-les-Moulineaux}


\begin{center}
  \includegraphics[width=3cm]{images/orga/b9_beecacecbd4d42ac10ef244f11e44e-100.png}
\end{center}


Portail open data d'Issy (http://data.issy.com) / Twitter d'Issy
{[}@data\_issy{]}(https://twitter.com/data\_issy) / Site
d'Issy-les-Moulineaux \href{http://www.issy.com}{Issy.com} / L'agence
numérique de Grand Paris Seine Ouest
\href{http://seineouestdigital.fr/}{SO digital}

La Ville d'Issy-les-Moulineaux libère des données pour stimuler la
création de nouveaux services.
\href{https://wiki.data.gouv.fr/wiki/Outillage_pour_les_datavisualisations}{Réutilisez
ces données, par exemple en créant des représentations graphiques avec
les outils conseillés par data.gouv.fr} ou des applications mobiles
grâce au projet européen \href{http://www.citadelonthemove.eu/}{Citadel
on the move.}

Testez aussi sur le portail open data d'Issy les outils de création de
cartes
\url{http://bit.ly/19Yt9Fy}{]}(http://bit.ly/19Yt9Fy{]}(http://bit.ly/19Yt9Fy))et
de création de graphiques
\url{http://bit.ly/1NhT5Kj}{]}(http://bit.ly/1NhT5Kj{]}(http://bit.ly/1NhT5Kj))


\vspace{0.5cm}

\needspace{12\baselineskip}
\subsection*{Agenda géolocalisé des événements
}\index{agenda}\index{evenements}\index{evenements!culturels}\index{evenementsportif}\index{issy!les!moulienaux}\index{sorties}
  \begin{wrapfigure}{r}{2.5cm}
    \centering
    \qrcode[nolink]{https://data.gouv.fr/dataset/53698e88a3a729239d2034bd}
  \end{wrapfigure}

Licence : \textbf{Licence Ouverte
}\newline
Créé le : 2014-02-10\newline
Modifié le : 2016-03-12\newline
Granularité : à la commune\newline
Mise à jour : quotienne\newline
Popularité : 2 réutilisations,  1 suivi\newline
Mots-clé : \emph{agenda, evenements, evenements-culturels, evenementsportif, issy-les-moulienaux, sorties
}\newline
Permalien : \url{https://data.gouv.fr/dataset/53698e88a3a729239d2034bd}\newline

\par
\noindent
    Ce fichier permet d'afficher sur un plan d'Issy tous les événements à
venir. Des filtres permettent de n'afficher que les événements près de
chez soi, et/ou uniquement en rapport avec un centre d'intérêt (les
sorties culturelles, les événements sportifs\ldots{}), à une période
déterminée. Très facilement, l'utilisateur qui s'intéresse à la culture
peut par exemple savoir que le week-end prochain, il peut aller près de
chez lui à une exposition au Musée de la Carte à Jouer ou assister à une
conférence à la Médiathèque!


\vspace{0.5cm}
\needspace{12\baselineskip}
\subsection*{Arbres remarquables
}\index{arbres}\index{arbres!remarquables}\index{biodiversite}\index{cadre!de!vie}\index{espaces!naturels}\index{espacesverts}
  \begin{wrapfigure}{r}{2.5cm}
    \centering
    \qrcode[nolink]{https://data.gouv.fr/dataset/53698f04a3a729239d203619}
  \end{wrapfigure}

Licence : \textbf{Licence Ouverte
}\newline
Créé le : 2014-02-14\newline
Modifié le : 2016-03-15\newline
Granularité : à la commune\newline
Mise à jour : ponctuelle\newline
Popularité : 2 réutilisations,  2 suivis\newline
Mots-clé : \emph{arbres, arbres-remarquables, biodiversite, cadre-de-vie, espaces-naturels, espacesverts
}\newline
Permalien : \url{https://data.gouv.fr/dataset/53698f04a3a729239d203619}\newline

\par
\noindent
    Ce jeu de données contient des informations sur les arbres remarquables
de la Ville d'Issy-les-Moulineaux: essences, localisation, lien vers du
contenu enrichi\ldots{}


\vspace{0.5cm}
\needspace{12\baselineskip}
\subsection*{Compte administratif 2014 d'Issy-les-Moulineaux
}\index{budget}\index{compte!administratif}
  \begin{wrapfigure}{r}{2.5cm}
    \centering
    \qrcode[nolink]{https://data.gouv.fr/dataset/5519478ec751df491d057c91}
  \end{wrapfigure}

Licence : \textbf{Licence Ouverte
}\newline
Créé le : 2015-03-30\newline
Modifié le : 2016-02-15\newline
Granularité : à la commune\newline
Mise à jour : annuelle\newline
Popularité : 1 réutilisation,  0 suivi\newline
Mots-clé : \emph{budget, compte-administratif
}\newline
Permalien : \url{https://data.gouv.fr/dataset/5519478ec751df491d057c91}\newline

\par
\noindent
    Ce jeu de données contient le compte administratif 2014 de la ville
d'Issy-les-Moulineaux. Le compte administratif a été adopté par le
Conseil Municipal du 9 avril 2015.


\vspace{0.5cm}
\needspace{12\baselineskip}
\subsection*{Défibrillateurs
}\index{cardiaque}\index{coeur}\index{sante}\index{secours}
  \begin{wrapfigure}{r}{2.5cm}
    \centering
    \qrcode[nolink]{https://data.gouv.fr/dataset/53699237a3a729239d203e7d}
  \end{wrapfigure}

Licence : \textbf{Licence Ouverte
}\newline
Créé le : 2014-02-14\newline
Modifié le : 2015-10-28\newline
Granularité : à la commune\newline
Mise à jour : ponctuelle\newline
Popularité : 2 réutilisations,  2 suivis\newline
Mots-clé : \emph{cardiaque, coeur, sante, secours
}\newline
Permalien : \url{https://data.gouv.fr/dataset/53699237a3a729239d203e7d}\newline

\par
\noindent
    Ce jeu de données contient des informations sur les défibrillateurs
présents à Issy-les-Moulineaux (bâtiments publics ou voie publique).


\vspace{0.5cm}
\needspace{12\baselineskip}
\subsection*{Entreprises
}\index{audiovisuel}\index{digital}\index{economie}\index{economie!numerique}\index{electronique}\index{entreprise}\index{entreprises}\index{informatique}\index{internet}\index{numerique}\index{telecommunications}\index{tic}
  \begin{wrapfigure}{r}{2.5cm}
    \centering
    \qrcode[nolink]{https://data.gouv.fr/dataset/5369945ba3a729239d204415}
  \end{wrapfigure}

Licence : \textbf{Licence Ouverte
}\newline
Créé le : 2014-02-28\newline
Modifié le : 2016-03-15\newline
Granularité : à la commune\newline
Mise à jour : ponctuelle\newline
Popularité : 1 réutilisation,  3 suivis\newline
Mots-clé : \emph{audiovisuel, digital, economie, economie-numerique, electronique, entreprise, entreprises, informatique, internet, numerique, telecommunications, tic
}\newline
Permalien : \url{https://data.gouv.fr/dataset/5369945ba3a729239d204415}\newline

\par
\noindent
    Ce jeu de données répertorie des entreprises isséennes dans le domaine
du numérique : nom, adresse, téléphone, site web, appartenance à un
réseau d'entreprises\ldots{}


\vspace{0.5cm}
\needspace{12\baselineskip}
\subsection*{Evolution de la Dotation Globale de Fonctionnement versée à
Issy-les-Moulineaux
}\index{dgf}\index{dotation!globale!de!fonctionneme}\index{recette}
  \begin{wrapfigure}{r}{2.5cm}
    \centering
    \qrcode[nolink]{https://data.gouv.fr/dataset/55194801c751df4c8b057c91}
  \end{wrapfigure}

Licence : \textbf{Licence Ouverte
}\newline
Créé le : 2015-03-30\newline
Modifié le : 2016-02-07\newline
Granularité : à la commune\newline
Mise à jour : annuelle\newline
Popularité : 1 réutilisation,  0 suivi\newline
Mots-clé : \emph{dgf, dotation-globale-de-fonctionneme, recette
}\newline
Permalien : \url{https://data.gouv.fr/dataset/55194801c751df4c8b057c91}\newline

\par
\noindent
    La Dotation Globale de Fonctionnement (DGF) est la principale composante
des dotations de l'État. C'est une recette de fonctionnement.

Le jeu de données contient les Dotations Globales de Fonctionnement déjà
perçues par la Ville d'Issy-les-Moulineaux et celles prévues.

A noter: l'État ayant décidé d'associer les collectivités à l'effort de
redressement des comptes publics, les montants devraient baisser très
fortement ces prochaines années.


\vspace{0.5cm}
\needspace{12\baselineskip}
\subsection*{La dette à Issy-les-Moulineaux
}\index{budget}\index{dette}\index{finances}
  \begin{wrapfigure}{r}{2.5cm}
    \centering
    \qrcode[nolink]{https://data.gouv.fr/dataset/54e46d10c751df7d3e467389}
  \end{wrapfigure}

Licence : \textbf{Licence Ouverte
}\newline
Créé le : 2015-02-18\newline
Modifié le : 2015-07-01\newline
Granularité : à la commune\newline
Mise à jour : annuelle\newline
Popularité : 1 réutilisation,  0 suivi\newline
Mots-clé : \emph{budget, dette, finances
}\newline
Permalien : \url{https://data.gouv.fr/dataset/54e46d10c751df7d3e467389}\newline

\par
\noindent
    Ce jeu de données donne l'encours de la dette (au 31 décembre de chaque
année), le montant remboursé chaque année et le montant des intérêts
versés.


\vspace{0.5cm}
\needspace{12\baselineskip}
\subsection*{Les abattements fiscaux votés par le Conseil Municipal
d'Issy-les-Moulineaux
}\index{abattement}\index{impots}
  \begin{wrapfigure}{r}{2.5cm}
    \centering
    \qrcode[nolink]{https://data.gouv.fr/dataset/551943f2c751df43f5057c91}
  \end{wrapfigure}

Licence : \textbf{Licence Ouverte
}\newline
Créé le : 2015-03-30\newline
Modifié le : 2015-10-09\newline
Granularité : à la commune\newline
Mise à jour : annuelle\newline
Popularité : 1 réutilisation,  0 suivi\newline
Mots-clé : \emph{abattement, impots
}\newline
Permalien : \url{https://data.gouv.fr/dataset/551943f2c751df43f5057c91}\newline

\par
\noindent
    Ce jeu de données contient les abattements votés par le Conseil
Municipal d'Issy-les-Moulineaux: généraux ou spéciaux, facultatifs ou
obligatoires. Pour chaque abattement, les taux légaux minimum et maximum
sont précisés, ainsi que le taux applicable à Issy-les-Moulineaux.


\vspace{0.5cm}
\needspace{12\baselineskip}
\subsection*{Les cimetières et pompes funèbres à Issy-les-Moulineaux
}\index{cimetiere}\index{pompes!funebres}
  \begin{wrapfigure}{r}{2.5cm}
    \centering
    \qrcode[nolink]{https://data.gouv.fr/dataset/54e46dfbc751df7d3e46738a}
  \end{wrapfigure}

Licence : \textbf{Licence Ouverte
}\newline
Créé le : 2015-02-18\newline
Modifié le : 2015-05-13\newline
Granularité : à la commune\newline
Mise à jour : annuelle\newline
Popularité : 1 réutilisation,  0 suivi\newline
Mots-clé : \emph{cimetiere, pompes-funebres
}\newline
Permalien : \url{https://data.gouv.fr/dataset/54e46dfbc751df7d3e46738a}\newline

\par
\noindent
    Liste des cimetières et pompes funèbres à Issy-les-Moulineaux.


\vspace{0.5cm}
\needspace{12\baselineskip}
\subsection*{Les dépenses de fonctionnement par types de dépenses
}\index{depenses}\index{fonctionnement}
  \begin{wrapfigure}{r}{2.5cm}
    \centering
    \qrcode[nolink]{https://data.gouv.fr/dataset/55194b6cc751df5025057c91}
  \end{wrapfigure}

Licence : \textbf{Licence Ouverte
}\newline
Créé le : 2015-03-30\newline
Modifié le : 2016-02-01\newline
Granularité : à la commune\newline
Mise à jour : annuelle\newline
Popularité : 1 réutilisation,  0 suivi\newline
Mots-clé : \emph{depenses, fonctionnement
}\newline
Permalien : \url{https://data.gouv.fr/dataset/55194b6cc751df5025057c91}\newline

\par
\noindent
    Ce jeu de données permet de comprendre où va l'argent. Les dépenses sont
classées en fonction de leur type.

Le FSRIF est le Fonds de Solidarité des communes de la Région
Ile-de-France.


\vspace{0.5cm}
\needspace{12\baselineskip}
\subsection*{Les dépenses de fonctionnement par types de services à la population
}\index{budget}\index{depenses}\index{services}
  \begin{wrapfigure}{r}{2.5cm}
    \centering
    \qrcode[nolink]{https://data.gouv.fr/dataset/55194b10c751df45a6057c92}
  \end{wrapfigure}

Licence : \textbf{Licence Ouverte
}\newline
Créé le : 2015-03-30\newline
Modifié le : 2015-07-28\newline
Granularité : à la commune\newline
Mise à jour : annuelle\newline
Popularité : 1 réutilisation,  0 suivi\newline
Mots-clé : \emph{budget, depenses, services
}\newline
Permalien : \url{https://data.gouv.fr/dataset/55194b10c751df45a6057c92}\newline

\par
\noindent
    Ce jeu de données permet de répondre aux questions ``Où va l'argent?''
et ``pour faire quoi?''.


\vspace{0.5cm}
\needspace{12\baselineskip}
\subsection*{Les principales actualités du moment à Issy-les-Moulineaux
}\index{flux}\index{issycom}
  \begin{wrapfigure}{r}{2.5cm}
    \centering
    \qrcode[nolink]{https://data.gouv.fr/dataset/54e464bfc751df72f8467389}
  \end{wrapfigure}

Licence : \textbf{Licence Ouverte
}\newline
Créé le : 2015-02-18\newline
Modifié le : 2015-07-06\newline
Granularité : à la commune\newline
Mise à jour : annuelle\newline
Popularité : 1 réutilisation,  0 suivi\newline
Mots-clé : \emph{flux, issycom
}\newline
Permalien : \url{https://data.gouv.fr/dataset/54e464bfc751df72f8467389}\newline

\par
\noindent
    Ce flux rss est celui des informations accessibles depuis la page
d'accueil du site de la ville
d'Issy-les-Moulineaux\url{http://www.issy.com}


\vspace{0.5cm}
\needspace{12\baselineskip}
\subsection*{Liste des points d'accès à internet
}\index{internet}\index{internet!acces}\index{wifi}\index{wifi!public}
  \begin{wrapfigure}{r}{2.5cm}
    \centering
    \qrcode[nolink]{https://data.gouv.fr/dataset/53f3e181a3a7291b78e8fc28}
  \end{wrapfigure}

Licence : \textbf{Licence Ouverte
}\newline
Créé le : 2014-08-19\newline
Modifié le : 2016-03-05\newline
Granularité : à la commune\newline
Mise à jour : annuelle\newline
Popularité : 2 réutilisations,  0 suivi\newline
Mots-clé : \emph{internet, internet-acces, wifi, wifi-public
}\newline
Permalien : \url{https://data.gouv.fr/dataset/53f3e181a3a7291b78e8fc28}\newline

\par
\noindent
    Le fichier contient la liste des structures publiques où une connexion
internet est accessible à Issy (ordinateur, borne ou wifi).


\vspace{0.5cm}
\needspace{12\baselineskip}
\subsection*{Monuments
}\index{art}\index{arts}\index{culture}\index{metal}\index{monument}\index{sculpture}
  \begin{wrapfigure}{r}{2.5cm}
    \centering
    \qrcode[nolink]{https://data.gouv.fr/dataset/53699a1ba3a729239d2053e5}
  \end{wrapfigure}

Licence : \textbf{Licence Ouverte
}\newline
Créé le : 2014-02-12\newline
Modifié le : 2016-03-12\newline
Granularité : à la commune\newline
Mise à jour : ponctuelle\newline
Popularité : 2 réutilisations,  1 suivi\newline
Mots-clé : \emph{art, arts, culture, metal, monument, sculpture
}\newline
Permalien : \url{https://data.gouv.fr/dataset/53699a1ba3a729239d2053e5}\newline

\par
\noindent
    La statuaire métallique publique constitue un pan important du décor à
Issy-les-Moulineaux. Le fichier contient les noms, descriptions,
adresses, données de géolocalisation, lien vers des photos\ldots{}


\vspace{0.5cm}
\needspace{12\baselineskip}
\subsection*{Principales dépenses d'équipement à Issy-les-Moulineaux en 2013
}\index{budget}\index{equipement}\index{finances}
  \begin{wrapfigure}{r}{2.5cm}
    \centering
    \qrcode[nolink]{https://data.gouv.fr/dataset/54e468a7c751df6fa9467389}
  \end{wrapfigure}

Licence : \textbf{Licence Ouverte
}\newline
Créé le : 2015-02-18\newline
Modifié le : 2015-07-23\newline
Granularité : à la commune\newline
Mise à jour : annuelle\newline
Popularité : 1 réutilisation,  0 suivi\newline
Mots-clé : \emph{budget, equipement, finances
}\newline
Permalien : \url{https://data.gouv.fr/dataset/54e468a7c751df6fa9467389}\newline

\par
\noindent
    Ce fichier donne en euros le montant des principales dépenses
d'équipement à Issy en 2013 dans les domaines de l'éducation, du sport
ou encore de la culture. Les équipements peuvent être localisés sur la
carte.


\vspace{0.5cm}
\needspace{12\baselineskip}
\subsection*{Répartition de la dette par prêteurs (au 31/12/N)
}\index{banques}\index{dette}\index{preteurs}
  \begin{wrapfigure}{r}{2.5cm}
    \centering
    \qrcode[nolink]{https://data.gouv.fr/dataset/550aee08c751df4bb2882844}
  \end{wrapfigure}

Licence : \textbf{Licence Ouverte
}\newline
Créé le : 2015-03-19\newline
Modifié le : 2015-03-20\newline
Granularité : à la commune\newline
Mise à jour : annuelle\newline
Popularité : 1 réutilisation,  0 suivi\newline
Mots-clé : \emph{banques, dette, preteurs
}\newline
Permalien : \url{https://data.gouv.fr/dataset/550aee08c751df4bb2882844}\newline

\par
\noindent
    Ce fichier contient la répartition de la dette par prêteurs au 31/12 de
chaque année. Les montants sont ceux du capital de la dette hors
intérêts.


\vspace{0.5cm}
\needspace{12\baselineskip}
\subsection*{Résultats des exercices budgétaires à Issy
}\index{budget}\index{resultat!exercice}\index{resultat!global}
  \begin{wrapfigure}{r}{2.5cm}
    \centering
    \qrcode[nolink]{https://data.gouv.fr/dataset/55194a58c751df4e40057c91}
  \end{wrapfigure}

Licence : \textbf{Licence Ouverte
}\newline
Créé le : 2015-03-30\newline
Modifié le : 2016-01-05\newline
Granularité : à la commune\newline
Mise à jour : annuelle\newline
Popularité : 1 réutilisation,  0 suivi\newline
Mots-clé : \emph{budget, resultat-exercice, resultat-global
}\newline
Permalien : \url{https://data.gouv.fr/dataset/55194a58c751df4e40057c91}\newline

\par
\noindent
    Ce jeu de données contient de nombreuses informations sur les grandes
masses budgétaires de la Ville d'Issy-les-Moulineaux: résultats
d'exercices, résultats antérieurs cumulés, reports d'investissement,
recettes, dépenses, capacités d'autofinancement, etc.


\vspace{0.5cm}
\needspace{12\baselineskip}
\subsection*{Résultats élections
}\index{election}\index{election!presidentielle}\index{elections}\index{elections!cantonales}\index{elections!europeennes}\index{elections!legislatives!assemblee}
  \begin{wrapfigure}{r}{2.5cm}
    \centering
    \qrcode[nolink]{https://data.gouv.fr/dataset/53699f41a3a729239d2060b5}
  \end{wrapfigure}

Licence : \textbf{Licence Ouverte
}\newline
Créé le : 2014-02-11\newline
Modifié le : 2016-03-14\newline
Granularité : à la commune\newline
Mise à jour : ponctuelle\newline
Popularité : 1 réutilisation,  1 suivi\newline
Mots-clé : \emph{election, election-presidentielle, elections, elections-cantonales, elections-europeennes, elections-legislatives-assemblee
}\newline
Permalien : \url{https://data.gouv.fr/dataset/53699f41a3a729239d2060b5}\newline

\par
\noindent
    Résultats électoraux à Issy-les-Moulineaux depuis 2012 par bureau de
vote. Sous les ressources visibles, cliquez sur ``voir les ressources''
pour voir toutes les ressources.


\vspace{0.5cm}
\needspace{12\baselineskip}
\subsection*{Stations et espaces AutoLib de la métropole parisienne
}\index{autolib}\index{bornes}\index{espaces}\index{stations}\index{transport}
  \begin{wrapfigure}{r}{2.5cm}
    \centering
    \qrcode[nolink]{https://data.gouv.fr/dataset/54e3637dc751df7960467389}
  \end{wrapfigure}

Licence : \textbf{Licence Ouverte
}\newline
Créé le : 2015-02-17\newline
Modifié le : 2015-10-16\newline
Granularité : à la commune\newline
Mise à jour : annuelle\newline
Popularité : 1 réutilisation,  0 suivi\newline
Mots-clé : \emph{autolib, bornes, espaces, stations, transport
}\newline
Permalien : \url{https://data.gouv.fr/dataset/54e3637dc751df7960467389}\newline

\par
\noindent
    Ce jeu de données comprend le nom de la station, sa localisation, le
nombre total de places\ldots{}


\vspace{0.5cm}
\needspace{12\baselineskip}
\subsection*{Structure des dépenses d'investissement à Issy-les-Moulineaux
}\index{budget}\index{depenses}\index{investissement}
  \begin{wrapfigure}{r}{2.5cm}
    \centering
    \qrcode[nolink]{https://data.gouv.fr/dataset/55194711c751df492f057c91}
  \end{wrapfigure}

Licence : \textbf{Licence Ouverte
}\newline
Créé le : 2015-03-30\newline
Modifié le : 2015-10-09\newline
Granularité : à la commune\newline
Mise à jour : annuelle\newline
Popularité : 1 réutilisation,  0 suivi\newline
Mots-clé : \emph{budget, depenses, investissement
}\newline
Permalien : \url{https://data.gouv.fr/dataset/55194711c751df492f057c91}\newline

\par
\noindent
    Ce jeu de données permet de comprendre dans quoi la ville
d'Issy-les-Moulineaux investit. Éducation, culture, sports et jeunesse,
social, etc.


\vspace{0.5cm}
\needspace{12\baselineskip}
\subsection*{Structure des recettes de fonctionnement
}\index{fonctionnement}\index{recette}
  \begin{wrapfigure}{r}{2.5cm}
    \centering
    \qrcode[nolink]{https://data.gouv.fr/dataset/5519489ec751df4945057c92}
  \end{wrapfigure}

Licence : \textbf{Licence Ouverte
}\newline
Créé le : 2015-03-30\newline
Modifié le : 2015-07-13\newline
Granularité : à la commune\newline
Mise à jour : annuelle\newline
Popularité : 1 réutilisation,  0 suivi\newline
Mots-clé : \emph{fonctionnement, recette
}\newline
Permalien : \url{https://data.gouv.fr/dataset/5519489ec751df4945057c92}\newline

\par
\noindent
    Les informations de ce jeu de données permettent de répondre à la
question ``d'où vient l'argent?''


\vspace{0.5cm}
\needspace{12\baselineskip}
\subsection*{Structure des recettes d'investissement
}\index{investissement}\index{recette}
  \begin{wrapfigure}{r}{2.5cm}
    \centering
    \qrcode[nolink]{https://data.gouv.fr/dataset/551948fdc751df4ca1057c91}
  \end{wrapfigure}

Licence : \textbf{Licence Ouverte
}\newline
Créé le : 2015-03-30\newline
Modifié le : 2016-01-19\newline
Granularité : à la commune\newline
Mise à jour : annuelle\newline
Popularité : 1 réutilisation,  0 suivi\newline
Mots-clé : \emph{investissement, recette
}\newline
Permalien : \url{https://data.gouv.fr/dataset/551948fdc751df4ca1057c91}\newline

\par
\noindent
    Ce jeu de données contient des informations sur les recettes
d'investissement de la Ville d'Issy-les-Moulineaux.


\vspace{0.5cm}
\needspace{12\baselineskip}
\subsection*{Taux des impôts locaux (Issy et villes de la même strate)
}\index{habitation}\index{impots}\index{taxe}
  \begin{wrapfigure}{r}{2.5cm}
    \centering
    \qrcode[nolink]{https://data.gouv.fr/dataset/550aec50c751df4696882844}
  \end{wrapfigure}

Licence : \textbf{Licence Ouverte
}\newline
Créé le : 2015-03-19\newline
Modifié le : 2016-03-05\newline
Granularité : à la commune\newline
Mise à jour : annuelle\newline
Popularité : 1 réutilisation,  0 suivi\newline
Mots-clé : \emph{habitation, impots, taxe
}\newline
Permalien : \url{https://data.gouv.fr/dataset/550aec50c751df4696882844}\newline

\par
\noindent
    Ce jeu de données contient les taux des impôts locaux à
Issy-les-Moulineaux depuis 1996 (taxe d'habitation et taxe foncière). Il
contient également des informations sur les taux moyens des villes de la
même strate (50000 à 100000 habitants).


\vspace{0.5cm}
\needspace{3\baselineskip} \rule{4cm}{0.25pt}\newline\textbf{Aussi disponible du même producteur :}\begin{itemize}
\item \href{https://data.gouv.fr/dataset/5982a09aa3a729783e0f56ee}{Activités sportives et clubs à Issy-les-Moulineaux}
\item \href{https://data.gouv.fr/dataset/5449061ac751df3edc5d0573}{Actualités}
\item \href{https://data.gouv.fr/dataset/595c26c0a3a7296408d69b71}{Aéroports, aérodrome et héliport}
\item \href{https://data.gouv.fr/dataset/53698fc6a3a729239d203827}{Budget primitif}
\item \href{https://data.gouv.fr/dataset/595c26e7a3a7296407d69b3c}{Budget primitif 2015 de la Ville d'Issy-les-Moulineaux}
\item \href{https://data.gouv.fr/dataset/550ae6fec751df33e3882844}{Budget primitif 2015 de la Ville d'Issy-les-Moulineaux}
\item \href{https://data.gouv.fr/dataset/595c2692a3a7296408d69b60}{Budget primitif 2016 de la Ville d'Issy-les-Moulineaux}
\item \href{https://data.gouv.fr/dataset/54e49ceec751df4662467389}{Commerces ou services en lien avec les animaux à Issy-les-Moulineaux}
\item \href{https://data.gouv.fr/dataset/595c267aa3a7296408d69b57}{Compte Administratif 2013 Issy-les-Moulineaux}
\item \href{https://data.gouv.fr/dataset/595c26e9a3a7296407d69b3d}{Compte administratif 2014 d'Issy-les-Moulineaux}
\item \href{https://data.gouv.fr/dataset/595c2694a3a7296407d69b1e}{Compte administratif 2015 d'Issy-les-Moulineaux}
\item \href{https://data.gouv.fr/dataset/550aeb1fc751df4694882844}{Diplôme du Brevet par établissement à Issy-les-Moulineaux}
\item \href{https://data.gouv.fr/dataset/5aacc506a3a7291e662c89f1}{Ecoles élémentaires - Secteurs scolaires à Issy-les-Moulineaux}
\item \href{https://data.gouv.fr/dataset/5aacc547b595086c4abc84b2}{Ecoles maternelles publiques et privées à Issy-les-Moulineaux}
\item \href{https://data.gouv.fr/dataset/550aeab0c751df432c882846}{Elections départementales 2015 - Candidatures 1er tour}
\item \href{https://data.gouv.fr/dataset/595c2728a3a7296408d69b96}{Evénements impactant les déplacements}
\item \href{https://data.gouv.fr/dataset/54e463dcc751df715d467389}{Flux Facebook des actualités d'Issy-les-Moulineaux}
\item \href{https://data.gouv.fr/dataset/550ae97ec751df433e882844}{Flux rss du Twitter @Issylesmoul}
\item \href{https://data.gouv.fr/dataset/595c26e4a3a7296408d69b7e}{Flux rss Twitter d'Issy-les-Moulineaux}
\item \href{https://data.gouv.fr/dataset/563c899bc751df3e01dcc3f1}{Histoire du Fort et des combats de 1870-1871 à Issy}
\item \href{https://data.gouv.fr/dataset/595c26a7a3a7296407d69b25}{Histoire du Fort et des combats de 1870-1871 à Issy}
\item \href{https://data.gouv.fr/dataset/563c8a52c751df097fdcc3f2}{Jumelages et partenariats internationaux d'Issy-les-Moulineaux}
\item \href{https://data.gouv.fr/dataset/56392e45c751df71f5dcc3f0}{Le patrimoine d'Issy-les-Moulineaux}
\item \href{https://data.gouv.fr/dataset/563c7e1bc751df097fdcc3f1}{Les associations à Issy-les-Moulineaux (hors associations sportives OMS)}
\item \href{https://data.gouv.fr/dataset/54e4b3fdc751df66fb467389}{Les commerces alimentaires à Issy-les-Moulineaux}
\item \href{https://data.gouv.fr/dataset/54e4a4efc751df4b9f46738a}{Les commerces pour la beauté et le bien-être à Issy-les-Moulineaux}
\item \href{https://data.gouv.fr/dataset/54e3646cc751df7967467389}{Les commerces pour l'automobile et les deux roues à Issy-les-Moulineaux}
\item \href{https://data.gouv.fr/dataset/54e4a8e4c751df50c0467389}{Les commerces pour l'équipement de la personne à Issy-les-Moulineaux}
\item \href{https://data.gouv.fr/dataset/54e4b311c751df654946738b}{Les commerces pour les loisirs et le tourisme à Issy-les-Moulineaux}
\item \href{https://data.gouv.fr/dataset/54e49366c751df38a2467389}{Les entreprises d'entretien, de rénovation et d'équipement à Issy-les-Moulineaux}
\item \href{https://data.gouv.fr/dataset/595c269da3a7296408d69b64}{Les entreprises isséennes membres de Cap Digital}
\item \href{https://data.gouv.fr/dataset/550aed97c751df4696882845}{Les entreprises isséennes membres de Cap Digital}
\item \href{https://data.gouv.fr/dataset/595c2714a3a7296408d69b8f}{Les entreprises TIC à Issy-les-Moulineaux}
\item \href{https://data.gouv.fr/dataset/54e365efc751df7b36467389}{Les grandes dates d'Issy-les-Moulineaux}
\item \href{https://data.gouv.fr/dataset/54e47362c751df08de467389}{Les lieux d'aide à l'emploi et aux entreprises à Issy-les-Moulineaux}
\item \href{https://data.gouv.fr/dataset/550ae9dcc751df433e882845}{Les panneaux électroniques d'information à Issy-les-Moulineaux}
\item \href{https://data.gouv.fr/dataset/550aeb7cc751df416c882845}{Les parcs, jardins et squares à Issy-les-Moulineaux}
\item \href{https://data.gouv.fr/dataset/5c6e2d969ce2e7324d5908de}{Les statistiques web de la Ville d'Issy-les-Moulineaux}
\item \href{https://data.gouv.fr/dataset/54e4540bc751df4f15467389}{Lieux de cultes à Issy-les-Moulineaux}
\item \href{https://data.gouv.fr/dataset/54e34365c751df47ed46738b}{Lieux pour les seniors à Issy-les-Moulineaux}
\item \href{https://data.gouv.fr/dataset/54e46f29c751df7bad467389}{Liste des administrations à Issy-les-Moulineaux}
\item \href{https://data.gouv.fr/dataset/550ae7a6c751df416c882844}{Liste des candélabres à Issy-les-Moulineaux}
\item \href{https://data.gouv.fr/dataset/54e4945dc751df38a246738a}{Liste des compagnies d'assurance et des banques à Issy-les-Moulineaux}
\item \href{https://data.gouv.fr/dataset/54e348ebc751df47ce46738a}{Liste des corbeilles de rue à Issy-les-Moulineaux}
\item \href{https://data.gouv.fr/dataset/54e49959c751df3f91467389}{Liste des cordonniers, serruriers, laveries et pressings à Issy-les-Moulineaux}
\item \href{https://data.gouv.fr/dataset/54e45063c751df544f467389}{Liste des équipements culturels et de loisirs à Issy-les-Moulineaux}
\item \href{https://data.gouv.fr/dataset/54e4557dc751df5cc3467389}{Liste des équipements sportifs à Issy-les-Moulineaux}
\item \href{https://data.gouv.fr/dataset/54e35fa9c751df70d3467389}{Liste des établissements scolaires à Issy-les-Moulineaux}
\item \href{https://data.gouv.fr/dataset/54e4a276c751df4b9f467389}{Liste des fleuristes et jardineries à Issy-les-Moulineaux}
\item \href{https://data.gouv.fr/dataset/54e34596c751df4cfd467389}{Liste des hébergements seniors à Issy-les-Moulineaux}
\item et 32 autres jeux de données\end{itemize}


    \clearpage

    \chapter{Autres jeux de données}

    Ce chapitre liste une sélection des autres jeux de données disponibles
    sur \emph{data.gouv.fr}.
    \par
    Sont sélectionnés les jeux de données avec au moins une réutilisation et publiés sous une licence libre.

    \minitoc

    
\clearpage
\section{ABSOLOM DESIGN}


\begin{center}
  \includegraphics[width=3cm]{images/orga/b5_1371853e684ffeb73527701a002fd7-100.png}
\end{center}


Web agency spécialisée dans les solutions internet et éditrice de
logiciels d'encaissement et de gestion de point de vente connectés.


\vspace{0.5cm}

\needspace{12\baselineskip}
\subsection*{Menus des écoles maternelles et élémentaires - Ville de Toulouse --
Novembre
}\index{cantine}\index{elementaires}\index{maternelles}\index{menus}\index{toulouse}
  \begin{wrapfigure}{r}{2.5cm}
    \centering
    \qrcode[nolink]{https://data.gouv.fr/dataset/5640607c88ee3844abe72049}
  \end{wrapfigure}

Licence : \textbf{Open Data Commons Open Database License (ODbL)
}\newline
Créé le : 2015-11-09\newline
Modifié le : 2016-02-07\newline
De 2015-11-01 à 2015-11-30\newline
Granularité : à la commune\newline
Mise à jour : mensuelle\newline
Popularité : 1 réutilisation,  0 suivi\newline
Mots-clé : \emph{cantine, elementaires, maternelles, menus, toulouse
}\newline
Permalien : \url{https://data.gouv.fr/dataset/5640607c88ee3844abe72049}\newline

\par
\noindent
    Menus des écoles maternelles et élémentaires - Ville de Toulouse --
Novembre


\vspace{0.5cm}

\clearpage
\section{ACTION Nogent-sur-Marne}


\begin{center}
  \includegraphics[width=3cm]{images/orga/cf_0534a48bee4c7497f88fdf8a83a191-100.png}
\end{center}


Association Citoyenne pour la Transparence et l'Initiative Populaire à
Nogent-sur-Marne (ACTION)

L'\textbf{\href{http://www.action-nogent.fr}{ACTION}} est une
\textbf{association citoyenne non partisane} militant pour la
transparence et l'implication citoyenne dans les décisions de la
\textbf{commune de
\href{http://www.ville-nogentsurmarne.fr}{Nogent-sur-Marne}}.

Nos objectifs principaux sont de promouvoir la \textbf{transparence et
la participation citoyenne} dans la gestion et les décisions de la
commune. Voici quelques grandes lignes des actions que nous souhaitons
mener :

\begin{itemize}

\item
  améliorer la transparence dans la gestion de la ville en mettant à
  disposition les documents de base (par exemples : les
  \href{http://action-nogent.fr/conseils-municipaux-2/}{délibérations du
  conseil municipal}, les
  \href{http://action-nogent.fr/communaute-dagglomeration/rapports-dactivite/}{rapports
  d'activité de la communauté d'agglomération}, etc.) ;
\item
  informer de manière pédagogique les citoyens sur les décisions prises
  par les élus, par exemple en expliquant certains points du budget
  municipal ou en mettant en ligne des services open data ;
\item
  sensibiliser et impliquer les citoyens dans la gestion de leur
  commune, en promouvant notamment la pratique de la
  {[}\href{http://fr.wikipedia.org/wiki/D\%C3\%A9mocratie_d\%C3\%A9lib\%C3\%A9rativeive}{démocratie
  délibérative}{]} ;
\item
  promouvoir l'utilisation de
  \href{http://fr.wikipedia.org/wiki/Logiciel_libre}{logiciels libres}.
\end{itemize}

\textbf{Site internet :
\url{http://www.action-nogent.fr}}{]}(http://www.action-nogent.fr{]}(http://www.action-nogent.fr)**)
Twitter : @ACTION\_Nogent - (https://twitter.com/ACTION\_Nogent)
Facebook : \href{https://www.facebook.com/ACTION94130}{ACTION94130}
Google+ : \href{https://plus.google.com/+ActionnogentFr94130}{ACTION
Nogent}


\vspace{0.5cm}

\needspace{12\baselineskip}
\subsection*{Législatives 2012 - Dépenses des candidats
}\index{2012}\index{cnccfp}\index{comptes!de!campagne}\index{elections}\index{elections!legislatives}\index{gilles!carrez}\index{legislatives}\index{val!de!marne}
  \begin{wrapfigure}{r}{2.5cm}
    \centering
    \qrcode[nolink]{https://data.gouv.fr/dataset/5420b973a3a7297b491991e2}
  \end{wrapfigure}

Licence : \textbf{Licence Ouverte
}\newline
Créé le : 2014-09-23\newline
Modifié le : 2015-09-06\newline
De 2012-06-10 à 2012-06-17\newline
Popularité : 1 réutilisation,  0 suivi\newline
Mots-clé : \emph{2012, cnccfp, comptes-de-campagne, elections, elections-legislatives, gilles-carrez, legislatives, val-de-marne
}\newline
Permalien : \url{https://data.gouv.fr/dataset/5420b973a3a7297b491991e2}\newline

\par
\noindent
    L'\href{http://www.action-nogent.fr}{ACTION} publie ici les dépenses des
candidats aux élections législatives de juin 2012 dans la
\href{http://fr.wikipedia.org/wiki/Cinqui\%C3\%A8me_circonscription_du_Val-de-Marne}{5ème
circonscription du Val-de-Marne} qui a vu
\href{http://www.nosdeputes.fr/gilles-carrez}{Gilles Carrez} être réélu
à l'issu du second tour.

Ces données sont l'extrait local du jeu publié par la
\href{http://www.cnccfp.fr/}{CNCCFP}.


\vspace{0.5cm}
\needspace{12\baselineskip}
\subsection*{Localisation des caméras de vidéosurveillance - Nogent-sur-Marne
}\index{nogent!sur!marne}\index{securite}\index{videoprotection}\index{videosurveillance}
  \begin{wrapfigure}{r}{2.5cm}
    \centering
    \qrcode[nolink]{https://data.gouv.fr/dataset/54708c38c751df2396ccf75e}
  \end{wrapfigure}

Licence : \textbf{Licence Ouverte
}\newline
Créé le : 2014-11-22\newline
Modifié le : 2015-12-08\newline
Granularité : à la commune\newline
Mise à jour : ponctuelle\newline
Popularité : 3 réutilisations,  0 suivi\newline
Mots-clé : \emph{nogent-sur-marne, securite, videoprotection, videosurveillance
}\newline
Permalien : \url{https://data.gouv.fr/dataset/54708c38c751df2396ccf75e}\newline

\par
\noindent
    L'\href{http://www.action-nogent.fr}{ACTION} publie ici la localisation
des 90 caméras de vidéosurveillance installées sur le territoire de la
commune de \href{http://www.ville-nogentsurmarne.fr}{Nogent-sur-Marne}.

Cette liste est basée sur l'arrêté préfectoral n\degree{}2013/1903
délivré par la Préfecture du Val de Marne en date du 24 juin 2013
portant autorisation du système de vidéoprotection sur la voir publique
et différents sites.

Les finalités du dispositif telles que présentes dans les données sont
les suivantes :

\begin{itemize}

\item
  Finalité 1 - Prévention des atteintes à la sécurité des personnes et
  des biens dans des lieux particulièrement exposés à des risques
  d'agression ou de vol
\item
  Finalité 2 - Protection de bâtiments et installations publics et de
  leurs abords
\item
  Finalité 3 - Régulation du trafic routier et constatation des
  infractions aux règles de la circulation (y compris deux-roues sur
  secteurs non autorisés)
\end{itemize}


\vspace{0.5cm}
\needspace{12\baselineskip}
\subsection*{Logement social : RPLS - Nogent-sur-Marne
}\index{bailleur}\index{hlm}\index{logement}\index{logement!social}\index{logements!sociaux}\index{nogent!sur!marne}\index{rpls}
  \begin{wrapfigure}{r}{2.5cm}
    \centering
    \qrcode[nolink]{https://data.gouv.fr/dataset/54f1936dc751df6f81882844}
  \end{wrapfigure}

Licence : \textbf{Licence Ouverte
}\newline
Créé le : 2015-02-28\newline
Modifié le : 2016-02-18\newline
De 2014-01-01 à 2014-01-01\newline
Granularité : à la commune\newline
Mise à jour : annuelle\newline
Popularité : 1 réutilisation,  1 suivi\newline
Mots-clé : \emph{bailleur, hlm, logement, logement-social, logements-sociaux, nogent-sur-marne, rpls
}\newline
Permalien : \url{https://data.gouv.fr/dataset/54f1936dc751df6f81882844}\newline

\par
\noindent
    L'\href{http://www.action-nogent.fr}{ACTION} publie ici l'extrait du
Répertoire des logements locatifs des bailleurs sociaux
(\href{http://www.statistiques.developpement-durable.gouv.fr/sources-methodes/enquete-nomenclature/1542/0/repertoire-logements-locatifs-bailleurs-sociaux-rpls.html}{RPLS})
pour la commune de
\href{http://www.ville-nogentsurmarne.fr}{Nogent-sur-Marne} aux 1er
janvier 2012-2013-2014.

Le répertoire des logements locatifs des bailleurs sociaux a pour
objectif de dresser l'état global du parc de logements locatifs de ces
bailleurs sociaux au 1er janvier d'une année (nombre de logements,
modifications intervenues au cours de l'année écoulée, localisation,
taux d'occupation, mobilité, niveau des loyers, financement et
conventionnement). Mis en place au 1er janvier 2011, il est alimenté par
les informations transmises par les bailleurs sociaux.


\vspace{0.5cm}
\needspace{12\baselineskip}
\subsection*{Municipales 2014 - Dépenses des candidats
}\index{elections!municipales}\index{elections!municipales!2014}\index{nogent!sur!marne}
  \begin{wrapfigure}{r}{2.5cm}
    \centering
    \qrcode[nolink]{https://data.gouv.fr/dataset/54b96cb4c751df38765fa5a2}
  \end{wrapfigure}

Licence : \textbf{Licence Ouverte
}\newline
Créé le : 2015-01-16\newline
Modifié le : 2016-02-06\newline
De 2014-03-23 à 2014-03-30\newline
Granularité : à la commune\newline
Mise à jour : ponctuelle\newline
Popularité : 1 réutilisation,  1 suivi\newline
Mots-clé : \emph{elections-municipales, elections-municipales-2014, nogent-sur-marne
}\newline
Permalien : \url{https://data.gouv.fr/dataset/54b96cb4c751df38765fa5a2}\newline

\par
\noindent
    L'\href{http://www.action-nogent.fr}{ACTION} publie ici les dépenses des
candidats aux élections municipales de mars 2014 sur la commune de
\href{https://www.ville-nogentsurmarne.fr}{Nogent-sur-Marne} qui a vu
\href{http://www.jacques-jp-martin.fr/}{Jacques Martin} être réélu à
l'issu du second tour.

Ces données ont été produites pas notre association sur la base de
documents fournis par la \href{http://www.cnccfp.fr/}{CNCCFP} et
certains candidats eux-mêmes.


\vspace{0.5cm}
\needspace{12\baselineskip}
\subsection*{Panneaux d'affichage libre à Nogent-sur-Marne
}\index{affichage!libre}\index{nogent!sur!marne}\index{panneaux}
  \begin{wrapfigure}{r}{2.5cm}
    \centering
    \qrcode[nolink]{https://data.gouv.fr/dataset/53a3772ba3a7297d730b2f57}
  \end{wrapfigure}

Licence : \textbf{Licence Ouverte
}\newline
Créé le : 2014-06-16\newline
Modifié le : 2015-11-25\newline
Granularité : à la commune\newline
Mise à jour : annuelle\newline
Popularité : 1 réutilisation,  0 suivi\newline
Mots-clé : \emph{affichage-libre, nogent-sur-marne, panneaux
}\newline
Permalien : \url{https://data.gouv.fr/dataset/53a3772ba3a7297d730b2f57}\newline

\par
\noindent
    L'\href{http://www.action-nogent.fr}{ACTION} publie ici un jeu de
données indiquant la liste des panneaux
d'\href{http://fr.wikipedia.org/wiki/Affichage_libre}{affichage libre}
mis à disposition par la
\href{http://www.ville-nogentsurmarne.fr}{commune de Nogent-sur-Marne}
sur son territoire.


\vspace{0.5cm}
\needspace{12\baselineskip}
\subsection*{Réserve parlementaire - Gilles Carrez - 2014
}\index{deputes}\index{gilles!carrez}\index{reserve!parlementaire}
  \begin{wrapfigure}{r}{2.5cm}
    \centering
    \qrcode[nolink]{https://data.gouv.fr/dataset/54df66d7c751df149e467389}
  \end{wrapfigure}

Licence : \textbf{Licence Ouverte
}\newline
Créé le : 2015-02-14\newline
Modifié le : 2016-02-15\newline
De 2014-01-01 à 2014-12-31\newline
Mise à jour : ponctuelle\newline
Popularité : 1 réutilisation,  0 suivi\newline
Mots-clé : \emph{deputes, gilles-carrez, reserve-parlementaire
}\newline
Permalien : \url{https://data.gouv.fr/dataset/54df66d7c751df149e467389}\newline

\par
\noindent
    L'\href{http://www.action-nogent.fr}{ACTION} (Association Citoyenne pour
la Transparence et l'Initiative Populaire à
\href{http://www.ville-nogentsurmarne.fr}{Nogent-sur-Marne}) publie ici
la distribution des 555 000 \euro{} de la réserve parlementaire de
\href{http://www.gillescarrez.fr/}{Gilles Carrez}, député de la
\href{http://www2.assemblee-nationale.fr/deputes/fiche/OMC_PA746}{5ème
circonscription du Val de Marne}.


\vspace{0.5cm}
\needspace{12\baselineskip}
\subsection*{Statistiques de la délinquance à Nogent-sur-Marne
}\index{criminalite}\index{delinquance}\index{nogent!sur!marne}\index{police}\index{statistiques}
  \begin{wrapfigure}{r}{2.5cm}
    \centering
    \qrcode[nolink]{https://data.gouv.fr/dataset/555f3165c751df28f3c98e11}
  \end{wrapfigure}

Licence : \textbf{Licence Ouverte
}\newline
Créé le : 2015-05-22\newline
Modifié le : 2016-03-13\newline
De 2008-09-01 à 2015-03-31\newline
Granularité : à la commune\newline
Mise à jour : mensuelle\newline
Popularité : 1 réutilisation,  1 suivi\newline
Mots-clé : \emph{criminalite, delinquance, nogent-sur-marne, police, statistiques
}\newline
Permalien : \url{https://data.gouv.fr/dataset/555f3165c751df28f3c98e11}\newline

\par
\noindent
    L'\href{http://www.action-nogent.fr}{ACTION} publie ici les statistiques
de la délinquance fournies de manière mensuelle par la
\href{http://www.val-de-marne.gouv.fr/}{Préfecture du Val de Marne} à la
municipalité de
\href{http://www.ville-nogentsurmarne.fr}{Nogent-sur-Marne}.

Ce jeu de données regroupe l'ensemble des typologies de faits recensés
mois par mois sur le territoire de la commune selon
l'\href{http://action-nogent.fr/wp-content/uploads/2015/05/indicateurs_2011.pdf}{État
4001}.


\vspace{0.5cm}
\needspace{3\baselineskip} \rule{4cm}{0.25pt}\newline\textbf{Aussi disponible du même producteur :}\begin{itemize}
\item \href{https://data.gouv.fr/dataset/53dfca07a3a729110ca8d362}{Bureaux de vote de la ville de Nogent-sur-Marne}
\item \href{https://data.gouv.fr/dataset/55169a53c751df565db33e55}{Chiffres clés évolution et structure de la population de Nogent-sur-Marne}
\item \href{https://data.gouv.fr/dataset/54e0b9e1c751df0a5b467389}{Construction de logements par commune dans le Val de Marne (2000-2013)}
\item \href{https://data.gouv.fr/dataset/53ed0cdda3a7296b65ed4d25}{Décisions prises par le maire de Nogent-sur-Marne (2014)}
\item \href{https://data.gouv.fr/dataset/53f0a4a8a3a72905a3504b19}{Délibérations du Conseil Municipal de la ville de Nogent-sur-Marne (2010)}
\item \href{https://data.gouv.fr/dataset/53ea9e63a3a7297931072b3c}{Délibérations du Conseil Municipal de la ville de Nogent-sur-Marne (2011)}
\item \href{https://data.gouv.fr/dataset/53e220e5a3a729615b14b20a}{Délibérations du Conseil Municipal de la ville de Nogent-sur-Marne (2012)}
\item \href{https://data.gouv.fr/dataset/53aff0a5a3a729757eddb439}{Délibérations du Conseil Municipal de la ville de Nogent-sur-Marne (2013)}
\item \href{https://data.gouv.fr/dataset/53dfccd9a3a729110ca8d365}{Délibérations du Conseil Municipal de la ville de Nogent-sur-Marne (2014)}
\item \href{https://data.gouv.fr/dataset/53f31ab3a3a72908bd9dfc0c}{Données comptables et fiscales - Nogent-sur-Marne (2000-2012)}
\item \href{https://data.gouv.fr/dataset/55113fe4c751df39b9882844}{Elections départementales 2015 - Nogent-sur-Marne}
\item \href{https://data.gouv.fr/dataset/5383c718a3a729162bf4f031}{Elections Européennes de mai 2014 - Résultats à Nogent-sur-Marne par bureau}
\item \href{https://data.gouv.fr/dataset/53dfce56a3a729110ca8d366}{Elections Législatives 2012 - Résultats par bureau de vote sur la commune de Nogent-sur-Marne}
\item \href{https://data.gouv.fr/dataset/53dfce63a3a729110ca8d367}{Elections Municipales 2014 - Résultats sur la commune de Nogent-sur-Marne }
\item \href{https://data.gouv.fr/dataset/53d60794a3a72919b65fbbb5}{Elections Présidentielles 2012 - Résultats par bureau de vote sur la commune de Nogent-sur-Marne}
\item \href{https://data.gouv.fr/dataset/53f0a959a3a72905a3504b1d}{Indemnités des élus de la Communauté d'Agglomération de la Vallée de la Marne (CAVM) (2014)}
\item \href{https://data.gouv.fr/dataset/53f0a959a3a72905a3504b1e}{Indemnités des élus municipaux de la ville de Nogent-sur-Marne}
\item \href{https://data.gouv.fr/dataset/5407ece2a3a7292ef9c20a5d}{Indemnités des membres du bureau du SIPPEREC (2014)}
\item \href{https://data.gouv.fr/dataset/53f0ac82a3a72905a3504b21}{Liste des Immeubles sur la commune de Nogent-sur-Marne protégés au titre des Monuments Historiques}
\item \href{https://data.gouv.fr/dataset/54f82e53c751df2b93882845}{Logement social : RPLS - Bry-sur-Marne}
\item \href{https://data.gouv.fr/dataset/54f82e58c751df2ef8882844}{Logement social : RPLS - Le Perreux-sur-Marne}
\item \href{https://data.gouv.fr/dataset/53ba55c4a3a729219b7beae2}{Plan Local d'Urbanisme (PLU) - Nogent-sur-Marne}
\item \href{https://data.gouv.fr/dataset/5416d19ba3a72937ecd41fa3}{Programme du maire de Nogent-sur-Marne (mandature 2014-2020)}
\item \href{https://data.gouv.fr/dataset/53a37db4a3a7297d730b2f5d}{Subventions municipales de la ville de Nogent-sur-Marne aux associations en 2012}
\item \href{https://data.gouv.fr/dataset/539f32a0a3a729478718a7f6}{Subventions municipales de la ville de Nogent-sur-Marne aux associations en 2013}
\end{itemize}

\clearpage
\section{Association Breizh Small Business Act}


\begin{center}
  \includegraphics[width=3cm]{images/orga/2015-03-17_93f954373a2b4b62b814bda82a2cd3fc_BreizhSBA-100.png}
\end{center}


\textbf{Breizh Small Business Act }est la première association créée en
Bretagne par des professionnels publics et privés autour des enjeux de
l'achat public.

Au sein de cette association (Loi 1901), il s'agit de favoriser en
Bretagne la rencontre des professionnels du secteur privé (TPE et PME)
et du secteur public (collectivités locales, administrations) qui
jusqu'à maintenant n'avaient pas de lieu neutre pour échanger et agir
ensemble dans le domaine de la commande publique.

L'association est également à l'origine de la création du prototype de
portail régional d'informations et de données de marchés publics dénommé
\textbf{My Breizh Open Data - Marchés Publics}.


\vspace{0.5cm}

\needspace{12\baselineskip}
\subsection*{Liste des marchés publics conclus par le Conseil régional de Bretagne
pour les années 2012, 2013 et 2014
}\index{dataconnexions!6}\index{impact!eco}
  \begin{wrapfigure}{r}{2.5cm}
    \centering
    \qrcode[nolink]{https://data.gouv.fr/dataset/56571e5588ee384aa0e72046}
  \end{wrapfigure}

Licence : \textbf{Creative Commons CCZero
}\newline
Créé le : 2015-11-26\newline
Modifié le : 2016-03-09\newline
De 2012-01-01 à 2014-12-31\newline
Granularité : à la région\newline
Mise à jour : annuelle\newline
Popularité : 1 réutilisation,  1 suivi\newline
Mots-clé : \emph{dataconnexions-6, impact-eco
}\newline
Permalien : \url{https://data.gouv.fr/dataset/56571e5588ee384aa0e72046}\newline

\par
\noindent
    Le jeu de données utilisé par My Breizh Open Data -- Marchés Publics a
été constitué à partir des données recensant tous les marchés publics
conclus par le Conseil régional de Bretagne pour les années 2012, 2013
et 2014. Ce jeu de données publiques réutilisables à vocation à
s'étoffer de l'ensemble des données de la commande publique des donneurs
d'ordres publics bretons. Les informations publiées sont : Numéro du
marché ; Donneur d'ordre ; Objet du marché ;Date ; Montant mandaté TTC ;
Entreprise ; Rôle de l'entreprise ; Code Postal de l'entreprise ;
Commune de l'entreprise ; Taille de l'entreprise ; Chiffre d'affaires ;
Secteur d'activité ; Type de marché ;Procédure du marché ; N\degree{} de
Siret

Ces données sont associées au format pivot déposé dans
l'Adullact\url{https://formats-pivots.adullact.net/}


\vspace{0.5cm}
\needspace{3\baselineskip} \rule{4cm}{0.25pt}\newline\textbf{Aussi disponible du même producteur :}\begin{itemize}
\item \href{https://data.gouv.fr/dataset/5507df0dc751df27a9882844}{Prototype du portail régional d'informations et de données de marchés publics}
\end{itemize}

\clearpage
\section{Astrolabe Expéditions}


\begin{center}
  \includegraphics[width=3cm]{images/orga/a1_ab9ed3f2454484bc469e52969a2f52-100.png}
\end{center}


Astrolabe Expéditions a pour objectif de rendre accessible les
expéditions scientifiques océaniques.

Ces missions de recherche citoyenne, organisées à bord de voiliers
visent à améliorer les connaissances et la préservation de
l'environnement marin. En collaboration avec des institutions
scientifiques et des espaces de recherche collaboratifs, tous les
citoyens peuvent s'impliquer dans le développement de programmes de
recherche et participer aux expéditions en mer.


\vspace{0.5cm}

\needspace{12\baselineskip}
\subsection*{Expédition Julo
}\index{oceanographie}\index{plancton}
  \begin{wrapfigure}{r}{2.5cm}
    \centering
    \qrcode[nolink]{https://data.gouv.fr/dataset/5895b64988ee3806289b81a4}
  \end{wrapfigure}

Licence : \textbf{Licence Ouverte
}\newline
Créé le : 2017-02-04\newline
Modifié le : 2017-02-04\newline
Mise à jour : ponctuelle\newline
Popularité : 1 réutilisation,  0 suivi\newline
Mots-clé : \emph{oceanographie, plancton
}\newline
Permalien : \url{https://data.gouv.fr/dataset/5895b64988ee3806289b81a4}\newline

\par
\noindent
    Données prélevées lors de l'expédition du bateau «Julo» en mai 2015.


\vspace{0.5cm}
\needspace{3\baselineskip} \rule{4cm}{0.25pt}\newline\textbf{Aussi disponible du même producteur :}\begin{itemize}
\item \href{https://data.gouv.fr/dataset/5895c26188ee380b259b81a4}{Astrolabe Expéditions - Islande 2016}
\item \href{https://data.gouv.fr/dataset/58970a6988ee380f079b81a4}{Cotentin 2013}
\end{itemize}

\clearpage
\section{Communauté d'agglomération Cannes Lérins}


\begin{center}
  \includegraphics[width=3cm]{images/orga/c2_8272a362ca451b81decfc0b93e31b8-100.png}
\end{center}


La communauté d'agglomération Cannes Lérins, située dans le département
des Alpes-Maritimes en région Provence-Alpes-Côte d'Azur, regroupe cinq
communes de l'extrême sud-ouest du département : Cannes, Le Cannet,
Mandelieu-La Napoule, Mougins et Théoule/Mer.


\vspace{0.5cm}

\needspace{12\baselineskip}
\subsection*{Localisations géographiques et caractéristiques techniques des stations
et des points de recharge pour véhicules électriques du réseau Wiiiz
implantés sur le territoire de la Communauté d'Agglomération Cannes
Lérins (France départment 06)
}\index{borne!de!recharge}\index{irve}\index{mobilite!electrique}\index{points!de!recharge}\index{station!de!recharge}\index{transition!energetique}\index{transports}\index{vehicule!electrique}\index{voiture!electrique}
  \begin{wrapfigure}{r}{2.5cm}
    \centering
    \qrcode[nolink]{https://data.gouv.fr/dataset/5b3f674ac751df051a5195c1}
  \end{wrapfigure}

Licence : \textbf{Licence Ouverte
}\newline
Créé le : 2018-07-06\newline
Modifié le : 2019-01-29\newline
Mise à jour : trimestrielle\newline
Popularité : 1 réutilisation,  2 suivis\newline
Mots-clé : \emph{borne-de-recharge, irve, mobilite-electrique, points-de-recharge, station-de-recharge, transition-energetique, transports, vehicule-electrique, voiture-electrique
}\newline
Permalien : \url{https://data.gouv.fr/dataset/5b3f674ac751df051a5195c1}\newline

\par
\noindent
    Ces données permettent de localiser et de connaître les caractéristiques
techniques des stations et des points de recharge pour véhicules
électrique ouverts au public, appartenant à la Communauté
d'Agglomération Cannes Lérins (CACPL)et implantés sur son territoire.

Ces stations et points de recharges font partie du réseau Wiiiz. Ce
réseau d'infrastructures de recharge pour véhicules électriques (IRVE)
se déploie non seulement sur le territoire de la CACPL mais également
sur celui de la Communauté d'Agglomération Sophia Antipolis et sur celui
de la Communauté d'Agglomération Pays de Grasse (plus d'information sur
le site internet Wiiiz.fr)

Cette ouverture de données est effectuée dans le cadre de la mise en
œuvre du décret n\degree{} 2017-26 du 12 janvier 2017 relatif aux
infrastructures de recharge pour véhicules électriques (IRVE) et portant
diverses mesures de transposition de la directive 2014/94/UE du
Parlement européen et du Conseil du 22 octobre 2014 sur le déploiement
d'une infrastructure pour carburants alternatifs.

Le jeu de données IRVE\_WiiizCASA\_AAAMMJJ comprend le fichier
IRVE\_WiiizCASA\_AAAMMJJ.csv réalisé dans le respect du standard
technique précisé dans l'annexe de l'arrêté du 12 janvier 2017 relatif
aux données concernant la localisation géographique et les
caractéristiques, techniques des stations et des points de recharge pour
véhicules électriques.


\vspace{0.5cm}

\clearpage
\section{Croix-Rouge française}


\begin{center}
  \includegraphics[width=3cm]{images/orga/46_8c80e04b29434a8a04a48515cd74c1-100.jpg}
\end{center}


La Croix-Rouge est un Mouvement humanitaire international présent dans
187 pays. Ses membres partagent les mêmes emblèmes et s'appuient sur
sept principes fondateurs communs garantissant la cohérence de leurs
actions.

La Croix-Rouge française, c'est à la fois une association de plus 54 000
bénévoles engagés depuis près de 150 ans sur de nombreux fronts de la
lutte contre les précarités et une entreprise à but non lucratif de
services dans les champs sanitaire, social, médico-social et de la
formation avec 18 000 salariés dans près de 600 établissements.


\vspace{0.5cm}

\needspace{12\baselineskip}
\subsection*{Liste des unités et établissements de la Croix-Rouge française (csv)
}\index{croix!rouge}\index{necmergitur}
  \begin{wrapfigure}{r}{2.5cm}
    \centering
    \qrcode[nolink]{https://data.gouv.fr/dataset/53699958a3a729239d205218}
  \end{wrapfigure}

Licence : \textbf{Licence Ouverte
}\newline
Créé le : 2013-12-10\newline
Modifié le : 2016-02-21\newline
De 2014-03-10 à 2014-04-10\newline
Granularité : au point d'intérêt\newline
Popularité : 1 réutilisation,  3 suivis\newline
Mots-clé : \emph{croix-rouge, necmergitur
}\newline
Permalien : \url{https://data.gouv.fr/dataset/53699958a3a729239d205218}\newline

\par
\noindent
    Au travers d'un réseau très dense de délégations et d'établissements, la
Croix-Rouge française est présente sur l'ensemble du territoire français
y compris dans les départements et territoires d'outre-mer (DOM-TOM). Ce
maillage lui permet d'intervenir au plus vite auprès des personnes en
difficulté quel que soit le point du territoire ou le type
d'intervention requis. Le réseau Croix-Rouge française en chiffres,
c'est plus de 1000 unités et implantations locales, 107 délégations
départementales et territoriales, 17 délégations régionales et 580
établissements ou antennes d'établissement (secteurs sanitaire,
médico-social, social et formation). Les unités et établissements de la
Croix-Rouge française sont regroupés ici par type d'actions. Le fichier
est au format csv.


\vspace{0.5cm}
\needspace{12\baselineskip}
\subsection*{Liste des unités et établissements de la Croix-Rouge française (xml)
}\index{action!internationale}\index{action!sociale}\index{aide!et!soin!a!domicile}\index{benevolat}\index{don}\index{formation}\index{humanitaire}\index{sante!autonomie}\index{urgence!secourisme}
  \begin{wrapfigure}{r}{2.5cm}
    \centering
    \qrcode[nolink]{https://data.gouv.fr/dataset/53699959a3a729239d205219}
  \end{wrapfigure}

Licence : \textbf{Licence Ouverte
}\newline
Créé le : 2013-12-10\newline
Modifié le : 2016-01-20\newline
De 2014-03-10 à 2014-04-10\newline
Granularité : au point d'intérêt\newline
Popularité : 2 réutilisations,  1 suivi\newline
Mots-clé : \emph{action-internationale, action-sociale, aide-et-soin-a-domicile, benevolat, don, formation, humanitaire, sante-autonomie, urgence-secourisme
}\newline
Permalien : \url{https://data.gouv.fr/dataset/53699959a3a729239d205219}\newline

\par
\noindent
    Au travers d'un réseau très dense de délégations et d'établissements, la
Croix-Rouge française est présente sur l'ensemble du territoire français
y compris dans les départements et territoires d'outre-mer (DOM-TOM). Ce
maillage lui permet d'intervenir au plus vite auprès des personnes en
difficulté quel que soit le point du territoire ou le type
d'intervention requis. Le réseau Croix-Rouge française en chiffres,
c'est plus de 1000 unités et implantations locales, 107 délégations
départementales et territoriales, 17 délégations régionales et 580
établissements ou antennes d'établissement (secteurs sanitaire,
médico-social, social et formation). Les unités et établissements de la
Croix-Rouge française sont regroupés ici par type d'actions. Le fichier
est au format xml.


\vspace{0.5cm}
\needspace{3\baselineskip} \rule{4cm}{0.25pt}\newline\textbf{Aussi disponible du même producteur :}\begin{itemize}
\item \href{https://data.gouv.fr/dataset/57cebf6188ee380b7db627a9}{Liste des actions des délégations de la Croix-Rouge française}
\item \href{https://data.gouv.fr/dataset/57cec0f788ee380ddbb627a9}{Liste des bénévoles actifs de la Croix-Rouge française}
\item \href{https://data.gouv.fr/dataset/57cebed888ee380992b627a9}{Liste des structures de la Croix-Rouge française}
\end{itemize}

\clearpage
\section{Cryptolia.fr}


\begin{center}
  \includegraphics[width=3cm]{images/orga/0f_3eca2af0704abfa5b187151d637e47-100.jpg}
\end{center}


Organisation dédiée à la publication de jeux de données relatives aux
cryptomonnaies, coins (pièces) et tokens (ICO).

Nous diffusons en temps réel les cours et le prix de plus de
\href{https://www.cryptolia.fr/crypto-monnaies/}{1700 cryptomonnaies}
(10-20 secondes de chargement).

Ces données peuvent servir de support à des organismes Gouvernementaux
(ou non) pour l'étude des données financières, la création de supports
visuels, la publication d'article de blog ou de presse, etc \ldots{}

N'hésitez pas à nous contacter pour obtenir les cours de cryptomonnaies
spécifiques, nous ne partageons ici que les principales cryptomonnaies,
mais nous avons les prix et les cours en direct de plus de 1300 pièces
et tokens numériques !

Contact : olivier@cryptolia.fr /
\href{https://www.cryptolia.fr}{cryptolia.fr}


\vspace{0.5cm}

\needspace{12\baselineskip}
\subsection*{Le cours du Bitcoin en Euros
}\index{bitcoin}\index{blockchain}\index{coin}\index{crypto}\index{cryptomonnaie}\index{ico}\index{token}
  \begin{wrapfigure}{r}{2.5cm}
    \centering
    \qrcode[nolink]{https://data.gouv.fr/dataset/5a89929888ee383cf671e8f2}
  \end{wrapfigure}

Licence : \textbf{Licence Ouverte
}\newline
Créé le : 2018-02-18\newline
Modifié le : 2018-06-15\newline
De 2013-01-28 à 2018-02-18\newline
Mise à jour : ponctuelle\newline
Popularité : 1 réutilisation,  0 suivi\newline
Mots-clé : \emph{bitcoin, blockchain, coin, crypto, cryptomonnaie, ico, token
}\newline
Permalien : \url{https://data.gouv.fr/dataset/5a89929888ee383cf671e8f2}\newline

\par
\noindent
    Retrouvez le cours historique du Bitcoin depuis le 28 Janvier 2013.
Détails du jeu de données

Fréquence des données : Journalière

Prix en Euros (BTC/EUR).

Le fichier sera mis à jour sur data.gouv.fr aussi fréquemment que
possible.Visualition du cours historique du BTC

\begin{itemize}
\item
  \href{https://www.data.gouv.fr/fr/reuses/cours-du-bitcoin/}{Sur
  Data.gouv.fr}
\item
  \href{https://www.cryptolia.fr/crypto-monnaies/bitcoin/}{Sur
  Cryptolia.fr} Autres crypto-monnaies à découvrir
\item
  \href{https://www.data.gouv.fr/fr/reuses/cours-eth-eur/}{Cours de
  l'Ethereum}
\item
  \href{https://www.data.gouv.fr/fr/reuses/prix-du-ripple-xrp/}{Cours du
  Ripple} Avertissements Ces données ont un but \textbf{purement
  informatif} : Vous fournir des supports utiles pour vos articles,
  blog, data visualisation, \textbf{il ne s'agit en aucun cas d'une
  incitation à la spéculation}.
\end{itemize}

\textbf{L'Etat Français n'a pas encore statué ni exprimé son avis sur
l'utilisation du Bitcoin}, l'investissement dans les cryptomonnaies et
le Bitcoin est donc risqué, prenez conseil auprès de votre conseiller
financier.


\vspace{0.5cm}

\clearpage
\section{CTS - Compagnie des transports Strasbourgeois}


\begin{center}
  \includegraphics[width=3cm]{images/orga/da_381cf63f7b407883adef58db357a3c-100.png}
\end{center}


Compagnie des transports Strasbourgeois


\vspace{0.5cm}

\needspace{12\baselineskip}
\subsection*{GTFS - Strasbourg
}
  \begin{wrapfigure}{r}{2.5cm}
    \centering
    \qrcode[nolink]{https://data.gouv.fr/dataset/5ae1715488ee384c8ba0342b}
  \end{wrapfigure}

Licence : \textbf{Licence Ouverte
}\newline
Créé le : 2018-04-26\newline
Modifié le : 2018-04-26\newline
Mise à jour : mensuelle\newline
Popularité : 1 réutilisation,  0 suivi\newline
Mots-clé : \emph{aucun
}\newline
Permalien : \url{https://data.gouv.fr/dataset/5ae1715488ee384c8ba0342b}\newline

\par
\noindent
    Fichiers GTFS du réseau de transport de l'Eurométropôle de Strasbourg.
Ne couvre pas les transports du Bas Rhin (CTBR)


\vspace{0.5cm}

\clearpage
\section{DAAF Martinique}


\begin{center}
  \includegraphics[width=3cm]{images/orga/4c_1569085b8d46fcaa4771c5075d9b83-100.png}
\end{center}


DAAF Martinique


\vspace{0.5cm}

\needspace{12\baselineskip}
\subsection*{Registre Parcellaire Graphique anonymisé de la Martinique pour l'année
2013
}\index{agriculture!parcellaire!agricole}\index{boundaries}\index{donnees!ouvertes}\index{occupation!des!terres}\index{passerelle!inspire}
  \begin{wrapfigure}{r}{2.5cm}
    \centering
    \qrcode[nolink]{https://data.gouv.fr/dataset/589b011f88ee3811189b81a4}
  \end{wrapfigure}

Licence : \textbf{Licence Ouverte version 2.0
}\newline
Créé le : 2017-02-08\newline
Modifié le : 2019-03-13\newline
Popularité : 1 réutilisation,  0 suivi\newline
Mots-clé : \emph{agriculture-parcellaire-agricole, boundaries, donnees-ouvertes, occupation-des-terres, passerelle-inspire
}\newline
Permalien : \url{https://data.gouv.fr/dataset/589b011f88ee3811189b81a4}\newline

\par
\noindent
    \begin{itemize}

\item
  Saisie initiale effectuée par l'exploitant sur fonds A4 réalisés
  spécifiquement, à partir de la BD-ORTHO, pour une saisie au 1: 5000 et
  pré-diffusés à l'exploitant (saisie selon procédure définie par la
  maîtrise d'ouvrage) ;- puis évolutions successives selon chaîne de
  traitement mise en oeuvre par l'ASP (gestionnaire) selon ses outils
  propres, dont l'application ISIS-TELEPAC;- Anonymisation des îlots
  (suppression des données nominatives des îlots);- Génération d'un
  identifiant numérique non significatif par îlot pour permettre le lien
  avec les données attributaires- Sélection géographique des parcelles
  intersectant le contour GeoFLA du département
\end{itemize}

\textbf{Origine}

\begin{itemize}

\item
  Saisie initiale effectuée par l'exploitant sur fonds A4 réalisés
  spécifiquement, à partir de la BD-ORTHO, pour une saisie au 1: 5000 et
  pré-diffusés à l'exploitant (saisie selon procédure définie par la
  maîtrise d'ouvrage) ;- puis évolutions successives selon chaîne de
  traitement mise en oeuvre par l'ASP (gestionnaire) selon ses outils
  propres, dont l'application ISIS-TELEPAC;- Anonymisation des îlots
  (suppression des données nominatives des îlots);- Génération d'un
  identifiant numérique non significatif par îlot pour permettre le lien
  avec les données attributaires- Sélection géographique des parcelles
  intersectant le contour GeoFLA du département
\end{itemize}

\textbf{Organisations partenaires}

DAAF Martinique

\textbf{Liens annexes}

\begin{itemize}

\item
  \href{http://ogc.geo-ide.developpement-durable.gouv.fr/csw/all-dataset?REQUEST=GetRecordById\&SERVICE=CSW\&VERSION=2.0.2\&RESULTTYPE=results\&elementSetName=full\&TYPENAMES=gmd:MD_Metadata\&OUTPUTSCHEMA\&\#x3D\%5Bhttp://www.isotc211.org/2005/gmd\&ID\&\%5D(http://www.isotc211.org/2005/gmd\&ID\&)x3D;fr-120066022-jdd-5b38fefc-50ac-4a45-ba46-e99674edafc1}{Vue
  XML des métadonnées}
\item
  \href{http://geostandards.developpement-durable.gouv.fr/afficherPageStandard.do?jeu=N_RPG2009_ANONYME_S}{Standard
  de données COVADIS : Îlots du Registre Parcellaire Graphique anonyme
  2009}
\item
  \href{http://ogc.geo-ide.developpement-durable.gouv.fr/wxs?map=/opt/data/carto/geoide-catalogue/1.4/org_38128/42c69310-9461-4fb5-80be-c809085c126c.internet.map}{URL
  de base des services wms/wfs sur internet}
\end{itemize}

➞
\href{https://geo.data.gouv.fr/fr/datasets/792dfb946b294aa52d49aec6abd371a0e7fe3ba3}{Consulter
cette fiche sur geo.data.gouv.fr}


\vspace{0.5cm}
\needspace{3\baselineskip} \rule{4cm}{0.25pt}\newline\textbf{Aussi disponible du même producteur :}\begin{itemize}
\item \href{https://data.gouv.fr/dataset/589b011f88ee38111a9b81a4}{CartoPLU 2018 Ile-de-France : Prescriptions surfaciques}
\item \href{https://data.gouv.fr/dataset/58a19552c751df6737ae0a65}{Potentialités agricoles des sols en Martinique}
\item \href{https://data.gouv.fr/dataset/58a1955288ee3866d79b81a4}{Registre Parcellaire Graphique 2014 anonymisé de la Martinique}
\item \href{https://data.gouv.fr/dataset/58a1955288ee3866d99b81a4}{Registre Parcellaire Graphique 2015 anonymisé de la Martinique}
\item \href{https://data.gouv.fr/dataset/5a099dec88ee3828e67aaf81}{Registre Parcellaire Graphique 2016 anonymisé de la Martinique}
\item \href{https://data.gouv.fr/dataset/5a099decc751df1c9ee02d61}{Registre Parcellaire Graphique 2016 anonymisé de la Martinique - parcelles}
\item \href{https://data.gouv.fr/dataset/589b011f88ee3811179b81a4}{Registre Parcellaire Graphique anonymisé de la Martinique pour l'année 2008}
\item \href{https://data.gouv.fr/dataset/589b011fc751df121bae0a65}{Registre Parcellaire Graphique anonymisé de la Martinique pour l'année 2011}
\item \href{https://data.gouv.fr/dataset/589b011fc751df121aae0a65}{Registre Parcellaire Graphique anonymisé de la Martinique pour l'année 2012}
\end{itemize}

\clearpage
\section{\#{}dataFin}


\begin{center}
  \includegraphics[width=3cm]{images/orga/fd_ec74197c4a405694934414c94bbe83-100.png}
\end{center}


Quand les institutions rencontrent la société civile pour travailler sur
l'ouverture des données financières publiques


\vspace{0.5cm}

\needspace{12\baselineskip}
\subsection*{Enquête sur le hackathon \#dataFin de juin 2018
}\index{enquete}\index{hackathon}\index{sondage}
  \begin{wrapfigure}{r}{2.5cm}
    \centering
    \qrcode[nolink]{https://data.gouv.fr/dataset/5b4c588988ee38292289c6db}
  \end{wrapfigure}

Licence : \textbf{Creative Commons Attribution
}\newline
Créé le : 2018-07-16\newline
Modifié le : 2018-07-16\newline
De 2018-06-15 à 2018-06-16\newline
Mise à jour : ponctuelle\newline
Popularité : 1 réutilisation,  0 suivi\newline
Mots-clé : \emph{enquete, hackathon, sondage
}\newline
Permalien : \url{https://data.gouv.fr/dataset/5b4c588988ee38292289c6db}\newline

\par
\noindent
    Les 15 et 16 juin 2018, l'Assemblée nationale, la Cour des comptes et le
Ministère de l'action et des comptes publics ont organisé un
\href{https://datafin.fr/}{hackathon pour exploiter les données
financières publiques}.

Lors de ce hackathon un questionnaire anonyme a été soumis aux
participants.

Ce jeu de données contient l'ensemble des réponses à ce questionnaire,
sans aucun traitement.


\vspace{0.5cm}

\clearpage
\section{Datalift}


\begin{center}
  \includegraphics[width=3cm]{images/orga/ed_9d68ab50c74401a52482626ecd92b2-100.jpg}
\end{center}


Datalift porte les données brutes structurées venant de plusieurs
formats (bases de données, CSV, XML) vers des données sémantiques
interconnectées sur le Web de données.

Datalift est une plateforme pour publier et interconnecter des jeux de
données sur le web de données. Datalift à la fois publie des jeux de
données provenant d'un réseau de partenaires et propose un ensemble
d'outils facilitant le processus de publication de jeux de données.

\url{http://datalift.org/}{]}(http://datalift.org/{]}(http://datalift.org/))


\vspace{0.5cm}

\needspace{12\baselineskip}
\subsection*{Indicateurs de valeur ajoutée des lycées d'enseignement professionnel,
modélisés sous forme de cubes de données
}\index{bac}\index{lycee}\index{resultat}\index{valeur}
  \begin{wrapfigure}{r}{2.5cm}
    \centering
    \qrcode[nolink]{https://data.gouv.fr/dataset/53699692a3a729239d204a8f}
  \end{wrapfigure}

Licence : \textbf{Licence Ouverte
}\newline
Créé le : 2014-03-26\newline
Modifié le : 2015-10-07\newline
De 2009-09-01 à 2012-08-31\newline
Granularité : au point d'intérêt\newline
Popularité : 1 réutilisation,  3 suivis\newline
Mots-clé : \emph{bac, lycee, resultat, valeur
}\newline
Permalien : \url{https://data.gouv.fr/dataset/53699692a3a729239d204a8f}\newline

\par
\noindent
    Ce jeux de données résulte de l'enrichissement du jeu suivant :
\emph{Indicateurs de valeur ajoutée des lycées d'enseignement
professionnel, publié le 14 septembre 2013 par Ministère de l'Education
Nationale sur la plateforme data.gouv.fr :
\url{https://www.data.gouv.fr/fr/dataset/indicateurs-de-valeur-ajoutee-des-lycees-d-enseignement-professionnel-00000000}{]}(https://www.data.gouv.fr/fr/dataset/indicateurs-de-valeur-ajoutee-des-lycees-d-enseignement-professionnel-00000000{]}(https://www.data.gouv.fr/fr/dataset/indicateurs-de-valeur-ajoutee-des-lycees-d-enseignement-professionnel-00000000))données
statistiques : les indicateurs de valeur ajoutée des lycées permettent
d'évaluer l'action propre de chaque lycée. Ils sont établis à partir des
résultats des élèves au baccalauréat et de leur parcours scolaire dans
l'établissement. Les lycées professionnels, publics et privés sous
contrat, sont concernés. Il ne s'agit aucunement d'un classement mais
d'un regard croisé sur les trois indicateurs et les « valeurs ajoutées »
correspondantes.}

Par rapport à ce dernier jeu, les statistiques sont modélisées sous
forme de cubes de données (datacube).


\vspace{0.5cm}
\needspace{12\baselineskip}
\subsection*{Liste des établissements des premier et second degrés pour les secteurs
publics et privés en France (coordonnées WGS84, données liées aux codes
INSEE)
}\index{ecole}\index{ecoles}\index{enseignement}\index{etablissement}\index{lycee}
  \begin{wrapfigure}{r}{2.5cm}
    \centering
    \qrcode[nolink]{https://data.gouv.fr/dataset/5369990aa3a729239d205126}
  \end{wrapfigure}

Licence : \textbf{Licence Ouverte
}\newline
Créé le : 2014-03-26\newline
Modifié le : 2016-01-20\newline
De 2012-01-01 à 2012-12-31\newline
Granularité : au point d'intérêt\newline
Popularité : 1 réutilisation,  4 suivis\newline
Mots-clé : \emph{ecole, ecoles, enseignement, etablissement, lycee
}\newline
Permalien : \url{https://data.gouv.fr/dataset/5369990aa3a729239d205126}\newline

\par
\noindent
    Liste des établissements d'enseignement du ministère de l'éducation
nationale, de la jeunesse et de la vie associative, des premier et
second degrés, des secteurs public et privé, comportant les coordonnées
X et Y au format WGS84 et reliés aux codes INSEE.

Ce jeux de données résulte de l'enrichissement du jeu suivant : Liste
des établissements d'enseignement des premier et second degrés du
ministère de l'éducation nationale, pour les secteurs public et privé,
comportant les données X et Y de géolocalisation fournies par l'IGN -
Actualisation juin 2012, publié le 14 septembre 2013 par Ministère de
l'Education Nationale sur la plateforme data.gouv.fr à l'adresse :
\url{https://www.data.gouv.fr/fr/dataset/liste-des-etablissements-d-enseignement-des-premier-et-second-degres-du-ministere-de-l-educat-564055}{]}(https://www.data.gouv.fr/fr/dataset/liste-des-etablissements-d-enseignement-des-premier-et-second-degres-du-ministere-de-l-educat-564055{]}(https://www.data.gouv.fr/fr/dataset/liste-des-etablissements-d-enseignement-des-premier-et-second-degres-du-ministere-de-l-educat-564055))
Par rapport à ce dernier jeu, les coordonnées Lambert93 ont été
transformée en coordonnées WGS84. Chaque établissement est également
relié au code INSEE de sa commune d'appartenance (URIs officielles
publiées sur
\url{http://rdf.insee.fr/}).{]}(http://rdf.insee.fr/{]}(http://rdf.insee.fr/)).)


\vspace{0.5cm}

\clearpage
\section{DDT de la Moselle}


\begin{center}
  \includegraphics[width=3cm]{images/orga/44_92356715984aeab05e1eea2667cf64-100.png}
\end{center}


DDT de la Moselle


\vspace{0.5cm}

\needspace{12\baselineskip}
\subsection*{Périmètre affecté à un lieutenant de louveterie en Moselle
}\index{chasse}\index{donnees!ouvertes}\index{lieutenant}\index{lieutenant!de!louveterie}\index{louveterie}\index{louvetier}\index{passerelle!inspire}\index{utilities!communication}
  \begin{wrapfigure}{r}{2.5cm}
    \centering
    \qrcode[nolink]{https://data.gouv.fr/dataset/5883d422c751df0f1dae0a7b}
  \end{wrapfigure}

Licence : \textbf{Licence Ouverte version 2.0
}\newline
Créé le : 2017-01-21\newline
Modifié le : 2019-03-16\newline
Popularité : 1 réutilisation,  0 suivi\newline
Mots-clé : \emph{chasse, donnees-ouvertes, lieutenant, lieutenant-de-louveterie, louveterie, louvetier, passerelle-inspire, utilities-communication
}\newline
Permalien : \url{https://data.gouv.fr/dataset/5883d422c751df0f1dae0a7b}\newline

\par
\noindent
    Partie du territoire départemental affectée à un lieutenant de
louveterie. Les dispositions relatives aux lieutenants de louveterie
figurent aux articles L. 427-1 à L. 427-7 et R. 427-1 à R. 427-4 du code
de l'environnement. Les lieutenants de louveterie sont nommés par le
préfet et concourent sous son autorité à la régu- lation et à la
destruction des animaux susceptibles d'occasionner des dégâts. Ils sont
assermentés et ont qualité pour constater, dans les limites de la
circonscription qui leur est fixée, les infractions à la police de la
chasse. Ils sont les conseillers techniques de l'administration sur les
problèmes posés par la gestion de la faune sauvage ; les chasses et
battues administratives sont organisées sous leur contrôle et sous leur
responsabilité technique. Leurs fonctions, exercées dans l'intérêt
général, sont bénévoles.

\textbf{Origine}

Par agglomération et/ou numérisation des communes et parties de communes
concernées, sur base du référentiel BD Carto de l'IGN.

\textbf{Organisations partenaires}

DDT Moselle

\textbf{Liens annexes}

\begin{itemize}

\item
  \href{http://catalogue.geo-ide.developpement-durable.gouv.fr/catalogue/srv/fre/catalog.search\#/metadata/fr-120066022-jdd-23828a59-dd6b-431c-9d4e-33e65775a83f}{Vue
  HTML des métadonnées sur internet}
\item
  \href{http://ogc.geo-ide.developpement-durable.gouv.fr/csw/all-dataset?REQUEST=GetRecordById\&SERVICE=CSW\&VERSION=2.0.2\&RESULTTYPE=results\&elementSetName=full\&TYPENAMES=gmd:MD_Metadata\&OUTPUTSCHEMA\&\#x3D\%5Bhttp://www.isotc211.org/2005/gmd\&ID\&\%5D(http://www.isotc211.org/2005/gmd\&ID\&)x3D;fr-120066022-jdd-23828a59-dd6b-431c-9d4e-33e65775a83f}{Vue
  XML des métadonnées}
\item
  \href{http://geostandards.developpement-durable.gouv.fr/afficherPageStandard.do?jeu=N_CHASSE_LOUV_ZINF_S}{Standard
  de données COVADIS : Périmètre affecté à un lieutenant de louveterie.}
\item
  \href{http://ogc.geo-ide.developpement-durable.gouv.fr/wxs?map=/opt/data/carto/geoide-catalogue/1.4/org_38056/c90c7b7c-f926-4f77-b1ac-aa0a3d26e1f7.internet.map}{URL
  de base des services wms/wfs sur internet}
\end{itemize}

➞
\href{https://geo.data.gouv.fr/fr/datasets/a117ca5b216e36a8dfc7c347326a08c92ed82ec1}{Consulter
cette fiche sur geo.data.gouv.fr}


\vspace{0.5cm}
\needspace{3\baselineskip} \rule{4cm}{0.25pt}\newline\textbf{Aussi disponible du même producteur :}\begin{itemize}
\item \href{https://data.gouv.fr/dataset/5883d422c751df5ab4ae0acb}{Directive Territoriale d'Aménagement des Bassins Miniers Nord-Lorrains (DTA)}
\item \href{https://data.gouv.fr/dataset/5883d422c751df16c9ae0a67}{Périmètre du PPRN le Giessen Ebersheim}
\item \href{https://data.gouv.fr/dataset/5883d42388ee38099f9b81a4}{Plan de protection de l'atmosphère (PPA) en Moselle}
\item \href{https://data.gouv.fr/dataset/5883d42288ee38061b9b81aa}{Unité cynégétique en Moselle}
\end{itemize}

\clearpage
\section{DDTM Manche}


\begin{center}
  \includegraphics[width=3cm]{images/orga/c3_309976e2fc40cfa25ffce4814d88eb-100.png}
\end{center}


DDTM Manche


\vspace{0.5cm}

\needspace{12\baselineskip}
\subsection*{Zonage de pâturage et de repli pour l'AOC pré-salés
}\index{agneaux}\index{agriculture!zonages!agricoles}\index{aoc}\index{donnees!ouvertes}\index{farming}\index{passerelle!inspire}\index{pre!sale}
  \begin{wrapfigure}{r}{2.5cm}
    \centering
    \qrcode[nolink]{https://data.gouv.fr/dataset/58aeeb42c751df507939044d}
  \end{wrapfigure}

Licence : \textbf{Licence Ouverte version 2.0
}\newline
Créé le : 2017-02-23\newline
Modifié le : 2019-03-17\newline
Popularité : 1 réutilisation,  0 suivi\newline
Mots-clé : \emph{agneaux, agriculture-zonages-agricoles, aoc, donnees-ouvertes, farming, passerelle-inspire, pre-sale
}\newline
Permalien : \url{https://data.gouv.fr/dataset/58aeeb42c751df507939044d}\newline

\par
\noindent
    Zone de pâturage et de repli quand les pré-salés sont recouverts à forte
marée définie par le décret n\degree{}2009-1245 du 15 octobre 2009. Les
zonages peuvent être des communes entières ou un ensemble de sections de
commune.

\textbf{Origine}

numérisation réalisée soit à l'échelle du cadastre pour les zones de
repli, soit par la BD-Ortho pour les zones de pâturage en pré-salés

\textbf{Organisations partenaires}

DDTM Manche, Institut NAtional de l'Origine et de la qualité

\textbf{Liens annexes}

\begin{itemize}

\item
  \href{http://ogc.geo-ide.developpement-durable.gouv.fr/wxs?map=/opt/data/carto/geoide-catalogue/1.4/org_38042/f9bd2e02-d279-4f3a-b2d5-1a69af239bb2.internet.map}{URL
  de base des services wms/wfs sur internet
  :{[}http://ogc.geo-ide.developpement-durable.gouv.fr/wxs?map\&{]}(http://ogc.geo-ide.developpement-durable.gouv.fr/wxs?map\&)x3D;/opt/data/carto/geoide-catalogue/1.4/org\_38042/f9bd2e02-d279-4f3a-b2d5-1a69af239bb2.internet.map}
\end{itemize}

➞
\href{https://geo.data.gouv.fr/fr/datasets/74c30be25e35b59e840362eb33c345d16ba2ac92}{Consulter
cette fiche sur geo.data.gouv.fr}


\vspace{0.5cm}
\needspace{3\baselineskip} \rule{4cm}{0.25pt}\newline\textbf{Aussi disponible du même producteur :}\begin{itemize}
\item \href{https://data.gouv.fr/dataset/5a097e45c751df6a16d463da}{Bâtiment d'exploitation agricole anonymisé de la Manche}
\item \href{https://data.gouv.fr/dataset/58aeeb3ec751df4ea775111c}{Cadastre des parcs conchylicoles du département de la Manche (50)}
\item \href{https://data.gouv.fr/dataset/58aeeb3cc751df508bc1758e}{Cadastre des parcs piscicoles sur le littoral du département de la Manche (50)}
\item \href{https://data.gouv.fr/dataset/58aeeb3e88ee381ced59285d}{Cadastre mytilicole du département de la Manche (50)}
\item \href{https://data.gouv.fr/dataset/584f202ec751df3643c0bb7f}{Carte de bruit stratégique - jour - A75 – voies de l’État dans le Cantal (Type A)}
\item \href{https://data.gouv.fr/dataset/58aebcbbc751df04b881c080}{CBS\_Dépassement des valeurs limites (type C - indice Ln)}
\item \href{https://data.gouv.fr/dataset/58aeeb3fc751df522e91079f}{Communes de la Manche couvertes par un document d'urbanisme opposable et numérisé à la date de saisie.}
\item \href{https://data.gouv.fr/dataset/558a820e88ee3871154a2316}{Communes disposant d'un règlement local de publicité (RLP) en Lorraine}
\item \href{https://data.gouv.fr/dataset/558e9b8dc751df2d9aa453be}{Communes du littoral du département de la Manche (50)}
\item \href{https://data.gouv.fr/dataset/58aeeb3f88ee381cf48415f1}{Communes du littoral du département de la Manche (50)}
\item \href{https://data.gouv.fr/dataset/58aeeb3e88ee381cd3010efa}{Cours d'eau non domaniaux générateurs de servitudes A4 dans la Manche}
\item \href{https://data.gouv.fr/dataset/58aeeb4388ee381cf48415f3}{Etiquette habillant le plan d'un PLU - Manche}
\item \href{https://data.gouv.fr/dataset/58aeeb3fc751df508bc17590}{Gestionnaire de SUP dans la Manche}
\item \href{https://data.gouv.fr/dataset/58aeeb3d88ee381b084f263b}{Halage et marchepied générateurs de servitudes EL3 dans la Manche}
\item \href{https://data.gouv.fr/dataset/58aeeb4388ee381cd3010efb}{ICPE ou sites de stockage souterrains générateurs de servitudes PM3 dans le cadre d'un PPRT dans la Manche}
\item \href{https://data.gouv.fr/dataset/58aeeb42c751df4ea775111d}{Limites Administratives Portuaires du département de la Manche (50)}
\item \href{https://data.gouv.fr/dataset/58aeeb3fc751df508bc17591}{Liste des servitudes d'utilité publique dans la Manche}
\item \href{https://data.gouv.fr/dataset/56220dafc751df0ce9cdbb4a}{Mesures compensatoires portant sur un linéaire et prévues par une autorisation de défrichement dans le département de la Manche}
\item \href{https://data.gouv.fr/dataset/58aeeb3c88ee381cd3010ef8}{Passage des piétons sur le littoral générateur de servitudes EL9 dans la Manche}
\item \href{https://data.gouv.fr/dataset/58aeeb43c751df508bc17593}{Périmètre informatif d'un PLU - Manche}
\item \href{https://data.gouv.fr/dataset/58aeeb3c88ee381ced59285c}{Point informatif d'un PLU - Manche}
\item \href{https://data.gouv.fr/dataset/58aeeb3ec751df508bc1758f}{Ports du département de la Manche (50)}
\item \href{https://data.gouv.fr/dataset/58aeeb42c751df508bc17592}{Prescription linéraire d'un PLU - Manche}
\item \href{https://data.gouv.fr/dataset/58aeeb45c751df507939044f}{Prescription surfacique d'un PLU - Manche}
\item \href{https://data.gouv.fr/dataset/58aeeb4288ee381b084f263d}{Schéma de cohérence territoriale}
\item \href{https://data.gouv.fr/dataset/58aeeb4088ee381b084f263c}{Secteur carte communale de la Manche}
\item \href{https://data.gouv.fr/dataset/58aeeb44c751df507939044e}{Table exprimant la relation (n - m) entre les SUP et les actes les instituant dans la Manche}
\item \href{https://data.gouv.fr/dataset/58aeeb44c751df508bc17594}{Tracé ponctuel habillant le plan d'un PLU - Manche}
\item \href{https://data.gouv.fr/dataset/56ec317e88ee3839ade1a63a}{[TRI de l’AGGLOMERATION MULHOUSIENNE] Enjeux correspondant aux établissements, infrastructures et installations sensibles}
\item \href{https://data.gouv.fr/dataset/58aeeb3dc751df4ea775111b}{Zonages rélementaires d'un PPRN ou d'un PPRM assiettes de servitudes PM1 dans la Manche}
\item \href{https://data.gouv.fr/dataset/58aeeb42c751df522e9107a0}{Zonages rélementaires d'un PPRN ou d'un PPRM générateurs de servitudes PM1 dans la Manche}
\item \href{https://data.gouv.fr/dataset/558ade91c751df04a0a453cd}{Zone réglementée du Plan de Prévention des Risques Mouvement de Terrain (PPRMVT Tunnel - 13DDTM19950058) sur la commune de Gignac-la-Nerthe - Bouches-du-Rhône}
\item \href{https://data.gouv.fr/dataset/58aeeb3dc751df5246070fb0}{Zones de protection liées aux servitudes de la catégorie EL9 (Passage des piétons sur le littoral) dans la Manche}
\end{itemize}

\clearpage
\section{DDTM Vendée}


\begin{center}
  \includegraphics[width=3cm]{images/orga/b5_c8a0de69014d9cbea035f17d9d4ca7-100.png}
\end{center}


DDTM Vendée


\vspace{0.5cm}

\needspace{12\baselineskip}
\subsection*{Périmètre affecté à un lieutenant de louveterie
}\index{donnees!ouvertes}\index{economy}\index{louveterie}\index{passerelle!inspire}
  \begin{wrapfigure}{r}{2.5cm}
    \centering
    \qrcode[nolink]{https://data.gouv.fr/dataset/58aef851c751df66175f5900}
  \end{wrapfigure}

Licence : \textbf{Licence Ouverte version 2.0
}\newline
Créé le : 2017-02-23\newline
Modifié le : 2019-03-07\newline
Popularité : 1 réutilisation,  0 suivi\newline
Mots-clé : \emph{donnees-ouvertes, economy, louveterie, passerelle-inspire
}\newline
Permalien : \url{https://data.gouv.fr/dataset/58aef851c751df66175f5900}\newline

\par
\noindent
    Le préfet nomme pour une durée déterminée les lieutenants de louveterie
et détermine les territoires qui leurs sont affectés (CE L 427-1). Ces
territoires peuvent être infra-communaux.

\textbf{Origine}

Le préfet nomme pour une durée déterminée les lieutenants de louveterie
et détermine les territoires qui leurs sont affectés (CE L 427-1) Tracé
réalisé avec la BDCARTO

\textbf{Organisations partenaires}

DDTM Vendée

\textbf{Liens annexes}

\begin{itemize}

\item
  \href{http://ogc.geo-ide.developpement-durable.gouv.fr/csw/all-dataset?REQUEST=GetRecordById\&SERVICE=CSW\&VERSION=2.0.2\&RESULTTYPE=results\&elementSetName=full\&TYPENAMES=gmd:MD_Metadata\&OUTPUTSCHEMA\&\#x3D\%5Bhttp://www.isotc211.org/2005/gmd\&ID\&\%5D(http://www.isotc211.org/2005/gmd\&ID\&)x3D;fr-120066022-jdd-a1f05512-b67f-4263-a2b7-70bdd6fbfe80}{Vue
  XML des métadonnées}
\item
  \href{http://ogc.geo-ide.developpement-durable.gouv.fr/wxs?map=/opt/data/carto/geoide-catalogue/1.4/org_38110/8e26b532-84b3-4490-9d6a-8fc58abce73e.internet.map}{URL
  de base des services wms/wfs sur internet}
\end{itemize}

➞
\href{https://geo.data.gouv.fr/fr/datasets/71b0dc9bcbe2c7ab9d9a8b9103b271efa92c2d3e}{Consulter
cette fiche sur geo.data.gouv.fr}


\vspace{0.5cm}
\needspace{3\baselineskip} \rule{4cm}{0.25pt}\newline\textbf{Aussi disponible du même producteur :}\begin{itemize}
\item \href{https://data.gouv.fr/dataset/5a09742088ee385e2bce2b60}{Cadastre conchylicole de la Vendée : linéaire}
\item \href{https://data.gouv.fr/dataset/5a097419c751df5890b7ce42}{Cadastre conchylicole de la Vendée : surfacique}
\item \href{https://data.gouv.fr/dataset/5883cd2d88ee3810b49b81e2}{DDT 70 - Classement sonore des lignes ferroviaires en Haute-Saône}
\item \href{https://data.gouv.fr/dataset/5883cd2ec751df5ab4ae0aae}{DDT 70 - Classement sonore des routes nationales et routes départementales en Haute-Saône}
\item \href{https://data.gouv.fr/dataset/5883cd3188ee386def9b81e2}{DDT 70 - SPANC (Service Public d'Assainissement Non Collectif)}
\item \href{https://data.gouv.fr/dataset/5883cd2f88ee3850419b81ed}{DDT 70- Zone réglementaire du PPRI n\degree{}20080026 Saône Basse Vallee}
\item \href{https://data.gouv.fr/dataset/5883cd3088ee3810c89b81cf}{DDT 70- Zone réglementaire du PPRI n\degree{}20090015 Durgeon\_Amont\_Hors\_Durgeon\_Aval}
\item \href{https://data.gouv.fr/dataset/58aef85188ee3830ed5fc99e}{Espaces remarquables en Vendée}
\item \href{https://data.gouv.fr/dataset/58aef856c751df65d9ad4b92}{Exposition sonore Lnu en Vendée sur Routes Départementales et Communales}
\item \href{https://data.gouv.fr/dataset/58aef84ec751df66175f58fe}{Exposition sonore sur la période Ln en Vendée sur Routes Départementales et Communales}
\item \href{https://data.gouv.fr/dataset/56ec31a288ee382820e1a645}{HAMSTER - Terriers de 2006 en Alsace}
\item \href{https://data.gouv.fr/dataset/56ec318988ee382820e1a642}{HAMSTER - Terriers de 2009 en Alsace}
\item \href{https://data.gouv.fr/dataset/56ec317f88ee3839ade1a63b}{HAMSTER - Terriers de 2013 en Alsace}
\item \href{https://data.gouv.fr/dataset/58aef85088ee38329d9a9215}{Intercommunalités en Vendée}
\item \href{https://data.gouv.fr/dataset/58aeeb45c751df4ea775111e}{Les stations d'épuration en Vendée}
\item \href{https://data.gouv.fr/dataset/58aef84ec751df67bc8b308e}{Les stations d'épuration en Vendée}
\item \href{https://data.gouv.fr/dataset/58aef85288ee38329d9a9217}{L\_GEOM\_S\_085}
\item \href{https://data.gouv.fr/dataset/58aef85188ee3831015cc6ed}{Limites portuaires en Vendée}
\item \href{https://data.gouv.fr/dataset/58aef84e88ee3830ed5fc99c}{Limites transversales de la mer en Vendée}
\item \href{https://data.gouv.fr/dataset/55f18efbc751df54351f92bd}{Pays des Olonnes - Aléas des risques inondation et submersion marine du Pays des Olonnes - Vendée}
\item \href{https://data.gouv.fr/dataset/5a097424c751df5890b7ce43}{Pays des Olonnes - Enjeux ponctuels des risques inondation et submersion marine du Pays des Olonnes - Vendée}
\item \href{https://data.gouv.fr/dataset/5a097425c751df56b75e0bd2}{Pays des Olonnes - Enjeux surfaciques des risques inondation et submersion marine du Pays des Olonnes - Vendée}
\item \href{https://data.gouv.fr/dataset/56175c7bc751df053ecdbb52}{Périmètre de la Directive Territoriale d''Aménagement (DTA) des Bassins Miniers Nord Lorrains (depuis référentiel GEOFLA)}
\item \href{https://data.gouv.fr/dataset/58aef85588ee3830ed5fc9a0}{PPRi de la rivière La Sèvre Nantaise - Profils du Plan de Prévention des Risques d'Inondation - Vendée}
\item \href{https://data.gouv.fr/dataset/58aef84dc751df643fab7070}{PPRi de la rivière La Sèvre Nantaise - Zonage du Plan de Prévention des Risques d'Inondation - Vendée}
\item \href{https://data.gouv.fr/dataset/58aef854c751df67cb9f5642}{PPRi de la rivière La Vendée - Fontenay-le-Comte - Profils du plan de prévention des risques d'inondation - Vendée}
\item \href{https://data.gouv.fr/dataset/58aef85388ee383295e73adb}{PPRi de la rivière La Vendée - Fontenay-le-comte - Zonage du plan de prévention des risques d'inondation - Vendée}
\item \href{https://data.gouv.fr/dataset/58aef85188ee38329d9a9216}{PPRi de la rivière La Vendée - Profils du Plan de Prévention des Risques d'Inondation - Vendée}
\item \href{https://data.gouv.fr/dataset/58aef84f88ee3830ed5fc99d}{PPRi de la rivière La Vendée - Zonage du Plan de Prévention des Risques d'Inondation - Vendée}
\item \href{https://data.gouv.fr/dataset/58aef851c751df67cb9f5640}{PPRi des rivières Le Lay, Le Grand Lay, le Petit Lay - Zonage du Plan de Prévention du Risque inondation - Vendée}
\item \href{https://data.gouv.fr/dataset/5a09743188ee385e2bce2b62}{PPRi du Lay Aval - Enjeux Ponctuels du Plan de Prévention des Risques naturels prévisibles inondation - Vendée}
\item \href{https://data.gouv.fr/dataset/5a09742dc751df5890b7ce44}{PPRi du Lay Aval - Enjeux surfaciques du Plan de Prévention des Risques naturels prévisibles inondation - Vendée}
\item \href{https://data.gouv.fr/dataset/58aef84e88ee3831015cc6eb}{PPRi du Lay Aval - Origine du risque du Plan de Prévention des Risques naturels prévisibles inondation - Vendée}
\item \href{https://data.gouv.fr/dataset/58aef84ec751df65d9ad4b8f}{PPRi du Lay Aval - Périmètre du Plan de Prévention des Risques naturels prévisibles inondation - Vendée}
\item \href{https://data.gouv.fr/dataset/58aef84dc751df67bc8b308d}{PPRi du Lay Aval - Zonage du Plan de Prévention des Risques naturels prévisibles inondation - Vendée}
\item \href{https://data.gouv.fr/dataset/58aef85688ee3830ed5fc9a1}{PPRL de la Baie de Bourgneuf - Cotes de référence 2100 du Plan de Prévention des Risques naturels prévisibles Littoraux - Vendée}
\item \href{https://data.gouv.fr/dataset/58aef855c751df67bc8b3092}{PPRL de la Baie de Bourgneuf - Cotes de référence actuelle du Plan de Prévention des Risques naturels prévisibles Littoraux - Vendée}
\item \href{https://data.gouv.fr/dataset/58aef85588ee3831015cc6ee}{PPRL de la Baie de Bourgneuf - Enjeux Ponctuels du Plan de Prévention des Risques naturels prévisibles Littoraux - Vendée}
\item \href{https://data.gouv.fr/dataset/58aef85688ee38329d9a9219}{PPRL de la Baie de Bourgneuf - Périmètre du Plan de Prévention des Risques naturels prévisibles Littoraux - Vendée}
\item \href{https://data.gouv.fr/dataset/5a097419c751df4bc5bd806d}{PPRL de La Faute-sur-Mer - Cotes de référence 2100 du Plan de Prévention des Risques naturels prévisibles Littoraux - Vendée}
\item \href{https://data.gouv.fr/dataset/5a09742088ee385e4555b51d}{PPRL de La Faute-sur-Mer - Cotes de référence actuelle du Plan de Prévention des Risques naturels prévisibles Littoraux - Vendée}
\item \href{https://data.gouv.fr/dataset/5a09741bc751df56b75e0bd0}{PPRL de La Faute-sur-Mer - Enjeux Ponctuels du Plan de Prévention des Risques naturels prévisibles Littoraux - Vendée}
\item \href{https://data.gouv.fr/dataset/5a097437c751df5890b7ce47}{PPRL de La Faute-sur-Mer - Périmètre du Plan de Prévention des Risques naturels prévisibles Littoraux - Vendée}
\item \href{https://data.gouv.fr/dataset/5a09744088ee386ab8dee916}{PPRL de La Faute-sur-Mer  - Zonage du Plan de Prévention des Risques naturels prévisibles Littoraux - Vendée}
\item \href{https://data.gouv.fr/dataset/58aef84e88ee38329d9a9213}{PPRL de l'Ile de Noirmoutier - Aléas du Plan de Prévention des Risques naturels prévisibles Littoraux - Vendée}
\item \href{https://data.gouv.fr/dataset/58aedc18c751df36bc2f1261}{PPRL de l'Ile de Noirmoutier - Aléas du Plan de Prévention des Risques naturels prévisibles Littoraux - Vendée}
\item \href{https://data.gouv.fr/dataset/58aef85388ee3830f429245e}{PPRL de l'Ile de Noirmoutier - Enjeux Pontuels du Plan de Prévention des Risques naturels prévisibles Littoraux - Vendée}
\item \href{https://data.gouv.fr/dataset/58aef84d88ee3830ed5fc99b}{PPRL de l'Ile de Noirmoutier - Périmètre du Plan de Prévention des Risques naturels prévisibles Littoraux - Vendée}
\item \href{https://data.gouv.fr/dataset/58aef85088ee3830f429245d}{PPRL de l'Ile de Noirmoutier - Zonage du Plan de Prévention des Risques naturels prévisibles Littoraux - Vendée}
\item \href{https://data.gouv.fr/dataset/5a09743bc751df56f03082f4}{PPRL de Pays de Monts - Enjeux Surfaciques du Plan de Prévention des Risques naturels prévisibles Littoraux - Vendée}
\item et 26 autres jeux de données\end{itemize}

\clearpage
\section{Deuxième labo}


\begin{center}
  \includegraphics[width=3cm]{images/orga/47_ce27a1c5b043eab11233cd115c4757-100.png}
\end{center}


Depuis 2009, Deuxième labo agit pour façonner de nouvelles formes de
relations science-société, et accompagner les mutations du monde de la
recherche et de l'innovation. Cette agence est pilotée par Elifsu
Sabuncu (docteure en biologie) et Antoine Blanchard (ingénieur
AgroParisTech).

Frappés par l'impuissance de l'institution scientifique à montrer sa
richesse, à s'ouvrir à la pollinisation croisée et à renouveler son
fonctionnement, nous sommes plus que jamais convaincus de l'urgence à
préparer la recherche de demain, autour de trois axes fondateurs :

\begin{itemize}

\item
  pour un/e chercheur/e qui est entendu/e et fait la différence
\item
  pour un labo ouvert et visible
\item
  pour une recherche fertile et fertilisée
\end{itemize}


\vspace{0.5cm}

\needspace{12\baselineskip}
\subsection*{Directeurs de thèse, auteurs de thèse et rapporteurs
}\index{chercheurs}\index{doctorants}\index{recherche}\index{science}\index{these}
  \begin{wrapfigure}{r}{2.5cm}
    \centering
    \qrcode[nolink]{https://data.gouv.fr/dataset/536992eca3a729239d204055}
  \end{wrapfigure}

Licence : \textbf{Licence Ouverte
}\newline
Créé le : 2013-12-18\newline
Modifié le : 2016-03-10\newline
De 2006-01-01 à 2013-12-31\newline
Granularité : au pays\newline
Mise à jour : ponctuelle\newline
Popularité : 1 réutilisation,  0 suivi\newline
Mots-clé : \emph{chercheurs, doctorants, recherche, science, these
}\newline
Permalien : \url{https://data.gouv.fr/dataset/536992eca3a729239d204055}\newline

\par
\noindent
    Liste des personnes impliquées dans la recherche doctorale française
(directeurs de thèse, auteurs de thèse en cours ou soutenue, et
rapporteurs) extraite du site theses.fr et classée dans l'ordre
alphabétique du nom. Tous les noms ont été dédoublonnés, pour que les
homonymes n'apparaissent qu'une seule fois.

theses.fr recense toutes les thèses soutenues depuis 2006 dans les
établissements ayant choisi d'abandonner le dépôt de la thèse papier au
profit du support électronique (plus de 6 000 thèses) et les données
issues du Fichier central des thèses (plus de 66 000 thèses en
préparation dans les universités, principalement en lettres, sciences
humaines et sociales).

Les données ont été récoltées par Clément Levallois depuis l'API de
theses.fr grâce au code disponible
sur\url{https://github.com/seinecle/fichiercentraldestheses}


\vspace{0.5cm}

\clearpage
\section{Direction Départementale des Territoires de la Corrèze}


\begin{center}
  \includegraphics[width=3cm]{images/orga/2015-06-25_71d4791614584a5989d87e3d12cece5b_logo_DDT-100.png}
\end{center}


Direction Départementale des Territoires de la Corrèze


\vspace{0.5cm}

\needspace{12\baselineskip}
\subsection*{Autoroutes de la Corrèze
}\index{donnee!generique!habillage}\index{donnees!ouvertes}\index{location}\index{passerelle!inspire}
  \begin{wrapfigure}{r}{2.5cm}
    \centering
    \qrcode[nolink]{https://data.gouv.fr/dataset/58aebcbdc751df04b0f8147a}
  \end{wrapfigure}

Licence : \textbf{Licence Ouverte version 2.0
}\newline
Créé le : 2017-02-23\newline
Modifié le : 2019-03-16\newline
Popularité : 1 réutilisation,  0 suivi\newline
Mots-clé : \emph{donnee-generique-habillage, donnees-ouvertes, location, passerelle-inspire
}\newline
Permalien : \url{https://data.gouv.fr/dataset/58aebcbdc751df04b0f8147a}\newline

\par
\noindent
    L\_TRONCON\_AUTOROUTE\_BDT\_019 Tronçons d'autoroutes de la Corrèze
issue de la BD Topo de l'IGN

\textbf{Origine}

Tronçons de routes et autoroutes issues de la couche des tronçons
routiers de la BD TOPO de l'IGN

\textbf{Organisations partenaires}

DDT Corrèze

\textbf{Liens annexes}

\begin{itemize}

\item
  \href{http://ogc.geo-ide.developpement-durable.gouv.fr/wxs?map=/opt/data/carto/geoide-catalogue/1.4/org_37978/b889e11d-12d8-4dd0-ac0a-91533029ee35.internet.map}{URL
  de base des services wms/wfs sur internet
  :{[}http://ogc.geo-ide.developpement-durable.gouv.fr/wxs?map\&{]}(http://ogc.geo-ide.developpement-durable.gouv.fr/wxs?map\&)x3D;/opt/data/carto/geoide-catalogue/1.4/org\_37978/b889e11d-12d8-4dd0-ac0a-91533029ee35.internet.map}
\end{itemize}

➞
\href{https://geo.data.gouv.fr/fr/datasets/bdee6990c94b61d58f7a2963dd646f1a7042a84e}{Consulter
cette fiche sur geo.data.gouv.fr}


\vspace{0.5cm}
\needspace{3\baselineskip} \rule{4cm}{0.25pt}\newline\textbf{Aussi disponible du même producteur :}\begin{itemize}
\item \href{https://data.gouv.fr/dataset/58aebcbfc751df04b881c082}{Agendas 21 des communes de la Corrèze}
\item \href{https://data.gouv.fr/dataset/58aebcba88ee384e16b93b66}{Assemblage de l’ensemble des lignes de hauteurs de submersion fournies par le Service de Prévision des Crues de la Dordogne (SPC) en Corrèze}
\item \href{https://data.gouv.fr/dataset/58aebcbac751df02e291bfa4}{Cantons électoraux 2013 de la Corrèze}
\item \href{https://data.gouv.fr/dataset/58aebcbcc751df04c132614f}{Cantons électoraux 2015 de la Corrèze}
\item \href{https://data.gouv.fr/dataset/5a09b1a488ee384a71189114}{Collectivité compétente en assainissement collectif par commune en Corrèze}
\item \href{https://data.gouv.fr/dataset/5a09b19088ee384c6ae266d0}{Collectivité compétente en assainissement non collectif par commune en Corrèze}
\item \href{https://data.gouv.fr/dataset/5a09b1a0c751df3bca5f3bed}{Collectivité compétente en gestion des milieux aquatiques par commune en Corrèze}
\item \href{https://data.gouv.fr/dataset/58aebcc1c751df04c1326154}{Communes couvertes par un document d'urbanisme opposable et numérisé à la date de saisie en Corrèze}
\item \href{https://data.gouv.fr/dataset/58aebcc0c751df04b0f8147b}{Cours d'eau classés et réservés au titre de l'article L432-6 du Code de l'environnement en Corrèze (obsolète)}
\item \href{https://data.gouv.fr/dataset/58aebcbac751df04b0f81478}{Digues contre les inondations et submersion, et digues de canaux et de rivières canalisées en Corrèze}
\item \href{https://data.gouv.fr/dataset/56290b9b88ee384d4212613f}{Documents d'urbanisme PLU, POS, carte communale existant sur le département sous forme numérique en Corrèze}
\item \href{https://data.gouv.fr/dataset/58aebcb9c751df04c132614d}{Enjeux linéaires pour la préparation de la gestion de crise inondation en Corrèze.}
\item \href{https://data.gouv.fr/dataset/58aebcbc88ee384ff8d01bd7}{Enjeux ponctuels pour la préparation de la gestion de crise inondation en Corrèze}
\item \href{https://data.gouv.fr/dataset/58aebcbc88ee384fde221b62}{Enjeux pour la préparation de la gestion de crise inondation en Corrèze}
\item \href{https://data.gouv.fr/dataset/58aebcbcc751df04b0f81479}{Enjeux surfaciques pour la préparation de la gestion de crise inondation en Corrèze.}
\item \href{https://data.gouv.fr/dataset/558bc1e688ee3848424a22ea}{Enveloppes des zonages liées aux servitudes de la catégorie PM1 (Plan de prévention des risques naturels prévisibles ou miniers) en Corrèze}
\item \href{https://data.gouv.fr/dataset/559c028bc751df484b390bd4}{Éoliennes construites dans le département de la Charente.}
\item \href{https://data.gouv.fr/dataset/58aebcbe88ee384fea29f6f4}{Îlots culturaux du Registre Parcellaire Graphique 2014 anonymes de Corrèze}
\item \href{https://data.gouv.fr/dataset/558bc1eb88ee3871c04a22ef}{Informations linéaires des PLU numérisés en Corrèze}
\item \href{https://data.gouv.fr/dataset/558bc1ec88ee38430d4a22ec}{Informations ponctuelles des PLU numérisés en Corrèze}
\item \href{https://data.gouv.fr/dataset/58aebcbc88ee384e16b93b68}{Informations surfaciques des PLU numérisés en Corrèze}
\item \href{https://data.gouv.fr/dataset/56ec30a088ee381877e1a63b}{Interdiction d'accès aux routes génératrice de servitudes EL11 de la Charente}
\item \href{https://data.gouv.fr/dataset/58aebcc088ee384fde221b64}{Liste des documents d'urbanisme de type PLU, POS dont la numérisation est engagée en Corrèze}
\item \href{https://data.gouv.fr/dataset/5a09780cc751df5ea1ff7e0e}{Liste des servitudes par commune en Corrèze}
\item \href{https://data.gouv.fr/dataset/5a09b19b88ee384ab01017ad}{Liste des servitudes par commune en Corrèze}
\item \href{https://data.gouv.fr/dataset/5a099812c751df1133b78e4b}{Parcelles forestières sous engagement fiscal en Corrèze}
\item \href{https://data.gouv.fr/dataset/58aebcc088ee384ff8d01bda}{Périmètre des associations agréées de pêche et de protection du milieu aquatique (AAPPMA) de la Corrèze}
\item \href{https://data.gouv.fr/dataset/58aebcbdc751df04c1326150}{Périmètre des pays en Corrèze}
\item \href{https://data.gouv.fr/dataset/5a0974dcc751df5a720e450a}{Périmètre des pays en Corrèze}
\item \href{https://data.gouv.fr/dataset/58aebcbb88ee384e16b93b67}{Périmètre informatif des cartes communales (CC) numérisées en Corrèze}
\item \href{https://data.gouv.fr/dataset/58aebcbec751df04c1326151}{Périmètres des communautés de communes et communautés d'agglomération en 2016 en Corrèze}
\item \href{https://data.gouv.fr/dataset/558bc1f2c751df5bfda453be}{Petite région agricole (PRA) en Corrèze}
\item \href{https://data.gouv.fr/dataset/56290b9188ee3850d7126142}{Plan local d'urbanisme v2.0 numérisé en Corrèze}
\item \href{https://data.gouv.fr/dataset/58aebcbcc751df02e291bfa6}{Pôles d'excellence rurale (PER) en Corrèze}
\item \href{https://data.gouv.fr/dataset/58aebcb9c751df04b881c07f}{Prescriptions linéaires des PLU numérisés en Corrèze}
\item \href{https://data.gouv.fr/dataset/58aebcbd88ee384ff8d01bd8}{Prescriptions ponctuelles des PLU numérisés en Corrèze}
\item \href{https://data.gouv.fr/dataset/558bc1efc751df4543a453bf}{Prescriptions surfaciques des cartes communales (CC) numérisées en Corrèze}
\item \href{https://data.gouv.fr/dataset/58aebcbc88ee384fea29f6f3}{Prescriptions surfaciques des PLU numérisés en Corrèze}
\item \href{https://data.gouv.fr/dataset/58aebcbf88ee384e16b93b6b}{Référentiel des obstacles à l'écoulement (ROE) pour les obstacles validés et situés sur les cours d'eau du département de la Corrèze et des départements voisins (Cantal, Creuse, Dordogne, Lot, Puy-de-Dôme et Haute-Vienne).}
\item \href{https://data.gouv.fr/dataset/58aebcc088ee384fea29f6f6}{Règlements locaux de publicité (RLP) en Corrèze}
\item \href{https://data.gouv.fr/dataset/58aebcbfc751df04c1326152}{Réseau des routes à grande circulation de la Corrèze}
\item \href{https://data.gouv.fr/dataset/558bc1f588ee3871c04a22f0}{Restriction de circulation sur la voirie locale en Corrèze}
\item \href{https://data.gouv.fr/dataset/558bc1e6c751df3fffa453ba}{Restrictions de circulation sur les routes départementales (RD) en Corrèze}
\item \href{https://data.gouv.fr/dataset/58aebcb988ee384e16b93b65}{Schémas de cohérence territoriale (Scot) en Corrèze}
\item \href{https://data.gouv.fr/dataset/58aebcc0c751df04c1326153}{Schémas de cohérence territoriale (Scot) en Corrèze en 2017 sur Geofla}
\item \href{https://data.gouv.fr/dataset/58aebcb9c751df02e291bfa3}{Secteurs constructibles des cartes communales (CC) numérisées de la Corrèze}
\item \href{https://data.gouv.fr/dataset/558bc1e6c751df39d6a453bb}{Surfaces de halage et marchepied liées aux servitudes de la catégorie EL3 (Halage et marchepied) en Corrèze}
\item \href{https://data.gouv.fr/dataset/58aebcbcc751df04b881c081}{Tampon de 100 m autour des autoroutes (application du L111-6 CU) de la Corrèze}
\item \href{https://data.gouv.fr/dataset/58aebcbd88ee384e16b93b69}{Tampon de 75 m autour des tronçons du réseau à grande circulation de la Corrèze}
\item \href{https://data.gouv.fr/dataset/55f1abe988ee380c05a46ef0}{Territoire bénéficiant de LEADER 4ème génération en Corrèze}
\item et 10 autres jeux de données\end{itemize}

\clearpage
\section{Direction Départementale des Territoires de l'Aube}


\begin{center}
  \includegraphics[width=3cm]{images/orga/2015-06-23_ace5864d5d0346229beb2a664ee6a1bc_logo_DDT-100.png}
\end{center}


Direction Départementale des Territoires de l'Aube


\vspace{0.5cm}

\needspace{12\baselineskip}
\subsection*{Circonscription des lieutenants de louveterie dans l'Aube
}\index{donnees!ouvertes}\index{economy}\index{passerelle!inspire}
  \begin{wrapfigure}{r}{2.5cm}
    \centering
    \qrcode[nolink]{https://data.gouv.fr/dataset/55896ea288ee386f654a22ea}
  \end{wrapfigure}

Licence : \textbf{Licence Ouverte version 2.0
}\newline
Créé le : 2015-06-23\newline
Modifié le : 2019-02-23\newline
Popularité : 1 réutilisation,  0 suivi\newline
Mots-clé : \emph{donnees-ouvertes, economy, passerelle-inspire
}\newline
Permalien : \url{https://data.gouv.fr/dataset/55896ea288ee386f654a22ea}\newline

\par
\noindent
    Périmètre des circonscriptions des lieutenants de louveterie 2014

\textbf{Origine}

Périmètre crée à partir des routes de la BD Topo

\textbf{Organisations partenaires}

DDT Aube

\textbf{Liens annexes}

\begin{itemize}

\item
  \href{http://ogc.geo-ide.developpement-durable.gouv.fr/csw/all-dataset?REQUEST=GetRecordById\&SERVICE=CSW\&VERSION=2.0.2\&RESULTTYPE=results\&elementSetName=full\&TYPENAMES=gmd:MD_Metadata\&OUTPUTSCHEMA\&\#x3D\%5Bhttp://www.isotc211.org/2005/gmd\&ID\&\%5D(http://www.isotc211.org/2005/gmd\&ID\&)x3D;fr-120066022-jdd-0c0bc9c3-2880-441c-b306-5ed46388616a}{Vue
  XML des métadonnées}
\item
  \href{http://ogc.geo-ide.developpement-durable.gouv.fr/wxs?map=/opt/data/carto/geoide-catalogue/1.4/org_37960/24f4288d-cf6c-4d13-bdb8-61995d3dbeaa.internet.map}{URL
  de base des services wms/wfs sur internet}
\end{itemize}

➞
\href{https://geo.data.gouv.fr/fr/datasets/c6babede2a318cb15f12cb5b4be36e42ce51d6a1}{Consulter
cette fiche sur geo.data.gouv.fr}


\vspace{0.5cm}
\needspace{3\baselineskip} \rule{4cm}{0.25pt}\newline\textbf{Aussi disponible du même producteur :}\begin{itemize}
\item \href{https://data.gouv.fr/dataset/55f1b1c2c751df29861f92c4}{Accueil des gens du voyage dans l'Aube}
\item \href{https://data.gouv.fr/dataset/55896eb688ee386f534a22ea}{Aire géographique d'une appelation d'origine protégée de type fromage dans l'Aube}
\item \href{https://data.gouv.fr/dataset/55896ea288ee386f574a22ea}{Aire géographique d'une indication géographique protégée de type autre que le vin dans l'Aube}
\item \href{https://data.gouv.fr/dataset/55f1b1d0c751df226d1f9301}{Assiette de la servitude canalisation d'eau potable dite A5 dans l'Aube}
\item \href{https://data.gouv.fr/dataset/55f1b1ce88ee3814d3a46edd}{Bruit dans l'Environnement - Cartographie du Bruit Stratégique dans le département du Gers - Assemblage des données pour les infrastructures routières - 3ème échéance 2017/2022 - Ensemble des cartes}
\item \href{https://data.gouv.fr/dataset/5a0ae9b6c751df2fe4b66937}{Carte Communale d'Aubeterre}
\item \href{https://data.gouv.fr/dataset/55f1b18ac751df226d1f92f6}{Carte communale d'Eaux Puiseaux}
\item \href{https://data.gouv.fr/dataset/55f1b1dcc751df226d1f9304}{Carte communale de Bagneux la Fosse}
\item \href{https://data.gouv.fr/dataset/55f1b18388ee380c27a46ee2}{Carte communale de Baroville}
\item \href{https://data.gouv.fr/dataset/55f1b18d88ee380a3ea46eef}{Carte communale de Boulages}
\item \href{https://data.gouv.fr/dataset/55f1b18988ee380c05a46efd}{Carte communale de Brillecourt}
\item \href{https://data.gouv.fr/dataset/55f1b18688ee380a3ea46eee}{Carte communale de Chalette sur Voire}
\item \href{https://data.gouv.fr/dataset/58adb5c588ee381d591901e7}{Carte communale de Champignol lez Mondeville}
\item \href{https://data.gouv.fr/dataset/55f1b18688ee380c05a46efb}{Carte communale de Chapelle Vallon}
\item \href{https://data.gouv.fr/dataset/55f1b1d888ee380c05a46f18}{Carte communale de Chessy les Prés}
\item \href{https://data.gouv.fr/dataset/55f1b18688ee3814d3a46ec7}{Carte communale de Faux Villecerf}
\item \href{https://data.gouv.fr/dataset/55f1b189c751df26061f92e7}{Carte communale de Ferreux Quincey}
\item \href{https://data.gouv.fr/dataset/55f1b189c751df26061f92e8}{Carte communale de Fralignes}
\item \href{https://data.gouv.fr/dataset/58adb5c6c751df4cb60d8be5}{Carte communale de La Fosse Corduan}
\item \href{https://data.gouv.fr/dataset/55f1b1a888ee380a3ea46ef2}{Carte communale de la Villeneuve au Chemin}
\item \href{https://data.gouv.fr/dataset/55f1b18ec751df29861f92b9}{Carte communale de Les Croûtes}
\item \href{https://data.gouv.fr/dataset/55f1b195c751df261a1f92d3}{Carte communale de Lirey}
\item \href{https://data.gouv.fr/dataset/55f1b19188ee3814d3a46ecc}{Carte communale de Longsols}
\item \href{https://data.gouv.fr/dataset/55f1b199c751df1b831f92e3}{Carte communale de Luyères}
\item \href{https://data.gouv.fr/dataset/55f1b19888ee380c05a46f01}{Carte communale de Maraye en Othe}
\item \href{https://data.gouv.fr/dataset/55f1b19788ee3814d3a46ece}{Carte communale de Mesnil Saint Loup}
\item \href{https://data.gouv.fr/dataset/55f1b19c88ee380c27a46ee3}{Carte communale de Montier en l'Isle}
\item \href{https://data.gouv.fr/dataset/55f1b1a6c751df226d1f92fc}{Carte communale de Neuville sur Seine}
\item \href{https://data.gouv.fr/dataset/55f1b1dcc751df323e1f92b2}{Carte communale de Paisy Cosdon}
\item \href{https://data.gouv.fr/dataset/58adb5cbc751df522d98b82b}{Carte communale de Plessis Barbuise}
\item \href{https://data.gouv.fr/dataset/55f1b1a688ee3814c3a46eda}{Carte communale de Poligny}
\item \href{https://data.gouv.fr/dataset/55f1b1a3c751df26061f92f0}{Carte communale de Polisy}
\item \href{https://data.gouv.fr/dataset/55f1b1a588ee380c27a46ee4}{Carte communale de Racines}
\item \href{https://data.gouv.fr/dataset/55f1b1a5c751df226d1f92fb}{Carte communale de Rouvres les Vignes}
\item \href{https://data.gouv.fr/dataset/55f1b1a088ee3814d3a46ed1}{Carte communale de Saint Benoist sur Vanne}
\item \href{https://data.gouv.fr/dataset/55f1b19e88ee380c05a46f05}{Carte communale de Saint Hilaire sous Romilly}
\item \href{https://data.gouv.fr/dataset/55f1b1a6c751df26061f92f1}{Carte communale de Saint Jean de Bonneval}
\item \href{https://data.gouv.fr/dataset/55f1b19dc751df261a1f92d4}{Carte communale de Saint Loup de Buffigny}
\item \href{https://data.gouv.fr/dataset/55f1b1ae88ee380c05a46f08}{Carte communale de Saint Oulph}
\item \href{https://data.gouv.fr/dataset/55f1b1aac751df226d1f92fe}{Carte communale de Saint Phal}
\item \href{https://data.gouv.fr/dataset/55f1b1a7c751df226d1f92fd}{Carte communale de Villemaur sur Vanne}
\item \href{https://data.gouv.fr/dataset/55f1b1ae88ee3814c3a46edc}{Carte communale de Villemereuil}
\item \href{https://data.gouv.fr/dataset/58adb5cec751df4ea673d5f8}{Carte communale de Villeneuve au Chemin}
\item \href{https://data.gouv.fr/dataset/55f1b1aa88ee3814d3a46ed5}{Carte communale de Villery}
\item \href{https://data.gouv.fr/dataset/55f1b1b088ee380c05a46f0a}{Carte communale de Vulaines}
\item \href{https://data.gouv.fr/dataset/55f1b1a1c751df226d1f92f9}{Carte communale d'Orvilliers Saint Julien}
\item \href{https://data.gouv.fr/dataset/58adb5c688ee381f1a122c86}{Centrale nucléaire dans l'Aube}
\item \href{https://data.gouv.fr/dataset/55896ea688ee386f574a22eb}{Contrat urbain de cohésion sociale de l'Aube}
\item \href{https://data.gouv.fr/dataset/58adb5c488ee381d591901e6}{Document d'urbanisme d'Avant les Ramerupt}
\item \href{https://data.gouv.fr/dataset/55f1b1bfc751df29861f92c3}{Document d'urbanisme de Barbery Saint Sulpice}
\item et 325 autres jeux de données\end{itemize}

\clearpage
\section{Direction Départementale des Territoires de Loir-et-Cher}


\begin{center}
  \includegraphics[width=3cm]{images/orga/2015-06-27_4c70d12fa81640779cf1e839015bc60e_logo_DDT-100.png}
\end{center}


Direction Départementale des Territoires de Loir-et-Cher


\vspace{0.5cm}

\needspace{12\baselineskip}
\subsection*{Chasse - Périmètre affecté à un lieutenant de louveterie en
Loir-et-Cher.
}\index{donnees!ouvertes}\index{passerelle!inspire}\index{utilities!communication}
  \begin{wrapfigure}{r}{2.5cm}
    \centering
    \qrcode[nolink]{https://data.gouv.fr/dataset/58aee85f88ee381566315a18}
  \end{wrapfigure}

Licence : \textbf{Licence Ouverte version 2.0
}\newline
Créé le : 2017-02-23\newline
Modifié le : 2019-03-06\newline
Popularité : 1 réutilisation,  0 suivi\newline
Mots-clé : \emph{donnees-ouvertes, passerelle-inspire, utilities-communication
}\newline
Permalien : \url{https://data.gouv.fr/dataset/58aee85f88ee381566315a18}\newline

\par
\noindent
    Le préfet nomme pour une durée déterminée les lieutenants de louveterie
et détermine les territoires qui leurs sont affectés (CE L 427-1 et
Arrêté du 14 juin 2010 NOR : DEVN1013973A) Ces territoires peuvent être
infra-communaux. La louveterie est une chasse aux loups et autres grands
animaux nuisibles, en vue de leur destruction. Le préfet décide par
arrêté des destructions collectives d'animaux nuisibles. Il nomme, pour
une durée de six ans, des lieutenants de louveterie qui géreront ces
destructions collectives sous le contrôle de la direction départementale
des territoires.Le nombre des lieutenants est fixé en fonction de la
superficie, du boisement et du relief du département. Ces lieutenants
sont les conseillers techniques de l'administration en matière de
problèmes posés par la gestion de la faune sauvage.

\textbf{Origine}

Par agglomération et/ou numérisation des communes et parties de communes
concernées.

\textbf{Organisations partenaires}

DDT Loir-et-Cher

\textbf{Liens annexes}

\begin{itemize}

\item
  \href{http://ogc.geo-ide.developpement-durable.gouv.fr/csw/all-dataset?REQUEST=GetRecordById\&SERVICE=CSW\&VERSION=2.0.2\&RESULTTYPE=results\&elementSetName=full\&TYPENAMES=gmd:MD_Metadata\&OUTPUTSCHEMA\&\#x3D\%5Bhttp://www.isotc211.org/2005/gmd\&ID\&\%5D(http://www.isotc211.org/2005/gmd\&ID\&)x3D;fr-120066022-jdd-d5e124c9-8547-4925-95b2-9eac6f47964d}{Vue
  XML des métadonnées}
\item
  \href{http://geostandards.developpement-durable.gouv.fr/afficherPageStandard.do?id=1025}{Standard
  de données COVADIS : N\_CHASSE\_LOUV\_ZINF\_S}
\item
  \href{http://ogc.geo-ide.developpement-durable.gouv.fr/wxs?map=/opt/data/carto/geoide-catalogue/1.4/org_38024/8249020e-e784-4082-9d6f-fa65eb7914e0.internet.map}{URL
  de base des services wms/wfs sur internet}
\end{itemize}

➞
\href{https://geo.data.gouv.fr/fr/datasets/21e5fe39b21399dcac1184d7990a6ae776bf44b0}{Consulter
cette fiche sur geo.data.gouv.fr}


\vspace{0.5cm}
\needspace{3\baselineskip} \rule{4cm}{0.25pt}\newline\textbf{Aussi disponible du même producteur :}\begin{itemize}
\item \href{https://data.gouv.fr/dataset/5a09ae5fc751df309c2a129f}{Administratif - Arrondissement en Loir-et-Cher}
\item \href{https://data.gouv.fr/dataset/5a09ade8c751df30a64217c0}{Administratif - Communes avec données des découpages territorial en 2016 du Loir-et-Cher}
\item \href{https://data.gouv.fr/dataset/5a09adeac751df309bf95a78}{Administratif - Communes avec données des découpages territorial en 2016 du Loir-et-Cher}
\item \href{https://data.gouv.fr/dataset/5a09ae6b88ee3844cf133d68}{Administratif - Contours IRIS 2015 en Loir-et-Cher}
\item \href{https://data.gouv.fr/dataset/5a09adf7c751df30a64217c1}{Administratif - Contours IRIS en Loir-et-Cher}
\item \href{https://data.gouv.fr/dataset/5a09ade9c751df30a55ef1f6}{Administratif - Découpage des circonscriptions en Loir-et-Cher}
\item \href{https://data.gouv.fr/dataset/5a09ae66c751df309bf95a7f}{Administratif - Les chefs-lieux en 2016 en Loir-et-Cher}
\item \href{https://data.gouv.fr/dataset/558e9d9e88ee383e2b4a22fa}{Administratif - Les chefs-lieux en Loir-et-Cher}
\item \href{https://data.gouv.fr/dataset/5a09ae61c751df30a55ef203}{Administration - Communes avec données des découpages territorial en 2015 de la Région Centre}
\item \href{https://data.gouv.fr/dataset/58aeef95c751df583fef489b}{Agriculture - Demandes d'engagement d'éléments linéaires CAD/CTE en loir-et-Cher}
\item \href{https://data.gouv.fr/dataset/56eee5a588ee381974908574}{Agriculture - Demandes d'engagement d'éléments surfaciques CAD/CTE en loir-et-Cher}
\item \href{https://data.gouv.fr/dataset/5a09adf388ee3846a0a845fa}{Agriculture - Elevages (Bovins, Caprins-Ovins, Porcins, Chevaux et Volailles) en Loir-Et-Cher}
\item \href{https://data.gouv.fr/dataset/58aee86188ee381386e173b1}{Agriculture - Ilots PAC utilisés dans le contrat CAD (programme 2006-2010) en Loir-et-Cher}
\item \href{https://data.gouv.fr/dataset/558e9db6c751df7292a453c2}{Agriculture - Installations classées ICPE en Loir-et-Cher}
\item \href{https://data.gouv.fr/dataset/58aee85488ee381386e173ab}{Agriculture - Parcelle de périmètre d'épandage de matière organique en Loir-et-Cher}
\item \href{https://data.gouv.fr/dataset/558e9db788ee385a1f4a2301}{Agriculture - Petite région agricole (PRA) en Loir-et-Cher}
\item \href{https://data.gouv.fr/dataset/558e9db688ee384a5e4a22fb}{Agriculture - Région Agricole (RA) en Région Centre}
\item \href{https://data.gouv.fr/dataset/5a09ae72c751df30a64217c7}{Aménagement - Périmètre de sauvegarde RGC (Route à grande circulation) en Loir-et-Cher}
\item \href{https://data.gouv.fr/dataset/5a09ae7ac751df30a55ef205}{Aménagement Zone de revitalisation rurale en Loir-et-Cher}
\item \href{https://data.gouv.fr/dataset/5a09ae0f88ee383f43ef7e1c}{Aménagement zones étude - Zone de revitalisation rurale 2015 en Loir-et-Cher}
\item \href{https://data.gouv.fr/dataset/5a09ae7c88ee3844cf133d6a}{Autre - Domaine public fluvial Loire en Loir-et-Cher}
\item \href{https://data.gouv.fr/dataset/558bf48f88ee383f5e4a2328}{Cartographie de l'aléa inondation sur la commune Barcelonne du Gers (Gers)}
\item \href{https://data.gouv.fr/dataset/58aee855c751df4a769e61ba}{Chasse - Massif cynégétique dans le Loir-et-Cher}
\item \href{https://data.gouv.fr/dataset/5a09ae7388ee3844cf133d69}{Culture - Localisation des musées et châteaux en Loir-et-Cher}
\item \href{https://data.gouv.fr/dataset/5a09ae7bc751df309bf95a80}{Culture société - Listes Circonscriptions Education Nationale en Loir-et-Cher}
\item \href{https://data.gouv.fr/dataset/5a09ae0c88ee3844cf133d60}{Culture société - Listes écoles primaires Direction Académique des Services de l'Education Nationale en Loir-et-Cher}
\item \href{https://data.gouv.fr/dataset/5883cd3188ee3850419b81ee}{DDT 70 - Atlas des Zones Inondables (AZI) du SALON}
\item \href{https://data.gouv.fr/dataset/5883cd33c751df558bae0a95}{DDT 70 - Atlas des Zones Inondables (AZI) du secteur de SENARGENT}
\item \href{https://data.gouv.fr/dataset/58aee85f88ee381715aaa7ba}{Eau Assainissement - Points de rejet des STEP en Loir-et-Cher}
\item \href{https://data.gouv.fr/dataset/5a09ae7988ee384693f1b140}{Eau - Assainissement - Stations d'épuration localisées en 2004 - Loir-et-Cher}
\item \href{https://data.gouv.fr/dataset/58aee857c751df4c640ff3c6}{Eau - Assainissement - Stations d'épuration localisées ponctuellement dans le Loir-et-Cher}
\item \href{https://data.gouv.fr/dataset/5a09ae74c751df30a55ef204}{Eau - Classement continuité écologique L214-17 (Liste 1) en Loir-et-Cher}
\item \href{https://data.gouv.fr/dataset/56ec2dc8c751df53e0cc7144}{Eau - Classement continuité écologique L214-17 (Liste 2) en Loir-et-Cher}
\item \href{https://data.gouv.fr/dataset/58aee85e88ee381723db05dd}{Eau - Classement continuité écologique L214-17 (Liste 2) en Loir-et-Cher}
\item \href{https://data.gouv.fr/dataset/5a09ae7dc751df30a64217c8}{Eau - Digue de protection contre les inondations des cours d'eau en Loir-et-Cher}
\item \href{https://data.gouv.fr/dataset/5a09ae1988ee384693f1b13d}{Eau - Disposition 1C-2 SDAGE en Loir-et-Cher}
\item \href{https://data.gouv.fr/dataset/5a09ae1888ee3846a0a845fc}{Eau - Masse d'eau souterraines en Loir-et-Cher}
\item \href{https://data.gouv.fr/dataset/5a09ae82c751df308a7a7390}{Eau - Masse d'eau superficielles en Loir-et-Cher}
\item \href{https://data.gouv.fr/dataset/58aee863c751df4c6f4e4366}{Eau - Ouvertrures de la Digue de la Loire en Loir-et-Cher}
\item \href{https://data.gouv.fr/dataset/5a09ae0f88ee384693f1b13c}{Eau - Ouvrage pour l'alimentation en eau potable (AEP) - Loir-et-Cher}
\item \href{https://data.gouv.fr/dataset/5a09ae0cc751df30a55ef1f8}{EAU - PAOT Programme d'actions opérationnel territorialisé des masses d'eau souterraine en Loir-et-Cher}
\item \href{https://data.gouv.fr/dataset/5a09ae07c751df309c2a129b}{Eau - Périmètres des Schémas d'Aménagement et de Gestion des Eaux en Loir-et-Cher}
\item \href{https://data.gouv.fr/dataset/5a09ae1788ee383f43ef7e1d}{Eau - Plan d'eau supérieur à 1000 m\textsuperscript{2} en Loir-et-Cher}
\item \href{https://data.gouv.fr/dataset/58aee86288ee381715aaa7bc}{Eau - Point de prélèvement d'eau au milieu naturel effectué pour l'irrigation en Loir-et-Cher}
\item \href{https://data.gouv.fr/dataset/5a09ae7cc751df308a7a738f}{Eau - Stations hydrométriques en Loir-et-Cher}
\item \href{https://data.gouv.fr/dataset/5a09ae0b88ee384693f1b13b}{Eau - Surface d'eau BDT étangs en Loir-et-Cher}
\item \href{https://data.gouv.fr/dataset/5a0975fd88ee385e2bce2b74}{EAU - Zones d'actions renforcées ZAR en loir et Cher}
\item \href{https://data.gouv.fr/dataset/5a09ae9088ee384693f1b141}{EAU - Zones d'actions renforcées ZAR en loir et Cher}
\item \href{https://data.gouv.fr/dataset/5a09adffc751df309c2a129a}{Eau - Zone Vulnérable à la Pollution par les Nitrates d'origine agricole en Loir-et-Cher}
\item \href{https://data.gouv.fr/dataset/58aee850c751df4a769e61b6}{Energie - EolienTerrestre en Loir-et-Cher}
\item et 164 autres jeux de données\end{itemize}

\clearpage
\section{Direction Départementale des Territoires de Seine-et-Marne}


\begin{center}
  \includegraphics[width=3cm]{images/orga/2015-06-22_66fee8a76b1e4efa82f19efad4609c5e_logo_DDT-100.png}
\end{center}


Direction Départementale des Territoires de Seine-et-Marne


\vspace{0.5cm}

\needspace{12\baselineskip}
\subsection*{Périmètres affectés aux lieutenants de louveterie sur le département de
Seine-et-Marne
}\index{donnees!ouvertes}\index{economy}\index{louveterie}\index{passerelle!inspire}
  \begin{wrapfigure}{r}{2.5cm}
    \centering
    \qrcode[nolink]{https://data.gouv.fr/dataset/585b979288ee381a893f4e5e}
  \end{wrapfigure}

Licence : \textbf{Licence Ouverte version 2.0
}\newline
Créé le : 2016-12-22\newline
Modifié le : 2019-03-15\newline
Popularité : 1 réutilisation,  0 suivi\newline
Mots-clé : \emph{donnees-ouvertes, economy, louveterie, passerelle-inspire
}\newline
Permalien : \url{https://data.gouv.fr/dataset/585b979288ee381a893f4e5e}\newline

\par
\noindent
    Les lieutenants de louveterie sont des personnes privées, collaborateurs
bénévoles de l'administration et collaborateur occasionnel du service
public. Ils sont nommés par le préfet de département sur proposition du
directeur départemental des territoires et de la mer et sur avis du
président de la Fédération départementale des chasseurs pour une durée
de cinq années renouvelable (art.R 427-2 du code de l'environnement).

Ils doivent être de nationalité française et jouir de leurs droits
civiques, être âgés de moins de 75 ans (décret du 22 septembre 2009),
avoir un permis de chasser depuis au moins cinq ans, posséder la
compétence cynégétique nécessaire pour remplir correctement leurs
fonctions, notamment par leurs connaissances de la vie, des mœurs des
animaux sauvages, de l'équilibre biologique à maintenir et la
législation de la chasse et des règles de sécurité, résider dans le
département où ils sont nommés (ou un canton limitrophe), ne pas avoir
fait l'objet de condamnation pénale en matière de chasse, de pêche et de
protection de la nature.

Ils sont assermentés.

Dans l'exercice de leurs fonctions, les louvetiers doivent être porteurs
de leur commission préfectorale et d'un insigne représentant une tête de
loup dorée avec en exergue une courroie de chasse émaillée bleue portant
l'inscription « lieutenant de louveterie » en doré. Ils s'engagent par
écrit à entretenir, à leurs frais, soit un minimum de quatre chiens
courants réservés à la chasse du sanglier ou du renard soit au moins
deux chiens de déterrage.

Pour la période 2015-2019, il y a dix lieutenants de louveterie pour lé
département de Seine-et-Marne.

\textbf{Origine}

Par agglomération et/ou numérisation des communes et parties de communes
concernées.

\textbf{Organisations partenaires}

DDT Seine-et-Marne

\textbf{Liens annexes}

\begin{itemize}

\item
  \href{http://ogc.geo-ide.developpement-durable.gouv.fr/csw/all-dataset?REQUEST=GetRecordById\&SERVICE=CSW\&VERSION=2.0.2\&RESULTTYPE=results\&elementSetName=full\&TYPENAMES=gmd:MD_Metadata\&OUTPUTSCHEMA\&\#x3D\%5Bhttp://www.isotc211.org/2005/gmd\&ID\&\%5D(http://www.isotc211.org/2005/gmd\&ID\&)x3D;fr-120066022-jdd-eadc870c-d060-492d-a5f6-524dc61de00f}{Vue
  XML des métadonnées}
\item
  \href{http://ogc.geo-ide.developpement-durable.gouv.fr/wxs?map=/opt/data/carto/geoide-catalogue/1.4/org_38094/d12b1d06-aa7e-48c3-91da-659a2b623e03.internet.map}{URL
  de base des services wms/wfs sur internet}
\end{itemize}

➞
\href{https://geo.data.gouv.fr/fr/datasets/acca2e1fde15c0df0c950cd3ba6e8e35ab109be9}{Consulter
cette fiche sur geo.data.gouv.fr}


\vspace{0.5cm}
\needspace{3\baselineskip} \rule{4cm}{0.25pt}\newline\textbf{Aussi disponible du même producteur :}\begin{itemize}
\item \href{https://data.gouv.fr/dataset/585b978f88ee38306b3f4e5d}{Cimetières générateurs de servitudes INT1 en Seine-et-Marne}
\item \href{https://data.gouv.fr/dataset/585b979088ee38452d3f4e5e}{Classement sonore des infrastructures ferroviaires conventionnelles et à grande vitesse en Seine-et-Marne}
\item \href{https://data.gouv.fr/dataset/558bf46b88ee382e004a2303}{Elements paysagers hors PLU du Gers}
\item \href{https://data.gouv.fr/dataset/584f2028c751df3640c0bb80}{Espaces bâtis 2005 du Cantal}
\item \href{https://data.gouv.fr/dataset/585b9792c751df5dfa2b7158}{Etablissement Public de Coopération Intercommunale en Seine-et-Marne (population 2015)}
\item \href{https://data.gouv.fr/dataset/58aef56888ee382d1f2255d6}{Etablissements publics d'aménagement (EPA) en Seine-et-Marne}
\item \href{https://data.gouv.fr/dataset/585b978fc751df4afc2b7154}{Forêts génératrices de servitudes A1 (Abrogée) en Seine-et-Marne}
\item \href{https://data.gouv.fr/dataset/558871d088ee3856924a22ff}{Gares ferroviaires de voyageurs en Seine-et-Marne}
\item \href{https://data.gouv.fr/dataset/585b979288ee38452d3f4e5f}{Groupement d'intérêt cynégétique Faisan en Seine-et-Marne}
\item \href{https://data.gouv.fr/dataset/585b979288ee38306a3f4e5f}{Les Établissements Publics de Coopération Intercommunale (EPCI) à Fiscalité Propre (FP) en vigueur au 01/01/2016 en Seine-et-Marne.}
\item \href{https://data.gouv.fr/dataset/585b9790c751df5dfa2b7157}{Les zones d'aménagement concerté (ZAC) en Seine-et-Marne}
\item \href{https://data.gouv.fr/dataset/585b979288ee38306b3f4e5e}{Ligne électrique à l'usage des Servitudes d'Utilité Publique (SUP I4) en Seine-et-Marne}
\item \href{https://data.gouv.fr/dataset/585b978f88ee381a893f4e5d}{Mesures de classement et d'inscription et protections des abords des monuments historiques en Seine-et-Marne}
\item \href{https://data.gouv.fr/dataset/585b978fc751df03f92b7155}{Mesures de classement et d'inscription et protections des abords des monuments historiques en Seine-et-Marne}
\item \href{https://data.gouv.fr/dataset/585b978fc751df4afd2b7156}{Monuments historiques générateurs de servitudes AC1 en Seine-et-Marne (type surfacique)}
\item \href{https://data.gouv.fr/dataset/558871e488ee3812f34a22ea}{Ouvrages liés aux réseaux d'assainissement en Seine-et-Marne}
\item \href{https://data.gouv.fr/dataset/558871c488ee3869a54a22fb}{Parc locatif social en Seine-et-Marne}
\item \href{https://data.gouv.fr/dataset/58aef56788ee382d101eb43c}{Périmètre des arrondissements au 01/01/2017 pour le département de Seine-et-Marne}
\item \href{https://data.gouv.fr/dataset/558871e7c751df60bea453c5}{Périmètres des diagnostics de territoires en Seine-et-Marne}
\item \href{https://data.gouv.fr/dataset/58aef568c751df6239c29d44}{Périmètres des opérations d'intérêt national (OIN) en Seine-et-Marne}
\item \href{https://data.gouv.fr/dataset/558871c4c751df60bea453ba}{Permis de construire éoliens en Seine-et-Marne}
\item \href{https://data.gouv.fr/dataset/58aef56788ee382d1749c409}{Plan de prévention des risques technologiques en Seine-et-Marne}
\item \href{https://data.gouv.fr/dataset/58aef56888ee382d27a9cf8d}{Plan local d'urbanisme intercommunal - périmètre}
\item \href{https://data.gouv.fr/dataset/558871bf88ee3856924a22fa}{Plans de Prévention des Risques Naturels (PPRN) en Seine-et-Marne}
\item \href{https://data.gouv.fr/dataset/558871d2c751df3f7ca453cc}{Plans Locaux de Déplacements (PLD) en Seine-et-Marne}
\item \href{https://data.gouv.fr/dataset/558871e4c751df3f7ca453d5}{Points de prélèvement au milieu naturel pour l'alimentation en eau potable (AEP) en Seine-et-Marne}
\item \href{https://data.gouv.fr/dataset/558871e688ee3869a54a2306}{Points d'introduction dans le milieu naturel des eaux épurées issues des STEP des collectivités en Seine-et-Marne}
\item \href{https://data.gouv.fr/dataset/585b9790c751df4afc2b7155}{Prescription surfacique d'un PLU ou d'un POS en Seine-et-Marne}
\item \href{https://data.gouv.fr/dataset/558871d7c751df60bea453be}{Réseau des routes à grande circulation en Seine-et-Marne}
\item \href{https://data.gouv.fr/dataset/558871e788ee3869a54a2307}{Réserve de Biosphère en Seine-et-Marne}
\item \href{https://data.gouv.fr/dataset/558871ce88ee384c144a22f8}{Secteur des cartes communales en Seine-et-Marne}
\item \href{https://data.gouv.fr/dataset/585b9792c751df55292b7159}{Servitude de passage dans le lit ou sur les berges de cours d'eau non domaniaux en Seine-et-Marne}
\item \href{https://data.gouv.fr/dataset/585b978f88ee38586f3f4e60}{Servitude de passage dans le lit ou sur les berges de cours d'eau non domaniaux en Seine-et-Marne}
\item \href{https://data.gouv.fr/dataset/585b978f88ee38586f3f4e5f}{Servitude relative aux canalisations publiques d'eau et d'assainissement en Seine-et-Marne}
\item \href{https://data.gouv.fr/dataset/5a04a53088ee3858568f8bf6}{Servitude relative aux installations classées et sites constituant une menace pour la sécurité et la salubrité publique en Seine-et-Marne}
\item \href{https://data.gouv.fr/dataset/558871cdc751df3f6fa453c4}{Servitude relative aux voies ferrées en Seine-et-Marne}
\item \href{https://data.gouv.fr/dataset/5a04a53dc751df3c3184016a}{Servitude relative aux voies ferrées en Seine-et-Marne}
\item \href{https://data.gouv.fr/dataset/558871d788ee384c144a22fa}{Servitudes de protection des centres de réception radioélectriques contre les perturbations électromagnétiques (PT2) en Seine-et-Marne}
\item \href{https://data.gouv.fr/dataset/5a04a52fc751df24a6bcc146}{Servitudes instituées au bénéfice des centres radioélectriques concernant la défense nationale ou la sécurité publique (PT1) en Seine-et-Marne}
\item \href{https://data.gouv.fr/dataset/5a04a53dc751df24a6bcc147}{Sites patrimoniaux remarquables (SPR) en Seine-et-Marne}
\item \href{https://data.gouv.fr/dataset/558871c488ee3856924a22fc}{Zone de développement de l'éolien (ZDE) en Seine-et-Marne}
\item \href{https://data.gouv.fr/dataset/58adbe7488ee382ba0653d23}{Zone réglementée du Plan de Prévention des Risques Inondation (PPRI- 13DDTM20110110) sur la commune d'Orgon - Bouches-du-Rhône}
\item \href{https://data.gouv.fr/dataset/58aef567c751df61fbc74a9e}{Zones agricoles protégées (A9) en Seine-et-Marne}
\item \href{https://data.gouv.fr/dataset/5a04a53c88ee3854b8df0039}{Zones agricoles protégées (A9) en Seine-et-Marne}
\item \href{https://data.gouv.fr/dataset/558871bf88ee3869a54a22f8}{Zones d'aléa des Plans de Prévention des Risques Inondation (PPRI) en Seine-et-Marne}
\item \href{https://data.gouv.fr/dataset/558871c188ee3869a54a22f9}{Zones d'aléa des Plans de Prévention des Risques Mouvement de Terrain (PPRMT) en Seine-et-Marne}
\item \href{https://data.gouv.fr/dataset/558871bfc751df3f7ca453c9}{Zones d'aléa des Plans de Prévention des Risques Technologiques (PPRT) en Seine-et-Marne}
\item \href{https://data.gouv.fr/dataset/558871bfc751df60bea453b9}{Zones d'enjeux prises en compte dans l'établissement des Plans de Prévention des Risques Inondation (PPRI) du département de Seine-et-Marne}
\item \href{https://data.gouv.fr/dataset/558871c988ee3869a54a22fd}{Zones de protection liées aux servitudes de la catégorie EL2 (Défense contre les inondations) en Seine-et-Marne}
\item \href{https://data.gouv.fr/dataset/558871d9c751df3f7aa453c8}{Zones de protection liées aux servitudes de la catégorie PM2 (Installations classées) en Seine-et-Marne}
\item et 8 autres jeux de données\end{itemize}

\clearpage
\section{Direction Départementale des Territoires d'Indre-et-Loire}


\begin{center}
  \includegraphics[width=3cm]{images/orga/2015-07-21_a45d975f46254197a238f47bbde15345_logo_DDT-100.png}
\end{center}


Direction Départementale des Territoires d'Indre-et-Loire


\vspace{0.5cm}

\needspace{12\baselineskip}
\subsection*{Quartier prioritaire de la politique de la ville dans l'Indre-et-Loire
}\index{donnees!ouvertes}\index{passerelle!inspire}\index{society}
  \begin{wrapfigure}{r}{2.5cm}
    \centering
    \qrcode[nolink]{https://data.gouv.fr/dataset/55ae44e6c751df56a110ccbb}
  \end{wrapfigure}

Licence : \textbf{Licence Ouverte version 2.0
}\newline
Créé le : 2015-07-21\newline
Modifié le : 2019-03-15\newline
Popularité : 1 réutilisation,  0 suivi\newline
Mots-clé : \emph{donnees-ouvertes, passerelle-inspire, society
}\newline
Permalien : \url{https://data.gouv.fr/dataset/55ae44e6c751df56a110ccbb}\newline

\par
\noindent
    La loi du 14 novembre 1996 de mise en œuvre du pacte de relance pour la
ville (PRV) distinguait trois niveaux d'intervention : les zones
urbaines sensibles, les zones de redynamisation urbaine (ZRU), les zones
franches urbaines (ZFU). Ces trois niveaux d'intervention ZUS, ZRU et
ZFU, caractérisés par des dispositifs d'importance croissante, visaient
à répondre à des degrés différents de difficultés rencontrées dans ces
quartiers.Depuis, la loi de programmation pour la ville et la cohésion
urbaine du 21 février 2014 a fixé (article 5) les modalités de la
réforme de la géographie prioritaire de la politique de la ville. Deux
décrets pris en 2014 (n\degree{} 2014-767 du 3 juillet 2014 et
n\degree{} 2014-1575 du 22 décembre 2014) ont détaillé, respectivement
pour la métropole et pour les territoires ultramarins, ces modalités.
Ainsi a pu être produite la liste nationale des quartiers prioritaires
de la politique de la ville (décrets n\degree{}2014-1750 et n\degree{}
2014-1751 du 30 décembre 2014) et la cartographie nationale de leurs
périmètres être publiée. Ces périmètres viennent se substituer aux zones
urbaines sensibles (ZUS) et aux quartiers en contrat urbain de cohésion
sociale (CUCS) à compter du 1er janvier 2015.

\textbf{Origine}

Un quartier prioritaire est un espace urbain continu, situé en
territoire urbain. Lorsque la limite d'un quartier correspond à une voie
publique, elle est réputée suivre l'axe central de cette voie.La liste
des quartiers prioritaires de la politique de la ville (QPPV), prévue au
II de l'article 5 de la loi n\degree{} 2014-173 du 21 février 2014,
comprend leur identification et la délimitation de leurs contours,
déterminée selon les modalités précisées par les décrets n\degree{}
2014-767 du 3 juillet 2014 (métropole) et n\degree{} 2014-1575 du 22
décembre 2014 (outre-mer). Les délimitations des quartiers concernés
sont consultables et téléchargeables auprès du Commissariat général à
l'égalité des territoires, 5, rue Pleyel, 93200 Saint-Denis
(www.ville.gouv.fr) et sur le Géoportail (www.geoportail.gouv.fr). La
liste, établie par décret, est actualisée dans l'année du renouvellement
général des conseils municipaux, ou, si la rapidité des évolutions
observées le justifie, tous les trois ans.

\textbf{Organisations partenaires}

DDT Indre-et-Loire

\textbf{Liens annexes}

\begin{itemize}

\item
  \href{http://ogc.geo-ide.developpement-durable.gouv.fr/csw/all-dataset?REQUEST=GetRecordById\&SERVICE=CSW\&VERSION=2.0.2\&RESULTTYPE=results\&elementSetName=full\&TYPENAMES=gmd:MD_Metadata\&OUTPUTSCHEMA\&\#x3D\%5Bhttp://www.isotc211.org/2005/gmd\&ID\&\%5D(http://www.isotc211.org/2005/gmd\&ID\&)x3D;fr-120066022-jdd-08c9e226-fdbc-4c72-ae79-99217a7d5386}{Vue
  XML des métadonnées}
\item
  \href{http://geostandards.developpement-durable.gouv.fr/afficherPageStandard.do?jeu=N_QPPV_ZINF}{Standard
  de données COVADIS : Quartier prioritaire de la politique de la ville}
\item
  \href{http://ogc.geo-ide.developpement-durable.gouv.fr/wxs?map=/opt/data/carto/geoide-catalogue/1.4/org_38016/5ea55c50-e955-4668-b4a2-ef0de01f280f.internet.map}{URL
  de base des services wms/wfs sur internet}
\end{itemize}

➞
\href{https://geo.data.gouv.fr/fr/datasets/66403587e9aa677d60be0296373c7875d43e3fb1}{Consulter
cette fiche sur geo.data.gouv.fr}


\vspace{0.5cm}
\needspace{3\baselineskip} \rule{4cm}{0.25pt}\newline\textbf{Aussi disponible du même producteur :}\begin{itemize}
\item \href{https://data.gouv.fr/dataset/55ae44cec751df786d10ccb6}{Aire de protection de biotope dans l'Indre-et-Loire}
\item \href{https://data.gouv.fr/dataset/55ae44d288ee38443d3ca28c}{Centres d'examen des permis de conduire dans l'Indre-et-Loire}
\item \href{https://data.gouv.fr/dataset/5a04a4e9c751df24a6bcc143}{Communes couvertes par un document d'urbanisme opposable et numérisé à la date de saisie dans l'Indre-et-Loire.}
\item \href{https://data.gouv.fr/dataset/55ae44d3c751df56a110ccb8}{DDT 70 - Atlas des Zones Inondables (AZI) de LA MANCE}
\item \href{https://data.gouv.fr/dataset/5883cd33c751df5ab4ae0ab3}{DDT 70 - Atlas des Zones Inondables (AZI) du CONEY}
\item \href{https://data.gouv.fr/dataset/5883cd3088ee386def9b81e1}{DDT 70 - Périmètre du PPRI n\degree{}20060010 du Durgeon Aval.}
\item \href{https://data.gouv.fr/dataset/5883cd2ec751df6aa8ae0aaf}{DDT 70 - Périmètre du PPRI n\degree{}20080023 du Bassin de la Semouse}
\item \href{https://data.gouv.fr/dataset/5ab9750788ee3862ea8f845c}{Digue de protection contre les inondations et submersion  et digue de canaux et de rivières canalisées dans l'Indre-et-Loire}
\item \href{https://data.gouv.fr/dataset/55ae44e188ee3836cb3ca28c}{Documents d'urbanisme carte communale existant sur le département sous forme numérique dans l'Indre-et-Loire}
\item \href{https://data.gouv.fr/dataset/55ae44de88ee38443d3ca28e}{Documents d'urbanisme PLU,  POS, carte communale existant sur le département sous forme numérique dans l'Indre-et-Loire}
\item \href{https://data.gouv.fr/dataset/58484dc5c751df51c0c0bb7e}{Elément linéaire engagé dans une mesure agro-environnementale (MAE) pour la campagne 2012 anonymisé dans l'Indre-et-Loire}
\item \href{https://data.gouv.fr/dataset/5883cd2f88ee3810d19b81cf}{Enjeu linéaire du Plan de Prévention des risques naturels prévisibles d'inondation de la Loire sur le Val de Bréhémont - Langeais dans l'Indre-et-Loire}
\item \href{https://data.gouv.fr/dataset/58aee44a88ee380f955b08c6}{Enjeu linéaire du Plan de Prévention des risques naturels prévisibles d'inondation de la Loire sur le Val de Tours}
\item \href{https://data.gouv.fr/dataset/55ae44d388ee3836cb3ca289}{Enjeu linéaire du Plan de Prévention des risques naturels prévisibles d'inondation de la Vallée de l'Indre dans l'Indre-et-Loire}
\item \href{https://data.gouv.fr/dataset/58484c5d88ee38220fc65bb3}{Enjeu linéaire du Plan de Prévention des Risques Technologiques sur le site EPC à Cigogné en Indre-et-Loire}
\item \href{https://data.gouv.fr/dataset/58484c5b88ee38220dc65bb4}{Enjeu linéaire du Plan de Prévention des Risques Technologiques sur le site SYNTHRON en Indre-et-Loire}
\item \href{https://data.gouv.fr/dataset/55ae44d3c751df56b510ccb8}{Enjeu ponctuel du Plan de Prévention des risques naturels prévisibles d'inondation de la Loire sur le Val de Cisse dans l'Indre-et-Loire}
\item \href{https://data.gouv.fr/dataset/55ae44cd88ee3824833ca288}{Enjeu ponctuel  du Plan de Prévention des risques naturels prévisibles d'inondation de la Vallée de l'Indre dans l'Indre-et-Loire}
\item \href{https://data.gouv.fr/dataset/55ae44d888ee381da83ca28d}{Enjeu ponctuel du Plan de Prévention des risques naturels prévisibles d'inondation du Val du Cher dans l'Indre-et-Loire}
\item \href{https://data.gouv.fr/dataset/58484c5dc751df4f7ac0bb7f}{Enjeu ponctuel du Plan de Prévention des Risques Technologiques sur le site EPC à Cigogné en Indre-et-Loire}
\item \href{https://data.gouv.fr/dataset/58484c5bc751df4f7cc0bb7f}{Enjeu ponctuel du Plan de Prévention des Risques Technologiques sur le site SYNTHRON en Indre-et-Loire}
\item \href{https://data.gouv.fr/dataset/58484c5988ee38220ac65bb3}{Enjeu surfacique du Plan de Prévention des Risques Technologiques sur le site SYNTHRON en Indre-et-Loire}
\item \href{https://data.gouv.fr/dataset/58484c5b88ee38220cc65bb4}{Entité ponctuelle à l'origine du risque du Plan de Prévention des Risques Technologiques sur le site SYNTHRON en Indre-et-Loire}
\item \href{https://data.gouv.fr/dataset/58aee44c88ee380f955b08c7}{Entité surfacique à l'origine du risque du Plan de Prévention des Risques Technologiques sur le site SYNTHRON en Indre-et-Loire}
\item \href{https://data.gouv.fr/dataset/58484c59c751df4f77c0bb7e}{Etablissement Public de Coopération Intercommunale dans l'Indre-et-Loire}
\item \href{https://data.gouv.fr/dataset/56ec308588ee381a3ae1a63e}{Générateurs de servitudes EL3 (halage et marchepied) en Charente}
\item \href{https://data.gouv.fr/dataset/56ec3176c751df5e92cc7140}{HAMSTER - Terriers de 2002 en Alsace}
\item \href{https://data.gouv.fr/dataset/5a04a4edc751df3c31840166}{Information ponctuelle des documents d'urbanisme dans l'Indre-et-Loire}
\item \href{https://data.gouv.fr/dataset/58484c59c751df4f7ac0bb7e}{Installations classées, entité ponctuelle à l'origine d'un risque dans l'Indre-et-Loire}
\item \href{https://data.gouv.fr/dataset/58484c5ac751df4f7bc0bb7e}{La base permanente des équipements (BPE) en 2013 dans l'Indre-et-Loire}
\item \href{https://data.gouv.fr/dataset/58484c5ac751df4f79c0bb7e}{Linéaire ajouté au plan de zonage d'un document d'urbanisme PLU (ou POS) à titre d'information dans l'Indre-et-Loire}
\item \href{https://data.gouv.fr/dataset/5a04a4e888ee3854862d2ed6}{Linéaire informatif des documents d'urbanisme dans l'Indre-et-Loire}
\item \href{https://data.gouv.fr/dataset/55ae44e688ee3845f13ca28d}{Liste des plans de prévention des risques technologiques du département d'Indre-et-Loire}
\item \href{https://data.gouv.fr/dataset/58484c5d88ee38220cc65bb5}{Périmètre ajouté au plan de zonage d'un document d'urbanisme PLU (ou POS) à titre d'information dans l'Indre-et-Loire}
\item \href{https://data.gouv.fr/dataset/55ae44d288ee381da83ca28c}{Périmètre du Plan de Prévention des risques naturels prévisibles d'inondation de la Loire sur le Val d'Authion dans l'Indre-et-Loire}
\item \href{https://data.gouv.fr/dataset/58484c5888ee38220bc65bb3}{Périmètre du Plan de Prévention des Risques Technologiques sur le site SYNTHRON en Indre-et-Loire}
\item \href{https://data.gouv.fr/dataset/5a04a4f188ee3854b8df0033}{Périmètre informatif des documents d'urbanisme dans l'Indre-et-Loire}
\item \href{https://data.gouv.fr/dataset/56ec2da688ee380c03e1a666}{Périmètres des pays dans l'Indre-et-Loire}
\item \href{https://data.gouv.fr/dataset/5ab96eb1c751df15e96b32b9}{Périmètres des pays dans l'Indre-et-Loire}
\item \href{https://data.gouv.fr/dataset/58aee44a88ee380fa2fac5d8}{Périmètres du Plan de Prévention des risques naturels prévisibles d'inondation de la Loire sur le Val de Tours - Val de Luynes 2016 dans l'Indre-et-Loire}
\item \href{https://data.gouv.fr/dataset/58484dc688ee382456c65bb3}{Plan de prévention des risques naturels prévisibles d'inondation de la Loire du Val d'Authion dans l'Indre-et-Loire}
\item \href{https://data.gouv.fr/dataset/5a04a4f7c751df3c128bfc72}{Plan de prévention des risques naturels prévisibles d'inondation de la Loire du Val de Bréhémont - Langeais dans l'Indre-et-Loire}
\item \href{https://data.gouv.fr/dataset/58aee44bc751df445d5e0ecd}{Plan de prévention des risques naturels prévisibles d'inondation de la Loire du Val de Tours - Val de Luynes dans l'Indre-et-Loire en 2016}
\item \href{https://data.gouv.fr/dataset/5a04a4f888ee384ef16115e1}{Plan de prévention des risques naturels prévisibles d'inondation du Val de Vienne dans l'Indre-et-Loire}
\item \href{https://data.gouv.fr/dataset/58484dc4c751df51bfc0bb7e}{Plan de prévention des risques naturels prévisibles d'inondation du Val du Cher dans l'Indre-et-Loire}
\item \href{https://data.gouv.fr/dataset/55f281fcc751df532c1f92b2}{Plan de prévention des risques naturels prévisibles d'inondation du Val du Cher dans l'Indre-et-Loire}
\item \href{https://data.gouv.fr/dataset/58484c5dc751df4f7cc0bb80}{Point ajouté au plan de zonage d'un document d'urbanisme PLU (ou POS) à titre d'information dans l'Indre-et-Loire}
\item \href{https://data.gouv.fr/dataset/58484dc588ee382454c65bb3}{Prescription Nationale pour les PLU, POS dans l'Indre-et-Loire}
\item \href{https://data.gouv.fr/dataset/55ae44cdc751df56b510ccb7}{Sites d'Interet Communautaire (SIC) Natura 2000 dans l'Indre-et-Loire}
\item \href{https://data.gouv.fr/dataset/55f1abd2c751df226d1f92e7}{Sites d'Interet Communautaire (SIC) Natura 2000 en France}
\item et 20 autres jeux de données\end{itemize}

\clearpage
\section{Direction Départementale des Territoires du Cher}


\begin{center}
  \includegraphics[width=3cm]{images/orga/51_0102fba7ec46be99d5efe903c26aa1-100.png}
\end{center}


Direction Départementale des Territoires du Cher


\vspace{0.5cm}

\needspace{12\baselineskip}
\subsection*{PLU (doc. du 25.02.2010) de la commune d'Aubigny-sur-Nère
}\index{donnees!ouvertes}\index{passerelle!inspire}\index{planning!cadastre}\index{usage!des!sols}
  \begin{wrapfigure}{r}{2.5cm}
    \centering
    \qrcode[nolink]{https://data.gouv.fr/dataset/56eedce2c751df414ad6e93b}
  \end{wrapfigure}

Licence : \textbf{Licence Ouverte version 2.0
}\newline
Créé le : 2016-03-20\newline
Modifié le : 2019-03-10\newline
Popularité : 1 réutilisation,  0 suivi\newline
Mots-clé : \emph{donnees-ouvertes, passerelle-inspire, planning-cadastre, usage-des-sols
}\newline
Permalien : \url{https://data.gouv.fr/dataset/56eedce2c751df414ad6e93b}\newline

\par
\noindent
    Le présent standard de données COVADIS concerne les documents de plans
locaux d'urbanisme (PLU) et les plans d'occupation des sols (POS qui
valent PLU). Ce standard de données offre un cadre technique décrivant
en détail la façon de dématérialiser ces documents d'urbanisme en une
base de données géographiques qui soit exploitable par un outil SIG et
interopérable. Ce standard de données concerne aussi bien les plans
graphiques de zonage, les prescriptions s'y superposant que les
règlements s'appliquant à chaque type de zone.Ce standard de données
COVADIS a été élaboré à partir du cahier des charges pour la
dématérialisation des documents d'urbanisme mis à jour en 2012 par le
CNIG, lui même basé sur la version consolidée du code de l'urbanisme en
date du 16 mars 2012. Les recommandations de ces deux documents sont
cohérentes même si leur objet n'est pas le même. Le standard de données
COVADIS propose des définitions et une structure pour organiser et
ranger dans une infrastructure les données géographiques de PLU/POS
existant sous forme numérique tandis que le cahier des charges du CNIG
sert à encadrer la numérisation de ces données. La partie `Structure des
données' présentée dans ce standard COVADIS donne des recommandations
complémentaires en matière de stockage des fichiers de données
(cf.~partie C). Il s'agit de choix spécifiques à l'infrastructure de
données du MAA et du MEDDE qui ne s'appliquent pas en dehors de leur
contexte.Les cartes communales font l'objet d'un autre standard de
données COVADIS.

\textbf{Origine}

Données issues de la numérisation des documents graphiques du PLU. Les
limites d'une zone sont numérisées sur le découpage cadastral de la
commune. La géométrie des parcelles cadastrales est une donnée issue du
référentiel géographique cadastral choisi au moment de l'élaboration
(plan cadastral informatisé (PCI) fourni par la DGI).

\textbf{Organisations partenaires}

DDT Cher

\textbf{Liens annexes}

\begin{itemize}

\item
  \href{http://ogc.geo-ide.developpement-durable.gouv.fr/csw/all-dataset?REQUEST=GetRecordById\&SERVICE=CSW\&VERSION=2.0.2\&RESULTTYPE=results\&elementSetName=full\&TYPENAMES=gmd:MD_Metadata\&OUTPUTSCHEMA\&\#x3D\%5Bhttp://www.isotc211.org/2005/gmd\&ID\&\%5D(http://www.isotc211.org/2005/gmd\&ID\&)x3D;fr-120066022-ldd-a1742d48-c282-4c6e-947a-4ac54d89c12d}{Vue
  XML des métadonnées}
\item
  \href{http://geostandards.developpement-durable.gouv.fr/afficherPageStandard.do?lot=Plan-local-d-urbanisme-v2.0}{Standard
  de données COVADIS : Plan local d'urbanisme v2.0}
\item
  \href{http://ogc.geo-ide.developpement-durable.gouv.fr/wxs?map=/opt/data/carto/geoide-catalogue/1.4/org_37976/fr-120066022-orphan-2f8a0a2e-8d75-4934-ad02-2322229033cd.internet.map}{URL
  de base des services wms/wfs sur internet}
\end{itemize}

➞
\href{https://geo.data.gouv.fr/fr/datasets/5e66cef445517658c7885dfcc55667f9d2108bb5}{Consulter
cette fiche sur geo.data.gouv.fr}


\vspace{0.5cm}
\needspace{3\baselineskip} \rule{4cm}{0.25pt}\newline\textbf{Aussi disponible du même producteur :}\begin{itemize}
\item \href{https://data.gouv.fr/dataset/56eed58dc751df33e4d6e93d}{Aires de protection de biotope de type ponctuel sur le département du Cher}
\item \href{https://data.gouv.fr/dataset/584f20c188ee380411c65bb4}{Aires de protection de biotope sur le département du Cher}
\item \href{https://data.gouv.fr/dataset/56eed59a88ee387a99908575}{Aires géographiques des appelations d'origine protégée de type fromage sur le département du Cher}
\item \href{https://data.gouv.fr/dataset/584f2027c751df363ec0bb7f}{Bassins de vie du Cantal}
\item \href{https://data.gouv.fr/dataset/584f202fc751df3644c0bb7f}{Carte de bruit stratégique - jour – voies des collectivités locales dans le Cantal (Type C)}
\item \href{https://data.gouv.fr/dataset/584f202cc751df3640c0bb83}{Carte de bruit stratégique - jour – voies des collectivités locales dans le Cantal (Type D)}
\item \href{https://data.gouv.fr/dataset/584f203188ee380223c65bb6}{Carte de bruit stratégique - nuit – voies de l’État dans le Cantal (Type C)}
\item \href{https://data.gouv.fr/dataset/56eed88e88ee38020a908575}{Carte de bruit stratégique - RN122 - nuit – voies de l’État dans le Cantal (Type A)}
\item \href{https://data.gouv.fr/dataset/56eedcc8c751df4139d6e93b}{Catégories piscicoles des cours d'eau et plans d'eau sur le département du Cher}
\item \href{https://data.gouv.fr/dataset/584f2026c751df363fc0bb7f}{Contours des bassins de vie du Cantal}
\item \href{https://data.gouv.fr/dataset/56eed5a2c751df33eed6e93b}{Cours d'eau pour la conditionnalité sur le département du Cher}
\item \href{https://data.gouv.fr/dataset/5a0ae96d88ee383d117b7c9f}{Cours d’eau prioritaires pour la protection des poissons migrateurs amphihalins avec échéance supérieure à 2021 sur le bassin Rhin-Meuse}
\item \href{https://data.gouv.fr/dataset/5883cd2f88ee3810b49b81e4}{DDT 70- Zone réglementaire PPRI n\degree{}20080025 CC Val de Semouse}
\item \href{https://data.gouv.fr/dataset/584f20c2c751df382cc0bb7e}{Documents d'urbanisme PLU, POS, carte communale existant sur le département sous forme numérique}
\item \href{https://data.gouv.fr/dataset/56eed5a188ee387aa4908574}{Emplacements des sites classés de type surfacique sur le département du Cher}
\item \href{https://data.gouv.fr/dataset/56eedcc988ee3808e5908574}{Emplacements des sites inscrits de type ponctuel sur la région Centre}
\item \href{https://data.gouv.fr/dataset/56eedca488ee3808d7908574}{Emplacements des sites inscrits de type ponctuel sur le département du Cher}
\item \href{https://data.gouv.fr/dataset/56eedca5c751df4128d6e93b}{Emplacements des sites inscrits de type surfacique sur la région Centre}
\item \href{https://data.gouv.fr/dataset/56eedcc1c751df4137d6e93b}{Emprises des plan d'eau douce sur le département du Cher}
\item \href{https://data.gouv.fr/dataset/558bf47088ee382e004a2307}{Ensemble des communes concernées par un contrat de rivière du Gers}
\item \href{https://data.gouv.fr/dataset/56eedcac88ee3808d9908574}{Espaces à dominante ubaine (aires urbaines) sur le département du Cher}
\item \href{https://data.gouv.fr/dataset/558a820488ee38069a4a22f4}{Établissement Public de Coopération Intercommunale (EPCI) au 01/01/2015 sur le Bassin Rhin-Meuse (GEOFLA)}
\item \href{https://data.gouv.fr/dataset/56eedd21c751df430ed6e93b}{Etiquette habillant le plan de la carte communale (doc. du 07.03.2008 ) de la commune de Meillant}
\item \href{https://data.gouv.fr/dataset/56eedd0a88ee380aa8908574}{Etiquette habillant le plan du PLU (doc. du 28-12-2014 ) de la commune de Charenton-du-Cher}
\item \href{https://data.gouv.fr/dataset/584f20c3c751df382ac0bb80}{Etiquette habillant le plan du POS (doc. du 15.01.2002) de la commune de Pigny}
\item \href{https://data.gouv.fr/dataset/56eed916c751df3a5cd6e93c}{Etiquette habillant le plan du POS (doc. du 24.06.2009) de la commune de Saint-Satur}
\item \href{https://data.gouv.fr/dataset/56eed82e88ee387f0d908574}{Habillages linéaires du PLU (doc. du 03.12.2012) de la commune d'Ivoy-le-Pré}
\item \href{https://data.gouv.fr/dataset/56eedc2388ee380892908574}{Habillages linéaires du PLU (doc. du 05.02.2004) de la commune de Sainte-Solange}
\item \href{https://data.gouv.fr/dataset/56eed97a88ee3802b2908576}{Habillages linéaires du PLU (doc. du 06.02.2008) de la commune de Sainte-Thorette}
\item \href{https://data.gouv.fr/dataset/56eed8f0c751df3a4ad6e93d}{Habillages linéaires du PLU (doc. du 08.09.2009) de la commune de Saint-Florent-sur-Cher}
\item \href{https://data.gouv.fr/dataset/56eed64588ee387c83908574}{Habillages linéaires du PLU (doc. du 12.10.2011) de la commune de Neuvy-sur-Barangeon}
\item \href{https://data.gouv.fr/dataset/56eedd5ac751df4331d6e93b}{Habillages linéaires du PLU (doc. du 13.05.2011) de la commune de Morthomiers}
\item \href{https://data.gouv.fr/dataset/56eed96688ee38027c908575}{Habillages linéaires du PLU (doc. du 13.06.2012) de la commune de Thénioux}
\item \href{https://data.gouv.fr/dataset/56eed945c751df3a6dd6e93d}{Habillages linéaires du PLU (doc. du 14.09.2010) de la commune de Saint-Martin-d'Auxigny}
\item \href{https://data.gouv.fr/dataset/56eed9a688ee380436908574}{Habillages linéaires du PLU (doc. du 14.12.2010) de la commune de Trouy}
\item \href{https://data.gouv.fr/dataset/56eed8a788ee380213908574}{Habillages linéaires du PLU (doc. du 15.06.2015) de la commune de Mehun-sur-Yèvre}
\item \href{https://data.gouv.fr/dataset/56eedd3488ee380abb908574}{Habillages linéaires du PLU (doc. du 16.06.2011) de la commune de Saint-Georges-sur-Moulon}
\item \href{https://data.gouv.fr/dataset/56eedd3688ee380abe908574}{Habillages linéaires du PLU (doc. du 17.06.2013) de la commune de La Groutte}
\item \href{https://data.gouv.fr/dataset/56eed94188ee380269908574}{Habillages linéaires du PLU (doc. du 18.06.2010) de la commune de Saint-Laurent}
\item \href{https://data.gouv.fr/dataset/56eed88088ee3801fb908575}{Habillages linéaires du PLU (doc. du 18.11.2005) de la commune de Quincy}
\item \href{https://data.gouv.fr/dataset/56eed8acc751df3a2bd6e93c}{Habillages linéaires du PLU (doc. du 18.12.2008) de la commune de Saint-Hilaire-de-Court}
\item \href{https://data.gouv.fr/dataset/56eed904c751df3a58d6e93b}{Habillages linéaires du PLU (doc. du 20.10.2005) de la commune de Nançay}
\item \href{https://data.gouv.fr/dataset/56eed87488ee380204908574}{Habillages linéaires du PLU (doc. du 22.01.2010) de la commune de Levet}
\item \href{https://data.gouv.fr/dataset/56eed8ef88ee380241908574}{Habillages linéaires du PLU (doc. du 24.10.2006) de la commune de Massay}
\item \href{https://data.gouv.fr/dataset/56eed888c751df3a1cd6e93c}{Habillages linéaires du PLU (doc. du 26.02.2010) de la commune de Preuilly}
\item \href{https://data.gouv.fr/dataset/56eedd7888ee380ad6908576}{Habillages linéaires du PLU (doc. du 26.04.2011) de la commune d'Oizon}
\item \href{https://data.gouv.fr/dataset/56eed85188ee3801f0908574}{Habillages linéaires du PLU (doc. du 28.02.2011) de la commune Mehun-sur-Yèvre}
\item \href{https://data.gouv.fr/dataset/56eed7d4c751df3825d6e93b}{Habillages linéaires du PLU (doc. du 28.12.2014) de la commune de Charenton-du-Cher}
\item \href{https://data.gouv.fr/dataset/56eed96388ee38027a908574}{Habillages linéaires du PLU (doc. du 29.07.2010) de la commune de Thénioux}
\item \href{https://data.gouv.fr/dataset/56eed8c788ee380221908575}{Habillages linéaires du PLU (doc. du 29.08.1987) de la commune d'Herry}
\item et 704 autres jeux de données\end{itemize}

\clearpage
\section{Direction Départementale des Territoires et de la Mer de la Somme}


\begin{center}
  \includegraphics[width=3cm]{images/orga/a6_356cda9fee45ca8ba96dac7ce097bb-100.png}
\end{center}


Direction Départementale des Territoires et de la Mer de la Somme


\vspace{0.5cm}

\needspace{12\baselineskip}
\subsection*{Démographie des communes 2013 selon INSEE dans la Somme
}\index{donnee!generique!demographie}\index{donnees!ouvertes}\index{passerelle!inspire}\index{society}
  \begin{wrapfigure}{r}{2.5cm}
    \centering
    \qrcode[nolink]{https://data.gouv.fr/dataset/56ec1672c751df2813cc716d}
  \end{wrapfigure}

Licence : \textbf{Licence Ouverte version 2.0
}\newline
Créé le : 2016-03-18\newline
Modifié le : 2019-03-11\newline
Popularité : 1 réutilisation,  0 suivi\newline
Mots-clé : \emph{donnee-generique-demographie, donnees-ouvertes, passerelle-inspire, society
}\newline
Permalien : \url{https://data.gouv.fr/dataset/56ec1672c751df2813cc716d}\newline

\par
\noindent
    Recensement de la population par commune pour 2013

\textbf{Origine}

les données attributaires sont issues des données INSEE géolocalisées
avec la BD CARTO.

\textbf{Organisations partenaires}

DDTM Somme

\textbf{Liens annexes}

\begin{itemize}

\item
  \href{http://ogc.geo-ide.developpement-durable.gouv.fr/csw/all-dataset?REQUEST=GetRecordById\&SERVICE=CSW\&VERSION=2.0.2\&RESULTTYPE=results\&elementSetName=full\&TYPENAMES=gmd:MD_Metadata\&OUTPUTSCHEMA\&\#x3D\%5Bhttp://www.isotc211.org/2005/gmd\&ID\&\%5D(http://www.isotc211.org/2005/gmd\&ID\&)x3D;fr-120066022-jdd-43872b3a-5bed-48de-b08d-b654347a5d76}{Vue
  XML des métadonnées}
\item
  \href{http://ogc.geo-ide.developpement-durable.gouv.fr/wxs?map=/opt/data/carto/geoide-catalogue/1.4/org_38100/b47f7e5b-acd8-41cf-9acf-ca3c08a16449.internet.map}{URL
  de base des services wms/wfs sur internet}
\end{itemize}

➞
\href{https://geo.data.gouv.fr/fr/datasets/d01135230d3466c9d3279c24a54423b648d784ae}{Consulter
cette fiche sur geo.data.gouv.fr}


\vspace{0.5cm}
\needspace{3\baselineskip} \rule{4cm}{0.25pt}\newline\textbf{Aussi disponible du même producteur :}\begin{itemize}
\item \href{https://data.gouv.fr/dataset/558a8535c751df039da453c5}{Aire urbaine issu de Unité urbaine selon l'INSEE 2008 dans la Somme}
\item \href{https://data.gouv.fr/dataset/56ec142088ee380c03e1a627}{Alignement des voies publiques générateurs de servitudes EL7 dans la Somme}
\item \href{https://data.gouv.fr/dataset/558a8526c751df6614a453c2}{Amers, phares et sémaphores générateurs de servitudes EL8 dans la Somme}
\item \href{https://data.gouv.fr/dataset/5a04a51488ee3858568f8bf3}{Ancien secteur d'irrigation collective par bassin versant dans la Somme - Secteur de sécheresse de 2015}
\item \href{https://data.gouv.fr/dataset/56ec1429c751df27f7cc7143}{Bande de construction de 100 mètres sur le littoral d'espace remarquable au titre du L.146-6 dans la Somme}
\item \href{https://data.gouv.fr/dataset/558a8532c751df039da453c4}{Bassin d'emploi 2008 dans la Somme}
\item \href{https://data.gouv.fr/dataset/55886ea488ee384c144a22ea}{Bassin d'emploi 2008 dans la Somme}
\item \href{https://data.gouv.fr/dataset/558a8514c751df039da453bf}{Borne du Domaine public Maritime dans la Somme}
\item \href{https://data.gouv.fr/dataset/558a8508c751df760ba453cb}{Borne repère de la laisse de crue de 2001 du PPRi de la vallée de la Somme et de ses affluents}
\item \href{https://data.gouv.fr/dataset/558a854488ee3871154a231d}{Canalisations d'eau et d'assainissement générateurs de servitudes A5 dans la Somme}
\item \href{https://data.gouv.fr/dataset/56ec31bbc751df5e98cc7149}{Carte à enjeux PNA - Crapaud vert en Alsace}
\item \href{https://data.gouv.fr/dataset/56ec317cc751df5e92cc7141}{Carte à enjeux PNA - Pélobate brun en Alsace}
\item \href{https://data.gouv.fr/dataset/5883cbc688ee3850419b81c9}{Cartographie de l'aléa inondation sur la commune de Vic Fezenzsac (Gers)}
\item \href{https://data.gouv.fr/dataset/558a853ec751df75e4a453cc}{Centres de réceptions radioélectriques générateurs de servitudes PT1 dans la Somme}
\item \href{https://data.gouv.fr/dataset/559c0290c751df4853390bd6}{Centres de réceptions radioélectriques générateurs de servitudes PT1 dans la Somme}
\item \href{https://data.gouv.fr/dataset/5a04a51dc751df3ded0d1da1}{Centre urbain de la zone réglementée du PPRn des Bas Champs du Sud de la Baie Somme}
\item \href{https://data.gouv.fr/dataset/558a8553c751df039da453cc}{Communes des établissements Public de Coopération Intercommunales dans la Somme}
\item \href{https://data.gouv.fr/dataset/558a853e88ee3871154a231c}{Conseillers généraux des cantons de la Somme}
\item \href{https://data.gouv.fr/dataset/558a853288ee3830734a22f0}{Conseillers régionaux de la Somme}
\item \href{https://data.gouv.fr/dataset/5a04a516c751df3c31840167}{Cours d'eau linéaire des points d'eau sur l'utilisation des produits phytopharmaceutiques et adjuvants}
\item \href{https://data.gouv.fr/dataset/584187a888ee382f78c65bb5}{Cours d'eau non domanial soumis à la loi sur l'eau dans la Somme}
\item \href{https://data.gouv.fr/dataset/558a8508c751df75e4a453c6}{Cours d'eau pour la conditionnalité au titre des bonnes conditions agricoles et environnementales dans la Somme}
\item \href{https://data.gouv.fr/dataset/558a8518c751df039da453c0}{Démographie des communes 2006 selon INSEE dans la Somme}
\item \href{https://data.gouv.fr/dataset/558a854a88ee3871154a231f}{Démographie des communes 2007 selon INSEE dans la Somme}
\item \href{https://data.gouv.fr/dataset/558a853bc751df5f10a453c2}{Démographie des communes 2008 selon INSEE dans la Somme}
\item \href{https://data.gouv.fr/dataset/558a853b88ee38069a4a2302}{Démographie des communes 2009 selon INSEE dans la Somme}
\item \href{https://data.gouv.fr/dataset/558a853dc751df5f10a453c3}{Démographie des communes 2011 selon INSEE dans la Somme}
\item \href{https://data.gouv.fr/dataset/58aedc1c88ee38021afdb08e}{Démographie des communes 2012 selon INSEE dans la Somme}
\item \href{https://data.gouv.fr/dataset/558a853bc751df039da453c7}{Démographie infracommunal en IRIS des communes 2007 selon INSEE dans la Somme}
\item \href{https://data.gouv.fr/dataset/56ec1463c751df27f7cc7149}{Démographie infracommunal en IRIS des communes 2014 dans la Somme}
\item \href{https://data.gouv.fr/dataset/558a8514c751df760ba453cf}{Députés de la Somme}
\item \href{https://data.gouv.fr/dataset/56ec146088ee380c0ce1a62b}{Document PPRT sur le département de la Somme}
\item \href{https://data.gouv.fr/dataset/558a852388ee3871154a2319}{Enceintes de site liées aux servitudes de la catégorie AC2 (Sites inscrits et classés) dans la Somme}
\item \href{https://data.gouv.fr/dataset/56ec309988ee381a3ae1a641}{Enceintes de site liées aux servitudes de la catégorie AC2 (Sites inscrits et classés) en Charente}
\item \href{https://data.gouv.fr/dataset/558a850ec751df039da453bd}{Enjeu linéaire d'un PPRT de Mesnil Saint Nicaise et de Nesle}
\item \href{https://data.gouv.fr/dataset/56ec1418c751df2813cc7145}{Enjeu linéaire du PPR Falaises Picardes}
\item \href{https://data.gouv.fr/dataset/558a854ac751df760ba453da}{Enjeu linéaire du PPRm de Montdidier}
\item \href{https://data.gouv.fr/dataset/584187a688ee382f76c65bb3}{Enjeu linéaire du PPRn du Marquenterre Baie de Somme}
\item \href{https://data.gouv.fr/dataset/558a852dc751df760ba453d2}{Enjeu ponctuel d'un PPRT de Mesnil Saint Nicaise et de Nesle}
\item \href{https://data.gouv.fr/dataset/56ec141fc751df73c1cc715f}{Enjeu ponctuel du PPR Falaises Picardes}
\item \href{https://data.gouv.fr/dataset/558a8520c751df75e4a453ca}{Enjeu ponctuel du PPRI de la vallée de la Somme}
\item \href{https://data.gouv.fr/dataset/558a8529c751df5f10a453c0}{Enjeu ponctuel du PPRI des cantons de Chaulnes et Bray sur somme}
\item \href{https://data.gouv.fr/dataset/584187a6c751df4f5bc0bb7f}{Enjeu ponctuel du PPRn du Marquenterre Baie de Somme}
\item \href{https://data.gouv.fr/dataset/558a850fc751df760ba453cd}{Enjeu surfacique d'un PPRT de Mesnil Saint Nicaise et de Nesle}
\item \href{https://data.gouv.fr/dataset/558a852088ee38791a4a22fb}{Enjeu surfacique du PPRI de la vallée de la Somme}
\item \href{https://data.gouv.fr/dataset/558a852388ee38791a4a22fd}{Enjeu surfacique du PPRI des cantons de Chaulnes et Bray sur Somme}
\item \href{https://data.gouv.fr/dataset/558a851ac751df75e4a453c8}{Enjeu surfacique du PPRm de Montdidier}
\item \href{https://data.gouv.fr/dataset/5a04a51388ee3854b8df0035}{Enjeu surfacique du PPRn des Bas Champs du Sud de la Baie Somme}
\item \href{https://data.gouv.fr/dataset/584187a888ee382f74c65bb3}{Enjeu surfacique du PPRn du Marquenterre Baie de Somme}
\item \href{https://data.gouv.fr/dataset/584187a688ee382f79c65bb3}{Entité à l'origine du risque du PPRn du Marquenterre Baie de Somme}
\item et 495 autres jeux de données\end{itemize}

\clearpage
\section{Direction Départementale des Territoires et de la Mer des Landes}


\begin{center}
  \includegraphics[width=3cm]{images/orga/2015-06-27_5ff76cca692a44abb6283cf26ac74750_logo_DDTM-100.png}
\end{center}


Direction Départementale des Territoires et de la Mer des Landes


\vspace{0.5cm}

\needspace{12\baselineskip}
\subsection*{Périmètre affecté à un lieutenant de louveterie dans le département des
Landes
}\index{chasse}\index{donnees!ouvertes}\index{environment}\index{louveterie}\index{passerelle!inspire}
  \begin{wrapfigure}{r}{2.5cm}
    \centering
    \qrcode[nolink]{https://data.gouv.fr/dataset/5882969088ee381ae19b81bc}
  \end{wrapfigure}

Licence : \textbf{Licence Ouverte version 2.0
}\newline
Créé le : 2017-01-21\newline
Modifié le : 2019-03-15\newline
Popularité : 1 réutilisation,  0 suivi\newline
Mots-clé : \emph{chasse, donnees-ouvertes, environment, louveterie, passerelle-inspire
}\newline
Permalien : \url{https://data.gouv.fr/dataset/5882969088ee381ae19b81bc}\newline

\par
\noindent
    Partie du territoire départemental affectée à un lieutenant de
louveterie dans le département des Landes Le préfet nomme pour une durée
déterminée les lieutenants de louveterie et détermine les territoires
qui leurs sont affectés (CE L 427-1 et Arrêté du 14 juin 2010 NOR :
DEVN1013973A)
-\url{https://www.legifrance.gouv.fr/affichTexte.do?cidTexte\&}x3D;JORFTEXT000022402572

Ces territoires peuvent être infra-communaux. La louveterie est une
chasse aux loups et autres grands animaux nuisibles, en vue de leur
destruction. Le préfet décide par arrêté des destructions collectives
d'animaux nuisibles. Il nomme, pour une durée de six ans, des
lieutenants de louveterie qui géreront ces destructions collectives sous
le contrôle de la direction départementale des territoires. Le nombre
des lieutenants est fixé en fonction de la superficie, du boisement et
du relief du département. Ces lieutenants sont les conseillers
techniques de l'administration en matière de problèmes posés par la
gestion de la faune sauvage.

\textbf{Origine}

Table réalisée à partir des communes de la BD Topo

\textbf{Organisations partenaires}

DDTM Landes

\textbf{Liens annexes}

\begin{itemize}

\item
  \href{http://ogc.geo-ide.developpement-durable.gouv.fr/csw/all-dataset?REQUEST=GetRecordById\&SERVICE=CSW\&VERSION=2.0.2\&RESULTTYPE=results\&elementSetName=full\&TYPENAMES=gmd:MD_Metadata\&OUTPUTSCHEMA\&\#x3D\%5Bhttp://www.isotc211.org/2005/gmd\&ID\&\%5D(http://www.isotc211.org/2005/gmd\&ID\&)x3D;fr-120066022-jdd-92d769fc-0c3c-48ec-b828-ade81b9ea99e}{Vue
  XML des métadonnées}
\item
  \href{http://ogc.geo-ide.developpement-durable.gouv.fr/wxs?map=/opt/data/carto/geoide-catalogue/1.4/org_38022/d8bb8b78-ccd5-48af-ad2c-79f0b2d4ee63.internet.map}{URL
  de base des services wms/wfs sur internet}
\end{itemize}

➞
\href{https://geo.data.gouv.fr/fr/datasets/7513e976f6cbaf8b54820c2a3d940ea36dad1d9c}{Consulter
cette fiche sur geo.data.gouv.fr}


\vspace{0.5cm}
\needspace{12\baselineskip}
\subsection*{Registre Parcellaire Graphique 2014 (RPG) - Ilots déclarés à la PAC et
les surfaces en 28 groupes de cultures - Bourgogne-Franche-Comté
}\index{agriculture!parcellaire!agricole}\index{boundaries}\index{donnees!ouvertes}\index{passerelle!inspire}\index{usage!des!sols}
  \begin{wrapfigure}{r}{2.5cm}
    \centering
    \qrcode[nolink]{https://data.gouv.fr/dataset/58a1ca5188ee3801759b81a4}
  \end{wrapfigure}

Licence : \textbf{Licence Ouverte version 2.0
}\newline
Créé le : 2017-02-13\newline
Modifié le : 2019-03-17\newline
Popularité : 1 réutilisation,  0 suivi\newline
Mots-clé : \emph{agriculture-parcellaire-agricole, boundaries, donnees-ouvertes, passerelle-inspire, usage-des-sols
}\newline
Permalien : \url{https://data.gouv.fr/dataset/58a1ca5188ee3801759b81a4}\newline

\par
\noindent
    Îlots culturaux (rendus anonymes) du registre parcellaire graphique
déclarés au titre des aides du 1er pilier de la politique agricole
commune, pour la campagne 2014. Seuls les îlots des exploitations
déclarant en Bourgogne-Franche-Comté sont disponibles.

\textbf{Origine}

Extraction des données issues des déclarations PAC, campagne 2014.
Calcul des surfaces déclarées pour chacun des 28 groupes de cultures
pour chacun des îlots. Traitements pour rendre les données anonymes
(identifiant non significatif).

\textbf{Organisations partenaires}

DRAAF Bourgogne-Franche-Comté (Direction Régionale de l'Alimentation, de
l'Agriculture et de la Forêt de Franche-Comté)

\textbf{Liens annexes}

\begin{itemize}

\item
  \href{http://ogc.geo-ide.developpement-durable.gouv.fr/csw/all-dataset?REQUEST=GetRecordById\&SERVICE=CSW\&VERSION=2.0.2\&RESULTTYPE=results\&elementSetName=full\&TYPENAMES=gmd:MD_Metadata\&OUTPUTSCHEMA\&\#x3D\%5Bhttp://www.isotc211.org/2005/gmd\&ID\&\%5D(http://www.isotc211.org/2005/gmd\&ID\&)x3D;fr-120066022-jdd-e0b2adb7-77ec-4e65-9f79-c595ddc9b3a3}{Vue
  XML des métadonnées}
\item
  \href{http://ogc.geo-ide.developpement-durable.gouv.fr/wxs?map=/opt/data/carto/geoide-catalogue/1.4/org_38198/e84d392d-6951-4e39-b97c-1446f0d2e9d5.internet.map}{URL
  de base des services wms/wfs sur internet}
\end{itemize}

➞
\href{https://geo.data.gouv.fr/fr/datasets/5b6fa339b06f587cab97fa4573db33c4128c50a8}{Consulter
cette fiche sur geo.data.gouv.fr}


\vspace{0.5cm}
\needspace{3\baselineskip} \rule{4cm}{0.25pt}\newline\textbf{Aussi disponible du même producteur :}\begin{itemize}
\item \href{https://data.gouv.fr/dataset/5882967fc751df1889ae0a73}{Aléa des incendies de forêt dans le département des Landes}
\item \href{https://data.gouv.fr/dataset/58829690c751df18c7ae0a7f}{Cartes de bruit en agglomération : Zonages dont le niveau sonore (jour, soir, nuit)   > 68 décibels dans le département des Landes}
\item \href{https://data.gouv.fr/dataset/58829690c751df191aae0a81}{Cartes de bruit en agglomération : Zonages du niveau sonore (nuit) dans le département des Landes}
\item \href{https://data.gouv.fr/dataset/5882968f88ee3807519b81c5}{Cartes de bruit sur les routes départementales : Zonages dont le niveau sonore (jour, soir, nuit) > 68 décibels dans le département des Landes}
\item \href{https://data.gouv.fr/dataset/5882968e88ee3819469b81c4}{Cartes de bruit sur les routes départementales : Zonages dont le niveau sonore (nuit) > 62 décibels dans le département des Landes}
\item \href{https://data.gouv.fr/dataset/5882968d88ee3807519b81c3}{Cartes de bruit sur les routes départementales : Zonages du niveau sonore (jour, soir, nuit) dans le département des Landes}
\item \href{https://data.gouv.fr/dataset/5882968ec751df1889ae0a7c}{Cartes de bruit sur les routes départementales : Zonages du niveau sonore (nuit) dans le département des Landes}
\item \href{https://data.gouv.fr/dataset/5882968dc751df18c7ae0a7e}{Cartes de bruit sur les routes nationales : Zonages dont le niveau sonore (jour, soir, nuit) > 68 décibels dans le département des Landes}
\item \href{https://data.gouv.fr/dataset/58829691c751df1891ae0a7b}{Cartes de bruit sur les routes nationales : Zonages dont le niveau sonore (nuit) > 62 décibels dans le département des Landes}
\item \href{https://data.gouv.fr/dataset/5882968dc751df191aae0a7e}{Cartes de bruit sur les routes nationales : Zonages du niveau sonore (jour, soir, nuit) dans le département des Landes}
\item \href{https://data.gouv.fr/dataset/5882968e88ee3808069b81bc}{Cartes de bruit sur les routes nationales : Zonages du niveau sonore (nuit) dans le département des Landes}
\item \href{https://data.gouv.fr/dataset/5882968dc751df18c7ae0a7d}{Cartes de bruit  en agglomération : Zonages du niveau sonore  (jour, soir, nuit) dans le département des Landes}
\item \href{https://data.gouv.fr/dataset/5882969088ee3819469b81c5}{Cartes de bruit  sur les autoroutes : Zonages dont le niveau sonore (jour, soir, nuit)   > 62 décibels dans le département des Landes}
\item \href{https://data.gouv.fr/dataset/5882968e88ee3807519b81c4}{Cartes de bruit  sur les autoroutes : Zonages dont le niveau sonore (jour, soir, nuit)   > 62 décibels dans le département des Landes}
\item \href{https://data.gouv.fr/dataset/5882969088ee3807519b81c6}{Cartes de bruit  sur les autoroutes : Zonages dont le niveau sonore (jour, soir, nuit)   > 68 décibels dans le département des Landes}
\item \href{https://data.gouv.fr/dataset/5882968ec751df187eae0a70}{Cartes de bruit  sur les autoroutes : Zonages du niveau sonore (nuit) dans le département des Landes}
\item \href{https://data.gouv.fr/dataset/588296cdc751df18c7ae0a80}{Cartes de bruit  sur les autoroutes : Zonages du niveau sonore (nuit) dans le département des Landes}
\item \href{https://data.gouv.fr/dataset/5a09ad9988ee3844cf133d57}{Communes concernées par le classement sonore des infrastructures de transports terrestres dans le département des Landes}
\item \href{https://data.gouv.fr/dataset/5882967bc751df1891ae0a70}{Cours d'eau retenus pour la mise en place de couverts environnementaux Pac dans le département des Landes}
\item \href{https://data.gouv.fr/dataset/5882968588ee3808069b81b6}{Localisation ponctuelle des plages dans le département des Landes}
\item \href{https://data.gouv.fr/dataset/558bf46f88ee3817714a231d}{Lots des zones d'activités dans le département du Gers}
\item \href{https://data.gouv.fr/dataset/5a09ada0c751df30a64217bc}{N\_EPCI\_ZSUP\_040}
\item \href{https://data.gouv.fr/dataset/5a09ad9588ee3846a0a845f5}{Ouvrages de protection recensés en 2017 dans le département des Landes}
\item \href{https://data.gouv.fr/dataset/58829690c751df1889ae0a7d}{Périmètres des cantons au sens de l'INSEE dans le département des Landes}
\item \href{https://data.gouv.fr/dataset/5882968c88ee3819469b81c2}{Périmètres des communes concernées par l'aléa sismique dans le département des Landes}
\item \href{https://data.gouv.fr/dataset/5882968888ee3807519b81bd}{Périmètres des Opérations Programmées d'Amélioration de l'Habitat (OPAH) dans le département des Landes}
\item \href{https://data.gouv.fr/dataset/5882969188ee3819469b81c6}{Périmètres des Programmes d'Intérêt Général (PIG) dans le département des Landes}
\item \href{https://data.gouv.fr/dataset/58aee31f88ee380d85fb083f}{Périmètres du Plan de Prévention des Risques Naturels (PPRN) d'Aire-sur-l'Adour – Landes (40)}
\item \href{https://data.gouv.fr/dataset/58829676c751df187eae0a6b}{Périmètres du Plan de Prévention des Risques Naturels (PPRN) d'Aire-sur-l'Adour – Landes (40)}
\item \href{https://data.gouv.fr/dataset/5882967688ee381ae19b81ac}{Périmètres du Plan de Prévention des Risques Naturels (PPRN) d'Angoumé – Landes (40)}
\item \href{https://data.gouv.fr/dataset/5882968cc751df18c7ae0a7c}{Périmètres du Plan de Prévention des Risques Naturels (PPRN) de Candresse – Landes (40)}
\item \href{https://data.gouv.fr/dataset/58829685c751df191aae0a79}{Périmètres du Plan de Prévention des Risques Naturels (PPRN)  de Dax – Landes (40)}
\item \href{https://data.gouv.fr/dataset/5882967d88ee3807519b81b1}{Périmètres du Plan de Prévention des Risques Naturels (PPRN) de Gousse – Landes (40)}
\item \href{https://data.gouv.fr/dataset/5882968088ee3808069b81b3}{Périmètres du Plan de Prévention des Risques Naturels (PPRN) de Mées – Landes (40)}
\item \href{https://data.gouv.fr/dataset/5882968ac751df187eae0a6e}{Périmètres du Plan de Prévention des Risques Naturels (PPRN) de Narosse – Landes (40)}
\item \href{https://data.gouv.fr/dataset/5882968588ee3807519b81ba}{Périmètres du Plan de Prévention des Risques Naturels (PPRN) de Peyrehorade – Landes (40)}
\item \href{https://data.gouv.fr/dataset/5882968a88ee381ae19b81b8}{Périmètres du Plan de Prévention des Risques Naturels (PPRN) de Rivière-Saas-et-Gourby – Landes (40)}
\item \href{https://data.gouv.fr/dataset/5882967988ee3808069b81af}{Périmètres du Plan de Prévention des Risques Naturels (PPRN) de Saint-Barthélemy – Landes (40)}
\item \href{https://data.gouv.fr/dataset/58829680c751df1889ae0a74}{Périmètres du Plan de Prévention des Risques Naturels (PPRN) de Sainte-Marie-de-Gosse – Landes (40)}
\item \href{https://data.gouv.fr/dataset/58829676c751df1891ae0a6d}{Périmètres du Plan de Prévention des Risques Naturels (PPRN) de Saint-Jean-de-Lier – Landes (40)}
\item \href{https://data.gouv.fr/dataset/5882968588ee3819469b81ba}{Périmètres du Plan de Prévention des Risques Naturels (PPRN) de Saint-Laurent-de-Gosse – Landes (40)}
\item \href{https://data.gouv.fr/dataset/58829675c751df18c7ae0a6d}{Périmètres du Plan de Prévention des Risques Naturels (PPRN) de Saint-Martin-de-Seignanx – Landes (40)}
\item \href{https://data.gouv.fr/dataset/58829677c751df18c7ae0a70}{Périmètres du Plan de Prévention des Risques Naturels (PPRN) de Saint Paul-lès-Dax – Landes (40)}
\item \href{https://data.gouv.fr/dataset/58829676c751df1889ae0a6b}{Périmètres du Plan de Prévention des Risques Naturels (PPRN) de Saint-Vincent-de-Paul – Landes (40)}
\item \href{https://data.gouv.fr/dataset/5882967988ee3807519b81ae}{Périmètres du Plan de Prévention des Risques Naturels (PPRN) de Seyresse – Landes (40)}
\item \href{https://data.gouv.fr/dataset/5882968788ee3807519b81bc}{Périmètres du Plan de Prévention des Risques Naturels (PPRN) de Tartas – Landes (40)}
\item \href{https://data.gouv.fr/dataset/5882967b88ee3819469b81b0}{Périmètres du Plan de Prévention des Risques Naturels (PPRN) de Tercis-les-Bains – Landes (40)}
\item \href{https://data.gouv.fr/dataset/5882967ec751df1891ae0a71}{Périmètres du Plan de Prévention des Risques Naturels (PPRN) de Téthieu – Landes (40)}
\item \href{https://data.gouv.fr/dataset/5882967d88ee3808069b81b1}{Périmètres du Plan de Prévention des Risques Naturels (PPRN) d'Hastingues – Landes (40)}
\item \href{https://data.gouv.fr/dataset/58829686c751df18c7ae0a78}{Périmètres du Plan de Prévention des Risques Naturels (PPRN) d'Oeyregave – Landes (40)}
\item et 91 autres jeux de données\end{itemize}

\clearpage
\section{Dispotrains}


\begin{center}
  \includegraphics[width=3cm]{images/orga/b0_48174cce41466181f7728af9927986-100.png}
\end{center}


DispoTrains: l'information (presque) temps-réel sur l'accessibilité du
réseau ferré d'Ile-de-France aux usagers à mobilité réduite.

Dispotrains compile les informations publiques d'accessibilité du réseau
ferré d'Ile-de-France pour les rendre plus facilement disponibles.


\vspace{0.5cm}

\needspace{12\baselineskip}
\subsection*{Historique de disponibilité des ascenseurs
}\index{accessibilite}\index{handicap}\index{ile!de!france}\index{ratp}\index{transilien}
  \begin{wrapfigure}{r}{2.5cm}
    \centering
    \qrcode[nolink]{https://data.gouv.fr/dataset/561b88dd88ee3825ff628efb}
  \end{wrapfigure}

Licence : \textbf{Creative Commons Attribution Share-Alike
}\newline
Créé le : 2015-10-12\newline
Modifié le : 2018-01-02\newline
De 2013-02-27 à 2018-01-02\newline
Mise à jour : quotienne\newline
Popularité : 1 réutilisation,  0 suivi\newline
Mots-clé : \emph{accessibilite, handicap, ile-de-france, ratp, transilien
}\newline
Permalien : \url{https://data.gouv.fr/dataset/561b88dd88ee3825ff628efb}\newline

\par
\noindent
    Historique de disponibilité des ascenseurs des stations de RER,
Transilien, Tram et Métro d'Ile-de-France.

Compilation des données d'état des ascenseurs du réseau STIF en
Ile-de-France. Ce jeu de donnée est mis à jour quotidiennement et
comporte, en général, trois prises d'état par jour.


\vspace{0.5cm}
\needspace{3\baselineskip} \rule{4cm}{0.25pt}\newline\textbf{Aussi disponible du même producteur :}\begin{itemize}
\item \href{https://data.gouv.fr/dataset/561b8e9688ee3825ff628efc}{Archive des disponibilité des ascenseurs}
\end{itemize}

\clearpage
\section{DRAAF Nouvelle Aquitaine}


\begin{center}
  \includegraphics[width=3cm]{images/orga/80_98f5e574a6439788192e07fadc995f-100.png}
\end{center}


DRAAF Nouvelle Aquitaine


\vspace{0.5cm}

\needspace{12\baselineskip}
\subsection*{Périmètre affecté à un lieutenant de louveterie dans le Gers
}\index{chasse}\index{donnees!ouvertes}\index{louveterie}\index{passerelle!inspire}\index{utilities!communication}
  \begin{wrapfigure}{r}{2.5cm}
    \centering
    \qrcode[nolink]{https://data.gouv.fr/dataset/558bf446c751df2c0ca453c9}
  \end{wrapfigure}

Licence : \textbf{Licence Ouverte version 2.0
}\newline
Créé le : 2015-06-25\newline
Modifié le : 2019-03-16\newline
Popularité : 1 réutilisation,  0 suivi\newline
Mots-clé : \emph{chasse, donnees-ouvertes, louveterie, passerelle-inspire, utilities-communication
}\newline
Permalien : \url{https://data.gouv.fr/dataset/558bf446c751df2c0ca453c9}\newline

\par
\noindent
    Le préfet nomme pour une durée déterminée les lieutenants de louveterie
et détermine les territoires qui leurs sont affectés (CE L 427-1 et
Arrêté du 14 juin 2010 NOR : DEVN1013973A) Ces territoires peuvent être
infra-communaux. La louveterie est une chasse aux loups et autres grands
animaux nuisibles, en vue de leur destruction. Le préfet décide par
arrêté des destructions collectives d'animaux nuisibles. Il nomme, pour
une durée de six ans, des lieutenants de louveterie qui géreront ces
destructions collectives sous le contrôle de la direction départementale
des territoires.Le nombre des lieutenants est fixé en fonction de la
superficie, du boisement et du relief du département. Ces lieutenants
sont les conseillers techniques de l'administration en matière de
problèmes posés par la gestion de la faune sauvage.

\textbf{Origine}

Par agglomération et/ou numérisation des communes et parties de communes
concernées à partir de la couche des communes de la BDTopo.

\textbf{Organisations partenaires}

DDT Gers

\textbf{Liens annexes}

\begin{itemize}

\item
  \href{http://geostandards.developpement-durable.gouv.fr/afficherPageStandard.do?jeu=N_CHASSE_LOUV_ZINF_S}{Standard
  de données COVADIS : Périmètre affecté à un lieutenant de louveterie.}
\item
  \href{http://ogc.geo-ide.developpement-durable.gouv.fr/wxs?map=/opt/data/carto/geoide-catalogue/1.4/org_38006/b8b44661-a8b9-4f55-af10-e19d56ead298.internet.map}{URL
  de base des services wms/wfs sur internet
  :{[}http://ogc.geo-ide.developpement-durable.gouv.fr/wxs?map\&{]}(http://ogc.geo-ide.developpement-durable.gouv.fr/wxs?map\&)x3D;/opt/data/carto/geoide-catalogue/1.4/org\_38006/b8b44661-a8b9-4f55-af10-e19d56ead298.internet.map}
\end{itemize}

➞
\href{https://geo.data.gouv.fr/fr/datasets/d299eeef37ef0d3f8981ddffd6b79e5082fbb86a}{Consulter
cette fiche sur geo.data.gouv.fr}


\vspace{0.5cm}
\needspace{3\baselineskip} \rule{4cm}{0.25pt}\newline\textbf{Aussi disponible du même producteur :}\begin{itemize}
\item \href{https://data.gouv.fr/dataset/5a099fc3c751df1c82537c61}{Coupes rases de bois en Dordogne 2013 - 2014}
\item \href{https://data.gouv.fr/dataset/589af489c751df7a63ae0a65}{Les régions agricoles de la région Nouvelle-Aquitaine}
\item \href{https://data.gouv.fr/dataset/58a196dd88ee38678f9b81a4}{Observatoire des sols à l'échelle communale de la zone Nord Est du département de la DORDOGNE pour 2013 - Version Bêta - Données agrégées}
\item \href{https://data.gouv.fr/dataset/58a196dbc751df67e7ae0a65}{Observatoire des sols à l'échelle communale de la zone Nord Est du département de la GIRONDE  pour 2013 - Version Bêta - Données brutes}
\item \href{https://data.gouv.fr/dataset/58a196dc88ee38678e9b81a4}{Observatoire des sols à l'échelle communale de la zone Nord Est du département des Landes pour 2013 - Version Bêta - Données brutes}
\item \href{https://data.gouv.fr/dataset/58a196ddc751df67ecae0a65}{Observatoire des sols à l'échelle communale de la zone Nord Est du département des Pyrénées Atlantiques pour 2013 - Version Bêta - Données agrégées}
\item \href{https://data.gouv.fr/dataset/58a196de88ee38678b9b81a5}{Observatoire des sols à l'échelle communale de la zone Sud Est du département de la GIRONDE pour 2013 - Version Bêta - Données brutes}
\item \href{https://data.gouv.fr/dataset/58a196ddc751df67eaae0a66}{Observatoire des sols à l'échelle communale de la zone Sud Est du département des LANDES pour 2013 - Version Bêta - Données agrégées}
\item \href{https://data.gouv.fr/dataset/58a196dcc751df67eaae0a65}{Observatoire des sols à l'échelle communale de la zone Sud Est du département des LANDES pour 2013 - Version Bêta - Données brutes}
\item \href{https://data.gouv.fr/dataset/58a196dd88ee38678d9b81a4}{Observatoire des sols à l'échelle communale de la zone Sud Est du département des Pyrénées Atlantiques pour 2013 - Version Bêta - Données agrégées}
\item \href{https://data.gouv.fr/dataset/5a099fc688ee3831d51556bb}{Observatoire des sols à l'échelle communale de la zone Sud Est du département des Pyrénées Atlantiques pour 2013 - Version Bêta - Données brutes}
\item \href{https://data.gouv.fr/dataset/58a196dbc751df67e8ae0a65}{Observatoire des sols à l'échelle communale de la zone Sud Ouest du département des Landes - Version Bêta - Données agrégées}
\item \href{https://data.gouv.fr/dataset/58a196de88ee38678d9b81a5}{Observatoire des sols à l'échelle communale de la zone Sud Ouest du département du LOT et GARONNE pour 2013 - Version Bêta - Données brutes}
\item \href{https://data.gouv.fr/dataset/559c029988ee38188e764f5f}{Organismes compétents en aménagement de rivières en Charente.}
\item \href{https://data.gouv.fr/dataset/589b2efa88ee38674f9b81a4}{Petites Régions Agricoles en Aquitaine}
\item \href{https://data.gouv.fr/dataset/589b2efac751df67a7ae0a65}{Peuplements classés graines et plants en Aquitaine 2015 (TAB)}
\item \href{https://data.gouv.fr/dataset/589af48988ee3879999b81a4}{Prescription linéraire du PLU de la commune de Clery sur Somme}
\item \href{https://data.gouv.fr/dataset/589b2efbc751df681bae0a65}{Région Agricole (RA) en Aquitaine}
\item \href{https://data.gouv.fr/dataset/589b2efac751df67a9ae0a65}{Territoire des GAL en Aquitaine}
\end{itemize}

\clearpage
\section{Easter-eggs}


\begin{center}
  \includegraphics[width=3cm]{images/orga/23_7c4fbb932b460983b7a73d01d67918-100.jpg}
\end{center}


Easter-eggs, diffuseur de données libres pour les collectivités et les
services de l'Etat. Entreprise solidaire, spécialiste logiciels libres
depuis 1997.

\href{http://www.donnees-libres.fr}{Données-libres.fr} est une
plateforme de service qui permet aux collectivités territoriales et aux
services de l'Etat de diffuser en co-marquage les données de
data.gouv.fr sur leur site Internet.

Easter-eggs est également éditrice du service de co-marquage de données
publiques libres \href{http://www.comarquage.fr}{Comarquage.fr}.


\vspace{0.5cm}

\needspace{12\baselineskip}
\subsection*{Base nationale des organismes publics
}\index{annuaire}\index{organismes!publics}\index{service!public}
  \begin{wrapfigure}{r}{2.5cm}
    \centering
    \qrcode[nolink]{https://data.gouv.fr/dataset/53698f5aa3a729239d203713}
  \end{wrapfigure}

Licence : \textbf{Licence Ouverte
}\newline
Créé le : 2013-09-20\newline
Modifié le : 2016-02-04\newline
Granularité : à la commune\newline
Popularité : 1 réutilisation,  2 suivis\newline
Mots-clé : \emph{annuaire, organismes-publics, service-public
}\newline
Permalien : \url{https://data.gouv.fr/dataset/53698f5aa3a729239d203713}\newline

\par
\noindent
    L'annuaire de l'Administration Comarquage.fr contient actuellement près
de 70 000 fiches d'organismes et de services géolocalisés sur le
territoire français.


\vspace{0.5cm}

\clearpage
\section{Enedis}


\begin{center}
  \includegraphics[width=3cm]{images/orga/1f_42d2d1de184464bff9501b4d99461b-100.jpg}
\end{center}


Enedis est une entreprise de service public, gestionnaire du réseau de
distribution d'électricité qui développe, exploite, modernise le réseau
électrique et gère les données associées. Indépendante des fournisseurs
d'énergie chargés de la vente et de la gestion du contrat d'électricité,
Enedis réalise les raccordements, le dépannage, le relevé des compteurs
et toutes interventions techniques.


\vspace{0.5cm}

\needspace{12\baselineskip}
\subsection*{Données relatives aux lignes et aux postes
}\index{basse!tension}\index{infrastructures}\index{lignes}\index{longueur}\index{moyenne!tension}\index{nombre}\index{postes}\index{reseau}
  \begin{wrapfigure}{r}{2.5cm}
    \centering
    \qrcode[nolink]{https://data.gouv.fr/dataset/5835148da3a7290df6f45ca6}
  \end{wrapfigure}

Licence : \textbf{Licence Ouverte version 2.0
}\newline
Créé le : 2016-11-23\newline
Modifié le : 2019-03-17\newline
Popularité : 1 réutilisation,  0 suivi\newline
Mots-clé : \emph{basse-tension, infrastructures, lignes, longueur, moyenne-tension, nombre, postes, reseau
}\newline
Permalien : \url{https://data.gouv.fr/dataset/5835148da3a7290df6f45ca6}\newline

\par
\noindent
    Ce jeu de données restitue la longueur en km des lignes moyenne tension
(HTA) et basse tension (BT) par type ainsi que le nombre de postes de
distribution publique (postes HTA/BT) exploités par Enedis.

Visualisez aussi les données relatives aux lignes et aux postes sur
notre
\href{http://www.enedis.fr/donnees-relatives-aux-lignes-et-aux-postes}{site
internet}.


\vspace{0.5cm}
\needspace{3\baselineskip} \rule{4cm}{0.25pt}\newline\textbf{Aussi disponible du même producteur :}\begin{itemize}
\item \href{https://data.gouv.fr/dataset/5ae44302b59508691380a891}{Agrégats segmentés de consommation électrique au pas 1/2 h des points de soutirage <= 36kVA}
\item \href{https://data.gouv.fr/dataset/5ae44301b59508691380a890}{Agrégats segmentés de consommation électrique au pas 1/2 h des points de soutirage > 36kVA}
\item \href{https://data.gouv.fr/dataset/5ae442eeb59508690d80a891}{Agrégats segmentés de production électrique au pas 1/2 h}
\item \href{https://data.gouv.fr/dataset/560d93bcb595086cd501d75c}{Bilan Électrique au pas demi-heure}
\item \href{https://data.gouv.fr/dataset/560d93bbb595086cd601d75c}{Bilan Électrique au pas journalier}
\item \href{https://data.gouv.fr/dataset/560d93bba3a7294ad6ff2d08}{Bilan Électrique - Puissance installée}
\item \href{https://data.gouv.fr/dataset/5a826539a3a7295dcfba013f}{Consommation électrique annuelle à la maille commune}
\item \href{https://data.gouv.fr/dataset/5a826557b595081fb6b66414}{Consommation électrique annuelle à la maille département}
\item \href{https://data.gouv.fr/dataset/5a82655ba3a7295dcfba0142}{Consommation électrique annuelle à la maille EPCI}
\item \href{https://data.gouv.fr/dataset/5a82653ab595081e27b66414}{Consommation électrique annuelle à la maille IRIS}
\item \href{https://data.gouv.fr/dataset/5a826547a3a7295dcfba0141}{Consommation électrique annuelle à la maille région}
\item \href{https://data.gouv.fr/dataset/560d93bba3a7294a8e94aa1b}{Consommation journalière par catégorie client}
\item \href{https://data.gouv.fr/dataset/5bbc656b06e3e70cc35aef6c}{Demandes de raccordement des installations de production au réseau Enedis par région, par département, par tranche de puissance et par modalité d’injection}
\item \href{https://data.gouv.fr/dataset/5a336090a3a729559d2d3329}{Durée moyenne de coupure par client BT}
\item \href{https://data.gouv.fr/dataset/5a33607eb5950807dd601d58}{Durée moyenne de coupure par client HTA}
\item \href{https://data.gouv.fr/dataset/5bbc656b9ce2e72528df4046}{Evolution des sorties des demandes de raccordement au réseau géré par Enedis par tranche de puissance et par modalités d’injection}
\item \href{https://data.gouv.fr/dataset/5bbc65899ce2e72534df4046}{Evolution trimestrielle du parc des installations de production raccordées sur le réseau Enedis par tranche de puissance et par modalités d’injection}
\item \href{https://data.gouv.fr/dataset/58c0d398a3a7293affefc206}{Evolution trimestrielle du parc raccordé en cours}
\item \href{https://data.gouv.fr/dataset/560d93bba3a7294a8d94aa1b}{Flexibilités participant au MA ou à NEBEF au périmètre d'Enedis}
\item \href{https://data.gouv.fr/dataset/560d93bba3a7294ad6ff2d07}{Flexibilités participant au MA ou à NEBEF par région}
\item \href{https://data.gouv.fr/dataset/5a3360a0b5950807dd601d5a}{Fréquence moyenne de coupure par client BT}
\item \href{https://data.gouv.fr/dataset/5a336094b59508096b601d58}{Fréquence moyenne de coupure par client HTA}
\item \href{https://data.gouv.fr/dataset/560d93bbb595086cd501d75a}{Galerie d'images}
\item \href{https://data.gouv.fr/dataset/5bbc658b9ce2e72534df4047}{Historique de l’état des entrées des demandes de raccordement de production au réseau Enedis  par tranche de puissance et par modalités d’injection}
\item \href{https://data.gouv.fr/dataset/58c0d39aa3a7293afeefc22d}{Historique de l’état la file d’attente des demandes de raccordement de production au réseau Enedis}
\item \href{https://data.gouv.fr/dataset/5bbc65699ce2e72528df4045}{Historique des projets en cours, des demandes de raccordement  entrées et sorties par tranche de puissance et par modalité d’injection}
\item \href{https://data.gouv.fr/dataset/560d93bbb595086cd501d75b}{Historique du parc des installations de production raccordées sur le réseau Enedis}
\item \href{https://data.gouv.fr/dataset/5bbc65859ce2e72534df4045}{Historique du parc des installations de production raccordées sur le réseau Enedis par tranche de puissance et par modalités d’injection}
\item \href{https://data.gouv.fr/dataset/5835148aa3a7290df5f45c60}{Indicateur réglementaire continuité d’alimentation}
\item \href{https://data.gouv.fr/dataset/5afd2d03a3a729485189da52}{Lignes aériennes HTA}
\item \href{https://data.gouv.fr/dataset/58a280d5a3a72974c1f1156d}{Nombre de mises en service sur installations existantes Entreprise et PME-PMI}
\item \href{https://data.gouv.fr/dataset/58a280d6a3a72974c1f1156e}{Nombre de mises en service sur installations existantes Résidentiel BT et Professionnel BT}
\item \href{https://data.gouv.fr/dataset/5a336090b5950807dd601d59}{Nombre de points de charge accessibles au public pour 100 000 habitants par région et son évolution sur les 12 derniers mois}
\item \href{https://data.gouv.fr/dataset/5a3360a3a3a72955bfc50966}{Nombre de points de charge par typologie}
\item \href{https://data.gouv.fr/dataset/5ae44304b59508691380a893}{Nombre de points de soutirage <= 36kVA par profil et puissance souscrite}
\item \href{https://data.gouv.fr/dataset/5ae44303b59508691380a892}{Nombre de points de soutirage > 36kVA par profil, puissance souscrite et secteur d'activité}
\item \href{https://data.gouv.fr/dataset/5ae442edb59508690d80a890}{Nombre de points d'injection par filière de production et puissance d'injection}
\item \href{https://data.gouv.fr/dataset/5a33609fa3a72955c32d3329}{Nombre total de points de charge}
\item \href{https://data.gouv.fr/dataset/560d93bba3a7294a8d94aa1c}{Parc des installations de production raccordées sur le réseau Enedis par région}
\item \href{https://data.gouv.fr/dataset/5bbc658406e3e7108a5aef6c}{Parc des installations de production raccordées sur le réseau Enedis par région, département, par tranche de puissance et par modalités d’injection}
\item \href{https://data.gouv.fr/dataset/560d93bba3a7294ad7ff2d07}{Parc des installations de production raccordées sur le réseau Enedis par tranche de puissance}
\item \href{https://data.gouv.fr/dataset/57868565a3a7294a196d121b}{Pertes modélisées}
\item \href{https://data.gouv.fr/dataset/5afd2d1bb5950855d4d11034}{Postes HTA/BT}
\item \href{https://data.gouv.fr/dataset/5afd2d05a3a729485189da53}{Postes source}
\item \href{https://data.gouv.fr/dataset/5a82655db595081fb6b66415}{Production électrique annuelle par filière à la maille commune}
\item \href{https://data.gouv.fr/dataset/5a826546a3a7295dcfba0140}{Production électrique annuelle par filière à la maille département}
\item \href{https://data.gouv.fr/dataset/5a826536a3a7295dcfba013e}{Production électrique annuelle par filière à la maille IRIS}
\item \href{https://data.gouv.fr/dataset/5a826544b595081e27b66415}{Production électrique annuelle par filière à la maille région}
\item \href{https://data.gouv.fr/dataset/5a826545b595081e27b66416}{Production électrique par filière à la maille EPCI}
\item \href{https://data.gouv.fr/dataset/5a3360a3a3a72955bfc50967}{Puissance installée par typologie}
\item et 7 autres jeux de données\end{itemize}

\clearpage
\section{George, le deuxième texte}


\begin{center}
  \includegraphics[width=3cm]{images/orga/9c_3e28e3bf654d0dae06c19a3e0de07e-100.jpg}
\end{center}


\emph{George, le deuxième texte} est un projet primé en 2017 par le
ministère chargé des droits des femmes au HackEgalitéFH. Il vise à
donner davantage de visibilité aux autrices, actuellement
sous-représentées dans les programmes, en proposant du contenu
pédagogique à destination des enseignants de lettres du secondaire et du
supérieur, par une plateforme web participative.


\vspace{0.5cm}

\needspace{12\baselineskip}
\subsection*{Auteurs et autrices dans des séquences de cours de français de première
}\index{education}\index{education!nationale}\index{egalite!femme!homme}\index{egalite!femmes!hommes}\index{egalite!homme!femme}\index{egalite!hommes!femmes}\index{lycee}
  \begin{wrapfigure}{r}{2.5cm}
    \centering
    \qrcode[nolink]{https://data.gouv.fr/dataset/5a6e07ae88ee38623246aa5a}
  \end{wrapfigure}

Licence : \textbf{Creative Commons Attribution Share-Alike
}\newline
Créé le : 2018-01-28\newline
Modifié le : 2018-01-28\newline
De 2018-01-21 à 2018-01-28\newline
Mise à jour : irrégulière\newline
Popularité : 1 réutilisation,  0 suivi\newline
Mots-clé : \emph{education, education-nationale, egalite-femme-homme, egalite-femmes-hommes, egalite-homme-femme, egalite-hommes-femmes, lycee
}\newline
Permalien : \url{https://data.gouv.fr/dataset/5a6e07ae88ee38623246aa5a}\newline

\par
\noindent
    Ce jeu de données recense les auteurs et autrices des extraits de textes
étudiés en classe de première en France, selon un recueil de ``listes de
bac'' qui résument les séquences présentées en cours de français et sont
transmises aux jurys d'oraux des épreuves anticipées de français du
baccalauréat.


\vspace{0.5cm}
\needspace{12\baselineskip}
\subsection*{Auteurs et autrices dans les annales de l'épreuve de français du diplôme
national du brevet
}\index{brevet}\index{college}\index{dnb}\index{education}\index{education!nationale}\index{egalite!femme!homme}\index{egalite!femmes!hommes}\index{egalite!homme!femme}\index{egalite!hommes!femmes}
  \begin{wrapfigure}{r}{2.5cm}
    \centering
    \qrcode[nolink]{https://data.gouv.fr/dataset/5a56034988ee3845f1b87d7d}
  \end{wrapfigure}

Licence : \textbf{Creative Commons Attribution Share-Alike
}\newline
Créé le : 2018-01-10\newline
Modifié le : 2018-01-10\newline
Mise à jour : annuelle\newline
Popularité : 1 réutilisation,  0 suivi\newline
Mots-clé : \emph{brevet, college, dnb, education, education-nationale, egalite-femme-homme, egalite-femmes-hommes, egalite-homme-femme, egalite-hommes-femmes
}\newline
Permalien : \url{https://data.gouv.fr/dataset/5a56034988ee3845f1b87d7d}\newline

\par
\noindent
    Ce jeu de données recense les auteurs et autrices des extraits de textes
proposés dans les épreuves de français du diplôme national du brevet, en
fournissant un lien vers les sujets qui les contiennent.


\vspace{0.5cm}
\needspace{12\baselineskip}
\subsection*{Auteurs et autrices dans les annales du baccalauréat de français
}\index{baccalaureat}\index{education}\index{education!nationale}\index{egalite!femme!homme}\index{egalite!femmes!hommes}\index{egalite!homme!femme}\index{egalite!hommes!femmes}
  \begin{wrapfigure}{r}{2.5cm}
    \centering
    \qrcode[nolink]{https://data.gouv.fr/dataset/594218bec751df6a8844711e}
  \end{wrapfigure}

Licence : \textbf{Creative Commons Attribution Share-Alike
}\newline
Créé le : 2017-06-15\newline
Modifié le : 2018-07-23\newline
De 2002-01-01 à 2018-07-05\newline
Mise à jour : annuelle\newline
Popularité : 2 réutilisations,  0 suivi\newline
Mots-clé : \emph{baccalaureat, education, education-nationale, egalite-femme-homme, egalite-femmes-hommes, egalite-homme-femme, egalite-hommes-femmes
}\newline
Permalien : \url{https://data.gouv.fr/dataset/594218bec751df6a8844711e}\newline

\par
\noindent
    Ce jeu de données recense les auteurs et autrices des extraits de textes
proposés dans les écrits des épreuves anticipées de français du
baccalauréat


\vspace{0.5cm}
\needspace{12\baselineskip}
\subsection*{Auteurs et autrices dans les bibliothèques numériques librement
accessibles en ligne
}\index{bibliotheque!numerique}\index{egalite!femme!homme}\index{egalite!femmes!hommes}\index{egalite!homme!femme}\index{egalite!hommes!femmes}\index{theatre}
  \begin{wrapfigure}{r}{2.5cm}
    \centering
    \qrcode[nolink]{https://data.gouv.fr/dataset/5b56220188ee383c3ce326e8}
  \end{wrapfigure}

Licence : \textbf{Creative Commons Attribution
}\newline
Créé le : 2018-07-23\newline
Modifié le : 2018-07-23\newline
De 2018-04-30 à 2018-05-01\newline
Popularité : 1 réutilisation,  0 suivi\newline
Mots-clé : \emph{bibliotheque-numerique, egalite-femme-homme, egalite-femmes-hommes, egalite-homme-femme, egalite-hommes-femmes, theatre
}\newline
Permalien : \url{https://data.gouv.fr/dataset/5b56220188ee383c3ce326e8}\newline

\par
\noindent
    Ce jeu de données recense les auteurs et autrices dont les oeuvres sont
présentes dans des bibliothèques numériques librement accessibles en
ligne : libretheatre.fr, theatregratuit.com et theatre-classique.net.


\vspace{0.5cm}
\needspace{12\baselineskip}
\subsection*{Auteurs et autrices dans les manuels scolaires
}\index{edition}\index{education}\index{education!nationale}\index{egalite!femme!homme}\index{egalite!femmes!hommes}\index{egalite!homme!femme}\index{egalite!hommes!femmes}\index{femme!homme}\index{litterature}
  \begin{wrapfigure}{r}{2.5cm}
    \centering
    \qrcode[nolink]{https://data.gouv.fr/dataset/5a344b9988ee3866027390e6}
  \end{wrapfigure}

Licence : \textbf{Creative Commons Attribution Share-Alike
}\newline
Créé le : 2017-12-15\newline
Modifié le : 2017-12-15\newline
Mise à jour : annuelle\newline
Popularité : 1 réutilisation,  0 suivi\newline
Mots-clé : \emph{edition, education, education-nationale, egalite-femme-homme, egalite-femmes-hommes, egalite-homme-femme, egalite-hommes-femmes, femme-homme, litterature
}\newline
Permalien : \url{https://data.gouv.fr/dataset/5a344b9988ee3866027390e6}\newline

\par
\noindent
    Ce jeu de données recense les auteurs et autrices qui apparaissent dans
les manuels scolaires


\vspace{0.5cm}
\needspace{3\baselineskip} \rule{4cm}{0.25pt}\newline\textbf{Aussi disponible du même producteur :}\begin{itemize}
\item \href{https://data.gouv.fr/dataset/59421608c751df2a1dea358d}{Auteurs et autrices dans les collections des maisons d'édition}
\item \href{https://data.gouv.fr/dataset/58f7d01ac751df583252dfde}{Auteurs et autrices dans les programmes d'enseignement ou de concours de lettres}
\end{itemize}

\clearpage
\section{Geowebservice}


\begin{center}
  \includegraphics[width=3cm]{images/orga/64_18cb3530064bedb1e30f5709972e41-100.png}
\end{center}


Outil d'aide à la décision en matière d'aménagement du territoire et
d'urbanisme et foncier.

Voici le lien de la chaîne Youtube
\url{https://www.youtube.com/user/tutorielgeo} Voici le lien vers la
page facebook
:\url{https://www.facebook.com/Tutorielgeo-Geomatic-Tutorial-GIS-Tutorial-Webmapping-Tutorial-325658277554574/}
Voici le lien vers le compte twitter
:\url{https://twitter.com/TutorielGeo} Voici le lien vers la page google
plus
:\url{https://plus.google.com/b/117203987416263637144/+tutorielgeo/posts}
Voici le lien vers le blog
:\url{http://freesoftwarestutorials.blogspot.fr/}


\vspace{0.5cm}

\needspace{12\baselineskip}
\subsection*{Liste des tutoriels - TutorielGeo
}\index{data!mining}\index{formation!professionnelle}\index{geographie}\index{geomatique}\index{sig}\index{tutoriel}\index{webmapping}
  \begin{wrapfigure}{r}{2.5cm}
    \centering
    \qrcode[nolink]{https://data.gouv.fr/dataset/5b070761c751df45f3f335f6}
  \end{wrapfigure}

Licence : \textbf{Licence Ouverte
}\newline
Créé le : 2018-05-24\newline
Modifié le : 2018-05-24\newline
Popularité : 1 réutilisation,  0 suivi\newline
Mots-clé : \emph{data-mining, formation-professionnelle, geographie, geomatique, sig, tutoriel, webmapping
}\newline
Permalien : \url{https://data.gouv.fr/dataset/5b070761c751df45f3f335f6}\newline

\par
\noindent
    Liste des 144 tutoriels de la chaîne Youtube TutorielGeo :
**\url{https://www.youtube.com/user/tutorielgeo/featured_**} Plus de 200
vidéos tutoriel gratuites sur Qgis, Postgis, Geoserver, Pentaho, Talend,
Google Earth Pro\ldots{} ainsi que sur les technologies webmapping et le
gestion de base de données : Oracle, Mysql, SQL Server. Voici le lien
vers le store
:\url{https://play.google.com/store/apps/details?id=com.tutorielgeo.mobileapps}
Voici le lien vers le site internet :\url{https://tutorielgeo.com} Voici
le lien de la chaîne Youtube
\url{https://www.youtube.com/user/tutorielgeo} Voici le lien vers la
page facebook
:\url{https://www.facebook.com/Tutorielgeo-Geomatic-Tutorial-GIS-Tutorial-Webmapping-Tutorial-325658277554574/}
Voici le lien vers le compte twitter
:\url{https://twitter.com/TutorielGeo} Voici le lien vers la page google
plus
:\url{https://plus.google.com/b/117203987416263637144/+tutorielgeo/posts}


\vspace{0.5cm}

\clearpage
\section{GIP Maximilien}


\begin{center}
  \includegraphics[width=3cm]{images/orga/e3_1211aa31a74161b140845e19af2f82-100.jpg}
\end{center}


Face à la difficulté des entrepreneurs pour accéder aux marchés publics,
et notamment les TPE-PME, et à celle des acheteurs publics cherchant à
concilier respect de la réglementation et efficacité des achats, la
création d'un portail commun de dématérialisation des marchés publics
franciliens a été le moteur de la création de Maximilien.

Maximilien est un Service public mutualisé initié par la Région aux
côtés des départements. Il a pour mission d'accompagner tous les
pouvoirs adjudicateurs franciliens à répondre à leurs obligations
réglementaires de cette année 2018 concernant les marchés publics :
réponse dématérialisée obligatoire, signature électronique et ouverture
des données essentielles de marchés. Reconnu comme un acteur clef de
l'achat public et de l'e-administration en Ile-de-France, Maximilien
fédère aujourd'hui plus de 300 acheteurs publics franciliens, de toutes
tailles (la Région Ile-de-France, l'ensemble des départements, des EPT,
la Métropole, des communes, des EPCI, des syndicats, etc) et de natures
juridiques très différentes (Collectivités, Lycée, OPH, CCAS, SEM, GIP,
SA, etc). La volonté partagée de tous les membres est de construire
collectivement un service public de diffusion des usages numériques sur
le territoire francilien, fondé sur la solidarité entre les structures
de grande et de petite taille.


\vspace{0.5cm}

\needspace{12\baselineskip}
\subsection*{Annuaire des profils acheteurs des adhérents du GIP Maximilien
}\index{commande!publique}\index{decp}\index{profil!acheteur}\index{profil!dacheteur}
  \begin{wrapfigure}{r}{2.5cm}
    \centering
    \qrcode[nolink]{https://data.gouv.fr/dataset/5c10d581634f4178275a8f58}
  \end{wrapfigure}

Licence : \textbf{Licence Ouverte version 2.0
}\newline
Créé le : 2018-12-12\newline
Modifié le : 2019-03-13\newline
De 2018-12-12 à 2019-12-31\newline
Granularité : à la région\newline
Mise à jour : ponctuelle\newline
Popularité : 1 réutilisation,  1 suivi\newline
Mots-clé : \emph{commande-publique, decp, profil-acheteur, profil-dacheteur
}\newline
Permalien : \url{https://data.gouv.fr/dataset/5c10d581634f4178275a8f58}\newline

\par
\noindent
    Ce fichier constitue l'annuaire des profils d'acheteurs des adhérents du
GIP Maximilien au sens de l'article 4 de l'arrêté du 14 avril 2017
relatif aux fonctionnalités et exigences minimales des profils
d'acheteurs.

Le fichier comprend les variables suivantes: - le SIRET des acheteurs
(colonne \texttt{siretAcheteur}) - l'adresse URL des profils d'acheteurs
(colonne \texttt{urlProfilAcheteur}) - l'adresse URL du catalogue DCAT
qui répertorie les données (colonne \texttt{urlDCAT}) - les coordonnées
du ou des acheteurs concernés (colonne \texttt{coordonnnees})


\vspace{0.5cm}

\clearpage
\section{global map solution}


\begin{center}
  \includegraphics[width=3cm]{images/orga/9d_6a9729e8564f63905f6f31cb00ba0d-100.jpg}
\end{center}


Global Map est une nouvelle solution de visualisation cartographique.
Notre ambition est de démocratiser l'accès à la donnée géographique au
travers d'une interface simple, intuitive et performante.

Global Map cherche à exploiter tout le potentiel de l'open data, avec
des premiers services permettant : -de géocoder et cartographier vos
adresses postales à partir de la Base Adresse Nationale -de présenter
une galerie de cartes avec les principaux indicateurs du marché tirés
principalement de l'open data


\vspace{0.5cm}

\needspace{12\baselineskip}
\subsection*{Base des logements, France, IRIS
}\index{logements}\index{logements!vacants}
  \begin{wrapfigure}{r}{2.5cm}
    \centering
    \qrcode[nolink]{https://data.gouv.fr/dataset/5a05df4388ee3848c27a2ff9}
  \end{wrapfigure}

Licence : \textbf{Creative Commons CCZero
}\newline
Créé le : 2017-11-10\newline
Modifié le : 2017-11-10\newline
Granularité : à l'IRIS INSEE\newline
Popularité : 1 réutilisation,  0 suivi\newline
Mots-clé : \emph{logements, logements-vacants
}\newline
Permalien : \url{https://data.gouv.fr/dataset/5a05df4388ee3848c27a2ff9}\newline

\par
\noindent
    la base des logements en France au niveau IRIS, du recensement INSEE
2013


\vspace{0.5cm}
\needspace{12\baselineskip}
\subsection*{Revenus des Français à la commune
}\index{commune}\index{france}\index{niveau!de!vie}\index{pauvrete}\index{revenus}\index{richesse}
  \begin{wrapfigure}{r}{2.5cm}
    \centering
    \qrcode[nolink]{https://data.gouv.fr/dataset/59faf660c751df1da4f5f5fc}
  \end{wrapfigure}

Licence : \textbf{Creative Commons CCZero
}\newline
Créé le : 2017-11-02\newline
Modifié le : 2017-11-02\newline
Granularité : à la commune\newline
Popularité : 1 réutilisation,  0 suivi\newline
Mots-clé : \emph{commune, france, niveau-de-vie, pauvrete, revenus, richesse
}\newline
Permalien : \url{https://data.gouv.fr/dataset/59faf660c751df1da4f5f5fc}\newline

\par
\noindent
    Données extraites de la
\href{https://www.insee.fr/fr/metadonnees/source/s1172}{base Filosofi}
de l'INSEE spécifiant le niveau de vie des Français pour chaque commune
française sur l'année 2013


\vspace{0.5cm}
\needspace{3\baselineskip} \rule{4cm}{0.25pt}\newline\textbf{Aussi disponible du même producteur :}\begin{itemize}
\item \href{https://data.gouv.fr/dataset/5a203dd7c751df1ac971203b}{mode de déplacement domicile-travail}
\item \href{https://data.gouv.fr/dataset/59fb571a88ee38602b5b3e57}{taux de motorisation des ménages}
\end{itemize}

\clearpage
\section{Havas Media}


\begin{center}
  \includegraphics[width=3cm]{images/orga/46_d9c189aac54626b3cfaaaf1822e2a8-100.png}
\end{center}


Havas Media est le réseau media du Groupe Havas, réseau présent dans 126
pays. Havas Media France, premier groupe media en France, regroupe des
agences conseil et stratégies media et déploie des dispositifs ad hoc
associant marques et contenus (stratégie, production et diffusion
nationale et internationale).

Havas Media accorde une importance particulière à la mesure des
performances, à la qualité et à la compétence des équipes, aux
développement de services en adéquation avec le marché des medias et les
attentes des annonceurs. Havas Media France a su intégrer les différents
métiers de la communication dans une approche globale et transversale et
accompagner ses clients vers le succès. La structure simplifiée et
diversifiée de Havas Media France l'impose comme l'agence de référence,
la plus intégrée et la plus réactive du marché. Pour plus
d'informations, allez sur notre site www.havasmedia.com ou suivez-nous
sur Twitter @HavasMedia.

Aujourd'hui, Havas Media souhaite partager les enseignements
structurants tirés de ses études avec le plus grand nombre. Des données
ouvertes à tous ceux qui souhaitent mieux comprendre et contribuer à
l'évolution, la révolution de notre écosystème.


\vspace{0.5cm}

\needspace{12\baselineskip}
\subsection*{Les Français et leurs données personnelles - Etude
}\index{data}\index{donnees!personnelles}\index{francais}\index{privacy}\index{protection}\index{vie!privee}
  \begin{wrapfigure}{r}{2.5cm}
    \centering
    \qrcode[nolink]{https://data.gouv.fr/dataset/542eb0c988ee387d53138113}
  \end{wrapfigure}

Licence : \textbf{Licence Ouverte
}\newline
Créé le : 2014-10-03\newline
Modifié le : 2016-02-02\newline
De 2014-08-05 à 2014-08-20\newline
Granularité : à la commune\newline
Popularité : 1 réutilisation,  0 suivi\newline
Mots-clé : \emph{data, donnees-personnelles, francais, privacy, protection, vie-privee
}\newline
Permalien : \url{https://data.gouv.fr/dataset/542eb0c988ee387d53138113}\newline

\par
\noindent
    Havas Media Group France livre les résultats de sa dernière étude qui
dresse un premier état des lieux sur le rapport des Français à la Data.
Principal enseignement, après des réactions d'inquiétude, les
comportements évoluent et ouvrent des perspectives. Si une grande
majorité des Français est conscient de la captation de ses données
personnelles et s'en déclarent inquiets, près d'un français sur deux
déclare pouvoir y trouver un intérêt, mais à certaines conditions.

Interviews online auprès d'un échantillon de 1 000 internautes français
âgés de 15-64 ans, représentatifs de la population française, menées du
05 août au 20 août 2014 par l'institut Toluna. Typologie réalisée sur
les critères d'attitudes et comportements vis-à-vis de la data.

Retrouvez l'intégralité de l'étude sur havasmediaopendata.com


\vspace{0.5cm}
\needspace{3\baselineskip} \rule{4cm}{0.25pt}\newline\textbf{Aussi disponible du même producteur :}\begin{itemize}
\item \href{https://data.gouv.fr/dataset/53698f48a3a729239d2036d3}{Baromètre POE 2014}
\item \href{https://data.gouv.fr/dataset/541c369888ee387c9de4a93c}{Social Rating Point Coupe du Monde 2014}
\end{itemize}

\clearpage
\section{Hooger SAS}


\begin{center}
  \includegraphics[width=3cm]{images/orga/2015-02-16_802d70a286a8414f93b14ed8ea7d60ea_logo_transparan-100.png}
\end{center}


Agence web / communication sur Auxerre

Développement d'un site internet et d'une application sur la ville
d'Auxerre


\vspace{0.5cm}

\needspace{12\baselineskip}
\subsection*{Délimitation des quartiers d'Auxerre
}\index{auxerre}\index{quartier!decoupage}\index{quartiers}
  \begin{wrapfigure}{r}{2.5cm}
    \centering
    \qrcode[nolink]{https://data.gouv.fr/dataset/5514825ac751df3d92703192}
  \end{wrapfigure}

Licence : \textbf{Open Data Commons Attribution License
}\newline
Créé le : 2015-03-26\newline
Modifié le : 2015-12-24\newline
Granularité : à la commune\newline
Mise à jour : semestrielle\newline
Popularité : 1 réutilisation,  0 suivi\newline
Mots-clé : \emph{auxerre, quartier-decoupage, quartiers
}\newline
Permalien : \url{https://data.gouv.fr/dataset/5514825ac751df3d92703192}\newline

\par
\noindent
    Délimitation géographique des quartiers d'habitation de la ville
d'Auxerre


\vspace{0.5cm}
\needspace{12\baselineskip}
\subsection*{Position GPS des arrêts de bus d'Auxerre
}\index{bus}\index{latitude}\index{lignes}\index{longitude}\index{positions}
  \begin{wrapfigure}{r}{2.5cm}
    \centering
    \qrcode[nolink]{https://data.gouv.fr/dataset/54e162a0c751df7b2146738a}
  \end{wrapfigure}

Licence : \textbf{Licence Ouverte
}\newline
Créé le : 2015-02-16\newline
Modifié le : 2016-02-03\newline
Granularité : à la commune\newline
Mise à jour : annuelle\newline
Popularité : 1 réutilisation,  1 suivi\newline
Mots-clé : \emph{bus, latitude, lignes, longitude, positions
}\newline
Permalien : \url{https://data.gouv.fr/dataset/54e162a0c751df7b2146738a}\newline

\par
\noindent
    Nom de l'arrêt, latitude et longitude des arrêts de la ville d'Auxerre
(Vivacité)


\vspace{0.5cm}
\needspace{12\baselineskip}
\subsection*{Position GPS des toutounets d'Auxerre
}\index{chiens}\index{dechets}\index{gps}\index{latitude}\index{longitude}\index{positions}\index{toutounet}
  \begin{wrapfigure}{r}{2.5cm}
    \centering
    \qrcode[nolink]{https://data.gouv.fr/dataset/54e16038c751df7e86467389}
  \end{wrapfigure}

Licence : \textbf{Licence Ouverte
}\newline
Créé le : 2015-02-16\newline
Modifié le : 2015-12-24\newline
De 2015-02-16 à 2015-10-16\newline
Granularité : à la commune\newline
Mise à jour : annuelle\newline
Popularité : 1 réutilisation,  0 suivi\newline
Mots-clé : \emph{chiens, dechets, gps, latitude, longitude, positions, toutounet
}\newline
Permalien : \url{https://data.gouv.fr/dataset/54e16038c751df7e86467389}\newline

\par
\noindent
    Latitude et longitude des ``toutounets'', distributeur de sacs pour
déjections canines, sur Auxerre


\vspace{0.5cm}

\clearpage
\section{IdeesLibres.org}


\begin{center}
  \includegraphics[width=3cm]{images/orga/0a_1673d8bcd4477592a76e8e5abad571-100.png}
\end{center}


OpenData et nouveaux usages
\href{http://www.ideeslibres.org}{www.ideeslibres.org}


\vspace{0.5cm}

\needspace{12\baselineskip}
\subsection*{Base de données des accidents corporels de la circulation
}\index{accidentologie}\index{accidents}
  \begin{wrapfigure}{r}{2.5cm}
    \centering
    \qrcode[nolink]{https://data.gouv.fr/dataset/53698f4da3a729239d2036ec}
  \end{wrapfigure}

Licence : \textbf{Open Data Commons Open Database License (ODbL)
}\newline
Créé le : 2014-05-05\newline
Modifié le : 2016-03-16\newline
De 2006-01-01 à 2011-12-31\newline
Granularité : à la commune\newline
Mise à jour : ponctuelle\newline
Popularité : 1 réutilisation,  7 suivis\newline
Mots-clé : \emph{accidentologie, accidents
}\newline
Permalien : \url{https://data.gouv.fr/dataset/53698f4da3a729239d2036ec}\newline

\par
\noindent
    Lors de l'OpenDataCamp organisé par Etalab et Devoxx le 16 avril 2014.
La
\href{https://www.data.gouv.fr/fr/dataset/base-de-donnees-accidents-corporels-de-la-circulation-sur-6-annees}{base
de données des accidents du Ministère de l'Intérieur} concernant la
période 2006-2011 a pu être reformatée (travail collectif) afin d'être
importable dans un SGBD classique et mis à disposition en ressource du
jeu de données initial. Le jeu données suivant en est une version
enrichie avec les codes INSEE du Geofla de 2011.

Une colonne ``geoflamatch'' indique soit ``O'', soit ``N'' pour
signifier s'il correspond ou non. Soit ``C'' lorsqu'il a du être corrigé
pour correspondre. L'outre-mer n'a pas été traité en ce sens, il est
noté ``N'' par défaut.

La mention obligatoire d'attribution doit être «
\href{http://www.ideeslibres.org/licence-odbl-1-0-fr/}{Licence ODbL} ©
\href{http://www.IdeesLibres.org}{IdeesLibres.org} 05/2014, Participants
de
l'\href{http://www.etalab.gouv.fr/article-invitation-a-l-open-data-camp-2-a-l-occasion-de-devoxx-2014-123124994.html}{OpenDataCamp}
16/04/2014,
\href{https://www.data.gouv.fr/fr/organization/ministere-de-l-interieur}{Ministère
de l'Intérieur} 04/2014 » avec les liens indiqués lors d'une publication
en ligne.


\vspace{0.5cm}
\needspace{12\baselineskip}
\subsection*{Base de données locales - Service-public.fr
}\index{cci}\index{gendarmerie}\index{mairie}\index{pole!emploi}\index{police}\index{service!public}\index{tribunaux}
  \begin{wrapfigure}{r}{2.5cm}
    \centering
    \qrcode[nolink]{https://data.gouv.fr/dataset/537d5ad5a3a72973a2dc0278}
  \end{wrapfigure}

Licence : \textbf{Open Data Commons Open Database License (ODbL)
}\newline
Créé le : 2014-05-21\newline
Modifié le : 2016-03-07\newline
De 2014-05-19 à 2014-05-19\newline
Granularité : à la commune\newline
Mise à jour : ponctuelle\newline
Popularité : 2 réutilisations,  5 suivis\newline
Mots-clé : \emph{cci, gendarmerie, mairie, pole-emploi, police, service-public, tribunaux
}\newline
Permalien : \url{https://data.gouv.fr/dataset/537d5ad5a3a72973a2dc0278}\newline

\par
\noindent
    Ce jeu de données est une extraction CSV de la base de données locales
V2 fournie à l'origine par les services du Premier Ministre en
\href{https://www.data.gouv.fr/fr/dataset/service-public-fr-annuaire-de-l-administration-base-de-donnees-locales}{format
XML} et notamment utilisée par
\href{http://www.Service-public.fr}{Service-public.fr} et le
\href{http://lecomarquage.service-public.fr/index.php}{Comarquage}.

Au 19 mai 2014, il contient 61 754 organismes publics géolocalisés plus
ou moins précisément selon les régions avec entre autres leurs horaires,
sites internet et compétences géographiques.

La documentation est disponible sur
\href{http://www.ideeslibres.org/service-public-fr-annuaire-de-ladministration-base-de-donnees-locales-v2/}{cette
page}.

La mention obligatoire d'attribution doit être «
\href{http://www.ideeslibres.org/licence-odbl-1-0-fr/}{Licence ODbL} ©
\href{http://www.IdeesLibres.org}{IdeesLibres.org} 05/2014,
\href{https://www.data.gouv.fr/fr/organization/premier-ministre}{Premier
ministre} 05/2014 » avec les liens indiqués lors d'une publication en
ligne.


\vspace{0.5cm}
\needspace{12\baselineskip}
\subsection*{Les candidats aux municipales par communes
}\index{candidats}\index{elections}\index{municipales}
  \begin{wrapfigure}{r}{2.5cm}
    \centering
    \qrcode[nolink]{https://data.gouv.fr/dataset/536997f9a3a729239d204e25}
  \end{wrapfigure}

Licence : \textbf{Open Data Commons Open Database License (ODbL)
}\newline
Créé le : 2014-03-20\newline
Modifié le : 2016-03-13\newline
De 2014-03-20 à 2014-03-21\newline
Granularité : à la commune\newline
Mise à jour : ponctuelle\newline
Popularité : 1 réutilisation,  2 suivis\newline
Mots-clé : \emph{candidats, elections, municipales
}\newline
Permalien : \url{https://data.gouv.fr/dataset/536997f9a3a729239d204e25}\newline

\par
\noindent
    Version CSV du XML du Ministère de l'Intérieur pour le 1er tour (complet
: communes de plus et moins de 1000 habitants) Plus de détails sur
\href{http://www.ideeslibres.org/blog/2014/03/20/nos-candidats-aux-municipales-2014-nuances-prenoms-et-patronymes/}{IdeesLibres.org}.

La mention obligatoire d'attribution doit être «
\href{http://www.ideeslibres.org/licence-odbl-1-0-fr/}{Licence ODbL} ©
\href{http://www.IdeesLibres.org}{IdeesLibres.org} 04/2014,
\href{https://www.data.gouv.fr/fr/organization/ministere-de-l-interieur}{Ministère
de l'Intérieur} 03/2014 » avec les liens indiqués lors d'une publication
en ligne.


\vspace{0.5cm}
\needspace{12\baselineskip}
\subsection*{Les élus municipaux (version enrichie)
}\index{elections}\index{elus}\index{municipales}
  \begin{wrapfigure}{r}{2.5cm}
    \centering
    \qrcode[nolink]{https://data.gouv.fr/dataset/53699831a3a729239d204eb9}
  \end{wrapfigure}

Licence : \textbf{Open Data Commons Open Database License (ODbL)
}\newline
Créé le : 2014-04-09\newline
Modifié le : 2016-03-14\newline
De 2014-03-30 à 2014-03-30\newline
Granularité : à la commune\newline
Mise à jour : ponctuelle\newline
Popularité : 5 réutilisations,  6 suivis\newline
Mots-clé : \emph{elections, elus, municipales
}\newline
Permalien : \url{https://data.gouv.fr/dataset/53699831a3a729239d204eb9}\newline

\par
\noindent
    Ce jeu de données a pu être construit grâce au croisement de ces trois
jeux de données :

\url{https://www.data.gouv.fr/fr/dataset/les-elus-municipaux}{]}(https://www.data.gouv.fr/fr/dataset/les-elus-municipaux{]}(https://www.data.gouv.fr/fr/dataset/les-elus-municipaux))
\url{https://www.data.gouv.fr/fr/dataset/les-resultats-du-premier-tour-des-municipales}{]}(https://www.data.gouv.fr/fr/dataset/les-resultats-du-premier-tour-des-municipales{]}(https://www.data.gouv.fr/fr/dataset/les-resultats-du-premier-tour-des-municipales))
\url{https://www.data.gouv.fr/fr/dataset/les-resultats-du-deuxieme-tour-des-municipales}{]}(https://www.data.gouv.fr/fr/dataset/les-resultats-du-deuxieme-tour-des-municipales{]}(https://www.data.gouv.fr/fr/dataset/les-resultats-du-deuxieme-tour-des-municipales))
Une
\href{http://www.ideeslibres.org/Municipales-2014/Nuances-et-Territoires/\#bbox=48.70365031117296,2.0537567138671875,48.97390736160892,2.991714477539062\&dpt=1\&lgd=1}{application
cartographique} a également été réalisée.

\textbf{Mise à jour :} Pour permettre une visualisation des alliances,
le jeu de données indique désormais les listes (liblisextinitiale) et
nuances initiales (codnualisteinitiale) des élus lorsque celles ci sont
différentes de celles du tour d'élection (liblisext et codnualiste).

La mention obligatoire d'attribution doit être «
\href{http://www.ideeslibres.org/licence-odbl-1-0-fr/}{Licence ODbL} ©
\href{http://www.IdeesLibres.org}{IdeesLibres.org} 04/2014,
\href{https://www.data.gouv.fr/fr/organization/ministere-de-l-interieur}{Ministère
de l'Intérieur} 04/2014 » avec les liens indiqués lors d'une publication
en ligne.


\vspace{0.5cm}
\needspace{12\baselineskip}
\subsection*{Les prénoms des conseillers municipaux
}\index{communales}\index{elections}\index{elus}\index{municipales}
  \begin{wrapfigure}{r}{2.5cm}
    \centering
    \qrcode[nolink]{https://data.gouv.fr/dataset/5369986da3a729239d204f60}
  \end{wrapfigure}

Licence : \textbf{Open Data Commons Open Database License (ODbL)
}\newline
Créé le : 2014-03-09\newline
Modifié le : 2016-02-13\newline
De 2008-03-09 à 2013-09-30\newline
Granularité : au pays\newline
Mise à jour : ponctuelle\newline
Popularité : 10 réutilisations,  3 suivis\newline
Mots-clé : \emph{communales, elections, elus, municipales
}\newline
Permalien : \url{https://data.gouv.fr/dataset/5369986da3a729239d204f60}\newline

\par
\noindent
    Les prénoms des conseillers municipaux en France, leur genre et leur
décompte. Couverture temporelle des élus de Paris étendue à mars 2014.
\href{http://www.ideeslibres.org/blog/2014/03/07/les-prenoms-de-nos-conseillers-municipaux/}{Plus
de détails sur IdeesLibres.org}.


\vspace{0.5cm}
\needspace{12\baselineskip}
\subsection*{Les prénoms des conseillers municipaux agrégés par departements
}\index{communales}\index{departements}\index{elections}\index{elus}\index{parite}\index{prenoms}
  \begin{wrapfigure}{r}{2.5cm}
    \centering
    \qrcode[nolink]{https://data.gouv.fr/dataset/5369986da3a729239d204f63}
  \end{wrapfigure}

Licence : \textbf{Open Data Commons Open Database License (ODbL)
}\newline
Créé le : 2014-03-01\newline
Modifié le : 2015-11-23\newline
De 2008-03-09 à 2013-09-30\newline
Granularité : au département\newline
Mise à jour : ponctuelle\newline
Popularité : 2 réutilisations,  1 suivi\newline
Mots-clé : \emph{communales, departements, elections, elus, parite, prenoms
}\newline
Permalien : \url{https://data.gouv.fr/dataset/5369986da3a729239d204f63}\newline

\par
\noindent
    Extrait du répertoire national des élus du Ministère de l'Intérieur.
Couverture temporelle étendue à mars 2014 pour les élus de Paris.
\href{http://www.ideeslibres.org/blog/2014/03/07/les-prenoms-de-nos-conseillers-municipaux/}{Plus
de détails sur IdeesLibres.org}. Existe aussi
\href{https://www.data.gouv.fr/dataset/les-prenoms-des-conseillers-municipaux}{sans
agrégation}


\vspace{0.5cm}
\needspace{12\baselineskip}
\subsection*{Stations services en France
}\index{stations!services}
  \begin{wrapfigure}{r}{2.5cm}
    \centering
    \qrcode[nolink]{https://data.gouv.fr/dataset/541d772d88ee38091a191ed5}
  \end{wrapfigure}

Licence : \textbf{Open Data Commons Open Database License (ODbL)
}\newline
Créé le : 2014-09-20\newline
Modifié le : 2016-03-16\newline
Mise à jour : ponctuelle\newline
Popularité : 4 réutilisations,  4 suivis\newline
Mots-clé : \emph{stations-services
}\newline
Permalien : \url{https://data.gouv.fr/dataset/541d772d88ee38091a191ed5}\newline

\par
\noindent
    Ce jeu de données est construit à partir du
\href{https://www.data.gouv.fr/fr/datasets/prix-des-carburants-en-france/}{flux
XML} de l'année 2014 (j-7\ldots{}) du Ministère de l'Economie
(http://donnees.roulez-eco.fr/opendata/annee). Il contient les stations
services (sous forme de points d'intérêts) dont le volume de vente à
l'année est supérieur à cinq cents mètres cubes (tous produits
confondus).

Le champ ``carburants'' est généré selon qu'il existe des tarifs
déclarés pour chacun de ceux-ci et tient compte de leurs dates de
ruptures (s'il y en a). Si une rupture est indiquée et qu'un prix est
ensuite mentionné à une date ultérieure pour ce type de carburant, ce
carburant sera présent. Si aucun prix n'est mentionné ultérieurement, ce
type de carburant ne sera pas inclus. Ce champ est donc fiable au moment
de la génération du CSV.

Le champ ``services'' n'est pas complétement fiable, au moins pour le
GPL. Après plusieurs vérifications, si ce champ contient ``GPL'' mais
que le champ ``carburants'' ne le contient pas, on peut considérer que
la station n'en vend pas. L'obligation légale d'alimentation du fichier
concernant uniquement les prix à la pompe, la liste des autres services
proposés peut ne pas être à jour.

On ne peux donc pas non plus faire complètement confiance aux horaires
d'ouverture. Le modèle du XML source ne se prêtant d'ailleurs pas à la
précision. Par exemple, un établissement est indiqué comme étant ouvert
de 8h à 18h sauf le samedi et le dimanche. Dans la réalité les horaires
de cet établissement sont : de 8h à 12h et de 14h à 18h du lundi au
vendredi, de 8h30 à 12h30 le samedi et fermeture le dimanche.

Heureusement, beaucoup de pompes sont aujourd'hui équipées en carte
bleue. Mais malheureusement cette information n'est pas précisée dans le
XML.

\textbf{Liste des champs et significations :}

``id'' : identifiant du Ministère

``latlng'' : coordonnées GPS de la forme ``latitude,longitude''

``typeroute'' : ``A'' pour autoroute et ``R'' pour route

``adresse'' : adresse dont les retours chariot sont matérialisés par le
caractère ``\textbar{}''

``commune'' : nom de la commune

``codepostal'' : code postal de la commune

``hdebut'' : heure d'ouverture

``hfin'' : heure de fermeture

``saufjour'' : les jours ou les horaires précités ne s'appliquent pas

``services'' : liste des services proposés séparés par le caractère
``\textbar{}''

``carburants'' : liste des carburants disponibles séparés par le
caractère ``\textbar{}''

``activite'' : ``O'' si l'établissement est en activité, ``N'' dans le
cas contraire et ``T'' pour une fermeture temporaire (certains le sont
depuis 2009). Ce champ résulte du même type de traitement que pour les
ruptures, il est donc fiable au moment de la génération du CSV.


\vspace{0.5cm}
\needspace{12\baselineskip}
\subsection*{Thématique: élections municipales
}\index{candidats}\index{elections}\index{municipales}\index{nuances}\index{referentiel}\index{resultats}
  \begin{wrapfigure}{r}{2.5cm}
    \centering
    \qrcode[nolink]{https://data.gouv.fr/dataset/5369a228a3a729239d20678e}
  \end{wrapfigure}

Licence : \textbf{Open Data Commons Open Database License (ODbL)
}\newline
Créé le : 2014-03-28\newline
Modifié le : 2016-03-12\newline
De 2014-03-20 à 2014-03-30\newline
Granularité : à la commune\newline
Mise à jour : ponctuelle\newline
Popularité : 1 réutilisation,  3 suivis\newline
Mots-clé : \emph{candidats, elections, municipales, nuances, referentiel, resultats
}\newline
Permalien : \url{https://data.gouv.fr/dataset/5369a228a3a729239d20678e}\newline

\par
\noindent
    Résultats, élus, candidats, référentiel et définitions des nuances
politiques. Cette page recense les jeux de données produits par
\href{http://www.ideeslibres.org}{IdeesLibres.org} principalement sur la
base du XML fourni par le Ministère de l'Intérieur.

La mention obligatoire d'attribution doit être «
\href{http://www.ideeslibres.org/licence-odbl-1-0-fr/}{Licence ODbL} ©
\href{http://www.IdeesLibres.org}{IdeesLibres.org} 04/2014,
\href{https://www.data.gouv.fr/fr/organization/ministere-de-l-interieur}{Ministère
de l'Intérieur} 03/2014 » avec les liens indiqués lors d'une publication
en ligne.


\vspace{0.5cm}
\needspace{3\baselineskip} \rule{4cm}{0.25pt}\newline\textbf{Aussi disponible du même producteur :}\begin{itemize}
\item \href{https://data.gouv.fr/dataset/54abdee9c751df2fcd04805b}{Données carroyées à 200m sur la population}
\item \href{https://data.gouv.fr/dataset/53699879a3a729239d204f85}{Les résultats du deuxième tour des municipales}
\item \href{https://data.gouv.fr/dataset/53699879a3a729239d204f87}{Les résultats du premier tour des municipales}
\item \href{https://data.gouv.fr/dataset/5369995ca3a729239d205224}{Listes et candidats aux élections européennes}
\item \href{https://data.gouv.fr/dataset/53699af3a3a729239d2055f5}{Nomenclature des catégories socio-professionnelles du Ministère de l'Intérieur (codes CSP)}
\item \href{https://data.gouv.fr/dataset/53699affa3a729239d205613}{Nuances politiques}
\item \href{https://data.gouv.fr/dataset/53699affa3a729239d205614}{Nuancier politique du Ministère de l'Intérieur}
\item \href{https://data.gouv.fr/dataset/53699bc3a3a729239d2057fa}{Participation aux municipales}
\item \href{https://data.gouv.fr/dataset/53699ed0a3a729239d205f92}{Référentiel des municipales - Appartenance des communes aux EPCI - Modes scrutins - Population}
\end{itemize}

\clearpage
\section{Infolocale}


\begin{center}
  \includegraphics[width=3cm]{images/orga/b2_be284ab67e425d8bca1660f8251621-100.png}
\end{center}


Infolocale est un outil de collecte et de mise à disposition de contenu
événementiel culturel.

Chaque année, 600 000 annonces sont transmises et publiées gratuitement
dans les médias locaux, et mises à disposition des réutilisateurs
sur\url{http://datainfolocale.opendatasoft.com}

Les annonces d'événements sont saisies sur Infolocale par : - 60 000
associations - 4 500 communes - 15 000 lieux culturels, de loisirs et
sportifs

Les données sont structurées, géo-localisées, modérées et corrigées
(orthographe, typographie).


\vspace{0.5cm}

\needspace{12\baselineskip}
\subsection*{Données événementielles Infolocale
}\index{agenda}\index{concert}\index{culture}\index{expositions}\index{opendata}\index{spectacle}
  \begin{wrapfigure}{r}{2.5cm}
    \centering
    \qrcode[nolink]{https://data.gouv.fr/dataset/53698e85a3a729239d2034b2}
  \end{wrapfigure}

Licence : \textbf{Open Data Commons Open Database License (ODbL)
}\newline
Créé le : 2013-12-30\newline
Modifié le : 2016-07-26\newline
De 2014-10-01 à 2020-01-01\newline
Granularité : à la commune\newline
Mise à jour : semestrielle\newline
Popularité : 2 réutilisations,  0 suivi\newline
Mots-clé : \emph{agenda, concert, culture, expositions, opendata, spectacle
}\newline
Permalien : \url{https://data.gouv.fr/dataset/53698e85a3a729239d2034b2}\newline

\par
\noindent
    Le jeu de données contient les annonces d'événements loisirs et
pratiques saisies par les organismes inscrits sur Infolocale. Il est mis
à jour quotidiennement.


\vspace{0.5cm}
\needspace{3\baselineskip} \rule{4cm}{0.25pt}\newline\textbf{Aussi disponible du même producteur :}\begin{itemize}
\item \href{https://data.gouv.fr/dataset/5c374cef06e3e766cf2594c6}{Données événementielles Infolocale (2019)}
\item \href{https://data.gouv.fr/dataset/5c374cee9ce2e74194a40e95}{Données événementielles Infolocale (année 2018)}
\item \href{https://data.gouv.fr/dataset/5bf1ef939ce2e702e40059b7}{Données événementielles Infolocale (échantillon)}
\end{itemize}

\clearpage
\section{Inria}


\begin{center}
  \includegraphics[width=3cm]{images/orga/8b_64b921d7674753ad7d0308ea4895ca-100.jpg}
\end{center}


Créé en 1967, Inria est le seul institut public de recherche entièrement
dédié aux sciences du numérique. A l'interface des sciences
informatiques et des mathématiques, les 3400 chercheurs d'Inria
inventent les technologies numériques de demain.


\vspace{0.5cm}

\needspace{12\baselineskip}
\subsection*{Barometre Inria 2014 - Les Français et le numerique
}\index{economie!numerique}\index{education}\index{innovation}\index{internet}\index{mooc}\index{numerique}\index{sciences!du!numerique}\index{societe!numerique}\index{vie!privee}
  \begin{wrapfigure}{r}{2.5cm}
    \centering
    \qrcode[nolink]{https://data.gouv.fr/dataset/53698f47a3a729239d2036cf}
  \end{wrapfigure}

Licence : \textbf{Licence Ouverte
}\newline
Créé le : 2014-03-10\newline
Modifié le : 2016-01-27\newline
De 2013-11-28 à 2013-12-02\newline
Mise à jour : annuelle\newline
Popularité : 3 réutilisations,  0 suivi\newline
Mots-clé : \emph{economie-numerique, education, innovation, internet, mooc, numerique, sciences-du-numerique, societe-numerique, vie-privee
}\newline
Permalien : \url{https://data.gouv.fr/dataset/53698f47a3a729239d2036cf}\newline

\par
\noindent
    Résultats de la 2ème édition du baromètre Inria TNS-Sofres sur les
Français et le numérique. 1145 personnes de 14 ans et plus ont été
interrogées en face à face, dans toute la France, entre les 28 novembre
et 2 décembre 2013.

Le Baromètre met en lumière les relations des Français au numérique,
avec des focus spécifiques sur l'économie numérique et l'éducation
numérique. Les résultats montrent une accélération dans l'appropriation
des usages, mais aussi une prise de conscience des responsabilités qui
vont de pair.


\vspace{0.5cm}
\needspace{3\baselineskip} \rule{4cm}{0.25pt}\newline\textbf{Aussi disponible du même producteur :}\begin{itemize}
\item \href{https://data.gouv.fr/dataset/53788ad6a3a7295dd332d9ce}{Barometre Inria 2011 - Les Français et le nouveau monde numerique}
\end{itemize}

\clearpage
\section{Institut des politiques publiques (IPP)}


\begin{center}
  \includegraphics[width=3cm]{images/orga/ce_6bad50cd3c4d28ad58d5ace8413b56-100.png}
\end{center}


L'\href{http://www.ipp.eu}{Institut des politiques publiques} (IPP) est
développé dans le cadre d'un partenariat scientifique conclu par
\href{http://parisschoolofeconomics.eu}{PSE-École d'Économie de Paris}
et le \href{http://www.crest.fr/}{Centre de Recherche en Économie et
Statistique} (CREST). L'IPP vise à promouvoir l'analyse et l'évaluation
quantitatives des politiques publiques en s'appuyant sur les méthodes
les plus récentes de la recherche en économie. Les travaux conduits par
l'IPP ont pour objectif de développer la recherche scientifique dans le
domaine des politiques publiques et de favoriser l'appropriation par les
citoyens des termes du débat public.


\vspace{0.5cm}

\needspace{12\baselineskip}
\subsection*{Barèmes IPP - système social et fiscal français
}
  \begin{wrapfigure}{r}{2.5cm}
    \centering
    \qrcode[nolink]{https://data.gouv.fr/dataset/53698f42a3a729239d2036be}
  \end{wrapfigure}

Licence : \textbf{Licence Ouverte
}\newline
Créé le : 2014-04-15\newline
Modifié le : 2016-03-14\newline
De 1915-01-01 à 2014-12-31\newline
Popularité : 6 réutilisations,  5 suivis\newline
Mots-clé : \emph{aucun
}\newline
Permalien : \url{https://data.gouv.fr/dataset/53698f42a3a729239d2036be}\newline

\par
\noindent
    L'Institut des politiques publiques (IPP) s'est donné pour mission de
rassembler l'ensemble de la législation des politiques publiques en
France dans une perspective historique et scientifique. L'objectif est
d'en faciliter l'évaluation, l'analyse et la diffusion. En premier lieu,
\href{http://www.ipp.eu/fr/outils/baremes-ipp/}{ces barèmes législatifs}
sont utilisés dans le modèle de microsimulation du système socio-fiscal
de l'IPP, \href{http://www.ipp.eu/fr/outils/taxipp-simulation/}{TAXIPP}.

Ce projet implique un travail de collecte considérable de sources
diverses et précises, jamais à ce jour rassemblées de façon cohérente.
L'IPP met à disposition sur cette page les premiers documents qui
présentent les barèmes du système socio-fiscal français. Les références
législatives (texte de loi, numéro du décret, arrêté ou accords des
partenaires sociaux) ainsi que la date de publication au Journal
Officiel de la République Française (JORF) sont indiquées dans la mesure
du possible. Les barèmes remontent à 1914 pour l'impôt sur le revenu, à
1945 dans certains cas (cotisations sociales, système de retraite) et
aux années 1980 pour la plupart des prestations.


\vspace{0.5cm}

\clearpage
\section{Institut national de l'audiovisuel}


\begin{center}
  \includegraphics[width=3cm]{images/orga/6f_55abd72223498eabffb11d8c81e31f-100.jpg}
\end{center}


Créé en 1975, l'Institut national de l'audiovisuel, entreprise publique
résolument engagée dans le XXIe siècle, collecte et conserve 80 ans de
fonds radiophoniques et 70 ans de programmes de télévision qui fondent
notre mémoire collective. Il les valorise et leur donne sens pour les
partager avec le plus large public en France et à l'étranger. Ses images
et ses sons sont accessibles, pour partie, sur son site grand public
ina.fr et dans leur totalité, dans ses centres de consultation Inathèque
au titre du dépôt légal. Ils sont aussi mis au service de la production
et de la diffusion de programmes, de l'édition, de l'éducation par
l'image et de l'animation culturelle.

L'Ina concentre des compétences d'expertise, une vocation d'observatoire
des médias, au service de l'excellence et de l'innovation. L'Institut
est l'un des premiers centres de formation initiale et continue aux
métiers de l'audiovisuel et des nouveaux médias et s'affirme comme un
laboratoire de recherche et d'expérimentation.


\vspace{0.5cm}

\needspace{12\baselineskip}
\subsection*{Classement thématique des sujets de journaux télévisés (janvier 2005 -
juin 2018)
}\index{audiovisuel}\index{television}
  \begin{wrapfigure}{r}{2.5cm}
    \centering
    \qrcode[nolink]{https://data.gouv.fr/dataset/57fe1446c751df182f79df72}
  \end{wrapfigure}

Licence : \textbf{Licence Ouverte
}\newline
Créé le : 2016-10-12\newline
Modifié le : 2018-10-20\newline
De 2015-01-01 à 2018-06-30\newline
Mise à jour : ponctuelle\newline
Popularité : 1 réutilisation,  3 suivis\newline
Mots-clé : \emph{audiovisuel, television
}\newline
Permalien : \url{https://data.gouv.fr/dataset/57fe1446c751df182f79df72}\newline

\par
\noindent
    Classement thématique des sujets diffusés sur les journaux télévisés du
soir de six chaînes (TF1, France 2, France 3, Canal +, Arte, M6) pour la
période janvier 2005 à juin 2018.

L'Ina indexe depuis 1995 la totalité des sujets diffusés sur les
journaux télévisés du soir des six chaînes dites « historiques » (TF1,
France 2, France 3, Canal +, Arte, M6), par une addition de mots clés
appartenant à un thésaurus. A partir de cette indexation, des
algorithmes thématisent les sujets entre quatorze rubriques. Un sujet
est classé sous une rubrique et une seule.

Le détail de la méthodologie d'indexation et de rubricage des sujets est
présenté sur le site internet de l'Inathèque
:\url{http://www.inatheque.fr/publications-evenements/ina-stat/ina-stat-methodologie.html.}


\vspace{0.5cm}
\needspace{12\baselineskip}
\subsection*{Temps de parole des hommes et des femmes à la télévision et à la radio
}\index{acoustique}\index{apprentissage!artificiel}\index{audiovisuel}\index{audiovisuel!public}\index{egalite!femme!homme}\index{egalite!femmes!hommes}\index{egalite!homme!femme}\index{egalite!hommes!femmes}\index{intelligence!artificielle}\index{musique}\index{parole}\index{radio}\index{taux!d!expression!des!femmes}\index{television}\index{temps!de!parole}\index{voix}
  \begin{wrapfigure}{r}{2.5cm}
    \centering
    \qrcode[nolink]{https://data.gouv.fr/dataset/5c6adbae634f4114a5c41776}
  \end{wrapfigure}

Licence : \textbf{Licence Ouverte
}\newline
Créé le : 2019-02-18\newline
Modifié le : 2019-03-12\newline
De 1995-01-01 à 2019-02-28\newline
Popularité : 4 réutilisations,  2 suivis\newline
Mots-clé : \emph{acoustique, apprentissage-artificiel, audiovisuel, audiovisuel-public, egalite-femme-homme, egalite-femmes-hommes, egalite-homme-femme, egalite-hommes-femmes, intelligence-artificielle, musique, parole, radio, taux-d-expression-des-femmes, television, temps-de-parole, voix
}\newline
Permalien : \url{https://data.gouv.fr/dataset/5c6adbae634f4114a5c41776}\newline

\par
\noindent
    Temps de parole des hommes et des femmes, correspondant à plus d'un
million d'heures de programmes diffusés de 1995 au 28 février 2019.

Les temps de parole sont obtenus automatiquement à l'aide du logiciel
libre
\href{http://github.com/ina-foss/inaSpeechSegmenter}{inaSpeechSegmenter},
développé à l'INA. Ce logiciel est basé sur des algorithmes
d'apprentissage automatique (sous-famille de l'intelligence
artificielle) entraînés sur un grand nombre d'exemples de musique, de
voix de femme et de voix d'homme, afin de détecter les zones de musique
et les zones de parole contenues dans les documents audiovisuels.

Les temps de parole sont estimés par tranches d'une heure, de 5h à
minuit pour la radio et de 10h à minuit pour la télévision.

Les 21 stations de radio analysées sont Chérie FM, Europe 1, France
Bleu, France Culture, France Info, France Inter, France Musique, Fun
Radio, Mouv', NRJ, Nostalgie, RFM, RMC, RTL, RTL 2, Radio Classique,
Radio France Internationale, Rire et Chansons, Skyrock, Sud Radio et
Virgin Radio.

Les 34 chaînes de TV analysées sont Arte, Animaux, BFM TV, Canal+,
Canal+ Sport, Chasse et pêche, Chérie 25, Comédie+, D8/C8, Euronews,
Eurosport France, France 2, France 24, France 3, France 5, France O,
Histoire, I-Télé/CNews, L'Equipe 21, LCI, LCP/Public Sénat, La chaîne
Météo, M6, Monte Carlo TMC, NRJ 12, Paris Première, Planète+, TF1, TV
Breizh, TV5 Monde, Toute l'Histoire, Téva, Voyage, W9

Le fonctionnement du logiciel d'analyse automatique utilisé sur les
documents audiovisuels, ainsi que l'estimation de sa fiabilité, sont
décrits dans l'article ci-dessous:

David Doukhan et Jean Carrive,
``\href{https://www.isca-speech.org/archive/JEP_2018/pdfs/192838.pdf}{Description
automatique du taux d'expression des femmes dans les flux télévisuels
français}'', XXXIIe Journées d'Études sur la Parole, Juin 2018


\vspace{0.5cm}

\clearpage
\section{La République En Marche !}


\begin{center}
  \includegraphics[width=3cm]{images/orga/e2_aceeaeef724544839ac48e2ae2732c-100.jpg}
\end{center}


La République En Marche ! est un mouvement politique et citoyen français
lancé le 6 avril 2016 par Emmanuel Macron.

Depuis notre création, nous organisons des grandes marches,
consultations et ateliers de réflexion avec les Françaises et les
Français afin de réfléchir ensemble à l'avenir du pays et de l'Europe.

Notre site internet est en open-source, et dans cette lignée nous
souhaitons partager un certain nombre de nos enquêtes avec le plus grand
nombre afin de démultiplier leur impact.

À bientôt sur \href{https://en-marche.fr}{en-marche.fr} !


\vspace{0.5cm}

\needspace{12\baselineskip}
\subsection*{Les consultations citoyennes
}\index{citoyen}\index{consultation}\index{enquete}\index{expertise}\index{loi}\index{opinion!publique}\index{reforme}\index{vie!publique}
  \begin{wrapfigure}{r}{2.5cm}
    \centering
    \qrcode[nolink]{https://data.gouv.fr/dataset/5a981a1ac751df07a20cc5c2}
  \end{wrapfigure}

Licence : \textbf{Other (Public Domain)
}\newline
Créé le : 2018-03-01\newline
Modifié le : 2018-03-02\newline
Mise à jour : ponctuelle\newline
Popularité : 1 réutilisation,  1 suivi\newline
Mots-clé : \emph{citoyen, consultation, enquete, expertise, loi, opinion-publique, reforme, vie-publique
}\newline
Permalien : \url{https://data.gouv.fr/dataset/5a981a1ac751df07a20cc5c2}\newline

\par
\noindent
    La République En Marche porte une \textbf{vision pragmatique de la
politique}. Nous avons la volonté de toujours partir de l'expérience des
Français, de leurs préoccupations, des problèmes qu'ils rencontrent, des
solutions qu'ils identifient.

Dans le cadre de la campagne présidentielle et grâce à nos comités
locaux, les adhérents d'En Marche ont participé à l'élaboration du
programme via la Grande Marche, l'organisation de marches thématiques
(``la marche des campagnes'', ``la marche des quartiers'', \ldots{}),
des consultations organisées sur des points du programmes, et les
kiosques organisés sur le terrain. De nombreuses orientations du
programme présidentiel ont ainsi été enrichies.

\textbf{Notre mouvement continue à développer ces nouveaux modes de
participation à la vie publique avec les consultations citoyennes}

Vous aimez comprendre, apprendre, débattre sur des les sujets variés du
monde qui nous entoure ? Vous avez des idées que vous voulez partager et
faire remonter aux parlementaires ou au gouvernement ? Les consultations
citoyennes permettent de faire remonter les expériences et les
réflexions issues du terrain.

\textbf{L'ensemble des résultats de nos consultations sont disponibles
en open-source ici}


\vspace{0.5cm}
\needspace{3\baselineskip} \rule{4cm}{0.25pt}\newline\textbf{Aussi disponible du même producteur :}\begin{itemize}
\item \href{https://data.gouv.fr/dataset/5bab7ea58b4c4129237253e5}{Restitution de la Grande Marche pour l'Europe}
\end{itemize}

\clearpage
\section{Le Figaro}


\begin{center}
  \includegraphics[width=3cm]{images/orga/fb_5ed5f7b1b74da999f644571f99774f-100.jpg}
\end{center}


Le Figaro, premier quotidien généraliste national, est devenu depuis
plusieurs années un acteur de référence dans l'univers du numérique.
Lefigaro.fr est le leader des sites de presse en ligne avec près de 18
millions de visiteurs uniques par mois. Avec le rachat du groupe CCM
Benchmark, le Groupe Figaro est le premier groupe média digital
français. Son audience globale sur tous les canaux digitaux dépasse les
32 millions de visiteurs uniques.


\vspace{0.5cm}

\needspace{12\baselineskip}
\subsection*{Niveau de vie des Français par commune
}\index{communes}\index{france}\index{niveau!de!vie}
  \begin{wrapfigure}{r}{2.5cm}
    \centering
    \qrcode[nolink]{https://data.gouv.fr/dataset/59f89adf88ee381016f69c0c}
  \end{wrapfigure}

Licence : \textbf{Creative Commons CCZero
}\newline
Créé le : 2017-10-31\newline
Modifié le : 2017-10-31\newline
Granularité : à la commune\newline
Mise à jour : annuelle\newline
Popularité : 1 réutilisation,  0 suivi\newline
Mots-clé : \emph{communes, france, niveau-de-vie
}\newline
Permalien : \url{https://data.gouv.fr/dataset/59f89adf88ee381016f69c0c}\newline

\par
\noindent
    L'Insee a publié les niveaux de vie des ménages par commune pour l'année
2014. Le dispositif d'analyse, appelé Filosofi, permet de détailler où
se situent les zones de pauvreté en France.


\vspace{0.5cm}

\clearpage
\section{Le Moulin Digital}


\begin{center}
  \includegraphics[width=3cm]{images/orga/18_636973b21f411cacb85970609cc521-100.jpg}
\end{center}


L'Association le Moulin Digital accompagne la transition numérique de
tous les métiers.


\vspace{0.5cm}

\needspace{12\baselineskip}
\subsection*{Emplacement des hotspots Wifi Cigale
}\index{cigale}\index{hotspots}\index{wifi!public}
  \begin{wrapfigure}{r}{2.5cm}
    \centering
    \qrcode[nolink]{https://data.gouv.fr/dataset/567c060288ee385ffaaf0bf4}
  \end{wrapfigure}

Licence : \textbf{Licence Ouverte
}\newline
Créé le : 2015-12-24\newline
Modifié le : 2016-07-07\newline
Granularité : au point d'intérêt\newline
Mise à jour : quotienne\newline
Popularité : 1 réutilisation,  0 suivi\newline
Mots-clé : \emph{cigale, hotspots, wifi-public
}\newline
Permalien : \url{https://data.gouv.fr/dataset/567c060288ee385ffaaf0bf4}\newline

\par
\noindent
    Emplacement des \href{http://www.cigale-hotspot.fr/}{hotspots Wifi
gratuit Cigale} avec reconnexion automatique.


\vspace{0.5cm}
\needspace{3\baselineskip} \rule{4cm}{0.25pt}\newline\textbf{Aussi disponible du même producteur :}\begin{itemize}
\item \href{https://data.gouv.fr/dataset/58d4ea1788ee3871dccd1164}{Evénements du numérique organisés par l’association le Moulin Digital}
\item \href{https://data.gouv.fr/dataset/58d4faa888ee3811d9d3f9b5}{Evénements French Tech in the Alps}
\item \href{https://data.gouv.fr/dataset/567c087c88ee385ff9af0bf4}{Statistiques d'utilisation du réseau de hotspots Wifi Cigale par jour.}
\item \href{https://data.gouv.fr/dataset/577e7d2e88ee380ad40ddd2c}{Typologie des utilisateurs du réseau de hotspots Wifi Cigale : appareil, navigateur, nationalité}
\end{itemize}

\clearpage
\section{Mapotempo}


\begin{center}
  \includegraphics[width=3cm]{images/orga/3d_fab77a20de4b68938f9771d121d25f-100.png}
\end{center}


Mapotempo édite des logiciels libre web de planification et
d'optimisation de tournées de livraison.


\vspace{0.5cm}

\needspace{12\baselineskip}
\subsection*{Contours détaillés des circonscriptions des législatives
}\index{carte}\index{circonscriptions!legislatives}\index{election}\index{elections!legislatives}\index{geojson}\index{openstreetmap}
  \begin{wrapfigure}{r}{2.5cm}
    \centering
    \qrcode[nolink]{https://data.gouv.fr/dataset/59131e7088ee3809491e322b}
  \end{wrapfigure}

Licence : \textbf{Open Data Commons Open Database License (ODbL)
}\newline
Créé le : 2017-05-10\newline
Modifié le : 2017-05-12\newline
Mise à jour : ponctuelle\newline
Popularité : 1 réutilisation,  0 suivi\newline
Mots-clé : \emph{carte, circonscriptions-legislatives, election, elections-legislatives, geojson, openstreetmap
}\newline
Permalien : \url{https://data.gouv.fr/dataset/59131e7088ee3809491e322b}\newline

\par
\noindent
    Contours détaillés des circonscriptions législatives (définition 2012)
construit depuis les données existantes dans OpenStreetMap avec un
niveau de détail approchant l'adresse. Les données manquantes pour 25
centres-villes sont complétés par le découpage proposé par le site
\href{http://www.toxicode.fr/circonscriptions}{Toxicode}.

Pour les élections législatives, le découpage est réalisé par
circonscriptions qui sont elles-mêmes un regroupement de cantons. Or,
pour les circonscriptions législatives l'ancien découpage des cantons
est toujours utilisé. Pour créer des données précises et détaillées du
découpage des circonscriptions législatives des contributeurs
d'OpenStreetMap France ont donc menés de nouveaux travaux. Pour ce
faire, ils ont procédé comme suit~:

\begin{itemize}

\item
  trouver les descriptions de l'ancien découpage des cantons au sein de
  diverses publications du Journal Officiel,
\item
  interpréter correctement ces descriptions,
\item
  regrouper l'ensemble des descriptions trouvées,
\item
  intégrer ces descriptions à OpenStreetMap afin de créer les données
  relatives au découpage des circonscriptions.
\end{itemize}

À partir de ces travaux réalisés par les contributeurs d'OpenStreetMap
nous avons pu créer le jeu de données présentant les contours des
circonscriptions législatives à un niveau de détail très fin,
s'approchant de l'adresse. Puisqu'il nous manquait des données pour une
vingtaine de centre-ville nous avons complété ce découpage grâce aux
travaux de Toxicode. Finalement, nous avons pu déterminer avec
exactitude la répartition des panneaux d'affichage électoraux entre
chaque circonscription.


\vspace{0.5cm}
\needspace{3\baselineskip} \rule{4cm}{0.25pt}\newline\textbf{Aussi disponible du même producteur :}\begin{itemize}
\item \href{https://data.gouv.fr/dataset/5b27873ac751df05a88224a2}{Pyramide de tuiles depuis la BD Ortho®}
\end{itemize}

\clearpage
\section{Master 2 Droit du Numérique Entreprise/ Administration }


\textbf{Master 2 Professionnel Droit du numérique - Administration -
Entreprise}

L'enseignement vise à la formation de juristes aptes au traitement des
questions posées par le développement des technologies de l'information
et de la communication, tant dans le domaine de la vie administrative et
des relations entre administrations, usagers et citoyens, que dans le
domaine de la vie des entreprises dans leurs relations avec les services
publics et les collectivités territoriales par le truchement de la
numérisation des procédures (téléservices, marchés publics) ou entre
elles par le biais des relations contractuelles et des places de marché
numériques ou avec leurs clients.


\vspace{0.5cm}

\needspace{12\baselineskip}
\subsection*{Clauses données personnelles des CGU
}\index{cgu}\index{donnees!personnelles}\index{open!personal!data}
  \begin{wrapfigure}{r}{2.5cm}
    \centering
    \qrcode[nolink]{https://data.gouv.fr/dataset/559a874fc751df17f0390bd4}
  \end{wrapfigure}

Licence : \textbf{Creative Commons Attribution Share-Alike
}\newline
Créé le : 2015-07-06\newline
Modifié le : 2016-01-24\newline
Popularité : 1 réutilisation,  0 suivi\newline
Mots-clé : \emph{cgu, donnees-personnelles, open-personal-data
}\newline
Permalien : \url{https://data.gouv.fr/dataset/559a874fc751df17f0390bd4}\newline

\par
\noindent
    Dans le cadre d'un Legal Hackathon Open Law mené sur mars et avril 2015
avec les promos 2015 du
\href{http://legal-innovation-paris.com/fr-FR/diu-dei/presentation-du-diu}{DIU
Informatique} et Droit et du
\href{http://www.univ-paris1.fr/ws/ws.php?_cmd=getFormation\&_oid=UP1-PROG28487\&_redirect=voir_presentation_diplome\&_lang=fr-FR}{Master
2 Droit du Numérique de l'Ecole de droit de la Sorbonne}, après collecte
et analyse de l'ensemble des clauses relatives aux données personnelles
des CGU des services numériques utilisés par les étudiants, nous avons
découvert des cas de réutilisation des données personnelles les plus
typiques.

Cette première étude a donc permis de dresser une liste des cas d'opt-in
les plus courants et des données pour lesquelles des permissions sont
régulièrement concédées aux opérateurs, via leurs CGU.

voici le résultat `'brut'' de ce travail de collecte et de
catégorisation:

Principaux services numériques utilisés par les étudiants
\url{https://github.com/M2DroitNumerique/audit_CGU}{]}(https://github.com/M2DroitNumerique/audit\_CGU{]}(https://github.com/M2DroitNumerique/audit\_CGU))Services
numériques aux juristes ( DIU informatique et droit):
\url{https://github.com/DIUIDEI/DIU-Droit-et-Informatique}{]}(https://github.com/DIUIDEI/DIU-Droit-et-Informatique{]}(https://github.com/DIUIDEI/DIU-Droit-et-Informatique))
Après analyse des données collectées et sur la base de certaines
propositions des étudiants, nous avons procédé à une formalisation du
projet de
\href{https://www.data.gouv.fr/fr/datasets/referentiel-cgu-1/}{référentiel
des CGU.}


\vspace{0.5cm}

\clearpage
\section{Montfort Communauté}


\begin{center}
  \includegraphics[width=3cm]{images/orga/21_6939224cdf412db18637678afc2e86-100.jpg}
\end{center}


Communauté de communes


\vspace{0.5cm}

\needspace{12\baselineskip}
\subsection*{Défibrillateurs Montfort Communauté
}\index{defibrillateur}\index{montfort!communaute}\index{sante}\index{secours}
  \begin{wrapfigure}{r}{2.5cm}
    \centering
    \qrcode[nolink]{https://data.gouv.fr/dataset/5baca9798b4c411829c4659a}
  \end{wrapfigure}

Licence : \textbf{Licence Ouverte version 2.0
}\newline
Créé le : 2018-09-27\newline
Modifié le : 2018-09-27\newline
De 2018-09-27 à 2018-09-28\newline
Granularité : à l'EPCI\newline
Popularité : 1 réutilisation,  0 suivi\newline
Mots-clé : \emph{defibrillateur, montfort-communaute, sante, secours
}\newline
Permalien : \url{https://data.gouv.fr/dataset/5baca9798b4c411829c4659a}\newline

\par
\noindent
    Liste et localisation des défibrillateurs de Montfort Communauté gérés
par les communes ou la communauté (intérieurs et extérieurs) Communes
concernées : Bédée, Breteil, Iffendic, Montfort-sur-Meu, La Nouaye,
Pleumeleuc,St-Gonlay, Talensac


\vspace{0.5cm}
\needspace{3\baselineskip} \rule{4cm}{0.25pt}\newline\textbf{Aussi disponible du même producteur :}\begin{itemize}
\item \href{https://data.gouv.fr/dataset/5ba0cffb634f413614b142cf}{Associations de Montfort Communauté}
\item \href{https://data.gouv.fr/dataset/5ba906f2634f41606a16e05b}{Entreprises de Montfort Communauté}
\item \href{https://data.gouv.fr/dataset/5ba0f5c1634f4164bacb4df5}{Nids de frelons asiatiques Montfort Communauté}
\end{itemize}

\clearpage
\section{Morbihan Tourisme}


\begin{center}
  \includegraphics[width=3cm]{images/orga/4a_4ed2e2888f4ac2b991567e60e8520b-100.jpg}
\end{center}


Comité Départemental du Tourisme du Morbihan


\vspace{0.5cm}

\needspace{12\baselineskip}
\subsection*{Campings du Morbihan
}\index{bretagne}\index{camping}\index{morbihan}\index{tourisme}\index{vacances}
  \begin{wrapfigure}{r}{2.5cm}
    \centering
    \qrcode[nolink]{https://data.gouv.fr/dataset/56aa3ea188ee380a798ffbc6}
  \end{wrapfigure}

Licence : \textbf{Licence Ouverte
}\newline
Créé le : 2016-01-28\newline
Modifié le : 2016-04-29\newline
Mise à jour : hebdomadaire\newline
Popularité : 1 réutilisation,  0 suivi\newline
Mots-clé : \emph{bretagne, camping, morbihan, tourisme, vacances
}\newline
Permalien : \url{https://data.gouv.fr/dataset/56aa3ea188ee380a798ffbc6}\newline

\par
\noindent
    Liste des campings du Morbihan


\vspace{0.5cm}
\needspace{3\baselineskip} \rule{4cm}{0.25pt}\newline\textbf{Aussi disponible du même producteur :}\begin{itemize}
\item \href{https://data.gouv.fr/dataset/56aa3ebcc751df5bccd77292}{Hotels du Morbihan}
\item \href{https://data.gouv.fr/dataset/56aa3efbc751df5676d77292}{Parcs et Jardins du Morbihan}
\item \href{https://data.gouv.fr/dataset/56aa3f2d88ee380a798ffbc7}{Patrimoine Naturel du Morbihan}
\item \href{https://data.gouv.fr/dataset/56aa3ed0c751df05b7d77292}{Restaurants du Morbihan}
\end{itemize}

\clearpage
\section{NosDonnées.fr}


\begin{center}
  \includegraphics[width=3cm]{images/orga/55_ae53576ff5482899020a1967a7490d-100.png}
\end{center}


NosDonnees.fr est un catalogue collaboratif de jeux de données, ouvert
aux contributions de chacun.

NosDonnées est géré par :

\begin{itemize}

\item
  \href{http://www.regardscitoyens.org/}{Regards Citoyens}
\item
  \href{http://fr.okfn.org/}{OKFN France}
\end{itemize}


\vspace{0.5cm}

\needspace{12\baselineskip}
\subsection*{Activité des députés de l'Assemblée nationale (13ème législature)
}\index{2007}\index{2007!2012}\index{amendements}\index{assemblee!nationale}\index{deputes}\index{ecrites}\index{interventions}\index{lois}\index{opendataday2010}\index{parlement}\index{presence}\index{questions}\index{rapports}
  \begin{wrapfigure}{r}{2.5cm}
    \centering
    \qrcode[nolink]{https://data.gouv.fr/dataset/53698e66a3a729239d20345b}
  \end{wrapfigure}

Licence : \textbf{Creative Commons Attribution Share-Alike
}\newline
Créé le : 2013-09-14\newline
Modifié le : 2015-11-30\newline
Popularité : 1 réutilisation,  0 suivi\newline
Mots-clé : \emph{2007, 2007-2012, amendements, assemblee-nationale, deputes, ecrites, interventions, lois, opendataday2010, parlement, presence, questions, rapports
}\newline
Permalien : \url{https://data.gouv.fr/dataset/53698e66a3a729239d20345b}\newline

\par
\noindent
    Ensemble des données parlementaires sur les députés. Informations
extraites du site de l'Assemblée nationale et du Journal Officiel pour
la constitution du site\url{http://NosDeputes.fr}Ces données comprennent
toutes les informations disponibles sur chaque député, toutes les
questions et interventions en séance publique et en commissions ainsi
que tous les amendements, propositions de lois et rapports produits
depuis le début de la législature. Ces données sont celles en date du
1er du mois en cours et sont remises à jour mensuellement.

Une API est également disponible pour accéder aux données
:\url{http://cpc.regardscitoyens.org/trac/wiki/API}


\vspace{0.5cm}
\needspace{12\baselineskip}
\subsection*{API codes postaux
}\index{codes!postaux}
  \begin{wrapfigure}{r}{2.5cm}
    \centering
    \qrcode[nolink]{https://data.gouv.fr/dataset/5878ee29a3a7291484cac7c9}
  \end{wrapfigure}

Licence : \textbf{Other (Public Domain)
}\newline
Créé le : 2013-02-23\newline
Modifié le : 2017-07-10\newline
Popularité : 2 réutilisations,  0 suivi\newline
Mots-clé : \emph{codes-postaux
}\newline
Permalien : \url{https://data.gouv.fr/dataset/5878ee29a3a7291484cac7c9}\newline

\par
\noindent
    

\vspace{0.5cm}
\needspace{12\baselineskip}
\subsection*{Carte des circonscriptions législatives françaises (2012+)
}\index{2012}\index{circonscriptions}\index{elections}\index{legislatives}\index{redecoupage}
  \begin{wrapfigure}{r}{2.5cm}
    \centering
    \qrcode[nolink]{https://data.gouv.fr/dataset/53699029a3a729239d203924}
  \end{wrapfigure}

Licence : \textbf{Open Data Commons Open Database License (ODbL)
}\newline
Créé le : 2013-09-14\newline
Modifié le : 2016-03-10\newline
Popularité : 1 réutilisation,  1 suivi\newline
Mots-clé : \emph{2012, circonscriptions, elections, legislatives, redecoupage
}\newline
Permalien : \url{https://data.gouv.fr/dataset/53699029a3a729239d203924}\newline

\par
\noindent
    Carte réalisée grâce aux données suivantes : France métropolitaine et
Outre-mer :\url{http://www.nosdonnees.fr/package/toxicode}; Partie de
l'Outre-mer manquante
:\url{http://www.nosdonnees.fr/package/carte-du-decoupage-des-circonscriptions-electorales-legislatives-2012-}


\vspace{0.5cm}
\needspace{12\baselineskip}
\subsection*{Carte du découpage des circonscriptions électorales législatives 2012
}\index{2012}\index{anciens!sites!industriels}\index{assemblee}\index{assemblee!nationale}\index{carte}\index{elections}\index{legislatives}\index{redecoupage}
  \begin{wrapfigure}{r}{2.5cm}
    \centering
    \qrcode[nolink]{https://data.gouv.fr/dataset/5878ee2fa3a7291484cac7d1}
  \end{wrapfigure}

Licence : \textbf{Open Data Commons Open Database License (ODbL)
}\newline
Créé le : 2013-02-22\newline
Modifié le : 2017-07-10\newline
Popularité : 1 réutilisation,  0 suivi\newline
Mots-clé : \emph{2012, anciens-sites-industriels, assemblee, assemblee-nationale, carte, elections, legislatives, redecoupage
}\newline
Permalien : \url{https://data.gouv.fr/dataset/5878ee2fa3a7291484cac7d1}\newline

\par
\noindent
    Carte réalisée par Jérôme Cukier pour ses projets de visualisation des
données électorales 2012
:\url{http://www.jeromecukier.net/projects/elections/projections950.html}
Identifiants des zones : DD = code département (01 à 95 971 à 977 987
988 999) CC = numéro circonscription (01 à 21) Départements : id=``gDD''
Circonscriptions : id = ``cDDCC''


\vspace{0.5cm}
\needspace{12\baselineskip}
\subsection*{Circulation automobile à Strasbourg
}\index{parking}\index{strasbourg}\index{trafic}
  \begin{wrapfigure}{r}{2.5cm}
    \centering
    \qrcode[nolink]{https://data.gouv.fr/dataset/536990b3a3a729239d203a86}
  \end{wrapfigure}

Licence : \textbf{Licence Ouverte
}\newline
Créé le : 2013-09-14\newline
Modifié le : 2015-11-30\newline
Popularité : 2 réutilisations,  1 suivi\newline
Mots-clé : \emph{parking, strasbourg, trafic
}\newline
Permalien : \url{https://data.gouv.fr/dataset/536990b3a3a729239d203a86}\newline

\par
\noindent
    

\vspace{0.5cm}
\needspace{12\baselineskip}
\subsection*{Comptes de campagne des candidats à l'élection présidentielle 2012
}\index{2012}\index{campagne}\index{candidats}\index{comptes}\index{elections}\index{frais}\index{presidentielle}\index{presidentielles}
  \begin{wrapfigure}{r}{2.5cm}
    \centering
    \qrcode[nolink]{https://data.gouv.fr/dataset/5369914da3a729239d203c11}
  \end{wrapfigure}

Licence : \textbf{Open Data Commons Open Database License (ODbL)
}\newline
Créé le : 2013-09-14\newline
Modifié le : 2015-07-01\newline
Popularité : 1 réutilisation,  0 suivi\newline
Mots-clé : \emph{2012, campagne, candidats, comptes, elections, frais, presidentielle, presidentielles
}\newline
Permalien : \url{https://data.gouv.fr/dataset/5369914da3a729239d203c11}\newline

\par
\noindent
    Le Journal Officiel propose en PDF toutes ces informations
:\url{http://www.regardscitoyens.org/temp/120731-JORF-Presidentielle-ComptesCampagne.pdf}
DataPublica les a numérises dans un tableur multi-feuillets
:\href{http://www.data-publica.com/data/14417--comptes-de-campagne-declares-des-candidats-a-l-election-presidentielle-2012}{http://www.data-publica.com/data/14417--comptes-de-campagne-declares-des-candidats-a-l-election-presidentielle-2012}
Les voici désormais en données exploitables.


\vspace{0.5cm}
\needspace{12\baselineskip}
\subsection*{Countours des circonscriptions des législatives
}\index{carte}\index{circonscriptions}\index{elections}\index{legislatives}
  \begin{wrapfigure}{r}{2.5cm}
    \centering
    \qrcode[nolink]{https://data.gouv.fr/dataset/536991dba3a729239d203d88}
  \end{wrapfigure}

Licence : \textbf{Open Data Commons Open Database License (ODbL)
}\newline
Créé le : 2013-09-14\newline
Modifié le : 2016-03-12\newline
Popularité : 1 réutilisation,  3 suivis\newline
Mots-clé : \emph{carte, circonscriptions, elections, legislatives
}\newline
Permalien : \url{https://data.gouv.fr/dataset/536991dba3a729239d203d88}\newline

\par
\noindent
    Pour construire ces données, nous avons utilisé ces deux sources :

\begin{itemize}
\item
  Les contours de communes fournis par l'IGN / GEOFLA
\item
  Les correspondances entre circonscriptions et communes, fournies par
  le Ministère de l'intérieur Format des données
\item
  code\_circonscription : le code du département le numéro de la
  circoncription, ex : 69002
\item
  departement : la code du département, ex : 69
\item
  numero : le numéro de circonscription, ex : 2,
\item
  communes : codes INSEE des communes qui sont (entièrement ou
  partiellement) dans la circonscription, séparés par le caractère `-'.
  ex : 69382-69381-69384-69389
\item
  kml\_shape : polygone(s) de la circonscription, au format KML
\item
  edited\_shape : `false' si le contour est simplement un regroupement
  des contours de communes. `true' si nous l'avons édité.
\end{itemize}

Davantage d'explications sur
\href{http://www.toxicode.fr/circonscriptions}{le site de Toxicode}


\vspace{0.5cm}
\needspace{12\baselineskip}
\subsection*{Déclarations d'intérêts des parlementaires publiées par la Haute
Autorité pour la Transparence
}
  \begin{wrapfigure}{r}{2.5cm}
    \centering
    \qrcode[nolink]{https://data.gouv.fr/dataset/53dfccb5a3a729110ca8d363}
  \end{wrapfigure}

Licence : \textbf{Licence Ouverte
}\newline
Créé le : 2014-08-03\newline
Modifié le : 2016-02-28\newline
De 2012-01-01 à 2017-12-31\newline
Granularité : au pays\newline
Popularité : 2 réutilisations,  3 suivis\newline
Mots-clé : \emph{aucun
}\newline
Permalien : \url{https://data.gouv.fr/dataset/53dfccb5a3a729110ca8d363}\newline

\par
\noindent
    Depuis la promulgation de la loi sur la transparence de la vie publique,
les parlementaires doivent déclarer leurs intérêts à la Haute Autorité
pour la Transparence de la Vie Publique en charge de les contrôler et de
les rendre publics afin que chaque citoyen puisse évaluer les possibles
risques de conflits d'intérêts de ses représentants. Activités annexes,
rémunérations extérieures, autres mandats, noms et activités des
collaborateurs\ldots{} Autant d'informations remplies à la main par les
parlementaires dans les formulaires publiés par la Haute Autorité :
\url{http://www.hatvp.fr/consulter-les-declarations-rechercher.html}{]}(http://www.hatvp.fr/consulter-les-declarations-rechercher.html{]}(http://www.hatvp.fr/consulter-les-declarations-rechercher.html))
Afin de permettre à tous de pouvoir explorer ces informations en
OpenData, Regards Citoyens a proposé une interface de crowdsourcing
permettant de numériser l'ensemble des déclarations en une semaine grâce
à la participation de près de 8000 personnes :
\url{http://www.regardscitoyens.org/8000-personnes-liberent-en-une-semaine-les-donnees-manuscrites-des-declarations-dinterets-des-parlementaires/}{]}(http://www.regardscitoyens.org/8000-personnes-liberent-en-une-semaine-les-donnees-manuscrites-des-declarations-dinterets-des-parlementaires/{]}(http://www.regardscitoyens.org/8000-personnes-liberent-en-une-semaine-les-donnees-manuscrites-des-declarations-dinterets-des-parlementaires/))
Retrouvez ici les données résultantes ainsi que l'ensemble des
numérisations anonymisées réalisées par les utilisateurs de
l'application.


\vspace{0.5cm}
\needspace{12\baselineskip}
\subsection*{Emplacement des campus de l'université de Strasbourg
}\index{campus}\index{strasbourg}\index{universite}
  \begin{wrapfigure}{r}{2.5cm}
    \centering
    \qrcode[nolink]{https://data.gouv.fr/dataset/536993cba3a729239d2042b7}
  \end{wrapfigure}

Licence : \textbf{Creative Commons Attribution Share-Alike
}\newline
Créé le : 2013-09-14\newline
Modifié le : 2016-03-06\newline
Popularité : 2 réutilisations,  1 suivi\newline
Mots-clé : \emph{campus, strasbourg, universite
}\newline
Permalien : \url{https://data.gouv.fr/dataset/536993cba3a729239d2042b7}\newline

\par
\noindent
    

\vspace{0.5cm}
\needspace{12\baselineskip}
\subsection*{Emplacement des stations de tramway de Strasbourg
}
  \begin{wrapfigure}{r}{2.5cm}
    \centering
    \qrcode[nolink]{https://data.gouv.fr/dataset/536993cda3a729239d2042bb}
  \end{wrapfigure}

Licence : \textbf{Creative Commons Attribution Share-Alike
}\newline
Créé le : 2013-09-14\newline
Modifié le : 2015-07-22\newline
Popularité : 2 réutilisations,  1 suivi\newline
Mots-clé : \emph{aucun
}\newline
Permalien : \url{https://data.gouv.fr/dataset/536993cda3a729239d2042bb}\newline

\par
\noindent
    

\vspace{0.5cm}
\needspace{12\baselineskip}
\subsection*{INSEE : codes et types d'équipements
}\index{equipements}\index{insee}
  \begin{wrapfigure}{r}{2.5cm}
    \centering
    \qrcode[nolink]{https://data.gouv.fr/dataset/53699720a3a729239d204bf4}
  \end{wrapfigure}

Licence : \textbf{Other (Public Domain)
}\newline
Créé le : 2013-09-14\newline
Modifié le : 2015-07-26\newline
Popularité : 2 réutilisations,  0 suivi\newline
Mots-clé : \emph{equipements, insee
}\newline
Permalien : \url{https://data.gouv.fr/dataset/53699720a3a729239d204bf4}\newline

\par
\noindent
    Liste des équipements tels qu'ils sont définis par l'INSEE et ses codes.


\vspace{0.5cm}
\needspace{12\baselineskip}
\subsection*{INSEE : codes et types d'équipements
}\index{equipements}\index{insee}
  \begin{wrapfigure}{r}{2.5cm}
    \centering
    \qrcode[nolink]{https://data.gouv.fr/dataset/5878ee54a3a7291484cac7fd}
  \end{wrapfigure}

Licence : \textbf{Other (Public Domain)
}\newline
Créé le : 2013-02-25\newline
Modifié le : 2017-07-10\newline
Popularité : 1 réutilisation,  0 suivi\newline
Mots-clé : \emph{equipements, insee
}\newline
Permalien : \url{https://data.gouv.fr/dataset/5878ee54a3a7291484cac7fd}\newline

\par
\noindent
    Liste des équipements tels qu'ils sont définis par l'INSEE et ses codes.


\vspace{0.5cm}
\needspace{12\baselineskip}
\subsection*{Liste des candidats aux élections législatives 2012 dans chaque
circonscription
}\index{2012}\index{candidats}\index{circonscriptions}\index{departements}\index{elections}\index{legislatives}
  \begin{wrapfigure}{r}{2.5cm}
    \centering
    \qrcode[nolink]{https://data.gouv.fr/dataset/5878ee58a3a7291485cac7e7}
  \end{wrapfigure}

Licence : \textbf{Licence Ouverte
}\newline
Créé le : 2013-02-28\newline
Modifié le : 2017-07-10\newline
Popularité : 1 réutilisation,  0 suivi\newline
Mots-clé : \emph{2012, candidats, circonscriptions, departements, elections, legislatives
}\newline
Permalien : \url{https://data.gouv.fr/dataset/5878ee58a3a7291485cac7e7}\newline

\par
\noindent
    Les données ont été progressivement corrigées grâce aux retours des
utilisateurs puis à la seconde version fournie par le Ministère.

Le second fichier enregistré identifie les corrections réalisées chez
nous et non présentes dans les données du ministère republiées sur
data.gouv.fr\href{http://www.data.gouv.fr/donnees/view/L\%C3\%A9gislatives-2012---Liste-des-candidats-du-1er-tour-551833}{http://www.data.gouv.fr/donnees/view/L\%C3\%A9gislatives-2012---Liste-des-candidats-du-1er-tour-551833}


\vspace{0.5cm}
\needspace{12\baselineskip}
\subsection*{Liste des réacteurs nucléaires en France
}\index{energie}\index{nucleaire}\index{open!transition!energie}\index{transition!energetique}
  \begin{wrapfigure}{r}{2.5cm}
    \centering
    \qrcode[nolink]{https://data.gouv.fr/dataset/5369994ea3a729239d2051fb}
  \end{wrapfigure}

Licence : \textbf{Open Data Commons Open Database License (ODbL)
}\newline
Créé le : 2013-09-14\newline
Modifié le : 2016-03-04\newline
Popularité : 1 réutilisation,  0 suivi\newline
Mots-clé : \emph{energie, nucleaire, open-transition-energie, transition-energetique
}\newline
Permalien : \url{https://data.gouv.fr/dataset/5369994ea3a729239d2051fb}\newline

\par
\noindent
    Liste des réacteurs nucléaires en France en activité - année 2012


\vspace{0.5cm}
\needspace{12\baselineskip}
\subsection*{Listes des communes géolocalisées par régions, départements,
circonscriptions
}\index{circonscriptions}\index{codes!postaux}\index{communes}\index{coordonnees}\index{departement}\index{gps}\index{insee}\index{postal}\index{prefectures}\index{regions}
  \begin{wrapfigure}{r}{2.5cm}
    \centering
    \qrcode[nolink]{https://data.gouv.fr/dataset/5369995ba3a729239d205221}
  \end{wrapfigure}

Licence : \textbf{Open Data Commons Open Database License (ODbL)
}\newline
Créé le : 2013-09-14\newline
Modifié le : 2016-03-16\newline
Popularité : 2 réutilisations,  3 suivis\newline
Mots-clé : \emph{circonscriptions, codes-postaux, communes, coordonnees, departement, gps, insee, postal, prefectures, regions
}\newline
Permalien : \url{https://data.gouv.fr/dataset/5369995ba3a729239d205221}\newline

\par
\noindent
    Tous les codes et informations géographique sur les 36000 communes
françaises. Il s'agit d'une compilation de différentes informations
tirées de Wikipedia et du travail
de\url{http://www.galichon.com/codesgeo/}donnant pour chaque commune de
France les informations précises sur leurs régions (et chefs-lieux),
départements (et préfectures), circonscriptions législatives et
européennes, codes INSEE, codes postaux et coordonnées GPS pour
l'essentiel des communes de métropole. (champs : EU\_circo code\_région
nom\_région chef-lieu\_région numéro\_département nom\_département
préfecture numéro\_circonscription nom\_commune codes\_postaux
code\_insee latitude longitude éloignement)


\vspace{0.5cm}
\needspace{12\baselineskip}
\subsection*{Localisation des salles de spectacles
}\index{cinema}\index{musee}\index{nfc}\index{qr!code}\index{salle!de!spectacle}\index{strasplus}
  \begin{wrapfigure}{r}{2.5cm}
    \centering
    \qrcode[nolink]{https://data.gouv.fr/dataset/5369996ea3a729239d20524f}
  \end{wrapfigure}

Licence : \textbf{Licence Ouverte
}\newline
Créé le : 2013-09-14\newline
Modifié le : 2015-11-08\newline
Popularité : 1 réutilisation,  0 suivi\newline
Mots-clé : \emph{cinema, musee, nfc, qr-code, salle-de-spectacle, strasplus
}\newline
Permalien : \url{https://data.gouv.fr/dataset/5369996ea3a729239d20524f}\newline

\par
\noindent
    Ce jeu de données contient la géolocalisation des signalétiques NFC
déployées dans le cadre du projet Stras+ et situés aux accès des salles
de spectacles de la CUS.


\vspace{0.5cm}
\needspace{12\baselineskip}
\subsection*{Municipales 2014
}\index{elections}
  \begin{wrapfigure}{r}{2.5cm}
    \centering
    \qrcode[nolink]{https://data.gouv.fr/dataset/53699a40a3a729239d205443}
  \end{wrapfigure}

Licence : \textbf{Licence Ouverte
}\newline
Créé le : 2014-03-20\newline
Modifié le : 2016-02-28\newline
Popularité : 1 réutilisation,  0 suivi\newline
Mots-clé : \emph{elections
}\newline
Permalien : \url{https://data.gouv.fr/dataset/53699a40a3a729239d205443}\newline

\par
\noindent
    Données non publiées sur data.gouv.fr relatives aux municipales 2014


\vspace{0.5cm}
\needspace{12\baselineskip}
\subsection*{Présidentielle 2012 : Résultats par commune et par circonscription des
français de l'étranger pour les deux tours
}\index{1er!tour}\index{2012}\index{2d!tour}\index{elections}\index{presidentielle}\index{presidentielles}
  \begin{wrapfigure}{r}{2.5cm}
    \centering
    \qrcode[nolink]{https://data.gouv.fr/dataset/53699da4a3a729239d205ca7}
  \end{wrapfigure}

Licence : \textbf{Licence Ouverte
}\newline
Créé le : 2013-10-20\newline
Modifié le : 2015-09-07\newline
Popularité : 1 réutilisation,  0 suivi\newline
Mots-clé : \emph{1er-tour, 2012, 2d-tour, elections, presidentielle, presidentielles
}\newline
Permalien : \url{https://data.gouv.fr/dataset/53699da4a3a729239d205ca7}\newline

\par
\noindent
    Données fournies par le ministère de l'intérieur divisées en deux
fichiers pour chacun des deux tours : un fichier communes françaises et
un fichier circonscriptions des français de l'étranger.


\vspace{0.5cm}
\needspace{12\baselineskip}
\subsection*{Réserve parlementaire 2013 de l'Assemblée nationale
}\index{2013}\index{assemblee}\index{assemblee!nationale}\index{associations}\index{collectivites}\index{parlement}\index{reserve}\index{reserve!parlementaire}\index{reserve!parlementaire!et!ministe}\index{subventions}
  \begin{wrapfigure}{r}{2.5cm}
    \centering
    \qrcode[nolink]{https://data.gouv.fr/dataset/53699f16a3a729239d20604b}
  \end{wrapfigure}

Licence : \textbf{Open Data Commons Open Database License (ODbL)
}\newline
Créé le : 2014-02-21\newline
Modifié le : 2016-03-12\newline
Popularité : 7 réutilisations,  0 suivi\newline
Mots-clé : \emph{2013, assemblee, assemblee-nationale, associations, collectivites, parlement, reserve, reserve-parlementaire, reserve-parlementaire-et-ministe, subventions
}\newline
Permalien : \url{https://data.gouv.fr/dataset/53699f16a3a729239d20604b}\newline

\par
\noindent
    Pour la première fois en 2014, l'Assemblée nationale publie e ligne
l'intégralité des subeventions accordées aux collectivités et
associations dans le cadre de la réserve parlementaire de 2013. Les
tableaux html scrapés et complétés des informations tirées de NosDéputés
sont rendus accessibles en CSV et mis à jour régulièrement.


\vspace{0.5cm}
\needspace{12\baselineskip}
\subsection*{Résultat des élections législatives françaises de 2012 au niveau bureau
de vote
}\index{2012}\index{bureaux!de!vote}\index{elections}\index{legislatives}
  \begin{wrapfigure}{r}{2.5cm}
    \centering
    \qrcode[nolink]{https://data.gouv.fr/dataset/53699f29a3a729239d20607a}
  \end{wrapfigure}

Licence : \textbf{Licence Ouverte
}\newline
Créé le : 2013-09-14\newline
Modifié le : 2016-01-09\newline
Popularité : 1 réutilisation,  0 suivi\newline
Mots-clé : \emph{2012, bureaux-de-vote, elections, legislatives
}\newline
Permalien : \url{https://data.gouv.fr/dataset/53699f29a3a729239d20607a}\newline

\par
\noindent
    Résultat de l'ensemble des bureaux de vote français des deux tours des
élections législatives de juin 2012


\vspace{0.5cm}
\needspace{12\baselineskip}
\subsection*{Résultat des élections présidentielles française de 2012 au niveau
bureau de vote
}\index{2012}\index{bureaux!de!vote}\index{elections}\index{presdientielles}
  \begin{wrapfigure}{r}{2.5cm}
    \centering
    \qrcode[nolink]{https://data.gouv.fr/dataset/53699f29a3a729239d20607b}
  \end{wrapfigure}

Licence : \textbf{Licence Ouverte
}\newline
Créé le : 2013-09-14\newline
Modifié le : 2016-03-14\newline
Popularité : 1 réutilisation,  0 suivi\newline
Mots-clé : \emph{2012, bureaux-de-vote, elections, presdientielles
}\newline
Permalien : \url{https://data.gouv.fr/dataset/53699f29a3a729239d20607b}\newline

\par
\noindent
    Résultat de l'ensemble des bureaux de vote français des deux tours des
élections présidentielles de 2012


\vspace{0.5cm}
\needspace{12\baselineskip}
\subsection*{Résultats des élections législatives 2007 et application au redécoupage
de 2012
}\index{2007}\index{2007!2012}\index{2012}\index{bureaux!de!vote}\index{circonscriptions}\index{deputes}\index{elections}\index{legislatives}\index{opendataday2010}\index{population}\index{redecoupage}\index{resultats!electoraux}
  \begin{wrapfigure}{r}{2.5cm}
    \centering
    \qrcode[nolink]{https://data.gouv.fr/dataset/53699f38a3a729239d2060a1}
  \end{wrapfigure}

Licence : \textbf{Open Data Commons Open Database License (ODbL)
}\newline
Créé le : 2013-09-14\newline
Modifié le : 2016-03-08\newline
Popularité : 1 réutilisation,  0 suivi\newline
Mots-clé : \emph{2007, 2007-2012, 2012, bureaux-de-vote, circonscriptions, deputes, elections, legislatives, opendataday2010, population, redecoupage, resultats-electoraux
}\newline
Permalien : \url{https://data.gouv.fr/dataset/53699f38a3a729239d2060a1}\newline

\par
\noindent
    Résultats de 2007 par circonscription et par bureau de vote avec
informations démographiques associées et prévisions de résultats sur la
carte prévisionnelle des circonscriptions pour l'élection 2012 à partir
de la relocalisation des résultats de 2007


\vspace{0.5cm}
\needspace{12\baselineskip}
\subsection*{Résultats des élections présidentielles 2007
}\index{bureaux!de!vote}\index{elections}\index{resultats!electoraux}
  \begin{wrapfigure}{r}{2.5cm}
    \centering
    \qrcode[nolink]{https://data.gouv.fr/dataset/53699f3ba3a729239d2060a8}
  \end{wrapfigure}

Licence : \textbf{Creative Commons Attribution Share-Alike
}\newline
Créé le : 2013-09-14\newline
Modifié le : 2015-11-23\newline
Popularité : 1 réutilisation,  0 suivi\newline
Mots-clé : \emph{bureaux-de-vote, elections, resultats-electoraux
}\newline
Permalien : \url{https://data.gouv.fr/dataset/53699f3ba3a729239d2060a8}\newline

\par
\noindent
    Données obtenues auprès du ministère de l'intérieur. Elles correspondent
aux résultats des deux tours de l'élection présidentielle de 2007,
bureaux de vote par bureaux de vote.


\vspace{0.5cm}
\needspace{12\baselineskip}
\subsection*{Service-public.fr - Annuaire de l'administration
}\index{administration}\index{annuaire}\index{csv}\index{service!public}
  \begin{wrapfigure}{r}{2.5cm}
    \centering
    \qrcode[nolink]{https://data.gouv.fr/dataset/53699fe4a3a729239d20622f}
  \end{wrapfigure}

Licence : \textbf{Open Data Commons Open Database License (ODbL)
}\newline
Créé le : 2013-09-14\newline
Modifié le : 2016-01-05\newline
Popularité : 1 réutilisation,  1 suivi\newline
Mots-clé : \emph{administration, annuaire, csv, service-public
}\newline
Permalien : \url{https://data.gouv.fr/dataset/53699fe4a3a729239d20622f}\newline

\par
\noindent
    Base de données locales V2 de Service-public.fr compilée au format CSV.
La documentation du jeu de données se trouve sur la page source.


\vspace{0.5cm}
\needspace{12\baselineskip}
\subsection*{Service Vélhop Strasbourg
}\index{strasbourg}\index{velhop}
  \begin{wrapfigure}{r}{2.5cm}
    \centering
    \qrcode[nolink]{https://data.gouv.fr/dataset/53699ff6a3a729239d206259}
  \end{wrapfigure}

Licence : \textbf{Licence Ouverte
}\newline
Créé le : 2013-09-14\newline
Modifié le : 2015-10-02\newline
Popularité : 3 réutilisations,  1 suivi\newline
Mots-clé : \emph{strasbourg, velhop
}\newline
Permalien : \url{https://data.gouv.fr/dataset/53699ff6a3a729239d206259}\newline

\par
\noindent
    

\vspace{0.5cm}
\needspace{12\baselineskip}
\subsection*{Stations Autotrement Strasbourg
}\index{autotrement}\index{strasbourg}
  \begin{wrapfigure}{r}{2.5cm}
    \centering
    \qrcode[nolink]{https://data.gouv.fr/dataset/5369a039a3a729239d2062fb}
  \end{wrapfigure}

Licence : \textbf{Licence Ouverte
}\newline
Créé le : 2013-09-14\newline
Modifié le : 2015-10-02\newline
Popularité : 2 réutilisations,  1 suivi\newline
Mots-clé : \emph{autotrement, strasbourg
}\newline
Permalien : \url{https://data.gouv.fr/dataset/5369a039a3a729239d2062fb}\newline

\par
\noindent
    Localisation des stations Autotrement sous la forme d'un flux XML au
format GML.


\vspace{0.5cm}
\needspace{12\baselineskip}
\subsection*{Statistiques démographiques INSEE sur les nouvelles circonscriptions
législatives de 2012
}\index{age}\index{cantons}\index{circonscriptions}\index{communes}\index{csp}\index{demographie}\index{elections}\index{insee}\index{legislatives}\index{menages}\index{population}\index{sexe}\index{statistiques}
  \begin{wrapfigure}{r}{2.5cm}
    \centering
    \qrcode[nolink]{https://data.gouv.fr/dataset/5369a086a3a729239d2063b1}
  \end{wrapfigure}

Licence : \textbf{Open Data Commons Open Database License (ODbL)
}\newline
Créé le : 2013-09-14\newline
Modifié le : 2015-12-22\newline
Popularité : 2 réutilisations,  0 suivi\newline
Mots-clé : \emph{age, cantons, circonscriptions, communes, csp, demographie, elections, insee, legislatives, menages, population, sexe, statistiques
}\newline
Permalien : \url{https://data.gouv.fr/dataset/5369a086a3a729239d2063b1}\newline

\par
\noindent
    Indicateurs statistiques sur les 556 circonscriptions des départements
de métropole, de Guadeloupe, de Martinique, de Guyane et de La Réunion
après le redécoupage de 2012. - population par tranche d'âge -
population par sexe et âge - population de 15 ans ou plus par sexe et
catégorie socioprofessionnelle - population selon la nationalité -
logements par catégorie - population des ménages par type de ménage


\vspace{0.5cm}
\needspace{3\baselineskip} \rule{4cm}{0.25pt}\newline\textbf{Aussi disponible du même producteur :}\begin{itemize}
\item \href{https://data.gouv.fr/dataset/53698e19a3a729239d203398}{1ère section "modernisation et mutations industrielles" du fonds stratégique pour le développement de la presse}
\item \href{https://data.gouv.fr/dataset/5878ee84a3a7291484cac82d}{1ère section "modernisation et mutations industrielles" du fonds stratégique pour le développement de la presse}
\item \href{https://data.gouv.fr/dataset/5878ee34a3a7291484cac7d7}{2014 - 04 - Conseils municipaux de Grenoble}
\item \href{https://data.gouv.fr/dataset/5878ee36a3a7291484cac7da}{2014 - 05 - Conseil municipal de Grenoble}
\item \href{https://data.gouv.fr/dataset/5878ee37a3a7291485cac7c3}{2014 - 06 - Conseil municipal de Grenoble}
\item \href{https://data.gouv.fr/dataset/5878ee35a3a7291485cac7c0}{2014 - 07 - Conseil municipal de Grenoble}
\item \href{https://data.gouv.fr/dataset/5878ee33a3a7291485cac7bd}{2014 - 09 - Conseil municipal de Grenoble}
\item \href{https://data.gouv.fr/dataset/5878ee37a3a7291484cac7dd}{2014 - 10 - Conseils municipaux de Grenoble}
\item \href{https://data.gouv.fr/dataset/5878ee36a3a7291484cac7db}{2014 - 11 - Conseil municipal de Grenoble}
\item \href{https://data.gouv.fr/dataset/5878ee35a3a7291484cac7d8}{2014 - 12 - Conseil municipal de Grenoble décembre 2014}
\item \href{https://data.gouv.fr/dataset/5878ee35a3a7291484cac7d9}{2015 - 01 - Conseil municipal de Grenoble}
\item \href{https://data.gouv.fr/dataset/5878ee34a3a7291485cac7bf}{2015 - 02 - Conseil municipal de grenoble}
\item \href{https://data.gouv.fr/dataset/5878ee36a3a7291485cac7c1}{2015 - 03 - Conseil municipal de Grenoble}
\item \href{https://data.gouv.fr/dataset/5878ee34a3a7291485cac7be}{2015 - 04 - Conseil municipal de Grenoble}
\item \href{https://data.gouv.fr/dataset/5878ee23a3a7291484cac7bc}{2015 - 05 - Conseil municipal de Grenoble}
\item \href{https://data.gouv.fr/dataset/5878ee23a3a7291485cac7a6}{2015 - 06 - Conseil municipal de Grenoble}
\item \href{https://data.gouv.fr/dataset/5878ee23a3a7291484cac7bd}{2015 - 07 - Conseil municipal de Grenoble}
\item \href{https://data.gouv.fr/dataset/5878ee37a3a7291484cac7dc}{2015 - 09 - Conseil municipal de Grenoble}
\item \href{https://data.gouv.fr/dataset/5878ee36a3a7291485cac7c2}{2015 - 10 - Conseil municipal de grenoble}
\item \href{https://data.gouv.fr/dataset/5878ee23a3a7291485cac7a7}{2015 - 11 - Conseil municipal de Grenoble}
\item \href{https://data.gouv.fr/dataset/5878ee24a3a7291484cac7be}{2015 - 12 - Conseil municipal de Grenoble}
\item \href{https://data.gouv.fr/dataset/5878ee24a3a7291485cac7a8}{2016 - 01 - Conseil municipal de grenoble}
\item \href{https://data.gouv.fr/dataset/5878ee24a3a7291484cac7bf}{2016 - 02 - Conseil municipal de Grenoble}
\item \href{https://data.gouv.fr/dataset/53698e20a3a729239d2033aa}{2e section "investissements des services de presse en ligne" du fonds stratégique pour le dév. de la presse}
\item \href{https://data.gouv.fr/dataset/5878ee24a3a7291484cac7c0}{2e section "investissements des services de presse en ligne" du fonds stratégique pour le dév. de la presse}
\item \href{https://data.gouv.fr/dataset/5878ee25a3a7291485cac7aa}{3e section "développement du lectorat de la presse" du fonds stratégique pour le développement de la presse}
\item \href{https://data.gouv.fr/dataset/53698e24a3a729239d2033b6}{3e section "développement du lectorat de la presse" du fonds stratégique pour le développement de la presse}
\item \href{https://data.gouv.fr/dataset/5878ee25a3a7291484cac7c1}{500 signatures des candidats aux élections présidentielles de 1995, 2002 et 2007}
\item \href{https://data.gouv.fr/dataset/53698e29a3a729239d2033c1}{500 signatures des candidats aux élections présidentielles de 1995, 2002 et 2007}
\item \href{https://data.gouv.fr/dataset/5878ee25a3a7291485cac7ab}{Actes des collectivités terrtoriales soumis au controle de légalité en 2012}
\item \href{https://data.gouv.fr/dataset/53698e59a3a729239d20343a}{Actes des collectivités terrtoriales soumis au controle de légalité en 2012}
\item \href{https://data.gouv.fr/dataset/53698e5ba3a729239d20343f}{Acteurs de l'aéroport Notre Dame des Landes}
\item \href{https://data.gouv.fr/dataset/5878ee25a3a7291484cac7c2}{Acteurs de l'aéroport Notre Dame des Landes}
\item \href{https://data.gouv.fr/dataset/5878ee25a3a7291485cac7ac}{Activité comparée des député-e-s}
\item \href{https://data.gouv.fr/dataset/53698e63a3a729239d203452}{Activité comparée des député-e-s}
\item \href{https://data.gouv.fr/dataset/5878ee26a3a7291484cac7c3}{Activité des députés de l'Assemblée nationale (13ème législature)}
\item \href{https://data.gouv.fr/dataset/5878ee26a3a7291485cac7ad}{Activités des députés dans la session 2014-2015}
\item \href{https://data.gouv.fr/dataset/5878ee26a3a7291485cac7ae}{Aide à la presse hebdomadaire régionale en 2012}
\item \href{https://data.gouv.fr/dataset/53698eb0a3a729239d203524}{Aide à la presse hebdomadaire régionale en 2012}
\item \href{https://data.gouv.fr/dataset/53698eb8a3a729239d203538}{Aide aux quotidiens locaux à faibles ressources de petites annonces en 2012}
\item \href{https://data.gouv.fr/dataset/5878ee26a3a7291484cac7c4}{Aide aux quotidiens locaux à faibles ressources de petites annonces en 2012}
\item \href{https://data.gouv.fr/dataset/53698ebba3a729239d203541}{Aide Publique au Développement France 2011}
\item \href{https://data.gouv.fr/dataset/5878ee27a3a7291484cac7c5}{Aide Publique au Développement France 2011}
\item \href{https://data.gouv.fr/dataset/53698ec1a3a729239d203550}{Aides aux quotidiens nationaux à faibles ressources publicitaires en  2012}
\item \href{https://data.gouv.fr/dataset/5878ee27a3a7291485cac7af}{Aides aux quotidiens nationaux à faibles ressources publicitaires en  2012}
\item \href{https://data.gouv.fr/dataset/5878ee27a3a7291484cac7c6}{Aides directes à la presse 2008-2014}
\item \href{https://data.gouv.fr/dataset/53698ecda3a729239d203583}{Aléa de retrait de gonflement des argiles}
\item \href{https://data.gouv.fr/dataset/5878ee27a3a7291485cac7b0}{Aléa de retrait de gonflement des argiles}
\item \href{https://data.gouv.fr/dataset/53698ed6a3a729239d20359b}{Amenagements cyclables et vélos en libre service dans les grandes villes françaises}
\item \href{https://data.gouv.fr/dataset/5878ee92a3a7291484cac839}{Amenagements cyclables et vélos en libre service dans les grandes villes françaises}
\item et 328 autres jeux de données\end{itemize}

\clearpage
\section{Observatoire de l'environnement en Bretagne}


\begin{center}
  \includegraphics[width=3cm]{images/orga/5c_8476cb81e4425c9d4607f2ec71ff09-100.png}
\end{center}


L'Observatoire de l'environnement en Bretagne (anciennement GIP Bretagne
environnement), créé par l'Etat et le Conseil régional de Bretagne en
2007, a pour mission de permettre à chacun de trouver les renseignements
qu'il recherche sur l'environnement en Bretagne, afin de développer ses
connaissances et d'être aidé dans ses prises de décisions.

\href{http://www.bretagne-environnement.org/}{Le portail web}

\href{http://cartographie.bretagne-environnement.org/}{Le site de
cartographie dynamique}

\href{http://www.bretagne-environnement.org/content/view/twitter/28987/}{Les
réseaux sociaux}


\vspace{0.5cm}

\needspace{12\baselineskip}
\subsection*{Nombre d'installations solaires photovoltaïques par commune en Bretagne
}\index{bretagne}\index{donnees!ouvertes}\index{economie}\index{economy}\index{energie}\index{energie!production}\index{passerelle!inspire}\index{photovoltaique}\index{solaire}\index{sources!denergie}
  \begin{wrapfigure}{r}{2.5cm}
    \centering
    \qrcode[nolink]{https://data.gouv.fr/dataset/56f001cb88ee387586908574}
  \end{wrapfigure}

Licence : \textbf{Licence Ouverte version 2.0
}\newline
Créé le : 2016-03-21\newline
Modifié le : 2019-02-08\newline
Popularité : 1 réutilisation,  0 suivi\newline
Mots-clé : \emph{bretagne, donnees-ouvertes, economie, economy, energie, energie-production, passerelle-inspire, photovoltaique, solaire, sources-denergie
}\newline
Permalien : \url{https://data.gouv.fr/dataset/56f001cb88ee387586908574}\newline

\par
\noindent
    Nombre d'installations solaires photovoltaïques raccordées au réseau de
distribution ERDF par commune en 2014 en Bretagne.

\textbf{Origine}

Nombre réel d'installations fourni par ERDF, sauf informations
commercialement sensibles (80\% de la puissance communale représentée
par 1 installation ou moins de 4 installations sur la commune). Les
données ICS sont ici remplacées par des estimations de l'Oreges basées
sur la connaissance des installations avec certificat d'obligation
d'achat recensé par la Dreal Bretagne.

\textbf{Organisations partenaires}

Observatoire de l'environnement en Bretagne

➞
\href{https://geo.data.gouv.fr/fr/datasets/8beb823eec2d90a9da4deabace568e98aed2c710}{Consulter
cette fiche sur geo.data.gouv.fr}


\vspace{0.5cm}
\needspace{3\baselineskip} \rule{4cm}{0.25pt}\newline\textbf{Aussi disponible du même producteur :}\begin{itemize}
\item \href{https://data.gouv.fr/dataset/56f001cec751df33fbd6e93d}{Communes de Bretagne concernées par au moins un site d'intérêt pour les chiroptères}
\item \href{https://data.gouv.fr/dataset/56f001c2c751df33f2d6e93b}{Consommation de bois bûche par commune en Bretagne}
\item \href{https://data.gouv.fr/dataset/5bd04726634f414815072c1c}{Consommation de bois déchiqueté par EPCI en Bretagne}
\item \href{https://data.gouv.fr/dataset/56f001c2c751df33f3d6e93b}{Consommation de gaz naturel sur le réseau de distribution par EPCI en Bretagne}
\item \href{https://data.gouv.fr/dataset/56f001c0c751df33efd6e93b}{Consommation d'électricité sur le réseau de distribution par EPCI en Bretagne}
\item \href{https://data.gouv.fr/dataset/56f001c188ee387584908574}{Consommation d'énergie par pays en Bretagne}
\item \href{https://data.gouv.fr/dataset/5bd047038b4c417d7efb6ee1}{Consommation estimée domestique de bois bûche et granulé par EPCI en Bretagne}
\item \href{https://data.gouv.fr/dataset/56f001cfc751df33fbd6e93e}{Emissions de gaz à effet de serre par habitant par pays en Bretagne}
\item \href{https://data.gouv.fr/dataset/58077f01c751df106c79df72}{Estimation du gisement régional de DNDAE en Bretagne}
\item \href{https://data.gouv.fr/dataset/5a09a87ec751df2789ef7b46}{Indice communal de richesse patrimoniale en Bretagne}
\item \href{https://data.gouv.fr/dataset/56f001c8c751df33f4d6e93c}{La collecte des recyclables secs en Bretagne}
\item \href{https://data.gouv.fr/dataset/58077f0188ee381f1f5ff491}{La collecte du verre en Bretagne}
\item \href{https://data.gouv.fr/dataset/58077f0188ee381f1d5ff492}{La collecte en déchèterie, sur les aires de déchets verts et autres collectes en Bretagne}
\item \href{https://data.gouv.fr/dataset/58077efe88ee381f0c5ff490}{Lauréats ZDZG en Bretagne}
\item \href{https://data.gouv.fr/dataset/56f001bc88ee387581908575}{Les communes et intercommunalités à compétence collecte des déchets ménagers et assimilés en Bretagne}
\item \href{https://data.gouv.fr/dataset/56f001da88ee387593908575}{Les EPCI à compétence collecte engagés dans une démarche de fiscalité incitative en Bretagne}
\item \href{https://data.gouv.fr/dataset/56f001c888ee387581908576}{Les programmes locaux de prévention des ordures ménagères et assimilées (PLP OMA) en Bretagne}
\item \href{https://data.gouv.fr/dataset/56f001ccc751df33fcd6e93c}{Les structures publiques gérant des installations de traitement des déchets ménagers et assimilés hors déchets verts en Bretagne}
\item \href{https://data.gouv.fr/dataset/56f001c6c751df33f2d6e93c}{Les tonnages entrants sur les centres de tri de recyclables secs en Bretagne}
\item \href{https://data.gouv.fr/dataset/5a09a87e88ee383b626c5ea0}{Les tonnages entrants sur les des installations de méthanisation multi-déchets en Bretagne}
\item \href{https://data.gouv.fr/dataset/56f001cac751df33fad6e93b}{Les tonnages entrants sur les installations de stockage de déchets non dangereux en Bretagne}
\item \href{https://data.gouv.fr/dataset/56f001c7c751df33f4d6e93b}{Les tonnages entrants sur les plateformes de compostage en Bretagne}
\item \href{https://data.gouv.fr/dataset/56f001bcc751df33ebd6e93c}{Les tonnages entrants sur les quais de transfert en Bretagne}
\item \href{https://data.gouv.fr/dataset/56f001d2c751df33fcd6e93e}{Les tonnages entrants sur les unités de traitement organique des OMR en Bretagne}
\item \href{https://data.gouv.fr/dataset/56f001cbc751df33fbd6e93b}{Localisation des déchèteries et des aires de déchets verts en Bretagne}
\item \href{https://data.gouv.fr/dataset/58077effc751df106179df73}{Modes de traitement des ordures ménagères résiduelles en Bretagne}
\item \href{https://data.gouv.fr/dataset/58077f0088ee381f1d5ff491}{Nombre de composteurs domestiques distribués en Bretagne}
\item \href{https://data.gouv.fr/dataset/58077f00c751df106b79df72}{Opérations de collectes séparatives des biodéchets en Bretagne}
\item \href{https://data.gouv.fr/dataset/56f001c388ee387582908575}{Patrimoine faunistique remarquable des eaux douces en Bretagne}
\item \href{https://data.gouv.fr/dataset/58077efe88ee381f0d5ff490}{Population des EPCI exerçant la compétence collecte des déchets en Bretagne}
\item \href{https://data.gouv.fr/dataset/58077f0088ee381f1f5ff490}{Pourcentage de foyers équipés d'un composteur individuel ou collectif en Bretagne}
\item \href{https://data.gouv.fr/dataset/56f001ce88ee38758c908574}{Pourcentage de logements chauffés à l'électricité par commune en Bretagne}
\item \href{https://data.gouv.fr/dataset/56f001bfc751df33ecd6e93c}{Production de chaleur et d'électricité des installations de méthanisation par commune en Bretagne}
\item \href{https://data.gouv.fr/dataset/56f001cbc751df33fcd6e93b}{Production de chaleur et d'électricité renouvelables des UIOM par commune en Bretagne}
\item \href{https://data.gouv.fr/dataset/56f001ba88ee387581908574}{Production d'électricité par EPCI en Bretagne}
\item \href{https://data.gouv.fr/dataset/56f001c488ee387583908576}{Production d'énergie finale des pays en Bretagne}
\item \href{https://data.gouv.fr/dataset/56f001bf88ee387583908575}{Puissance des chaufferies au bois déchiqueté par commune en Bretagne}
\item \href{https://data.gouv.fr/dataset/56f001c4c751df33edd6e93c}{Puissance éolienne en fonctionnement par commune en Bretagne}
\item \href{https://data.gouv.fr/dataset/56f001c0c751df33ecd6e93d}{Puissance hydroélectrique installée par commune en Bretagne}
\item \href{https://data.gouv.fr/dataset/56f001c888ee38757f908577}{Qualité des cours d'eau vis-à-vis de l'ammonium (Percentile 90) en Bretagne}
\item \href{https://data.gouv.fr/dataset/56f001c888ee387581908577}{Qualité des cours d'eau vis-à-vis de l'indice biologique diatomées (IBD) en Bretagne}
\item \href{https://data.gouv.fr/dataset/56f001d988ee387593908574}{Qualité des cours d'eau vis-à-vis de l'indice biologique macro-invertébrés (IBGN) en Bretagne}
\item \href{https://data.gouv.fr/dataset/56f001d588ee38758f908574}{Qualité des cours d'eau vis-à-vis de l'indice biologique macrophytes (IBMR) en Bretagne}
\item \href{https://data.gouv.fr/dataset/56f001bc88ee38757f908575}{Qualité des cours d'eau vis-à-vis de l'indice poissons (IPR) en Bretagne}
\item \href{https://data.gouv.fr/dataset/56f001bac751df33ecd6e93b}{Qualité des cours d'eau vis-à-vis des matières phosphorées en Bretagne}
\item \href{https://data.gouv.fr/dataset/56f001cfc751df33fcd6e93d}{Qualité des cours d'eau vis-à-vis des nitrites (Percentile 90) en Bretagne}
\item \href{https://data.gouv.fr/dataset/5a09a87e88ee383d5c485dcd}{Qualité des cours d'eau vis-à-vis des orthophosphates en Bretagne}
\item \href{https://data.gouv.fr/dataset/56f001d9c751df3404d6e93b}{Qualité des cours d'eau vis-à-vis des pesticides (concentration moyenne) en Bretagne}
\item \href{https://data.gouv.fr/dataset/56f001d588ee387591908575}{Qualité des cours d'eau vis-à-vis des pesticides en Bretagne - Dépassement des seuils fixés pour l’alimentation en eau potable}
\item \href{https://data.gouv.fr/dataset/5a09a88188ee383b626c5ea1}{Qualité des cours d'eau vis-à-vis du phosphore total en Bretagne}
\item et 21 autres jeux de données\end{itemize}

\clearpage
\section{Office de tourisme So Toulouse}
\needspace{12\baselineskip}
\subsection*{Agenda des manifestations culturelles
}\index{agenda}\index{concert}\index{conference}\index{evenement}\index{exposition}\index{manifestation!culturelle}\index{musique}\index{sortie}\index{spectacle}\index{theatre}
  \begin{wrapfigure}{r}{2.5cm}
    \centering
    \qrcode[nolink]{https://data.gouv.fr/dataset/53698e8aa3a729239d2034c5}
  \end{wrapfigure}

Licence : \textbf{Open Data Commons Open Database License (ODbL)
}\newline
Créé le : 2013-11-18\newline
Modifié le : 2016-03-14\newline
Mise à jour : hebdomadaire\newline
Popularité : 2 réutilisations,  0 suivi\newline
Mots-clé : \emph{agenda, concert, conference, evenement, exposition, manifestation-culturelle, musique, sortie, spectacle, theatre
}\newline
Permalien : \url{https://data.gouv.fr/dataset/53698e8aa3a729239d2034c5}\newline

\par
\noindent
    Agenda des manifestations culturelles, des événements, spectacles,
festivals, expositions, animations, etc. de la ville de Toulouse et des
villes alentours.


\vspace{0.5cm}

\clearpage
\section{OpenAgenda}


\begin{center}
  \includegraphics[width=3cm]{images/orga/2015-04-16_7560b0ba4f9844f1a4040b4a8417929f_avatar_OA_fond_gris-100.png}
\end{center}


Des agendas ouverts pour tous! OpenAgenda permet de créer gratuitement
des agendas participatifs, géolocalisés et OpenData.


\vspace{0.5cm}

\needspace{12\baselineskip}
\subsection*{Agenda de la 1ère édition de la Journée des Arts et de la Culture dans
l'Enseignement Supérieur
}\index{agenda}\index{arts}\index{concert}\index{crous}\index{culture}\index{ecole}\index{education}\index{enseignement}\index{evenements}\index{exposition}\index{theatre}\index{universite}
  \begin{wrapfigure}{r}{2.5cm}
    \centering
    \qrcode[nolink]{https://data.gouv.fr/dataset/53698e86a3a729239d2034b6}
  \end{wrapfigure}

Licence : \textbf{Open Data Commons Open Database License (ODbL)
}\newline
Créé le : 2014-04-11\newline
Modifié le : 2015-04-11\newline
De 2014-04-10 à 2014-04-10\newline
Mise à jour : annuelle\newline
Popularité : 1 réutilisation,  1 suivi\newline
Mots-clé : \emph{agenda, arts, concert, crous, culture, ecole, education, enseignement, evenements, exposition, theatre, universite
}\newline
Permalien : \url{https://data.gouv.fr/dataset/53698e86a3a729239d2034b6}\newline

\par
\noindent
    \url{http://journee-arts-culture-sup.fr/} LA 1ÈRE ÉDITION

LA CONVENTION CADRE « UNIVERSITÉ, LIEU DE CULTURE »

Les ministères chargés de la culture et de l'enseignement supérieur et
la Conférence des présidents d'université, ont signé en Avignon le 12
juillet 2013 : la convention cadre « Université, lieu de culture ».\\
Cette convention met en lumière le rôle important que jouent les
universités dans la création et la diffusion culturelles et artistiques
sur le plan national et international. L'animation et les pratiques
culturelles et artistiques favorisent indéniablement :\\
- le développement de leur rayonnement et de leur attractivité\\
- la réussite des étudiants dans leurs études et leur insertion
professionnelle\\
- l'enrichissement de leur cursus\\
- la stimulation de leur créativité

UN NOUVEL ÉLAN POUR LA CULTURE ET LES ARTS DANS L'ENSEIGNEMENT SUPÉRIEUR

Cette convention institue une nouvelle collaboration entre les deux
ministères en vue : d'intensifier les pratiques culturelles et
artistiques des étudiants et des communautés universitaires d'ouvrir ces
pratiques à toutes les dimensions de la culture (y compris scientifiques
et techniques) de développer la présence artistique dans les universités
de renforcer les échanges entre les universités et leur environnement de
manière à en faire des lieux ouverts sur la cité et des acteurs
culturels locaux reconnus Comme le rappelle la convention, la mission
commune des deux ministères est de : favoriser l'accès du plus grand
nombre à la culture en poursuivant dans l'enseignement supérieur
l'ambitieux projet d'une éducation artistique et culturelle commencée
dès l'école.

LA JOURNÉE DES ARTS ET DE LA CULTURE DANS L'ENSEIGNEMENT SUPÉRIEUR

C'est dans ce contexte que les deux ministres ont décidé du lancement
d'une Journée des arts et de la culture dans l'enseignement supérieur
fixée le 10 avril 2014. Cette journée , vitrine des actions culturelles
et artistiques menées dans l'enseignement supérieur, a vocation à
devenir chaque année un moment exceptionnel de partage et de rencontres
notamment avec le grand public qui pourra découvrir la diversité et la
qualité des offres culturelles et artistiques. L'ensemble des
universités et des écoles sont donc conviées à se joindre à cette
manifestation nationale et à présenter les réalisations qui leur
paraissent les plus emblématiques de leur production.


\vspace{0.5cm}
\needspace{12\baselineskip}
\subsection*{Evénements géolocalisés et OpenData
}\index{cinema}\index{concert}\index{conference}\index{expositions}\index{festivals}\index{salons}\index{theatres}
  \begin{wrapfigure}{r}{2.5cm}
    \centering
    \qrcode[nolink]{https://data.gouv.fr/dataset/5369952aa3a729239d204648}
  \end{wrapfigure}

Licence : \textbf{Licence Ouverte
}\newline
Créé le : 2014-01-06\newline
Modifié le : 2016-03-15\newline
Popularité : 1 réutilisation,  2 suivis\newline
Mots-clé : \emph{cinema, concert, conference, expositions, festivals, salons, theatres
}\newline
Permalien : \url{https://data.gouv.fr/dataset/5369952aa3a729239d204648}\newline

\par
\noindent
    OpenAgenda est un plateforme d'agenda pour organisateur d'événements:
festivals, théâtres, concerts, expositions, salons, conférences, etc.

Voici la base de données complètes des événements mis en ligne par nos
utilisateurs, avec les coordonnées géographiques précises et les plages
horaires détaillées. Cette base de données est en alimentation continue.

Un aperçu de la base est disponible
ici:\url{http://public.opendatasoft.com/explore/dataset/evenements-publics-cibul/?tab=map\&location=2,20.65089,6.15314\&basemap=mapquest}


\vspace{0.5cm}

\clearpage
\section{Open Data Bourgogne}


\begin{center}
  \includegraphics[width=3cm]{images/orga/4d_b1897d7fd240a19ec02119e5f52ee0-100.png}
\end{center}


Regroupement informel des membres du canal IRC \#OpenDataBourgogne


\vspace{0.5cm}

\needspace{12\baselineskip}
\subsection*{Subvention Besançon 2008-2012
}\index{subventions}
  \begin{wrapfigure}{r}{2.5cm}
    \centering
    \qrcode[nolink]{https://data.gouv.fr/dataset/55853496c751df5ea9a453b9}
  \end{wrapfigure}

Licence : \textbf{Open Data Commons Open Database License (ODbL)
}\newline
Créé le : 2015-06-20\newline
Modifié le : 2015-06-22\newline
De 2008-01-01 à 2012-12-31\newline
Mise à jour : ponctuelle\newline
Popularité : 1 réutilisation,  0 suivi\newline
Mots-clé : \emph{subventions
}\newline
Permalien : \url{https://data.gouv.fr/dataset/55853496c751df5ea9a453b9}\newline

\par
\noindent
    Ces données ont été extraites des fichiers PDF du site de la
\href{http://www.besancon.fr/}{ville de Besançon} lors du
\href{http://wiki.sequanux.org/index.php/Hackathon_Open_Data}{Hackathon
Sequanux} du 7 décembre 2013.


\vspace{0.5cm}
\needspace{3\baselineskip} \rule{4cm}{0.25pt}\newline\textbf{Aussi disponible du même producteur :}\begin{itemize}
\item \href{https://data.gouv.fr/dataset/543c57f888ee3805943c1649}{Contours des quartiers de Dijon}
\end{itemize}

\clearpage
\section{OpenDataSoft}


\begin{center}
  \includegraphics[width=3cm]{images/orga/2014-11-17_e32209b275224854bc00f953505e7872_250_blue-100.jpg}
\end{center}


OpenDataSoft est une société française proposant une solution Open Data
complète : publication - visualisation - partage - réutilisation des
données. Les clients d'OpenDataSoft, collectivités territoriales ou
administrations, grandes entreprises et médias disposent ainsi d'une
plateforme clef-en-main sur le cloud pour piloter la collecte, la
préparation, la visualisation et la diffusion via API de tous types de
données, sans contrainte de format ou de volume.


\vspace{0.5cm}

\needspace{12\baselineskip}
\subsection*{Correspondance entre les codes postaux et codes INSEE des communes
Françaises.
}\index{codes!insee}\index{codes!postaux}\index{communes}
  \begin{wrapfigure}{r}{2.5cm}
    \centering
    \qrcode[nolink]{https://data.gouv.fr/dataset/536991c5a3a729239d203d4b}
  \end{wrapfigure}

Licence : \textbf{Licence Ouverte
}\newline
Créé le : 2013-12-21\newline
Modifié le : 2016-03-16\newline
Granularité : à la commune\newline
Popularité : 5 réutilisations,  10 suivis\newline
Mots-clé : \emph{codes-insee, codes-postaux, communes
}\newline
Permalien : \url{https://data.gouv.fr/dataset/536991c5a3a729239d203d4b}\newline

\par
\noindent
    Ce jeu de données intègre par ailleurs des données de référence sur les
communes:

\begin{itemize}

\item
  Population
\item
  Superficie
\item
  Altitude moyenne
\item
  Forme géographique
\end{itemize}

Ce jeu de données a été constitué à partir des données de la base
GEOFLA® mise à disposition par l'IGN (codes INSEE) ainsi qu'à partir de
données Wikipédia (codes postaux).


\vspace{0.5cm}
\needspace{12\baselineskip}
\subsection*{Jeux de données - IAU-ÎdF
}\index{liste!de!jeux!de!donnees}
  \begin{wrapfigure}{r}{2.5cm}
    \centering
    \qrcode[nolink]{https://data.gouv.fr/dataset/53699775a3a729239d204ccf}
  \end{wrapfigure}

Licence : \textbf{Licence Ouverte
}\newline
Créé le : 2013-09-14\newline
Modifié le : 2016-03-04\newline
Popularité : 1 réutilisation,  0 suivi\newline
Mots-clé : \emph{liste-de-jeux-de-donnees
}\newline
Permalien : \url{https://data.gouv.fr/dataset/53699775a3a729239d204ccf}\newline

\par
\noindent
    Les jeux de données fournis par IAU-ÎdF pour data.gouv.fr.


\vspace{0.5cm}
\needspace{12\baselineskip}
\subsection*{Jeux de données - JCDecaux developer
}\index{liste!de!jeux!de!donnees}
  \begin{wrapfigure}{r}{2.5cm}
    \centering
    \qrcode[nolink]{https://data.gouv.fr/dataset/53699776a3a729239d204cd2}
  \end{wrapfigure}

Licence : \textbf{Licence Ouverte
}\newline
Créé le : 2013-09-14\newline
Modifié le : 2016-02-01\newline
Popularité : 2 réutilisations,  0 suivi\newline
Mots-clé : \emph{liste-de-jeux-de-donnees
}\newline
Permalien : \url{https://data.gouv.fr/dataset/53699776a3a729239d204cd2}\newline

\par
\noindent
    Les jeux de données fournis par JCDecaux developer pour data.gouv.fr.


\vspace{0.5cm}
\needspace{3\baselineskip} \rule{4cm}{0.25pt}\newline\textbf{Aussi disponible du même producteur :}\begin{itemize}
\item \href{https://data.gouv.fr/dataset/53698e48a3a729239d20340e}{Accessibilité des équipements de la Ville de Paris}
\item \href{https://data.gouv.fr/dataset/53698e7ba3a729239d20349b}{Adresses des panneaux d'affichage des Conseils de Quartier}
\item \href{https://data.gouv.fr/dataset/53698f14a3a729239d203644}{Arrondissements}
\item \href{https://data.gouv.fr/dataset/5369918da3a729239d203cb7}{conseils-quartierswgs84}
\item \href{https://data.gouv.fr/dataset/5369925ea3a729239d203ee3}{DemandeursEmploi-25-49-ABC-brutes-Corse}
\item \href{https://data.gouv.fr/dataset/53699761a3a729239d204c9a}{Jeux de données - Agence des espaces verts - IDF}
\item \href{https://data.gouv.fr/dataset/53699767a3a729239d204ca6}{Jeux de données - Autolib}
\item \href{https://data.gouv.fr/dataset/5369976da3a729239d204cb7}{Jeux de données - Conseil général des Hauts-de-Seine}
\item \href{https://data.gouv.fr/dataset/53699773a3a729239d204cca}{Jeux de données - Fédération Nationale des Bistrots de Pays}
\item \href{https://data.gouv.fr/dataset/53699774a3a729239d204ccc}{Jeux de données - GIP Corse Compétences}
\item \href{https://data.gouv.fr/dataset/53699777a3a729239d204cd4}{Jeux de données - Le RIF}
\item \href{https://data.gouv.fr/dataset/53699779a3a729239d204cd8}{Jeux de données - Mairie de Paris}
\item \href{https://data.gouv.fr/dataset/5369977fa3a729239d204ce8}{Jeux de données - OpenDataSoft}
\item \href{https://data.gouv.fr/dataset/53699788a3a729239d204cfd}{Jeux de données - Société nationale des chemins de fer français}
\item \href{https://data.gouv.fr/dataset/536997e5a3a729239d204df1}{Les 1000 titres les plus empruntés de 2012}
\item \href{https://data.gouv.fr/dataset/536997e7a3a729239d204df6}{les 1000 titres les plus empruntés par bibliothèque en 2012}
\item \href{https://data.gouv.fr/dataset/536998dfa3a729239d2050a3}{Liste des cafés à un euro}
\item \href{https://data.gouv.fr/dataset/53699920a3a729239d20516b}{Liste des kiosques presse à Paris}
\item \href{https://data.gouv.fr/dataset/53699929a3a729239d205182}{Liste des marchés de fournitures passés par la Ville de Paris}
\item \href{https://data.gouv.fr/dataset/5369993ba3a729239d2051c2}{Liste des ouvrages des bibliothèques en janvier 2009}
\item \href{https://data.gouv.fr/dataset/5369997ea3a729239d205279}{Logements sociaux financés à Paris}
\item \href{https://data.gouv.fr/dataset/536999f5a3a729239d205389}{Mobiliers et emprises au sol de signalisation routière et piétonne - Données géographiques}
\item \href{https://data.gouv.fr/dataset/53699a42a3a729239d205447}{Murs et Clôtures}
\item \href{https://data.gouv.fr/dataset/53699b2ca3a729239d20568a}{Ordres du jour du Conseil Général}
\item \href{https://data.gouv.fr/dataset/53699b2da3a729239d20568c}{Ordres du jour du Conseil Municipal}
\item \href{https://data.gouv.fr/dataset/53699b6aa3a729239d205723}{Panneaux d'affichages associatifs}
\item \href{https://data.gouv.fr/dataset/53699b6ca3a729239d205728}{Panneaux indicateurs de signalisation routière et piétonne}
\item \href{https://data.gouv.fr/dataset/53699c59a3a729239d20595c}{Petits Marchands sur l'espace public parisien}
\item \href{https://data.gouv.fr/dataset/53699c98a3a729239d2059f7}{Plan de l'hémicycle du conseil de Paris}
\item \href{https://data.gouv.fr/dataset/580dee9da3a7292dcea9d307}{POD Database - Brand Owners - 2013/12/20}
\item \href{https://data.gouv.fr/dataset/580dee9da3a7292dcea9d306}{POD Database - Brands - 2013/12/20}
\item \href{https://data.gouv.fr/dataset/580dee9da3a7292dcea9d308}{POD Database - GCP Codes - 2013/12/20}
\item \href{https://data.gouv.fr/dataset/580dee9da3a7292dcfa9d32f}{POD Database - GCP Codes, Aggregates per Logistical Owner - 2013/12/20}
\item \href{https://data.gouv.fr/dataset/580dee9da3a7292dcfa9d32e}{POD Database - GCP Codes, Aggregates per Registration Owner - 2013/12/20}
\item \href{https://data.gouv.fr/dataset/580dee9da3a7292dcea9d309}{POD Database - GS1 Prefix List}
\item \href{https://data.gouv.fr/dataset/580dee9da3a7292dcfa9d32d}{POD Database - GTIN Codes - 2013/12/20}
\item \href{https://data.gouv.fr/dataset/580dee9da3a7292dcfa9d32c}{POD Database - Nutrition, GTIN Codes - 2013/12/20}
\item \href{https://data.gouv.fr/dataset/53699ecba3a729239d205f86}{Références les plus réservées en 2011}
\item \href{https://data.gouv.fr/dataset/54ddb7c0c751df19c7467389}{Réserve Parlementaire 2014}
\item \href{https://data.gouv.fr/dataset/5369a06fa3a729239d20637a}{Statistiques de création d'actes d'état-civil}
\item \href{https://data.gouv.fr/dataset/5369a163a3a729239d2065c3}{Tarification du stationnement au mois d’août 2012 selon les rues}
\item \href{https://data.gouv.fr/dataset/5369a290a3a729239d206880}{Transport Aerien}
\item \href{https://data.gouv.fr/dataset/5369a32fa3a729239d2069e5}{Utilisations mensuelles des hotspots Paris Wi-Fi}
\end{itemize}

\clearpage
\section{OpenEventDatabase}


\begin{center}
  \includegraphics[width=3cm]{images/orga/96_c3f91a58b248c6a3e97d3a33dc029a-100.png}
\end{center}


OpenEventDatabase est un projet dont le but est de partager des données
concernant des événements publics passés, présents ou futurs localisés
dans l'espace et le temps.

On peut ainsi y déposer des informations concernant par exemple: - la
circulation routière: bouchons, accidents, routes coupées, obstacles,
dangers divers - la météo: alertes météo et phénomènes exceptionnels
(vent, brouillard, orages, inondations, etc) - des mesures diverse:
hauteurs des cours d'eau, informations météo, niveau de pollution - des
horaires de transports, d'évènement sportifs, culturels - des événements
touristiques: festival, concert, brocante, vide-grenier, visites\ldots{}
etc\ldots{}

OpenEventDatabase est accessible via une API REST documentée
sur\url{https://github.com/openeventdatabase/backend/wiki} Les jeux de
données publiés ici sont des extraits provenant en temps réel de cette
API.


\vspace{0.5cm}

\needspace{12\baselineskip}
\subsection*{Fête de la musique 2016
}\index{concerts}\index{culture}\index{fete}\index{fete!de!la!musique}\index{musique}
  \begin{wrapfigure}{r}{2.5cm}
    \centering
    \qrcode[nolink]{https://data.gouv.fr/dataset/57680400c751df0bdb537a11}
  \end{wrapfigure}

Licence : \textbf{Open Data Commons Open Database License (ODbL)
}\newline
Créé le : 2016-06-20\newline
Modifié le : 2016-06-20\newline
Granularité : au point d'intérêt\newline
Mise à jour : ponctuelle\newline
Popularité : 3 réutilisations,  0 suivi\newline
Mots-clé : \emph{concerts, culture, fete, fete-de-la-musique, musique
}\newline
Permalien : \url{https://data.gouv.fr/dataset/57680400c751df0bdb537a11}\newline

\par
\noindent
    Les données ont été extraites le 17 juin depuis le flux json disponible
sur\url{https://openagenda.com/fetedelamusique2016/actions/}et remises à
disposition via l'API OpenEventDatabase.


\vspace{0.5cm}
\needspace{12\baselineskip}
\subsection*{Paris 2016 OGP Summit events
}\index{ogp}
  \begin{wrapfigure}{r}{2.5cm}
    \centering
    \qrcode[nolink]{https://data.gouv.fr/dataset/583c7823c751df548dc0bb7e}
  \end{wrapfigure}

Licence : \textbf{Licence Ouverte
}\newline
Créé le : 2016-11-28\newline
Modifié le : 2016-11-28\newline
De 2016-12-07 à 2016-12-09\newline
Granularité : au point d'intérêt\newline
Mise à jour : ponctuelle\newline
Popularité : 1 réutilisation,  2 suivis\newline
Mots-clé : \emph{ogp
}\newline
Permalien : \url{https://data.gouv.fr/dataset/583c7823c751df548dc0bb7e}\newline

\par
\noindent
    This dataset contains the list of OGP Summit events from dec 7th to 9th
2016.


\vspace{0.5cm}
\needspace{3\baselineskip} \rule{4cm}{0.25pt}\newline\textbf{Aussi disponible du même producteur :}\begin{itemize}
\item \href{https://data.gouv.fr/dataset/57a35375c751df7317bb5dd4}{Evénements de la Nuit des Étoiles 2016}
\item \href{https://data.gouv.fr/dataset/5c18f912634f412d50a47cc2}{Incidents sur ligne RER B}
\item \href{https://data.gouv.fr/dataset/57fa503bc751df60f679df72}{Liste des photographies aériennes anciennes de Montreuil-en-Touraine}
\item \href{https://data.gouv.fr/dataset/5759cb3dc751df04baac31a0}{Matchs de football de l'Euro2016}
\item \href{https://data.gouv.fr/dataset/579e32e088ee386754d73ff6}{Prévisions de trafic de bison futé}
\end{itemize}

\clearpage
\section{Open Food Facts}


\begin{center}
  \includegraphics[width=3cm]{images/orga/b3_9cc0c7e967416c8994d485701ae5e1-100.png}
\end{center}


Open Food Facts répertorie les produits alimentaires du monde entier.
Les informations sur les aliments (photos, ingrédients, composition
nutritionnelle etc.) sont collectées de façon collaborative et mises à
disposition de tous et pour tous usages dans une base de données
ouverte, libre et gratuite.

Ces données sont ensuite réutilisables et redistribuables librement et
gratuitement pour :

\begin{itemize}

\item
  Vous aider à faire de meilleurs choix
\item
  Inciter les industriels à proposer des produits plus sains
\item
  Aider la recherche
\end{itemize}


\vspace{0.5cm}

\needspace{12\baselineskip}
\subsection*{Open Beauty Facts
}\index{cosmetiques}\index{crowdsourcing}
  \begin{wrapfigure}{r}{2.5cm}
    \centering
    \qrcode[nolink]{https://data.gouv.fr/dataset/5985d205c751df6d505e7e86}
  \end{wrapfigure}

Licence : \textbf{Open Data Commons Open Database License (ODbL)
}\newline
Créé le : 2017-08-05\newline
Modifié le : 2017-08-05\newline
Popularité : 1 réutilisation,  1 suivi\newline
Mots-clé : \emph{cosmetiques, crowdsourcing
}\newline
Permalien : \url{https://data.gouv.fr/dataset/5985d205c751df6d505e7e86}\newline

\par
\noindent
    Open Food Facts répertorie les informations sur les produits cosmétiques
: ingrédients, additifs, labels etc. Les données proviennent
majoritairement de la collecte citoyenne (crowdsourcing) des
informations.


\vspace{0.5cm}

\clearpage
\section{OpenHealth Company}


\begin{center}
  \includegraphics[width=3cm]{images/orga/4c_e483811d8a4b38996995946f5f3f74-100.png}
\end{center}


\textbf{OpenHealth Company}

Née du rapprochement de 3 entités complémentaires, OpenHealth Company
est devenu un leader dans la collecte et le traitement des données de
santé. Nos équipes travaillent jour après jour pour explorer les
possibilités d'analyse ouvertes par les nouvelles technologies.
Préserver des systèmes de santé soumis à des contraintes financières
croissantes, anticiper et gérer les crises sanitaires, améliorer
l'efficience de nos politiques, rendre le patient acteur de son
parcours, effectuer des choix stratégiques en matière de
recherche\ldots{} autant de défis qui pourront être relevés grâce
notamment à l'usage raisonné des données de santé, avec rigueur, éthique
et efficacité.


\vspace{0.5cm}

\needspace{12\baselineskip}
\subsection*{Etude sur la consommation de traitements contre la dysfonction érectile
}\index{dysfonctionnement!erectile}\index{etude}\index{etudes}\index{etudes!medicales}\index{sante}
  \begin{wrapfigure}{r}{2.5cm}
    \centering
    \qrcode[nolink]{https://data.gouv.fr/dataset/53699509a3a729239d2045c4}
  \end{wrapfigure}

Licence : \textbf{Open Data Commons Open Database License (ODbL)
}\newline
Créé le : 2014-01-17\newline
Modifié le : 2016-03-15\newline
De 2009-01-01 à 2013-12-31\newline
Granularité : à la région\newline
Mise à jour : ponctuelle\newline
Popularité : 3 réutilisations,  53 suivis\newline
Mots-clé : \emph{dysfonctionnement-erectile, etude, etudes, etudes-medicales, sante
}\newline
Permalien : \url{https://data.gouv.fr/dataset/53699509a3a729239d2045c4}\newline

\par
\noindent
    Etude sur la consommation de traitements contre la dysfonction érectile.

Données issues des sorties consommateurs des officines du réseau
CELTIPHARM.

Plus d'infos sur nos
méthodes:\url{http://www.openhealth.fr/methode}\url{http://www.openhealth.fr/}


\vspace{0.5cm}
\needspace{12\baselineskip}
\subsection*{Indicateur Avancé Sanitaire IAS® - GASTRO-ENTERITE
}\index{epidemiologie}\index{gastro!enterite}\index{ias}\index{sante}
  \begin{wrapfigure}{r}{2.5cm}
    \centering
    \qrcode[nolink]{https://data.gouv.fr/dataset/53699667a3a729239d2049ac}
  \end{wrapfigure}

Licence : \textbf{Open Data Commons Open Database License (ODbL)
}\newline
Créé le : 2013-12-29\newline
Modifié le : 2016-06-21\newline
De 2008-07-01 à 2016-06-01\newline
Granularité : au département\newline
Mise à jour : quotienne\newline
Popularité : 4 réutilisations,  59 suivis\newline
Mots-clé : \emph{epidemiologie, gastro-enterite, ias, sante
}\newline
Permalien : \url{https://data.gouv.fr/dataset/53699667a3a729239d2049ac}\newline

\par
\noindent
    L'objectif de l'IAS® Incidence de la gastro-entérite est de réaliser une
surveillance syndromique de la gastro-entérite en France. La
gastro-entérite est un syndrome pouvant avoir de nombreuses origines :
bactérienne (consommation d'eau ou de nourriture contaminée par des
bactéries), parasitaire (protozoaires, amibes) ou virales (Rotavirus,
Norovirus\ldots{}). La gastro-entérite sévit toute l'année, mais il y a
chaque hiver une épidémie d'origine virale.

Données issues des sorties consommateurs des officines du réseau
CELTIPHARM.

Plus d'infos sur nos méthodes
:\url{http://ias.openhealth.fr/methode}\url{http://ias.openhealth.fr/}
Les données sont mises à jour quotidiennement. Adopter le J+1 !


\vspace{0.5cm}
\needspace{12\baselineskip}
\subsection*{Indicateur Avancé Sanitaire IAS® - SYNDROME GRIPPAL
}\index{epidemiologie}\index{grippe}\index{ias}\index{sante}\index{syndrome!grippal}
  \begin{wrapfigure}{r}{2.5cm}
    \centering
    \qrcode[nolink]{https://data.gouv.fr/dataset/53699668a3a729239d2049e8}
  \end{wrapfigure}

Licence : \textbf{Open Data Commons Open Database License (ODbL)
}\newline
Créé le : 2014-01-08\newline
Modifié le : 2016-06-21\newline
De 2009-07-01 à 2016-06-01\newline
Granularité : au pays\newline
Mise à jour : quotienne\newline
Popularité : 7 réutilisations,  59 suivis\newline
Mots-clé : \emph{epidemiologie, grippe, ias, sante, syndrome-grippal
}\newline
Permalien : \url{https://data.gouv.fr/dataset/53699668a3a729239d2049e8}\newline

\par
\noindent
    L'objectif de l'Indicateur Avancé Sanitaire (IAS®) ``Syndrome Grippal''
est de contribuer à la surveillance des syndromes grippaux en France en
apportant des informations complémentaires à celles du réseau
Sentinelles. Cet indicateur a été validé par comparaison avec les
données du réseau Sentinelles. Au niveau national, La corrélation
croisée avec le réseau Sentinelle est forte (0,88). Cette corrélation
valide la pertinence de l'IAS®. L'IAS® est calculé chaque jour en
employant une méthode de lissage temporel : les informations des sept
jours précédents et des 7 jours suivants sont prises en compte pour
calculer la valeur d'un jour donné. Ceci fait que l'indice d'un jour J
peut légèrement évoluer jusqu'à J+7.

Données issues des sorties consommateurs des officines du réseau
CELTIPHARM.

Plus d'infos sur nos
méthodes:\url{http://ias.openhealth.fr/methode}\url{http://ias.openhealth.fr/}
Les données sont mises à jour quotidiennement. Adopter le J+1 !


\vspace{0.5cm}
\needspace{3\baselineskip} \rule{4cm}{0.25pt}\newline\textbf{Aussi disponible du même producteur :}\begin{itemize}
\item \href{https://data.gouv.fr/dataset/53699327a3a729239d2040f6}{Données de l’Institut de Veille Sanitaire sur la surveillance liée au syndrome grippal}
\item \href{https://data.gouv.fr/dataset/5432a99f88ee38259e396b4d}{Indicateur Avancé Sanitaire IAS® - BACLOFENE ET DEPENDANCE ALCOOLIQUE}
\item \href{https://data.gouv.fr/dataset/563898e688ee38015a531575}{Indicateur Avancé Sanitaire IAS®  : INCIDENCE DE LA GALE}
\item \href{https://data.gouv.fr/dataset/5432a6f888ee38259f396b41}{Indicateur Avancé Sanitaire IAS® - INCIDENCE DES POUX}
\end{itemize}

\clearpage
\section{OpenStreetMap}


\begin{center}
  \includegraphics[width=3cm]{images/orga/c7_513a9d32e84c1dbcc6ac7e2ce42cb8-100.png}
\end{center}


Le wiki cartographique mondial qui crée et fournit des données
géographiques sous licence libre \href{http://osm.org/copyright}{ODbL}.
OSM est représenté en France par
\href{http://openstreetmap.fr/}{OpenStreetMap France}, association régie
par la loi de 1901 créée en Octobre 2011.

Pour tout usage de ces données, vous avez obligation de préciser la
source: ``© les contributeurs OpenStreetMap'' et de partager à
l'identique les données dérivées. Voir
\url{http://osm.org/copyright}{]}(http://osm.org/copyright{]}(http://osm.org/copyright))pour
plus de détails.


\vspace{0.5cm}

\needspace{12\baselineskip}
\subsection*{Base d'Adresses Nationale Ouverte - BANO
}\index{adresse}\index{adresses}\index{bano}\index{geocodage}\index{openstreetmap}\index{plaque}
  \begin{wrapfigure}{r}{2.5cm}
    \centering
    \qrcode[nolink]{https://data.gouv.fr/dataset/538071a9a3a7297e4d35d6ce}
  \end{wrapfigure}

Licence : \textbf{Open Data Commons Open Database License (ODbL)
}\newline
Créé le : 2014-05-24\newline
Modifié le : 2018-11-29\newline
Granularité : au point d'intérêt\newline
Mise à jour : quotienne\newline
Popularité : 7 réutilisations,  25 suivis\newline
Mots-clé : \emph{adresse, adresses, bano, geocodage, openstreetmap, plaque
}\newline
Permalien : \url{https://data.gouv.fr/dataset/538071a9a3a7297e4d35d6ce}\newline

\par
\noindent
    Ce jeu de données constitué de plus de 16 millions d'adresses provient
du projet de Base d'Adresses Nationale Ouverte initié par OpenStreetMap
France.

Pour plus de renseignements sur ce projet :
\url{http://prev.openstreetmap.fr/bano}{]}(http://prev.openstreetmap.fr/bano{]}(http://prev.openstreetmap.fr/bano))
\textbf{Origine des données}

BANO est une base de données composite, constituée à partir de
différentes sources:

\begin{itemize}

\item
  OpenStreetMap (ODbL) - (mise à jour quotidienne)
\item
  données diffusées en opendata par les collectivités
\item
  Arles Crau Camargue Montagnette (Licence Ouverte) - avril 2016
\item
  Ville de Montpellier (Licence Ouverte) - mai 2016
\item
  Toulouse Métropole (ODbL) - février 2015
\item
  Rennes Métropole (Licence Ouverte) - avril 2016
\item
  Mulhouse Alsace Agglomération (Licence Ouverte) - février 2015
\item
  Grand Nancy (ODbL) - avril 2016
\item
  Nantes Métropole (ODbL) - janvier 2015
\item
  Grand Lyon (Licence Ouverte) - janvier 2015
\item
  Bordeaux Métropole (ODbL) - février 2015
\item
  Strasbourg EuroMétropole (Licence Ouverte) - janvier 2014
\item
  Lille Métropole Européenne (Licence Ouverte) - juin 2014
\item
  Métropole Nice Côte d'Azur (Licence Ouverte) - janvier 2015
\item
  Ville de Limoges (ODbL) - décembre 2014
\item
  Ville de La Rochelle (Licence Ouverte) - octobre 2015
\item
  Pays de Brest (Licence Ouverte) - mars 2016
\item
  Mairie de Nanterre (Licence Ouverte) - février 2015
\item
  Ville de Grenoble (ODbL) - mars 2015
\item
  Ville de Paris (ODbL) - janvier 2015
\item
  Ville de Poitiers (Licence Ouverte) - avril 2015
\item
  Ville d'Angers (ODbL) - avril 2016
\item
  données adresses collectées sur le site du cadastre et fichier FANTOIR
  (Licence Ouverte) - janvier 2016
\item
  Base officielle des codes postaux du Groupe La Poste (Licence Ouverte)
  - novembre 2014
\end{itemize}

\textbf{Licence}

Ces données sont sous licence \textbf{ODbL} (Open Database Licence).
Cette licence implique: l'attribution et le partage à l'identique.

\begin{itemize}

\item
  Pour la mention d'attribution veuillez indiquer ``\textbf{source:
  BANO}'' ainsi que la date du jeu de données.
\item
  Pour le partage à l'identique, \textbf{toute amélioration des données
  de BANO doit être repartagée sous licence identique}. Merci de nous
  faire remonter: les adresses manquantes, les adresses mal
  positionnées, les erreurs de libellés.
\end{itemize}

\textbf{Format de diffusion}

Ces fichiers sont proposés sous forme de :

\begin{itemize}

\item
  fichiers csv (projection WGS84/EPSG:4326 et textes en UTF-8)
\item
  fichiers shapefile (EPSG:4326 et UTF-8)
\item
  données RDF au format turtle
\end{itemize}

Ces données sont téléchargeables sur
\url{http://bano.openstreetmap.fr/data/}{]}(http://bano.openstreetmap.fr/data/{]}(http://bano.openstreetmap.fr/data/))et
mises à jour \textbf{quotidiennement}.

\textbf{Contenu des fichiers}

Pour chaque adresse:

\begin{itemize}

\item
  id (unique) : code\_insee + codefantoir + numero
\item
  numero : numéro dans la voie avec suffixe (ex: 1, 1BIS, 1D)
\item
  voie : nom de voie
\item
  code\_post : code postal sur 5 caractères
\item
  nom\_comm : nom de la commune
\item
  source : OSM = donnée directement issue d'OpenStreetMap, OD = donnée
  provenant de source open data locales, O+O = donnée opendata enrichie
  par OSM, CAD = donnée directement issue du cadastre, C+O = donnée du
  cadastre enrichie par OSM (nom de voie par exemple)
\item
  lat : latitude en degrés décimaux WGS84
\item
  lon : longitude en degrés décimaux WGS84
\end{itemize}

\textbf{Historique du jeu de données}

1 septembre 2014 : 15.109.578 adresses

9 juin 2014 : 14.9 millions d'adresses

\begin{itemize}

\item
  Première version à intégrer des données adresses disponibles en
  opendata.
\item
  Amélioration des rapprochements avec les données OSM
\item
  Prise en compte des améliorations des données OSM au 7 juin 2014
\item
  Elimination des numéros d'adresses non conformes.
\end{itemize}

\textbf{Mise à jour, corrections}

Pour mettre à jour et corriger les données de BANO, il est déjà possible
de faire des améliorations directement dans OpenStreetMap (par exemple
par l'ajout d'adresses à une meilleure position, par la correction des
noms de rues). Ces modifications seront prises en compte au prochain
cycle de mise à jour de BANO.

\emph{Un guichet unique collaboratif de signalement/correction va
prochainement être mis en place pour simplifier le processus
d'amélioration du contenu de la base. Pour participer à sa
co-construction, n'hésitez pas à nous contacter !}

Pour toute question concernant le projet ou ce jeu de données, vous
pouvez contacter bano@openstreetmap.fr


\vspace{0.5cm}
\needspace{12\baselineskip}
\subsection*{Cartes des régions et départements issus d'OpenStreetMap
}\index{administratif}\index{crowdsourcing}\index{decoupage}\index{departements}\index{openstreetmap}\index{regions}
  \begin{wrapfigure}{r}{2.5cm}
    \centering
    \qrcode[nolink]{https://data.gouv.fr/dataset/53699035a3a729239d203942}
  \end{wrapfigure}

Licence : \textbf{Creative Commons Attribution Share-Alike
}\newline
Créé le : 2013-11-17\newline
Modifié le : 2018-09-25\newline
Granularité : à la région\newline
Mise à jour : ponctuelle\newline
Popularité : 4 réutilisations,  1 suivi\newline
Mots-clé : \emph{administratif, crowdsourcing, decoupage, departements, openstreetmap, regions
}\newline
Permalien : \url{https://data.gouv.fr/dataset/53699035a3a729239d203942}\newline

\par
\noindent
    Fichiers SVG du découpage des régions et départements métropolitains
issus des données OpenStreetMap.

Ils ont été produits à partir de données sous licence ODbL issues du
crowdsourcing effectué par les contributeurs au projet OpenStreetMap.

La simplification géométrique a été faite avec mapshaper.


\vspace{0.5cm}
\needspace{12\baselineskip}
\subsection*{Communes nouvelles au 1er janvier 2016
}\index{commune}\index{communes}\index{communes!nouvelles}\index{decoupage!adminsitratif}\index{fusion}\index{fusion!de!communes}
  \begin{wrapfigure}{r}{2.5cm}
    \centering
    \qrcode[nolink]{https://data.gouv.fr/dataset/568558dec751df503ac664bc}
  \end{wrapfigure}

Licence : \textbf{Open Data Commons Open Database License (ODbL)
}\newline
Créé le : 2015-12-31\newline
Modifié le : 2017-01-23\newline
Granularité : à la commune\newline
Mise à jour : ponctuelle\newline
Popularité : 3 réutilisations,  1 suivi\newline
Mots-clé : \emph{commune, communes, communes-nouvelles, decoupage-adminsitratif, fusion, fusion-de-communes
}\newline
Permalien : \url{https://data.gouv.fr/dataset/568558dec751df503ac664bc}\newline

\par
\noindent
    Ce jeu de données contient la liste des communes nouvelles créées au 1er
janvier 2016. Il a été créé par ``crowdsourcing'' par plusieurs
contributeurs, \textbf{il ne s'agit donc pas d'un fichier officiel.}

Sa constitution s'est faite à partir des arrêtés publiés au JORF, des
arrêtés préfectoraux consultables sur les sites web des préfectures
(dans les Registres des Actes Administratifs) et de sources locales
(presse, sites web municipaux), en s'appuyant sur deux pages de wiki: -
\href{https://wiki.openstreetmap.org/wiki/WikiProject_France/Limites_administratives/Modifications_planifi\%C3\%A9es\#Fusions_au_01.2F01.2F2016}{Wiki
OpenStreetMap} -
\href{https://fr.wikipedia.org/wiki/Projet_de_communes_fran\%C3\%A7aises_nouvelles_en_2016}{Wikipédia}

\textbf{Si vous relevez des erreurs ou omissions, merci de les signaler
en ouvrant une discussion ci-dessous ou
\href{https://github.com/cquest/fusion-communes-2016/issues}{via une
issue sur github}.}

\textbf{Complément du 7 janvier 2016:}

Ajout d'un fichier ``détail'' avec une ligne par ancienne commune,
incluant son code INSEE. Pour les anciennes communes qui avaient été
fusionnées dans les années 70, leur code INSEE de l'époque a été utilisé
(source historique du Code Officiel Géographique publié par l'INSEE). Le
code INSEE des communes nouvelles a été fixé par défaut à celui du
chef-lieu lorsque celui-ci est connu.

\textbf{Complément du 18 janvier 2016:}

Le fichier détail a été complété à partir des
\href{http://www.insee.fr/fr/methodes/default.asp?page=nomenclatures/cog/communes-nouvelles-2015.htm}{données
diffusées par l'INSEE depuis le 13 janvier 2016 sur son site}.


\vspace{0.5cm}
\needspace{12\baselineskip}
\subsection*{Contours des arrondissements français issus d'OpenStreetMap
}
  \begin{wrapfigure}{r}{2.5cm}
    \centering
    \qrcode[nolink]{https://data.gouv.fr/dataset/536991b0a3a729239d203d11}
  \end{wrapfigure}

Licence : \textbf{Open Data Commons Open Database License (ODbL)
}\newline
Créé le : 2013-12-20\newline
Modifié le : 2016-02-29\newline
De 2013-01-01 à 2013-12-31\newline
Mise à jour : mensuelle\newline
Popularité : 1 réutilisation,  2 suivis\newline
Mots-clé : \emph{aucun
}\newline
Permalien : \url{https://data.gouv.fr/dataset/536991b0a3a729239d203d11}\newline

\par
\noindent
    Exports du découpage administratif français au niveau des
arrondissements issu d'OpenStreetMap produit dans sa grande majorité à
partir du cadastre.

Ces données sont issues du crowdsourcing effectué par les contributeurs
au projet OpenStreetMap et sont sous licence ODbL qui impose un partage
à l'identique et la mention obligatoire d'attribution doit être ``© les
contributeurs d'OpenStreetMap sous licence ODbL'' conformément à
\url{http://osm.org/copyright}{]}(http://osm.org/copyright{]}(http://osm.org/copyright))
Il s'agit d'un export semi-automatique avec des géométries allégées et
vérifiées topologiquement (pas de chevauchement).

\subsubsection{\texorpdfstring{Descriptif du contenu des fichiers
``arrondissements''}{Descriptif du contenu des fichiers arrondissements}}\label{descriptif-du-contenu-des-fichiers-arrondissements}

\textbf{Origine}

Les données proviennent de la base de données cartographiques
OpenStreetMap. Celles-ci ont été constituées à partir du cadastre mis à
disposition par la DGFiP sur cadastre.gouv.fr. En complément sur Mayotte
où le cadastre n'est pas disponible sur cadastre.gouv.fr, ce sont les
limites du GEOFLA de l'IGN qui ont été utilisées ainsi que le tracé des
côtes à partir des images aériennes de Bing.

Plus d'infos:\url{http://openstreetmap.fr/36680-communes}

\textbf{Format}

Ces fichiers sont proposés au format shapefile, en projection WGS84 avec
plusieurs niveaux de détails:

\begin{itemize}

\item
  simplification à 5m
\item
  simplification à 50m
\item
  simplification à 100m
\end{itemize}

La topologie est conservée lors du processus de simplification
(cf:\href{http://openstreetmap.fr/blogs/cquest/limites-administratives-simplifiees}{http://openstreetmap.fr/blogs/cquest/limites-administratives-simplifiees)})

\textbf{Contenu}

Ces fichiers contiennent l'ensemble des arrondissements français, y
compris les DOM, mais sans Mayotte.

Pour chaque arrondissement, les attributs suivants sont ajoutés:

\begin{itemize}

\item
  insee\_ar: code INSEE de l'arrondissement
\item
  nom: nom de l'arondissement
\item
  wikipedia: entrée wikipédia (code langue suivi du nom de l'article)
\item
  nb\_comm: nombre de communes dans l'arrondissement
\item
  surf\_km2: superficie de l'arrondissement en km2 sur le sphéroid WGS84
\end{itemize}

Pour toute question concernant ces exports, vous pouvez contacter
exports@openstreetmap.fr

Voir aussi :

\begin{itemize}

\item
  \href{https://www.data.gouv.fr/fr/dataset/decoupage-administratif-communal-francais-issu-d-openstreetmap}{Contours
  des communes françaises}
\item
  \href{https://www.data.gouv.fr/fr/dataset/contours-des-epci-2014}{Contours
  des EPCI 2014} et
  \href{https://www.data.gouv.fr/fr/dataset/contours-des-epci-2013}{Contours
  des EPCI 2013}
\item
  \href{https://www.data.gouv.fr/fr/dataset/contours-des-departements-francais-issus-d-openstreetmap}{Contours
  des départements français} et
  \href{https://www.data.gouv.fr/fr/dataset/cartes-des-regions-et-departements-issus-d-openstreetmap}{Cartes
  SVG des départements}
\item
  \href{https://www.data.gouv.fr/fr/dataset/contours-des-regions-francaises-sur-openstreetmap}{Contours
  de régions françaises}
\end{itemize}


\vspace{0.5cm}
\needspace{12\baselineskip}
\subsection*{Contours des EPCI 2013
}\index{crowdsourcing}\index{decoupage}\index{epci}\index{epci!a!fiscalite!propre}\index{limites}\index{openstreetmap}\index{osm}
  \begin{wrapfigure}{r}{2.5cm}
    \centering
    \qrcode[nolink]{https://data.gouv.fr/dataset/536991b1a3a729239d203d15}
  \end{wrapfigure}

Licence : \textbf{Open Data Commons Open Database License (ODbL)
}\newline
Créé le : 2013-12-20\newline
Modifié le : 2016-02-27\newline
De 2013-01-01 à 2013-12-31\newline
Granularité : à l'EPCI\newline
Mise à jour : ponctuelle\newline
Popularité : 2 réutilisations,  1 suivi\newline
Mots-clé : \emph{crowdsourcing, decoupage, epci, epci-a-fiscalite-propre, limites, openstreetmap, osm
}\newline
Permalien : \url{https://data.gouv.fr/dataset/536991b1a3a729239d203d15}\newline

\par
\noindent
    Contours géographiques des EPCI issu du croisement des limites
communales d'OpenStreetMap et des données du Ministère de l'Intérieur
sur les EPCI datant de 2013.

Ces données sont en partie issues du crowdsourcing effectué par les
contributeurs au projet OpenStreetMap et sont donc sous licence ODbL qui
impose un partage à l'identique et la mention obligatoire d'attribution
doit être ``© les contributeurs d'OpenStreetMap sous licence ODbL''
conformément à
\url{http://osm.org/copyright}{]}(http://osm.org/copyright{]}(http://osm.org/copyright))
\#\#Descriptif du contenu des fichiers ``epci''

\subsubsection{Origine des données}\label{origine-des-donnuxe9es}

Les données proviennent du Ministère de l'Intérieur croisées avec le
découpage communal issu de la base de données cartographiques
OpenStreetMap. Ces dernières ont été constituées à partir du cadastre
mis à disposition par la DGFiP sur cadastre.gouv.fr. En complément sur
Mayotte où le cadastre n'est pas disponible sur cadastre.gouv.fr, ce
sont les limites du GEOFLA de l'IGN qui ont été utilisées ainsi que le
tracé des côtes à partir des images aériennes de Bing.

Source pour les EPCI
2013:\url{https://www.data.gouv.fr/fr/dataset/adhesion-des-communes-a-un-etablissement-public-de-cooperation-intercommunale-epci-a-fiscal-00000000}
Plus d'infos:\url{http://openstreetmap.fr/36680-communes}

\subsubsection{Format}\label{format}

Ces fichiers sont proposés au format shapefile, en projection WGS84 avec
plusieurs niveaux de détails:

\begin{itemize}

\item
  simplification à 5m
\item
  simplification à 50m
\item
  simplification à 100m
\end{itemize}

La topologie est conservée lors du processus de simplification
(cf:\href{http://openstreetmap.fr/blogs/cquest/limites-administratives-simplifiees}{http://openstreetmap.fr/blogs/cquest/limites-administratives-simplifiees)})

\subsubsection{Contenu}\label{contenu}

Ces fichiers contiennent l'ensemble des EPCI figurant dans le fichier du
Ministère de l'Intérieur (voir ``Origine des données'').

Pour chaque EPCI, les attributs suivants sont fournis:

\begin{itemize}

\item
  siren\_epci : code SIREN attribué par l'INSEE à l'EPCI (source Min.
  Intérieur)
\item
  nom\_epci : nom de l'EPCI (source Min. Intérieur)
\item
  ptot\_epci : population totale de l'EPCI (source Min. Intérieur)
\item
  nb\_comm : nombre de communes dans l'EPCI
\item
  surf\_km2 : superficie de l'EPCI en km2 sur le sphéroid WGS84
\end{itemize}

\subsubsection{Historique}\label{historique}

\begin{itemize}

\item
  20-12-2013 : première génération du fichier, basé sur le découpage
  communal OSM au 19-12-2013
\end{itemize}

Pour toute question concernant ces exports, vous pouvez contacter
exports@openstreetmap.fr

Voir aussi :

\begin{itemize}

\item
  \href{https://www.data.gouv.fr/fr/dataset/contours-des-epci-2014}{Contours
  des EPCI 2014}
\item
  \href{https://www.data.gouv.fr/fr/dataset/decoupage-administratif-communal-francais-issu-d-openstreetmap}{Contours
  des communes françaises}
\item
  \href{https://www.data.gouv.fr/fr/dataset/contours-des-arrondissements-francais-issus-d-openstreetmap}{Contours
  des arrondissements français}
\item
  \href{https://www.data.gouv.fr/fr/dataset/contours-des-departements-francais-issus-d-openstreetmap}{Contours
  des départements français} et
  \href{https://www.data.gouv.fr/fr/dataset/cartes-des-regions-et-departements-issus-d-openstreetmap}{Cartes
  SVG des départements}
\item
  \href{https://www.data.gouv.fr/fr/dataset/contours-des-regions-francaises-sur-openstreetmap}{Contours
  de régions françaises}
\end{itemize}


\vspace{0.5cm}
\needspace{12\baselineskip}
\subsection*{Contours des EPCI 2014
}\index{communes}\index{decoupage}\index{decoupage!administratif}\index{decoupage!adminsitratif}\index{epci}\index{epci!a!fiscalite!propre}\index{openstreetmap}
  \begin{wrapfigure}{r}{2.5cm}
    \centering
    \qrcode[nolink]{https://data.gouv.fr/dataset/536991b1a3a729239d203d17}
  \end{wrapfigure}

Licence : \textbf{Open Data Commons Open Database License (ODbL)
}\newline
Créé le : 2014-03-06\newline
Modifié le : 2016-03-16\newline
De 2014-01-01 à 2014-12-31\newline
Granularité : à l'EPCI\newline
Mise à jour : semestrielle\newline
Popularité : 1 réutilisation,  4 suivis\newline
Mots-clé : \emph{communes, decoupage, decoupage-administratif, decoupage-adminsitratif, epci, epci-a-fiscalite-propre, openstreetmap
}\newline
Permalien : \url{https://data.gouv.fr/dataset/536991b1a3a729239d203d17}\newline

\par
\noindent
    Contours géographiques des EPCI issu du croisement des limites
communales d'OpenStreetMap et des données de la Direction Générale des
Collectivités Locales datant de 2014.

Ces données sont en partie issues du crowdsourcing effectué par les
contributeurs au projet OpenStreetMap et sont donc sous licence ODbL qui
impose un partage à l'identique et la mention obligatoire d'attribution
doit être ``\textbf{© les contributeurs d'OpenStreetMap sous licence
ODbL}'' conformément à
\url{http://osm.org/copyright}{]}(http://osm.org/copyright{]}(http://osm.org/copyright))
\#\#Descriptif du contenu des fichiers ``epci''

\subsubsection{Origine des données}\label{origine-des-donnuxe9es}

Les données proviennent de la Direction Générale des Collectivités
Locales (DGCL) croisées avec le découpage communal issu de la base de
données cartographiques OpenStreetMap. Ces dernières ont été constituées
à partir du cadastre mis à disposition par la DGFiP sur
cadastre.gouv.fr.

Source pour les EPCI 2014:
\url{http://www.collectivites-locales.gouv.fr/liste-et-composition-2014}{]}(http://www.collectivites-locales.gouv.fr/liste-et-composition-2014{]}(http://www.collectivites-locales.gouv.fr/liste-et-composition-2014))

\subsubsection{Format}\label{format}

Ces fichiers sont proposés au format shapefile, en projection WGS84 avec
plusieurs niveaux de détails:

\begin{itemize}

\item
  simplification à 5m
\item
  simplification à 50m
\item
  simplification à 100m
\end{itemize}

La topologie est conservée lors du processus de simplification (cf:
\url{http://openstreetmap.fr/blogs/cquest/limites-administratives-simplifiees}){]}(http://openstreetmap.fr/blogs/cquest/limites-administratives-simplifiees{]}(http://openstreetmap.fr/blogs/cquest/limites-administratives-simplifiees)))

\subsubsection{Contenu}\label{contenu}

Ces fichiers contiennent l'ensemble des EPCI figurant dans le fichier de
la DGCL (voir ``Origine des données'').

Pour chaque EPCI, les attributs suivants sont fournis:

\begin{itemize}

\item
  siren\_epci : code SIREN attribué par l'INSEE à l'EPCI (source Min.
  Intérieur)
\item
  nom\_epci : nom de l'EPCI (source Min. Intérieur)
\item
  ptot\_epci : population totale de l'EPCI (source Min. Intérieur)
\item
  nb\_comm : nombre de communes dans l'EPCI
\item
  surf\_km2 : superficie de l'EPCI en km2 sur le sphéroid WGS84
  (arrondie à l'hectare avant simplification)
\item
  short\_name : nom abbrégé (source OSM)
\item
  wikipedia: article wikipédia concernant l'EPCI (code langue + nom de
  l'article, exemple: ``fr:Communauté de communes du Larmont'')
\item
  web : site web de l'EPCI (source OSM)
\item
  osm\_id : ID de la relation OSM au moment de l'export
\item
  nom\_osm : nom de l'EPCI dans OSM
\item
  type\_epci: type d'EPCI (source OSM)
\end{itemize}

\subsubsection{Historique}\label{historique}

\begin{itemize}

\item
  20-12-2013 : première génération du fichier, basé sur le découpage
  communal OSM au 19-12-2013
\item
  12-02-2014 : seconde génération du fichier avec les EPCI 2014, et
  découpage communal OSM du 19-12-2013
\item
  13-02-2014 : ajout des champs web, wikipedia et osm\_id
\item
  06-03-2014 : troisième génération du fichier avec les EPCI 2014 et le
  découpage communal OSM au 06-03-2014
\item
  06-03-2014 : ajout du nom figurant dans OpenStreetMap et du type
  d'EPCI
\end{itemize}

Versions prédécentes disponibles sur:
\url{http://osm13.openstreetmap.fr/}cquest/openfla/export/{]}(http://osm13.openstreetmap.fr/\textasciitilde{}cquest/openfla/export/)

Pour toute question concernant ces exports, vous pouvez contacter
exports@openstreetmap.fr

Voir aussi :

\begin{itemize}

\item
  \href{https://www.data.gouv.fr/fr/dataset/contours-des-epci-2013}{Contours
  des EPCI 2013}
\item
  \href{https://www.data.gouv.fr/fr/dataset/decoupage-administratif-communal-francais-issu-d-openstreetmap}{Contours
  des communes françaises}
\item
  \href{https://www.data.gouv.fr/fr/dataset/contours-des-arrondissements-francais-issus-d-openstreetmap}{Contours
  des arrondissements français}
\item
  \href{https://www.data.gouv.fr/fr/dataset/contours-des-departements-francais-issus-d-openstreetmap}{Contours
  des départements français} et
  \href{https://www.data.gouv.fr/fr/dataset/cartes-des-regions-et-departements-issus-d-openstreetmap}{Cartes
  SVG des départements}
\item
  \href{https://www.data.gouv.fr/fr/dataset/contours-des-regions-francaises-sur-openstreetmap}{Contours
  de régions françaises}
\end{itemize}


\vspace{0.5cm}
\needspace{12\baselineskip}
\subsection*{Contours des EPCI 2015
}\index{communaute!de!communes}\index{decoupage!administratif}\index{epci}\index{epci!a!fiscalite!propre}\index{frontieres}\index{limites!administratives}\index{osm}
  \begin{wrapfigure}{r}{2.5cm}
    \centering
    \qrcode[nolink]{https://data.gouv.fr/dataset/54f63501c751df466f882844}
  \end{wrapfigure}

Licence : \textbf{Open Data Commons Open Database License (ODbL)
}\newline
Créé le : 2015-03-03\newline
Modifié le : 2016-03-16\newline
De 2015-01-01 à 2015-01-01\newline
Granularité : à l'EPCI\newline
Mise à jour : annuelle\newline
Popularité : 3 réutilisations,  2 suivis\newline
Mots-clé : \emph{communaute-de-communes, decoupage-administratif, epci, epci-a-fiscalite-propre, frontieres, limites-administratives, osm
}\newline
Permalien : \url{https://data.gouv.fr/dataset/54f63501c751df466f882844}\newline

\par
\noindent
    Contours géographiques des EPCI issu du croisement des limites
communales d'OpenStreetMap et des données de la Direction Générale des
Collectivités Locales datant de 2015.

Ces données sont en partie issues du crowdsourcing effectué par les
contributeurs au projet OpenStreetMap et sont donc sous licence ODbL qui
impose un partage à l'identique et la mention obligatoire d'attribution
doit être ``\textbf{© les contributeurs d'OpenStreetMap sous licence
ODbL}'' conformément à
\url{http://osm.org/copyright}{]}(http://osm.org/copyright{]}(http://osm.org/copyright))
\#\#Descriptif du contenu des fichiers ``epci''

\subsubsection{Origine des données}\label{origine-des-donnuxe9es}

Les données proviennent de la Direction Générale des Collectivités
Locales (DGCL) croisées avec le découpage communal issu de la base de
données cartographiques OpenStreetMap. Ces dernières ont été constituées
à partir du cadastre mis à disposition par la DGFiP sur
cadastre.gouv.fr.

Source pour les EPCI 2015:
\url{http://www.collectivites-locales.gouv.fr/liste-et-composition-2015}{]}(http://www.collectivites-locales.gouv.fr/liste-et-composition-2015{]}(http://www.collectivites-locales.gouv.fr/liste-et-composition-2015))

\subsubsection{Format}\label{format}

Ces fichiers sont proposés au format shapefile, en projection WGS84 avec
plusieurs niveaux de détails:

\begin{itemize}

\item
  simplification à 5m
\item
  simplification à 50m
\item
  simplification à 100m
\end{itemize}

La topologie est conservée lors du processus de simplification (cf:
\url{http://openstreetmap.fr/blogs/cquest/limites-administratives-simplifiees}){]}(http://openstreetmap.fr/blogs/cquest/limites-administratives-simplifiees{]}(http://openstreetmap.fr/blogs/cquest/limites-administratives-simplifiees)))

\subsubsection{Contenu}\label{contenu}

Ces fichiers contiennent l'ensemble des EPCI figurant dans le fichier de
la DGCL (voir ``Origine des données'').

Les attributs suivants sont fournis:

\begin{itemize}

\item
  siren\_epci : code SIREN attribué par l'INSEE à l'EPCI (source Min.
  Intérieur)
\item
  nom\_epci : nom de l'EPCI (source Min. Intérieur)
\item
  ptot\_epci : population totale de l'EPCI (source Min. Intérieur)
\item
  nb\_comm : nombre de communes dans l'EPCI
\item
  surf\_km2 : superficie de l'EPCI en km2 sur le sphéroid WGS84
  (arrondie à l'hectare avant simplification)
\item
  short\_name : nom abbrégé (source OSM)
\item
  wikipedia: article wikipédia concernant l'EPCI (code langue + nom de
  l'article, exemple: ``fr:Communauté de communes du Larmont'')
\item
  web : site web de l'EPCI (source OSM)
\item
  osm\_id : ID de la relation OSM au moment de l'export
\item
  nom\_osm : nom de l'EPCI dans OSM
\item
  type\_epci: type d'EPCI (source OSM)
\end{itemize}

\subsubsection{Historique}\label{historique}

\begin{itemize}

\item
  20-12-2013 : première génération du fichier, basé sur le découpage
  communal OSM au 19-12-2013
\item
  12-02-2014 : seconde génération du fichier avec les EPCI 2014, et
  découpage communal OSM du 19-12-2013
\item
  13-02-2014 : ajout des champs web, wikipedia et osm\_id
\item
  06-03-2014 : troisième génération du fichier avec les EPCI 2014 et le
  découpage communal OSM au 06-03-2014
\item
  06-03-2014 : ajout du nom figurant dans OpenStreetMap et du type
  d'EPCI
\item
  \textbf{03-03-2014 : quatrième génération du fichier avec les EPCI
  2015 et le découpage communal OSM au 01-01-2015}
\end{itemize}

Versions précédentes disponibles sur:
\url{http://osm13.openstreetmap.fr/}cquest/openfla/export/{]}(http://osm13.openstreetmap.fr/\textasciitilde{}cquest/openfla/export/)

Pour toute question concernant ces exports, vous pouvez contacter
exports@openstreetmap.fr

Voir aussi :

\begin{itemize}

\item
  \href{https://www.data.gouv.fr/fr/dataset/contours-des-epci-2014}{Contours
  des EPCI 2014}
\item
  \href{https://www.data.gouv.fr/fr/dataset/decoupage-administratif-communal-francais-issu-d-openstreetmap}{Contours
  des communes françaises}
\item
  \href{https://www.data.gouv.fr/fr/dataset/contours-des-arrondissements-francais-issus-d-openstreetmap}{Contours
  des arrondissements français}
\item
  \href{https://www.data.gouv.fr/fr/dataset/contours-des-departements-francais-issus-d-openstreetmap}{Contours
  des départements français} et
  \href{https://www.data.gouv.fr/fr/dataset/cartes-des-regions-et-departements-issus-d-openstreetmap}{Cartes
  SVG des départements}
\item
  \href{https://www.data.gouv.fr/fr/dataset/contours-des-regions-francaises-sur-openstreetmap}{Contours
  de régions françaises}
\item
  \href{https://www.data.gouv.fr/fr/datasets/projet-de-redecoupages-des-regions/}{Contours
  des futures régions}
\end{itemize}


\vspace{0.5cm}
\needspace{12\baselineskip}
\subsection*{Contours OSM des cantons électoraux départementaux 2015
}\index{canton}\index{cantonales}\index{cantons}\index{election}\index{elections}\index{elections!cantonales}\index{elections!departementales!2015}\index{limites!politiques}
  \begin{wrapfigure}{r}{2.5cm}
    \centering
    \qrcode[nolink]{https://data.gouv.fr/dataset/54f32631c751df46a0882844}
  \end{wrapfigure}

Licence : \textbf{Open Data Commons Open Database License (ODbL)
}\newline
Créé le : 2015-03-01\newline
Modifié le : 2017-06-30\newline
De 2015-03-01 à 2015-03-31\newline
Granularité : au canton\newline
Mise à jour : ponctuelle\newline
Popularité : 7 réutilisations,  5 suivis\newline
Mots-clé : \emph{canton, cantonales, cantons, election, elections, elections-cantonales, elections-departementales-2015, limites-politiques
}\newline
Permalien : \url{https://data.gouv.fr/dataset/54f32631c751df46a0882844}\newline

\par
\noindent
    Contours géographiques des nouveaux cantons électoraux départementaux
issu des limites communales d'OpenStreetMap et du découpage publié au
Journal Officiel.

Ces données sont issues du crowdsourcing effectué par les contributeurs
au projet OpenStreetMap et sont donc sous licence ODbL qui impose un
partage à l'identique et la mention obligatoire d'attribution doit être
``\textbf{© les contributeurs d'OpenStreetMap sous licence ODbL}''
conformément à
\url{http://osm.org/copyright}{]}(http://osm.org/copyright{]}(http://osm.org/copyright))Descriptif
du contenu des fichiers ``cantons 2015'' Origine des données

Les données proviennent du découpage communal issu de la base de données
cartographiques OpenStreetMap (constituées à partir du cadastre mis à
disposition par la DGFiP sur cadastre.gouv.fr) et de la lecture du
Journal Officiel. Les informations sur la population légale
\href{http://insee.fr/fr/themes/detail.asp?reg_id=0\&ref_id=cantons-2015}{proviennent
de l'INSEE}.

Pour plus de détail, consultez le wiki OpenStreetMap:
\url{http://wiki.openstreetmap.org/wiki/FR:Cantons_in_France}{]}(http://wiki.openstreetmap.org/wiki/FR:Cantons\_in\_France{]}(http://wiki.openstreetmap.org/wiki/FR:Cantons\_in\_France))
Format

Ces fichiers sont proposés au format shapefile (en WGS84, EPSG:4326,
textes en UTF-8), ainsi qu'en KML et GeoJSON avec leurs géométries
simplifiées. Contenu

Pour chaque canton, les attributs suivants sont fournis:

\begin{itemize}
\item
  code du canton, composé du code du département et du N\degree{} de
  canton (ex: 094-01, 02B-04)
\item
  nom du canton figurant au JORF (source OSM)
\item
  wikipedia: lien wikipedia vers l'article du canton (source OSM)
\item
  code INSEE du département (source INSEE)
\item
  N\degree{} du canton dans le département (source INSEE)
\item
  nom\_insee: Nom du canton dans le fichier INSEE (source INSEE)
\item
  pop\_legale: population légale dans le canton selon l'INSEE
\item
  N\degree{} de publication au JORF Historique
\item
  01-03-2015 : première génération du fichier pour la métropole
\item
  15-03-2015 : deuxième génération du fichier incluant les DOM
\end{itemize}

Versions prédécentes disponibles sur:
\url{http://osm13.openstreetmap.fr/}cquest/openfla/export/{]}(http://osm13.openstreetmap.fr/\textasciitilde{}cquest/openfla/export/)

Pour toute question concernant ces exports, vous pouvez contacter
exports@openstreetmap.fr

Voir aussi :

\begin{itemize}

\item
  \href{https://www.data.gouv.fr/fr/dataset/decoupage-administratif-communal-francais-issu-d-openstreetmap}{Contours
  des communes françaises}
\item
  \href{https://www.data.gouv.fr/fr/dataset/contours-des-arrondissements-francais-issus-d-openstreetmap}{Contours
  des arrondissements français}
\item
  \href{https://www.data.gouv.fr/fr/dataset/contours-des-departements-francais-issus-d-openstreetmap}{Contours
  des départements français} et
  \href{https://www.data.gouv.fr/fr/dataset/cartes-des-regions-et-departements-issus-d-openstreetmap}{Cartes
  SVG des départements}
\item
  \href{https://www.data.gouv.fr/fr/dataset/contours-des-regions-francaises-sur-openstreetmap}{Contours
  de régions françaises}
\item
  \href{https://www.data.gouv.fr/fr/datasets/projet-de-redecoupages-des-regions/}{Contours
  des futures régions}
\end{itemize}


\vspace{0.5cm}
\needspace{12\baselineskip}
\subsection*{Monuments aux morts présents en France dans les données OpenStreetMap
}\index{guerre}\index{monuments}
  \begin{wrapfigure}{r}{2.5cm}
    \centering
    \qrcode[nolink]{https://data.gouv.fr/dataset/5be5645e8b4c41645a007f47}
  \end{wrapfigure}

Licence : \textbf{Open Data Commons Open Database License (ODbL)
}\newline
Créé le : 2018-11-09\newline
Modifié le : 2018-11-09\newline
Granularité : au point d'intérêt\newline
Mise à jour : ponctuelle\newline
Popularité : 1 réutilisation,  0 suivi\newline
Mots-clé : \emph{guerre, monuments
}\newline
Permalien : \url{https://data.gouv.fr/dataset/5be5645e8b4c41645a007f47}\newline

\par
\noindent
    Ce jeu de données est un extrait de la base de données géographiques
libre et collaborative OpenStreetMap.

Il contient près de 5000 monuments aux morts cartographiés sur le
territoire français.

L'extraction a été faite avec l'outil Overpass Turbo, à l'aide de la
requête suivante:\url{https://overpass-turbo.eu/s/DwX}


\vspace{0.5cm}
\needspace{12\baselineskip}
\subsection*{Réseau électrique issu d'OpenStreetMap
}\index{crowdsourcing}\index{electrique}\index{openstreetmap}\index{reseau}
  \begin{wrapfigure}{r}{2.5cm}
    \centering
    \qrcode[nolink]{https://data.gouv.fr/dataset/53699f0fa3a729239d206036}
  \end{wrapfigure}

Licence : \textbf{Open Data Commons Open Database License (ODbL)
}\newline
Créé le : 2013-11-22\newline
Modifié le : 2018-09-05\newline
Granularité : au point d'intérêt\newline
Popularité : 2 réutilisations,  8 suivis\newline
Mots-clé : \emph{crowdsourcing, electrique, openstreetmap, reseau
}\newline
Permalien : \url{https://data.gouv.fr/dataset/53699f0fa3a729239d206036}\newline

\par
\noindent
    Requête d'extraction de l'ensemble du réseau électrique français présent
dans OpenStreetMap à l'aide de l'API de requêtage overpass

Ces données sont issues du crowdsourcing effectué par les contributeurs
au projet OpenStreetMap et sont sous licence ODbL et la mention
d'attribution doit être ``© les contributeurs d'OpenStreetMap sous
licence ODbL'' conformément à
\url{http://osm.org/copyright}{]}(http://osm.org/copyright{]}(http://osm.org/copyright))


\vspace{0.5cm}
\needspace{12\baselineskip}
\subsection*{Réseau Transilien / RER C
}\index{crowdsourcing}\index{deplacements!transports}\index{openstreetmap}\index{rer}\index{reseau!de!transport}\index{sncf}\index{transilien}
  \begin{wrapfigure}{r}{2.5cm}
    \centering
    \qrcode[nolink]{https://data.gouv.fr/dataset/53699f12a3a729239d206042}
  \end{wrapfigure}

Licence : \textbf{Open Data Commons Open Database License (ODbL)
}\newline
Créé le : 2013-11-16\newline
Modifié le : 2016-03-16\newline
Granularité : au point d'intérêt\newline
Popularité : 1 réutilisation,  1 suivi\newline
Mots-clé : \emph{crowdsourcing, deplacements-transports, openstreetmap, rer, reseau-de-transport, sncf, transilien
}\newline
Permalien : \url{https://data.gouv.fr/dataset/53699f12a3a729239d206042}\newline

\par
\noindent
    Description du réseau ferré de la ligne C du RER comprenant les points
d'arrêt et les voies ferrées empruntées.

Ces données sont issues du crowdsourcing effectué par les contributeurs
au projet OpenStreetMap et sont sous licence ODbL et la mention
d'attribution doit être ``© les contributeurs d'OpenStreetMap sous
licence ODbL'' conformément à
\url{http://osm.org/copyright}{]}(http://osm.org/copyright{]}(http://osm.org/copyright))
Les coordonnées géographiques sont en WGS84.


\vspace{0.5cm}
\needspace{12\baselineskip}
\subsection*{Stations et gares de métro, RER et Tramway de la région Ile de France
}\index{crowdsourcing}\index{deplacements!transports}\index{openstreetmap}\index{organisation!des!transports}\index{transports!en!commun}
  \begin{wrapfigure}{r}{2.5cm}
    \centering
    \qrcode[nolink]{https://data.gouv.fr/dataset/5369a042a3a729239d206311}
  \end{wrapfigure}

Licence : \textbf{Open Data Commons Open Database License (ODbL)
}\newline
Créé le : 2013-11-23\newline
Modifié le : 2016-03-16\newline
Granularité : au point d'intérêt\newline
Popularité : 2 réutilisations,  5 suivis\newline
Mots-clé : \emph{crowdsourcing, deplacements-transports, openstreetmap, organisation-des-transports, transports-en-commun
}\newline
Permalien : \url{https://data.gouv.fr/dataset/5369a042a3a729239d206311}\newline

\par
\noindent
    Documentation sur ces données disponibles sur
\url{http://wiki.openstreetmap.org/wiki/WikiProject_France/Transports_en_}le-de-France{]}(http://wiki.openstreetmap.org/wiki/WikiProject\_France/Transports\_en\_Île-de-France)

Ces données sont issues du crowdsourcing effectué par les contributeurs
au projet OpenStreetMap et sont sous licence ODbL et la mention
d'attribution doit être ``© les contributeurs d'OpenStreetMap sous
licence ODbL'' conformément à
\url{http://osm.org/copyright}{]}(http://osm.org/copyright{]}(http://osm.org/copyright))
Les requêtes API ont la forme:

{[}http://oapi-fr.openstreetmap.fr/oapi/interpreter?data={[}out:FORMAT{]};relation\href{http://oapi-fr.openstreetmap.fr/oapi/interpreter?data=\%5Bout:FORMAT\%5D;relation\%5B}{}type:RATP``\textasciitilde{}''metro\textbar{}rer\textbar{}tram``{]};out
META;\textgreater{};out META;\textgreater{};out META;

avec: - FORMAT = xml ou json - META = skel (seules les lignes autoroutes
sont décrites), body (description de tout les objets), meta (description
intégrale avec les metadonnées OSM comme le timestamp de dernière
modification de l'objet et l'auteur de cette modification)

Quelques exemples de requêtes ci-dessous :


\vspace{0.5cm}
\needspace{3\baselineskip} \rule{4cm}{0.25pt}\newline\textbf{Aussi disponible du même producteur :}\begin{itemize}
\item \href{https://data.gouv.fr/dataset/5a2687b488ee387aac2d85b4}{Liste des adjacences des communes françaises}
\item \href{https://data.gouv.fr/dataset/539a6f33a3a7293bc2728390}{Projet de redécoupages des régions}
\item \href{https://data.gouv.fr/dataset/53699fcea3a729239d2061f9}{Sentiers et points d'intérêt d'Abbaretz (Loire-Atlantique)}
\item \href{https://data.gouv.fr/dataset/53699fcea3a729239d2061fa}{Sentiers et points d'intérêt de la Garenne-Lemot (Loire-Atlantique)}
\item \href{https://data.gouv.fr/dataset/5797a382c751df2e36fa6070}{Zones Touristiques Internationales à Paris}
\end{itemize}

\clearpage
\section{oqp.io}


\begin{center}
  \includegraphics[width=3cm]{images/orga/11_b60971cd934abb8a0b56066341dd33-100.png}
\end{center}


Sur oqp.io, nous aidons les propriétaires de logements vacants à s'en
occuper et à les remettre sur le marché. Un par un, ensemble.


\vspace{0.5cm}

\needspace{12\baselineskip}
\subsection*{Ressources / logements vacants recensés sur oqp.io
}\index{crowdsourcing}\index{immobilier}\index{logement}\index{logement!politique!de!la!ville!i}\index{logement!vacant}\index{territoires}\index{vacance}\index{vacants}
  \begin{wrapfigure}{r}{2.5cm}
    \centering
    \qrcode[nolink]{https://data.gouv.fr/dataset/56c3603988ee38268c585a59}
  \end{wrapfigure}

Licence : \textbf{Open Data Commons Open Database License (ODbL)
}\newline
Créé le : 2016-02-16\newline
Modifié le : 2018-04-09\newline
Granularité : au point d'intérêt\newline
Popularité : 1 réutilisation,  0 suivi\newline
Mots-clé : \emph{crowdsourcing, immobilier, logement, logement-politique-de-la-ville-i, logement-vacant, territoires, vacance, vacants
}\newline
Permalien : \url{https://data.gouv.fr/dataset/56c3603988ee38268c585a59}\newline

\par
\noindent
    Pour \href{https://oqp.io}{oqp.io}, un logement vacant est avant tout
une ressource, une matière première pour créer du logement. Pas
forcément déjà un produit, un logement disponible ou prêt à trouver
preneur sur le marché immobilier, mais un bien inoccupé (ou bientôt
inoccupé) et potentiellement habitable. Un appartement vide, une maison
dont les habitants vont déménager, un corps de ferme qui tombe en
ruines, un étage d'immeuble comprenant des bureaux désaffectés ou encore
des combles aménageables dans un immeuble sont à ce titre des
ressources. Une résidence secondaire, même occupée très épisodiquement,
n'en est pas une.

Sur oqp.io, chacun peut signaler un logement vacant depuis un mobile ou
un ordinateur ou enrichir les informations déjà partagées. On peut
ensuite imaginer et détailler différents projets
\href{https://www.data.gouv.fr/fr/datasets/projets-de-reutilisations-de-logements-vacants-sur-oqp-io/}{(voir
le jeu de données des projets)}.

Ce jeu de données est publié à titre expérimental, avec
\href{https://github.com/oqpio/oqpio/issues/6}{quelques limitations déjà
connues}. Tout avis sur son format ou même sur la structure des données
est bienvenu!


\vspace{0.5cm}
\needspace{3\baselineskip} \rule{4cm}{0.25pt}\newline\textbf{Aussi disponible du même producteur :}\begin{itemize}
\item \href{https://data.gouv.fr/dataset/5767ff7b88ee384572ab6512}{Alertes - recherche logement vacant}
\end{itemize}

\clearpage
\section{Pas-de-Calais tourisme}
\needspace{12\baselineskip}
\subsection*{Loisirs (Français)
}
  \begin{wrapfigure}{r}{2.5cm}
    \centering
    \qrcode[nolink]{https://data.gouv.fr/dataset/53699983a3a729239d205286}
  \end{wrapfigure}

Licence : \textbf{Open Data Commons Open Database License (ODbL)
}\newline
Créé le : 2013-12-10\newline
Modifié le : 2015-11-02\newline
Popularité : 1 réutilisation,  0 suivi\newline
Mots-clé : \emph{aucun
}\newline
Permalien : \url{https://data.gouv.fr/dataset/53699983a3a729239d205286}\newline

\par
\noindent
    Loisirs dans le Pas-de-Calais.

Ce jeu de données est en Français.


\vspace{0.5cm}
\needspace{12\baselineskip}
\subsection*{Manifestations (Français)
}
  \begin{wrapfigure}{r}{2.5cm}
    \centering
    \qrcode[nolink]{https://data.gouv.fr/dataset/536999aaa3a729239d2052d9}
  \end{wrapfigure}

Licence : \textbf{Open Data Commons Open Database License (ODbL)
}\newline
Créé le : 2013-12-10\newline
Modifié le : 2015-02-11\newline
Popularité : 1 réutilisation,  0 suivi\newline
Mots-clé : \emph{aucun
}\newline
Permalien : \url{https://data.gouv.fr/dataset/536999aaa3a729239d2052d9}\newline

\par
\noindent
    Manifestations dans le Pas-de-Calais.

Ce jeu de données est en Français.


\vspace{0.5cm}
\needspace{12\baselineskip}
\subsection*{Patrimoine Culturel (Français)
}
  \begin{wrapfigure}{r}{2.5cm}
    \centering
    \qrcode[nolink]{https://data.gouv.fr/dataset/53699bdaa3a729239d205832}
  \end{wrapfigure}

Licence : \textbf{Open Data Commons Open Database License (ODbL)
}\newline
Créé le : 2013-12-10\newline
Modifié le : 2015-11-25\newline
Popularité : 1 réutilisation,  1 suivi\newline
Mots-clé : \emph{aucun
}\newline
Permalien : \url{https://data.gouv.fr/dataset/53699bdaa3a729239d205832}\newline

\par
\noindent
    Patrimoine culturel dans le Pas-de-Calais.

Ce jeu de données est en Français.


\vspace{0.5cm}
\needspace{3\baselineskip} \rule{4cm}{0.25pt}\newline\textbf{Aussi disponible du même producteur :}\begin{itemize}
\item \href{https://data.gouv.fr/dataset/536995efa3a729239d20486c}{Gastronomie (Anglais)}
\item \href{https://data.gouv.fr/dataset/536995efa3a729239d20486e}{Gastronomie (Néerlandais)}
\item \href{https://data.gouv.fr/dataset/5369960fa3a729239d2048c8}{Hébergements (Anglais)}
\item \href{https://data.gouv.fr/dataset/53699610a3a729239d2048ca}{Hébergements (Français)}
\item \href{https://data.gouv.fr/dataset/53699611a3a729239d2048cd}{Hébergements (Néerlandais)}
\item \href{https://data.gouv.fr/dataset/53699781a3a729239d204cec}{Jeux de données - Pas-de-Calais tourisme}
\item \href{https://data.gouv.fr/dataset/53699982a3a729239d205284}{Loisirs (Anglais)}
\item \href{https://data.gouv.fr/dataset/53699984a3a729239d205288}{Loisirs (Néerlandais)}
\item \href{https://data.gouv.fr/dataset/536999aaa3a729239d2052d8}{Manifestations (Anglais)}
\item \href{https://data.gouv.fr/dataset/536999aba3a729239d2052db}{Manifestations (Néerlandais)}
\item \href{https://data.gouv.fr/dataset/53699b35a3a729239d2056a0}{Organismes Touristiques (Anglais)}
\item \href{https://data.gouv.fr/dataset/53699b35a3a729239d2056a1}{Organismes Touristiques (Français)}
\item \href{https://data.gouv.fr/dataset/53699b36a3a729239d2056a2}{Organismes Touristiques (Néerlandais)}
\item \href{https://data.gouv.fr/dataset/53699bd9a3a729239d205830}{Patrimoine Culturel (Anglais)}
\item \href{https://data.gouv.fr/dataset/53699bdaa3a729239d205834}{Patrimoine Culturel (Néerlandais)}
\item \href{https://data.gouv.fr/dataset/53699cc7a3a729239d205a69}{Pleine Nature (Anglais)}
\item \href{https://data.gouv.fr/dataset/53699cc8a3a729239d205a6a}{Pleine Nature (Français)}
\item \href{https://data.gouv.fr/dataset/53699cc8a3a729239d205a6b}{Pleine Nature (Néerlandais)}
\item \href{https://data.gouv.fr/dataset/53699f27a3a729239d206075}{Restauration (Anglais)}
\item \href{https://data.gouv.fr/dataset/53699f27a3a729239d206077}{Restauration (Français)}
\item \href{https://data.gouv.fr/dataset/53699f28a3a729239d206078}{Restauration (Néerlandais)}
\item \href{https://data.gouv.fr/dataset/5369a032a3a729239d2062e9}{Sport (Anglais)}
\item \href{https://data.gouv.fr/dataset/5369a032a3a729239d2062ea}{Sport (Français)}
\item \href{https://data.gouv.fr/dataset/5369a033a3a729239d2062eb}{Sport (Néerlandais)}
\item \href{https://data.gouv.fr/dataset/5369a035a3a729239d2062ef}{Stages et Ateliers (Anglais)}
\item \href{https://data.gouv.fr/dataset/5369a035a3a729239d2062f0}{Stages et Ateliers (Français)}
\item \href{https://data.gouv.fr/dataset/5369a035a3a729239d2062f1}{Stages et Ateliers (Néerlandais)}
\end{itemize}

\clearpage
\section{Polytech'Tours}


\begin{center}
  \includegraphics[width=3cm]{images/orga/2015-01-12_d2616f9579624de99bf0458b0d0bb813_PolytechDAE-100.jpg}
\end{center}


Département Aménagement et Environnement


\vspace{0.5cm}

\needspace{12\baselineskip}
\subsection*{Remembrement et aménagement foncier
}
  \begin{wrapfigure}{r}{2.5cm}
    \centering
    \qrcode[nolink]{https://data.gouv.fr/dataset/54ad3e16c751df373bde6534}
  \end{wrapfigure}

Licence : \textbf{Licence Ouverte
}\newline
Créé le : 2015-01-07\newline
Modifié le : 2016-02-10\newline
De 1945-01-01 à 2005-12-31\newline
Granularité : à la commune\newline
Popularité : 1 réutilisation,  1 suivi\newline
Mots-clé : \emph{aucun
}\newline
Permalien : \url{https://data.gouv.fr/dataset/54ad3e16c751df373bde6534}\newline

\par
\noindent
    60 ans d'opérations d'aménagement foncier en France


\vspace{0.5cm}

\clearpage
\section{Projet de recherche ECCE Carto}


\begin{center}
  \includegraphics[width=3cm]{images/orga/9a_6ef7641f9845e8b675e5364c098d20-100.jpg}
\end{center}


\textbf{ECCE Carto (des Espaces de la Contribution à la Contribution sur
l'Espace)} est un projet de recherche mené au sein du laboratoire
PASSAGES (UMR 5319) qui associe enseignants-chercheurs, chercheurs du
CNRS, doctorants et étudiants de l'Université Bordeaux Montaigne.

Son objectif est de porter un regard géographique sur les nouveaux modes
de fabrique cartographique en s'intéressant, en particulier, au projet
de cartographie collaborative OpenStreetMap (OSM).

Le projet ECCE Carto vise à donner corps aux contributions présentes sur
OSM en mesurant, visualisant, interrogeant les relations entre pratiques
territoriales et pratiques de production de l'information géographique
numérique. Ce décryptage permet d'appréhender d'une manière originale
(par les pratiques et à l'échelle de l'individu) les glissements de
légitimité actuellement en cours dans les manières de « dire et d'écrire
» le territoire : des acteurs en position historique d'autorité vers les
citoyens « amateurs », capteurs et/ou contributeurs.


\vspace{0.5cm}

\needspace{12\baselineskip}
\subsection*{Localisation des contributeurs OSM
}\index{openstreetmap}
  \begin{wrapfigure}{r}{2.5cm}
    \centering
    \qrcode[nolink]{https://data.gouv.fr/dataset/57e3ae12c751df1e3d79df72}
  \end{wrapfigure}

Licence : \textbf{Open Data Commons Open Database License (ODbL)
}\newline
Créé le : 2016-09-22\newline
Modifié le : 2016-09-22\newline
Granularité : à la commune\newline
Popularité : 1 réutilisation,  0 suivi\newline
Mots-clé : \emph{openstreetmap
}\newline
Permalien : \url{https://data.gouv.fr/dataset/57e3ae12c751df1e3d79df72}\newline

\par
\noindent
    Localisation des contributeurs OpenStreetMap issue de l'enquête en ligne
ECCE Carto, menée entre décembre 2015 et janvier 2016, qui a permis de
recueillir 298 réponses auprès de contributeurs français. Les données
sont sous forme de points, avec un fichier de localisations par communes
pour la France et un fichier de localisation par pays pour les autres
pays. Pour chaque point est renseigné le nombre de contributeurs qui y
sont rattachés.


\vspace{0.5cm}
\needspace{3\baselineskip} \rule{4cm}{0.25pt}\newline\textbf{Aussi disponible du même producteur :}\begin{itemize}
\item \href{https://data.gouv.fr/dataset/57e3afa8c751df1e4079df73}{Ages et sexe des contributeurs OSM}
\item \href{https://data.gouv.fr/dataset/57e3a63cc751df1e3f79df72}{Résultats de l'enquête ECCE Carto 2015 auprès de contributeurs OSM}
\end{itemize}

\clearpage
\section{QUORUM }


\begin{center}
  \includegraphics[width=3cm]{images/orga/99_3caea0485c4f38889e8c05acb3b694-100.png}
\end{center}


Quorum est le premier logiciel de mobilisation en Europe. ONG, campagnes
électorales, concertation, mouvement Quorum pousse les frontières de la
data sciences et des nouvelles technologies pour construire les
campagnes de terrain et les mobilisations de demain.


\vspace{0.5cm}

\needspace{12\baselineskip}
\subsection*{Contours précis des circonscriptions législatives
}\index{cantons}\index{carte}\index{circonscription}\index{circonscriptions}\index{circonscriptions!legislatives}\index{election}\index{elections}\index{elections!legislatives}\index{geojson}\index{legislatives}\index{opendata}\index{openstreetmap}\index{politique}
  \begin{wrapfigure}{r}{2.5cm}
    \centering
    \qrcode[nolink]{https://data.gouv.fr/dataset/5914930088ee384fd1b28666}
  \end{wrapfigure}

Licence : \textbf{Open Data Commons Open Database License (ODbL)
}\newline
Créé le : 2017-05-11\newline
Modifié le : 2017-05-11\newline
Popularité : 4 réutilisations,  0 suivi\newline
Mots-clé : \emph{cantons, carte, circonscription, circonscriptions, circonscriptions-legislatives, election, elections, elections-legislatives, geojson, legislatives, opendata, openstreetmap, politique
}\newline
Permalien : \url{https://data.gouv.fr/dataset/5914930088ee384fd1b28666}\newline

\par
\noindent
    \textbf{La carte des législatives : découpage par circonscription}

Contours détaillés des circonscriptions législatives (définition 2012)
construit depuis les données existantes dans OpenStreetMap avec un
niveau de détail approchant l'adresse. Les données manquantes pour 25
centres-villes sont complétés par le découpage proposé par le site
Toxicode.

Pour les élections législatives, le découpage est réalisé \textbf{par
circonscriptions} qui sont elles-mêmes un regroupement de cantons.

Or, pour les circonscriptions législatives l'ancien découpage des
cantons est toujours utilisé. Pour créer des données précises et
détaillées du découpage des circonscriptions législatives des
contributeurs d'OpenStreetMap France ont donc menés de nouveaux travaux.
Pour ce faire, ils ont procédé comme suit :

\begin{itemize}

\item
  trouver les descriptions de l'ancien découpage des cantons au sein de
  diverses publications du Journal Officiel,
\item
  interpréter correctement ces descriptions,
\item
  regrouper l'ensemble des descriptions trouvées, intégrer ces
  descriptions à OpenStreetMap afin de créer les données relatives au
  découpage des circonscriptions.
\end{itemize}

À partir de ces travaux réalisés par les contributeurs d'OpenStreetMap
nous avons pu créer le jeu de données présentant les contours des
circonscriptions législatives à** un niveau de détail très fin,
s'approchant de l'adresse.**

Puisqu'il nous manquait des données pour une vingtaine de centre-ville
nous avons complété ce découpage grâce aux travaux de Toxicode.
Finalement, nous avons pu déterminer avec exactitude la répartition des
panneaux d'affichage électoraux entre chaque circonscription.

La production de ces données présentant un niveau de détail très fin, à
partir des contributions de citoyens engagés dans l'Open Data et le
Libre, témoigne une nouvelle fois de l'intelligence que peut produire
l'OpenData ! Aussi, nous utilisons ce découpage au sein du projet
panneaux-election.fr pour les législatives 2017 avec mapotempo.


\vspace{0.5cm}

\clearpage
\section{Rennes Métropole en accès libre}


\begin{center}
  \includegraphics[width=3cm]{images/orga/9d_5043e604c94ed58a8a1501eb69e5b4-100.jpg}
\end{center}


Le programme open data de Rennes Métropole qui publie les données de la
Ville de Rennes, de Rennes Métropole et de ses partenaires.
\url{http://www.data.rennes-metropole.fr}


\vspace{0.5cm}

\needspace{12\baselineskip}
\subsection*{Jeux de données - Association Trans Musicales
}\index{liste!de!jeux!de!donnees}
  \begin{wrapfigure}{r}{2.5cm}
    \centering
    \qrcode[nolink]{https://data.gouv.fr/dataset/53699766a3a729239d204ca5}
  \end{wrapfigure}

Licence : \textbf{Licence Ouverte
}\newline
Créé le : 2013-11-13\newline
Modifié le : 2015-12-13\newline
Popularité : 1 réutilisation,  0 suivi\newline
Mots-clé : \emph{liste-de-jeux-de-donnees
}\newline
Permalien : \url{https://data.gouv.fr/dataset/53699766a3a729239d204ca5}\newline

\par
\noindent
    Les jeux de données fournis par Association Trans Musicales pour
data.gouv.fr.


\vspace{0.5cm}
\needspace{3\baselineskip} \rule{4cm}{0.25pt}\newline\textbf{Aussi disponible du même producteur :}\begin{itemize}
\item \href{https://data.gouv.fr/dataset/586db27aa3a7290df6f4bea9}{Accidents corporels de la circulation 2015 - Rennes Métropole}
\item \href{https://data.gouv.fr/dataset/580a5746a3a7292dcfa9d1d6}{Actes d'état civil à Rennes}
\item \href{https://data.gouv.fr/dataset/5a275e43a3a7293578c3f7c1}{Adresses du référentiel voies et adresses au format BAL}
\item \href{https://data.gouv.fr/dataset/580a577ba3a7292dcfa9d1fa}{Adresses du référentiel voies et adresses de Rennes Métropole}
\item \href{https://data.gouv.fr/dataset/580a5771a3a7292dcfa9d1f5}{Agences commerciales du réseau STAR}
\item \href{https://data.gouv.fr/dataset/5b208a36b595082912f1ace8}{Aire de camping car - Cesson-Sévigné}
\item \href{https://data.gouv.fr/dataset/5b208a3cb595082912f1ace9}{Aires de jeux - Cesson-Sévigné}
\item \href{https://data.gouv.fr/dataset/580a5735a3a7292dcea9d1d3}{Alertes trafic en temps réel sur les lignes du réseau STAR}
\item \href{https://data.gouv.fr/dataset/580a5786a3a7292dcea9d20b}{Aménagements vélo et zones de circulation apaisée sur Rennes Métropole}
\item \href{https://data.gouv.fr/dataset/586db2c8a3a7290df5f4be9e}{Ancienneté d'emménagement par age}
\item \href{https://data.gouv.fr/dataset/589e7ecba3a72974c1f1125b}{Arbres d'ornement des espaces verts de la Ville de Rennes}
\item \href{https://data.gouv.fr/dataset/580a574fa3a7292dcea9d1e8}{Arbres du jardin du Thabor}
\item \href{https://data.gouv.fr/dataset/580a5737a3a7292dcfa9d1cb}{Artistes et concerts aux Transmusicales}
\item \href{https://data.gouv.fr/dataset/59b0b6d0b59508797d111c83}{Artistes et concerts aux Transmusicales}
\item \href{https://data.gouv.fr/dataset/5b2c6827a3a7292d14b11d6f}{Base élus de Rennes Ville et Métropole 2018}
\item \href{https://data.gouv.fr/dataset/580a5741a3a7292dcea9d1dd}{Base élus Rennes Ville et Métropole 2017}
\item \href{https://data.gouv.fr/dataset/586db267a3a7290df5f4be91}{Base organismes et équipements "Vivre à Rennes"}
\item \href{https://data.gouv.fr/dataset/5a5d6c3db5950807eeecba32}{Base SIRENE - Cesson-Sévigné}
\item \href{https://data.gouv.fr/dataset/5a5ebd65b5950876ccecba32}{Base SIRENE - Noyal-Châtillon-sur-Seiche}
\item \href{https://data.gouv.fr/dataset/5a73d543b5950827d4a2256b}{Base SIRENE - Saint-Jacques de la Lande}
\item \href{https://data.gouv.fr/dataset/5c89c8cd9ce2e72817e27c1e}{Base Sirene v3 ß - Rennes Métropole}
\item \href{https://data.gouv.fr/dataset/580a5736a3a7292dcfa9d1ca}{Bornes de recharge dédiées aux véhicules électriques sur le territoire de Rennes Métropole}
\item \href{https://data.gouv.fr/dataset/580a572ba3a7292dcfa9d1c2}{BP 2012 - Ville de Rennes - Budget Principal}
\item \href{https://data.gouv.fr/dataset/580a5766a3a7292dcea9d1f9}{BP 2012 - Ville de Rennes - Budgets Annexes}
\item \href{https://data.gouv.fr/dataset/580a572ea3a7292dcea9d1cd}{BP 2012 - Ville de Rennes - Subventions d'équipement aux associations}
\item \href{https://data.gouv.fr/dataset/580a572ea3a7292dcfa9d1c4}{BP 2012 - Ville de Rennes - Subventions exceptionnelles aux associations}
\item \href{https://data.gouv.fr/dataset/580a5766a3a7292dcfa9d1ee}{BP 2012 - Ville de Rennes - Subventions ordinaires aux associations}
\item \href{https://data.gouv.fr/dataset/580a5795a3a7292dcfa9d20b}{BP 2013 - BUDGET PRINCIPAL}
\item \href{https://data.gouv.fr/dataset/580a5763a3a7292dcea9d1f7}{BP 2013 - Ville de Rennes - Budget Annexes}
\item \href{https://data.gouv.fr/dataset/580a5729a3a7292dcfa9d1c0}{BP 2013 - Ville de Rennes - Subventions d'équipement aux associations}
\item \href{https://data.gouv.fr/dataset/580a578ba3a7292dcea9d20e}{BP 2013 - Ville de Rennes - Subventions exceptionnelles aux associations}
\item \href{https://data.gouv.fr/dataset/580a5763a3a7292dcfa9d1ec}{BP 2013 - Ville de Rennes - Subventions ordinaires aux associations}
\item \href{https://data.gouv.fr/dataset/580a575fa3a7292dcea9d1f4}{BP 2014 - Ville de Rennes - Budget Annexes}
\item \href{https://data.gouv.fr/dataset/580a5726a3a7292dcea9d1c6}{BP 2014 - Ville de Rennes - Budget Principal par article}
\item \href{https://data.gouv.fr/dataset/580a5725a3a7292dcfa9d1bc}{BP 2014 - Ville de Rennes - Budget Principal par sous fonction}
\item \href{https://data.gouv.fr/dataset/580a5727a3a7292dcea9d1c7}{BP 2014 - Ville de Rennes - Subventions d'équipement aux associations}
\item \href{https://data.gouv.fr/dataset/580a5726a3a7292dcfa9d1bd}{BP 2014 - Ville de Rennes - Subventions exceptionnelles aux associations}
\item \href{https://data.gouv.fr/dataset/580a575ea3a7292dcfa9d1e9}{BP 2014 - Ville de Rennes - Subventions ordinaires aux associations}
\item \href{https://data.gouv.fr/dataset/580a5791a3a7292dcea9d212}{BP 2015 - Ville de Rennes - Budget Principal par article}
\item \href{https://data.gouv.fr/dataset/580a5722a3a7292dcfa9d1ba}{BP 2015 - Ville de Rennes - Budget Principal par sous fonctions}
\item \href{https://data.gouv.fr/dataset/580a5752a3a7292dcea9d1ea}{BP 2015 - Ville de Rennes - Budgets Annexes}
\item \href{https://data.gouv.fr/dataset/580a575aa3a7292dcea9d1f0}{BP 2015 - Ville de Rennes - Subventions d'équipement aux associations}
\item \href{https://data.gouv.fr/dataset/580a5759a3a7292dcfa9d1e5}{BP 2015 - Ville de Rennes - Subventions exceptionnelles aux associations}
\item \href{https://data.gouv.fr/dataset/580a5723a3a7292dcea9d1c4}{BP 2015 - Ville de Rennes - Subventions ordinaires aux associations}
\item \href{https://data.gouv.fr/dataset/580a5791a3a7292dcfa9d209}{BP 2016 - Rennes Métropole}
\item \href{https://data.gouv.fr/dataset/580a575ba3a7292dcea9d1f1}{BP 2016 - Ville de Rennes - Budget Principal}
\item \href{https://data.gouv.fr/dataset/580a575ba3a7292dcfa9d1e6}{BP 2016 - Ville De Rennes - Budgets Annexes}
\item \href{https://data.gouv.fr/dataset/580a5749a3a7292dcfa9d1d9}{BP 2016 - Ville de Rennes - Subventions aux associations}
\item \href{https://data.gouv.fr/dataset/586f0854a3a7291134880609}{BP 2017 - Rennes Métropole}
\item \href{https://data.gouv.fr/dataset/58be2261a3a7293affefbd25}{BP 2017 - Ville de Rennes - Budget Principal}
\item et 216 autres jeux de données\end{itemize}

\clearpage
\section{Sciences Po}


\begin{center}
  \includegraphics[width=3cm]{images/orga/2015-01-29_9db46d8621cc476c9485c256e4a5846a_SCIENCES_PO_logo_RGB-100.jpg}
\end{center}


Fondée en 1872, Sciences Po est une université de recherche de rang
international spécialisée en sciences sociales qui délivre des
formations de niveau Bac+3, Master, Doctorat et en formation continue.

Modèle unique et innovant, elle propose une formation intellectuelle
pluridisciplinaire, nourrie par une recherche de pointe et ancrée dans
le monde professionnel. Son ambition est de former de futurs
professionnels capables de comprendre le monde et de faire bouger les
lignes.

A la fois sélective et ouverte à tous les talents, Sciences Po compte
sur ses sept campus en France 13 000 étudiants, dont 46\% sont d'une
nationalité autre que française.

Lieu de rencontre entre la pensée et l'action, Sciences Po propose, aux
côtés de ses enseignements académiques, une mise en pratique des
savoirs, grâce notamment à sa communauté d'enseignants issus du monde
professionnel.

Acteur majeur de la recherche en sciences humaines et sociales, Sciences
Po rassemble 220 chercheurs investis dans des unités à la pointe de
l'analyse de nos sociétés.

L'une d'entre elles, le Centre de données socio-politiques (CDSP, UMS
Sciences Po \& CNRS) a pour mission l'archivage, la documentation et la
diffusion, auprès de la communauté scientifique, de données qualitatives
et quantitatives, notamment celles issues de la recherche.


\vspace{0.5cm}

\needspace{12\baselineskip}
\subsection*{Carte des circonscriptions législatives 2012 et 2017
}\index{circonscriptions!legislatives}\index{fond!de!carte}\index{geojson}
  \begin{wrapfigure}{r}{2.5cm}
    \centering
    \qrcode[nolink]{https://data.gouv.fr/dataset/58f87fdcc751df03a7be51e4}
  \end{wrapfigure}

Licence : \textbf{Open Data Commons Open Database License (ODbL)
}\newline
Créé le : 2017-04-20\newline
Modifié le : 2017-07-21\newline
Popularité : 7 réutilisations,  1 suivi\newline
Mots-clé : \emph{circonscriptions-legislatives, fond-de-carte, geojson
}\newline
Permalien : \url{https://data.gouv.fr/dataset/58f87fdcc751df03a7be51e4}\newline

\par
\noindent
    Ce fond de carte au format shapefile et geojson est une reprise du
travail de \href{http://www.toxicode.fr/circonscriptions}{Toxicode}.
L'Atelier de cartographie de Sciences Po à ensuite vérifié, nettoyé et
généralisé le fond.

\textbf{Couverture} : France métropolitaine, départements d'outre-mer
(DOM) et collectivités d'outre-mer (COM)

\textbf{Format des données}

\begin{itemize}

\item
  ID : code du département et numéro de la circonscription, ex : 69002
\item
  num\_circ : numéro de circonscription, ex : 2
\item
  code\_dpt : code du département, ex : 69
\item
  nom\_dpt : nom du département
\item
  code\_reg : code du département
\item
  nom\_reg : nom de la région
\end{itemize}

\textbf{Historique}\\
20/04/2017 - première version avec France métropolitaine + département
d'outre-mer (DOM)\\
15/05/2017 - corrections multiples d'attributs et de géométries (merci à
F. Rodrigo et Th. Gratier) + ajout des collectivités d'outre-mer (COM)
après nettoyage et simplification du
\href{https://www.data.gouv.fr/fr/datasets/contours-detailles-des-circonscriptions-des-legislatives/}{fond
de mapotempo}\\
21/07/2017 - corrections des tracés (géométries invalides, superposition
de points\ldots{}) grâce à la contribution d'Arthur Cheysson (merci!)


\vspace{0.5cm}
\needspace{12\baselineskip}
\subsection*{Élections cantonales 1988-2011
}\index{cantonales}\index{cantons}\index{elections}\index{elections!cantonales}
  \begin{wrapfigure}{r}{2.5cm}
    \centering
    \qrcode[nolink]{https://data.gouv.fr/dataset/54774905c751df0eb372f5d4}
  \end{wrapfigure}

Licence : \textbf{Open Data Commons Open Database License (ODbL)
}\newline
Créé le : 2014-11-27\newline
Modifié le : 2016-03-08\newline
De 1988-01-01 à 2011-12-31\newline
Granularité : au canton\newline
Mise à jour : ponctuelle\newline
Popularité : 2 réutilisations,  0 suivi\newline
Mots-clé : \emph{cantonales, cantons, elections, elections-cantonales
}\newline
Permalien : \url{https://data.gouv.fr/dataset/54774905c751df0eb372f5d4}\newline

\par
\noindent
    Les résultats des élections cantonales depuis 1988 sont disponibles au
niveau des cantons.

Jusqu'aux élections de 1998, les résultats ont été obtenus auprès de la
Banque de données sociopolitiques de Grenoble, dont le Centre de données
sociopolitiques \href{http://cdsp.sciences-po.fr}{CDSP} de Sciences Po a
pris la suite en 2005. Ils sont agrégés par tendances politiques avec
plus ou moins de détails selon les années d'élection. À partir de 2001,
les résultats proviennent du Ministère de l'intérieur et sont détaillés
par candidats.

Le CDSP a procédé à des vérifications sur les fichiers, essentiellement
par des tests de cohérence (par exemple, comparaison du nombre
d'exprimés et de la somme des voix pour chaque parti/candidat), mais des
erreurs peuvent subsister.

La citation des fichiers est disponible dans la description de chaque
ressource. Dans le cas d'utilisation de l'ensemble des fichiers, il
convient d'utiliser les modèles suivants.

\textbf{Citation lors de l'utilisation de l'ensemble des fichiers}:
Résultats des élections cantonales 1988-2011 par canton,
CANTONALES1988-2011.zip {[}fichier informatique{]}, Grenoble : Banque de
Données Socio-Politiques, Paris : Ministère de l'Intérieur, France
{[}producteurs{]}, Centre de Données Socio-Politiques {[}diffuseur{]},
septembre 2009.

La citation de chaque fichier est indiquée dans la description de la
ressource. L'inventaire des données d'élections de la Ve République
renseigne le fonds des résultats électoraux du CDSP, selon plusieurs
niveaux territoriaux.


\vspace{0.5cm}
\needspace{12\baselineskip}
\subsection*{Elections présidentielles 1965-2012
}\index{circonscriptions}\index{elections}\index{elections!presidentielles}\index{presidentielles}
  \begin{wrapfigure}{r}{2.5cm}
    \centering
    \qrcode[nolink]{https://data.gouv.fr/dataset/54aeb906c751df6643de6534}
  \end{wrapfigure}

Licence : \textbf{Open Data Commons Open Database License (ODbL)
}\newline
Créé le : 2015-01-08\newline
Modifié le : 2016-02-13\newline
De 1965-01-01 à 2012-12-31\newline
Mise à jour : ponctuelle\newline
Popularité : 4 réutilisations,  1 suivi\newline
Mots-clé : \emph{circonscriptions, elections, elections-presidentielles, presidentielles
}\newline
Permalien : \url{https://data.gouv.fr/dataset/54aeb906c751df6643de6534}\newline

\par
\noindent
    Les résultats des élections présidentielles sont disponibles par
circonscription de 1965 à 2012. Les premiers tours des élections de 1981
à 2002 et le second tour de l'élection de 1988 sont disponibles par
commune pour les communes de plus de 9000 habitants. A partir de 2002,
les résultats pour les DOM-TOM figurent dans les fichiers.

Jusqu'à l'élection de 2002, les résultats ont été obtenus auprès de la
Banque de données sociopolitiques (BDSP) de Grenoble dont le Centre de
données sociopolitiques (\href{http://cdsp.sciences-po.fr/}{CDSP}) de
Sciences Po a pris la suite en 2005. À partir de 2007, les résultats
proviennent du Ministère de l'intérieur.

Le CDSP a procédé à des vérifications sur les fichiers, essentiellement
par des tests de cohérence (par exemple, comparaison du nombre
d'exprimés et de la somme des voix pour chaque parti/candidat), mais des
erreurs peuvent subsister.

\textbf{Citation lors de l'utilisation de l'ensemble des fichiers:}
Résultats des élections présidentielles 1965-2012,
PRESIDENTIELLES1965-2012.zip {[}fichier informatique{]}, Grenoble :
Banque de Données Socio-Politiques, Paris : Ministère de l'Intérieur,
France {[}producteurs{]}, Centre de Données Socio-Politiques
{[}diffuseur{]}, 2014.

La citation de chaque fichier est indiquée dans la description de la
ressource.

L'inventaire des données d'élections de la Ve République renseigne le
fonds des résultats électoraux du CDSP, selon plusieurs niveaux
territoriaux.


\vspace{0.5cm}
\needspace{12\baselineskip}
\subsection*{Elections régionales 1986-2010
}\index{circonscriptions}\index{communes}\index{departements}\index{elections}\index{regionales}\index{regions}
  \begin{wrapfigure}{r}{2.5cm}
    \centering
    \qrcode[nolink]{https://data.gouv.fr/dataset/54aebd5fc751df69b7de6536}
  \end{wrapfigure}

Licence : \textbf{Open Data Commons Open Database License (ODbL)
}\newline
Créé le : 2015-01-08\newline
Modifié le : 2016-02-16\newline
De 1986-01-01 à 2010-12-31\newline
Granularité : au département\newline
Mise à jour : ponctuelle\newline
Popularité : 1 réutilisation,  0 suivi\newline
Mots-clé : \emph{circonscriptions, communes, departements, elections, regionales, regions
}\newline
Permalien : \url{https://data.gouv.fr/dataset/54aebd5fc751df69b7de6536}\newline

\par
\noindent
    Les résultats des élections régionales de 1998 à 2010 sont disponibles
par circonscription législative et ceux des élections de 1986 et de 1992
par département.

Jusqu'aux élections de 1992, les résultats ont été obtenus auprès de la
Banque de données sociopolitiques (BDSP) de Grenoble dont le Centre de
données sociopolitiques (\href{http://cdsp.sciences-po.fr/}{CDSP}) de
Sciences Po a pris la suite en 2005. Les élections suivantes proviennent
du Ministère de l'intérieur.

Pour les élections de 1986 et de 1992, le mode de scrutin est
proportionnel (à un tour) et l'offre politique était proposée au niveau
département. Pour la Corse, l'élection de 1992 s'est déroulée en 2
tours, or seuls les résultats du 1e tour sont disponibles dans la
ressource existante. Pour ces deux élections, les données sont agrégées
par tendances politiques. Lors des élections suivantes, les listes de
candidats étaient présentées au niveau régional (avec des sections
départementales).

Certaines élections sont également disponibles au niveau des communes :

\begin{itemize}

\item
  pour 1986, 1992 et 1998, seuls les résultats des communes de plus de
  9000 habitants;
\item
  pour 2004, uniquement les communes de plus de 3500 habitants au 1er
  tour, et l'ensemble des communes pour le 2nd tour.
\end{itemize}

Le CDSP a procédé à des vérifications sur les fichiers, essentiellement
par des tests de cohérence (par exemple, comparaison du nombre
d'exprimés et de la somme des voix pour chaque parti/candidat), mais des
erreurs peuvent subsister.

\textbf{Citation lors de l'utilisation de l'ensemble des données:}
Résultats des élections régionales 1986-2010, REGIONALES1986-2010.zip
{[}fichier informatique{]}, Grenoble : Banque de Données
Socio-Politiques, Paris : Ministère de l'Intérieur, France
{[}producteurs{]}, Centre de Données Socio-Politiques {[}diffuseur{]},
2010.

La citation de chaque fichier est indiquée dans la description de la
ressource.

L'inventaire des données d'élections de la Ve République renseigne le
fonds des résultats électoraux du CDSP, selon plusieurs niveaux
territoriaux.


\vspace{0.5cm}

\clearpage
\section{Semitan}
\needspace{12\baselineskip}
\subsection*{API temps réel de la TAN
}
  \begin{wrapfigure}{r}{2.5cm}
    \centering
    \qrcode[nolink]{https://data.gouv.fr/dataset/53698ee6a3a729239d2035cc}
  \end{wrapfigure}

Licence : \textbf{Open Data Commons Open Database License (ODbL)
}\newline
Créé le : 2013-11-16\newline
Modifié le : 2016-03-16\newline
Popularité : 1 réutilisation,  1 suivi\newline
Mots-clé : \emph{aucun
}\newline
Permalien : \url{https://data.gouv.fr/dataset/53698ee6a3a729239d2035cc}\newline

\par
\noindent
    Cette API proposée par la Semitan permet effectuer les opérations :

\begin{itemize}

\item
  Lister les arrêts (à proximité ou non)
\item
  Obtenir le temps d'attente à une zone d'arrêt
\item
  Récupérer les horaires à un arrêt
\end{itemize}

Le format de retour est JSON.

Cette même API est utilisée par l'application mobile TAN officielle
(téléchargeable gratuitement pour iPhone et Android).


\vspace{0.5cm}
\needspace{3\baselineskip} \rule{4cm}{0.25pt}\newline\textbf{Aussi disponible du même producteur :}\begin{itemize}
\item \href{https://data.gouv.fr/dataset/536993b0a3a729239d204273}{Eléments de la charte graphique TAN}
\item \href{https://data.gouv.fr/dataset/5369970ba3a729239d204bbe}{Info-trafic TAN prévisionnel}
\item \href{https://data.gouv.fr/dataset/5369970ba3a729239d204bbf}{Info-trafic TAN temps réel}
\end{itemize}

\clearpage
\section{Service départemental d'incendie et de secours du Tarn}


\begin{center}
  \includegraphics[width=3cm]{images/orga/2015-04-02_a0c1eee01c064d1da6e02d5acc78609f_logo_sdis81-100.png}
\end{center}


Le Service Départemental d'Incendie et de Secours du Tarn est un
établissement public administratif, dont la particularité est d'être
placé sous une double autorité :

\begin{itemize}
\item
  celle du président de son conseil d'administration, pour la gestion
  administrative et financière de l'établissement public ;
\item
  celle du Préfet du Tarn, pour le domaine opérationnel.
\end{itemize}

Il dépend du Ministère de l'Intérieur. En France, les SDIS sont
regroupés en 5 catégories définies suivant différents critères : la
taille de la population défendue, le budget et les effectifs. Le SDIS du
Tarn est un SDIS de 3ème catégorie défendant 323 communes. Il est dirigé
par un officier supérieur de sapeurs-pompiers.Le conseil
d'administration du SDIS comprend des conseillers généraux, des maires
et des élus des établissements de coopération intercommunaux (EPCI). Il
règle, par ses délibérations, les affaires relatives à l'administration
du SDIS 81. Il est garant du bon fonctionnement de la structure et a
comme objectif d'optimiser la gestion du service public.


\vspace{0.5cm}

\needspace{12\baselineskip}
\subsection*{Centres d'incendie et de secours du Tarn (CIS)
}\index{batiments}\index{boundaries}\index{donnees!ouvertes}\index{passerelle!inspire}\index{service!dincendie}
  \begin{wrapfigure}{r}{2.5cm}
    \centering
    \qrcode[nolink]{https://data.gouv.fr/dataset/551d22fb88ee383095cf2007}
  \end{wrapfigure}

Licence : \textbf{Licence Ouverte version 2.0
}\newline
Créé le : 2015-04-02\newline
Modifié le : 2019-02-08\newline
Popularité : 1 réutilisation,  0 suivi\newline
Mots-clé : \emph{batiments, boundaries, donnees-ouvertes, passerelle-inspire, service-dincendie
}\newline
Permalien : \url{https://data.gouv.fr/dataset/551d22fb88ee383095cf2007}\newline

\par
\noindent
    Centres d'incendie et de secours du Tarn est la représentation numérique
des 31 casernes de sapeurs-pompiers présentes sur le territoire
départemental.

Les centres d'incendie et de secours sont les unités territoriales
chargées principalement des missions de secours. Ils sont créés et
classés par arrêté du préfet de département en centres de secours
principaux, centres de secours et centres de première intervention en
application de l'article 1424-1 du code général des collectivités
territoriales, en fonction du schéma départemental d'analyse et de
couverture des risques et du règlement opérationnel. Dans le Tarn, les
CIS sont répartis en centres de secours principaux (CSP), centres de
1ière catégorie (CS1), centres de 2ième catégorie (CS2) et centres de
3ième catégorie (CS3).

\textbf{Origine}

Saisie sur orthophotoplan IGN.

\textbf{Organisations partenaires}

SDIS 81

➞
\href{https://geo.data.gouv.fr/fr/datasets/5d14e9e14286f15e87f321bd68d1f4ac21040106}{Consulter
cette fiche sur geo.data.gouv.fr}


\vspace{0.5cm}
\needspace{12\baselineskip}
\subsection*{Hydrants du Tarn
}\index{donnees!ouvertes}\index{eau}\index{passerelle!inspire}\index{service!dincendie}\index{structure}
  \begin{wrapfigure}{r}{2.5cm}
    \centering
    \qrcode[nolink]{https://data.gouv.fr/dataset/551d22ffc751df126d0cd8a0}
  \end{wrapfigure}

Licence : \textbf{Licence Ouverte version 2.0
}\newline
Créé le : 2015-04-02\newline
Modifié le : 2019-02-08\newline
Popularité : 1 réutilisation,  0 suivi\newline
Mots-clé : \emph{donnees-ouvertes, eau, passerelle-inspire, service-dincendie, structure
}\newline
Permalien : \url{https://data.gouv.fr/dataset/551d22ffc751df126d0cd8a0}\newline

\par
\noindent
    Hydrants du Tarn est la représentation numérique de l'implantation des
poteaux et bouches incendie installés dans le département du Tarn.

\textbf{Origine}

Les données sont essentiellement collectées sur le terrain par les
sapeurs-pompiers du département lors des contrôles annuels et lors de
leurs déplacements ou missions qu'ils sont amenés à effectuer sur le
terrain. Les services des eaux contribuent aussi pour une large part à
l'enrichissement de cette base départementale. La mise à jour de la base
se fait à partir de relevés gps fournis par les acteurs de terrain ou de
saisies basées sur l'orthophotoplan de l'IGN.

\textbf{Organisations partenaires}

SDIS 81

➞
\href{https://geo.data.gouv.fr/fr/datasets/9e845cb922fa75662bd2867cf31acd0c45bab7f0}{Consulter
cette fiche sur geo.data.gouv.fr}


\vspace{0.5cm}
\needspace{12\baselineskip}
\subsection*{Points d'aspiration du Tarn (PA)
}\index{donnees!ouvertes}\index{eau}\index{passerelle!inspire}\index{service!dincendie}\index{structure}
  \begin{wrapfigure}{r}{2.5cm}
    \centering
    \qrcode[nolink]{https://data.gouv.fr/dataset/551d230288ee3834e6cf2007}
  \end{wrapfigure}

Licence : \textbf{Licence Ouverte version 2.0
}\newline
Créé le : 2015-04-02\newline
Modifié le : 2019-02-08\newline
Popularité : 1 réutilisation,  0 suivi\newline
Mots-clé : \emph{donnees-ouvertes, eau, passerelle-inspire, service-dincendie, structure
}\newline
Permalien : \url{https://data.gouv.fr/dataset/551d230288ee3834e6cf2007}\newline

\par
\noindent
    Points d'aspiration du Tarn est la représentation numérique de points
d'eau naturels et artificiels, de citernes aériennes ou enterrées, de
réserves à l'air libre, de réserves souples ou bâches à eau et autres
réservoirs fixes, tous permettant de concourir efficacement à la défense
extérieure contre l'incendie (DECI). Il peuvent être situés tant sur le
domaine publique que sur le domaine privé.

\textbf{Origine}

Les données sont essentiellement collectées sur le terrain par les
sapeurs-pompiers du département lors des contrôles annuels et lors de
leurs déplacements ou missions qu'ils sont amenés à effectuer sur le
terrain. La mise à jour de la base se fait à partir de relevés gps
fournis par les acteurs de terrain ou de saisies basées sur
l'orthophotoplan de l'IGN.

\textbf{Organisations partenaires}

SDIS 81

➞
\href{https://geo.data.gouv.fr/fr/datasets/0703bfd2340721f26eaf1fa8a094bfddcd9605ff}{Consulter
cette fiche sur geo.data.gouv.fr}


\vspace{0.5cm}

\clearpage
\section{SETRAM}


\begin{center}
  \includegraphics[width=3cm]{images/orga/2a_c96b6e14aa4f0eb9cb9b757ccd7813-100.jpg}
\end{center}


La Setram gère le réseau des transports en commun de Le Mans Métropole.
C'est une Société d'Économie Mixte délégataire de service public depuis
1974. 19 communes sont desservies par le réseau qui propose divers
services de mobilité : bus, bus à haut niveau de service, tramway, ou
encore vélo à assistance électrique.


\vspace{0.5cm}

\needspace{12\baselineskip}
\subsection*{GTFS du réseau des transports Bus et Tramway SETRAM circulant sur le
territoire Le Mans Métropole
}
  \begin{wrapfigure}{r}{2.5cm}
    \centering
    \qrcode[nolink]{https://data.gouv.fr/dataset/5acb75f0c751df341652c886}
  \end{wrapfigure}

Licence : \textbf{Open Data Commons Open Database License (ODbL)
}\newline
Créé le : 2018-04-09\newline
Modifié le : 2018-11-06\newline
De 2018-04-09 à 2018-07-06\newline
Granularité : à l'EPCI\newline
Mise à jour : irrégulière\newline
Popularité : 1 réutilisation,  1 suivi\newline
Mots-clé : \emph{aucun
}\newline
Permalien : \url{https://data.gouv.fr/dataset/5acb75f0c751df341652c886}\newline

\par
\noindent
    Données arrêts, parcours et horaires théoriques de tous les bus et
tramways du réseau SETRAM circulant sur le territoire de le Mans
Métropole.


\vspace{0.5cm}

\clearpage
\section{SyDEV}


\begin{center}
  \includegraphics[width=3cm]{images/orga/2015-01-15_424f865e93f04623b5dec68714cdfa71_LOGOCOULEUR-_ECRITURE_NOIRE_1-100.jpg}
\end{center}


\textbf{Le SyDEV : le syndicat garant du service public de la
distribution des énergies en Vendée}

Le SyDEV est le syndicat départemental d'énergie auquel adhèrent les 282
communes et 28 intercommunalités vendéennes. A ce titre, il agit pour
leur compte par transfert de compétences.

\textbf{Ses missions dans le domaines des énergies pour le compte de ses
adhérents}

Propriétaire des 22 000 Km de réseaux électriques basse et moyenne
tension et de 2400 Km de réseaux de gaz naturel, le SyDEV a confié à
ERDF la gestion et l'exploitation de ses réseaux électriques par la
signature, en 1992, d'un contrat de concession d'une durée de 40 ans. Il
a également signé un contrat de concession avec GrDF en 1998, pour une
durée de 40 ans et développé ensuite la desserte gazière par voie de
délégations de service public, attribuées à GrDF et Sorégies. Le SyDEV
contrôle le bon accomplissement de ces missions de service public
déléguées.

Il assure la maîtrise d'ouvrage des travaux d'effacements, d'extensions,
de renforcements et sécurisations des réseaux électriques.

S'inscrivant dans la transition énergétique avec le Plan Climat Energie
Collectivité, il conseille et accompagne ses adhérents pour toutes leurs
actions de maîtrise de la demande en énergie et de développement de
l'utilisation des énergies renouvelables. Le SyDEV a également créé la
SEML Vendée Energie pour la production de ces énergies nouvelles.

Il contribue à la solidarité en assurant l'information sur les tarifs
sociaux de l'énergie auprès des élus et acteurs sociaux vendéens,
interlocuteurs des usagers en état de précarité énergétique.

\textbf{Ses missions sur l'éclairage public et la signalisation
lumineuse}

Le SyDEV assure la maîtrise d'ouvrage des installations d'éclairage
public et de signalisation lumineuse liée à la circulation routière. Il
assure également la maintenance et le fonctionnement de ces
installations. Le SyDEV participe à la mise en place d'un éclairage
public économe, sécurisé et respectueux de l'environnement.

\textbf{Son implication dans de grands projets vendéens}

Le SyDEV est impliqué dans de grands projets tels que le développement
du très haut débit avec le Conseil Général de la Vendée, la
participation active à la transition énergétique avec le projet Smart
Grid (réseaux intelligents) Vendée en partenariat avec ERDF notamment,
le développement de l'infrastructure de recharge pour véhicules
électriques et la participation à l'étude pour une usine de dessalement
de l'eau de mer pilotée par Vendée Eau.

\textbf{Contact SyDEV : 02 51 45 88 00 -- www.sydev-vendee.fr}


\vspace{0.5cm}

\needspace{12\baselineskip}
\subsection*{Stations de recharge pour véhicules électriques sur le territoire
vendéen
}\index{bornes!de!recharge}\index{infrastructure!de!recharge}\index{irve}\index{stations!de!recharge}\index{sydev}\index{vehicules!electriques}\index{vendee}\index{voitures!electriques}
  \begin{wrapfigure}{r}{2.5cm}
    \centering
    \qrcode[nolink]{https://data.gouv.fr/dataset/54b7ce3ec751df56d15fa5a2}
  \end{wrapfigure}

Licence : \textbf{Licence Ouverte
}\newline
Créé le : 2015-01-15\newline
Modifié le : 2019-02-25\newline
Granularité : au département\newline
Mise à jour : ponctuelle\newline
Popularité : 1 réutilisation,  3 suivis\newline
Mots-clé : \emph{bornes-de-recharge, infrastructure-de-recharge, irve, stations-de-recharge, sydev, vehicules-electriques, vendee, voitures-electriques
}\newline
Permalien : \url{https://data.gouv.fr/dataset/54b7ce3ec751df56d15fa5a2}\newline

\par
\noindent
    Déploiement des infrastructures de recharge pour les véhicules
électriques sur le territoire vendéen.


\vspace{0.5cm}

\clearpage
\section{Territoires Numériques Bourgogne-Franche-Comté}


\begin{center}
  \includegraphics[width=3cm]{images/orga/9c_66c43f80bd4099b1f20618731b47ca-100.jpg}
\end{center}


\href{http://www.territoires-numeriques-bfc.fr/}{Territoires Numériques
Bourgogne-Franche-Comté} est un groupement d'intérêt public. Il a pour
objet de mettre en œuvre une plate-forme électronique de services
dématérialisés fournis aux usagers (particuliers, entreprises,
associations, etc.) par l'ensemble des organismes publics ou privés
chargés d'une mission de service public ou d'intérêt général, dans une
perspective de modernisation de l'administration et d'amélioration de
l'accès aux services publics. L'offre de services inclut la
dématérialisation, des services à destination des entreprises et des
citoyens.

\textbf{\emph{C'est le premier GIP d'intérêt général dédié à
l'administration numérique}} Nos valeurs

\begin{itemize}

\item
  Mutualisation \& Partage
\item
  Solidarité financière
\item
  Libre administration
\item
  Egalité d'accès
\end{itemize}

\#\#\#\# Nos motivations

\begin{itemize}

\item
  Développer une coopération territoriale et une gouvernance partagée
\item
  Donner à « comprendre le numérique »
\item
  Contribuer à la simplification
\item
  Promouvoir le Logiciel Libre
\item
  Proposer une pépinière à projets innovants
\item
  Développer l'écoute et la participation européenne et française
\end{itemize}


\vspace{0.5cm}

\needspace{12\baselineskip}
\subsection*{Adhérents de Territoires Numériques Bourgogne-Franche-Comté
}\index{adherents}\index{ternum}
  \begin{wrapfigure}{r}{2.5cm}
    \centering
    \qrcode[nolink]{https://data.gouv.fr/dataset/5b4f5453c751df0bb1c28757}
  \end{wrapfigure}

Licence : \textbf{Licence Ouverte
}\newline
Créé le : 2018-07-18\newline
Modifié le : 2019-03-07\newline
Granularité : à la commune\newline
Mise à jour : trimestrielle\newline
Popularité : 1 réutilisation,  0 suivi\newline
Mots-clé : \emph{adherents, ternum
}\newline
Permalien : \url{https://data.gouv.fr/dataset/5b4f5453c751df0bb1c28757}\newline

\par
\noindent
    Liste des adhérents de Territoires Numériques Bourgogne-Franche-Comté
sous forme de fichier tabulaire ouvert.

Description des attributs : département : département de l'adhérent name
: nom de l'organisation (ex : collective territoriale) adhérente à
Territoires numériques postalAddress : Adresse postale de l'organisation
postalCode : Code postal city : Commune ou se situe l'organisation phone
: numéro de téléphone siren : Code SIREN fourni par l'INSEE


\vspace{0.5cm}
\needspace{12\baselineskip}
\subsection*{Liste des données demandées et produites des participants de l'Atelier
Opendata aux IntercoTOUR de Bourgogne-Franche-Comté
}\index{interconnectes}\index{liste!de!demande!de!donnees}\index{liste!de!donnees!produites}\index{scdl}
  \begin{wrapfigure}{r}{2.5cm}
    \centering
    \qrcode[nolink]{https://data.gouv.fr/dataset/5bbb1ab9634f4151b3507f00}
  \end{wrapfigure}

Licence : \textbf{Licence Ouverte
}\newline
Créé le : 2018-10-08\newline
Modifié le : 2018-10-08\newline
De 2018-09-28 à 2018-09-29\newline
Granularité : à la région\newline
Mise à jour : irrégulière\newline
Popularité : 1 réutilisation,  0 suivi\newline
Mots-clé : \emph{interconnectes, liste-de-demande-de-donnees, liste-de-donnees-produites, scdl
}\newline
Permalien : \url{https://data.gouv.fr/dataset/5bbb1ab9634f4151b3507f00}\newline

\par
\noindent
    Le fichier disponible a été construit lors d'un atelier Opendata
\href{http://www.interconnectes.com/bourgognefc-2018/}{dans le cadre des
IntercoTOUR de Bourgogne-Franche-Comté} du 28 Septembre 2018.

Le tableur présente les acteurs présents, ce qu'il produisent et ce
qu'ils demandent comme données ouvertes du
\href{http://opendatalocale.net/scdl/}{Socle Commun des Données
Locales}.


\vspace{0.5cm}
\needspace{3\baselineskip} \rule{4cm}{0.25pt}\newline\textbf{Aussi disponible du même producteur :}\begin{itemize}
\item \href{https://data.gouv.fr/dataset/5b505320c751df2603927d42}{Annuaire des profils acheteurs des adhérents de Territoires Numériques Bourgogne-Franche-Comté}
\end{itemize}

\clearpage
\section{Tesla}


\begin{center}
  \includegraphics[width=3cm]{images/orga/a1_623e0cb97e412b9635688efb554b2f-100.png}
\end{center}


Tesla


\vspace{0.5cm}

\needspace{12\baselineskip}
\subsection*{Stations Supercharger Tesla
}\index{irve}\index{tesla}
  \begin{wrapfigure}{r}{2.5cm}
    \centering
    \qrcode[nolink]{https://data.gouv.fr/dataset/54637a75c751df446989ec20}
  \end{wrapfigure}

Licence : \textbf{Licence Ouverte
}\newline
Créé le : 2014-11-12\newline
Modifié le : 2018-11-30\newline
De 2014-06-01 à 2018-11-28\newline
Granularité : au pays\newline
Mise à jour : ponctuelle\newline
Popularité : 2 réutilisations,  2 suivis\newline
Mots-clé : \emph{irve, tesla
}\newline
Permalien : \url{https://data.gouv.fr/dataset/54637a75c751df446989ec20}\newline

\par
\noindent
    Description et localisation des stations Supercharger TESLA - Bornes de
recharge rapides pour véhicules électriques Tesla uniquement


\vspace{0.5cm}

\clearpage
\section{Toulouse-Métropole.data}


\begin{center}
  \includegraphics[width=3cm]{images/orga/b0_a9a37d5416408798c88bd29c57708f-100.png}
\end{center}


Référencement des catalogues des données hébergées sur le portail


\vspace{0.5cm}

\needspace{12\baselineskip}
\subsection*{Jeux de données - Tisseo SMTC
}\index{liste!de!jeux!de!donnees}
  \begin{wrapfigure}{r}{2.5cm}
    \centering
    \qrcode[nolink]{https://data.gouv.fr/dataset/53699789a3a729239d204d00}
  \end{wrapfigure}

Licence : \textbf{Licence Ouverte
}\newline
Créé le : 2013-11-18\newline
Modifié le : 2016-02-12\newline
Popularité : 1 réutilisation,  0 suivi\newline
Mots-clé : \emph{liste-de-jeux-de-donnees
}\newline
Permalien : \url{https://data.gouv.fr/dataset/53699789a3a729239d204d00}\newline

\par
\noindent
    Les jeux de données fournis par Tisseo SMTC pour data.gouv.fr.


\vspace{0.5cm}
\needspace{12\baselineskip}
\subsection*{Jeux de données - Toulouse métrople
}\index{liste!de!jeux!de!donnees}
  \begin{wrapfigure}{r}{2.5cm}
    \centering
    \qrcode[nolink]{https://data.gouv.fr/dataset/53699789a3a729239d204d01}
  \end{wrapfigure}

Licence : \textbf{Licence Ouverte
}\newline
Créé le : 2013-11-18\newline
Modifié le : 2016-01-27\newline
Popularité : 1 réutilisation,  0 suivi\newline
Mots-clé : \emph{liste-de-jeux-de-donnees
}\newline
Permalien : \url{https://data.gouv.fr/dataset/53699789a3a729239d204d01}\newline

\par
\noindent
    Les jeux de données fournis par Toulouse métrople pour data.gouv.fr.


\vspace{0.5cm}
\needspace{3\baselineskip} \rule{4cm}{0.25pt}\newline\textbf{Aussi disponible du même producteur :}\begin{itemize}
\item \href{https://data.gouv.fr/dataset/53699778a3a729239d204cd7}{Jeux de données - Mairie de Balma}
\item \href{https://data.gouv.fr/dataset/5369977aa3a729239d204cda}{Jeux de données - Mairie de Seilh}
\item \href{https://data.gouv.fr/dataset/5369977aa3a729239d204cdb}{Jeux de données - Mairie de Toulouse}
\item \href{https://data.gouv.fr/dataset/5369977da3a729239d204ce4}{Jeux de données - Office de tourisme So Toulouse}
\end{itemize}

\clearpage
\section{UFC Que Choisir}


\begin{center}
  \includegraphics[width=3cm]{images/orga/55_7d42e53d36467694dc568109cd4e00-100.png}
\end{center}


UFC-Que Choisir, association à but non lucratif créée en 1951, est la
doyenne des associations de consommateurs d'Europe occidentale.

L'UFC-Que Choisir est au service des consommateurs pour les informer,
les conseiller et les défendre. \url{https://www.quechoisir.org/}


\vspace{0.5cm}

\needspace{12\baselineskip}
\subsection*{Délai d'attente rendez-vous gynécologue
}\index{depassements!honoraires}\index{fracture!sanitaire}\index{gynecologues}\index{medecins}\index{medecins!specialistes}\index{sante}\index{tarifs!medecins}\index{ufc!que!choisir}
  \begin{wrapfigure}{r}{2.5cm}
    \centering
    \qrcode[nolink]{https://data.gouv.fr/dataset/5369923ea3a729239d203e93}
  \end{wrapfigure}

Licence : \textbf{Licence Ouverte
}\newline
Créé le : 2013-12-20\newline
Modifié le : 2016-03-08\newline
De 2012-01-01 à 2012-12-31\newline
Granularité : à la commune\newline
Popularité : 1 réutilisation,  3 suivis\newline
Mots-clé : \emph{depassements-honoraires, fracture-sanitaire, gynecologues, medecins, medecins-specialistes, sante, tarifs-medecins, ufc-que-choisir
}\newline
Permalien : \url{https://data.gouv.fr/dataset/5369923ea3a729239d203e93}\newline

\par
\noindent
    Pour connaître les délais d'attente moyens pour une prise de rendez-vous
auprès d'un médecin spécialiste, l'UFC-Que Choisir a, en septembre et
octobre 2012, mené une campagne d'appels auprès de 230 gynécologues.


\vspace{0.5cm}
\needspace{3\baselineskip} \rule{4cm}{0.25pt}\newline\textbf{Aussi disponible du même producteur :}\begin{itemize}
\item \href{https://data.gouv.fr/dataset/5369923fa3a729239d203e96}{Délai d'attente rendez-vous ophtalmologiste}
\item \href{https://data.gouv.fr/dataset/5369923fa3a729239d203e97}{Délai d'attente rendez-vous pédiatre}
\end{itemize}

\clearpage
\section{UFO Urbanisme Collaboratif}


\begin{center}
  \includegraphics[width=3cm]{images/orga/b9_6dc610047045d6bcda792604c58535-100.png}
\end{center}


\href{http://unlimitedcities.org}{unlimitedcities.org}

UFO a été créée par des architectes urbanistes et des designers en 2010.
Rejoints par des philosophes, des codeurs et des sociologues nous
pensons que les mutations positives des villes dépendront de la capacité
à faire fonctionner l'intelligence collective de la société civile pour
\textbf{transformer les territoires de façon collaborative}. Nous sommes
certains que les visions des villes numériques actuelles, qui proposent
l'optimisation industrielle de systèmes urbains dépassés sont plus des
freins que des progrès, même si c'est bien agréable de savoir dans
combien de minutes arrive le bus. Nous pensons aussi que les villes
numériques vues du coté des usages ne sont qu'une petite partie de la
réponse qui ne prend pas assez en compte l'évolution des milieux humains
que sont nos villes, où la nature devra se combiner d'une façon ou d'une
autre avec l'information omniprésente.

Ce qui nous intéresse c'est \textbf{la ville d'après la numérisation
technique et servicielle}. C'est la ville contributive, conçue en amont
comme telle, à la fois neutre, comme on dit neutralité des réseaux, non
programmée définitivement, c'est-à-dire disponible pour l'émancipation
des individus et des organisations, et diversifiée, en l'occurrence
urba-diversifiée où les vides, les pleins, le banal et l'exceptionnel se
tissent à toutes les échelles. Cette ville fractale, respectueuse des
devenirs individuels et collectifs, des biens communs et de l'intérêt
public, nous avons le sentiment qu'elle peut naitre plus aujourd'hui
qu'hier parce que l'humanité à des moyens d'accès à la connaissance et
d'échanges comme elle n'en a jamais eu. Mais en même temps, d'autres
forces de concentration, de surveillance et de coercition sont à
l'oeuvre favorisées par ces mêmes technologies.

Les jeux sont donc ouverts et les activités d'UFO sont d'apporter des
méthodes d'urbanisme collaboratif et \textbf{des outils d'intelligence
collective pour commencer à agir ensemble}, non pas en 2025, mais ici et
maintenant. Depuis 2013, des villes, des concepteurs, des citoyens et
des organisations qui partagent notre désir de co-imagination et de
co-conception des villes contributives de demain ont commencé à
appliquer nos méthodes d'innovations ouvertes. L'intérêt du surf c'est
que l'équilibre rend difficile des mouvements disgracieux. On tombe vite
si on ne perçoit pas la dynamique de la vague. Avec l'urbanisme
collaboratif, l'équilibre et les échanges sont basés sur la confiance en
l'autre, qui n'est pas là seulement pour participer à un jeu de rôle
plus ou moins sincère, mais pour faire, pour collaborer, pour apporter
un élément du puzzle. Alors un des meilleurs alliés et des meilleurs
atouts pour créer cette confiance ce sont les données ouvertes, et en
particulier d'ouvrir l'accès aux données brutes, non traitées, non
manipulées pour permettre d'autres approches, d'autres conclusions.

Alors nous sommes très, très content, en tant que \textbf{lilliputiens
des producteurs de données}, de vous donner accès à des premiers jeux de
données, issus de l'outil Unlimited Cities / Villes sans limite,
élaborés par les habitants de Rennes, de Montpellier et d'Evreux, qui
seront bientôt rejoints par ceux d'Helsinki, de Rio et de Sendai. Très
rapidement, nous allons vous fournir du gras et du lourd, car nous
commençons à développer des collaborations avec des opérateurs de
téléphone pour développer notre indicateur universel de la qualité de
vie urbaine URBANDASH, et là ça va chiffrer\ldots{} \textbf{il va
falloir du bon gros gun de dataviz, de l'algo bien extreme et de la
compression affutée.}

Alain Renk pour UFO**


\vspace{0.5cm}

\needspace{12\baselineskip}
\subsection*{Unlimited Cities - Evreux Saint-Louis --
}\index{collaboratif}\index{evreux}\index{participation}\index{urbanisme}
  \begin{wrapfigure}{r}{2.5cm}
    \centering
    \qrcode[nolink]{https://data.gouv.fr/dataset/5369a31ba3a729239d2069af}
  \end{wrapfigure}

Licence : \textbf{Creative Commons Attribution
}\newline
Créé le : 2014-03-19\newline
Modifié le : 2015-09-29\newline
De 2013-06-20 à 2013-10-24\newline
Popularité : 1 réutilisation,  1 suivi\newline
Mots-clé : \emph{collaboratif, evreux, participation, urbanisme
}\newline
Permalien : \url{https://data.gouv.fr/dataset/5369a31ba3a729239d2069af}\newline

\par
\noindent
    Les données en Open Data de - Unlimited Cities / Evreux Saint-Louis -
contiennent les résultats issus des choix émis par les participants lors
de l'expérimentation et de la manipulation de l'application Unlimited
Cities sur le site d'Evreux Saint-Louis. Elles contiennent: - le nombre
de mix validés - le point de vue choisi par l'utilisateur - La date de
validation du mix - Le commentaire formulé par l'utilisateur - le niveau
choisi par l'utilisateur pour chacun des 6 critères proposés (densité,
nature, mobilité, vie de quartier, numérique et créativité) -
l'emplacement de l'utilisateur s'il a accepté d'être géolocalisé.

Afin de garder le respect de la vie privée, les données ont été
anonymisées.

\textbf{Description de l'outil :} L'application numérique nomade
«Unlimited Cities» permet de déployer un dispositif d'urbanisme
collaboratif en phase de programmation, c'est à dire au moment où les
projets d'aménagement des espaces publics, d'urbanisme et d'architecture
peuvent encore évoluer.

Concrètement, l'application Unlimited Cities, propose trois vues
photographiques par quartier étudié et permet aux usagers de faire
varier des curseurs d'intensités sur une échelle allant de un à cinq
pour différents critères : densité, nature, mobilité, vie de quartier,
créativité et numérique. Ces variations entraînent des modifications
instantanées de l'image photographique du site (il est possible
d'ajouter ou d'enlever des éléments urbains, de faire pousser des
arbres, de faire monter les bâtiments\ldots{}). Les six thèmes
s'articulent de sorte à proposer 15625 combinaisons possibles par point
de vue. Dans l'interface ces compositions sont appelées des mix.
L'utilisateur peut ainsi composer sa vision de l'évolution du quartier
et l'agrémenter d'un commentaire. Les mixs validés sont hyperréalistes.


\vspace{0.5cm}
\needspace{12\baselineskip}
\subsection*{Unlimited Cities - Montpellier la Pompignane --
}\index{collaboratif}\index{montpellier}\index{participation}\index{urbanisme}\index{villes!sans!limite}
  \begin{wrapfigure}{r}{2.5cm}
    \centering
    \qrcode[nolink]{https://data.gouv.fr/dataset/5369a31ba3a729239d2069b1}
  \end{wrapfigure}

Licence : \textbf{Creative Commons Attribution
}\newline
Créé le : 2014-01-22\newline
Modifié le : 2017-01-03\newline
De 2013-06-01 à 2013-09-30\newline
Popularité : 1 réutilisation,  2 suivis\newline
Mots-clé : \emph{collaboratif, montpellier, participation, urbanisme, villes-sans-limite
}\newline
Permalien : \url{https://data.gouv.fr/dataset/5369a31ba3a729239d2069b1}\newline

\par
\noindent
    Les données en Open Data de - Unlimited Cities / Montpellier la
Pompignane - contiennent les résultats issus des choix émis par les
participants lors de l'expérimentation et de la manipulation de
l'application. Elles contiennent: - le nombre de mix validés - le point
de vue choisi par l'utilisateur - La date de validation du mix - Le
commentaire formulé par l'utilisateur - le niveau choisi par
l'utilisateur pour chacun des 6 critères proposés (densité, nature,
mobilité, vie de quartier, numérique et créativité) - emplacement de
l'utilisateur s'il a accepté d'être géolocalisé.

Afin de garder le respect de la vie privée, les données ont été
anonymisées.

\textbf{Description de l'outil :} L'application numérique nomade
«Unlimited Cities» permet de déployer un dispositif d'urbanisme
collaboratif en phase de programmation, c'est à dire au moment où les
projets d'aménagement des espaces publics, d'urbanisme et d'architecture
peuvent encore évoluer.

Concrètement, l'application Unlimited Cities, propose trois vues
photographiques par quartier étudié et permet aux usagers de faire
varier des curseurs d'intensités sur une échelle allant de un à cinq
pour différents critères : densité, nature, mobilité, vie de quartier,
créativité et numérique. Ces variations entraînent des modifications
instantanées de l'image photographique du site (il est possible
d'ajouter ou d'enlever des éléments urbains, de faire pousser des
arbres, de faire monter les bâtiments\ldots{}). Les six thèmes
s'articulent de sorte à proposer 15625 combinaisons possibles par point
de vue. Dans l'interface ces compositions sont appelées des mix.
L'utilisateur peut ainsi composer sa vision de l'évolution du quartier
et l'agrémenter d'un commentaire. Les mixs validés sont hyperréalistes.


\vspace{0.5cm}
\needspace{3\baselineskip} \rule{4cm}{0.25pt}\newline\textbf{Aussi disponible du même producteur :}\begin{itemize}
\item \href{https://data.gouv.fr/dataset/5369a31ca3a729239d2069b4}{Unlimited Cities - Rennes Gare Sud --}
\item \href{https://data.gouv.fr/dataset/586baa6fc751df0fda2b7154}{Unlimited Cities - Saint-Nazaire}
\end{itemize}

\clearpage
\section{Université Paris 13}


\begin{center}
  \includegraphics[width=3cm]{images/orga/d0_47dfc1604d4930b847deccbecad5aa-100.png}
\end{center}


L'université Paris 13 est l'une des treize universités qui ont succédé à
la Sorbonne après 1968.

Elle compte aujourd'hui 23 650 étudiant·e·s, répartis sur cinq campus,
en formation initiale ou continue. Réellement pluridisciplinaire,
l'Université Paris 13 est un pôle majeur d'enseignement et de recherche
au nord de Paris.


\vspace{0.5cm}

\needspace{12\baselineskip}
\subsection*{Données de scolarité de l'université Paris 13
}\index{enseignement!superieur}\index{scolarite}\index{seine!saint!denis}\index{universite}\index{universite!paris!13}\index{up13}
  \begin{wrapfigure}{r}{2.5cm}
    \centering
    \qrcode[nolink]{https://data.gouv.fr/dataset/58e34f7dc751df5d2777388c}
  \end{wrapfigure}

Licence : \textbf{Licence Ouverte
}\newline
Créé le : 2017-04-04\newline
Modifié le : 2018-06-22\newline
Mise à jour : annuelle\newline
Popularité : 1 réutilisation,  1 suivi\newline
Mots-clé : \emph{enseignement-superieur, scolarite, seine-saint-denis, universite, universite-paris-13, up13
}\newline
Permalien : \url{https://data.gouv.fr/dataset/58e34f7dc751df5d2777388c}\newline

\par
\noindent
    L'université Paris 13 a enregistré dans son système d'information
(logiciel Apogée), des données sur l'inscription des étudiant·e·s pour
chaque année universitaire entre 2006(-2007) et 2015(-2016). Ces données
portent sur les diplômes préparés, les étapes pour y parvenir, le régime
(s'il s'agit de formation initiale ou d'apprentissage), les composantes
concernées (UFR, IUT, etc.), et l'origine des étudiant·e·s (type de Bac,
académie d'origine, nationalité). Chaque entrée concerne l'inscription
principale d'un·e étudiant·e à l'université pour une année. Les
attributs de ces données sont les suivants.

\begin{itemize}

\item
  CODE\_INDIVIDU Donnée masquée
\item
  ANNEE\_INSCRIPTION Année d'inscription : 2006 pour 2006-2007, etc.
\item
  LIB\_DIPLOME Nom du diplôme
\item
  NIVEAU\_DANS\_LE\_DIPLOME 1, 2,\ldots{} pour master 1, licence 2, etc.
\item
  NIVEAU\_APRES\_BAC 1, 2,\ldots{} pour Bac+1, Bac+2,\ldots{}
\item
  LIBELLE\_DISCIPLINE\_DIPLOME Rattachement du diplôme à une discipline
\item
  CODE\_SISE\_DIPLOME Code du système d'information sur le suivi de
  l'étudiant
\item
  CODE\_ETAPE Code interne d'une étape (année, parcours) de diplôme
\item
  LIBELLE\_COURT\_ETAPE Nom court de l'étape
\item
  LIBELLE\_LONG\_ETAPE Nom plus intelligible de l'étape
\item
  LIBELLE\_COURT\_COMPOSANTE Nom de la composante (UFR, IUT etc.)
\item
  CODE\_COMPOSANTE Code numérique de la composante (inutilisé)
\item
  REGROUPEMENT\_BAC Type de Bac (L, ES, S, techno STMG, techno
  ST2S,\ldots{})
\item
  LIBELLE\_ACADEMIE\_BAC Académie du Bac (Créteil, Versailles,
  étranger,\ldots{})
\item
  CONTINENT Déduit de la nationalité qui est une donnée masquée
\item
  LIBELLE\_REGIME Formation initiale, continue, pro, apprentissage
\end{itemize}

L'université Paris 13 rend publique une partie de ce jeu de données à
travers plusieurs ressources, dans le respect de l'anonymat de ses
étudiant·e·s.

Partant de 213 289 entrées qui correspondent à toutes les inscriptions
des 106 088 individus ayant étudié à l'université Paris 13 au cours des
dix années universitaires entre 2006(-2007) et 2015(-2016), nous avons
sélectionné plusieurs ressources correspondant chacune à une partie des
données. Pour produire chaque ressource nous avons choisi un petit
nombre d'attributs, puis nous avons supprimé une petite proportion des
entrées, de façon à satisfaire une contrainte de k-anonymisation avec k
= 5, c'est à dire de faire en sorte que, dans chaque ressource, chaque
entrée apparait au moins 5 fois à l'identique (autrement l'entrée est
supprimée). Les quatre ressources produites sont matérialisés par les
fichiers suivants.

\begin{itemize}
\item
  Le fichier \texttt{up13\_etapes.csv} concerne les étapes de diplôme,
  il contient les attributs ``CODE\_ETAPE'', ``LIBELLE\_COURT\_ETAPE'',
  ``LIBELLE\_LONG\_ETAPE'', ``NIVEAU\_APRES\_BAC'',
  ``LIBELLE\_COURT\_COMPOSANTE'', ``LIB\_DIPLOME'',
  ``LIBELLE\_DISCIPLINE\_DIPLOME'', ``CODE\_SISE\_DIPLOME'',
  ``NIVEAU\_DANS\_LE\_DIPLOME'' et son anonymisation occasionne une
  perte de 918 entrées.
\item
  Le fichier \texttt{up13\_Academie.csv} concerne l'académie du Bac et
  il contient les attributs ``LIBELLE\_ACADEMIE\_BAC'',
  ``NIVEAU\_APRES\_BAC'', ``NIVEAU\_DANS\_LE\_DIPLOME'', ``CONTINENT'',
  ``LIBELLE\_REGIME'', ``LIB\_DIPLOME'', ``LIBELLE\_COURT\_COMPOSANTE''
  et son anoymisation occasionne la perte de 7525 entrées.
\item
  Le fichier \texttt{up13\_Bac.csv} concerne le type de Bac et le niveau
  atteint après le Bac, il contient les colonnes ``REGROUPEMENT\_BAC'',
  ``NIVEAU\_APRES\_BAC'', ``LIBELLE\_REGIME'', ``CONTINENT'',
  ``LIBELLE\_COURT\_COMPOSANTE'', ``LIB\_DIPLOME'',
  ``NIVEAU\_DANS\_LE\_DIPLOME'' et son anonymisation occasionne la perte
  de 3 933 entrées.
\item
  Le fichier \texttt{up13\_annees\_etapes.csv} concerne le inscriptions
  dans les étapes de diplôme année après année, il contient les colonnes
  ``ANNEE\_INSCRIPTION'', ``LIBELLE\_COURT\_COMPOSANTE'',
  ``NIVEAU\_APRES\_BAC'', ``LIB\_DIPLOME'', ``CODE\_ETAPE'' et son
  anonymisation occasionne la perte de 3 532 entrées.
\end{itemize}

D'autres tableaux extraits de la même donnée initiale et construits
selon la même méthode d'anonymisation, peuvent être fourni sur demande
(préciser les colonnes souhaitées).

Un second ensemble de ressources propose le suivi des étudiant·e·s année
après année, d'étape de diplôme en étape de diplôme. Dans ce jeu de
données, nous appelons \textbf{trace} un tel suivi lorsque l'année
d'inscription a été oubliée et que seule subsiste la séquence. Et nous
appelons \textbf{cursus} une donnée décrivant cette succession d'étapes
au fil des années. Pour l'anonymisation nous avons regroupé les traces
ou les parcours identiques et dès lors qu'il y en avait moins de 10 nous
n'indiquons pas leur nombre, ou, ce qui revient au même, nous mettons ce
nombre à 1 (l'information étant qu'il y a au moins un·e étudiant·e ayant
laissé cette trace ou suivi ce cursus). Cela conduit à oublier un
certain nombre de parcours étudiants trop spécifiques et à n'en
conserver qu'un seul comme témoin.

Partant de 106 088 parcours ou traces, nous produisons les ressources
suivantes.

\begin{itemize}
\item
  Le fichier \texttt{up13\_traces.csv} contient la séquence des code
  d'étapes de diplôme (une trace) et l'anonymisation nous fait oublier
  10 089 traces.
\item
  Le fichier \texttt{up13\_traces\_wt\_etape.csv} contient des traces
  similaires, mais sans le code étape. C'est à dire que seul subsistent
  le diplôme, le niveau après bac et la composante concernée.
  L'anonymisation nous fait oublier 4 447 traces.
\item
  Le fichier \texttt{up13\_traces\_bac\_wt\_etape.csv} contient les
  mêmes données que dans le fichier \texttt{up13\_traces\_wt\_etape.csv}
  mais avec le type de Bac en plus. L'anonymisation nous fait oublier 8
  067 traces.
\item
  Le fichier \texttt{up13\_cursus\_wt\_etape.csv} contient les mêmes
  données que dans le fichier \texttt{up13\_traces\_wt\_etape.csv} avec
  les années d'inscription en plus. L'anonymisation nous fait oublier 8
  324 cursus.
\end{itemize}


\vspace{0.5cm}

\clearpage
\section{Voxe.org}


\begin{center}
  \includegraphics[width=3cm]{images/orga/d9_fba9b7c803472e96416dd9e9ec12d3-100.jpg}
\end{center}


\href{http://www.voxe.org}{Voxe.org} est le comparateur neutre et
international des programmes des candidats aux élections.


\vspace{0.5cm}

\needspace{12\baselineskip}
\subsection*{Programmes des candidats à l'élection présidentielle de 2012
}\index{candidats}\index{elections}\index{programmes!politiques}\index{propositions}
  \begin{wrapfigure}{r}{2.5cm}
    \centering
    \qrcode[nolink]{https://data.gouv.fr/dataset/53699e5ea3a729239d205e6d}
  \end{wrapfigure}

Licence : \textbf{Licence Ouverte
}\newline
Créé le : 2014-03-04\newline
Modifié le : 2016-01-28\newline
Popularité : 1 réutilisation,  0 suivi\newline
Mots-clé : \emph{candidats, elections, programmes-politiques, propositions
}\newline
Permalien : \url{https://data.gouv.fr/dataset/53699e5ea3a729239d205e6d}\newline

\par
\noindent
    Ce jeu de données inclut toutes les propositions des candidats aux
élections présidentielles de 2012.

Documentation de l'API :
\url{http://voxe.org/platform}{]}(http://voxe.org/platform{]}(http://voxe.org/platform))


\vspace{0.5cm}

\clearpage
\section{VroomVroom.fr}


\begin{center}
  \includegraphics[width=3cm]{images/orga/4e_ce04088ce642f0b0237916aba905ba-100.png}
\end{center}


VroomVroom.fr est le comparateur des auto-écoles pour le permis de
conduire. Il permet de comparer les taux de réussite, les
\href{http://www.vroomvroom.fr/}{tarifs auto-écoles}, les
\href{http://www.vroomvroom.fr/prix-permis-de-conduire}{prix du permis
de conduire} et consulter le
\href{http://www.vroomvroom.fr/code-de-la-route}{code de la route} .


\vspace{0.5cm}

\needspace{12\baselineskip}
\subsection*{Taux de réussite des auto-écoles du Morbihan
}\index{auto!ecoles}\index{permis!de!conduire}\index{taux!de!reussite}
  \begin{wrapfigure}{r}{2.5cm}
    \centering
    \qrcode[nolink]{https://data.gouv.fr/dataset/5369a1b9a3a729239d20668b}
  \end{wrapfigure}

Licence : \textbf{Creative Commons Attribution
}\newline
Créé le : 2013-12-16\newline
Modifié le : 2016-02-20\newline
De 2012-01-01 à 2012-12-31\newline
Granularité : au point d'intérêt\newline
Popularité : 1 réutilisation,  0 suivi\newline
Mots-clé : \emph{auto-ecoles, permis-de-conduire, taux-de-reussite
}\newline
Permalien : \url{https://data.gouv.fr/dataset/5369a1b9a3a729239d20668b}\newline

\par
\noindent
    Il s'agit des taux de réussite à tous les permis des auto-écoles du
Morbihan, ainsi que les coordonnées géographiques, adresses et nom des
auto-écoles. Ce document a été réalisé par les équipes de VroomVroom.fr


\vspace{0.5cm}

\clearpage
\section{World Resources Institute}


\begin{center}
  \includegraphics[width=3cm]{images/orga/23_d77d6335754cf1a467339f49107ff7-100.png}
\end{center}


WRI is a global research organization that turns big ideas into action
at the nexus of environment, economic opportunity and human well-being.


\vspace{0.5cm}

\needspace{12\baselineskip}
\subsection*{CAIT - Country Greenhouse Gas Emissions Data
}\index{climate}\index{emissions}\index{energy}\index{greenhouse!gas}
  \begin{wrapfigure}{r}{2.5cm}
    \centering
    \qrcode[nolink]{https://data.gouv.fr/dataset/57c6dff7c751df24c697bae5}
  \end{wrapfigure}

Licence : \textbf{Creative Commons Attribution
}\newline
Créé le : 2016-08-31\newline
Modifié le : 2016-08-31\newline
Granularité : au pays\newline
Mise à jour : annuelle\newline
Popularité : 1 réutilisation,  0 suivi\newline
Mots-clé : \emph{climate, emissions, energy, greenhouse-gas
}\newline
Permalien : \url{https://data.gouv.fr/dataset/57c6dff7c751df24c697bae5}\newline

\par
\noindent
    The CAIT Country GHG emissions collection applies a consistent
methodology to create a six-gas, multi-sector, and internationally
comparable data set for 186 countries.

CAIT enables data analysis by allowing users to quickly narrow down by
year, gas, country/state, and sector. Automatic calculations for percent
changes from prior year, per capita, and per GDP are also available.
Users are presented with clear and customizable data visualizations that
can be readily shared through unique URLs or embedded for further use
online.

Data for Land-Use and Forestry indicator are provided by the Food and
Agriculture Organization of the United Nations (FAO). WRI has been
granted a non-exclusive, non-transferrable right to publish these data.
Therefore, if users wish to republish this dataset in whole or in part,
they should contact FAO directly at copyright@fao.org

Data sources: - Boden, T.A., G. Marland, and R.J. Andres. 2015. Global,
Regional, and National Fossil-Fuel CO2 Emissions. Carbon Dioxide
Information Analysis Center, Oak Ridge National Laboratory, U.S.
Department of Energy, Oak Ridge, Tenn., U.S.A. doi
10.3334/CDIAC/00001\_V2015 Available online
at\url{http://cdiac.ornl.gov/trends/emis/overview_2011.html}. - Food and
Agriculture Organization of the United Nations (FAO). 2014. FAOSTAT
Emissions Database. Rome, Italy: FAO. Available
at:\url{http://faostat3.fao.org/download/G1/*/E}- International Energy
Agency (IEA). 2014. CO2 Emissions from Fuel Combustion (2014 edition).
Paris, France: OECD/IEA. Available online
at\url{http://data.iea.org/ieastore/statslisting.asp.} © OECD/IEA,
{[}2014{]}. - World Bank. 2014. World Development Indicators 2014.
Washington, DC. Available at:\url{http://data.worldbank.org/}Last
Accessed May 18th, 2015 - U.S. Energy Information Administration (EIA).
2014. International Energy Statistics Washington, DC: U.S. Department of
Energy. Available online
at\url{http://www.eia.gov/cfapps/ipdbproject/IEDIndex3.cfm?tid=90\&pid=44\&aid=8}-
U.S. Environmental Protection Agency (EPA). 2012. ``Global Non-CO2 GHG
Emissions: 1990-2030.'' Washington, DC: EPA. Available
at:\url{http://www.epa.gov/climatechange/EPAactivities/economics/nonco2projections.html.}


\vspace{0.5cm}
\needspace{3\baselineskip} \rule{4cm}{0.25pt}\newline\textbf{Aussi disponible du même producteur :}\begin{itemize}
\item \href{https://data.gouv.fr/dataset/57c6de60c751df225897bae5}{CAIT Paris Contributions Data}
\end{itemize}

\clearpage
\section{WWF}


\begin{center}
  \includegraphics[width=3cm]{images/orga/b8_a0041f6cc544acad37a0f0da901c05-100.jpg}
\end{center}


Le WWF est la première organisation mondiale de protection de la nature.

Une organisation indépendante. + Le WWF compte plus de 5 millions de
donateurs à travers le monde. + L'organisation dispose d'un réseau
opérationnel dans 100 pays proposant 1 200 programmes de protection de
la nature. + Les compétences du WWF dans le domaine scientifique sont
mondialement reconnues.

Le WWF : une philosophie fondée sur le dialogue. Le WWF recherche dans
le monde entier la concertation pour la mise en œuvre de solutions
concrètes et durable. L'organisation a une réelle volonté d'impliquer
l'ensemble des acteurs concernés: communautés locales, entreprises,
gouvernements, organisations internationales et non gouvernementales.

Des réalisations d'envergure nationale et internationale. + La
protection de l'un des derniers fleuves sauvages d'Europe : la Loire +
La protection des espèces menacées: dauphins et baleines en
Méditerranée, ours brun dans les Pyrénées + La lutte contre le commerce
des espèces menacées grâce au réseau TRAFFIC + La création d'une réserve
naturelle de 17000 hectares pour la protection des tortues luth en
Guyane + Le développement d'un programme de gestion durable des forêts
(plus de 3 millions d'hectares) en collaboration avec les entreprises


\vspace{0.5cm}

\needspace{12\baselineskip}
\subsection*{Empreinte écologique et biocapacité
}\index{biocapacite}\index{empreinte!ecologique}\index{idh}\index{mdg}\index{mdg2015}\index{wwf}
  \begin{wrapfigure}{r}{2.5cm}
    \centering
    \qrcode[nolink]{https://data.gouv.fr/dataset/5369941aa3a729239d204372}
  \end{wrapfigure}

Licence : \textbf{Licence Ouverte
}\newline
Créé le : 2013-09-30\newline
Modifié le : 2015-11-22\newline
De 2008-01-01 à 2008-12-31\newline
Popularité : 1 réutilisation,  0 suivi\newline
Mots-clé : \emph{biocapacite, empreinte-ecologique, idh, mdg, mdg2015, wwf
}\newline
Permalien : \url{https://data.gouv.fr/dataset/5369941aa3a729239d204372}\newline

\par
\noindent
    Tableau de données de l'empreinte écologique.

Chiffres issus du rapport WWF ``Planète Vivante 2012'' sur l'empreinte
écologique totale et la biocapacité.


\vspace{0.5cm}

\clearpage
\section{Yanport}


\begin{center}
  \includegraphics[width=3cm]{images/orga/4a_1a7eead3b24707bd895ed4aface684-100.png}
\end{center}


Yanport est un logiciel immobilier dédié aux professionnels du secteur
et propose 3 grandes services : la pige immobilière, l'estimation
immobilière et la veille concurrentielle.

Nous publions certaines de nos données en OpenData comme l'encadrement
des loyers, le classement des agences immobilières et le classement des
portails.


\vspace{0.5cm}

\needspace{12\baselineskip}
\subsection*{Classement des agences immobilières de Levallois
}\index{agence!immobiliere}\index{classement}
  \begin{wrapfigure}{r}{2.5cm}
    \centering
    \qrcode[nolink]{https://data.gouv.fr/dataset/583f4357c751df7e00c0bb7e}
  \end{wrapfigure}

Licence : \textbf{Open Data Commons Open Database License (ODbL)
}\newline
Créé le : 2016-11-30\newline
Modifié le : 2016-11-30\newline
Granularité : à la commune\newline
Mise à jour : ponctuelle\newline
Popularité : 1 réutilisation,  1 suivi\newline
Mots-clé : \emph{agence-immobiliere, classement
}\newline
Permalien : \url{https://data.gouv.fr/dataset/583f4357c751df7e00c0bb7e}\newline

\par
\noindent
    Classement des agences immobilières de Levallois


\vspace{0.5cm}
\needspace{12\baselineskip}
\subsection*{Classements des portails immobiliers
}\index{annonces}\index{immobilier}\index{portail}
  \begin{wrapfigure}{r}{2.5cm}
    \centering
    \qrcode[nolink]{https://data.gouv.fr/dataset/583f4018c751df77a0c0bb7e}
  \end{wrapfigure}

Licence : \textbf{Open Data Commons Open Database License (ODbL)
}\newline
Créé le : 2016-11-30\newline
Modifié le : 2016-11-30\newline
Granularité : à la région\newline
Mise à jour : mensuelle\newline
Popularité : 3 réutilisations,  2 suivis\newline
Mots-clé : \emph{annonces, immobilier, portail
}\newline
Permalien : \url{https://data.gouv.fr/dataset/583f4018c751df77a0c0bb7e}\newline

\par
\noindent
    Ce jeu de données contient le nombres d'annonces sur chaque portail
surveillé par \href{https://www.yanport.com}{Yanport}.

Pour plus de détails, consulter le
\href{https://github.com/yanport/classement-portails}{projet github}.


\vspace{0.5cm}
\needspace{12\baselineskip}
\subsection*{Volume d'annonces par département des grands portails immobiliers
}\index{immobilier}\index{portail}
  \begin{wrapfigure}{r}{2.5cm}
    \centering
    \qrcode[nolink]{https://data.gouv.fr/dataset/5889d257c751df24abae0a65}
  \end{wrapfigure}

Licence : \textbf{Open Data Commons Open Database License (ODbL)
}\newline
Créé le : 2017-01-26\newline
Modifié le : 2017-01-26\newline
Granularité : au département\newline
Mise à jour : mensuelle\newline
Popularité : 1 réutilisation,  1 suivi\newline
Mots-clé : \emph{immobilier, portail
}\newline
Permalien : \url{https://data.gouv.fr/dataset/5889d257c751df24abae0a65}\newline

\par
\noindent
    Les données et le code pour représenter sur des cartes sont disponible
sur le site\url{https://github.com/yanport/meilleursportails.immo}


\vspace{0.5cm}
\needspace{3\baselineskip} \rule{4cm}{0.25pt}\newline\textbf{Aussi disponible du même producteur :}\begin{itemize}
\item \href{https://data.gouv.fr/dataset/5780fd2288ee38629f67b0c8}{Données de l'encadrement des loyers à Paris et Lille}
\end{itemize}
\backmatter

\printindex

\clearpage
\thispagestyle{empty}
\vspace*{\fill}\begin{center}
\rule{\textwidth}{0.2pt}
  Finit d'imprimé le 1er Avril 2019\\
  Dépot presque légal Avril 2019\\
  Ce document est publié sous \href{https://www.etalab.gouv.fr/wp-content/uploads/2017/04/ETALAB-Licence-Ouverte-v2.0.pdf}{Licence Ouverte 2.0}
\rule{\textwidth}{0.2pt}
\end{center}
\vspace*{\fill}
\includegraphics[width=\textwidth]{images/equipe-etalab.jpg}
\pagebreak

}
\end{document}
